% Complete documentation on the extended LaTeX markup used for Python
% documentation is available in ``Documenting Python'', which is part
% of the standard documentation for Python.  It may be found online
% at:
%
%     http://www.python.org/doc/current/doc/doc.html

\documentclass{manual}

\title{Python compiler package}

\author{Jeremy Hylton}

% Please at least include a long-lived email address;
% the rest is at your discretion.
\authoraddress{
        PythonLabs \\
        Zope Corp. \\
        Email: \email{jeremy@zope.com}
}

\date{August 15, 2001}           % update before release!
                                % Use an explicit date so that reformatting
                                % doesn't cause a new date to be used.  Setting
                                % the date to \today can be used during draft
                                % stages to make it easier to handle versions.

\release{2.2}                   % release version; this is used to define the
                                % \version macro

\makeindex                      % tell \index to actually write the .idx file
\makemodindex                   % If this contains a lot of module sections.


\begin{document}

\maketitle

% This makes the contents more accessible from the front page of the HTML.
\ifhtml
\chapter*{Front Matter\label{front}}
\fi

%\begin{small}
Copyright \copyright{} 2001 Python Software Foundation.
All rights reserved.

Copyright \copyright{} 2000 BeOpen.com.
All rights reserved.

Copyright \copyright{} 1995-2000 Corporation for National Research Initiatives.
All rights reserved.

Copyright \copyright{} 1991-1995 Stichting Mathematisch Centrum.
All rights reserved.

%%begin{latexonly}
\vskip 4mm
%%end{latexonly}

\centerline{\strong{BEOPEN.COM TERMS AND CONDITIONS FOR PYTHON 2.0}}

\centerline{\strong{BEOPEN PYTHON OPEN SOURCE LICENSE AGREEMENT VERSION 1}}

\begin{enumerate}

\item
This LICENSE AGREEMENT is between BeOpen.com (``BeOpen''), having an
office at 160 Saratoga Avenue, Santa Clara, CA 95051, and the
Individual or Organization (``Licensee'') accessing and otherwise
using this software in source or binary form and its associated
documentation (``the Software'').

\item
Subject to the terms and conditions of this BeOpen Python License
Agreement, BeOpen hereby grants Licensee a non-exclusive,
royalty-free, world-wide license to reproduce, analyze, test, perform
and/or display publicly, prepare derivative works, distribute, and
otherwise use the Software alone or in any derivative version,
provided, however, that the BeOpen Python License is retained in the
Software, alone or in any derivative version prepared by Licensee.

\item
BeOpen is making the Software available to Licensee on an ``AS IS''
basis.  BEOPEN MAKES NO REPRESENTATIONS OR WARRANTIES, EXPRESS OR
IMPLIED.  BY WAY OF EXAMPLE, BUT NOT LIMITATION, BEOPEN MAKES NO AND
DISCLAIMS ANY REPRESENTATION OR WARRANTY OF MERCHANTABILITY OR FITNESS
FOR ANY PARTICULAR PURPOSE OR THAT THE USE OF THE SOFTWARE WILL NOT
INFRINGE ANY THIRD PARTY RIGHTS.

\item
BEOPEN SHALL NOT BE LIABLE TO LICENSEE OR ANY OTHER USERS OF THE
SOFTWARE FOR ANY INCIDENTAL, SPECIAL, OR CONSEQUENTIAL DAMAGES OR LOSS
AS A RESULT OF USING, MODIFYING OR DISTRIBUTING THE SOFTWARE, OR ANY
DERIVATIVE THEREOF, EVEN IF ADVISED OF THE POSSIBILITY THEREOF.

\item
This License Agreement will automatically terminate upon a material
breach of its terms and conditions.

\item
This License Agreement shall be governed by and interpreted in all
respects by the law of the State of California, excluding conflict of
law provisions.  Nothing in this License Agreement shall be deemed to
create any relationship of agency, partnership, or joint venture
between BeOpen and Licensee.  This License Agreement does not grant
permission to use BeOpen trademarks or trade names in a trademark
sense to endorse or promote products or services of Licensee, or any
third party.  As an exception, the ``BeOpen Python'' logos available
at http://www.pythonlabs.com/logos.html may be used according to the
permissions granted on that web page.

\item
By copying, installing or otherwise using the software, Licensee
agrees to be bound by the terms and conditions of this License
Agreement.
\end{enumerate}


\centerline{\strong{CNRI OPEN SOURCE GPL-COMPATIBLE LICENSE AGREEMENT}}

Python 1.6.1 is made available subject to the terms and conditions in
CNRI's License Agreement.  This Agreement together with Python 1.6.1 may
be located on the Internet using the following unique, persistent
identifier (known as a handle): 1895.22/1013.  This Agreement may also
be obtained from a proxy server on the Internet using the following
URL: \url{http://hdl.handle.net/1895.22/1013}.


\centerline{\strong{CWI PERMISSIONS STATEMENT AND DISCLAIMER}}

Copyright \copyright{} 1991 - 1995, Stichting Mathematisch Centrum
Amsterdam, The Netherlands.  All rights reserved.

Permission to use, copy, modify, and distribute this software and its
documentation for any purpose and without fee is hereby granted,
provided that the above copyright notice appear in all copies and that
both that copyright notice and this permission notice appear in
supporting documentation, and that the name of Stichting Mathematisch
Centrum or CWI not be used in advertising or publicity pertaining to
distribution of the software without specific, written prior
permission.

STICHTING MATHEMATISCH CENTRUM DISCLAIMS ALL WARRANTIES WITH REGARD TO
THIS SOFTWARE, INCLUDING ALL IMPLIED WARRANTIES OF MERCHANTABILITY AND
FITNESS, IN NO EVENT SHALL STICHTING MATHEMATISCH CENTRUM BE LIABLE
FOR ANY SPECIAL, INDIRECT OR CONSEQUENTIAL DAMAGES OR ANY DAMAGES
WHATSOEVER RESULTING FROM LOSS OF USE, DATA OR PROFITS, WHETHER IN AN
ACTION OF CONTRACT, NEGLIGENCE OR OTHER TORTIOUS ACTION, ARISING OUT
OF OR IN CONNECTION WITH THE USE OR PERFORMANCE OF THIS SOFTWARE.
\end{small}


\begin{abstract}

\noindent
The Python compiler package is a tool for analyzing Python source code
and generating Python bytecode.  The compiler contains libraries to
generate an abstract syntax tree from Python source code and to
generate Python bytecode from the tree.

\end{abstract}

\tableofcontents

\chapter{Introduction\label{Introduction}}

XXX Need basic intro

XXX what are the major advantages...  the abstract syntax is much
closer to the python source...

\section{The basic interface}

The top-level of the package defines four functions.

\begin{funcdesc}{parse}{buf}
Returns an abstract syntax tree for the Python source code in \var{buf}.
The function raises SyntaxError if there is an error in the source
code.  The return value is a \class{compiler.ast.Module} instance that
contains the tree.  
\end{funcdesc}

\begin{funcdesc}{parseFile}{path}
Return an abstract syntax tree for the Python source code in the file
specified by \var{path}.  It is equivalent to \code{parse(open(path).read())}.
\end{funcdesc}

\begin{funcdesc}{walk}{ast, visitor, \optional{verbose=None}}
Do a pre-order walk over the abstract syntax tree \var{ast}.  Call the
appropriate method on the \var{visitor} instance for each node
encountered. 
\end{funcdesc}

\begin{funcdesc}{compile}{filename}
Compile the file \var{filename} and generated \var{filename}.pyc.
\end{funcdesc}

The \module{compiler} package contains the following modules:
\module{ast}, \module{consts}, \module{future}, \module{misc},
\module{pyassem}, \module{pycodegen}, \module{symbols},
\module{transformer}, and \module{visitor}.

\section{Limitations}

There are some problems with the error checking of the compiler
package.  The interpreter detects syntax errors in two distinct
phases.  One set of errors is detected by the interpreter's parser,
the other set by the compiler.  The compiler package relies on the
interpreter's parser, so it get the first phases of error checking for
free.  It implements the second phase itself, and that implement is
incomplete.  For example, the compiler package does not raise an error
if a name appears more than once in an argument list: 
\code{def f(x, x): ...}

\chapter{Python Abstract Syntax}

\section{Introduction}

The \module{compiler.ast} module defines an abstract syntax for
Python.  In the abstract syntax tree, each node represents a syntactic
construct.  The root of the tree is \class{Module} object.

The abstract syntax offers a higher level interface to parsed Python
source code.  The \module{parser} module and the compiler written in C
for the Python interpreter use a concrete syntax tree.  The concrete
syntax is tied closely to the grammar description used for the Python
parser.  Instead of a single node for a construct, there are often
several levels of nested nodes that are introduced by Python's
precedence rules.

The abstract syntax tree is created by the
\module{compiler.transformer} module.  The transformer relies on the
builtin Python parser to generate a concrete syntax tree.  It
generates an abstract syntax tree from the concrete tree.  

The \module{transformer} module was created by Greg Stein and Bill
Tutt for the Python-to-C compiler.  The current version contains a
number of modifications and improvements, but the basic form of the
abstract syntax and of the transformer are due to Stein and Tutt.

\section{AST Nodes}

The \module{ast} module is generated from a text file that describes
each node type and its elements.  Each node type is represented as a
class that inherits from the abstract base class \class{ast.Node} and
defines a set of named attributes for child nodes.

\begin{classdesc}{Node}{}
  
  The \class{Node} instances are created automatically by the parser
  generator.  The recommended interface for specific \class{Node}
  instances is to use the public attributes to access child nodes.  A
  public attribute may be bound to a single node or to a sequence of
  nodes, depending on the \class{Node} type.  For example, the
  \member{bases} attribute of the \class{Class} node, is bound to a
  list of base class nodes, and the \member{doc} attribute is bound to
  a single node.
  
  Each \class{Node} instance has a \member{lineno} attribute which may
  be \code{None}.  XXX Not sure what the rules are for which nodes
  will have a useful lineno.

  \begin{methoddesc}{getChildren}{}
    Returns a flattened list of the child nodes and objects in the
    order they occur.  Specifically, the order of the nodes is the
    order in which they appear in the Python grammar.  Not all of the
    children are \class{Node} instances.  The names of functions and
    classes, for example, are plain strings.
  \end{methoddesc}

  \begin{methoddesc}{getChildNodes}{}
    Returns a flattened list of the child nodes in the order they
    occur.  This method is like \method{getChildNodes}, except that it
    only returns those children that are \class{Node} instances.
  \end{methoddesc}

\end{classdesc}

Two examples illustrate the general structure of \class{Node}
classes.  The while statement is defined by the following grammar
production: 

\begin{verbatim}
while_stmt:     "while" expression ":" suite
               ["else" ":" suite]
\end{verbatim}

The \class{While} node has three attributes: \member{test},
\member{body}, and \member{else_}.  (If the natural name for an
attribute is also a Python reserved word, it can't be used as an
attribute name.  An underscore is appended to the word to make it
legal, hence \code{else_} instead of \code{else}.)

The if statement is more complicated because it can include several
tests.  

\begin{verbatim}
if_stmt: 'if' test ':' suite ('elif' test ':' suite)* ['else' ':' suite]
\end{verbatim}

The \class{If} node only defines two attributes: \member{tests} and
\member{else_}.  The \member{tests} attribute is a sequence of test
expression, consequent body pairs.  There is one pair of each if/elif
clause.  The first element of the pair is the test expression.  The
second elements is a \class{Stmt} node that contains the code to
execute if the test is true.

The \method{getChildren()} method of \class{If} returns a flat list of
child nodes.  If there are three if/elif clauses and no else clause,
then \method{getChildren()} will return a list of six elements: the
first test expression, the first \class{Stmt}, the second text
expression, etc.

The following table lists each of the \class{Node} subclasses defined
in \module{compiler.ast} and each of the public attributes available
on their instances.  The values of most of the attributes are
themselves \class{Node} instances or sequences of instances.  When the
value is something other than an instance, the type is noted in the
comment.  The attributes are listed in the order in which they are
returned by \method{getChildren} and \method{getChildNodes}.

\begin{longtableiii}{lll}{class}{Node type}{Attribute}{Value}

\lineiii{Add}{left}{left operand}
\lineiii{}{right}{right operand}
\hline 

\lineiii{And}{nodes}{list of operands}
\hline 

\lineiii{AssAttr}{}{\emph{attribute as target of assignment}}
\lineiii{}{expr}{expression on the left-hand side of the dot}
\lineiii{}{attrname}{the attribute name, a string}
\lineiii{}{flags}{XXX}
\hline 

\lineiii{AssList}{nodes}{list of list elements being assigned to}
\hline 

\lineiii{AssName}{name}{name being assigned to}
\lineiii{}{flags}{XXX}
\hline 

\lineiii{AssTuple}{nodes}{list of tuple elements being assigned to}
\hline 

\lineiii{Assert}{test}{the expression to be tested}
\lineiii{}{fail}{the value of the \exception{AssertionError}}
\hline 

\lineiii{Assign}{nodes}{a list of assignment targets, one per equal sign}
\lineiii{}{expr}{the value being assigned}
\hline 

\lineiii{AugAssign}{node}{}
\lineiii{}{op}{}
\lineiii{}{expr}{}
\hline 

\lineiii{Backquote}{expr}{}
\hline 

\lineiii{Bitand}{nodes}{}
\hline 

\lineiii{Bitor}{nodes}{}
\hline 

\lineiii{Bitxor}{nodes}{}
\hline 

\lineiii{Break}{}{}
\hline 

\lineiii{CallFunc}{node}{expression for the callee}
\lineiii{}{args}{a list of arguments}
\lineiii{}{star_args}{the extended *-arg value}
\lineiii{}{dstar_args}{the extended **-arg value}
\hline 

\lineiii{Class}{name}{the name of the class, a string}
\lineiii{}{bases}{a list of base classes}
\lineiii{}{doc}{doc string, a string or \code{None}}
\lineiii{}{code}{the body of the class statement}
\hline 

\lineiii{Compare}{expr}{}
\lineiii{}{ops}{}
\hline 

\lineiii{Const}{value}{}
\hline 

\lineiii{Continue}{}{}
\hline 

\lineiii{Dict}{items}{}
\hline 

\lineiii{Discard}{expr}{}
\hline 

\lineiii{Div}{left}{}
\lineiii{}{right}{}
\hline 

\lineiii{Ellipsis}{}{}
\hline 

\lineiii{Exec}{expr}{}
\lineiii{}{locals}{}
\lineiii{}{globals}{}
\hline 

\lineiii{For}{assign}{}
\lineiii{}{list}{}
\lineiii{}{body}{}
\lineiii{}{else_}{}
\hline 

\lineiii{From}{modname}{}
\lineiii{}{names}{}
\hline 

\lineiii{Function}{name}{name used in def, a string}
\lineiii{}{argnames}{list of argument names, as strings}
\lineiii{}{defaults}{list of default values}
\lineiii{}{flags}{xxx}
\lineiii{}{doc}{doc string, a string or \code{None}}
\lineiii{}{code}{the body of the function}
\hline

\lineiii{Getattr}{expr}{}
\lineiii{}{attrname}{}
\hline 

\lineiii{Global}{names}{}
\hline 

\lineiii{If}{tests}{}
\lineiii{}{else_}{}
\hline 

\lineiii{Import}{names}{}
\hline 

\lineiii{Invert}{expr}{}
\hline 

\lineiii{Keyword}{name}{}
\lineiii{}{expr}{}
\hline 

\lineiii{Lambda}{argnames}{}
\lineiii{}{defaults}{}
\lineiii{}{flags}{}
\lineiii{}{code}{}
\hline 

\lineiii{LeftShift}{left}{}
\lineiii{}{right}{}
\hline 

\lineiii{List}{nodes}{}
\hline 

\lineiii{ListComp}{expr}{}
\lineiii{}{quals}{}
\hline 

\lineiii{ListCompFor}{assign}{}
\lineiii{}{list}{}
\lineiii{}{ifs}{}
\hline 

\lineiii{ListCompIf}{test}{}
\hline 

\lineiii{Mod}{left}{}
\lineiii{}{right}{}
\hline 

\lineiii{Module}{doc}{doc string, a string or \code{None}}
\lineiii{}{node}{body of the module, a \class{Stmt}}
\hline 

\lineiii{Mul}{left}{}
\lineiii{}{right}{}
\hline 

\lineiii{Name}{name}{}
\hline 

\lineiii{Not}{expr}{}
\hline 

\lineiii{Or}{nodes}{}
\hline 

\lineiii{Pass}{}{}
\hline 

\lineiii{Power}{left}{}
\lineiii{}{right}{}
\hline 

\lineiii{Print}{nodes}{}
\lineiii{}{dest}{}
\hline 

\lineiii{Printnl}{nodes}{}
\lineiii{}{dest}{}
\hline 

\lineiii{Raise}{expr1}{}
\lineiii{}{expr2}{}
\lineiii{}{expr3}{}
\hline 

\lineiii{Return}{value}{}
\hline 

\lineiii{RightShift}{left}{}
\lineiii{}{right}{}
\hline 

\lineiii{Slice}{expr}{}
\lineiii{}{flags}{}
\lineiii{}{lower}{}
\lineiii{}{upper}{}
\hline 

\lineiii{Sliceobj}{nodes}{list of statements}
\hline 

\lineiii{Stmt}{nodes}{}
\hline 

\lineiii{Sub}{left}{}
\lineiii{}{right}{}
\hline 

\lineiii{Subscript}{expr}{}
\lineiii{}{flags}{}
\lineiii{}{subs}{}
\hline 

\lineiii{TryExcept}{body}{}
\lineiii{}{handlers}{}
\lineiii{}{else_}{}
\hline 

\lineiii{TryFinally}{body}{}
\lineiii{}{final}{}
\hline 

\lineiii{Tuple}{nodes}{}
\hline 

\lineiii{UnaryAdd}{expr}{}
\hline 

\lineiii{UnarySub}{expr}{}
\hline 

\lineiii{While}{test}{}
\lineiii{}{body}{}
\lineiii{}{else_}{}
\hline 

\lineiii{Yield}{value}{}
\hline 

\end{longtableiii}


\section{Assignment nodes}

There is a collection of nodes used to represent assignments.  Each
assignment statement in the source code becomes a single
\class{Assign} node in the AST.  The \member{nodes} attribute is a
list that contains a node for each assignment target.  This is
necessary because assignment can be chained, e.g. \code{a = b = 2}.
Each \class{Node} in the list will be one of the following classes: 
\class{AssAttr}, \class{AssList}, \class{AssName}, or
\class{AssTuple}. 

XXX Explain what the AssXXX nodes are for.  Mention \code{a.b.c = 2}
as an example.  Explain what the flags are for.

\chapter{Using Visitors to Walk ASTs}

The visitor pattern is ...  The \module{compiler} package uses a
variant on the visitor pattern that takes advantage of Python's
introspection features to elminiate the need for much of the visitor's
infrastructure.

The classes being visited do not need to be programmed to accept
visitors.  The visitor need only define visit methods for classes it
is specifically interested in; a default visit method can handle the
rest. 

XXX The magic \method{visit()} method for visitors.

\begin{classdesc}{ASTVisitor}{}

The \class{ASTVisitor} is responsible for walking over the tree in the
correct order.  A walk begins with a call to \method{preorder()}.  For
each node, it checks the \var{visitor} argument to \method{preorder{}}
for a method named `visitNodeType,' where NodeType is the name of the
node's class, e.g. for a \class{While} node a \method{visitWhile}
would be called .  If the method exists, it is called with the node as
its first argument.

The visitor method for a particular node type can control how child
nodes are visited during the walk.  The \class{ASTVisitor} modifies
the visitor argument by adding a visit method to the visitor; this
method can be used to visit a particular child node.  If no visitor is
found for a particular node type, the \method{default} method is
called. 

XXX describe extra arguments

\begin{methoddesc}{default}{node\optional{, *args}}
\end{methoddesc}

\begin{methoddesc}{dispatch}{node\optional{, *args}}
\end{methoddesc}

\begin{methoddesc}{preorder}{tree, visitor}
\end{methoddesc}

\end{classdesc}

\begin{funcdesc}{walk}{tree, visitor\optional{, verbose=None}}
\end{funcdesc}

\chapter{Bytecode Generation}

The code generator is a visit that emits bytecodes.  Each visit method
can call the \method{emit} method to emit a new bytecode.  The basic
code generator is specialized for modules, classes, and functions.  An
assembler converts that emitted instructions to the low-level bytecode
format.  It handles things like generator of constant lists of code
objects and calculation of jump offsets.

%
%  The ugly "%begin{latexonly}" pseudo-environments are really just to
%  keep LaTeX2HTML quiet during the \renewcommand{} macros; they're
%  not really valuable.
%
%  If you don't want the Module Index, you can remove all of this up
%  until the second \input line.
%
%begin{latexonly}
\renewcommand{\indexname}{Module Index}
%end{latexonly}
\input{mod\jobname.ind}         % Module Index

%begin{latexonly}
\renewcommand{\indexname}{Index}
%end{latexonly}
\input{\jobname.ind}                    % Index

\end{document}
