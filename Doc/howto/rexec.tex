\documentclass{howto}

\title{Restricted Execution HOWTO}

\release{2.1}

\author{A.M. Kuchling}
\authoraddress{\email{amk@amk.ca}}

\begin{document}

\maketitle

\begin{abstract}
\noindent

Python 2.2.2 and earlier provided a \module{rexec} module running
untrusted code.  However, it's never been exhaustively audited for
security and it hasn't been updated to take into account recent
changes to Python such as new-style classes. Therefore, the
\module{rexec} module should not be trusted.  To discourage use of 
\module{rexec}, this HOWTO has been withdrawn.

The \module{rexec} and \module{Bastion} modules have been disabled in
the Python CVS tree, both on the trunk (which will eventually become
Python 2.3alpha2 and later 2.3final) and on the release22-maint branch
(which will become Python 2.2.3, if someone ever volunteers to issue
2.2.3).

For discussion of the problems with \module{rexec}, see the python-dev
threads starting at the following URLs:
\url{http://mail.python.org/pipermail/python-dev/2002-December/031160.html},
and
\url{http://mail.python.org/pipermail/python-dev/2003-January/031848.html}.

\end{abstract}


\section{Version History}

Sep. 12, 1998: Minor revisions and added the reference to the Janus
project.

Feb. 26, 1998: First version.  Suggestions are welcome.

Mar. 16, 1998: Made some revisions suggested by Jeff Rush.  Some minor
changes and clarifications, and a sizable section on exceptions added.

Oct. 4, 2000: Checked with Python 2.0.  Minor rewrites and fixes made.
Version number increased to 2.0.

Dec. 17, 2002: Withdrawn.

Jan. 8, 2003: Mention that \module{rexec} will be disabled in Python 2.3,
and added links to relevant python-dev threads.

\end{document}




