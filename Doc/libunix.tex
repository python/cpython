\chapter{Unix Specific Services}

The modules described in this chapter provide interfaces to features
that are unique to the \UNIX{} operating system, or in some cases to
some or many variants of it.  Here's an overview:

\begin{description}

\item[posix]
--- The most common \POSIX{} system calls (normally used via module \code{os}).

\item[posixpath]
--- Common \POSIX{} pathname manipulations (normally used via \code{os.path}).

\item[pwd]
--- The password database (\code{getpwnam()} and friends).

\item[grp]
--- The group database (\code{getgrnam()} and friends).

\item[crypt]
--- The \code{crypt()} function used to check \UNIX{} passwords.

\item[dbm]
--- The standard ``database'' interface, based on \code{ndbm}.

\item[gdbm]
--- GNU's reinterpretation of dbm.

\item[termios]
--- \POSIX{} style tty control.

\item[TERMIOS]
--- The symbolic constants required to use the \code{termios} module.

\item[fcntl]
--- The \code{fcntl()} and \code{ioctl()} system calls.

\item[posixfile]
--- A file-like object with support for locking.

\item[resource]
--- An interface to provide resource usage information on the current
process.

\item[syslog]
--- An interface to the \UNIX{} \code{syslog} library routines.

\end{description}
