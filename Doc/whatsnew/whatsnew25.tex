\documentclass{howto}
\usepackage{distutils}
% $Id$


\title{What's New in Python 2.5}
\release{0.0}
\author{A.M. Kuchling}
\authoraddress{\email{amk@amk.ca}}

\begin{document}
\maketitle
\tableofcontents

This article explains the new features in Python 2.5.  No release date
for Python 2.5 has been set; it will probably be released in late 2005.

% Compare with previous release in 2 - 3 sentences here.

This article doesn't attempt to provide a complete specification of
the new features, but instead provides a convenient overview.  For
full details, you should refer to the documentation for Python 2.5.
% add hyperlink when the documentation becomes available online.
If you want to understand the complete implementation and design
rationale, refer to the PEP for a particular new feature.


%======================================================================
\section{PEP 309: Partial Function Application}

The \module{functional} module is intended to contain tools for
functional-style programming.  Currently it only contains
\class{partial}, but new functions will probably be added in future
versions of Python.

For programs written in a functional style, it can be useful to
construct variants of existing functions that have some of the
parameters filled in.  Consider a Python function \code{f(a, b, c)};
you could create a new function \code{g(b, c)} that was equivalent to
\code{f(1, b, c)}.  This is called ``partial function application'',
and is provided by the \class{partial} class in the new
\module{functional} module.

The constructor for \class{partial} takes the arguments
\code{(\var{function}, \var{arg1}, \var{arg2}, ...
\var{kwarg1}=\var{value1}, \var{kwarg2}=\var{value2})}.  The resulting
object is callable, so you can just call it to invoke \var{function}
with the filled-in arguments.

Here's a small but realistic example:

\begin{verbatim}
import functional

def log (message, subsystem):
    "Write the contents of 'message' to the specified subsystem."
    print '%s: %s' % (subsystem, message)
    ...

server_log = functional.partial(log, subsystem='server')
\end{verbatim}

Here's another example, from a program that uses PyGTk.  

% XXX add example from my GTk programming


\begin{seealso}

\seepep{309}{Partial Function Application}{PEP proposed and written by
Peter Harris; implemented by Hye-Shik Chang, with adaptations by
Raymond Hettinger.}

\end{seealso}


%======================================================================
\section{PEP 314: Metadata for Python Software Packages v1.1}

Some simple dependency support was added to Distutils.  The
\function{setup()} function now has \code{requires},\code{provides},
and \code{obsoletes}.  When you build a source distribution using the
\code{sdist} command, the dependency information will be recorded in
the \file{PKG-INFO} file.  

Another new keyword is \code{download_url}, which should be set to a
URL for the package's source code.  This means it's now possible to
look up an entry in the package index, determine the dependencies for
a package, and download the required packages.  

% XXX put example here
 
\begin{seealso}

\seepep{314}{Metadata for Python Software Packages v1.1}{PEP proposed
and written by A.M. Kuchling, Richard Jones, and Fred Drake; 
implemented by Richard Jones and Fred Drake.}

\end{seealso}


%======================================================================
\section{PEP 342: New Generator Features}

XXX write this section

\begin{seealso}

\seepep{342}{Coroutines via Enhanced Generators}{PEP written by 
Guido van Rossum and Phillip J. Eby;
implemented by Phillip J. Eby.}

\end{seealso}


%======================================================================
\section{Other Language Changes}

Here are all of the changes that Python 2.5 makes to the core Python
language.

\begin{itemize}

\item The \function{min()} and \function{max()} built-in functions
gained a \code{key} keyword argument analogous to the \code{key}
argument for \method{sort()}.  This argument supplies a function
that takes a single argument and is called for every value in the list; 
\function{min()}/\function{max()} will return the element with the 
smallest/largest return value from this function.
For example, to find the longest string in a list, you can do:

\begin{verbatim}
L = ['medium', 'longest', 'short']
# Prints 'longest'
print max(L, key=len)		   
# Prints 'short', because lexicographically 'short' has the largest value
print max(L)         
\end{verbatim}

(Contributed by Steven Bethard and Raymond Hettinger.)

\item The list of base classes in a class definition can now be empty.  
As an example, this is now legal:

\begin{verbatim}
class C():
    pass
\end{verbatim}
(Implemented by Brett Cannon.)

\end{itemize}


%======================================================================
\subsection{Optimizations}

\begin{itemize}

\item Optimizations should be described here.

\end{itemize}

The net result of the 2.5 optimizations is that Python 2.5 runs the
pystone benchmark around XX\% faster than Python 2.4.


%======================================================================
\section{New, Improved, and Deprecated Modules}

As usual, Python's standard library received a number of enhancements and
bug fixes.  Here's a partial list of the most notable changes, sorted
alphabetically by module name. Consult the
\file{Misc/NEWS} file in the source tree for a more
complete list of changes, or look through the CVS logs for all the
details.

\begin{itemize}

% the cPickle module no longer accepts the deprecated None option in the
% args tuple returned by __reduce__().

% csv module improvements

% datetime.datetime() now has a strptime class method which can be used to
% create datetime object using a string and format.

\item The \function{nsmallest()} and 
\function{nlargest()} functions in the \module{heapq} module 
now support a \code{key} keyword argument similar to the one
provided by the \function{min()}/\function{max()} functions
and the \method{sort()} methods.  For example:
Example:

\begin{verbatim}
>>> import heapq
>>> L = ["short", 'medium', 'longest', 'longer still']
>>> heapq.nsmallest(2, L)  # Return two lowest elements, lexicographically
['longer still', 'longest']
>>> heapq.nsmallest(2, L, key=len)   # Return two shortest elements
['short', 'medium']
\end{verbatim}

(Contributed by Raymond Hettinger.)

\item The \function{itertools.islice()} function now accepts
\code{None} for the start and step arguments.  This makes it more
compatible with the attributes of slice objects, so that you can now write
the following:

\begin{verbatim}
s = slice(5)     # Create slice object
itertools.islice(iterable, s.start, s.stop, s.step)
\end{verbatim}

(Contributed by Raymond Hettinger.)

\item New module: \module{spwd} provides functions for accessing the
shadow password database on systems that support it.  
% XXX give example

\item The \module{os} module underwent a number of changes.  The
\member{stat_float_times} variable now defaults to true, meaning that
\function{os.stat()} will now return time values as floats.  (This
doesn't necessarily mean that \function{os.stat()} will return times
that are precise to fractions of a second; not all systems support
such precision.)

Also, constants named \member{os.SEEK_SET}, \member{os.SEEK_CUR}, and
\member{os.SEEK_END} have been added; these are the parameters to the
\function{os.lseek()} function.

\item The \class{TarFile} class in the \module{tarfile} module now has
an \method{extractall()} method that extracts all members from the
archive into the current working directory.  It's also possible to set
a different directory as the extraction target, and to unpack only a
subset of the archive's members.  (Contributed by Lars Gust\"abel.)

\end{itemize}



%======================================================================
% whole new modules get described in \subsections here


% ======================================================================
\section{Build and C API Changes}

Changes to Python's build process and to the C API include:

\begin{itemize}

\item The \cfunction{PyRange_New()} function was removed.  It was never documented,
never used in the core code, and had dangerously lax error checking.  

\end{itemize}


%======================================================================
\subsection{Port-Specific Changes}

Platform-specific changes go here.


%======================================================================
\section{Other Changes and Fixes \label{section-other}}

As usual, there were a bunch of other improvements and bugfixes
scattered throughout the source tree.  A search through the CVS change
logs finds there were XXX patches applied and YYY bugs fixed between
Python 2.4 and 2.5.  Both figures are likely to be underestimates.

Some of the more notable changes are:

\begin{itemize}

\item Details go here.

\end{itemize}


%======================================================================
\section{Porting to Python 2.5}

This section lists previously described changes that may require
changes to your code:

\begin{itemize}

\item Some old deprecated modules (\module{statcache}, \module{tzparse},
      \module{whrandom})  have been moved to \file{Lib/lib-old}.
You can get access to these modules  again by adding the directory 
to your \code{sys.path}:

\begin{verbatim}
import os
from distutils import sysconfig

lib_dir = sysconfig.get_python_lib(standard_lib=True)
old_dir = os.path.join(lib_dir, 'lib-old')
sys.path.append(old_dir)
\end{verbatim}

Doing so is discouraged, however; it's better to update any code that
still uses these modules.

% the pickle module no longer uses the deprecated bin parameter.

\end{itemize}


%======================================================================
\section{Acknowledgements \label{acks}}

The author would like to thank the following people for offering
suggestions, corrections and assistance with various drafts of this
article: .

\end{document}
