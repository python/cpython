\documentclass{howto}

% $Id$

\title{What's New in Python 2.1}
\release{0.03}
\author{A.M. Kuchling}
\authoraddress{\email{amk1@bigfoot.com}}
\begin{document}
\maketitle\tableofcontents

\section{Introduction}

{\large This document is a draft, and is subject to change until
Python 2.1 is released.  Please send any comments, bug reports, or questions,
no matter how minor, to \email{amk1@bigfoot.com}.
}

It's that time again... time for a new Python release, version 2.1.
One recent goal of the Python development team has been to accelerate
the pace of new releases, with a new release coming every 6 to 9
months. 2.1 is the first release to come out at this faster pace, with
the first alpha appearing in January, 3 months after the final version
of 2.0 was released.

This article explains the new features in 2.1.  While there aren't as
many changes in 2.1 as there were in Python 2.0, there are still some
pleasant surprises in store.  2.1 is the first release to be steered
through the use of Python Enhancement Proposals, or PEPs, so most of
the sizable changes have accompanying PEPs that provide more complete
documentation and a design rationale for the change.  This article
doesn't attempt to document the new features completely, but simply
provides an overview of the new features for Python programmers.
Refer to the Python 2.1 documentation, or to the specific PEP, for
more details about any new feature that particularly interests you.

Currently 2.1 is available in an alpha release, but the release
schedule calls for a beta release by late February 2001, and a final
release in April 2001.

% ======================================================================
\section{PEP 232: Function Attributes}

In Python 2.1, functions can now have arbitrary
information attached to them.  People were often using docstrings to hold
information about functions and methods, because the \code{__doc__}
attribute was the only way of attaching any information to a function.
For example, in the Zope Web application server, functions are marked
as safe for public access by having a docstring, and in John Aycock's
SPARK parsing framework, docstrings hold parts of the BNF grammar to
be parsed.  This overloading is unfortunate, since docstrings are
really intended to hold a function's documentation, and it 
means you
can't properly document functions intended for private use in Zope.

Attributes can now be set and retrieved on functions, using the
regular Python syntax:

\begin{verbatim}
def f(): pass

f.publish = 1
f.secure = 1
f.grammar = "A ::= B (C D)*"
\end{verbatim}    

The dictionary containing attributes can be accessed as
\member{__dict__}. Unlike the \member{__dict__} attribute of
class instances, in functions you can actually assign a new dictionary
to \member{__dict__}, though the new value is restricted to a
regular Python dictionary; you can't be tricky and set it to a
\class{UserDict} instance, a DBM file, or any other random mapping
object.

\begin{seealso}

\seepep{232}{Function Attributes}{Written and implemented by Barry Warsaw.}

\end{seealso}

% ======================================================================
\section{PEP 207: Rich Comparisons}

In earlier versions, Python's support for implementing comparisons on
user-defined classes and extension types was quite simple. Classes
could implement a \method{__cmp__} method that was given two instances
of a class, and could only return 0 if they were equal or +1 or -1 if
they weren't; the method couldn't raise an exception or return
anything other than a Boolean value.  Users of Numeric Python often
found this model too weak and restrictive, because in the
number-crunching programs that numeric Python is used for, it would be
more useful to be able to perform elementwise comparisons of two
matrices, returning a matrix containing the results of a given
comparison for each element.  If the two matrices are of different
sizes, then the compare has to be able to raise an exception to signal
the error.

In Python 2.1, rich comparisons were added in order to support this need.
Python classes can now individually overload each of the \code{<},
\code{<=}, \code{>}, \code{>=}, \code{==}, and \code{!=} operations.
The new magic method names are:

\begin{tableii}{c|l}{code}{Operation}{Method name}
  \lineii{<}{\method{__lt__}}
  \lineii{<=}{\method{__le__}}
  \lineii{>}{\method{__gt__}}
  \lineii{>=}{\method{__ge__}}
  \lineii{==}{\method{__eq__}}
  \lineii{!=}{\method{__ne__}}
\end{tableii}

(The magic methods are named after the corresponding Fortran operators
\code{.LT.}. \code{.LE.}, \&c.  Numeric programmers are almost
certainly quite familar with these names and will find them easy to
remember.)
 
Each of these magic methods is of the form \code{\var{method}(self,
other)}, where \code{self} will be the object on the left-hand side of
the operator, while \code{other} will be the object on the right-hand
side.  For example, the expression \code{A < B} will cause
\code{A.__lt__(B)} to be called.

Each of these magic methods can return anything at all: a Boolean, a
matrix, a list, or any other Python object.  Alternatively they can
raise an exception if the comparison is impossible, inconsistent, or
otherwise meaningless.

The built-in \function{cmp(A,B)} function can use the rich comparison
machinery, and now accepts an optional argument specifying which
comparison operation to use; this is given as one of the strings
\code{"<"}, \code{"<="}, \code{">"}, \code{">="}, \code{"=="}, or
\code{"!="}.  If called without the optional third argument,
\function{cmp()} will only return -1, 0, or +1 as in previous versions
of Python; otherwise it will call the appropriate method and can
return any Python object.

There are also corresponding changes of interest to C programmers;
there's a new slot \code{tp_richcmp} in type objects and an API for
performing a given rich comparison.  I won't cover the C API here, but
will refer you to PEP 207, or the documentation for Python's C API,
for the full list of related functions.

\begin{seealso}

\seepep{207}{Rich Comparisions}{Written by Guido van Rossum, heavily
based on earlier work by David Ascher, and implemented by Guido van Rossum.}

\end{seealso}

% ======================================================================
\section{PEP 230: Warning Framework}

Over its 10 years of existence, Python has accumulated a certain
number of obsolete modules and features along the way.  It's difficult
to know when a feature is safe to remove, since there's no way of
knowing how much code uses it --- perhaps no programs depend on the
feature, or perhaps many do.  To enable removing old features in a
more structured way, a warning framework was added.  When the Python
developers want to get rid of a feature, it will first trigger a
warning in the next version of Python.  The following Python version
can then drop the feature, and users will have had a full release
cycle to remove uses of the old feature.  

Python 2.1 adds the warning framework to be used in this scheme.  It
adds a \module{warnings} module that provide functions to issue
warnings, and to filter out warnings that you don't want to be
displayed. Third-party modules can also use this framework to
deprecate old features that they no longer wish to support.

For example, in Python 2.1 the \module{regex} module is deprecated,
so importing it causes a warning to be printed:

\begin{verbatim}
>>> import regex
__main__:1: DeprecationWarning: the regex module is deprecated; please use the re module
>>> 
\end{verbatim}

Warnings can be issued by calling the \function{warnings.warn} function:

\begin{verbatim}
warnings.warn("feature X no longer supported")              
\end{verbatim}

The first parameter is the warning message; an additional optional
parameters can be used to specify a particular warning category.

Filters can be added to disable certain warnings; a regular expression
pattern can be applied to the message or to the module name in order
to suppress a warning.  For example, you may have a program that uses
the \module{regex} module and not want to spare the time to convert it
to use the \module{re} module right now.  The warning can be
suppressed by calling

\begin{verbatim}
import warnings
warnings.filterwarnings(action = 'ignore',
                        message='.*regex module is deprecated',
                        category=DeprecationWarning,
                        module = '__main__')
\end{verbatim}

This adds a filter that will apply only to warnings of the class
\class{DeprecationWarning} triggered in the \module{__main__} module, and applies a regular expression to only match the message about the \module{regex} module being deprecated, and will cause such warnings to be ignored.  Warnings can also be printed only once, printed every time the offending code is executed, or turned into exceptions that will cause the program to stop (unless the exceptions are caught in the usual way, of course). 

Functions were also added to Python's C API for issuing warnings;
refer to PEP 230 or to Python's API documentation for the details.

\begin{seealso}
\seepep{5}{Guidelines for Language Evolution}{Written by Paul Prescod,
to specify procedures to be followed when removing old features from
Python.  The policy described in this PEP hasn't been officially
adopted, but the eventual policy probably won't be too different from
Prescod's proposal.}

\seepep{230}{Warning Framework}{Written and implemented by Guido van Rossum.}
\end{seealso}
    
% ======================================================================
\section{PEP 229: New Build System}

When compiling Python, the user had to go in and edit the
\file{Modules/Setup} file in order to enable various additional
modules; the default set is relatively small and limited to modules
that compile on most Unix platforms.  This means that on Unix
platforms with many more features, most notably Linux, Python
installations often don't contain all useful modules they could.

Python 2.0 added the Distutils, a set of modules for distributing and
installing extensions.  In Python 2.1, the Distutils are used to
compile much of the standard library of extension modules,
autodetecting which ones are supported on the current machine.  It's
hoped that this will make Python installations easier and more
featureful.

Instead of having to edit the \file{Modules/Setup} file in order to
enable modules, a \file{setup.py} script in the top directory of the
Python source distribution is run at build time, and attempts to
discover which modules can be enabled by examining the modules and
header files on the system.  In 2.1alpha1, there's very little you can
do to change \file{setup.py}'s behaviour, or to discover why a given
module isn't compiled.  If you run into problems in 2.1alpha1, please
report them, and be prepared to dive into \file{setup.py} in order to
fix autodetection of a given library on your system.  In the alpha2
release I plan to add ways to have more control over what the script
does (probably command-line arguments to \file{configure} or to
\file{setup.py}).

If it turns out to be impossible to make autodetection work reliably,
it's possible that this change may become an optional build method
instead of the default, or it may even be backed out completely.

\begin{seealso}
\seepep{229}{Using Distutils to Build Python}{Written and implemented by A.M. Kuchling.}
\end{seealso}

% ======================================================================
\section{PEP 217: Interactive Display Hook}

When using the Python interpreter interactively, the output of
commands is displayed using the built-in \function{repr()} function.
In Python 2.1, the variable \module{sys.displayhook} can be set to a
callable object which will be called instead of \function{repr()}.
For example, you can set it to a special pretty-printing function:

\begin{verbatim}
>>> # Create a recursive data structure
... L = [1,2,3]
>>> L.append(L)
>>> L   # Show Python's default output
[1, 2, 3, [...]]
>>> # Use pprint.pprint() as the display function
... import sys, pprint
>>> sys.displayhook = pprint.pprint
>>> L
[1, 2, 3, <Recursion on list with id=135143996>]
>>> 
\end{verbatim}

\begin{seealso}

\seepep{217}{Display Hook for Interactive Use}{Written and implemented by Moshe Zadka.}

\end{seealso}

% ======================================================================
\section{PEP 208: New Coercion Model}

How numeric coercion is done at the C level was significantly
modified.  This will only affect the authors of C extensions to
Python, allowing them more flexibility in writing extension types that
support numeric operations.

Extension types can now set the type flag
\code{Py_TPFLAGS_CHECKTYPES} in their \code{PyTypeObject}
structure to indicate that they support the new coercion model.  In
such extension types, the numeric slot functions can no longer assume
that they'll be passed two arguments of the same type; instead they
may be passed two arguments of differing types, and can then perform
their own internal coercion.  If the slot function is passed a type it
can't handle, it can indicate the failure by returning a reference to
the \code{Py_NotImplemented} singleton value.  The numeric functions
of the other type will then be tried, and perhaps they can handle the
operation; if the other type also returns \code{Py_NotImplemented},
then a \exception{TypeError} will be raised.  Numeric methods written
in Python can also return \code{Py_NotImplemented}, causing the
interpreter to act as if the method did not exist (perhaps raising a
\exception{TypeError}, perhaps trying another object's numeric
methods).

\begin{seealso}

\seepep{208}{Reworking the Coercion Model}{Written and implemented by
Neil Schemenauer, heavily based upon earlier work by Marc-Andr\'e
Lemburg.  Read this to understand the fine points of how numeric
operations will now be processed at the C level.}

\end{seealso}

% ======================================================================
\section{Minor Changes and Fixes}

There were relatively few smaller changes made in Python 2.1 due to
the shorter release cycle.  A search through the CVS change logs turns
up 57 patches applied, and 86 bugs fixed; both figures are likely to
be underestimates.  Some of the more notable changes are:

\begin{itemize}


\item The speed of line-oriented file I/O has been improved because
people often complain about its lack of speed, and because it's often
been used as a na\"ive benchmark.  The \method{readline()} method of
file objects has therefore been rewritten to be much faster.  The
exact amount of the speedup will vary from platform to platform
depending on how slow the C library's \function{getc()} was, but is
around 66\%, and potentially much faster on some particular operating
systems.  Tim Peters did much of the benchmarking and coding for this
change, motivated by a discussion in comp.lang.python.

A new module and method for file objects was also added, contributed
by Jeff Epler. The new method, \method{xreadlines()}, is similar to
the existing \function{xrange()} built-in.  \function{xreadlines()}
returns an opaque sequence object that only supports being iterated
over, reading a line on every iteration but not reading the entire
file into memory as the existing \method{readlines()} method does.
You'd use it like this:

\begin{verbatim}
for line in sys.stdin.xreadlines():
    # ... do something for each line ...
    ...
\end{verbatim}

For a fuller discussion of the line I/O changes, see the python-dev
summary for January 1-15, 2001.  

\item A new method, \method{popitem()}, was added to dictionaries to enable 
destructively iterating through the contents of a dictionary; this can be faster for large dictionaries because .  
\code{D.popitem()} removes a random \code{(\var{key}, \var{value})} pair
from the dictionary and returns it as a 2-tuple.  This was implemented
mostly by Tim Peters and Guido van Rossum, after a suggestion and
preliminary patch by Moshe Zadka.
 
\item \module{curses.panel}, a wrapper for the panel library, part of
ncurses and of SYSV curses, was contributed by Thomas Gellekum.  The
panel library provides windows with the additional feature of depth.
Windows can be moved higher or lower in the depth ordering, and the
panel library figures out where panels overlap and which sections are
visible.

\item Modules can now control which names are imported when \code{from
\var{module} import *} is used, by defining an \code{__all__} attribute
containing a list of names that will be imported.  One common
complaint is that if the module imports other modules such as
\module{sys} or \module{string}, \code{from \var{module} import *}
will add them to the importing module's namespace.  To fix this,
simply list the public names in \code{__all__}:

\begin{verbatim}
# List public names
__all__ = ['Database', 'open']
\end{verbatim}

A stricter version of this patch was first suggested and implemented
by Ben Wolfson, but after some python-dev discussion, a weaker
final version was checked in.

\item The PyXML package has gone through a few releases since Python
2.0, and Python 2.1 includes an updated version of the \module{xml}
package.  Some of the noteworthy changes include support for Expat
1.2, the ability for Expat parsers to handle files in any encoding
supported by Python, and various bugfixes for SAX, DOM, and the
\module{minidom} module.

\item Various functions in the \module{time} module, such as
\function{asctime()} and \function{localtime()},
require a floating point argument containing the time in seconds since
the epoch.  The most common use of these functions is to work with the
current time, so the floating point argument has been made optional;
when a value isn't provided, the current time will be used.  For
example, log file entries usually need a string containing the current
time; in Python 2.1, \code{time.asctime()} can be used, instead of the
lengthier \code{time.asctime(time.localtime(time.time()))} that was
previously required.
 
This change was proposed and implemented by Thomas Wouters.

\item The \module{ftplib} module now defaults to retrieving files in passive mode,
because passive mode is more likely to work from behind a firewall.
This request came from the Debian bug tracking system, since other
Debian packages use \module{ftplib} to retrieve files and then don't
work from behind a firewall.  It's deemed unlikely that this will
cause problems for anyone, because Netscape defaults to passive mode
and few people complain, but if passive mode is unsuitable for your
application or network setup, call
\method{set_pasv(0)} on FTP objects to disable passive mode.  

\item The size of the Unicode character database was compressed by another 55K thanks to Fredrik Lundh.

\end{itemize}

And there's the usual list of bugfixes, minor memory leaks, docstring
edits, and other tweaks, too lengthy to be worth itemizing; see the
CVS logs for the full details if you want them.


% ======================================================================
\section{Nested Scopes}

% XXX
The PEP for this new feature hasn't been completed yet, and the
requisite changes haven't been checked into CVS yet.

\begin{seealso}

\seepep{227}{Statically Nested Scopes}{Written and implemented by Jeremy Hylton.}

\end{seealso}


% ======================================================================
\section{Weak References}

% XXX
The PEP for this new feature hasn't been completed yet, and the
requisite changes haven't been checked into CVS yet.


\begin{seealso}

\seepep{205}{Statically Nested Scopes}{Written and implemented by Jeremy Hylton.}

\end{seealso}


% ======================================================================
\section{Acknowledgements}

The author would like to thank the following people for offering
suggestions on various drafts of this article: David Goodger, Michael
Hudson, Marc-Andr\'e Lemburg, Neil Schemenauer, Thomas Wouters.

\end{document}
