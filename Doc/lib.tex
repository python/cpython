\documentstyle[myformat]{report}

% Underscores are not magic throughout this document
\catcode`_=12

% Dummy \cbstart and \cbend so I can leave the changebars in...
\newcommand{\cbstart}{}
\newcommand{\cbend}{}

% Define \itembreak: force the text after an item to start on a new line
\newcommand{\itembreak}{
\mbox{}
\\*[0mm]
}

% Command to generate two index entries (using subentries)
\newcommand{\indexii}[2]{
\index{#1!#2}
\index{#2!#1}
}

% And three entries (using only one level of subentries)
\newcommand{\indexiii}[3]{
\index{#1!#2 #3}
\index{#2!#3, #1}
\index{#3!#1 #2}
}

% And four (again, using only one level of subentries)
\newcommand{\indexiv}[4]{
\index{#1!#2 #3 #4}
\index{#2!#3 #4, #1}
\index{#3!#4, #1 #2}
\index{#4!#1 #2 #3}
}

% Command to generate a reference to a function, statement, keyword, operator
\newcommand{\stindex}[1]{\indexii{statement}{#1@{\tt#1}}}
\newcommand{\kwindex}[1]{\indexii{keyword}{#1@{\tt#1}}}
\newcommand{\opindex}[1]{\indexii{operator}{#1@{\tt#1}}}
\newcommand{\bifuncindex}[1]{\index{#1@{\tt#1} (built-in function)}}

% Add an index entry for a module
\newcommand{\modindex}[2]{\index{#1@{\tt#1} (#2module)}}
\newcommand{\bimodindex}[1]{\modindex{#1}{built-in }}
\newcommand{\stmodindex}[1]{\modindex{#1}{standard }}

% Additional string for an index entry
\newcommand{\indexsubitem}{}
\newcommand{\ttindex}[1]{\index{#1@{\tt#1} \indexsubitem}}

% Define \itemjoin: some negative vspace to join two items together
\newcommand{\itemjoin}{
\mbox{}
\vspace{-\itemsep}
\vspace{-\parsep}
}

% Define \funcitem{func}{args}: define a function item
\newcommand{\funcitem}[2]{
\ttindex{#1}
\item[{\tt #1(#2)}]
\ 
}

% Define \dataitem{name}: define a data item
\newcommand{\dataitem}[1]{
\ttindex{#1}
\item[{\tt #1}]
\ 
}

% Define \excitem{name}: define an exception item
\newcommand{\excitem}[1]{
\ttindex{#1}
\item[{\tt #1}]
\itembreak
}

\title{\bf
	Python Library Reference
}

\author{
	Guido van Rossum \\
	Dept. CST, CWI, Kruislaan 413 \\
	1098 SJ Amsterdam, The Netherlands \\
	E-mail: {\tt guido@cwi.nl}
}

% Tell \index to actually write the .idx file
\makeindex

\begin{document}

\pagenumbering{roman}

\maketitle

\begin{abstract}

\noindent
This document describes the built-in types, exceptions and functions
and the standard modules that come with the Python system.  It assumes
basic knowledge about the Python language.  For an informal
introduction to the language, see the {\em Python Tutorial}.  The {\em
Python Reference Manual} gives a more formal definition of the
language.

\end{abstract}

\pagebreak

{
\parskip = 0mm
\tableofcontents
}

\pagebreak

\pagenumbering{arabic}

\input{lib1.tex}	% intro; built-in types, functions and exceptions
\input{lib2.tex}	% built-in modules
\input{lib3.tex}	% standard modules
\input{lib4.tex}	% OS-dependent chapters
\input{lib5.tex}	% Graphics chapters
\documentstyle[twoside,11pt,myformat]{report}

% NOTE: this file controls which chapters/sections of the library
% manual are actually printed.  It is easy to customize your manual
% by commenting out sections that you're not interested in.

\title{Python Library Reference}

\author{Guido van Rossum\\
	Fred L. Drake, Jr., editor}
\authoraddress{
	BeOpen PythonLabs\\
	E-mail: \email{python-docs@python.org}
}

\date{September 5, 2000}			% XXX update before release!
\release{2.0b1}


\makeindex			% tell \index to actually write the .idx file


\begin{document}

\pagenumbering{roman}

\maketitle

\begin{small}
Copyright \copyright{} 2001 Python Software Foundation.
All rights reserved.

Copyright \copyright{} 2000 BeOpen.com.
All rights reserved.

Copyright \copyright{} 1995-2000 Corporation for National Research Initiatives.
All rights reserved.

Copyright \copyright{} 1991-1995 Stichting Mathematisch Centrum.
All rights reserved.

%%begin{latexonly}
\vskip 4mm
%%end{latexonly}

\centerline{\strong{BEOPEN.COM TERMS AND CONDITIONS FOR PYTHON 2.0}}

\centerline{\strong{BEOPEN PYTHON OPEN SOURCE LICENSE AGREEMENT VERSION 1}}

\begin{enumerate}

\item
This LICENSE AGREEMENT is between BeOpen.com (``BeOpen''), having an
office at 160 Saratoga Avenue, Santa Clara, CA 95051, and the
Individual or Organization (``Licensee'') accessing and otherwise
using this software in source or binary form and its associated
documentation (``the Software'').

\item
Subject to the terms and conditions of this BeOpen Python License
Agreement, BeOpen hereby grants Licensee a non-exclusive,
royalty-free, world-wide license to reproduce, analyze, test, perform
and/or display publicly, prepare derivative works, distribute, and
otherwise use the Software alone or in any derivative version,
provided, however, that the BeOpen Python License is retained in the
Software, alone or in any derivative version prepared by Licensee.

\item
BeOpen is making the Software available to Licensee on an ``AS IS''
basis.  BEOPEN MAKES NO REPRESENTATIONS OR WARRANTIES, EXPRESS OR
IMPLIED.  BY WAY OF EXAMPLE, BUT NOT LIMITATION, BEOPEN MAKES NO AND
DISCLAIMS ANY REPRESENTATION OR WARRANTY OF MERCHANTABILITY OR FITNESS
FOR ANY PARTICULAR PURPOSE OR THAT THE USE OF THE SOFTWARE WILL NOT
INFRINGE ANY THIRD PARTY RIGHTS.

\item
BEOPEN SHALL NOT BE LIABLE TO LICENSEE OR ANY OTHER USERS OF THE
SOFTWARE FOR ANY INCIDENTAL, SPECIAL, OR CONSEQUENTIAL DAMAGES OR LOSS
AS A RESULT OF USING, MODIFYING OR DISTRIBUTING THE SOFTWARE, OR ANY
DERIVATIVE THEREOF, EVEN IF ADVISED OF THE POSSIBILITY THEREOF.

\item
This License Agreement will automatically terminate upon a material
breach of its terms and conditions.

\item
This License Agreement shall be governed by and interpreted in all
respects by the law of the State of California, excluding conflict of
law provisions.  Nothing in this License Agreement shall be deemed to
create any relationship of agency, partnership, or joint venture
between BeOpen and Licensee.  This License Agreement does not grant
permission to use BeOpen trademarks or trade names in a trademark
sense to endorse or promote products or services of Licensee, or any
third party.  As an exception, the ``BeOpen Python'' logos available
at http://www.pythonlabs.com/logos.html may be used according to the
permissions granted on that web page.

\item
By copying, installing or otherwise using the software, Licensee
agrees to be bound by the terms and conditions of this License
Agreement.
\end{enumerate}


\centerline{\strong{CNRI OPEN SOURCE GPL-COMPATIBLE LICENSE AGREEMENT}}

Python 1.6.1 is made available subject to the terms and conditions in
CNRI's License Agreement.  This Agreement together with Python 1.6.1 may
be located on the Internet using the following unique, persistent
identifier (known as a handle): 1895.22/1013.  This Agreement may also
be obtained from a proxy server on the Internet using the following
URL: \url{http://hdl.handle.net/1895.22/1013}.


\centerline{\strong{CWI PERMISSIONS STATEMENT AND DISCLAIMER}}

Copyright \copyright{} 1991 - 1995, Stichting Mathematisch Centrum
Amsterdam, The Netherlands.  All rights reserved.

Permission to use, copy, modify, and distribute this software and its
documentation for any purpose and without fee is hereby granted,
provided that the above copyright notice appear in all copies and that
both that copyright notice and this permission notice appear in
supporting documentation, and that the name of Stichting Mathematisch
Centrum or CWI not be used in advertising or publicity pertaining to
distribution of the software without specific, written prior
permission.

STICHTING MATHEMATISCH CENTRUM DISCLAIMS ALL WARRANTIES WITH REGARD TO
THIS SOFTWARE, INCLUDING ALL IMPLIED WARRANTIES OF MERCHANTABILITY AND
FITNESS, IN NO EVENT SHALL STICHTING MATHEMATISCH CENTRUM BE LIABLE
FOR ANY SPECIAL, INDIRECT OR CONSEQUENTIAL DAMAGES OR ANY DAMAGES
WHATSOEVER RESULTING FROM LOSS OF USE, DATA OR PROFITS, WHETHER IN AN
ACTION OF CONTRACT, NEGLIGENCE OR OTHER TORTIOUS ACTION, ARISING OUT
OF OR IN CONNECTION WITH THE USE OR PERFORMANCE OF THIS SOFTWARE.
\end{small}


\begin{abstract}

\noindent
This document describes the built-in and standard types, exceptions,
functions and modules that come with the Python system.  It assumes
basic knowledge about the Python language.  For an informal
introduction to the language, see the {\em Python Tutorial}.  The {\em
Python Reference Manual} gives a more formal definition of the
language.

\end{abstract}

\pagebreak

{
\parskip = 0mm
\tableofcontents
}

\pagebreak

\pagenumbering{arabic}

				% Chapter title:

\chapter{Introduction}

The Python library consists of three parts, with different levels of
integration with the interpreter.
Closest to the interpreter are built-in types, exceptions and functions.
Next are built-in modules, which are written in \C{} and linked statically
with the interpreter.
Finally there are standard modules that are implemented entirely in
Python, but are always available.
For efficiency, some standard modules may become built-in modules in
future versions of the interpreter.
\indexii{built-in}{types}
\indexii{built-in}{exceptions}
\indexii{built-in}{functions}
\indexii{built-in}{modules}
\indexii{standard}{modules}
\indexii{\C{}}{language}
		% Introduction

\chapter{Built-In Objects \label{builtin}}

Names for built-in exceptions and functions and a number of constants are
found in a separate 
symbol table.  This table is searched last when the interpreter looks
up the meaning of a name, so local and global
user-defined names can override built-in names.  Built-in types are
described together here for easy reference.\footnote{
	Most descriptions sorely lack explanations of the exceptions
	that may be raised --- this will be fixed in a future version of
	this manual.}
\indexii{built-in}{types}
\indexii{built-in}{exceptions}
\indexii{built-in}{functions}
\indexii{built-in}{constants}
\index{symbol table}

The tables in this chapter document the priorities of operators by
listing them in order of ascending priority (within a table) and
grouping operators that have the same priority in the same box.
Binary operators of the same priority group from left to right.
(Unary operators group from right to left, but there you have no real
choice.)  See chapter 5 of the \citetitle[../ref/ref.html]{Python
Reference Manual} for the complete picture on operator priorities.
			% Built-in Types, Exceptions and Functions
\section{\module{types} ---
         Names for all built-in types}

\declaremodule{standard}{types}
\modulesynopsis{Names for all built-in types.}


This module defines names for all object types that are used by the
standard Python interpreter, but not for the types defined by various
extension modules.  It is safe to use \samp{from types import *} ---
the module does not export any names besides the ones listed here.
New names exported by future versions of this module will all end in
\samp{Type}.

Typical use is for functions that do different things depending on
their argument types, like the following:

\begin{verbatim}
from types import *
def delete(list, item):
    if type(item) is IntType:
       del list[item]
    else:
       list.remove(item)
\end{verbatim}

The module defines the following names:

\begin{datadesc}{NoneType}
The type of \code{None}.
\end{datadesc}

\begin{datadesc}{TypeType}
The type of type objects (such as returned by
\function{type()}\bifuncindex{type}).
\end{datadesc}

\begin{datadesc}{IntType}
The type of integers (e.g. \code{1}).
\end{datadesc}

\begin{datadesc}{LongType}
The type of long integers (e.g. \code{1L}).
\end{datadesc}

\begin{datadesc}{FloatType}
The type of floating point numbers (e.g. \code{1.0}).
\end{datadesc}

\begin{datadesc}{ComplexType}
The type of complex numbers (e.g. \code{1.0j}).
\end{datadesc}

\begin{datadesc}{StringType}
The type of character strings (e.g. \code{'Spam'}).
\end{datadesc}

\begin{datadesc}{UnicodeType}
The type of Unicode character strings (e.g. \code{u'Spam'}).
\end{datadesc}

\begin{datadesc}{TupleType}
The type of tuples (e.g. \code{(1, 2, 3, 'Spam')}).
\end{datadesc}

\begin{datadesc}{ListType}
The type of lists (e.g. \code{[0, 1, 2, 3]}).
\end{datadesc}

\begin{datadesc}{DictType}
The type of dictionaries (e.g. \code{\{'Bacon': 1, 'Ham': 0\}}).
\end{datadesc}

\begin{datadesc}{DictionaryType}
An alternate name for \code{DictType}.
\end{datadesc}

\begin{datadesc}{FunctionType}
The type of user-defined functions and lambdas.
\end{datadesc}

\begin{datadesc}{LambdaType}
An alternate name for \code{FunctionType}.
\end{datadesc}

\begin{datadesc}{CodeType}
The type for code objects such as returned by
\function{compile()}\bifuncindex{compile}.
\end{datadesc}

\begin{datadesc}{ClassType}
The type of user-defined classes.
\end{datadesc}

\begin{datadesc}{InstanceType}
The type of instances of user-defined classes.
\end{datadesc}

\begin{datadesc}{MethodType}
The type of methods of user-defined class instances.
\end{datadesc}

\begin{datadesc}{UnboundMethodType}
An alternate name for \code{MethodType}.
\end{datadesc}

\begin{datadesc}{BuiltinFunctionType}
The type of built-in functions like \function{len()} or
\function{sys.exit()}.
\end{datadesc}

\begin{datadesc}{BuiltinMethodType}
An alternate name for \code{BuiltinFunction}.
\end{datadesc}

\begin{datadesc}{ModuleType}
The type of modules.
\end{datadesc}

\begin{datadesc}{FileType}
The type of open file objects such as \code{sys.stdout}.
\end{datadesc}

\begin{datadesc}{XRangeType}
The type of range objects returned by
\function{xrange()}\bifuncindex{xrange}.
\end{datadesc}

\begin{datadesc}{SliceType}
The type of objects returned by
\function{slice()}\bifuncindex{slice}.
\end{datadesc}

\begin{datadesc}{EllipsisType}
The type of \code{Ellipsis}.
\end{datadesc}

\begin{datadesc}{TracebackType}
The type of traceback objects such as found in
\code{sys.exc_traceback}.
\end{datadesc}

\begin{datadesc}{FrameType}
The type of frame objects such as found in \code{tb.tb_frame} if
\code{tb} is a traceback object.
\end{datadesc}

\begin{datadesc}{BufferType}
The type of buffer objects created by the
\function{buffer()}\bifuncindex{buffer} function.
\end{datadesc}

\section{Built-in Exceptions}
\label{module-exceptions}
\stmodindex{exceptions}

Exceptions can be class objects or string objects.  While
traditionally, most exceptions have been string objects, in Python
1.5, all standard exceptions have been converted to class objects,
and users are encouraged to the the same.  The source code for those
exceptions is present in the standard library module
\code{exceptions}; this module never needs to be imported explicitly.

For backward compatibility, when Python is invoked with the \code{-X}
option, the standard exceptions are strings.  This may be needed to
run some code that breaks because of the different semantics of class
based exceptions.  The \code{-X} option will become obsolete in future
Python versions, so the recommended solution is to fix the code.

Two distinct string objects with the same value are considered different
exceptions.  This is done to force programmers to use exception names
rather than their string value when specifying exception handlers.
The string value of all built-in exceptions is their name, but this is
not a requirement for user-defined exceptions or exceptions defined by
library modules.

For class exceptions, in a \code{try} statement with an \code{except}
clause that mentions a particular class, that clause also handles
any exception classes derived from that class (but not exception
classes from which \emph{it} is derived).  Two exception classes
that are not related via subclassing are never equivalent, even if
they have the same name.
\stindex{try}
\stindex{except}

The built-in exceptions listed below can be generated by the
interpreter or built-in functions.  Except where mentioned, they have
an ``associated value'' indicating the detailed cause of the error.
This may be a string or a tuple containing several items of
information (e.g., an error code and a string explaining the code).
The associated value is the second argument to the \code{raise}
statement.  For string exceptions, the associated value itself will be
stored in the variable named as the second argument of the
\code{except} clause (if any).  For class exceptions derived from
the root class \code{Exception}, that variable receives the exception
instance, and the associated value is present as the exception
instance's \code{args} attribute; this is a tuple even if the second
argument to \code{raise} was not (then it is a singleton tuple).
\stindex{raise}

User code can raise built-in exceptions.  This can be used to test an
exception handler or to report an error condition ``just like'' the
situation in which the interpreter raises the same exception; but
beware that there is nothing to prevent user code from raising an
inappropriate error.

\setindexsubitem{(built-in exception base class)}

The following exceptions are only used as base classes for other
exceptions.  When string-based standard exceptions are used, they
are tuples containing the directly derived classes.

\begin{excdesc}{Exception}
The root class for exceptions.  All built-in exceptions are derived
from this class.  All user-defined exceptions should also be derived
from this class, but this is not (yet) enforced.  The \code{str()}
function, when applied to an instance of this class (or most derived
classes) returns the string value of the argument or arguments, or an
empty string if no arguments were given to the constructor.  When used
as a sequence, this accesses the arguments given to the constructor
(handy for backward compatibility with old code).
\end{excdesc}

\begin{excdesc}{StandardError}
The base class for built-in exceptions.  All built-in exceptions are
derived from this class, which is itself derived from the root class
\code{Exception}.
\end{excdesc}

\begin{excdesc}{ArithmeticError}
The base class for those built-in exceptions that are raised for
various arithmetic errors: \code{OverflowError},
\code{ZeroDivisionError}, \code{FloatingPointError}.
\end{excdesc}

\begin{excdesc}{LookupError}
The base class for thise exceptions that are raised when a key or
index used on a mapping or sequence is invalid: \code{IndexError},
\code{KeyError}.
\end{excdesc}

\setindexsubitem{(built-in exception)}

The following exceptions are the exceptions that are actually raised.
They are class objects, except when the \code{-X} option is used to
revert back to string-based standard exceptions.

\begin{excdesc}{AssertionError}
Raised when an \code{assert} statement fails.
\stindex{assert}
\end{excdesc}

\begin{excdesc}{AttributeError}
% xref to attribute reference?
  Raised when an attribute reference or assignment fails.  (When an
  object does not support attribute references or attribute assignments
  at all, \code{TypeError} is raised.)
\end{excdesc}

\begin{excdesc}{EOFError}
% XXXJH xrefs here
  Raised when one of the built-in functions (\code{input()} or
  \code{raw_input()}) hits an end-of-file condition (\EOF{}) without
  reading any data.
% XXXJH xrefs here
  (N.B.: the \code{read()} and \code{readline()} methods of file
  objects return an empty string when they hit \EOF{}.)  No associated value.
\end{excdesc}

\begin{excdesc}{FloatingPointError}
Raised when a floating point operation fails.  This exception is
always defined, but can only be raised when Python is configured with
the \code{--with-fpectl} option, or the \code{WANT_SIGFPE_HANDLER}
symbol is defined in the \file{config.h} file.
\end{excdesc}

\begin{excdesc}{IOError}
% XXXJH xrefs here
  Raised when an I/O operation (such as a \code{print} statement, the
  built-in \code{open()} function or a method of a file object) fails
  for an I/O-related reason, e.g., ``file not found'' or ``disk full''.

When class exceptions are used, and this exception is instantiated as
\code{IOError(errno, strerror)}, the instance has two additional
attributes \code{errno} and \code{strerror} set to the error code and
the error message, respectively.  These attributes default to
\code{None}.
\end{excdesc}

\begin{excdesc}{ImportError}
% XXXJH xref to import statement?
  Raised when an \code{import} statement fails to find the module
  definition or when a \code{from {\rm \ldots} import} fails to find a
  name that is to be imported.
\end{excdesc}

\begin{excdesc}{IndexError}
% XXXJH xref to sequences
  Raised when a sequence subscript is out of range.  (Slice indices are
  silently truncated to fall in the allowed range; if an index is not a
  plain integer, \code{TypeError} is raised.)
\end{excdesc}

\begin{excdesc}{KeyError}
% XXXJH xref to mapping objects?
  Raised when a mapping (dictionary) key is not found in the set of
  existing keys.
\end{excdesc}

\begin{excdesc}{KeyboardInterrupt}
  Raised when the user hits the interrupt key (normally
  \kbd{Control-C} or \kbd{DEL}).  During execution, a check for
  interrupts is made regularly.
% XXXJH xrefs here
  Interrupts typed when a built-in function \function{input()} or
  \function{raw_input()}) is waiting for input also raise this
  exception.  This exception has no associated value.
\end{excdesc}

\begin{excdesc}{MemoryError}
  Raised when an operation runs out of memory but the situation may
  still be rescued (by deleting some objects).  The associated value is
  a string indicating what kind of (internal) operation ran out of memory.
  Note that because of the underlying memory management architecture
  (\C{}'s \code{malloc()} function), the interpreter may not always be able
  to completely recover from this situation; it nevertheless raises an
  exception so that a stack traceback can be printed, in case a run-away
  program was the cause.
\end{excdesc}

\begin{excdesc}{NameError}
  Raised when a local or global name is not found.  This applies only
  to unqualified names.  The associated value is the name that could
  not be found.
\end{excdesc}

\begin{excdesc}{OverflowError}
% XXXJH reference to long's and/or int's?
  Raised when the result of an arithmetic operation is too large to be
  represented.  This cannot occur for long integers (which would rather
  raise \code{MemoryError} than give up).  Because of the lack of
  standardization of floating point exception handling in \C{}, most
  floating point operations also aren't checked.  For plain integers,
  all operations that can overflow are checked except left shift, where
  typical applications prefer to drop bits than raise an exception.
\end{excdesc}

\begin{excdesc}{RuntimeError}
  Raised when an error is detected that doesn't fall in any of the
  other categories.  The associated value is a string indicating what
  precisely went wrong.  (This exception is mostly a relic from a
  previous version of the interpreter; it is not used very much any
  more.)
\end{excdesc}

\begin{excdesc}{SyntaxError}
% XXXJH xref to these functions?
  Raised when the parser encounters a syntax error.  This may occur in
  an \code{import} statement, in an \code{exec} statement, in a call
  to the built-in function \code{eval()} or \code{input()}, or
  when reading the initial script or standard input (also
  interactively).

When class exceptions are used, instances of this class have
atttributes \code{filename}, \code{lineno}, \code{offset} and
\code{text} for easier access to the details; for string exceptions,
the associated value is usually a tuple of the form
\code{(message, (filename, lineno, offset, text))}.
For class exceptions, \code{str()} returns only the message.
\end{excdesc}

\begin{excdesc}{SystemError}
  Raised when the interpreter finds an internal error, but the
  situation does not look so serious to cause it to abandon all hope.
  The associated value is a string indicating what went wrong (in
  low-level terms).
  
  You should report this to the author or maintainer of your Python
  interpreter.  Be sure to report the version string of the Python
  interpreter (\code{sys.version}; it is also printed at the start of an
  interactive Python session), the exact error message (the exception's
  associated value) and if possible the source of the program that
  triggered the error.
\end{excdesc}

\begin{excdesc}{SystemExit}
% XXXJH xref to module sys?
  This exception is raised by the \code{sys.exit()} function.  When it
  is not handled, the Python interpreter exits; no stack traceback is
  printed.  If the associated value is a plain integer, it specifies the
  system exit status (passed to \C{}'s \code{exit()} function); if it is
  \code{None}, the exit status is zero; if it has another type (such as
  a string), the object's value is printed and the exit status is one.

When class exceptions are used, the instance has an attribute
\code{code} which is set to the proposed exit status or error message
(defaulting to \code{None}).
  
  A call to \code{sys.exit()} is translated into an exception so that
  clean-up handlers (\code{finally} clauses of \code{try} statements)
  can be executed, and so that a debugger can execute a script without
  running the risk of losing control.  The \code{os._exit()} function
  can be used if it is absolutely positively necessary to exit
  immediately (e.g., after a \code{fork()} in the child process).
\end{excdesc}

\begin{excdesc}{TypeError}
  Raised when a built-in operation or function is applied to an object
  of inappropriate type.  The associated value is a string giving
  details about the type mismatch.
\end{excdesc}

\begin{excdesc}{ValueError}
  Raised when a built-in operation or function receives an argument
  that has the right type but an inappropriate value, and the
  situation is not described by a more precise exception such as
  \code{IndexError}.
\end{excdesc}

\begin{excdesc}{ZeroDivisionError}
  Raised when the second argument of a division or modulo operation is
  zero.  The associated value is a string indicating the type of the
  operands and the operation.
\end{excdesc}

\section{Built-in Functions}

The Python interpreter has a number of functions built into it that
are always available.  They are listed here in alphabetical order.


\renewcommand{\indexsubitem}{(built-in function)}
\begin{funcdesc}{abs}{x}
  Return the absolute value of a number.  The argument may be a plain
  or long integer or a floating point number.
\end{funcdesc}

\begin{funcdesc}{apply}{function\, args}
The \var{function} argument must be a callable object (a user-defined or
built-in function or method, or a class object) and the \var{args}
argument must be a tuple.  The \var{function} is called with
\var{args} as argument list; the number of arguments is the the length
of the tuple.  (This is different from just calling
\code{\var{func}(\var{args})}, since in that case there is always
exactly one argument.)
\end{funcdesc}

\begin{funcdesc}{chr}{i}
  Return a string of one character whose \ASCII{} code is the integer
  \var{i}, e.g., \code{chr(97)} returns the string \code{'a'}.  This is the
  inverse of \code{ord()}.  The argument must be in the range [0..255],
  inclusive.
\end{funcdesc}

\begin{funcdesc}{cmp}{x\, y}
  Compare the two objects \var{x} and \var{y} and return an integer
  according to the outcome.  The return value is negative if \code{\var{x}
  < \var{y}}, zero if \code{\var{x} == \var{y}} and strictly positive if
  \code{\var{x} > \var{y}}.
\end{funcdesc}

\begin{funcdesc}{coerce}{x\, y}
  Return a tuple consisting of the two numeric arguments converted to
  a common type, using the same rules as used by arithmetic
  operations.
\end{funcdesc}

\begin{funcdesc}{compile}{string\, filename\, kind}
  Compile the \var{string} into a code object.  Code objects can be
  executed by a \code{exec()} statement or evaluated by a call to
  \code{eval()}.  The \var{filename} argument should
  give the file from which the code was read; pass e.g. \code{'<string>'}
  if it wasn't read from a file.  The \var{kind} argument specifies
  what kind of code must be compiled; it can be \code{'exec'} if
  \var{string} consists of a sequence of statements, or \code{'eval'}
  if it consists of a single expression.
\end{funcdesc}

\begin{funcdesc}{delattr}{object\, name}
  This is a relative of \code{setattr}.  The arguments are an
  object and a string.  The string must be the name
  of one of the object's attributes.  The function deletes
  the named attribute, provided the object allows it.  For example,
  \code{setattr(\var{x}, '\var{foobar}')} is equivalent to
  \code{del \var{x}.\var{foobar}}.
\end{funcdesc}

\begin{funcdesc}{dir}{}
  Without arguments, return the list of names in the current local
  symbol table.  With a module, class or class instance object as
  argument (or anything else that has a \code{__dict__} attribute),
  returns the list of names in that object's attribute dictionary.
  The resulting list is sorted.  For example:

\bcode\begin{verbatim}
>>> import sys
>>> dir()
['sys']
>>> dir(sys)
['argv', 'exit', 'modules', 'path', 'stderr', 'stdin', 'stdout']
>>> 
\end{verbatim}\ecode
\end{funcdesc}

\begin{funcdesc}{divmod}{a\, b}
  Take two numbers as arguments and return a pair of integers
  consisting of their integer quotient and remainder.  With mixed
  operand types, the rules for binary arithmetic operators apply.  For
  plain and long integers, the result is the same as
  \code{(\var{a} / \var{b}, \var{a} \%{} \var{b})}.
  For floating point numbers the result is the same as
  \code{(math.floor(\var{a} / \var{b}), \var{a} \%{} \var{b})}.
\end{funcdesc}

\begin{funcdesc}{eval}{expression\optional{\, globals\optional{\, locals}}}
  The arguments are a string and two optional dictionaries.  The
  \var{expression} argument is parsed and evaluated as a Python
  expression (technically speaking, a condition list) using the
  \var{globals} and \var{locals} dictionaries as global and local name
  space.  If the \var{globals} dictionary is omitted it defaults to
  the \var{locals} dictionary.  If both dictionaries are omitted, the
  expression is executed in the environment where \code{eval} is
  called.  The return value is the result of the evaluated expression.
  Syntax errors are reported as exceptions.  Example:

\bcode\begin{verbatim}
>>> x = 1
>>> print eval('x+1')
2
>>> 
\end{verbatim}\ecode

  This function can also be used to execute arbitrary code objects
  (e.g. created by \code{compile()}).  In this case pass a code
  object instead of a string.  The code object must have been compiled
  passing \code{'eval'} to the \var{kind} argument.

  Note: dynamic execution of statements is supported by the
  \code{exec} statement.  Execution of statements from a file is
  supported by the \code{execfile()} function.

\end{funcdesc}

\begin{funcdesc}{execfile}{file\optional{\, globals\optional{\, locals}}}
  This function is similar to the \code{eval()} function or the
  \code{exec} statement, but parses a file instead of a string.  It is
  different from the \code{import} statement in that it does not use
  the module administration -- it reads the file unconditionally and
  does not create a new module.

  The arguments are a file name and two optional dictionaries.  The
  file is parsed and evaluated as a sequence of Python statements
  (similarly to a module) using the \var{globals} and \var{locals}
  dictionaries as global and local name space.  If the \var{globals}
  dictionary is omitted it defaults to the \var{locals} dictionary.
  If both dictionaries are omitted, the expression is executed in the
  environment where \code{execfile} is called.  The return value is
  None.
\end{funcdesc}

\begin{funcdesc}{filter}{function\, list}
Construct a list from those elements of \var{list} for which
\var{function} returns true.  If \var{list} is a string or a tuple,
the result also has that type; otherwise it is always a list.  If
\var{function} is \code{None}, the identity function is assumed,
i.e. all elements of \var{list} that are false (zero or empty) are
removed.
\end{funcdesc}

\begin{funcdesc}{float}{x}
  Convert a number to floating point.  The argument may be a plain or
  long integer or a floating point number.
\end{funcdesc}

\begin{funcdesc}{getattr}{object\, name}
  The arguments are an object and a string.  The string must be the
  name
  of one of the object's attributes.  The result is the value of that
  attribute.  For example, \code{getattr(\var{x}, '\var{foobar}')} is equivalent to
  \code{\var{x}.\var{foobar}}.
\end{funcdesc}

\begin{funcdesc}{hasattr}{object\, name}
  The arguments are an object and a string.  The result is 1 if the
  string is the name of one of the object's attributes, 0 if not.
  (This is implemented by calling \code{getattr(object, name)} and
  seeing whether it raises an exception or not.)
\end{funcdesc}

\begin{funcdesc}{hash}{object}
  Return the hash value of the object (if it has one).  Hash values
  are 32-bit integers.  They are used to quickly compare dictionary
  keys during a dictionary lookup.  Numeric values that compare equal
  have the same hash value (even if they are of different types, e.g.
  1 and 1.0).
\end{funcdesc}

\begin{funcdesc}{hex}{x}
  Convert a number to a hexadecimal string.  The result is a valid
  Python expression.
\end{funcdesc}

\begin{funcdesc}{id}{object}
  Return the `identity' of an object.  This is an integer which is
  guaranteed to be unique and constant for this object during its
  lifetime.  (Two objects whose lifetimes are disjunct may have the
  same id() value.)  (Implementation note: this is the address of the
  object.)
\end{funcdesc}

\begin{funcdesc}{input}{\optional{prompt}}
  Almost equivalent to \code{eval(raw_input(\var{prompt}))}.  Like
  \code{raw_input()}, the \var{prompt} argument is optional.  The difference
  is that a long input expression may be broken over multiple lines using
  the backslash convention.
\end{funcdesc}

\begin{funcdesc}{int}{x}
  Convert a number to a plain integer.  The argument may be a plain or
  long integer or a floating point number.
\end{funcdesc}

\begin{funcdesc}{len}{s}
  Return the length (the number of items) of an object.  The argument
  may be a sequence (string, tuple or list) or a mapping (dictionary).
\end{funcdesc}

\begin{funcdesc}{long}{x}
  Convert a number to a long integer.  The argument may be a plain or
  long integer or a floating point number.
\end{funcdesc}

\begin{funcdesc}{map}{function\, list\, ...}
Apply \var{function} to every item of \var{list} and return a list
of the results.  If additional \var{list} arguments are passed, 
\var{function} must take that many arguments and is applied to
the items of all lists in parallel; if a list is shorter than another
it is assumed to be extended with \code{None} items.  If
\var{function} is \code{None}, the identity function is assumed; if
there are multiple list arguments, \code{map} returns a list
consisting of tuples containing the corresponding items from all lists
(i.e. a kind of transpose operation).  The \var{list} arguments may be
any kind of sequence; the result is always a list.
\end{funcdesc}

\begin{funcdesc}{max}{s}
  Return the largest item of a non-empty sequence (string, tuple or
  list).
\end{funcdesc}

\begin{funcdesc}{min}{s}
  Return the smallest item of a non-empty sequence (string, tuple or
  list).
\end{funcdesc}

\begin{funcdesc}{oct}{x}
  Convert a number to an octal string.  The result is a valid Python
  expression.
\end{funcdesc}

\begin{funcdesc}{open}{filename\, \optional{mode\optional{\, bufsize}}}
  Return a new file object (described earlier under Built-in Types).
  The first two arguments are the same as for \code{stdio}'s
  \code{fopen()}: \var{filename} is the file name to be opened,
  \var{mode} indicates how the file is to be opened: \code{'r'} for
  reading, \code{'w'} for writing (truncating an existing file), and
  \code{'a'} opens it for appending.  Modes \code{'r+'}, \code{'w+'} and
  \code{'a+'} open the file for updating, provided the underlying
  \code{stdio} library understands this.  On systems that differentiate
  between binary and text files, \code{'b'} appended to the mode opens
  the file in binary mode.  If the file cannot be opened, \code{IOError}
  is raised.
If \var{mode} is omitted, it defaults to \code{'r'}.
The optional \var{bufsize} argument specifies the file's desired
buffer size: 0 means unbuffered, 1 means line buffered, any other
positive value means use a buffer of (approximately) that size.  A
negative \var{bufsize} means to use the system default, which is
usually line buffered for for tty devices and fully buffered for other
files.%
\footnote{Specifying a buffer size currently has no effect on systems
that don't have \code{setvbuf()}.  The interface to specify the buffer
size is not done using a method that calls \code{setvbuf()}, because
that may dump core when called after any I/O has been performed, and
there's no reliable way to determine whether this is the case.}
\end{funcdesc}

\begin{funcdesc}{ord}{c}
  Return the \ASCII{} value of a string of one character.  E.g.,
  \code{ord('a')} returns the integer \code{97}.  This is the inverse of
  \code{chr()}.
\end{funcdesc}

\begin{funcdesc}{pow}{x\, y\optional{\, z}}
  Return \var{x} to the power \var{y}; if \var{z} is present, return
  \var{x} to the power \var{y}, modulo \var{z} (computed more
  efficiently that \code{pow(\var{x}, \var{y}) \% \var{z}}).
  The arguments must have
  numeric types.  With mixed operand types, the rules for binary
  arithmetic operators apply.  The effective operand type is also the
  type of the result; if the result is not expressible in this type, the
  function raises an exception; e.g., \code{pow(2, -1)} or \code{pow(2,
  35000)} is not allowed.
\end{funcdesc}

\begin{funcdesc}{range}{\optional{start\,} end\optional{\, step}}
  This is a versatile function to create lists containing arithmetic
  progressions.  It is most often used in \code{for} loops.  The
  arguments must be plain integers.  If the \var{step} argument is
  omitted, it defaults to \code{1}.  If the \var{start} argument is
  omitted, it defaults to \code{0}.  The full form returns a list of
  plain integers \code{[\var{start}, \var{start} + \var{step},
  \var{start} + 2 * \var{step}, \ldots]}.  If \var{step} is positive,
  the last element is the largest \code{\var{start} + \var{i} *
  \var{step}} less than \var{end}; if \var{step} is negative, the last
  element is the largest \code{\var{start} + \var{i} * \var{step}}
  greater than \var{end}.  \var{step} must not be zero.  Example:

\bcode\begin{verbatim}
>>> range(10)
[0, 1, 2, 3, 4, 5, 6, 7, 8, 9]
>>> range(1, 11)
[1, 2, 3, 4, 5, 6, 7, 8, 9, 10]
>>> range(0, 30, 5)
[0, 5, 10, 15, 20, 25]
>>> range(0, 10, 3)
[0, 3, 6, 9]
>>> range(0, -10, -1)
[0, -1, -2, -3, -4, -5, -6, -7, -8, -9]
>>> range(0)
[]
>>> range(1, 0)
[]
>>> 
\end{verbatim}\ecode
\end{funcdesc}

\begin{funcdesc}{raw_input}{\optional{prompt}}
  If the \var{prompt} argument is present, it is written to standard output
  without a trailing newline.  The function then reads a line from input,
  converts it to a string (stripping a trailing newline), and returns that.
  When \EOF{} is read, \code{EOFError} is raised. Example:

\bcode\begin{verbatim}
>>> s = raw_input('--> ')
--> Monty Python's Flying Circus
>>> s
'Monty Python\'s Flying Circus'
>>> 
\end{verbatim}\ecode
\end{funcdesc}

\begin{funcdesc}{reduce}{function\, list\optional{\, initializer}}
Apply the binary \var{function} to the items of \var{list} so as to
reduce the list to a single value.  E.g.,
\code{reduce(lambda x, y: x*y, \var{list}, 1)} returns the product of
the elements of \var{list}.  The optional \var{initializer} can be
thought of as being prepended to \var{list} so as to allow reduction
of an empty \var{list}.  The \var{list} arguments may be any kind of
sequence.
\end{funcdesc}

\begin{funcdesc}{reload}{module}
  Re-parse and re-initialize an already imported \var{module}.  The
  argument must be a module object, so it must have been successfully
  imported before.  This is useful if you have edited the module source
  file using an external editor and want to try out the new version
  without leaving the Python interpreter.  Note that if a module is
  syntactically correct but its initialization fails, the first
  \code{import} statement for it does not import the name, but does
  create a (partially initialized) module object; to reload the module
  you must first \code{import} it again (this will just make the
  partially initialized module object available) before you can
  \code{reload()} it.
\end{funcdesc}

\begin{funcdesc}{repr}{object}
Return a string containing a printable representation of an object.
This is the same value yielded by conversions (reverse quotes).
It is sometimes useful to be able to access this operation as an
ordinary function.  For many types, this function makes an attempt
to return a string that would yield an object with the same value
when passed to \code{eval()}.
\end{funcdesc}

\begin{funcdesc}{round}{x\, n}
  Return the floating point value \var{x} rounded to \var{n} digits
  after the decimal point.  If \var{n} is omitted, it defaults to zero.
  The result is a floating point number.  Values are rounded to the
  closest multiple of 10 to the power minus \var{n}; if two multiples
  are equally close, rounding is done away from 0 (so e.g.
  \code{round(0.5)} is \code{1.0} and \code{round(-0.5)} is \code{-1.0}).
\end{funcdesc}

\begin{funcdesc}{setattr}{object\, name\, value}
  This is the counterpart of \code{getattr}.  The arguments are an
  object, a string and an arbitrary value.  The string must be the name
  of one of the object's attributes.  The function assigns the value to
  the attribute, provided the object allows it.  For example,
  \code{setattr(\var{x}, '\var{foobar}', 123)} is equivalent to
  \code{\var{x}.\var{foobar} = 123}.
\end{funcdesc}

\begin{funcdesc}{str}{object}
Return a string containing a nicely printable representation of an
object.  For strings, this returns the string itself.  The difference
with \code{repr(\var{object}} is that \code{str(\var{object}} does not
always attempt to return a string that is acceptable to \code{eval()};
its goal is to return a printable string.
\end{funcdesc}

\begin{funcdesc}{tuple}{object}
Return a tuple whose items are the same and in the same order as
\var{object}'s items.  If \var{object} is alread a tuple, it
is returned unchanged.  For instance, \code{tuple('abc')} returns
returns \code{('a', 'b', 'c')} and \code{tuple([1, 2, 3])} returns
\code{(1, 2, 3)}.
\end{funcdesc}

\begin{funcdesc}{type}{object}
% XXXJH xref to buil-in objects here?
  Return the type of an \var{object}.  The return value is a type
  object.  There is not much you can do with type objects except compare
  them to other type objects; e.g., the following checks if a variable
  is a string:

\bcode\begin{verbatim}
>>> if type(x) == type(''): print 'It is a string'
\end{verbatim}\ecode
\end{funcdesc}

\begin{funcdesc}{vars}{}
Without arguments, return a dictionary corresponding to the current
local symbol table.  With a module, class or class instance object as
argument (or anything else that has a \code{__dict__} attribute),
returns a dictionary corresponding to the object's symbol table.
The returned dictionary should not be modified: the effects on the
corresponding symbol table are undefined.%
\footnote{In the current implementation, local variable bindings
cannot normally be affected this way, but variables retrieved from
other scopes can be.  This may change.}
\end{funcdesc}

\begin{funcdesc}{xrange}{\optional{start\,} end\optional{\, step}}
This function is very similar to \code{range()}, but returns an
``xrange object'' instead of a list.  This is an opaque sequence type
which yields the same values as the corresponding list, without
actually storing them all simultaneously.  The advantage of
\code{xrange()} over \code{range()} is minimal (since \code{xrange()}
still has to create the values when asked for them) except when a very
large range is used on a memory-starved machine (e.g. DOS) or when all
of the range's elements are never used (e.g. when the loop is usually
terminated with \code{break}).
\end{funcdesc}


\chapter{Python Services}
\label{python}

The modules described in this chapter provide a wide range of services
related to the Python interpreter and its interaction with its
environment.  Here's an overview:

\begin{description}

\item[sys]
--- Access system specific parameters and functions.

\item[types]
--- Names for all built-in types.

\item[UserDict]
--- Class wrapper for dictionary objects.

\item[UserList]
--- Class wrapper for list objects.

\item[operator]
--- All Python's standard operators as built-in functions.

\item[traceback]
--- Print or retrieve a stack traceback.

\item[pickle]
--- Convert Python objects to streams of bytes and back.

\item[cPickle]
--- Faster version of \module{pickle}, but not subclassable.

\item[copy_reg]
--- Register \module{pickle} support functions.

\item[shelve]
--- Python object persistency.

\item[copy]
--- Shallow and deep copy operations.

\item[marshal]
--- Convert Python objects to streams of bytes and back (with
different constraints).

\item[imp]
--- Access the implementation of the \keyword{import} statement.

\item[parser]
--- Retrieve and submit parse trees from and to the runtime support
environment.

\item[symbol]
--- Constants representing internal nodes of the parse tree.

\item[token]
--- Constants representing terminal nodes of the parse tree.

\item[keyword]
--- Test whether a string is a keyword in the Python language.

\item[code]
--- Code object services.

\item[pprint]
--- Data pretty printer.

\item[dis]
--- Disassembler.

\item[site]
--- A standard way to reference site-specific modules.

\item[user]
--- A standard way to reference user-specific modules.

\item[__builtin__]
--- The set of built-in functions.

\item[__main__]
--- The environment where the top-level script is run.

\end{description}
		% Python Services
\section{Built-in Module \sectcode{sys}}
\label{module-sys}

\bimodindex{sys}
This module provides access to some variables used or maintained by the
interpreter and to functions that interact strongly with the interpreter.
It is always available.

\setindexsubitem{(in module sys)}

\begin{datadesc}{argv}
  The list of command line arguments passed to a Python script.
  \code{argv[0]} is the script name (it is operating system
  dependent whether this is a full pathname or not).
  If the command was executed using the \samp{-c} command line option
  to the interpreter, \code{argv[0]} is set to the string
  \code{"-c"}.
  If no script name was passed to the Python interpreter,
  \code{argv} has zero length.
\end{datadesc}

\begin{datadesc}{builtin_module_names}
  A tuple of strings giving the names of all modules that are compiled
  into this Python interpreter.  (This information is not available in
  any other way --- \code{modules.keys()} only lists the imported
  modules.)
\end{datadesc}

\begin{funcdesc}{exc_info}{}
This function returns a tuple of three values that give information
about the exception that is currently being handled.  The information
returned is specific both to the current thread and to the current
stack frame.  If the current stack frame is not handling an exception,
the information is taken from the calling stack frame, or its caller,
and so on until a stack frame is found that is handling an exception.
Here, ``handling an exception'' is defined as ``executing or having
executed an except clause.''  For any stack frame, only
information about the most recently handled exception is accessible.

If no exception is being handled anywhere on the stack, a tuple
containing three \code{None} values is returned.  Otherwise, the
values returned are
\code{(\var{type}, \var{value}, \var{traceback})}.
Their meaning is: \var{type} gets the exception type of the exception
being handled (a string or class object); \var{value} gets the
exception parameter (its \dfn{associated value} or the second argument
to \keyword{raise}, which is always a class instance if the exception
type is a class object); \var{traceback} gets a traceback object (see
the Reference Manual) which encapsulates the call stack at the point
where the exception originally occurred.
\obindex{traceback}

\strong{Warning:} assigning the \var{traceback} return value to a
local variable in a function that is handling an exception will cause
a circular reference. This will prevent anything referenced by a local
variable in the same function or by the traceback from being garbage
collected.  Since most functions don't need access to the traceback,
the best solution is to use something like
\code{type, value = sys.exc_info()[:2]}
to extract only the exception type and value.  If you do need the
traceback, make sure to delete it after use (best done with a
\keyword{try} ... \keyword{finally} statement) or to call
\function{exc_info()} in a function that does not itself handle an
exception.
\end{funcdesc}

\begin{datadesc}{exc_type}
\dataline{exc_value}
\dataline{exc_traceback}
\deprecated {1.5}
            {Use \function{exc_info()} instead.}
Since they are global variables, they are not specific to the current
thread, so their use is not safe in a multi-threaded program.  When no
exception is being handled, \code{exc_type} is set to \code{None} and
the other two are undefined.
\end{datadesc}

\begin{datadesc}{exec_prefix}
A string giving the site-specific
directory prefix where the platform-dependent Python files are
installed; by default, this is also \code{"/usr/local"}.  This can be
set at build time with the \code{-}\code{-exec-prefix} argument to the
\program{configure} script.  Specifically, all configuration files
(e.g. the \file{config.h} header file) are installed in the directory
\code{exec_prefix + "/lib/python\var{version}/config"}, and shared library
modules are installed in
\code{exec_prefix + "/lib/python\var{version}/lib-dynload"},
where \var{version} is equal to \code{version[:3]}.
\end{datadesc}

\begin{funcdesc}{exit}{n}
  Exit from Python with numeric exit status \var{n}.  This is
  implemented by raising the \exception{SystemExit} exception, so cleanup
  actions specified by finally clauses of \keyword{try} statements
  are honored, and it is possible to catch the exit attempt at an outer
  level.
\end{funcdesc}

\begin{datadesc}{exitfunc}
  This value is not actually defined by the module, but can be set by
  the user (or by a program) to specify a clean-up action at program
  exit.  When set, it should be a parameterless function.  This function
  will be called when the interpreter exits in any way (except when a
  fatal error occurs: in that case the interpreter's internal state
  cannot be trusted).
\end{datadesc}

\begin{funcdesc}{getrefcount}{object}
Return the reference count of the \var{object}.  The count returned is
generally one higher than you might expect, because it includes the
(temporary) reference as an argument to \code{getrefcount()}.
\end{funcdesc}

\begin{datadesc}{last_type}
\dataline{last_value}
\dataline{last_traceback}
These three variables are not always defined; they are set when an
exception is not handled and the interpreter prints an error message
and a stack traceback.  Their intended use is to allow an interactive
user to import a debugger module and engage in post-mortem debugging
without having to re-execute the command that caused the error.
(Typical use is \samp{import pdb; pdb.pm()} to enter the post-mortem
debugger; see the chapter ``The Python Debugger'' for more
information.)
\refstmodindex{pdb}

The meaning of the variables is the same
as that of the return values from \function{exc_info()} above.
(Since there is only one interactive thread, thread-safety is not a
concern for these variables, unlike for \code{exc_type} etc.)
\end{datadesc}

\begin{datadesc}{modules}
  This is a dictionary that maps module names to modules which have
  already been loaded.  This can be manipulated to force reloading of
  modules and other tricks.  Note that removing a module from this
  dictionary is \emph{not} the same as calling
  \function{reload()}\bifuncindex{reload} on the corresponding module
  object.
\end{datadesc}

\begin{datadesc}{path}
\indexiii{module}{search}{path}
  A list of strings that specifies the search path for modules.
  Initialized from the environment variable \code{\$PYTHONPATH}, or an
  installation-dependent default.  

The first item of this list, \code{path[0]}, is the 
directory containing the script that was used to invoke the Python 
interpreter.  If the script directory is not available (e.g.  if the 
interpreter is invoked interactively or if the script is read from 
standard input), \code{path[0]} is the empty string, which directs 
Python to search modules in the current directory first.  Notice that 
the script directory is inserted \emph{before} the entries inserted as 
a result of \code{\$PYTHONPATH}.  
\end{datadesc}

\begin{datadesc}{platform}
This string contains a platform identifier, e.g. \code{'sunos5'} or
\code{'linux1'}.  This can be used to append platform-specific
components to \code{path}, for instance. 
\end{datadesc}

\begin{datadesc}{prefix}
A string giving the site-specific directory prefix where the platform
independent Python files are installed; by default, this is the string
\code{"/usr/local"}.  This can be set at build time with the
\code{-}\code{-prefix} argument to the \program{configure} script.  The main
collection of Python library modules is installed in the directory
\code{prefix + "/lib/python\var{version}"} while the platform
independent header files (all except \file{config.h}) are stored in
\code{prefix + "/include/python\var{version}"},
where \var{version} is equal to \code{version[:3]}.

\end{datadesc}

\begin{datadesc}{ps1}
\dataline{ps2}
\index{interpreter prompts}
\index{prompts, interpreter}
  Strings specifying the primary and secondary prompt of the
  interpreter.  These are only defined if the interpreter is in
  interactive mode.  Their initial values in this case are
  \code{'>>> '} and \code{'... '}.  If a non-string object is assigned
  to either variable, its \function{str()} is re-evaluated each time
  the interpreter prepares to read a new interactive command; this can
  be used to implement a dynamic prompt.
\end{datadesc}

\begin{funcdesc}{setcheckinterval}{interval}
Set the interpreter's ``check interval''.  This integer value
determines how often the interpreter checks for periodic things such
as thread switches and signal handlers.  The default is \code{10}, meaning
the check is performed every 10 Python virtual instructions.  Setting
it to a larger value may increase performance for programs using
threads.  Setting it to a value \code{<=} 0 checks every virtual instruction,
maximizing responsiveness as well as overhead.
\end{funcdesc}

\begin{funcdesc}{settrace}{tracefunc}
  Set the system's trace function, which allows you to implement a
  Python source code debugger in Python.  See section ``How It Works''
  in the chapter on the Python Debugger.
\end{funcdesc}
\index{trace function}
\index{debugger}

\begin{funcdesc}{setprofile}{profilefunc}
  Set the system's profile function, which allows you to implement a
  Python source code profiler in Python.  See the chapter on the
  Python Profiler.  The system's profile function
  is called similarly to the system's trace function (see
  \function{settrace()}), but it isn't called for each executed line of
  code (only on call and return and when an exception occurs).  Also,
  its return value is not used, so it can just return \code{None}.
\end{funcdesc}
\index{profile function}
\index{profiler}

\begin{datadesc}{stdin}
\dataline{stdout}
\dataline{stderr}
  File objects corresponding to the interpreter's standard input,
  output and error streams.  \code{stdin} is used for all
  interpreter input except for scripts but including calls to
  \function{input()}\bifuncindex{input} and
  \function{raw_input()}\bifuncindex{raw_input}.  \code{stdout} is used
  for the output of \keyword{print} and expression statements and for the
  prompts of \function{input()} and \function{raw_input()}.  The interpreter's
  own prompts and (almost all of) its error messages go to
  \code{stderr}.  \code{stdout} and \code{stderr} needn't
  be built-in file objects: any object is acceptable as long as it has
  a \method{write()} method that takes a string argument.  (Changing these
  objects doesn't affect the standard I/O streams of processes
  executed by \function{os.popen()}, \function{os.system()} or the
  \function{exec*()} family of functions in the \module{os} module.)
\refstmodindex{os}
\end{datadesc}

\begin{datadesc}{tracebacklimit}
When this variable is set to an integer value, it determines the
maximum number of levels of traceback information printed when an
unhandled exception occurs.  The default is \code{1000}.  When set to
0 or less, all traceback information is suppressed and only the
exception type and value are printed.
\end{datadesc}

\begin{datadesc}{version}
A string containing the version number of the Python interpreter.  
\end{datadesc}

\input{libtypes2}		% types is already taken :-(
\section{\module{traceback} ---
         Print or retrieve a stack traceback}

\declaremodule{standard}{traceback}
\modulesynopsis{Print or retrieve a stack traceback.}


This module provides a standard interface to extract, format and print
stack traces of Python programs.  It exactly mimics the behavior of
the Python interpreter when it prints a stack trace.  This is useful
when you want to print stack traces under program control, e.g. in a
``wrapper'' around the interpreter.

The module uses traceback objects --- this is the object type
that is stored in the variables \code{sys.exc_traceback} and
\code{sys.last_traceback} and returned as the third item from
\function{sys.exc_info()}.
\obindex{traceback}

The module defines the following functions:

\begin{funcdesc}{print_tb}{traceback\optional{, limit\optional{, file}}}
Print up to \var{limit} stack trace entries from \var{traceback}.  If
\var{limit} is omitted or \code{None}, all entries are printed.
If \var{file} is omitted or \code{None}, the output goes to
\code{sys.stderr}; otherwise it should be an open file or file-like
object to receive the output.
\end{funcdesc}

\begin{funcdesc}{print_exception}{type, value, traceback\optional{,
                                  limit\optional{, file}}}
Print exception information and up to \var{limit} stack trace entries
from \var{traceback} to \var{file}.
This differs from \function{print_tb()} in the
following ways: (1) if \var{traceback} is not \code{None}, it prints a
header \samp{Traceback (innermost last):}; (2) it prints the
exception \var{type} and \var{value} after the stack trace; (3) if
\var{type} is \exception{SyntaxError} and \var{value} has the appropriate
format, it prints the line where the syntax error occurred with a
caret indicating the approximate position of the error.
\end{funcdesc}

\begin{funcdesc}{print_exc}{\optional{limit\optional{, file}}}
This is a shorthand for `\code{print_exception(sys.exc_type,}
\code{sys.exc_value,} \code{sys.exc_traceback,} \var{limit}\code{,}
\var{file}\code{)}'.  (In fact, it uses \code{sys.exc_info()} to
retrieve the same information in a thread-safe way.)
\end{funcdesc}

\begin{funcdesc}{print_last}{\optional{limit\optional{, file}}}
This is a shorthand for `\code{print_exception(sys.last_type,}
\code{sys.last_value,} \code{sys.last_traceback,} \var{limit}\code{,}
\var{file}\code{)}'.
\end{funcdesc}

\begin{funcdesc}{print_stack}{\optional{f\optional{, limit\optional{, file}}}}
This function prints a stack trace from its invocation point.  The
optional \var{f} argument can be used to specify an alternate stack
frame to start.  The optional \var{limit} and \var{file} arguments have the
same meaning as for \function{print_exception()}.
\end{funcdesc}

\begin{funcdesc}{extract_tb}{traceback\optional{, limit}}
Return a list of up to \var{limit} ``pre-processed'' stack trace
entries extracted from the traceback object \var{traceback}.  It is
useful for alternate formatting of stack traces.  If \var{limit} is
omitted or \code{None}, all entries are extracted.  A
``pre-processed'' stack trace entry is a quadruple (\var{filename},
\var{line number}, \var{function name}, \var{text}) representing
the information that is usually printed for a stack trace.  The
\var{text} is a string with leading and trailing whitespace
stripped; if the source is not available it is \code{None}.
\end{funcdesc}

\begin{funcdesc}{extract_stack}{\optional{f\optional{, limit}}}
Extract the raw traceback from the current stack frame.  The return
value has the same format as for \function{extract_tb()}.  The
optional \var{f} and \var{limit} arguments have the same meaning as
for \function{print_stack()}.
\end{funcdesc}

\begin{funcdesc}{format_list}{list}
Given a list of tuples as returned by \function{extract_tb()} or
\function{extract_stack()}, return a list of strings ready for
printing.  Each string in the resulting list corresponds to the item
with the same index in the argument list.  Each string ends in a
newline; the strings may contain internal newlines as well, for those
items whose source text line is not \code{None}.
\end{funcdesc}

\begin{funcdesc}{format_exception_only}{type, value}
Format the exception part of a traceback.  The arguments are the
exception type and value such as given by \code{sys.last_type} and
\code{sys.last_value}.  The return value is a list of strings, each
ending in a newline.  Normally, the list contains a single string;
however, for \code{SyntaxError} exceptions, it contains several lines
that (when printed) display detailed information about where the
syntax error occurred.  The message indicating which exception
occurred is the always last string in the list.
\end{funcdesc}

\begin{funcdesc}{format_exception}{type, value, tb\optional{, limit}}
Format a stack trace and the exception information.  The arguments 
have the same meaning as the corresponding arguments to
\function{print_exception()}.  The return value is a list of strings,
each ending in a newline and some containing internal newlines.  When
these lines are concatenated and printed, exactly the same text is
printed as does \function{print_exception()}.
\end{funcdesc}

\begin{funcdesc}{format_tb}{tb\optional{, limit}}
A shorthand for \code{format_list(extract_tb(\var{tb}, \var{limit}))}.
\end{funcdesc}

\begin{funcdesc}{format_stack}{\optional{f\optional{, limit}}}
A shorthand for \code{format_list(extract_stack(\var{f}, \var{limit}))}.
\end{funcdesc}

\begin{funcdesc}{tb_lineno}{tb}
This function returns the current line number set in the traceback
object.  This is normally the same as the \code{\var{tb}.tb_lineno}
field of the object, but when optimization is used (the -O flag) this
field is not updated correctly; this function calculates the correct
value.
\end{funcdesc}


\subsection{Traceback Example \label{traceback-example}}

This simple example implements a basic read-eval-print loop, similar
to (but less useful than) the standard Python interactive interpreter
loop.  For a more complete implementation of the interpreter loop,
refer to the \refmodule{code} module.

\begin{verbatim}
import sys, traceback

def run_user_code(envdir):
    source = raw_input(">>> ")
    try:
        exec source in envdir
    except:
        print "Exception in user code:"
        print '-'*60
        traceback.print_exc(file=sys.stdout)
        print '-'*60

envdir = {}
while 1:
    run_user_code(envdir)
\end{verbatim}

\section{\module{pickle} --- Python object serialization}

\declaremodule{standard}{pickle}
\modulesynopsis{Convert Python objects to streams of bytes and back.}
% Substantial improvements by Jim Kerr <jbkerr@sr.hp.com>.
% Rewritten by Barry Warsaw <barry@zope.com>

\index{persistence}
\indexii{persistent}{objects}
\indexii{serializing}{objects}
\indexii{marshalling}{objects}
\indexii{flattening}{objects}
\indexii{pickling}{objects}

The \module{pickle} module implements a fundamental, but powerful
algorithm for serializing and de-serializing a Python object
structure.  ``Pickling'' is the process whereby a Python object
hierarchy is converted into a byte stream, and ``unpickling'' is the
inverse operation, whereby a byte stream is converted back into an
object hierarchy.  Pickling (and unpickling) is alternatively known as
``serialization'', ``marshalling,''\footnote{Don't confuse this with
the \refmodule{marshal} module} or ``flattening'',
however, to avoid confusion, the terms used here are ``pickling'' and
``unpickling''.

This documentation describes both the \module{pickle} module and the 
\refmodule{cPickle} module.

\subsection{Relationship to other Python modules}

The \module{pickle} module has an optimized cousin called the
\module{cPickle} module.  As its name implies, \module{cPickle} is
written in C, so it can be up to 1000 times faster than
\module{pickle}.  However it does not support subclassing of the
\function{Pickler()} and \function{Unpickler()} classes, because in
\module{cPickle} these are functions, not classes.  Most applications
have no need for this functionality, and can benefit from the improved
performance of \module{cPickle}.  Other than that, the interfaces of
the two modules are nearly identical; the common interface is
described in this manual and differences are pointed out where
necessary.  In the following discussions, we use the term ``pickle''
to collectively describe the \module{pickle} and
\module{cPickle} modules.

The data streams the two modules produce are guaranteed to be
interchangeable.

Python has a more primitive serialization module called
\refmodule{marshal}, but in general
\module{pickle} should always be the preferred way to serialize Python
objects.  \module{marshal} exists primarily to support Python's
\file{.pyc} files.

The \module{pickle} module differs from \refmodule{marshal} several
significant ways:

\begin{itemize}

\item The \module{pickle} module keeps track of the objects it has
      already serialized, so that later references to the same object
      won't be serialized again.  \module{marshal} doesn't do this.

      This has implications both for recursive objects and object
      sharing.  Recursive objects are objects that contain references
      to themselves.  These are not handled by marshal, and in fact,
      attempting to marshal recursive objects will crash your Python
      interpreter.  Object sharing happens when there are multiple
      references to the same object in different places in the object
      hierarchy being serialized.  \module{pickle} stores such objects
      only once, and ensures that all other references point to the
      master copy.  Shared objects remain shared, which can be very
      important for mutable objects.

\item \module{marshal} cannot be used to serialize user-defined
      classes and their instances.  \module{pickle} can save and
      restore class instances transparently, however the class
      definition must be importable and live in the same module as
      when the object was stored.

\item The \module{marshal} serialization format is not guaranteed to
      be portable across Python versions.  Because its primary job in
      life is to support \file{.pyc} files, the Python implementers
      reserve the right to change the serialization format in
      non-backwards compatible ways should the need arise.  The
      \module{pickle} serialization format is guaranteed to be
      backwards compatible across Python releases.

\end{itemize}

\begin{notice}[warning]
The \module{pickle} module is not intended to be secure against
erroneous or maliciously constructed data.  Never unpickle data
received from an untrusted or unauthenticated source.
\end{notice}

Note that serialization is a more primitive notion than persistence;
although
\module{pickle} reads and writes file objects, it does not handle the
issue of naming persistent objects, nor the (even more complicated)
issue of concurrent access to persistent objects.  The \module{pickle}
module can transform a complex object into a byte stream and it can
transform the byte stream into an object with the same internal
structure.  Perhaps the most obvious thing to do with these byte
streams is to write them onto a file, but it is also conceivable to
send them across a network or store them in a database.  The module
\refmodule{shelve} provides a simple interface
to pickle and unpickle objects on DBM-style database files.

\subsection{Data stream format}

The data format used by \module{pickle} is Python-specific.  This has
the advantage that there are no restrictions imposed by external
standards such as XDR\index{XDR}\index{External Data Representation}
(which can't represent pointer sharing); however it means that
non-Python programs may not be able to reconstruct pickled Python
objects.

By default, the \module{pickle} data format uses a printable \ASCII{}
representation.  This is slightly more voluminous than a binary
representation.  The big advantage of using printable \ASCII{} (and of
some other characteristics of \module{pickle}'s representation) is that
for debugging or recovery purposes it is possible for a human to read
the pickled file with a standard text editor.

There are currently 3 different protocols which can be used for pickling.

\begin{itemize}

\item Protocol version 0 is the original ASCII protocol and is backwards
compatible with earlier versions of Python.

\item Protocol version 1 is the old binary format which is also compatible
with earlier versions of Python.

\item Protocol version 2 was introduced in Python 2.3.  It provides
much more efficient pickling of new-style classes.

\end{itemize}

Refer to PEP 307 for more information.

If a \var{protocol} is not specified, protocol 0 is used.
If \var{protocol} is specified as a negative value
or \constant{HIGHEST_PROTOCOL},
the highest protocol version available will be used.

\versionchanged[The \var{bin} parameter is deprecated and only provided
for backwards compatibility.  You should use the \var{protocol}
parameter instead]{2.3}

A binary format, which is slightly more efficient, can be chosen by
specifying a true value for the \var{bin} argument to the
\class{Pickler} constructor or the \function{dump()} and \function{dumps()}
functions.  A \var{protocol} version >= 1 implies use of a binary format.

\subsection{Usage}

To serialize an object hierarchy, you first create a pickler, then you
call the pickler's \method{dump()} method.  To de-serialize a data
stream, you first create an unpickler, then you call the unpickler's
\method{load()} method.  The \module{pickle} module provides the
following constant:

\begin{datadesc}{HIGHEST_PROTOCOL}
The highest protocol version available.  This value can be passed
as a \var{protocol} value.
\versionadded{2.3}
\end{datadesc}

The \module{pickle} module provides the
following functions to make this process more convenient:

\begin{funcdesc}{dump}{obj, file\optional{, protocol\optional{, bin}}}
Write a pickled representation of \var{obj} to the open file object
\var{file}.  This is equivalent to
\code{Pickler(\var{file}, \var{protocol}, \var{bin}).dump(\var{obj})}.

If the \var{protocol} parameter is omitted, protocol 0 is used.
If \var{protocol} is specified as a negative value
or \constant{HIGHEST_PROTOCOL},
the highest protocol version will be used.

\versionchanged[The \var{protocol} parameter was added.
The \var{bin} parameter is deprecated and only provided
for backwards compatibility.  You should use the \var{protocol}
parameter instead]{2.3}

If the optional \var{bin} argument is true, the binary pickle format
is used; otherwise the (less efficient) text pickle format is used
(for backwards compatibility, this is the default).

\var{file} must have a \method{write()} method that accepts a single
string argument.  It can thus be a file object opened for writing, a
\refmodule{StringIO} object, or any other custom
object that meets this interface.
\end{funcdesc}

\begin{funcdesc}{load}{file}
Read a string from the open file object \var{file} and interpret it as
a pickle data stream, reconstructing and returning the original object
hierarchy.  This is equivalent to \code{Unpickler(\var{file}).load()}.

\var{file} must have two methods, a \method{read()} method that takes
an integer argument, and a \method{readline()} method that requires no
arguments.  Both methods should return a string.  Thus \var{file} can
be a file object opened for reading, a
\module{StringIO} object, or any other custom
object that meets this interface.

This function automatically determines whether the data stream was
written in binary mode or not.
\end{funcdesc}

\begin{funcdesc}{dumps}{obj\optional{, protocol\optional{, bin}}}
Return the pickled representation of the object as a string, instead
of writing it to a file.

If the \var{protocol} parameter is omitted, protocol 0 is used.
If \var{protocol} is specified as a negative value
or \constant{HIGHEST_PROTOCOL},
the highest protocol version will be used.

\versionchanged[The \var{protocol} parameter was added.
The \var{bin} parameter is deprecated and only provided
for backwards compatibility.  You should use the \var{protocol}
parameter instead]{2.3}

If the optional \var{bin} argument is
true, the binary pickle format is used; otherwise the (less efficient)
text pickle format is used (this is the default).
\end{funcdesc}

\begin{funcdesc}{loads}{string}
Read a pickled object hierarchy from a string.  Characters in the
string past the pickled object's representation are ignored.
\end{funcdesc}

The \module{pickle} module also defines three exceptions:

\begin{excdesc}{PickleError}
A common base class for the other exceptions defined below.  This
inherits from \exception{Exception}.
\end{excdesc}

\begin{excdesc}{PicklingError}
This exception is raised when an unpicklable object is passed to
the \method{dump()} method.
\end{excdesc}

\begin{excdesc}{UnpicklingError}
This exception is raised when there is a problem unpickling an object.
Note that other exceptions may also be raised during unpickling,
including (but not necessarily limited to) \exception{AttributeError},
\exception{EOFError}, \exception{ImportError}, and \exception{IndexError}.
\end{excdesc}

The \module{pickle} module also exports two callables\footnote{In the
\module{pickle} module these callables are classes, which you could
subclass to customize the behavior.  However, in the \refmodule{cPickle}
module these callables are factory functions and so cannot be
subclassed.  One common reason to subclass is to control what
objects can actually be unpickled.  See section~\ref{pickle-sub} for
more details.}, \class{Pickler} and \class{Unpickler}:

\begin{classdesc}{Pickler}{file\optional{, protocol\optional{, bin}}}
This takes a file-like object to which it will write a pickle data
stream.  

If the \var{protocol} parameter is omitted, protocol 0 is used.
If \var{protocol} is specified as a negative value,
the highest protocol version will be used.

\versionchanged[The \var{bin} parameter is deprecated and only provided
for backwards compatibility.  You should use the \var{protocol}
parameter instead]{2.3}

Optional \var{bin} if true, tells the pickler to use the more
efficient binary pickle format, otherwise the \ASCII{} format is used
(this is the default).

\var{file} must have a \method{write()} method that accepts a single
string argument.  It can thus be an open file object, a
\module{StringIO} object, or any other custom
object that meets this interface.
\end{classdesc}

\class{Pickler} objects define one (or two) public methods:

\begin{methoddesc}[Pickler]{dump}{obj}
Write a pickled representation of \var{obj} to the open file object
given in the constructor.  Either the binary or \ASCII{} format will
be used, depending on the value of the \var{bin} flag passed to the
constructor.
\end{methoddesc}

\begin{methoddesc}[Pickler]{clear_memo}{}
Clears the pickler's ``memo''.  The memo is the data structure that
remembers which objects the pickler has already seen, so that shared
or recursive objects pickled by reference and not by value.  This
method is useful when re-using picklers.

\begin{notice}
Prior to Python 2.3, \method{clear_memo()} was only available on the
picklers created by \refmodule{cPickle}.  In the \module{pickle} module,
picklers have an instance variable called \member{memo} which is a
Python dictionary.  So to clear the memo for a \module{pickle} module
pickler, you could do the following:

\begin{verbatim}
mypickler.memo.clear()
\end{verbatim}

Code that does not need to support older versions of Python should
simply use \method{clear_memo()}.
\end{notice}
\end{methoddesc}

It is possible to make multiple calls to the \method{dump()} method of
the same \class{Pickler} instance.  These must then be matched to the
same number of calls to the \method{load()} method of the
corresponding \class{Unpickler} instance.  If the same object is
pickled by multiple \method{dump()} calls, the \method{load()} will
all yield references to the same object.\footnote{\emph{Warning}: this
is intended for pickling multiple objects without intervening
modifications to the objects or their parts.  If you modify an object
and then pickle it again using the same \class{Pickler} instance, the
object is not pickled again --- a reference to it is pickled and the
\class{Unpickler} will return the old value, not the modified one.
There are two problems here: (1) detecting changes, and (2)
marshalling a minimal set of changes.  Garbage Collection may also
become a problem here.}

\class{Unpickler} objects are defined as:

\begin{classdesc}{Unpickler}{file}
This takes a file-like object from which it will read a pickle data
stream.  This class automatically determines whether the data stream
was written in binary mode or not, so it does not need a flag as in
the \class{Pickler} factory.

\var{file} must have two methods, a \method{read()} method that takes
an integer argument, and a \method{readline()} method that requires no
arguments.  Both methods should return a string.  Thus \var{file} can
be a file object opened for reading, a
\module{StringIO} object, or any other custom
object that meets this interface.
\end{classdesc}

\class{Unpickler} objects have one (or two) public methods:

\begin{methoddesc}[Unpickler]{load}{}
Read a pickled object representation from the open file object given
in the constructor, and return the reconstituted object hierarchy
specified therein.
\end{methoddesc}

\begin{methoddesc}[Unpickler]{noload}{}
This is just like \method{load()} except that it doesn't actually
create any objects.  This is useful primarily for finding what's
called ``persistent ids'' that may be referenced in a pickle data
stream.  See section~\ref{pickle-protocol} below for more details.

\strong{Note:} the \method{noload()} method is currently only
available on \class{Unpickler} objects created with the
\module{cPickle} module.  \module{pickle} module \class{Unpickler}s do
not have the \method{noload()} method.
\end{methoddesc}

\subsection{What can be pickled and unpickled?}

The following types can be pickled:

\begin{itemize}

\item \code{None}, \code{True}, and \code{False}

\item integers, long integers, floating point numbers, complex numbers

\item normal and Unicode strings

\item tuples, lists, sets, and dictionaries containing only picklable objects

\item functions defined at the top level of a module

\item built-in functions defined at the top level of a module

\item classes that are defined at the top level of a module

\item instances of such classes whose \member{__dict__} or
\method{__setstate__()} is picklable  (see
section~\ref{pickle-protocol} for details)

\end{itemize}

Attempts to pickle unpicklable objects will raise the
\exception{PicklingError} exception; when this happens, an unspecified
number of bytes may have already been written to the underlying file.

Note that functions (built-in and user-defined) are pickled by ``fully
qualified'' name reference, not by value.  This means that only the
function name is pickled, along with the name of module the function
is defined in.  Neither the function's code, nor any of its function
attributes are pickled.  Thus the defining module must be importable
in the unpickling environment, and the module must contain the named
object, otherwise an exception will be raised.\footnote{The exception
raised will likely be an \exception{ImportError} or an
\exception{AttributeError} but it could be something else.}

Similarly, classes are pickled by named reference, so the same
restrictions in the unpickling environment apply.  Note that none of
the class's code or data is pickled, so in the following example the
class attribute \code{attr} is not restored in the unpickling
environment:

\begin{verbatim}
class Foo:
    attr = 'a class attr'

picklestring = pickle.dumps(Foo)
\end{verbatim}

These restrictions are why picklable functions and classes must be
defined in the top level of a module.

Similarly, when class instances are pickled, their class's code and
data are not pickled along with them.  Only the instance data are
pickled.  This is done on purpose, so you can fix bugs in a class or
add methods to the class and still load objects that were created with
an earlier version of the class.  If you plan to have long-lived
objects that will see many versions of a class, it may be worthwhile
to put a version number in the objects so that suitable conversions
can be made by the class's \method{__setstate__()} method.

\subsection{The pickle protocol
\label{pickle-protocol}}\setindexsubitem{(pickle protocol)}

This section describes the ``pickling protocol'' that defines the
interface between the pickler/unpickler and the objects that are being
serialized.  This protocol provides a standard way for you to define,
customize, and control how your objects are serialized and
de-serialized.  The description in this section doesn't cover specific
customizations that you can employ to make the unpickling environment
slightly safer from untrusted pickle data streams; see section~\ref{pickle-sub}
for more details.

\subsubsection{Pickling and unpickling normal class
    instances\label{pickle-inst}}

When a pickled class instance is unpickled, its \method{__init__()}
method is normally \emph{not} invoked.  If it is desirable that the
\method{__init__()} method be called on unpickling, an old-style class
can define a method \method{__getinitargs__()}, which should return a
\emph{tuple} containing the arguments to be passed to the class
constructor (i.e. \method{__init__()}).  The
\method{__getinitargs__()} method is called at
pickle time; the tuple it returns is incorporated in the pickle for
the instance.
\withsubitem{(copy protocol)}{\ttindex{__getinitargs__()}}
\withsubitem{(instance constructor)}{\ttindex{__init__()}}

\withsubitem{(copy protocol)}{\ttindex{__getnewargs__()}}

New-style types can provide a \method{__getnewargs__()} method that is
used for protocol 2.  Implementing this method is needed if the type
establishes some internal invariants when the instance is created, or
if the memory allocation is affected by the values passed to the
\method{__new__()} method for the type (as it is for tuples and
strings).  Instances of a new-style type \class{C} are created using

\begin{alltt}
obj = C.__new__(C, *\var{args})
\end{alltt}

where \var{args} is the result of calling \method{__getnewargs__()} on
the original object; if there is no \method{__getnewargs__()}, an
empty tuple is assumed.

\withsubitem{(copy protocol)}{
  \ttindex{__getstate__()}\ttindex{__setstate__()}}
\withsubitem{(instance attribute)}{
  \ttindex{__dict__}}

Classes can further influence how their instances are pickled; if the
class defines the method \method{__getstate__()}, it is called and the
return state is pickled as the contents for the instance, instead of
the contents of the instance's dictionary.  If there is no
\method{__getstate__()} method, the instance's \member{__dict__} is
pickled.

Upon unpickling, if the class also defines the method
\method{__setstate__()}, it is called with the unpickled
state.\footnote{These methods can also be used to implement copying
class instances.}  If there is no \method{__setstate__()} method, the
pickled state must be a dictionary and its items are assigned to the
new instance's dictionary.  If a class defines both
\method{__getstate__()} and \method{__setstate__()}, the state object
needn't be a dictionary and these methods can do what they
want.\footnote{This protocol is also used by the shallow and deep
copying operations defined in the
\refmodule{copy} module.}

\begin{notice}[warning]
  For new-style classes, if \method{__getstate__()} returns a false
  value, the \method{__setstate__()} method will not be called.
\end{notice}


\subsubsection{Pickling and unpickling extension types}

When the \class{Pickler} encounters an object of a type it knows
nothing about --- such as an extension type --- it looks in two places
for a hint of how to pickle it.  One alternative is for the object to
implement a \method{__reduce__()} method.  If provided, at pickling
time \method{__reduce__()} will be called with no arguments, and it
must return either a string or a tuple.

If a string is returned, it names a global variable whose contents are
pickled as normal.  The string returned by \method{__reduce__} should
be the object's local name relative to its module; the pickle module
searches the module namespace to determine the object's module.

When a tuple is returned, it must be between two and five elements
long. Optional elements can either be omitted, or \code{None} can be provided 
as their value.  The semantics of each element are:

\begin{itemize}

\item A callable object that will be called to create the initial
version of the object.  The next element of the tuple will provide
arguments for this callable, and later elements provide additional
state information that will subsequently be used to fully reconstruct
the pickled date.

In the unpickling environment this object must be either a class, a
callable registered as a ``safe constructor'' (see below), or it must
have an attribute \member{__safe_for_unpickling__} with a true value.
Otherwise, an \exception{UnpicklingError} will be raised in the
unpickling environment.  Note that as usual, the callable itself is
pickled by name.

\item A tuple of arguments for the callable object, or \code{None}.
\deprecated{2.3}{If this item is \code{None}, then instead of calling
the callable directly, its \method{__basicnew__()} method is called
without arguments; this method should also return the unpickled
object.  Providing \code{None} is deprecated, however; return a
tuple of arguments instead.}

\item Optionally, the object's state, which will be passed to
      the object's \method{__setstate__()} method as described in
      section~\ref{pickle-inst}.  If the object has no
      \method{__setstate__()} method, then, as above, the value must
      be a dictionary and it will be added to the object's
      \member{__dict__}.

\item Optionally, an iterator (and not a sequence) yielding successive
list items.  These list items will be pickled, and appended to the
object using either \code{obj.append(\var{item})} or
\code{obj.extend(\var{list_of_items})}.  This is primarily used for
list subclasses, but may be used by other classes as long as they have
\method{append()} and \method{extend()} methods with the appropriate
signature.  (Whether \method{append()} or \method{extend()} is used
depends on which pickle protocol version is used as well as the number
of items to append, so both must be supported.)

\item Optionally, an iterator (not a sequence)
yielding successive dictionary items, which should be tuples of the
form \code{(\var{key}, \var{value})}.  These items will be pickled
and stored to the object using \code{obj[\var{key}] = \var{value}}.
This is primarily used for dictionary subclasses, but may be used by
other classes as long as they implement \method{__setitem__}.

\end{itemize}

It is sometimes useful to know the protocol version when implementing
\method{__reduce__}.  This can be done by implementing a method named
\method{__reduce_ex__} instead of \method{__reduce__}.
\method{__reduce_ex__}, when it exists, is called in preference over
\method{__reduce__} (you may still provide \method{__reduce__} for
backwards compatibility).  The \method{__reduce_ex__} method will be
called with a single integer argument, the protocol version.

The \class{object} class implements both \method{__reduce__} and
\method{__reduce_ex__}; however, if a subclass overrides
\method{__reduce__} but not \method{__reduce_ex__}, the
\method{__reduce_ex__} implementation detects this and calls
\method{__reduce__}.

An alternative to implementing a \method{__reduce__()} method on the
object to be pickled, is to register the callable with the
\refmodule[copyreg]{copy_reg} module.  This module provides a way
for programs to register ``reduction functions'' and constructors for
user-defined types.   Reduction functions have the same semantics and
interface as the \method{__reduce__()} method described above, except
that they are called with a single argument, the object to be pickled.

The registered constructor is deemed a ``safe constructor'' for purposes
of unpickling as described above.


\subsubsection{Pickling and unpickling external objects}

For the benefit of object persistence, the \module{pickle} module
supports the notion of a reference to an object outside the pickled
data stream.  Such objects are referenced by a ``persistent id'',
which is just an arbitrary string of printable \ASCII{} characters.
The resolution of such names is not defined by the \module{pickle}
module; it will delegate this resolution to user defined functions on
the pickler and unpickler.\footnote{The actual mechanism for
associating these user defined functions is slightly different for
\module{pickle} and \module{cPickle}.  The description given here
works the same for both implementations.  Users of the \module{pickle}
module could also use subclassing to effect the same results,
overriding the \method{persistent_id()} and \method{persistent_load()}
methods in the derived classes.}

To define external persistent id resolution, you need to set the
\member{persistent_id} attribute of the pickler object and the
\member{persistent_load} attribute of the unpickler object.

To pickle objects that have an external persistent id, the pickler
must have a custom \function{persistent_id()} method that takes an
object as an argument and returns either \code{None} or the persistent
id for that object.  When \code{None} is returned, the pickler simply
pickles the object as normal.  When a persistent id string is
returned, the pickler will pickle that string, along with a marker
so that the unpickler will recognize the string as a persistent id.

To unpickle external objects, the unpickler must have a custom
\function{persistent_load()} function that takes a persistent id
string and returns the referenced object.

Here's a silly example that \emph{might} shed more light:

\begin{verbatim}
import pickle
from cStringIO import StringIO

src = StringIO()
p = pickle.Pickler(src)

def persistent_id(obj):
    if hasattr(obj, 'x'):
        return 'the value %d' % obj.x
    else:
        return None

p.persistent_id = persistent_id

class Integer:
    def __init__(self, x):
        self.x = x
    def __str__(self):
        return 'My name is integer %d' % self.x

i = Integer(7)
print i
p.dump(i)

datastream = src.getvalue()
print repr(datastream)
dst = StringIO(datastream)

up = pickle.Unpickler(dst)

class FancyInteger(Integer):
    def __str__(self):
        return 'I am the integer %d' % self.x

def persistent_load(persid):
    if persid.startswith('the value '):
        value = int(persid.split()[2])
        return FancyInteger(value)
    else:
        raise pickle.UnpicklingError, 'Invalid persistent id'

up.persistent_load = persistent_load

j = up.load()
print j
\end{verbatim}

In the \module{cPickle} module, the unpickler's
\member{persistent_load} attribute can also be set to a Python
list, in which case, when the unpickler reaches a persistent id, the
persistent id string will simply be appended to this list.  This
functionality exists so that a pickle data stream can be ``sniffed''
for object references without actually instantiating all the objects
in a pickle.\footnote{We'll leave you with the image of Guido and Jim
sitting around sniffing pickles in their living rooms.}  Setting
\member{persistent_load} to a list is usually used in conjunction with
the \method{noload()} method on the Unpickler.

% BAW: Both pickle and cPickle support something called
% inst_persistent_id() which appears to give unknown types a second
% shot at producing a persistent id.  Since Jim Fulton can't remember
% why it was added or what it's for, I'm leaving it undocumented.

\subsection{Subclassing Unpicklers \label{pickle-sub}}

By default, unpickling will import any class that it finds in the
pickle data.  You can control exactly what gets unpickled and what
gets called by customizing your unpickler.  Unfortunately, exactly how
you do this is different depending on whether you're using
\module{pickle} or \module{cPickle}.\footnote{A word of caution: the
mechanisms described here use internal attributes and methods, which
are subject to change in future versions of Python.  We intend to
someday provide a common interface for controlling this behavior,
which will work in either \module{pickle} or \module{cPickle}.}

In the \module{pickle} module, you need to derive a subclass from
\class{Unpickler}, overriding the \method{load_global()}
method.  \method{load_global()} should read two lines from the pickle
data stream where the first line will the name of the module
containing the class and the second line will be the name of the
instance's class.  It then looks up the class, possibly importing the
module and digging out the attribute, then it appends what it finds to
the unpickler's stack.  Later on, this class will be assigned to the
\member{__class__} attribute of an empty class, as a way of magically
creating an instance without calling its class's \method{__init__()}.
Your job (should you choose to accept it), would be to have
\method{load_global()} push onto the unpickler's stack, a known safe
version of any class you deem safe to unpickle.  It is up to you to
produce such a class.  Or you could raise an error if you want to
disallow all unpickling of instances.  If this sounds like a hack,
you're right.  Refer to the source code to make this work.

Things are a little cleaner with \module{cPickle}, but not by much.
To control what gets unpickled, you can set the unpickler's
\member{find_global} attribute to a function or \code{None}.  If it is
\code{None} then any attempts to unpickle instances will raise an
\exception{UnpicklingError}.  If it is a function,
then it should accept a module name and a class name, and return the
corresponding class object.  It is responsible for looking up the
class and performing any necessary imports, and it may raise an
error to prevent instances of the class from being unpickled.

The moral of the story is that you should be really careful about the
source of the strings your application unpickles.

\subsection{Example \label{pickle-example}}

Here's a simple example of how to modify pickling behavior for a
class.  The \class{TextReader} class opens a text file, and returns
the line number and line contents each time its \method{readline()}
method is called. If a \class{TextReader} instance is pickled, all
attributes \emph{except} the file object member are saved. When the
instance is unpickled, the file is reopened, and reading resumes from
the last location. The \method{__setstate__()} and
\method{__getstate__()} methods are used to implement this behavior.

\begin{verbatim}
class TextReader:
    """Print and number lines in a text file."""
    def __init__(self, file):
        self.file = file
        self.fh = open(file)
        self.lineno = 0

    def readline(self):
        self.lineno = self.lineno + 1
        line = self.fh.readline()
        if not line:
            return None
        if line.endswith("\n"):
            line = line[:-1]
        return "%d: %s" % (self.lineno, line)

    def __getstate__(self):
        odict = self.__dict__.copy() # copy the dict since we change it
        del odict['fh']              # remove filehandle entry
        return odict

    def __setstate__(self,dict):
        fh = open(dict['file'])      # reopen file
        count = dict['lineno']       # read from file...
        while count:                 # until line count is restored
            fh.readline()
            count = count - 1
        self.__dict__.update(dict)   # update attributes
        self.fh = fh                 # save the file object
\end{verbatim}

A sample usage might be something like this:

\begin{verbatim}
>>> import TextReader
>>> obj = TextReader.TextReader("TextReader.py")
>>> obj.readline()
'1: #!/usr/local/bin/python'
>>> # (more invocations of obj.readline() here)
... obj.readline()
'7: class TextReader:'
>>> import pickle
>>> pickle.dump(obj,open('save.p','w'))
\end{verbatim}

If you want to see that \refmodule{pickle} works across Python
processes, start another Python session, before continuing.  What
follows can happen from either the same process or a new process.

\begin{verbatim}
>>> import pickle
>>> reader = pickle.load(open('save.p'))
>>> reader.readline()
'8:     "Print and number lines in a text file."'
\end{verbatim}


\begin{seealso}
  \seemodule[copyreg]{copy_reg}{Pickle interface constructor
                                registration for extension types.}

  \seemodule{shelve}{Indexed databases of objects; uses \module{pickle}.}

  \seemodule{copy}{Shallow and deep object copying.}

  \seemodule{marshal}{High-performance serialization of built-in types.}
\end{seealso}


\section{\module{cPickle} --- A faster \module{pickle}}

\declaremodule{builtin}{cPickle}
\modulesynopsis{Faster version of \refmodule{pickle}, but not subclassable.}
\moduleauthor{Jim Fulton}{jim@zope.com}
\sectionauthor{Fred L. Drake, Jr.}{fdrake@acm.org}

The \module{cPickle} module supports serialization and
de-serialization of Python objects, providing an interface and
functionality nearly identical to the
\refmodule{pickle}\refstmodindex{pickle} module.  There are several
differences, the most important being performance and subclassability.

First, \module{cPickle} can be up to 1000 times faster than
\module{pickle} because the former is implemented in C.  Second, in
the \module{cPickle} module the callables \function{Pickler()} and
\function{Unpickler()} are functions, not classes.  This means that
you cannot use them to derive custom pickling and unpickling
subclasses.  Most applications have no need for this functionality and
should benefit from the greatly improved performance of the
\module{cPickle} module.

The pickle data stream produced by \module{pickle} and
\module{cPickle} are identical, so it is possible to use
\module{pickle} and \module{cPickle} interchangeably with existing
pickles.\footnote{Since the pickle data format is actually a tiny
stack-oriented programming language, and some freedom is taken in the
encodings of certain objects, it is possible that the two modules
produce different data streams for the same input objects.  However it
is guaranteed that they will always be able to read each other's
data streams.}

There are additional minor differences in API between \module{cPickle}
and \module{pickle}, however for most applications, they are
interchangeable.  More documentation is provided in the
\module{pickle} module documentation, which
includes a list of the documented differences.



\section{\module{shelve} ---
         Python object persistence}

\declaremodule{standard}{shelve}
\modulesynopsis{Python object persistence.}


A ``shelf'' is a persistent, dictionary-like object.  The difference
with ``dbm'' databases is that the values (not the keys!) in a shelf
can be essentially arbitrary Python objects --- anything that the
\refmodule{pickle} module can handle.  This includes most class
instances, recursive data types, and objects containing lots of shared 
sub-objects.  The keys are ordinary strings.
\refstmodindex{pickle}

To summarize the interface (\code{key} is a string, \code{data} is an
arbitrary object):

\begin{verbatim}
import shelve

d = shelve.open(filename) # open -- file may get suffix added by low-level
                          # library

d[key] = data   # store data at key (overwrites old data if
                # using an existing key)
data = d[key]   # retrieve data at key (raise KeyError if no
                # such key)
del d[key]      # delete data stored at key (raises KeyError
                # if no such key)
flag = d.has_key(key)   # true if the key exists
list = d.keys() # a list of all existing keys (slow!)

d.close()       # close it
\end{verbatim}

In addition to the above, shelve supports all methods that are
supported by dictionaries.  This eases the transition from dictionary
based scripts to those requiring persistent storage.

Restrictions:

\begin{itemize}

\item
The choice of which database package will be used
(e.g. \refmodule{dbm} or \refmodule{gdbm}) depends on which interface
is available.  Therefore it is not safe to open the database directly
using \refmodule{dbm}.  The database is also (unfortunately) subject
to the limitations of \refmodule{dbm}, if it is used --- this means
that (the pickled representation of) the objects stored in the
database should be fairly small, and in rare cases key collisions may
cause the database to refuse updates.
\refbimodindex{dbm}
\refbimodindex{gdbm}

\item
Depending on the implementation, closing a persistent dictionary may
or may not be necessary to flush changes to disk.  The \method{__del__}
method of the \class{Shelf} class calls the \method{close} method, so the
programmer generally need not do this explicitly.

\item
The \module{shelve} module does not support \emph{concurrent} read/write
access to shelved objects.  (Multiple simultaneous read accesses are
safe.)  When a program has a shelf open for writing, no other program
should have it open for reading or writing.  \UNIX{} file locking can
be used to solve this, but this differs across \UNIX{} versions and
requires knowledge about the database implementation used.

\end{itemize}

\begin{classdesc}{Shelf}{dict\optional{, binary=False}}
A subclass of \class{UserDict.DictMixin} which stores pickled values in the
\var{dict} object.  If the \var{binary} parameter is \constant{True}, binary
pickles will be used.  This can provide much more compact storage than plain
text pickles, depending on the nature of the objects stored in the databse.
\end{classdesc}

\begin{classdesc}{BsdDbShelf}{dict\optional{, binary=False}}
A subclass of \class{Shelf} which exposes \method{first}, \method{next},
{}\method{previous}, \method{last} and \method{set_location} which are
available in the \module{bsddb} module but not in other database modules.
The \var{dict} object passed to the constructor must support those methods.
This is generally accomplished by calling one of \function{bsddb.hashopen},
\function{bsddb.btopen} or \function{bsddb.rnopen}.  The optional
\var{binary} parameter has the same interpretation as for the \class{Shelf}
class. 
\end{classdesc}

\begin{classdesc}{DbfilenameShelf}{dict\optional{, flag='c'}\optional{, binary=False}}
A subclass of \class{Shelf} which accepts a filename instead of a dict-like
object.  The underlying file will be opened using \function{anydbm.open}.
By default, the file will be created and opened for both read and write.
The optional \var{binary} parameter has the same interpretation as for the
\class{Shelf} class.
\end{classdesc}

\begin{seealso}
  \seemodule{anydbm}{Generic interface to \code{dbm}-style databases.}
  \seemodule{bsddb}{BSD \code{db} database interface.}
  \seemodule{dbhash}{Thin layer around the \module{bsddb} which provides an
  \function{open} function like the other database modules.}
  \seemodule{dbm}{Standard \UNIX{} database interface.}
  \seemodule{dumbdbm}{Portable implementation of the \code{dbm} interface.}
  \seemodule{gdbm}{GNU database interface, based on the \code{dbm} interface.}
  \seemodule{pickle}{Object serialization used by \module{shelve}.}
  \seemodule{cPickle}{High-performance version of \refmodule{pickle}.}
\end{seealso}

\section{Built-in module \sectcode{copy}}
\stmodindex{copy}
\ttindex{copy}
\ttindex{deepcopy}

This module provides generic (shallow and deep) copying operations.

Interface summary:

\begin{verbatim}
import copy

x = copy.copy(y)	# make a shallow copy of y
x = copy.deepcopy(y)	# make a deep copy of y
\end{verbatim}

For module specific errors, \code{copy.Error} is raised.

The difference between shallow and deep copying is only relevant for
compound objects (objects that contain other objects, like lists or
class instances):

\begin{itemize}

\item
A {\em shallow copy} constructs a new compound object and then (to the
extent possible) inserts {\em references} into it to the objects found
in the original.

\item
A {\em deep copy} constructs a new compound object and then,
recursively, inserts {\em copies} into it of the objects found in the
original.

\end{itemize}

Two problems often exist with deep copy operations that don't exist
with shallow copy operations:

\begin{itemize}

\item
Recursive objects (compound objects that, directly or indirectly,
contain a reference to themselves) may cause a recursive loop.

\item
Because deep copy copies {\em everything} it may copy too much, e.g.
administrative data structures that should be shared even between
copies.

\end{itemize}

Python's \code{deepcopy()} operation avoids these problems by:

\begin{itemize}

\item
keeping a table of objects already copied during the current
copying pass; and

\item
letting user-defined classes override the copying operation or the
set of components copied.

\end{itemize}

This version does not copy types like module, class, function, method,
nor stack trace, stack frame, nor file, socket, window, nor array, nor
any similar types.

Classes can use the same interfaces to control copying that they use
to control pickling: they can define methods called
\code{__getinitargs__()}, \code{__getstate__()} and
\code{__setstate__()}.  See the description of module \code{pickle}
for information on these methods.
\stmodindex{pickle}
\ttindex{__getinitargs__}
\ttindex{__getstate__}
\ttindex{__setstate__}

\section{Built-in Module \sectcode{marshal}}
\label{module-marshal}

\bimodindex{marshal}
This module contains functions that can read and write Python
values in a binary format.  The format is specific to Python, but
independent of machine architecture issues (e.g., you can write a
Python value to a file on a PC, transport the file to a Sun, and read
it back there).  Details of the format are undocumented on purpose;
it may change between Python versions (although it rarely does).%
\footnote{The name of this module stems from a bit of terminology used
by the designers of Modula-3 (amongst others), who use the term
``marshalling'' for shipping of data around in a self-contained form.
Strictly speaking, ``to marshal'' means to convert some data from
internal to external form (in an RPC buffer for instance) and
``unmarshalling'' for the reverse process.}

This is not a general ``persistency'' module.  For general persistency
and transfer of Python objects through RPC calls, see the modules
\code{pickle} and \code{shelve}.  The \code{marshal} module exists
mainly to support reading and writing the ``pseudo-compiled'' code for
Python modules of \samp{.pyc} files.
\refstmodindex{pickle}
\refstmodindex{shelve}
\obindex{code}

Not all Python object types are supported; in general, only objects
whose value is independent from a particular invocation of Python can
be written and read by this module.  The following types are supported:
\code{None}, integers, long integers, floating point numbers,
strings, tuples, lists, dictionaries, and code objects, where it
should be understood that tuples, lists and dictionaries are only
supported as long as the values contained therein are themselves
supported; and recursive lists and dictionaries should not be written
(they will cause infinite loops).

{\bf Caveat:} On machines where C's \code{long int} type has more than
32 bits (such as the DEC Alpha), it
is possible to create plain Python integers that are longer than 32
bits.  Since the current \code{marshal} module uses 32 bits to
transfer plain Python integers, such values are silently truncated.
This particularly affects the use of very long integer literals in
Python modules --- these will be accepted by the parser on such
machines, but will be silently be truncated when the module is read
from the \code{.pyc} instead.%
\footnote{A solution would be to refuse such literals in the parser,
since they are inherently non-portable.  Another solution would be to
let the \code{marshal} module raise an exception when an integer value
would be truncated.  At least one of these solutions will be
implemented in a future version.}

There are functions that read/write files as well as functions
operating on strings.

The module defines these functions:

\renewcommand{\indexsubitem}{(in module marshal)}

\begin{funcdesc}{dump}{value\, file}
  Write the value on the open file.  The value must be a supported
  type.  The file must be an open file object such as
  \code{sys.stdout} or returned by \code{open()} or
  \code{posix.popen()}.
  
  If the value has (or contains an object that has) an unsupported type,
  a \code{ValueError} exception is raised -- but garbage data will also
  be written to the file.  The object will not be properly read back by
  \code{load()}.
\end{funcdesc}

\begin{funcdesc}{load}{file}
  Read one value from the open file and return it.  If no valid value
  is read, raise \code{EOFError}, \code{ValueError} or
  \code{TypeError}.  The file must be an open file object.

  Warning: If an object containing an unsupported type was marshalled
  with \code{dump()}, \code{load()} will substitute \code{None} for the
  unmarshallable type.
\end{funcdesc}

\begin{funcdesc}{dumps}{value}
  Return the string that would be written to a file by
  \code{dump(value, file)}.  The value must be a supported type.
  Raise a \code{ValueError} exception if value has (or contains an
  object that has) an unsupported type.
\end{funcdesc}

\begin{funcdesc}{loads}{string}
  Convert the string to a value.  If no valid value is found, raise
  \code{EOFError}, \code{ValueError} or \code{TypeError}.  Extra
  characters in the string are ignored.
\end{funcdesc}

\section{\module{imp} ---
         Access the \keyword{import} internals}

\declaremodule{builtin}{imp}
\modulesynopsis{Access the implementation of the \keyword{import} statement.}


This\stindex{import} module provides an interface to the mechanisms
used to implement the \keyword{import} statement.  It defines the
following constants and functions:


\begin{funcdesc}{get_magic}{}
\indexii{file}{byte-code}
Return the magic string value used to recognize byte-compiled code
files (\file{.pyc} files).  (This value may be different for each
Python version.)
\end{funcdesc}

\begin{funcdesc}{get_suffixes}{}
Return a list of triples, each describing a particular type of module.
Each triple has the form \code{(\var{suffix}, \var{mode},
\var{type})}, where \var{suffix} is a string to be appended to the
module name to form the filename to search for, \var{mode} is the mode
string to pass to the built-in \function{open()} function to open the
file (this can be \code{'r'} for text files or \code{'rb'} for binary
files), and \var{type} is the file type, which has one of the values
\constant{PY_SOURCE}, \constant{PY_COMPILED}, or
\constant{C_EXTENSION}, described below.
\end{funcdesc}

\begin{funcdesc}{find_module}{name\optional{, path}}
Try to find the module \var{name} on the search path \var{path}.  If
\var{path} is a list of directory names, each directory is searched
for files with any of the suffixes returned by \function{get_suffixes()}
above.  Invalid names in the list are silently ignored (but all list
items must be strings).  If \var{path} is omitted or \code{None}, the
list of directory names given by \code{sys.path} is searched, but
first it searches a few special places: it tries to find a built-in
module with the given name (\constant{C_BUILTIN}), then a frozen module
(\constant{PY_FROZEN}), and on some systems some other places are looked
in as well (on the Mac, it looks for a resource (\constant{PY_RESOURCE});
on Windows, it looks in the registry which may point to a specific
file).

If search is successful, the return value is a triple
\code{(\var{file}, \var{pathname}, \var{description})} where
\var{file} is an open file object positioned at the beginning,
\var{pathname} is the pathname of the
file found, and \var{description} is a triple as contained in the list
returned by \function{get_suffixes()} describing the kind of module found.
If the module does not live in a file, the returned \var{file} is
\code{None}, \var{filename} is the empty string, and the
\var{description} tuple contains empty strings for its suffix and
mode; the module type is as indicate in parentheses above.  If the
search is unsuccessful, \exception{ImportError} is raised.  Other
exceptions indicate problems with the arguments or environment.

This function does not handle hierarchical module names (names
containing dots).  In order to find \var{P}.\var{M}, that is, submodule
\var{M} of package \var{P}, use \function{find_module()} and
\function{load_module()} to find and load package \var{P}, and then use
\function{find_module()} with the \var{path} argument set to
\code{\var{P}.__path__}.  When \var{P} itself has a dotted name, apply
this recipe recursively.
\end{funcdesc}

\begin{funcdesc}{load_module}{name, file, filename, description}
Load a module that was previously found by \function{find_module()} (or by
an otherwise conducted search yielding compatible results).  This
function does more than importing the module: if the module was
already imported, it is equivalent to a
\function{reload()}\bifuncindex{reload}!  The \var{name} argument
indicates the full module name (including the package name, if this is
a submodule of a package).  The \var{file} argument is an open file,
and \var{filename} is the corresponding file name; these can be
\code{None} and \code{''}, respectively, when the module is not being
loaded from a file.  The \var{description} argument is a tuple, as
would be returned by \function{get_suffixes()}, describing what kind
of module must be loaded.

If the load is successful, the return value is the module object;
otherwise, an exception (usually \exception{ImportError}) is raised.

\strong{Important:} the caller is responsible for closing the
\var{file} argument, if it was not \code{None}, even when an exception
is raised.  This is best done using a \keyword{try}
... \keyword{finally} statement.
\end{funcdesc}

\begin{funcdesc}{new_module}{name}
Return a new empty module object called \var{name}.  This object is
\emph{not} inserted in \code{sys.modules}.
\end{funcdesc}

\begin{funcdesc}{lock_held}{}
Return \code{True} if the import lock is currently held, else \code{False}.
On platforms without threads, always return \code{False}.

On platforms with threads, a thread executing an import holds an internal
lock until the import is complete.
This lock blocks other threads from doing an import until the original
import completes, which in turn prevents other threads from seeing
incomplete module objects constructed by the original thread while in
the process of completing its import (and the imports, if any,
triggered by that).
\end{funcdesc}

\begin{funcdesc}{acquire_lock}{}
Acquires the interpreter's import lock for the current thread.  This lock
should be used by import hooks to ensure thread-safety when importing modules.
On platforms without threads, this function does nothing.
\versionadded{2.3}
\end{funcdesc}

\begin{funcdesc}{release_lock}{}
Release the interpreter's import lock.
On platforms without threads, this function does nothing.
\versionadded{2.3}
\end{funcdesc}

The following constants with integer values, defined in this module,
are used to indicate the search result of \function{find_module()}.

\begin{datadesc}{PY_SOURCE}
The module was found as a source file.
\end{datadesc}

\begin{datadesc}{PY_COMPILED}
The module was found as a compiled code object file.
\end{datadesc}

\begin{datadesc}{C_EXTENSION}
The module was found as dynamically loadable shared library.
\end{datadesc}

\begin{datadesc}{PY_RESOURCE}
The module was found as a Mac OS 9 resource.  This value can only be
returned on a Mac OS 9 or earlier Macintosh.
\end{datadesc}

\begin{datadesc}{PKG_DIRECTORY}
The module was found as a package directory.
\end{datadesc}

\begin{datadesc}{C_BUILTIN}
The module was found as a built-in module.
\end{datadesc}

\begin{datadesc}{PY_FROZEN}
The module was found as a frozen module (see \function{init_frozen()}).
\end{datadesc}

The following constant and functions are obsolete; their functionality
is available through \function{find_module()} or \function{load_module()}.
They are kept around for backward compatibility:

\begin{datadesc}{SEARCH_ERROR}
Unused.
\end{datadesc}

\begin{funcdesc}{init_builtin}{name}
Initialize the built-in module called \var{name} and return its module
object.  If the module was already initialized, it will be initialized
\emph{again}.  A few modules cannot be initialized twice --- attempting
to initialize these again will raise an \exception{ImportError}
exception.  If there is no
built-in module called \var{name}, \code{None} is returned.
\end{funcdesc}

\begin{funcdesc}{init_frozen}{name}
Initialize the frozen module called \var{name} and return its module
object.  If the module was already initialized, it will be initialized
\emph{again}.  If there is no frozen module called \var{name},
\code{None} is returned.  (Frozen modules are modules written in
Python whose compiled byte-code object is incorporated into a
custom-built Python interpreter by Python's \program{freeze} utility.
See \file{Tools/freeze/} for now.)
\end{funcdesc}

\begin{funcdesc}{is_builtin}{name}
Return \code{1} if there is a built-in module called \var{name} which
can be initialized again.  Return \code{-1} if there is a built-in
module called \var{name} which cannot be initialized again (see
\function{init_builtin()}).  Return \code{0} if there is no built-in
module called \var{name}.
\end{funcdesc}

\begin{funcdesc}{is_frozen}{name}
Return \code{True} if there is a frozen module (see
\function{init_frozen()}) called \var{name}, or \code{False} if there is
no such module.
\end{funcdesc}

\begin{funcdesc}{load_compiled}{name, pathname, \optional{file}}
\indexii{file}{byte-code}
Load and initialize a module implemented as a byte-compiled code file
and return its module object.  If the module was already initialized,
it will be initialized \emph{again}.  The \var{name} argument is used
to create or access a module object.  The \var{pathname} argument
points to the byte-compiled code file.  The \var{file}
argument is the byte-compiled code file, open for reading in binary
mode, from the beginning.
It must currently be a real file object, not a
user-defined class emulating a file.
\end{funcdesc}

\begin{funcdesc}{load_dynamic}{name, pathname\optional{, file}}
Load and initialize a module implemented as a dynamically loadable
shared library and return its module object.  If the module was
already initialized, it will be initialized \emph{again}.  Some modules
don't like that and may raise an exception.  The \var{pathname}
argument must point to the shared library.  The \var{name} argument is
used to construct the name of the initialization function: an external
C function called \samp{init\var{name}()} in the shared library is
called.  The optional \var{file} argument is ignored.  (Note: using
shared libraries is highly system dependent, and not all systems
support it.)
\end{funcdesc}

\begin{funcdesc}{load_source}{name, pathname\optional{, file}}
Load and initialize a module implemented as a Python source file and
return its module object.  If the module was already initialized, it
will be initialized \emph{again}.  The \var{name} argument is used to
create or access a module object.  The \var{pathname} argument points
to the source file.  The \var{file} argument is the source
file, open for reading as text, from the beginning.
It must currently be a real file
object, not a user-defined class emulating a file.  Note that if a
properly matching byte-compiled file (with suffix \file{.pyc} or
\file{.pyo}) exists, it will be used instead of parsing the given
source file.
\end{funcdesc}


\subsection{Examples}
\label{examples-imp}

The following function emulates what was the standard import statement
up to Python 1.4 (no hierarchical module names).  (This
\emph{implementation} wouldn't work in that version, since
\function{find_module()} has been extended and
\function{load_module()} has been added in 1.4.)

\begin{verbatim}
import imp
import sys

def __import__(name, globals=None, locals=None, fromlist=None):
    # Fast path: see if the module has already been imported.
    try:
        return sys.modules[name]
    except KeyError:
        pass

    # If any of the following calls raises an exception,
    # there's a problem we can't handle -- let the caller handle it.

    fp, pathname, description = imp.find_module(name)
    
    try:
        return imp.load_module(name, fp, pathname, description)
    finally:
        # Since we may exit via an exception, close fp explicitly.
        if fp:
            fp.close()
\end{verbatim}

A more complete example that implements hierarchical module names and
includes a \function{reload()}\bifuncindex{reload} function can be
found in the module \module{knee}\refmodindex{knee}.  The
\module{knee} module can be found in \file{Demo/imputil/} in the
Python source distribution.

\section{Built-in Module \sectcode{__builtin__}}
\bimodindex{__builtin__}

This module provides direct access to all `built-in' identifier of
Python; e.g. \code{__builtin__.open} is the full name for the built-in
function \code{open}.
		% really __builtin__
\section{Built-in Module \module{__main__}}
\label{module-main}
\bimodindex{__main__}
This module represents the (otherwise anonymous) scope in which the
interpreter's main program executes --- commands read either from
standard input or from a script file.
			% really __main__

\chapter{String Services}
\label{strings}

The modules described in this chapter provide a wide range of string
manipulation operations.  Here's an overview:

\begin{description}

\item[string]
--- Common string operations.

\item[re]
--- New Perl-style regular expression search and match operations.

\item[regex]
--- Regular expression search and match operations.

\item[regsub]
--- Substitution and splitting operations that use regular expressions.

\item[struct]
--- Interpret strings as packed binary data.

\item[StringIO]
--- Read and write strings as if they were files.

\end{description}
		% String Services
\section{Standard Module \sectcode{string}}
\label{module-string}
\stmodindex{string}

This module defines some constants useful for checking character
classes and some useful string functions.  See the module
\module{re}\refstmodindex{re} for string functions based on regular
expressions.

The constants defined in this module are are:

\setindexsubitem{(data in module string)}
\begin{datadesc}{digits}
  The string \code{'0123456789'}.
\end{datadesc}

\begin{datadesc}{hexdigits}
  The string \code{'0123456789abcdefABCDEF'}.
\end{datadesc}

\begin{datadesc}{letters}
  The concatenation of the strings \function{lowercase()} and
  \function{uppercase()} described below.
\end{datadesc}

\begin{datadesc}{lowercase}
  A string containing all the characters that are considered lowercase
  letters.  On most systems this is the string
  \code{'abcdefghijklmnopqrstuvwxyz'}.  Do not change its definition ---
  the effect on the routines \function{upper()} and
  \function{swapcase()} is undefined.
\end{datadesc}

\begin{datadesc}{octdigits}
  The string \code{'01234567'}.
\end{datadesc}

\begin{datadesc}{uppercase}
  A string containing all the characters that are considered uppercase
  letters.  On most systems this is the string
  \code{'ABCDEFGHIJKLMNOPQRSTUVWXYZ'}.  Do not change its definition ---
  the effect on the routines \function{lower()} and
  \function{swapcase()} is undefined.
\end{datadesc}

\begin{datadesc}{whitespace}
  A string containing all characters that are considered whitespace.
  On most systems this includes the characters space, tab, linefeed,
  return, formfeed, and vertical tab.  Do not change its definition ---
  the effect on the routines \function{strip()} and \function{split()}
  is undefined.
\end{datadesc}

The functions defined in this module are:


\begin{funcdesc}{atof}{s}
Convert a string to a floating point number.  The string must have
the standard syntax for a floating point literal in Python, optionally
preceded by a sign (\samp{+} or \samp{-}).  Note that this behaves
identical to the built-in function
\function{float()}\bifuncindex{float} when passed a string.
\end{funcdesc}

\begin{funcdesc}{atoi}{s\optional{, base}}
Convert string \var{s} to an integer in the given \var{base}.  The
string must consist of one or more digits, optionally preceded by a
sign (\samp{+} or \samp{-}).  The \var{base} defaults to 10.  If it is
0, a default base is chosen depending on the leading characters of the
string (after stripping the sign): \samp{0x} or \samp{0X} means 16,
\samp{0} means 8, anything else means 10.  If \var{base} is 16, a
leading \samp{0x} or \samp{0X} is always accepted.  Note that when
invoked without \var{base} or with \var{base} set to 10, this behaves
identical to the built-in function \function{int()} when passed a string.
(Also note: for a more flexible interpretation of numeric literals,
use the built-in function \function{eval()}\bifuncindex{eval}.)
\end{funcdesc}

\begin{funcdesc}{atol}{s\optional{, base}}
Convert string \var{s} to a long integer in the given \var{base}.  The 
string must consist of one or more digits, optionally preceded by a
sign (\samp{+} or \samp{-}).  The \var{base} argument has the same
meaning as for \function{atoi()}.  A trailing \samp{l} or \samp{L} is
not allowed, except if the base is 0.  Note that when invoked without
\var{base} or with \var{base} set to 10, this behaves identical to the
built-in function \function{long()}\bifuncindex{long} when passed a
string.
\end{funcdesc}

\begin{funcdesc}{capitalize}{word}
Capitalize the first character of the argument.
\end{funcdesc}

\begin{funcdesc}{capwords}{s}
Split the argument into words using \function{split()}, capitalize
each word using \function{capitalize()}, and join the capitalized
words using \function{join()}.  Note that this replaces runs of
whitespace characters by a single space, and removes leading and
trailing whitespace.
\end{funcdesc}

\begin{funcdesc}{expandtabs}{s, tabsize}
Expand tabs in a string, i.e.\ replace them by one or more spaces,
depending on the current column and the given tab size.  The column
number is reset to zero after each newline occurring in the string.
This doesn't understand other non-printing characters or escape
sequences.
\end{funcdesc}

\begin{funcdesc}{find}{s, sub\optional{, start\optional{,end}}}
Return the lowest index in \var{s} where the substring \var{sub} is
found such that \var{sub} is wholly contained in
\code{\var{s}[\var{start}:\var{end}]}.  Return \code{-1} on failure.
Defaults for \var{start} and \var{end} and interpretation of negative
values is the same as for slices.
\end{funcdesc}

\begin{funcdesc}{rfind}{s, sub\optional{, start\optional{, end}}}
Like \function{find()} but find the highest index.
\end{funcdesc}

\begin{funcdesc}{index}{s, sub\optional{, start\optional{, end}}}
Like \function{find()} but raise \exception{ValueError} when the
substring is not found.
\end{funcdesc}

\begin{funcdesc}{rindex}{s, sub\optional{, start\optional{, end}}}
Like \function{rfind()} but raise \exception{ValueError} when the
substring is not found.
\end{funcdesc}

\begin{funcdesc}{count}{s, sub\optional{, start\optional{, end}}}
Return the number of (non-overlapping) occurrences of substring
\var{sub} in string \code{\var{s}[\var{start}:\var{end}]}.
Defaults for \var{start} and \var{end} and interpretation of negative
values is the same as for slices.
\end{funcdesc}

\begin{funcdesc}{lower}{s}
Convert letters to lower case.
\end{funcdesc}

\begin{funcdesc}{maketrans}{from, to}
Return a translation table suitable for passing to
\function{translate()} or \function{regex.compile()}, that will map
each character in \var{from} into the character at the same position
in \var{to}; \var{from} and \var{to} must have the same length. 
\end{funcdesc}

\begin{funcdesc}{split}{s\optional{, sep\optional{, maxsplit}}}
Return a list of the words of the string \var{s}.  If the optional
second argument \var{sep} is absent or \code{None}, the words are
separated by arbitrary strings of whitespace characters (space, tab,
newline, return, formfeed).  If the second argument \var{sep} is
present and not \code{None}, it specifies a string to be used as the
word separator.  The returned list will then have one more items than
the number of non-overlapping occurrences of the separator in the
string.  The optional third argument \var{maxsplit} defaults to 0.  If
it is nonzero, at most \var{maxsplit} number of splits occur, and the
remainder of the string is returned as the final element of the list
(thus, the list will have at most \code{\var{maxsplit}+1} elements).
\end{funcdesc}

\begin{funcdesc}{splitfields}{s\optional{, sep\optional{, maxsplit}}}
This function behaves identically to \function{split()}.  (In the
past, \function{split()} was only used with one argument, while
\function{splitfields()} was only used with two arguments.)
\end{funcdesc}

\begin{funcdesc}{join}{words\optional{, sep}}
Concatenate a list or tuple of words with intervening occurrences of
\var{sep}.  The default value for \var{sep} is a single space
character.  It is always true that
\samp{string.join(string.split(\var{s}, \var{sep}), \var{sep})}
equals \var{s}.
\end{funcdesc}

\begin{funcdesc}{joinfields}{words\optional{, sep}}
This function behaves identical to \function{join()}.  (In the past,
\function{join()} was only used with one argument, while
\function{joinfields()} was only used with two arguments.)
\end{funcdesc}

\begin{funcdesc}{lstrip}{s}
Remove leading whitespace from the string \var{s}.
\end{funcdesc}

\begin{funcdesc}{rstrip}{s}
Remove trailing whitespace from the string \var{s}.
\end{funcdesc}

\begin{funcdesc}{strip}{s}
Remove leading and trailing whitespace from the string \var{s}.
\end{funcdesc}

\begin{funcdesc}{swapcase}{s}
Convert lower case letters to upper case and vice versa.
\end{funcdesc}

\begin{funcdesc}{translate}{s, table\optional{, deletechars}}
Delete all characters from \var{s} that are in \var{deletechars} (if
present), and then translate the characters using \var{table}, which
must be a 256-character string giving the translation for each
character value, indexed by its ordinal.  
\end{funcdesc}

\begin{funcdesc}{upper}{s}
Convert letters to upper case.
\end{funcdesc}

\begin{funcdesc}{ljust}{s, width}
\funcline{rjust}{s, width}
\funcline{center}{s, width}
These functions respectively left-justify, right-justify and center a
string in a field of given width.
They return a string that is at least
\var{width}
characters wide, created by padding the string
\var{s}
with spaces until the given width on the right, left or both sides.
The string is never truncated.
\end{funcdesc}

\begin{funcdesc}{zfill}{s, width}
Pad a numeric string on the left with zero digits until the given
width is reached.  Strings starting with a sign are handled correctly.
\end{funcdesc}

\begin{funcdesc}{replace}{str, old, new\optional{, maxsplit}}
Return a copy of string \var{str} with all occurrences of substring
\var{old} replaced by \var{new}.  If the optional argument
\var{maxsplit} is given, the first \var{maxsplit} occurrences are
replaced.
\end{funcdesc}

This module is implemented in Python.  Much of its functionality has
been reimplemented in the built-in module
\module{strop}\refbimodindex{strop}.  However, you
should \emph{never} import the latter module directly.  When
\module{string} discovers that \module{strop} exists, it transparently
replaces parts of itself with the implementation from \module{strop}.
After initialization, there is \emph{no} overhead in using
\module{string} instead of \module{strop}.

\section{Built-in Module \sectcode{regex}}

\bimodindex{regex}
This module provides regular expression matching operations similar to
those found in Emacs.  It is always available.

By default the patterns are Emacs-style regular expressions; there is
a way to change the syntax to match that of several well-known
\UNIX{} utilities.

This module is 8-bit clean: both patterns and strings may contain null
bytes and characters whose high bit is set.

\strong{Please note:} There is a little-known fact about Python string
literals which means that you don't usually have to worry about
doubling backslashes, even though they are used to escape special
characters in string literals as well as in regular expressions.  This
is because Python doesn't remove backslashes from string literals if
they are followed by an unrecognized escape character.
\emph{However}, if you want to include a literal \dfn{backslash} in a
regular expression represented as a string literal, you have to
\emph{quadruple} it.  E.g.  to extract LaTeX \samp{\e section\{{\rm
\ldots}\}} headers from a document, you can use this pattern:
\code{'\e \e \e\e section\{\e (.*\e )\}'}.

The module defines these functions, and an exception:

\renewcommand{\indexsubitem}{(in module regex)}

\begin{funcdesc}{match}{pattern\, string}
  Return how many characters at the beginning of \var{string} match
  the regular expression \var{pattern}.  Return \code{-1} if the
  string does not match the pattern (this is different from a
  zero-length match!).
\end{funcdesc}

\begin{funcdesc}{search}{pattern\, string}
  Return the first position in \var{string} that matches the regular
  expression \var{pattern}.  Return -1 if no position in the string
  matches the pattern (this is different from a zero-length match
  anywhere!).
\end{funcdesc}

\begin{funcdesc}{compile}{pattern\optional{\, translate}}
  Compile a regular expression pattern into a regular expression
  object, which can be used for matching using its \code{match} and
  \code{search} methods, described below.  The optional
  \var{translate}, if present, must be a 256-character string
  indicating how characters (both of the pattern and of the strings to
  be matched) are translated before comparing them; the \code{i}-th
  element of the string gives the translation for the character with
  ASCII code \code{i}.

  The sequence

\bcode\begin{verbatim}
prog = regex.compile(pat)
result = prog.match(str)
\end{verbatim}\ecode

is equivalent to

\bcode\begin{verbatim}
result = regex.match(pat, str)
\end{verbatim}\ecode

but the version using \code{compile()} is more efficient when multiple
regular expressions are used concurrently in a single program.  (The
compiled version of the last pattern passed to \code{regex.match()} or
\code{regex.search()} is cached, so programs that use only a single
regular expression at a time needn't worry about compiling regular
expressions.)
\end{funcdesc}

\begin{funcdesc}{set_syntax}{flags}
  Set the syntax to be used by future calls to \code{compile},
  \code{match} and \code{search}.  (Already compiled expression objects
  are not affected.)  The argument is an integer which is the OR of
  several flag bits.  The return value is the previous value of
  the syntax flags.  Names for the flags are defined in the standard
  module \code{regex_syntax}; read the file \file{regex_syntax.py} for
  more information.
\end{funcdesc}

\begin{funcdesc}{symcomp}{pattern\optional{\, translate}}
This is like \code{compile}, but supports symbolic group names: if a
parentheses-enclosed group begins with a group name in angular
brackets, e.g. \code{'\e(<id>[a-z][a-z0-9]*\e)'}, the group can
be referenced by its name in arguments to the \code{group} method of
the resulting compiled regular expression object, like this:
\code{p.group('id')}.
\end{funcdesc}

\begin{excdesc}{error}
  Exception raised when a string passed to one of the functions here
  is not a valid regular expression (e.g., unmatched parentheses) or
  when some other error occurs during compilation or matching.  (It is
  never an error if a string contains no match for a pattern.)
\end{excdesc}

\begin{datadesc}{casefold}
A string suitable to pass as \var{translate} argument to
\code{compile} to map all upper case characters to their lowercase
equivalents.
\end{datadesc}

\noindent
Compiled regular expression objects support these methods:

\renewcommand{\indexsubitem}{(regex method)}
\begin{funcdesc}{match}{string\optional{\, pos}}
  Return how many characters at the beginning of \var{string} match
  the compiled regular expression.  Return \code{-1} if the string
  does not match the pattern (this is different from a zero-length
  match!).
  
  The optional second parameter \var{pos} gives an index in the string
  where the search is to start; it defaults to \code{0}.  This is not
  completely equivalent to slicing the string; the \code{'\^'} pattern
  character matches at the real begin of the string and at positions
  just after a newline, not necessarily at the index where the search
  is to start.
\end{funcdesc}

\begin{funcdesc}{search}{string\optional{\, pos}}
  Return the first position in \var{string} that matches the regular
  expression \code{pattern}.  Return \code{-1} if no position in the
  string matches the pattern (this is different from a zero-length
  match anywhere!).
  
  The optional second parameter has the same meaning as for the
  \code{match} method.
\end{funcdesc}

\begin{funcdesc}{group}{index\, index\, ...}
This method is only valid when the last call to the \code{match}
or \code{search} method found a match.  It returns one or more
groups of the match.  If there is a single \var{index} argument,
the result is a single string; if there are multiple arguments, the
result is a tuple with one item per argument.  If the \var{index} is
zero, the corresponding return value is the entire matching string; if
it is in the inclusive range [1..99], it is the string matching the
the corresponding parenthesized group (using the default syntax,
groups are parenthesized using \code{\\(} and \code{\\)}).  If no
such group exists, the corresponding result is \code{None}.

If the regular expression was compiled by \code{symcomp} instead of
\code{compile}, the \var{index} arguments may also be strings
identifying groups by their group name.
\end{funcdesc}

\noindent
Compiled regular expressions support these data attributes:

\renewcommand{\indexsubitem}{(regex attribute)}

\begin{datadesc}{regs}
When the last call to the \code{match} or \code{search} method found a
match, this is a tuple of pairs of indices corresponding to the
beginning and end of all parenthesized groups in the pattern.  Indices
are relative to the string argument passed to \code{match} or
\code{search}.  The 0-th tuple gives the beginning and end or the
whole pattern.  When the last match or search failed, this is
\code{None}.
\end{datadesc}

\begin{datadesc}{last}
When the last call to the \code{match} or \code{search} method found a
match, this is the string argument passed to that method.  When the
last match or search failed, this is \code{None}.
\end{datadesc}

\begin{datadesc}{translate}
This is the value of the \var{translate} argument to
\code{regex.compile} that created this regular expression object.  If
the \var{translate} argument was omitted in the \code{regex.compile}
call, this is \code{None}.
\end{datadesc}

\begin{datadesc}{givenpat}
The regular expression pattern as passed to \code{compile} or
\code{symcomp}.
\end{datadesc}

\begin{datadesc}{realpat}
The regular expression after stripping the group names for regular
expressions compiled with \code{symcomp}.  Same as \code{givenpat}
otherwise.
\end{datadesc}

\begin{datadesc}{groupindex}
A dictionary giving the mapping from symbolic group names to numerical
group indices for regular expressions compiled with \code{symcomp}.
\code{None} otherwise.
\end{datadesc}

\section{\module{regsub} ---
         String operations using regular expressions}

\declaremodule{standard}{regsub}
\modulesynopsis{Substitution and splitting operations that use
                regular expressions.  \strong{Obsolete!}}


This module defines a number of functions useful for working with
regular expressions (see built-in module \refmodule{regex}).

Warning: these functions are not thread-safe.

\strong{Obsolescence note:}
This module is obsolete as of Python version 1.5; it is still being
maintained because much existing code still uses it.  All new code in
need of regular expressions should use the new \refmodule{re} module, which
supports the more powerful and regular Perl-style regular expressions.
Existing code should be converted.  The standard library module
\module{reconvert} helps in converting \refmodule{regex} style regular
expressions to \refmodule{re} style regular expressions.  (For more
conversion help, see Andrew Kuchling's\index{Kuchling, Andrew}
``regex-to-re HOWTO'' at
\url{http://www.python.org/doc/howto/regex-to-re/}.)


\begin{funcdesc}{sub}{pat, repl, str}
Replace the first occurrence of pattern \var{pat} in string
\var{str} by replacement \var{repl}.  If the pattern isn't found,
the string is returned unchanged.  The pattern may be a string or an
already compiled pattern.  The replacement may contain references
\samp{\e \var{digit}} to subpatterns and escaped backslashes.
\end{funcdesc}

\begin{funcdesc}{gsub}{pat, repl, str}
Replace all (non-overlapping) occurrences of pattern \var{pat} in
string \var{str} by replacement \var{repl}.  The same rules as for
\code{sub()} apply.  Empty matches for the pattern are replaced only
when not adjacent to a previous match, so e.g.
\code{gsub('', '-', 'abc')} returns \code{'-a-b-c-'}.
\end{funcdesc}

\begin{funcdesc}{split}{str, pat\optional{, maxsplit}}
Split the string \var{str} in fields separated by delimiters matching
the pattern \var{pat}, and return a list containing the fields.  Only
non-empty matches for the pattern are considered, so e.g.
\code{split('a:b', ':*')} returns \code{['a', 'b']} and
\code{split('abc', '')} returns \code{['abc']}.  The \var{maxsplit}
defaults to 0. If it is nonzero, only \var{maxsplit} number of splits
occur, and the remainder of the string is returned as the final
element of the list.
\end{funcdesc}

\begin{funcdesc}{splitx}{str, pat\optional{, maxsplit}}
Split the string \var{str} in fields separated by delimiters matching
the pattern \var{pat}, and return a list containing the fields as well
as the separators.  For example, \code{splitx('a:::b', ':*')} returns
\code{['a', ':::', 'b']}.  Otherwise, this function behaves the same
as \code{split}.
\end{funcdesc}

\begin{funcdesc}{capwords}{s\optional{, pat}}
Capitalize words separated by optional pattern \var{pat}.  The default
pattern uses any characters except letters, digits and underscores as
word delimiters.  Capitalization is done by changing the first
character of each word to upper case.
\end{funcdesc}

\begin{funcdesc}{clear_cache}{}
The regsub module maintains a cache of compiled regular expressions,
keyed on the regular expression string and the syntax of the regex
module at the time the expression was compiled.  This function clears
that cache.
\end{funcdesc}

\section{Built-in Module \module{struct}}
\declaremodule{builtin}{struct}

\modulesynopsis{Interpret strings as packed binary data.}

\indexii{C@\C{}}{structures}

This module performs conversions between Python values and C
structs represented as Python strings.  It uses \dfn{format strings}
(explained below) as compact descriptions of the lay-out of the C
structs and the intended conversion to/from Python values.

The module defines the following exception and functions:


\begin{excdesc}{error}
  Exception raised on various occasions; argument is a string
  describing what is wrong.
\end{excdesc}

\begin{funcdesc}{pack}{fmt, v1, v2, {\rm \ldots}}
  Return a string containing the values
  \code{\var{v1}, \var{v2}, {\rm \ldots}} packed according to the given
  format.  The arguments must match the values required by the format
  exactly.
\end{funcdesc}

\begin{funcdesc}{unpack}{fmt, string}
  Unpack the string (presumably packed by \code{pack(\var{fmt}, {\rm \ldots})})
  according to the given format.  The result is a tuple even if it
  contains exactly one item.  The string must contain exactly the
  amount of data required by the format (i.e.  \code{len(\var{string})} must
  equal \code{calcsize(\var{fmt})}).
\end{funcdesc}

\begin{funcdesc}{calcsize}{fmt}
  Return the size of the struct (and hence of the string)
  corresponding to the given format.
\end{funcdesc}

Format characters have the following meaning; the conversion between C
and Python values should be obvious given their types:

\begin{tableiii}{c|l|l}{samp}{Format}{C Type}{Python}
  \lineiii{x}{pad byte}{no value}
  \lineiii{c}{char}{string of length 1}
  \lineiii{b}{signed char}{integer}
  \lineiii{B}{unsigned char}{integer}
  \lineiii{h}{short}{integer}
  \lineiii{H}{unsigned short}{integer}
  \lineiii{i}{int}{integer}
  \lineiii{I}{unsigned int}{integer}
  \lineiii{l}{long}{integer}
  \lineiii{L}{unsigned long}{integer}
  \lineiii{f}{float}{float}
  \lineiii{d}{double}{float}
  \lineiii{s}{char[]}{string}
\end{tableiii}

A format character may be preceded by an integral repeat count; e.g.\
the format string \code{'4h'} means exactly the same as \code{'hhhh'}.

Whitespace characters between formats are ignored; a count and its
format must not contain whitespace though.

For the \code{'s'} format character, the count is interpreted as the
size of the string, not a repeat count like for the other format
characters; e.g. \code{'10s'} means a single 10-byte string, while
\code{'10c'} means 10 characters.  For packing, the string is
truncated or padded with null bytes as appropriate to make it fit.
For unpacking, the resulting string always has exactly the specified
number of bytes.  As a special case, \code{'0s'} means a single, empty
string (while \code{'0c'} means 0 characters).

For the \code{'I'} and \code{'L'} format characters, the return
value is a Python long integer.

By default, C numbers are represented in the machine's native format
and byte order, and properly aligned by skipping pad bytes if
necessary (according to the rules used by the C compiler).

Alternatively, the first character of the format string can be used to
indicate the byte order, size and alignment of the packed data,
according to the following table:

\begin{tableiii}{c|l|l}{samp}{Character}{Byte order}{Size and alignment}
  \lineiii{@}{native}{native}
  \lineiii{=}{native}{standard}
  \lineiii{<}{little-endian}{standard}
  \lineiii{>}{big-endian}{standard}
  \lineiii{!}{network (= big-endian)}{standard}
\end{tableiii}

If the first character is not one of these, \code{'@'} is assumed.

Native byte order is big-endian or little-endian, depending on the
host system (e.g. Motorola and Sun are big-endian; Intel and DEC are
little-endian).

Native size and alignment are determined using the C compiler's sizeof
expression.  This is always combined with native byte order.

Standard size and alignment are as follows: no alignment is required
for any type (so you have to use pad bytes); short is 2 bytes; int and
long are 4 bytes.  Float and double are 32-bit and 64-bit IEEE floating
point numbers, respectively.

Note the difference between \code{'@'} and \code{'='}: both use native
byte order, but the size and alignment of the latter is standardized.

The form \code{'!'} is available for those poor souls who claim they
can't remember whether network byte order is big-endian or
little-endian.

There is no way to indicate non-native byte order (i.e. force
byte-swapping); use the appropriate choice of \code{'<'} or
\code{'>'}.

Examples (all using native byte order, size and alignment, on a
big-endian machine):

\begin{verbatim}
>>> from struct import *
>>> pack('hhl', 1, 2, 3)
'\000\001\000\002\000\000\000\003'
>>> unpack('hhl', '\000\001\000\002\000\000\000\003')
(1, 2, 3)
>>> calcsize('hhl')
8
>>> 
\end{verbatim}
%
Hint: to align the end of a structure to the alignment requirement of
a particular type, end the format with the code for that type with a
repeat count of zero, e.g.\ the format \code{'llh0l'} specifies two
pad bytes at the end, assuming longs are aligned on 4-byte boundaries.
This only works when native size and alignment are in effect;
standard size and alignment does not enforce any alignment.

\begin{seealso}
\seemodule{array}{packed binary storage of homogeneous data}
\end{seealso}


\chapter{Miscellaneous Services}
\label{misc}

The modules described in this chapter provide miscellaneous services
that are available in all Python versions.  Here's an overview:

\localmoduletable
			% Miscellaneous Services
\section{Built-in Module \sectcode{math}}
\label{module-math}

\bimodindex{math}
\renewcommand{\indexsubitem}{(in module math)}
This module is always available.
It provides access to the mathematical functions defined by the C
standard.
They are:

\begin{funcdesc}{acos}{x}
Return the arc cosine of \var{x}.
\end{funcdesc}

\begin{funcdesc}{asin}{x}
Return the arc sine of \var{x}.
\end{funcdesc}

\begin{funcdesc}{atan}{x}
Return the arc tangent of \var{x}.
\end{funcdesc}

\begin{funcdesc}{atan2}{x, y}
Return \code{atan(x / y)}.
\end{funcdesc}

\begin{funcdesc}{ceil}{x}
Return the ceiling of \var{x}.
\end{funcdesc}

\begin{funcdesc}{cos}{x}
Return the cosine of \var{x}.
\end{funcdesc}

\begin{funcdesc}{cosh}{x}
Return the hyperbolic cosine of \var{x}.
\end{funcdesc}

\begin{funcdesc}{exp}{x}
Return the exponential value $\mbox{e}^x$.
\end{funcdesc}

\begin{funcdesc}{fabs}{x}
Return the absolute value of the real \var{x}.
\end{funcdesc}

\begin{funcdesc}{floor}{x}
Return the floor of \var{x}.
\end{funcdesc}

\begin{funcdesc}{fmod}{x, y}
Return \code{x \% y}.
\end{funcdesc}

\begin{funcdesc}{frexp}{x}
Return the matissa and exponent for \var{x}.  The mantissa is
positive.
\end{funcdesc}

\begin{funcdesc}{hypot}{x, y}
Return the Euclidean distance, \code{sqrt(x*x + y*y)}.
\end{funcdesc}

\begin{funcdesc}{ldexp}{x, i}
Return $x {\times} 2^i$.
\end{funcdesc}

\begin{funcdesc}{modf}{x}
Return the fractional and integer parts of \var{x}.  Both results
carry the sign of \var{x}.
\end{funcdesc}

\begin{funcdesc}{pow}{x, y}
Return $x^y$.
\end{funcdesc}

\begin{funcdesc}{sin}{x}
Return the sine of \var{x}.
\end{funcdesc}

\begin{funcdesc}{sinh}{x}
Return the hyperbolic sine of \var{x}.
\end{funcdesc}

\begin{funcdesc}{sqrt}{x}
Return the square root of \var{x}.
\end{funcdesc}

\begin{funcdesc}{tan}{x}
Return the tangent of \var{x}.
\end{funcdesc}

\begin{funcdesc}{tanh}{x}
Return the hyperbolic tangent of \var{x}.
\end{funcdesc}

Note that \code{frexp} and \code{modf} have a different call/return
pattern than their C equivalents: they take a single argument and
return a pair of values, rather than returning their second return
value through an `output parameter' (there is no such thing in Python).

The module also defines two mathematical constants:

\begin{datadesc}{pi}
The mathematical constant \emph{pi}.
\end{datadesc}

\begin{datadesc}{e}
The mathematical constant \emph{e}.
\end{datadesc}

\begin{seealso}
  \seemodule{cmath}{Complex number versions of many of these functions.}
\end{seealso}

\section{Standard Module \sectcode{rand}}

\stmodindex{rand} This module implements a pseudo-random number
generator with an interface similar to \code{rand()} in C\@.  It defines
the following functions:

\renewcommand{\indexsubitem}{(in module rand)}
\begin{funcdesc}{rand}{}
Returns an integer random number in the range [0 ... 32768).
\end{funcdesc}

\begin{funcdesc}{choice}{s}
Returns a random element from the sequence (string, tuple or list)
\var{s}.
\end{funcdesc}

\begin{funcdesc}{srand}{seed}
Initializes the random number generator with the given integral seed.
When the module is first imported, the random number is initialized with
the current time.
\end{funcdesc}

\section{Standard Module \sectcode{whrandom}}

\stmodindex{whrandom}
This module implements a Wichmann-Hill pseudo-random number generator.
It defines the following functions:

\renewcommand{\indexsubitem}{(in module whrandom)}
\begin{funcdesc}{random}{}
Returns the next random floating point number in the range [0.0 ... 1.0).
\end{funcdesc}

\begin{funcdesc}{seed}{x\, y\, z}
Initializes the random number generator from the integers
\var{x},
\var{y}
and
\var{z}.
When the module is first imported, the random number is initialized
using values derived from the current time.
\end{funcdesc}

\section{\module{array} ---
         Efficient arrays of numeric values}

\declaremodule{builtin}{array}
\modulesynopsis{Efficient arrays of uniformly typed numeric values.}


This module defines an object type which can efficiently represent
an array of basic values: characters, integers, floating point
numbers.  Arrays\index{arrays} are sequence types and behave very much
like lists, except that the type of objects stored in them is
constrained.  The type is specified at object creation time by using a
\dfn{type code}, which is a single character.  The following type
codes are defined:

\begin{tableiv}{c|l|l|c}{code}{Type code}{C Type}{Python Type}{Minimum size in bytes}
  \lineiv{'c'}{char}          {character}        {1}
  \lineiv{'b'}{signed char}   {int}              {1}
  \lineiv{'B'}{unsigned char} {int}              {1}
  \lineiv{'u'}{Py_UNICODE}    {Unicode character}{2}
  \lineiv{'h'}{signed short}  {int}              {2}
  \lineiv{'H'}{unsigned short}{int}              {2}
  \lineiv{'i'}{signed int}    {int}              {2}
  \lineiv{'I'}{unsigned int}  {long}             {2}
  \lineiv{'l'}{signed long}   {int}              {4}
  \lineiv{'L'}{unsigned long} {long}             {4}
  \lineiv{'f'}{float}         {float}            {4}
  \lineiv{'d'}{double}        {float}            {8}
\end{tableiv}

The actual representation of values is determined by the machine
architecture (strictly speaking, by the C implementation).  The actual
size can be accessed through the \member{itemsize} attribute.  The values
stored  for \code{'L'} and \code{'I'} items will be represented as
Python long integers when retrieved, because Python's plain integer
type cannot represent the full range of C's unsigned (long) integers.


The module defines the following type:

\begin{funcdesc}{array}{typecode\optional{, initializer}}
Return a new array whose items are restricted by \var{typecode},
and initialized from the optional \var{initializer} value, which
must be a list, string, or iterable over elements of the
appropriate type.
\versionchanged[Formerly, only lists or strings were accepted]{2.4}
If given a list or string, the initializer is passed to the
new array's \method{fromlist()}, \method{fromstring()}, or
\method{fromunicode()} method (see below) to add initial items to
the array.  Otherwise, the iterable initializer is passed to the
\method{extend()} method.
\end{funcdesc}

\begin{datadesc}{ArrayType}
Obsolete alias for \function{array}.
\end{datadesc}


Array objects support the ordinary sequence operations of
indexing, slicing, concatenation, and multiplication.  When using
slice assignment, the assigned value must be an array object with the
same type code; in all other cases, \exception{TypeError} is raised.
Array objects also implement the buffer interface, and may be used
wherever buffer objects are supported.

The following data items and methods are also supported:

\begin{memberdesc}[array]{typecode}
The typecode character used to create the array.
\end{memberdesc}

\begin{memberdesc}[array]{itemsize}
The length in bytes of one array item in the internal representation.
\end{memberdesc}


\begin{methoddesc}[array]{append}{x}
Append a new item with value \var{x} to the end of the array.
\end{methoddesc}

\begin{methoddesc}[array]{buffer_info}{}
Return a tuple \code{(\var{address}, \var{length})} giving the current
memory address and the length in elements of the buffer used to hold
array's contents.  The size of the memory buffer in bytes can be
computed as \code{\var{array}.buffer_info()[1] *
\var{array}.itemsize}.  This is occasionally useful when working with
low-level (and inherently unsafe) I/O interfaces that require memory
addresses, such as certain \cfunction{ioctl()} operations.  The
returned numbers are valid as long as the array exists and no
length-changing operations are applied to it.

\note{When using array objects from code written in C or
\Cpp{} (the only way to effectively make use of this information), it
makes more sense to use the buffer interface supported by array
objects.  This method is maintained for backward compatibility and
should be avoided in new code.  The buffer interface is documented in
the \citetitle[../api/newTypes.html]{Python/C API Reference Manual}.}
\end{methoddesc}

\begin{methoddesc}[array]{byteswap}{}
``Byteswap'' all items of the array.  This is only supported for
values which are 1, 2, 4, or 8 bytes in size; for other types of
values, \exception{RuntimeError} is raised.  It is useful when reading
data from a file written on a machine with a different byte order.
\end{methoddesc}

\begin{methoddesc}[array]{count}{x}
Return the number of occurrences of \var{x} in the array.
\end{methoddesc}

\begin{methoddesc}[array]{extend}{iterable}
Append items from \var{iterable} to the end of the array.  If
\var{iterable} is another array, it must have \emph{exactly} the same
type code; if not, \exception{TypeError} will be raised.  If
\var{iterable} is not an array, it must be iterable and its
elements must be the right type to be appended to the array.
\versionchanged[Formerly, the argument could only be another array]{2.4}
\end{methoddesc}

\begin{methoddesc}[array]{fromfile}{f, n}
Read \var{n} items (as machine values) from the file object \var{f}
and append them to the end of the array.  If less than \var{n} items
are available, \exception{EOFError} is raised, but the items that were
available are still inserted into the array.  \var{f} must be a real
built-in file object; something else with a \method{read()} method won't
do.
\end{methoddesc}

\begin{methoddesc}[array]{fromlist}{list}
Append items from the list.  This is equivalent to
\samp{for x in \var{list}:\ a.append(x)}
except that if there is a type error, the array is unchanged.
\end{methoddesc}

\begin{methoddesc}[array]{fromstring}{s}
Appends items from the string, interpreting the string as an
array of machine values (as if it had been read from a
file using the \method{fromfile()} method).
\end{methoddesc}

\begin{methoddesc}[array]{fromunicode}{s}
Extends this array with data from the given unicode string.
The array must be a type 'u' array; otherwise a ValueError
is raised.  Use \samp{array.fromstring(ustr.decode(enc))} to
append Unicode data to an array of some other type.
\end{methoddesc}

\begin{methoddesc}[array]{index}{x}
Return the smallest \var{i} such that \var{i} is the index of
the first occurrence of \var{x} in the array.
\end{methoddesc}

\begin{methoddesc}[array]{insert}{i, x}
Insert a new item with value \var{x} in the array before position
\var{i}. Negative values are treated as being relative to the end
of the array.
\end{methoddesc}

\begin{methoddesc}[array]{pop}{\optional{i}}
Removes the item with the index \var{i} from the array and returns
it. The optional argument defaults to \code{-1}, so that by default
the last item is removed and returned.
\end{methoddesc}

\begin{methoddesc}[array]{read}{f, n}
\deprecated {1.5.1}
  {Use the \method{fromfile()} method.}
Read \var{n} items (as machine values) from the file object \var{f}
and append them to the end of the array.  If less than \var{n} items
are available, \exception{EOFError} is raised, but the items that were
available are still inserted into the array.  \var{f} must be a real
built-in file object; something else with a \method{read()} method won't
do.
\end{methoddesc}

\begin{methoddesc}[array]{remove}{x}
Remove the first occurrence of \var{x} from the array.
\end{methoddesc}

\begin{methoddesc}[array]{reverse}{}
Reverse the order of the items in the array.
\end{methoddesc}

\begin{methoddesc}[array]{tofile}{f}
Write all items (as machine values) to the file object \var{f}.
\end{methoddesc}

\begin{methoddesc}[array]{tolist}{}
Convert the array to an ordinary list with the same items.
\end{methoddesc}

\begin{methoddesc}[array]{tostring}{}
Convert the array to an array of machine values and return the
string representation (the same sequence of bytes that would
be written to a file by the \method{tofile()} method.)
\end{methoddesc}

\begin{methoddesc}[array]{tounicode}{}
Convert the array to a unicode string.  The array must be
a type 'u' array; otherwise a ValueError is raised.  Use
array.tostring().decode(enc) to obtain a unicode string
from an array of some other type.
\end{methoddesc}

\begin{methoddesc}[array]{write}{f}
\deprecated {1.5.1}
  {Use the \method{tofile()} method.}
Write all items (as machine values) to the file object \var{f}.
\end{methoddesc}

When an array object is printed or converted to a string, it is
represented as \code{array(\var{typecode}, \var{initializer})}.  The
\var{initializer} is omitted if the array is empty, otherwise it is a
string if the \var{typecode} is \code{'c'}, otherwise it is a list of
numbers.  The string is guaranteed to be able to be converted back to
an array with the same type and value using reverse quotes
(\code{``}), so long as the \function{array()} function has been
imported using \code{from array import array}.  Examples:

\begin{verbatim}
array('l')
array('c', 'hello world')
array('u', u'hello \textbackslash u2641')
array('l', [1, 2, 3, 4, 5])
array('d', [1.0, 2.0, 3.14])
\end{verbatim}


\begin{seealso}
  \seemodule{struct}{Packing and unpacking of heterogeneous binary data.}
  \seemodule{xdrlib}{Packing and unpacking of External Data
                     Representation (XDR) data as used in some remote
                     procedure call systems.}
  \seetitle[http://numpy.sourceforge.net/numdoc/HTML/numdoc.htm]{The
           Numerical Python Manual}{The Numeric Python extension
           (NumPy) defines another array type; see
           \url{http://numpy.sourceforge.net/} for further information
           about Numerical Python.  (A PDF version of the NumPy manual
           is available at
           \url{http://numpy.sourceforge.net/numdoc/numdoc.pdf}).}
\end{seealso}


\chapter{Generic Operating System Services}

The modules described in this chapter provide interfaces to operating
system features that are available on (almost) all operating systems,
such as files and a clock.  The interfaces are generally modelled
after the \UNIX{} or C interfaces but they are available on most other
systems as well.  Here's an overview:

\begin{description}

\item[os]
--- Miscellaneous OS interfaces.

\item[time]
--- Time access and conversions.

\item[getopt]
--- Parser for command line options.

\item[tempfile]
--- Generate temporary file names.

\item[errno]
--- Standard errno system symbols.

\item[glob]
--- \UNIX{} shell style pathname pattern expansion.

\item[fnmatch]
--- \UNIX{} shell style pathname pattern matching.

\item[locale]
--- Internationalization services.

\end{description}
		% Generic Operating System Services
\section{\module{os} ---
         Miscellaneous operating system interfaces}

\declaremodule{standard}{os}
\modulesynopsis{Miscellaneous operating system interfaces.}


This module provides a more portable way of using operating system
dependent functionality than importing a operating system dependent
built-in module like \refmodule{posix} or \module{nt}.

This module searches for an operating system dependent built-in module like
\module{mac} or \refmodule{posix} and exports the same functions and data
as found there.  The design of all Python's built-in operating system dependent
modules is such that as long as the same functionality is available,
it uses the same interface; for example, the function
\code{os.stat(\var{path})} returns stat information about \var{path} in
the same format (which happens to have originated with the
\POSIX{} interface).

Extensions peculiar to a particular operating system are also
available through the \module{os} module, but using them is of course a
threat to portability!

Note that after the first time \module{os} is imported, there is
\emph{no} performance penalty in using functions from \module{os}
instead of directly from the operating system dependent built-in module,
so there should be \emph{no} reason not to use \module{os}!


% Frank Stajano <fstajano@uk.research.att.com> complained that it
% wasn't clear that the entries described in the subsections were all
% available at the module level (most uses of subsections are
% different); I think this is only a problem for the HTML version,
% where the relationship may not be as clear.
%
\ifhtml
The \module{os} module contains many functions and data values.
The items below and in the following sub-sections are all available
directly from the \module{os} module.
\fi


\begin{excdesc}{error}
This exception is raised when a function returns a system-related
error (not for illegal argument types or other incidental errors).
This is also known as the built-in exception \exception{OSError}.  The
accompanying value is a pair containing the numeric error code from
\cdata{errno} and the corresponding string, as would be printed by the
C function \cfunction{perror()}.  See the module
\refmodule{errno}\refbimodindex{errno}, which contains names for the
error codes defined by the underlying operating system.

When exceptions are classes, this exception carries two attributes,
\member{errno} and \member{strerror}.  The first holds the value of
the C \cdata{errno} variable, and the latter holds the corresponding
error message from \cfunction{strerror()}.  For exceptions that
involve a file system path (such as \function{chdir()} or
\function{unlink()}), the exception instance will contain a third
attribute, \member{filename}, which is the file name passed to the
function.
\end{excdesc}

\begin{datadesc}{name}
The name of the operating system dependent module imported.  The
following names have currently been registered: \code{'posix'},
\code{'nt'}, \code{'mac'}, \code{'os2'}, \code{'ce'},
\code{'java'}, \code{'riscos'}.
\end{datadesc}

\begin{datadesc}{path}
The corresponding operating system dependent standard module for pathname
operations, such as \module{posixpath} or \module{macpath}.  Thus,
given the proper imports, \code{os.path.split(\var{file})} is
equivalent to but more portable than
\code{posixpath.split(\var{file})}.  Note that this is also an
importable module: it may be imported directly as
\refmodule{os.path}.
\end{datadesc}



\subsection{Process Parameters \label{os-procinfo}}

These functions and data items provide information and operate on the
current process and user.

\begin{datadesc}{environ}
A mapping object representing the string environment. For example,
\code{environ['HOME']} is the pathname of your home directory (on some
platforms), and is equivalent to \code{getenv("HOME")} in C.

This mapping is captured the first time the \module{os} module is
imported, typically during Python startup as part of processing
\file{site.py}.  Changes to the environment made after this time are
not reflected in \code{os.environ}, except for changes made by modifying
\code{os.environ} directly.

If the platform supports the \function{putenv()} function, this
mapping may be used to modify the environment as well as query the
environment.  \function{putenv()} will be called automatically when
the mapping is modified.
\note{Calling \function{putenv()} directly does not change
\code{os.environ}, so it's better to modify \code{os.environ}.}
\note{On some platforms, including FreeBSD and Mac OS X, setting
\code{environ} may cause memory leaks.  Refer to the system documentation
for \cfunction{putenv()}.}

If \function{putenv()} is not provided, this mapping may be passed to
the appropriate process-creation functions to cause child processes to
use a modified environment.
\end{datadesc}

\begin{funcdescni}{chdir}{path}
\funclineni{fchdir}{fd}
\funclineni{getcwd}{}
These functions are described in ``Files and Directories'' (section
\ref{os-file-dir}).
\end{funcdescni}

\begin{funcdesc}{ctermid}{}
Return the filename corresponding to the controlling terminal of the
process.
Availability: \UNIX.
\end{funcdesc}

\begin{funcdesc}{getegid}{}
Return the effective group id of the current process.  This
corresponds to the `set id' bit on the file being executed in the
current process.
Availability: \UNIX.
\end{funcdesc}

\begin{funcdesc}{geteuid}{}
\index{user!effective id}
Return the current process' effective user id.
Availability: \UNIX.
\end{funcdesc}

\begin{funcdesc}{getgid}{}
\index{process!group}
Return the real group id of the current process.
Availability: \UNIX.
\end{funcdesc}

\begin{funcdesc}{getgroups}{}
Return list of supplemental group ids associated with the current
process.
Availability: \UNIX.
\end{funcdesc}

\begin{funcdesc}{getlogin}{}
Return the name of the user logged in on the controlling terminal of
the process.  For most purposes, it is more useful to use the
environment variable \envvar{LOGNAME} to find out who the user is,
or \code{pwd.getpwuid(os.getuid())[0]} to get the login name
of the currently effective user ID.
Availability: \UNIX.
\end{funcdesc}

\begin{funcdesc}{getpgid}{pid}
Return the process group id of the process with process id \var{pid}.
If \var{pid} is 0, the process group id of the current process is
returned. Availability: \UNIX.
\versionadded{2.3}
\end{funcdesc}

\begin{funcdesc}{getpgrp}{}
\index{process!group}
Return the id of the current process group.
Availability: \UNIX.
\end{funcdesc}

\begin{funcdesc}{getpid}{}
\index{process!id}
Return the current process id.
Availability: \UNIX, Windows.
\end{funcdesc}

\begin{funcdesc}{getppid}{}
\index{process!id of parent}
Return the parent's process id.
Availability: \UNIX.
\end{funcdesc}

\begin{funcdesc}{getuid}{}
\index{user!id}
Return the current process' user id.
Availability: \UNIX.
\end{funcdesc}

\begin{funcdesc}{getenv}{varname\optional{, value}}
Return the value of the environment variable \var{varname} if it
exists, or \var{value} if it doesn't.  \var{value} defaults to
\code{None}.
Availability: most flavors of \UNIX, Windows.
\end{funcdesc}

\begin{funcdesc}{putenv}{varname, value}
\index{environment variables!setting}
Set the environment variable named \var{varname} to the string
\var{value}.  Such changes to the environment affect subprocesses
started with \function{os.system()}, \function{popen()} or
\function{fork()} and \function{execv()}.
Availability: most flavors of \UNIX, Windows.

\note{On some platforms, including FreeBSD and Mac OS X,
setting \code{environ} may cause memory leaks.
Refer to the system documentation for putenv.}

When \function{putenv()} is
supported, assignments to items in \code{os.environ} are automatically
translated into corresponding calls to \function{putenv()}; however,
calls to \function{putenv()} don't update \code{os.environ}, so it is
actually preferable to assign to items of \code{os.environ}.
\end{funcdesc}

\begin{funcdesc}{setegid}{egid}
Set the current process's effective group id.
Availability: \UNIX.
\end{funcdesc}

\begin{funcdesc}{seteuid}{euid}
Set the current process's effective user id.
Availability: \UNIX.
\end{funcdesc}

\begin{funcdesc}{setgid}{gid}
Set the current process' group id.
Availability: \UNIX.
\end{funcdesc}

\begin{funcdesc}{setgroups}{groups}
Set the list of supplemental group ids associated with the current
process to \var{groups}. \var{groups} must be a sequence, and each
element must be an integer identifying a group. This operation is
typical available only to the superuser.
Availability: \UNIX.
\versionadded{2.2}
\end{funcdesc}

\begin{funcdesc}{setpgrp}{}
Calls the system call \cfunction{setpgrp()} or \cfunction{setpgrp(0,
0)} depending on which version is implemented (if any).  See the
\UNIX{} manual for the semantics.
Availability: \UNIX.
\end{funcdesc}

\begin{funcdesc}{setpgid}{pid, pgrp} Calls the system call
\cfunction{setpgid()} to set the process group id of the process with
id \var{pid} to the process group with id \var{pgrp}.  See the \UNIX{}
manual for the semantics.
Availability: \UNIX.
\end{funcdesc}

\begin{funcdesc}{setreuid}{ruid, euid}
Set the current process's real and effective user ids.
Availability: \UNIX.
\end{funcdesc}

\begin{funcdesc}{setregid}{rgid, egid}
Set the current process's real and effective group ids.
Availability: \UNIX.
\end{funcdesc}

\begin{funcdesc}{getsid}{pid}
Calls the system call \cfunction{getsid()}.  See the \UNIX{} manual
for the semantics.
Availability: \UNIX. \versionadded{2.4}
\end{funcdesc}

\begin{funcdesc}{setsid}{}
Calls the system call \cfunction{setsid()}.  See the \UNIX{} manual
for the semantics.
Availability: \UNIX.
\end{funcdesc}

\begin{funcdesc}{setuid}{uid}
\index{user!id, setting}
Set the current process' user id.
Availability: \UNIX.
\end{funcdesc}

% placed in this section since it relates to errno.... a little weak
\begin{funcdesc}{strerror}{code}
Return the error message corresponding to the error code in
\var{code}.
Availability: \UNIX, Windows.
\end{funcdesc}

\begin{funcdesc}{umask}{mask}
Set the current numeric umask and returns the previous umask.
Availability: \UNIX, Windows.
\end{funcdesc}

\begin{funcdesc}{uname}{}
Return a 5-tuple containing information identifying the current
operating system.  The tuple contains 5 strings:
\code{(\var{sysname}, \var{nodename}, \var{release}, \var{version},
\var{machine})}.  Some systems truncate the nodename to 8
characters or to the leading component; a better way to get the
hostname is \function{socket.gethostname()}
\withsubitem{(in module socket)}{\ttindex{gethostname()}}
or even
\withsubitem{(in module socket)}{\ttindex{gethostbyaddr()}}
\code{socket.gethostbyaddr(socket.gethostname())}.
Availability: recent flavors of \UNIX.
\end{funcdesc}



\subsection{File Object Creation \label{os-newstreams}}

These functions create new file objects.


\begin{funcdesc}{fdopen}{fd\optional{, mode\optional{, bufsize}}}
Return an open file object connected to the file descriptor \var{fd}.
\index{I/O control!buffering}
The \var{mode} and \var{bufsize} arguments have the same meaning as
the corresponding arguments to the built-in \function{open()}
function.
Availability: Macintosh, \UNIX, Windows.

\versionchanged[When specified, the \var{mode} argument must now start
  with one of the letters \character{r}, \character{w}, or \character{a},
  otherwise a \exception{ValueError} is raised]{2.3}
\end{funcdesc}

\begin{funcdesc}{popen}{command\optional{, mode\optional{, bufsize}}}
Open a pipe to or from \var{command}.  The return value is an open
file object connected to the pipe, which can be read or written
depending on whether \var{mode} is \code{'r'} (default) or \code{'w'}.
The \var{bufsize} argument has the same meaning as the corresponding
argument to the built-in \function{open()} function.  The exit status of
the command (encoded in the format specified for \function{wait()}) is
available as the return value of the \method{close()} method of the file
object, except that when the exit status is zero (termination without
errors), \code{None} is returned.
Availability: Macintosh, \UNIX, Windows.

\versionchanged[This function worked unreliably under Windows in
  earlier versions of Python.  This was due to the use of the
  \cfunction{_popen()} function from the libraries provided with
  Windows.  Newer versions of Python do not use the broken
  implementation from the Windows libraries]{2.0}
\end{funcdesc}

\begin{funcdesc}{tmpfile}{}
Return a new file object opened in update mode (\samp{w+b}).  The file
has no directory entries associated with it and will be automatically
deleted once there are no file descriptors for the file.
Availability: Macintosh, \UNIX, Windows.
\end{funcdesc}


For each of these \function{popen()} variants, if \var{bufsize} is
specified, it specifies the buffer size for the I/O pipes.
\var{mode}, if provided, should be the string \code{'b'} or
\code{'t'}; on Windows this is needed to determine whether the file
objects should be opened in binary or text mode.  The default value
for \var{mode} is \code{'t'}.

Also, for each of these variants, on \UNIX, \var{cmd} may be a sequence, in
which case arguments will be passed directly to the program without shell
intervention (as with \function{os.spawnv()}). If \var{cmd} is a string it will
be passed to the shell (as with \function{os.system()}).

These methods do not make it possible to retrieve the return code from
the child processes.  The only way to control the input and output
streams and also retrieve the return codes is to use the
\class{Popen3} and \class{Popen4} classes from the \refmodule{popen2}
module; these are only available on \UNIX.

For a discussion of possible deadlock conditions related to the use
of these functions, see ``\ulink{Flow Control
Issues}{popen2-flow-control.html}''
(section~\ref{popen2-flow-control}).

\begin{funcdesc}{popen2}{cmd\optional{, mode\optional{, bufsize}}}
Executes \var{cmd} as a sub-process.  Returns the file objects
\code{(\var{child_stdin}, \var{child_stdout})}.
Availability: Macintosh, \UNIX, Windows.
\versionadded{2.0}
\end{funcdesc}

\begin{funcdesc}{popen3}{cmd\optional{, mode\optional{, bufsize}}}
Executes \var{cmd} as a sub-process.  Returns the file objects
\code{(\var{child_stdin}, \var{child_stdout}, \var{child_stderr})}.
Availability: Macintosh, \UNIX, Windows.
\versionadded{2.0}
\end{funcdesc}

\begin{funcdesc}{popen4}{cmd\optional{, mode\optional{, bufsize}}}
Executes \var{cmd} as a sub-process.  Returns the file objects
\code{(\var{child_stdin}, \var{child_stdout_and_stderr})}.
Availability: Macintosh, \UNIX, Windows.
\versionadded{2.0}
\end{funcdesc}

(Note that \code{\var{child_stdin}, \var{child_stdout}, and
\var{child_stderr}} are named from the point of view of the child
process, i.e. \var{child_stdin} is the child's standard input.)

This functionality is also available in the \refmodule{popen2} module
using functions of the same names, but the return values of those
functions have a different order.


\subsection{File Descriptor Operations \label{os-fd-ops}}

These functions operate on I/O streams referred to
using file descriptors.


\begin{funcdesc}{close}{fd}
Close file descriptor \var{fd}.
Availability: Macintosh, \UNIX, Windows.

\begin{notice}
This function is intended for low-level I/O and must be applied
to a file descriptor as returned by \function{open()} or
\function{pipe()}.  To close a ``file object'' returned by the
built-in function \function{open()} or by \function{popen()} or
\function{fdopen()}, use its \method{close()} method.
\end{notice}
\end{funcdesc}

\begin{funcdesc}{dup}{fd}
Return a duplicate of file descriptor \var{fd}.
Availability: Macintosh, \UNIX, Windows.
\end{funcdesc}

\begin{funcdesc}{dup2}{fd, fd2}
Duplicate file descriptor \var{fd} to \var{fd2}, closing the latter
first if necessary.
Availability: Macintosh, \UNIX, Windows.
\end{funcdesc}

\begin{funcdesc}{fdatasync}{fd}
Force write of file with filedescriptor \var{fd} to disk.
Does not force update of metadata.
Availability: \UNIX.
\end{funcdesc}

\begin{funcdesc}{fpathconf}{fd, name}
Return system configuration information relevant to an open file.
\var{name} specifies the configuration value to retrieve; it may be a
string which is the name of a defined system value; these names are
specified in a number of standards (\POSIX.1, \UNIX{} 95, \UNIX{} 98, and
others).  Some platforms define additional names as well.  The names
known to the host operating system are given in the
\code{pathconf_names} dictionary.  For configuration variables not
included in that mapping, passing an integer for \var{name} is also
accepted.
Availability: Macintosh, \UNIX.

If \var{name} is a string and is not known, \exception{ValueError} is
raised.  If a specific value for \var{name} is not supported by the
host system, even if it is included in \code{pathconf_names}, an
\exception{OSError} is raised with \constant{errno.EINVAL} for the
error number.
\end{funcdesc}

\begin{funcdesc}{fstat}{fd}
Return status for file descriptor \var{fd}, like \function{stat()}.
Availability: Macintosh, \UNIX, Windows.
\end{funcdesc}

\begin{funcdesc}{fstatvfs}{fd}
Return information about the filesystem containing the file associated
with file descriptor \var{fd}, like \function{statvfs()}.
Availability: \UNIX.
\end{funcdesc}

\begin{funcdesc}{fsync}{fd}
Force write of file with filedescriptor \var{fd} to disk.  On \UNIX,
this calls the native \cfunction{fsync()} function; on Windows, the
MS \cfunction{_commit()} function.

If you're starting with a Python file object \var{f}, first do
\code{\var{f}.flush()}, and then do \code{os.fsync(\var{f}.fileno())},
to ensure that all internal buffers associated with \var{f} are written
to disk.
Availability: Macintosh, \UNIX, and Windows starting in 2.2.3.
\end{funcdesc}

\begin{funcdesc}{ftruncate}{fd, length}
Truncate the file corresponding to file descriptor \var{fd},
so that it is at most \var{length} bytes in size.
Availability: Macintosh, \UNIX.
\end{funcdesc}

\begin{funcdesc}{isatty}{fd}
Return \code{True} if the file descriptor \var{fd} is open and
connected to a tty(-like) device, else \code{False}.
Availability: Macintosh, \UNIX.
\end{funcdesc}

\begin{funcdesc}{lseek}{fd, pos, how}
Set the current position of file descriptor \var{fd} to position
\var{pos}, modified by \var{how}: \code{0} to set the position
relative to the beginning of the file; \code{1} to set it relative to
the current position; \code{2} to set it relative to the end of the
file.
Availability: Macintosh, \UNIX, Windows.
\end{funcdesc}

\begin{funcdesc}{open}{file, flags\optional{, mode}}
Open the file \var{file} and set various flags according to
\var{flags} and possibly its mode according to \var{mode}.
The default \var{mode} is \code{0777} (octal), and the current umask
value is first masked out.  Return the file descriptor for the newly
opened file.
Availability: Macintosh, \UNIX, Windows.

For a description of the flag and mode values, see the C run-time
documentation; flag constants (like \constant{O_RDONLY} and
\constant{O_WRONLY}) are defined in this module too (see below).

\begin{notice}
This function is intended for low-level I/O.  For normal usage,
use the built-in function \function{open()}, which returns a ``file
object'' with \method{read()} and \method{write()} methods (and many
more).
\end{notice}
\end{funcdesc}

\begin{funcdesc}{openpty}{}
Open a new pseudo-terminal pair. Return a pair of file descriptors
\code{(\var{master}, \var{slave})} for the pty and the tty,
respectively. For a (slightly) more portable approach, use the
\refmodule{pty}\refstmodindex{pty} module.
Availability: Macintosh, Some flavors of \UNIX.
\end{funcdesc}

\begin{funcdesc}{pipe}{}
Create a pipe.  Return a pair of file descriptors \code{(\var{r},
\var{w})} usable for reading and writing, respectively.
Availability: Macintosh, \UNIX, Windows.
\end{funcdesc}

\begin{funcdesc}{read}{fd, n}
Read at most \var{n} bytes from file descriptor \var{fd}.
Return a string containing the bytes read.  If the end of the file
referred to by \var{fd} has been reached, an empty string is
returned.
Availability: Macintosh, \UNIX, Windows.

\begin{notice}
This function is intended for low-level I/O and must be applied
to a file descriptor as returned by \function{open()} or
\function{pipe()}.  To read a ``file object'' returned by the
built-in function \function{open()} or by \function{popen()} or
\function{fdopen()}, or \code{sys.stdin}, use its
\method{read()} or \method{readline()} methods.
\end{notice}
\end{funcdesc}

\begin{funcdesc}{tcgetpgrp}{fd}
Return the process group associated with the terminal given by
\var{fd} (an open file descriptor as returned by \function{open()}).
Availability: Macintosh, \UNIX.
\end{funcdesc}

\begin{funcdesc}{tcsetpgrp}{fd, pg}
Set the process group associated with the terminal given by
\var{fd} (an open file descriptor as returned by \function{open()})
to \var{pg}.
Availability: Macintosh, \UNIX.
\end{funcdesc}

\begin{funcdesc}{ttyname}{fd}
Return a string which specifies the terminal device associated with
file-descriptor \var{fd}.  If \var{fd} is not associated with a terminal
device, an exception is raised.
Availability:Macintosh,  \UNIX.
\end{funcdesc}

\begin{funcdesc}{write}{fd, str}
Write the string \var{str} to file descriptor \var{fd}.
Return the number of bytes actually written.
Availability: Macintosh, \UNIX, Windows.

\begin{notice}
This function is intended for low-level I/O and must be applied
to a file descriptor as returned by \function{open()} or
\function{pipe()}.  To write a ``file object'' returned by the
built-in function \function{open()} or by \function{popen()} or
\function{fdopen()}, or \code{sys.stdout} or \code{sys.stderr}, use
its \method{write()} method.
\end{notice}
\end{funcdesc}


The following data items are available for use in constructing the
\var{flags} parameter to the \function{open()} function.

\begin{datadesc}{O_RDONLY}
\dataline{O_WRONLY}
\dataline{O_RDWR}
\dataline{O_APPEND}
\dataline{O_CREAT}
\dataline{O_EXCL}
\dataline{O_TRUNC}
Options for the \var{flag} argument to the \function{open()} function.
These can be bit-wise OR'd together.
Availability: Macintosh, \UNIX, Windows.
\end{datadesc}

\begin{datadesc}{O_DSYNC}
\dataline{O_RSYNC}
\dataline{O_SYNC}
\dataline{O_NDELAY}
\dataline{O_NONBLOCK}
\dataline{O_NOCTTY}
More options for the \var{flag} argument to the \function{open()} function.
Availability: Macintosh, \UNIX.
\end{datadesc}

\begin{datadesc}{O_BINARY}
Option for the \var{flag} argument to the \function{open()} function.
This can be bit-wise OR'd together with those listed above.
Availability: Windows.
% XXX need to check on the availability of this one.
\end{datadesc}

\begin{datadesc}{O_NOINHERIT}
\dataline{O_SHORT_LIVED}
\dataline{O_TEMPORARY}
\dataline{O_RANDOM}
\dataline{O_SEQUENTIAL}
\dataline{O_TEXT}
Options for the \var{flag} argument to the \function{open()} function.
These can be bit-wise OR'd together.
Availability: Windows.
\end{datadesc}

\subsection{Files and Directories \label{os-file-dir}}

\begin{funcdesc}{access}{path, mode}
Use the real uid/gid to test for access to \var{path}.  Note that most
operations will use the effective uid/gid, therefore this routine can
be used in a suid/sgid environment to test if the invoking user has the
specified access to \var{path}.  \var{mode} should be \constant{F_OK}
to test the existence of \var{path}, or it can be the inclusive OR of
one or more of \constant{R_OK}, \constant{W_OK}, and \constant{X_OK} to
test permissions.  Return \constant{True} if access is allowed,
\constant{False} if not.
See the \UNIX{} man page \manpage{access}{2} for more information.
Availability: Macintosh, \UNIX, Windows.
\end{funcdesc}

\begin{datadesc}{F_OK}
  Value to pass as the \var{mode} parameter of \function{access()} to
  test the existence of \var{path}.
\end{datadesc}

\begin{datadesc}{R_OK}
  Value to include in the \var{mode} parameter of \function{access()}
  to test the readability of \var{path}.
\end{datadesc}

\begin{datadesc}{W_OK}
  Value to include in the \var{mode} parameter of \function{access()}
  to test the writability of \var{path}.
\end{datadesc}

\begin{datadesc}{X_OK}
  Value to include in the \var{mode} parameter of \function{access()}
  to determine if \var{path} can be executed.
\end{datadesc}

\begin{funcdesc}{chdir}{path}
\index{directory!changing}
Change the current working directory to \var{path}.
Availability: Macintosh, \UNIX, Windows.
\end{funcdesc}

\begin{funcdesc}{fchdir}{fd}
Change the current working directory to the directory represented by
the file descriptor \var{fd}.  The descriptor must refer to an opened
directory, not an open file.
Availability: \UNIX.
\versionadded{2.3}
\end{funcdesc}

\begin{funcdesc}{getcwd}{}
Return a string representing the current working directory.
Availability: Macintosh, \UNIX, Windows.
\end{funcdesc}

\begin{funcdesc}{getcwdu}{}
Return a Unicode object representing the current working directory.
Availability: Macintosh, \UNIX, Windows.
\versionadded{2.3}
\end{funcdesc}

\begin{funcdesc}{chroot}{path}
Change the root directory of the current process to \var{path}.
Availability: Macintosh, \UNIX.
\versionadded{2.2}
\end{funcdesc}

\begin{funcdesc}{chmod}{path, mode}
Change the mode of \var{path} to the numeric \var{mode}.
\var{mode} may take one of the following values
(as defined in the \module{stat} module):
\begin{itemize}
  \item \code{S_ISUID}
  \item \code{S_ISGID}
  \item \code{S_ENFMT}
  \item \code{S_ISVTX}
  \item \code{S_IREAD}
  \item \code{S_IWRITE}
  \item \code{S_IEXEC}
  \item \code{S_IRWXU}
  \item \code{S_IRUSR}
  \item \code{S_IWUSR}
  \item \code{S_IXUSR}
  \item \code{S_IRWXG}
  \item \code{S_IRGRP}
  \item \code{S_IWGRP}
  \item \code{S_IXGRP}
  \item \code{S_IRWXO}
  \item \code{S_IROTH}
  \item \code{S_IWOTH}
  \item \code{S_IXOTH}
\end{itemize}
Availability: Macintosh, \UNIX, Windows.
\end{funcdesc}

\begin{funcdesc}{chown}{path, uid, gid}
Change the owner and group id of \var{path} to the numeric \var{uid}
and \var{gid}.
Availability: Macintosh, \UNIX.
\end{funcdesc}

\begin{funcdesc}{lchown}{path, uid, gid}
Change the owner and group id of \var{path} to the numeric \var{uid}
and gid. This function will not follow symbolic links.
Availability: Macintosh, \UNIX.
\versionadded{2.3}
\end{funcdesc}

\begin{funcdesc}{link}{src, dst}
Create a hard link pointing to \var{src} named \var{dst}.
Availability: Macintosh, \UNIX.
\end{funcdesc}

\begin{funcdesc}{listdir}{path}
Return a list containing the names of the entries in the directory.
The list is in arbitrary order.  It does not include the special
entries \code{'.'} and \code{'..'} even if they are present in the
directory.
Availability: Macintosh, \UNIX, Windows.

\versionchanged[On Windows NT/2k/XP and Unix, if \var{path} is a Unicode
object, the result will be a list of Unicode objects.]{2.3}
\end{funcdesc}

\begin{funcdesc}{lstat}{path}
Like \function{stat()}, but do not follow symbolic links.
Availability: Macintosh, \UNIX.
\end{funcdesc}

\begin{funcdesc}{mkfifo}{path\optional{, mode}}
Create a FIFO (a named pipe) named \var{path} with numeric mode
\var{mode}.  The default \var{mode} is \code{0666} (octal).  The current
umask value is first masked out from the mode.
Availability: Macintosh, \UNIX.

FIFOs are pipes that can be accessed like regular files.  FIFOs exist
until they are deleted (for example with \function{os.unlink()}).
Generally, FIFOs are used as rendezvous between ``client'' and
``server'' type processes: the server opens the FIFO for reading, and
the client opens it for writing.  Note that \function{mkfifo()}
doesn't open the FIFO --- it just creates the rendezvous point.
\end{funcdesc}

\begin{funcdesc}{mknod}{path\optional{, mode=0600, device}}
Create a filesystem node (file, device special file or named pipe)
named filename. \var{mode} specifies both the permissions to use and
the type of node to be created, being combined (bitwise OR) with one
of S_IFREG, S_IFCHR, S_IFBLK, and S_IFIFO (those constants are
available in \module{stat}). For S_IFCHR and S_IFBLK, \var{device}
defines the newly created device special file (probably using
\function{os.makedev()}), otherwise it is ignored.
\versionadded{2.3}
\end{funcdesc}

\begin{funcdesc}{major}{device}
Extracts a device major number from a raw device number.
\versionadded{2.3}
\end{funcdesc}

\begin{funcdesc}{minor}{device}
Extracts a device minor number from a raw device number.
\versionadded{2.3}
\end{funcdesc}

\begin{funcdesc}{makedev}{major, minor}
Composes a raw device number from the major and minor device numbers.
\versionadded{2.3}
\end{funcdesc}

\begin{funcdesc}{mkdir}{path\optional{, mode}}
Create a directory named \var{path} with numeric mode \var{mode}.
The default \var{mode} is \code{0777} (octal).  On some systems,
\var{mode} is ignored.  Where it is used, the current umask value is
first masked out.
Availability: Macintosh, \UNIX, Windows.
\end{funcdesc}

\begin{funcdesc}{makedirs}{path\optional{, mode}}
Recursive directory creation function.\index{directory!creating}
\index{UNC paths!and \function{os.makedirs()}}
Like \function{mkdir()},
but makes all intermediate-level directories needed to contain the
leaf directory.  Throws an \exception{error} exception if the leaf
directory already exists or cannot be created.  The default \var{mode}
is \code{0777} (octal).  This function does not properly handle UNC
paths (only relevant on Windows systems; Universal Naming Convention
paths are those that use the `\code{\e\e host\e path}' syntax).
\versionadded{1.5.2}
\end{funcdesc}

\begin{funcdesc}{pathconf}{path, name}
Return system configuration information relevant to a named file.
\var{name} specifies the configuration value to retrieve; it may be a
string which is the name of a defined system value; these names are
specified in a number of standards (\POSIX.1, \UNIX{} 95, \UNIX{} 98, and
others).  Some platforms define additional names as well.  The names
known to the host operating system are given in the
\code{pathconf_names} dictionary.  For configuration variables not
included in that mapping, passing an integer for \var{name} is also
accepted.
Availability: Macintosh, \UNIX.

If \var{name} is a string and is not known, \exception{ValueError} is
raised.  If a specific value for \var{name} is not supported by the
host system, even if it is included in \code{pathconf_names}, an
\exception{OSError} is raised with \constant{errno.EINVAL} for the
error number.
\end{funcdesc}

\begin{datadesc}{pathconf_names}
Dictionary mapping names accepted by \function{pathconf()} and
\function{fpathconf()} to the integer values defined for those names
by the host operating system.  This can be used to determine the set
of names known to the system.
Availability: Macintosh, \UNIX.
\end{datadesc}

\begin{funcdesc}{readlink}{path}
Return a string representing the path to which the symbolic link
points.  The result may be either an absolute or relative pathname; if
it is relative, it may be converted to an absolute pathname using
\code{os.path.join(os.path.dirname(\var{path}), \var{result})}.
Availability: Macintosh, \UNIX.
\end{funcdesc}

\begin{funcdesc}{remove}{path}
Remove the file \var{path}.  If \var{path} is a directory,
\exception{OSError} is raised; see \function{rmdir()} below to remove
a directory.  This is identical to the \function{unlink()} function
documented below.  On Windows, attempting to remove a file that is in
use causes an exception to be raised; on \UNIX, the directory entry is
removed but the storage allocated to the file is not made available
until the original file is no longer in use.
Availability: Macintosh, \UNIX, Windows.
\end{funcdesc}

\begin{funcdesc}{removedirs}{path}
\index{directory!deleting}
Removes directories recursively.  Works like
\function{rmdir()} except that, if the leaf directory is
successfully removed, directories corresponding to rightmost path
segments will be pruned way until either the whole path is consumed or
an error is raised (which is ignored, because it generally means that
a parent directory is not empty).  Throws an \exception{error}
exception if the leaf directory could not be successfully removed.
\versionadded{1.5.2}
\end{funcdesc}

\begin{funcdesc}{rename}{src, dst}
Rename the file or directory \var{src} to \var{dst}.  If \var{dst} is
a directory, \exception{OSError} will be raised.  On \UNIX, if
\var{dst} exists and is a file, it will be removed silently if the
user has permission.  The operation may fail on some \UNIX{} flavors
if \var{src} and \var{dst} are on different filesystems.  If
successful, the renaming will be an atomic operation (this is a
\POSIX{} requirement).  On Windows, if \var{dst} already exists,
\exception{OSError} will be raised even if it is a file; there may be
no way to implement an atomic rename when \var{dst} names an existing
file.
Availability: Macintosh, \UNIX, Windows.
\end{funcdesc}

\begin{funcdesc}{renames}{old, new}
Recursive directory or file renaming function.
Works like \function{rename()}, except creation of any intermediate
directories needed to make the new pathname good is attempted first.
After the rename, directories corresponding to rightmost path segments
of the old name will be pruned away using \function{removedirs()}.
\versionadded{1.5.2}

\begin{notice}
This function can fail with the new directory structure made if
you lack permissions needed to remove the leaf directory or file.
\end{notice}
\end{funcdesc}

\begin{funcdesc}{rmdir}{path}
Remove the directory \var{path}.
Availability: Macintosh, \UNIX, Windows.
\end{funcdesc}

\begin{funcdesc}{stat}{path}
Perform a \cfunction{stat()} system call on the given path.  The
return value is an object whose attributes correspond to the members of
the \ctype{stat} structure, namely:
\member{st_mode} (protection bits),
\member{st_ino} (inode number),
\member{st_dev} (device),
\member{st_nlink} (number of hard links),
\member{st_uid} (user ID of owner),
\member{st_gid} (group ID of owner),
\member{st_size} (size of file, in bytes),
\member{st_atime} (time of most recent access),
\member{st_mtime} (time of most recent content modification),
\member{st_ctime}
(platform dependent; time of most recent metadata change on \UNIX, or
the time of creation on Windows).

\versionchanged [If \function{stat_float_times} returns true, the time
values are floats, measuring seconds. Fractions of a second may be
reported if the system supports that. On Mac OS, the times are always
floats. See \function{stat_float_times} for further discussion. ]{2.3}

On some Unix systems (such as Linux), the following attributes may
also be available:
\member{st_blocks} (number of blocks allocated for file),
\member{st_blksize} (filesystem blocksize),
\member{st_rdev} (type of device if an inode device).

On Mac OS systems, the following attributes may also be available:
\member{st_rsize},
\member{st_creator},
\member{st_type}.

On RISCOS systems, the following attributes are also available:
\member{st_ftype} (file type),
\member{st_attrs} (attributes),
\member{st_obtype} (object type).

For backward compatibility, the return value of \function{stat()} is
also accessible as a tuple of at least 10 integers giving the most
important (and portable) members of the \ctype{stat} structure, in the
order
\member{st_mode},
\member{st_ino},
\member{st_dev},
\member{st_nlink},
\member{st_uid},
\member{st_gid},
\member{st_size},
\member{st_atime},
\member{st_mtime},
\member{st_ctime}.
More items may be added at the end by some implementations.
The standard module \refmodule{stat}\refstmodindex{stat} defines
functions and constants that are useful for extracting information
from a \ctype{stat} structure.
(On Windows, some items are filled with dummy values.)

\note{The exact meaning and resolution of the \member{st_atime},
 \member{st_mtime}, and \member{st_ctime} members depends on the
 operating system and the file system.  For example, on Windows systems
 using the FAT or FAT32 file systems, \member{st_mtime} has 2-second
 resolution, and \member{st_atime} has only 1-day resolution.  See
 your operating system documentation for details.}

Availability: Macintosh, \UNIX, Windows.

\versionchanged
[Added access to values as attributes of the returned object]{2.2}
\end{funcdesc}

\begin{funcdesc}{stat_float_times}{\optional{newvalue}}
Determine whether \class{stat_result} represents time stamps as float
objects.  If newval is True, future calls to stat() return floats, if
it is False, future calls return ints.  If newval is omitted, return
the current setting.

For compatibility with older Python versions, accessing
\class{stat_result} as a tuple always returns integers. For
compatibility with Python 2.2, accessing the time stamps by field name
also returns integers. Applications that want to determine the
fractions of a second in a time stamp can use this function to have
time stamps represented as floats. Whether they will actually observe
non-zero fractions depends on the system.

Future Python releases will change the default of this setting;
applications that cannot deal with floating point time stamps can then
use this function to turn the feature off.

It is recommended that this setting is only changed at program startup
time in the \var{__main__} module; libraries should never change this
setting. If an application uses a library that works incorrectly if
floating point time stamps are processed, this application should turn
the feature off until the library has been corrected.

\end{funcdesc}

\begin{funcdesc}{statvfs}{path}
Perform a \cfunction{statvfs()} system call on the given path.  The
return value is an object whose attributes describe the filesystem on
the given path, and correspond to the members of the
\ctype{statvfs} structure, namely:
\member{f_frsize},
\member{f_blocks},
\member{f_bfree},
\member{f_bavail},
\member{f_files},
\member{f_ffree},
\member{f_favail},
\member{f_flag},
\member{f_namemax}.
Availability: \UNIX.

For backward compatibility, the return value is also accessible as a
tuple whose values correspond to the attributes, in the order given above.
The standard module \refmodule{statvfs}\refstmodindex{statvfs}
defines constants that are useful for extracting information
from a \ctype{statvfs} structure when accessing it as a sequence; this
remains useful when writing code that needs to work with versions of
Python that don't support accessing the fields as attributes.

\versionchanged
[Added access to values as attributes of the returned object]{2.2}
\end{funcdesc}

\begin{funcdesc}{symlink}{src, dst}
Create a symbolic link pointing to \var{src} named \var{dst}.
Availability: \UNIX.
\end{funcdesc}

\begin{funcdesc}{tempnam}{\optional{dir\optional{, prefix}}}
Return a unique path name that is reasonable for creating a temporary
file.  This will be an absolute path that names a potential directory
entry in the directory \var{dir} or a common location for temporary
files if \var{dir} is omitted or \code{None}.  If given and not
\code{None}, \var{prefix} is used to provide a short prefix to the
filename.  Applications are responsible for properly creating and
managing files created using paths returned by \function{tempnam()};
no automatic cleanup is provided.
On \UNIX, the environment variable \envvar{TMPDIR} overrides
\var{dir}, while on Windows the \envvar{TMP} is used.  The specific
behavior of this function depends on the C library implementation;
some aspects are underspecified in system documentation.
\warning{Use of \function{tempnam()} is vulnerable to symlink attacks;
consider using \function{tmpfile()} instead.}
Availability: Macintosh, \UNIX, Windows.
\end{funcdesc}

\begin{funcdesc}{tmpnam}{}
Return a unique path name that is reasonable for creating a temporary
file.  This will be an absolute path that names a potential directory
entry in a common location for temporary files.  Applications are
responsible for properly creating and managing files created using
paths returned by \function{tmpnam()}; no automatic cleanup is
provided.
\warning{Use of \function{tmpnam()} is vulnerable to symlink attacks;
consider using \function{tmpfile()} instead.}
Availability: \UNIX, Windows.  This function probably shouldn't be used
on Windows, though:  Microsoft's implementation of \function{tmpnam()}
always creates a name in the root directory of the current drive, and
that's generally a poor location for a temp file (depending on
privileges, you may not even be able to open a file using this name).
\end{funcdesc}

\begin{datadesc}{TMP_MAX}
The maximum number of unique names that \function{tmpnam()} will
generate before reusing names.
\end{datadesc}

\begin{funcdesc}{unlink}{path}
Remove the file \var{path}.  This is the same function as
\function{remove()}; the \function{unlink()} name is its traditional
\UNIX{} name.
Availability: Macintosh, \UNIX, Windows.
\end{funcdesc}

\begin{funcdesc}{utime}{path, times}
Set the access and modified times of the file specified by \var{path}.
If \var{times} is \code{None}, then the file's access and modified
times are set to the current time.  Otherwise, \var{times} must be a
2-tuple of numbers, of the form \code{(\var{atime}, \var{mtime})}
which is used to set the access and modified times, respectively.
Whether a directory can be given for \var{path} depends on whether the
operating system implements directories as files (for example, Windows
does not).  Note that the exact times you set here may not be returned
by a subsequent \function{stat()} call, depending on the resolution
with which your operating system records access and modification times;
see \function{stat()}.
\versionchanged[Added support for \code{None} for \var{times}]{2.0}
Availability: Macintosh, \UNIX, Windows.
\end{funcdesc}

\begin{funcdesc}{walk}{top\optional{, topdown\code{=True}
                       \optional{, onerror\code{=None}}}}
\index{directory!walking}
\index{directory!traversal}
\function{walk()} generates the file names in a directory tree, by
walking the tree either top down or bottom up.
For each directory in the tree rooted at directory \var{top} (including
\var{top} itself), it yields a 3-tuple
\code{(\var{dirpath}, \var{dirnames}, \var{filenames})}.

\var{dirpath} is a string, the path to the directory.  \var{dirnames} is
a list of the names of the subdirectories in \var{dirpath}
(excluding \code{'.'} and \code{'..'}).  \var{filenames} is a list of
the names of the non-directory files in \var{dirpath}.  Note that the
names in the lists contain no path components.  To get a full
path (which begins with \var{top}) to a file or directory in
\var{dirpath}, do \code{os.path.join(\var{dirpath}, \var{name})}.

If optional argument \var{topdown} is true or not specified, the triple
for a directory is generated before the triples for any of its
subdirectories (directories are generated top down).  If \var{topdown} is
false, the triple for a directory is generated after the triples for all
of its subdirectories (directories are generated bottom up).

When \var{topdown} is true, the caller can modify the \var{dirnames} list
in-place (perhaps using \keyword{del} or slice assignment), and
\function{walk()} will only recurse into the subdirectories whose names
remain in \var{dirnames}; this can be used to prune the search,
impose a specific order of visiting, or even to inform \function{walk()}
about directories the caller creates or renames before it resumes
\function{walk()} again.  Modifying \var{dirnames} when \var{topdown} is
false is ineffective, because in bottom-up mode the directories in
\var{dirnames} are generated before \var{dirnames} itself is generated.

By default errors from the \code{os.listdir()} call are ignored.  If
optional argument \var{onerror} is specified, it should be a function;
it will be called with one argument, an os.error instance.  It can
report the error to continue with the walk, or raise the exception
to abort the walk.  Note that the filename is available as the
\code{filename} attribute of the exception object.

\begin{notice}
If you pass a relative pathname, don't change the current working
directory between resumptions of \function{walk()}.  \function{walk()}
never changes the current directory, and assumes that its caller
doesn't either.
\end{notice}

\begin{notice}
On systems that support symbolic links, links to subdirectories appear
in \var{dirnames} lists, but \function{walk()} will not visit them
(infinite loops are hard to avoid when following symbolic links).
To visit linked directories, you can identify them with
\code{os.path.islink(\var{path})}, and invoke \code{walk(\var{path})}
on each directly.
\end{notice}

This example displays the number of bytes taken by non-directory files
in each directory under the starting directory, except that it doesn't
look under any CVS subdirectory:

\begin{verbatim}
import os
from os.path import join, getsize
for root, dirs, files in os.walk('python/Lib/email'):
    print root, "consumes",
    print sum(getsize(join(root, name)) for name in files),
    print "bytes in", len(files), "non-directory files"
    if 'CVS' in dirs:
        dirs.remove('CVS')  # don't visit CVS directories
\end{verbatim}

In the next example, walking the tree bottom up is essential:
\function{rmdir()} doesn't allow deleting a directory before the
directory is empty:

\begin{verbatim}
# Delete everything reachable from the directory named in 'top',
# assuming there are no symbolic links.
# CAUTION:  This is dangerous!  For example, if top == '/', it
# could delete all your disk files.
import os
for root, dirs, files in os.walk(top, topdown=False):
    for name in files:
        os.remove(os.path.join(root, name))
    for name in dirs:
        os.rmdir(os.path.join(root, name))
\end{verbatim}

\versionadded{2.3}
\end{funcdesc}

\subsection{Process Management \label{os-process}}

These functions may be used to create and manage processes.

The various \function{exec*()} functions take a list of arguments for
the new program loaded into the process.  In each case, the first of
these arguments is passed to the new program as its own name rather
than as an argument a user may have typed on a command line.  For the
C programmer, this is the \code{argv[0]} passed to a program's
\cfunction{main()}.  For example, \samp{os.execv('/bin/echo', ['foo',
'bar'])} will only print \samp{bar} on standard output; \samp{foo}
will seem to be ignored.


\begin{funcdesc}{abort}{}
Generate a \constant{SIGABRT} signal to the current process.  On
\UNIX, the default behavior is to produce a core dump; on Windows, the
process immediately returns an exit code of \code{3}.  Be aware that
programs which use \function{signal.signal()} to register a handler
for \constant{SIGABRT} will behave differently.
Availability: Macintosh, \UNIX, Windows.
\end{funcdesc}

\begin{funcdesc}{execl}{path, arg0, arg1, \moreargs}
\funcline{execle}{path, arg0, arg1, \moreargs, env}
\funcline{execlp}{file, arg0, arg1, \moreargs}
\funcline{execlpe}{file, arg0, arg1, \moreargs, env}
\funcline{execv}{path, args}
\funcline{execve}{path, args, env}
\funcline{execvp}{file, args}
\funcline{execvpe}{file, args, env}
These functions all execute a new program, replacing the current
process; they do not return.  On \UNIX, the new executable is loaded
into the current process, and will have the same process ID as the
caller.  Errors will be reported as \exception{OSError} exceptions.

The \character{l} and \character{v} variants of the
\function{exec*()} functions differ in how command-line arguments are
passed.  The \character{l} variants are perhaps the easiest to work
with if the number of parameters is fixed when the code is written;
the individual parameters simply become additional parameters to the
\function{execl*()} functions.  The \character{v} variants are good
when the number of parameters is variable, with the arguments being
passed in a list or tuple as the \var{args} parameter.  In either
case, the arguments to the child process should start with the name of
the command being run, but this is not enforced.

The variants which include a \character{p} near the end
(\function{execlp()}, \function{execlpe()}, \function{execvp()},
and \function{execvpe()}) will use the \envvar{PATH} environment
variable to locate the program \var{file}.  When the environment is
being replaced (using one of the \function{exec*e()} variants,
discussed in the next paragraph), the
new environment is used as the source of the \envvar{PATH} variable.
The other variants, \function{execl()}, \function{execle()},
\function{execv()}, and \function{execve()}, will not use the
\envvar{PATH} variable to locate the executable; \var{path} must
contain an appropriate absolute or relative path.

For \function{execle()}, \function{execlpe()}, \function{execve()},
and \function{execvpe()} (note that these all end in \character{e}),
the \var{env} parameter must be a mapping which is used to define the
environment variables for the new process; the \function{execl()},
\function{execlp()}, \function{execv()}, and \function{execvp()}
all cause the new process to inherit the environment of the current
process.
Availability: Macintosh, \UNIX, Windows.
\end{funcdesc}

\begin{funcdesc}{_exit}{n}
Exit to the system with status \var{n}, without calling cleanup
handlers, flushing stdio buffers, etc.
Availability: Macintosh, \UNIX, Windows.

\begin{notice}
The standard way to exit is \code{sys.exit(\var{n})}.
\function{_exit()} should normally only be used in the child process
after a \function{fork()}.
\end{notice}
\end{funcdesc}

The following exit codes are a defined, and can be used with
\function{_exit()}, although they are not required.  These are
typically used for system programs written in Python, such as a
mail server's external command delivery program.

\begin{datadesc}{EX_OK}
Exit code that means no error occurred.
Availability: Macintosh, \UNIX.
\versionadded{2.3}
\end{datadesc}

\begin{datadesc}{EX_USAGE}
Exit code that means the command was used incorrectly, such as when
the wrong number of arguments are given.
Availability: Macintosh, \UNIX.
\versionadded{2.3}
\end{datadesc}

\begin{datadesc}{EX_DATAERR}
Exit code that means the input data was incorrect.
Availability: Macintosh, \UNIX.
\versionadded{2.3}
\end{datadesc}

\begin{datadesc}{EX_NOINPUT}
Exit code that means an input file did not exist or was not readable.
Availability: Macintosh, \UNIX.
\versionadded{2.3}
\end{datadesc}

\begin{datadesc}{EX_NOUSER}
Exit code that means a specified user did not exist.
Availability: Macintosh, \UNIX.
\versionadded{2.3}
\end{datadesc}

\begin{datadesc}{EX_NOHOST}
Exit code that means a specified host did not exist.
Availability: Macintosh, \UNIX.
\versionadded{2.3}
\end{datadesc}

\begin{datadesc}{EX_UNAVAILABLE}
Exit code that means that a required service is unavailable.
Availability: Macintosh, \UNIX.
\versionadded{2.3}
\end{datadesc}

\begin{datadesc}{EX_SOFTWARE}
Exit code that means an internal software error was detected.
Availability: Macintosh, \UNIX.
\versionadded{2.3}
\end{datadesc}

\begin{datadesc}{EX_OSERR}
Exit code that means an operating system error was detected, such as
the inability to fork or create a pipe.
Availability: Macintosh, \UNIX.
\versionadded{2.3}
\end{datadesc}

\begin{datadesc}{EX_OSFILE}
Exit code that means some system file did not exist, could not be
opened, or had some other kind of error.
Availability: Macintosh, \UNIX.
\versionadded{2.3}
\end{datadesc}

\begin{datadesc}{EX_CANTCREAT}
Exit code that means a user specified output file could not be created.
Availability: Macintosh, \UNIX.
\versionadded{2.3}
\end{datadesc}

\begin{datadesc}{EX_IOERR}
Exit code that means that an error occurred while doing I/O on some file.
Availability: Macintosh, \UNIX.
\versionadded{2.3}
\end{datadesc}

\begin{datadesc}{EX_TEMPFAIL}
Exit code that means a temporary failure occurred.  This indicates
something that may not really be an error, such as a network
connection that couldn't be made during a retryable operation.
Availability: Macintosh, \UNIX.
\versionadded{2.3}
\end{datadesc}

\begin{datadesc}{EX_PROTOCOL}
Exit code that means that a protocol exchange was illegal, invalid, or
not understood.
Availability: Macintosh, \UNIX.
\versionadded{2.3}
\end{datadesc}

\begin{datadesc}{EX_NOPERM}
Exit code that means that there were insufficient permissions to
perform the operation (but not intended for file system problems).
Availability: Macintosh, \UNIX.
\versionadded{2.3}
\end{datadesc}

\begin{datadesc}{EX_CONFIG}
Exit code that means that some kind of configuration error occurred.
Availability: Macintosh, \UNIX.
\versionadded{2.3}
\end{datadesc}

\begin{datadesc}{EX_NOTFOUND}
Exit code that means something like ``an entry was not found''.
Availability: Macintosh, \UNIX.
\versionadded{2.3}
\end{datadesc}

\begin{funcdesc}{fork}{}
Fork a child process.  Return \code{0} in the child, the child's
process id in the parent.
Availability: Macintosh, \UNIX.
\end{funcdesc}

\begin{funcdesc}{forkpty}{}
Fork a child process, using a new pseudo-terminal as the child's
controlling terminal. Return a pair of \code{(\var{pid}, \var{fd})},
where \var{pid} is \code{0} in the child, the new child's process id
in the parent, and \var{fd} is the file descriptor of the master end
of the pseudo-terminal.  For a more portable approach, use the
\refmodule{pty} module.
Availability: Macintosh, Some flavors of \UNIX.
\end{funcdesc}

\begin{funcdesc}{kill}{pid, sig}
\index{process!killing}
\index{process!signalling}
Kill the process \var{pid} with signal \var{sig}.  Constants for the
specific signals available on the host platform are defined in the
\refmodule{signal} module.
Availability: Macintosh, \UNIX.
\end{funcdesc}

\begin{funcdesc}{killpg}{pgid, sig}
\index{process!killing}
\index{process!signalling}
Kill the process group \var{pgid} with the signal \var{sig}.
Availability: Macintosh, \UNIX.
\versionadded{2.3}
\end{funcdesc}

\begin{funcdesc}{nice}{increment}
Add \var{increment} to the process's ``niceness''.  Return the new
niceness.
Availability: Macintosh, \UNIX.
\end{funcdesc}

\begin{funcdesc}{plock}{op}
Lock program segments into memory.  The value of \var{op}
(defined in \code{<sys/lock.h>}) determines which segments are locked.
Availability: Macintosh, \UNIX.
\end{funcdesc}

\begin{funcdescni}{popen}{\unspecified}
\funclineni{popen2}{\unspecified}
\funclineni{popen3}{\unspecified}
\funclineni{popen4}{\unspecified}
Run child processes, returning opened pipes for communications.  These
functions are described in section \ref{os-newstreams}.
\end{funcdescni}

\begin{funcdesc}{spawnl}{mode, path, \moreargs}
\funcline{spawnle}{mode, path, \moreargs, env}
\funcline{spawnlp}{mode, file, \moreargs}
\funcline{spawnlpe}{mode, file, \moreargs, env}
\funcline{spawnv}{mode, path, args}
\funcline{spawnve}{mode, path, args, env}
\funcline{spawnvp}{mode, file, args}
\funcline{spawnvpe}{mode, file, args, env}
Execute the program \var{path} in a new process.  If \var{mode} is
\constant{P_NOWAIT}, this function returns the process ID of the new
process; if \var{mode} is \constant{P_WAIT}, returns the process's
exit code if it exits normally, or \code{-\var{signal}}, where
\var{signal} is the signal that killed the process.  On Windows, the
process ID will actually be the process handle, so can be used with
the \function{waitpid()} function.

The \character{l} and \character{v} variants of the
\function{spawn*()} functions differ in how command-line arguments are
passed.  The \character{l} variants are perhaps the easiest to work
with if the number of parameters is fixed when the code is written;
the individual parameters simply become additional parameters to the
\function{spawnl*()} functions.  The \character{v} variants are good
when the number of parameters is variable, with the arguments being
passed in a list or tuple as the \var{args} parameter.  In either
case, the arguments to the child process must start with the name of
the command being run.

The variants which include a second \character{p} near the end
(\function{spawnlp()}, \function{spawnlpe()}, \function{spawnvp()},
and \function{spawnvpe()}) will use the \envvar{PATH} environment
variable to locate the program \var{file}.  When the environment is
being replaced (using one of the \function{spawn*e()} variants,
discussed in the next paragraph), the new environment is used as the
source of the \envvar{PATH} variable.  The other variants,
\function{spawnl()}, \function{spawnle()}, \function{spawnv()}, and
\function{spawnve()}, will not use the \envvar{PATH} variable to
locate the executable; \var{path} must contain an appropriate absolute
or relative path.

For \function{spawnle()}, \function{spawnlpe()}, \function{spawnve()},
and \function{spawnvpe()} (note that these all end in \character{e}),
the \var{env} parameter must be a mapping which is used to define the
environment variables for the new process; the \function{spawnl()},
\function{spawnlp()}, \function{spawnv()}, and \function{spawnvp()}
all cause the new process to inherit the environment of the current
process.

As an example, the following calls to \function{spawnlp()} and
\function{spawnvpe()} are equivalent:

\begin{verbatim}
import os
os.spawnlp(os.P_WAIT, 'cp', 'cp', 'index.html', '/dev/null')

L = ['cp', 'index.html', '/dev/null']
os.spawnvpe(os.P_WAIT, 'cp', L, os.environ)
\end{verbatim}

Availability: \UNIX, Windows.  \function{spawnlp()},
\function{spawnlpe()}, \function{spawnvp()} and \function{spawnvpe()}
are not available on Windows.
\versionadded{1.6}
\end{funcdesc}

\begin{datadesc}{P_NOWAIT}
\dataline{P_NOWAITO}
Possible values for the \var{mode} parameter to the \function{spawn*()}
family of functions.  If either of these values is given, the
\function{spawn*()} functions will return as soon as the new process
has been created, with the process ID as the return value.
Availability: Macintosh, \UNIX, Windows.
\versionadded{1.6}
\end{datadesc}

\begin{datadesc}{P_WAIT}
Possible value for the \var{mode} parameter to the \function{spawn*()}
family of functions.  If this is given as \var{mode}, the
\function{spawn*()} functions will not return until the new process
has run to completion and will return the exit code of the process the
run is successful, or \code{-\var{signal}} if a signal kills the
process.
Availability: Macintosh, \UNIX, Windows.
\versionadded{1.6}
\end{datadesc}

\begin{datadesc}{P_DETACH}
\dataline{P_OVERLAY}
Possible values for the \var{mode} parameter to the
\function{spawn*()} family of functions.  These are less portable than
those listed above.
\constant{P_DETACH} is similar to \constant{P_NOWAIT}, but the new
process is detached from the console of the calling process.
If \constant{P_OVERLAY} is used, the current process will be replaced;
the \function{spawn*()} function will not return.
Availability: Windows.
\versionadded{1.6}
\end{datadesc}

\begin{funcdesc}{startfile}{path}
Start a file with its associated application.  This acts like
double-clicking the file in Windows Explorer, or giving the file name
as an argument to the \program{start} command from the interactive
command shell: the file is opened with whatever application (if any)
its extension is associated.

\function{startfile()} returns as soon as the associated application
is launched.  There is no option to wait for the application to close,
and no way to retrieve the application's exit status.  The \var{path}
parameter is relative to the current directory.  If you want to use an
absolute path, make sure the first character is not a slash
(\character{/}); the underlying Win32 \cfunction{ShellExecute()}
function doesn't work if it is.  Use the \function{os.path.normpath()}
function to ensure that the path is properly encoded for Win32.
Availability: Windows.
\versionadded{2.0}
\end{funcdesc}

\begin{funcdesc}{system}{command}
Execute the command (a string) in a subshell.  This is implemented by
calling the Standard C function \cfunction{system()}, and has the
same limitations.  Changes to \code{posix.environ}, \code{sys.stdin},
etc.\ are not reflected in the environment of the executed command.

On \UNIX, the return value is the exit status of the process encoded in the
format specified for \function{wait()}.  Note that \POSIX{} does not
specify the meaning of the return value of the C \cfunction{system()}
function, so the return value of the Python function is system-dependent.

On Windows, the return value is that returned by the system shell after
running \var{command}, given by the Windows environment variable
\envvar{COMSPEC}: on \program{command.com} systems (Windows 95, 98 and ME)
this is always \code{0}; on \program{cmd.exe} systems (Windows NT, 2000
and XP) this is the exit status of the command run; on systems using
a non-native shell, consult your shell documentation.

Availability: Macintosh, \UNIX, Windows.
\end{funcdesc}

\begin{funcdesc}{times}{}
Return a 5-tuple of floating point numbers indicating accumulated
(processor or other)
times, in seconds.  The items are: user time, system time, children's
user time, children's system time, and elapsed real time since a fixed
point in the past, in that order.  See the \UNIX{} manual page
\manpage{times}{2} or the corresponding Windows Platform API
documentation.
Availability: Macintosh, \UNIX, Windows.
\end{funcdesc}

\begin{funcdesc}{wait}{}
Wait for completion of a child process, and return a tuple containing
its pid and exit status indication: a 16-bit number, whose low byte is
the signal number that killed the process, and whose high byte is the
exit status (if the signal number is zero); the high bit of the low
byte is set if a core file was produced.
Availability: Macintosh, \UNIX.
\end{funcdesc}

\begin{funcdesc}{waitpid}{pid, options}
The details of this function differ on \UNIX{} and Windows.

On \UNIX:
Wait for completion of a child process given by process id \var{pid},
and return a tuple containing its process id and exit status
indication (encoded as for \function{wait()}).  The semantics of the
call are affected by the value of the integer \var{options}, which
should be \code{0} for normal operation.

If \var{pid} is greater than \code{0}, \function{waitpid()} requests
status information for that specific process.  If \var{pid} is
\code{0}, the request is for the status of any child in the process
group of the current process.  If \var{pid} is \code{-1}, the request
pertains to any child of the current process.  If \var{pid} is less
than \code{-1}, status is requested for any process in the process
group \code{-\var{pid}} (the absolute value of \var{pid}).

On Windows:
Wait for completion of a process given by process handle \var{pid},
and return a tuple containing \var{pid},
and its exit status shifted left by 8 bits (shifting makes cross-platform
use of the function easier).
A \var{pid} less than or equal to \code{0} has no special meaning on
Windows, and raises an exception.
The value of integer \var{options} has no effect.
\var{pid} can refer to any process whose id is known, not necessarily a
child process.
The \function{spawn()} functions called with \constant{P_NOWAIT}
return suitable process handles.
\end{funcdesc}

\begin{datadesc}{WNOHANG}
The option for \function{waitpid()} to avoid hanging if no child
process status is available immediately.
Availability: Macintosh, \UNIX.
\end{datadesc}

\begin{datadesc}{WCONTINUED}
This option causes child processes to be reported if they have been
continued from a job control stop since their status was last
reported.
Availability: Some \UNIX{} systems.
\versionadded{2.3}
\end{datadesc}

\begin{datadesc}{WUNTRACED}
This option causes child processes to be reported if they have been
stopped but their current state has not been reported since they were
stopped.
Availability: Macintosh, \UNIX.
\versionadded{2.3}
\end{datadesc}

The following functions take a process status code as returned by
\function{system()}, \function{wait()}, or \function{waitpid()} as a
parameter.  They may be used to determine the disposition of a
process.

\begin{funcdesc}{WCOREDUMP}{status}
Returns \code{True} if a core dump was generated for the process,
otherwise it returns \code{False}.
Availability: Macintosh, \UNIX.
\versionadded{2.3}
\end{funcdesc}

\begin{funcdesc}{WIFCONTINUED}{status}
Returns \code{True} if the process has been continued from a job
control stop, otherwise it returns \code{False}.
Availability: \UNIX.
\versionadded{2.3}
\end{funcdesc}

\begin{funcdesc}{WIFSTOPPED}{status}
Returns \code{True} if the process has been stopped, otherwise it
returns \code{False}.
Availability: \UNIX.
\end{funcdesc}

\begin{funcdesc}{WIFSIGNALED}{status}
Returns \code{True} if the process exited due to a signal, otherwise
it returns \code{False}.
Availability: Macintosh, \UNIX.
\end{funcdesc}

\begin{funcdesc}{WIFEXITED}{status}
Returns \code{True} if the process exited using the \manpage{exit}{2}
system call, otherwise it returns \code{False}.
Availability: Macintosh, \UNIX.
\end{funcdesc}

\begin{funcdesc}{WEXITSTATUS}{status}
If \code{WIFEXITED(\var{status})} is true, return the integer
parameter to the \manpage{exit}{2} system call.  Otherwise, the return
value is meaningless.
Availability: Macintosh, \UNIX.
\end{funcdesc}

\begin{funcdesc}{WSTOPSIG}{status}
Return the signal which caused the process to stop.
Availability: Macintosh, \UNIX.
\end{funcdesc}

\begin{funcdesc}{WTERMSIG}{status}
Return the signal which caused the process to exit.
Availability: Macintosh, \UNIX.
\end{funcdesc}


\subsection{Miscellaneous System Information \label{os-path}}


\begin{funcdesc}{confstr}{name}
Return string-valued system configuration values.
\var{name} specifies the configuration value to retrieve; it may be a
string which is the name of a defined system value; these names are
specified in a number of standards (\POSIX, \UNIX{} 95, \UNIX{} 98, and
others).  Some platforms define additional names as well.  The names
known to the host operating system are given in the
\code{confstr_names} dictionary.  For configuration variables not
included in that mapping, passing an integer for \var{name} is also
accepted.
Availability: Macintosh, \UNIX.

If the configuration value specified by \var{name} isn't defined, the
empty string is returned.

If \var{name} is a string and is not known, \exception{ValueError} is
raised.  If a specific value for \var{name} is not supported by the
host system, even if it is included in \code{confstr_names}, an
\exception{OSError} is raised with \constant{errno.EINVAL} for the
error number.
\end{funcdesc}

\begin{datadesc}{confstr_names}
Dictionary mapping names accepted by \function{confstr()} to the
integer values defined for those names by the host operating system.
This can be used to determine the set of names known to the system.
Availability: Macintosh, \UNIX.
\end{datadesc}

\begin{funcdesc}{getloadavg}{}
Return the number of processes in the system run queue averaged over
the last 1, 5, and 15 minutes or raises OSError if the load average
was unobtainable.

\versionadded{2.3}
\end{funcdesc}

\begin{funcdesc}{sysconf}{name}
Return integer-valued system configuration values.
If the configuration value specified by \var{name} isn't defined,
\code{-1} is returned.  The comments regarding the \var{name}
parameter for \function{confstr()} apply here as well; the dictionary
that provides information on the known names is given by
\code{sysconf_names}.
Availability: Macintosh, \UNIX.
\end{funcdesc}

\begin{datadesc}{sysconf_names}
Dictionary mapping names accepted by \function{sysconf()} to the
integer values defined for those names by the host operating system.
This can be used to determine the set of names known to the system.
Availability: Macintosh, \UNIX.
\end{datadesc}


The follow data values are used to support path manipulation
operations.  These are defined for all platforms.

Higher-level operations on pathnames are defined in the
\refmodule{os.path} module.


\begin{datadesc}{curdir}
The constant string used by the operating system to refer to the current
directory.
For example: \code{'.'} for \POSIX{} or \code{':'} for Mac OS 9.
Also available via \module{os.path}.
\end{datadesc}

\begin{datadesc}{pardir}
The constant string used by the operating system to refer to the parent
directory.
For example: \code{'..'} for \POSIX{} or \code{'::'} for Mac OS 9.
Also available via \module{os.path}.
\end{datadesc}

\begin{datadesc}{sep}
The character used by the operating system to separate pathname components,
for example, \character{/} for \POSIX{} or \character{:} for
Mac OS 9.  Note that knowing this is not sufficient to be able to
parse or concatenate pathnames --- use \function{os.path.split()} and
\function{os.path.join()} --- but it is occasionally useful.
Also available via \module{os.path}.
\end{datadesc}

\begin{datadesc}{altsep}
An alternative character used by the operating system to separate pathname
components, or \code{None} if only one separator character exists.  This is
set to \character{/} on Windows systems where \code{sep} is a
backslash.
Also available via \module{os.path}.
\end{datadesc}

\begin{datadesc}{extsep}
The character which separates the base filename from the extension;
for example, the \character{.} in \file{os.py}.
Also available via \module{os.path}.
\versionadded{2.2}
\end{datadesc}

\begin{datadesc}{pathsep}
The character conventionally used by the operating system to separate
search path components (as in \envvar{PATH}), such as \character{:} for
\POSIX{} or \character{;} for Windows.
Also available via \module{os.path}.
\end{datadesc}

\begin{datadesc}{defpath}
The default search path used by \function{exec*p*()} and
\function{spawn*p*()} if the environment doesn't have a \code{'PATH'}
key.
Also available via \module{os.path}.
\end{datadesc}

\begin{datadesc}{linesep}
The string used to separate (or, rather, terminate) lines on the
current platform.  This may be a single character, such as \code{'\e
n'} for \POSIX{} or \code{'\e r'} for Mac OS, or multiple characters,
for example, \code{'\e r\e n'} for Windows.
\end{datadesc}

\begin{datadesc}{devnull}
The file path of the null device.
For example: \code{'/dev/null'} for \POSIX{} or \code{'Dev:Nul'} for
Mac OS 9.
Also available via \module{os.path}.
\versionadded{2.4}
\end{datadesc}


\subsection{Miscellaneous Functions \label{os-miscfunc}}

\begin{funcdesc}{urandom}{n}
Return a string of \var{n} random bytes suitable for cryptographic use.

This function returns random bytes from an OS-specific
randomness source.  The returned data should be unpredictable enough for
cryptographic applications, though its exact quality depends on the OS
implementation.  On a UNIX-like system this will query /dev/urandom, and
on Windows it will use CryptGenRandom.  If a randomness source is not
found, \exception{NotImplementedError} will be raised.
\versionadded{2.4}
\end{funcdesc}





\section{\module{time} ---
         Time access and conversions}

\declaremodule{builtin}{time}
\modulesynopsis{Time access and conversions.}


This module provides various time-related functions.
It is always available, but not all functions are available
on all platforms.

An explanation of some terminology and conventions is in order.

\begin{itemize}

\item
The \dfn{epoch}\index{epoch} is the point where the time starts.  On
January 1st of that year, at 0 hours, the ``time since the epoch'' is
zero.  For \UNIX{}, the epoch is 1970.  To find out what the epoch is,
look at \code{gmtime(0)}.

\item
The functions in this module do not handle dates and times before the
epoch or far in the future.  The cut-off point in the future is
determined by the C library; for \UNIX{}, it is typically in
2038\index{Year 2038}.

\item
\strong{Year 2000 (Y2K) issues}:\index{Year 2000}\index{Y2K}  Python
depends on the platform's C library, which generally doesn't have year
2000 issues, since all dates and times are represented internally as
seconds since the epoch.  Functions accepting a time tuple (see below)
generally require a 4-digit year.  For backward compatibility, 2-digit
years are supported if the module variable \code{accept2dyear} is a
non-zero integer; this variable is initialized to \code{1} unless the
environment variable \envvar{PYTHONY2K} is set to a non-empty string,
in which case it is initialized to \code{0}.  Thus, you can set
\envvar{PYTHONY2K} to a non-empty string in the environment to require 4-digit
years for all year input.  When 2-digit years are accepted, they are
converted according to the \POSIX{} or X/Open standard: values 69-99
are mapped to 1969-1999, and values 0--68 are mapped to 2000--2068.
Values 100--1899 are always illegal.  Note that this is new as of
Python 1.5.2(a2); earlier versions, up to Python 1.5.1 and 1.5.2a1,
would add 1900 to year values below 1900.

\item
UTC\index{UTC} is Coordinated Universal Time\index{Coordinated
Universal Time} (formerly known as Greenwich Mean
Time,\index{Greenwich Mean Time} or GMT).  The acronym UTC is not a
mistake but a compromise between English and French.

\item
DST is Daylight Saving Time,\index{Daylight Saving Time} an adjustment
of the timezone by (usually) one hour during part of the year.  DST
rules are magic (determined by local law) and can change from year to
year.  The C library has a table containing the local rules (often it
is read from a system file for flexibility) and is the only source of
True Wisdom in this respect.

\item
The precision of the various real-time functions may be less than
suggested by the units in which their value or argument is expressed.
E.g.\ on most \UNIX{} systems, the clock ``ticks'' only 50 or 100 times a
second, and on the Mac, times are only accurate to whole seconds.

\item
On the other hand, the precision of \function{time()} and
\function{sleep()} is better than their \UNIX{} equivalents: times are
expressed as floating point numbers, \function{time()} returns the
most accurate time available (using \UNIX{} \cfunction{gettimeofday()}
where available), and \function{sleep()} will accept a time with a
nonzero fraction (\UNIX{} \cfunction{select()} is used to implement
this, where available).

\item

The time tuple as returned by \function{gmtime()},
\function{localtime()}, and \function{strptime()}, and accepted by
\function{asctime()}, \function{mktime()} and \function{strftime()},
is a tuple of 9 integers:

\begin{tableiii}{r|l|l}{textrm}{Index}{Field}{Values}
  \lineiii{0}{year}{(e.g.\ 1993)}
  \lineiii{1}{month}{range [1,12]}
  \lineiii{2}{day}{range [1,31]}
  \lineiii{3}{hour}{range [0,23]}
  \lineiii{4}{minute}{range [0,59]}
  \lineiii{5}{second}{range [0,61]; see \strong{(1)} in \function{strftime()} description}
  \lineiii{6}{weekday}{range [0,6], Monday is 0}
  \lineiii{7}{Julian day}{range [1,366]}
  \lineiii{8}{daylight savings flag}{0, 1 or -1; see below}
\end{tableiii}

Note that unlike the C structure, the month value is a
range of 1-12, not 0-11.  A year value will be handled as described
under ``Year 2000 (Y2K) issues'' above.  A \code{-1} argument as
daylight savings flag, passed to \function{mktime()} will usually
result in the correct daylight savings state to be filled in.

\end{itemize}

The module defines the following functions and data items:


\begin{datadesc}{accept2dyear}
Boolean value indicating whether two-digit year values will be
accepted.  This is true by default, but will be set to false if the
environment variable \envvar{PYTHONY2K} has been set to a non-empty
string.  It may also be modified at run time.
\end{datadesc}

\begin{datadesc}{altzone}
The offset of the local DST timezone, in seconds west of UTC, if one
is defined.  This is negative if the local DST timezone is east of UTC
(as in Western Europe, including the UK).  Only use this if
\code{daylight} is nonzero.
\end{datadesc}

\begin{funcdesc}{asctime}{\optional{tuple}}
Convert a tuple representing a time as returned by \function{gmtime()}
or \function{localtime()} to a 24-character string of the following form:
\code{'Sun Jun 20 23:21:05 1993'}.  If \var{tuple} is not provided, the
current time as returned by \function{localtime()} is used.  Note: unlike
the C function of the same name, there is no trailing newline.
\versionchanged[Allowed \var{tuple} to be omitted]{2.1}
\end{funcdesc}

\begin{funcdesc}{clock}{}
On \UNIX, return
the current processor time as a floating point number expressed in
seconds.  The precision, and in fact the very definition of the meaning
of ``processor time''\index{CPU time}\index{processor time}, depends
on that of the C function of the same name, but in any case, this is
the function to use for benchmarking\index{benchmarking} Python or
timing algorithms.

On Windows, this function returns the nearest approximation to
wall-clock time since the first call to this function, based on the
Win32 function \cfunction{QueryPerformanceCounter()}.  The resolution
is typically better than one microsecond.
\end{funcdesc}

\begin{funcdesc}{ctime}{\optional{secs}}
Convert a time expressed in seconds since the epoch to a string
representing local time. If \var{secs} is not provided, the current time
as returned by \function{time()} is used.  \code{ctime(\var{secs})}
is equivalent to \code{asctime(localtime(\var{secs}))}.
\versionchanged[Allowed \var{secs} to be omitted]{2.1}
\end{funcdesc}

\begin{datadesc}{daylight}
Nonzero if a DST timezone is defined.
\end{datadesc}

\begin{funcdesc}{gmtime}{\optional{secs}}
Convert a time expressed in seconds since the epoch to a time tuple
in UTC in which the dst flag is always zero.  If \var{secs} is not
provided, the current time as returned by \function{time()} is used.
Fractions of a second are ignored.  See above for a description of the
tuple lay-out.
\versionchanged[Allowed \var{secs} to be omitted]{2.1}
\end{funcdesc}

\begin{funcdesc}{localtime}{\optional{secs}}
Like \function{gmtime()} but converts to local time.  The dst flag is
set to \code{1} when DST applies to the given time.
\versionchanged[Allowed \var{secs} to be omitted]{2.1}
\end{funcdesc}

\begin{funcdesc}{mktime}{tuple}
This is the inverse function of \function{localtime()}.  Its argument
is the full 9-tuple (since the dst flag is needed; use \code{-1} as
the dst flag if it is unknown) which expresses the time in
\emph{local} time, not UTC.  It returns a floating point number, for
compatibility with \function{time()}.  If the input value cannot be
represented as a valid time, \exception{OverflowError} is raised.
\end{funcdesc}

\begin{funcdesc}{sleep}{secs}
Suspend execution for the given number of seconds.  The argument may
be a floating point number to indicate a more precise sleep time.
The actual suspension time may be less than that requested because any
caught signal will terminate the \function{sleep()} following
execution of that signal's catching routine.  Also, the suspension
time may be longer than requested by an arbitrary amount because of
the scheduling of other activity in the system.
\end{funcdesc}

\begin{funcdesc}{strftime}{format\optional{, tuple}}
Convert a tuple representing a time as returned by \function{gmtime()}
or \function{localtime()} to a string as specified by the \var{format}
argument.  If \var{tuple} is not provided, the current time as returned by
\function{localtime()} is used.  \var{format} must be a string.
\versionchanged[Allowed \var{tuple} to be omitted]{2.1}

The following directives can be embedded in the \var{format} string.
They are shown without the optional field width and precision
specification, and are replaced by the indicated characters in the
\function{strftime()} result:

\begin{tableiii}{c|p{24em}|c}{code}{Directive}{Meaning}{Notes}
  \lineiii{\%a}{Locale's abbreviated weekday name.}{}
  \lineiii{\%A}{Locale's full weekday name.}{}
  \lineiii{\%b}{Locale's abbreviated month name.}{}
  \lineiii{\%B}{Locale's full month name.}{}
  \lineiii{\%c}{Locale's appropriate date and time representation.}{}
  \lineiii{\%d}{Day of the month as a decimal number [01,31].}{}
  \lineiii{\%H}{Hour (24-hour clock) as a decimal number [00,23].}{}
  \lineiii{\%I}{Hour (12-hour clock) as a decimal number [01,12].}{}
  \lineiii{\%j}{Day of the year as a decimal number [001,366].}{}
  \lineiii{\%m}{Month as a decimal number [01,12].}{}
  \lineiii{\%M}{Minute as a decimal number [00,59].}{}
  \lineiii{\%p}{Locale's equivalent of either AM or PM.}{}
  \lineiii{\%S}{Second as a decimal number [00,61].}{(1)}
  \lineiii{\%U}{Week number of the year (Sunday as the first day of the
                week) as a decimal number [00,53].  All days in a new year
                preceding the first Sunday are considered to be in week 0.}{}
  \lineiii{\%w}{Weekday as a decimal number [0(Sunday),6].}{}
  \lineiii{\%W}{Week number of the year (Monday as the first day of the
                week) as a decimal number [00,53].  All days in a new year
                preceding the first Sunday are considered to be in week 0.}{}
  \lineiii{\%x}{Locale's appropriate date representation.}{}
  \lineiii{\%X}{Locale's appropriate time representation.}{}
  \lineiii{\%y}{Year without century as a decimal number [00,99].}{}
  \lineiii{\%Y}{Year with century as a decimal number.}{}
  \lineiii{\%Z}{Time zone name (or by no characters if no time zone exists).}{}
  \lineiii{\%\%}{A literal \character{\%} character.}{}
\end{tableiii}

\noindent
Notes:

\begin{description}
  \item[(1)]
    The range really is \code{0} to \code{61}; this accounts for leap
    seconds and the (very rare) double leap seconds.
\end{description}

Here is an example, a format for dates compatible with that specified 
in the \rfc{2822} Internet email standard.
	\footnote{The use of \code{\%Z} is now
	deprecated, but the \code{\%z} escape that expands to the preferred 
	hour/minute offset is not supported by all ANSI C libraries. Also,
	a strict reading of the original 1982 \rfc{822} standard calls for
	a two-digit year (\%y rather than \%Y), but practice moved to
	4-digit years long before the year 2000.  The 4-digit year has
        been mandated by \rfc{2822}, which obsoletes \rfc{822}.}

\begin{verbatim}
>>> from time import gmtime, strftime
>>> strftime("%a, %d %b %Y %H:%M:%S +0000", gmtime())
'Thu, 28 Jun 2001 14:17:15 +0000'
\end{verbatim}

Additional directives may be supported on certain platforms, but
only the ones listed here have a meaning standardized by ANSI C.

On some platforms, an optional field width and precision
specification can immediately follow the initial \character{\%} of a
directive in the following order; this is also not portable.
The field width is normally 2 except for \code{\%j} where it is 3.
\end{funcdesc}

\begin{funcdesc}{strptime}{string\optional{, format}}
Parse a string representing a time according to a format.  The return 
value is a tuple as returned by \function{gmtime()} or
\function{localtime()}.  The \var{format} parameter uses the same
directives as those used by \function{strftime()}; it defaults to
\code{"\%a \%b \%d \%H:\%M:\%S \%Y"} which matches the formatting
returned by \function{ctime()}.  The same platform caveats apply; see
the local \UNIX{} documentation for restrictions or additional
supported directives.  If \var{string} cannot be parsed according to
\var{format}, \exception{ValueError} is raised.  Values which are not
provided as part of the input string are filled in with default
values; the specific values are platform-dependent as the XPG standard
does not provide sufficient information to constrain the result.

\strong{Note:} This function relies entirely on the underlying
platform's C library for the date parsing, and some of these libraries
are buggy.  There's nothing to be done about this short of a new,
portable implementation of \cfunction{strptime()}.

Availability: Most modern \UNIX{} systems.
\end{funcdesc}

\begin{funcdesc}{time}{}
Return the time as a floating point number expressed in seconds since
the epoch, in UTC.  Note that even though the time is always returned
as a floating point number, not all systems provide time with a better
precision than 1 second.
\end{funcdesc}

\begin{datadesc}{timezone}
The offset of the local (non-DST) timezone, in seconds west of UTC
(i.e. negative in most of Western Europe, positive in the US, zero in
the UK).
\end{datadesc}

\begin{datadesc}{tzname}
A tuple of two strings: the first is the name of the local non-DST
timezone, the second is the name of the local DST timezone.  If no DST
timezone is defined, the second string should not be used.
\end{datadesc}


\begin{seealso}
  \seemodule{locale}{Internationalization services.  The locale
                     settings can affect the return values for some of 
                     the functions in the \module{time} module.}
\end{seealso}

\section{Standard Module \sectcode{getopt}}

\stmodindex{getopt}
This module helps scripts to parse the command line arguments in
\code{sys.argv}.
It uses the same conventions as the \UNIX{}
\code{getopt()}
function (including the special meanings of arguments of the form
\samp{-} and \samp{--}).
It defines the function
\code{getopt.getopt(args, options)}
and the exception
\code{getopt.error}.

The first argument to
\code{getopt()}
is the argument list passed to the script with its first element
chopped off (i.e.,
\code{sys.argv[1:]}).
The second argument is the string of option letters that the
script wants to recognize, with options that require an argument
followed by a colon (i.e., the same format that \UNIX{}
\code{getopt()}
uses).
The return value consists of two elements: the first is a list of
option-and-value pairs; the second is the list of program arguments
left after the option list was stripped (this is a trailing slice of the
first argument).
Each option-and-value pair returned has the option as its first element,
prefixed with a hyphen (e.g.,
\code{'-x'}),
and the option argument as its second element, or an empty string if the
option has no argument.
The options occur in the list in the same order in which they were
found, thus allowing multiple occurrences.
Example:

\bcode\begin{verbatim}
>>> import getopt, string
>>> args = string.split('-a -b -cfoo -d bar a1 a2')
>>> args
['-a', '-b', '-cfoo', '-d', 'bar', 'a1', 'a2']
>>> optlist, args = getopt.getopt(args, 'abc:d:')
>>> optlist
[('-a', ''), ('-b', ''), ('-c', 'foo'), ('-d', 'bar')]
>>> args
['a1', 'a2']
>>> 
\end{verbatim}\ecode

The exception
\code{getopt.error = 'getopt error'}
is raised when an unrecognized option is found in the argument list or
when an option requiring an argument is given none.
The argument to the exception is a string indicating the cause of the
error.

\section{\module{tempfile} ---
         Generate temporary files and directories}
\sectionauthor{Zack Weinberg}{zack@codesourcery.com}

\declaremodule{standard}{tempfile}
\modulesynopsis{Generate temporary files and directories.}

\indexii{temporary}{file name}
\indexii{temporary}{file}

This module generates temporary files and directories.  It works on
all supported platforms.

In version 2.3 of Python, this module was overhauled for enhanced
security.  It now provides three new functions,
\function{NamedTemporaryFile()}, \function{mkstemp()}, and
\function{mkdtemp()}, which should eliminate all remaining need to use
the insecure \function{mktemp()} function.  Temporary file names created
by this module no longer contain the process ID; instead a string of
six random characters is used.

Also, all the user-callable functions now take additional arguments
which allow direct control over the location and name of temporary
files.  It is no longer necessary to use the global \var{tempdir} and
\var{template} variables.  To maintain backward compatibility, the
argument order is somewhat odd; it is recommended to use keyword
arguments for clarity.

The module defines the following user-callable functions:

\begin{funcdesc}{TemporaryFile}{\optional{mode=\code{'w+b'}\optional{,
                                bufsize=\code{-1}\optional{,
                                suffix\optional{, prefix\optional{, dir}}}}}}
Return a file (or file-like) object that can be used as a temporary
storage area.  The file is created using \function{mkstemp}. It will
be destroyed as soon as it is closed (including an implicit close when
the object is garbage collected).  Under \UNIX, the directory entry
for the file is removed immediately after the file is created.  Other
platforms do not support this; your code should not rely on a
temporary file created using this function having or not having a
visible name in the file system.

The \var{mode} parameter defaults to \code{'w+b'} so that the file
created can be read and written without being closed.  Binary mode is
used so that it behaves consistently on all platforms without regard
for the data that is stored.  \var{bufsize} defaults to \code{-1},
meaning that the operating system default is used.

The \var{dir}, \var{prefix} and \var{suffix} parameters are passed to
\function{mkstemp()}.
\end{funcdesc}

\begin{funcdesc}{NamedTemporaryFile}{\optional{mode=\code{'w+b'}\optional{,
                                     bufsize=\code{-1}\optional{,
                                     suffix\optional{, prefix\optional{,
                                     dir}}}}}}
This function operates exactly as \function{TemporaryFile()} does,
except that the file is guaranteed to have a visible name in the file
system (on \UNIX, the directory entry is not unlinked).  That name can
be retrieved from the \member{name} member of the file object.  Whether
the name can be used to open the file a second time, while the
named temporary file is still open, varies across platforms (it can
be so used on \UNIX; it cannot on Windows NT or later).
\versionadded{2.3}
\end{funcdesc}

\begin{funcdesc}{mkstemp}{\optional{suffix\optional{,
                          prefix\optional{, dir\optional{, text}}}}}
Creates a temporary file in the most secure manner possible.  There
are no race conditions in the file's creation, assuming that the
platform properly implements the \constant{O_EXCL} flag for
\function{os.open()}.  The file is readable and writable only by the
creating user ID.  If the platform uses permission bits to indicate
whether a file is executable, the file is executable by no one.  The
file descriptor is not inherited by child processes.

Unlike \function{TemporaryFile()}, the user of \function{mkstemp()} is
responsible for deleting the temporary file when done with it.

If \var{suffix} is specified, the file name will end with that suffix,
otherwise there will be no suffix.  \function{mkstemp()} does not put a
dot between the file name and the suffix; if you need one, put it at
the beginning of \var{suffix}.

If \var{prefix} is specified, the file name will begin with that
prefix; otherwise, a default prefix is used.

If \var{dir} is specified, the file will be created in that directory;
otherwise, a default directory is used.

If \var{text} is specified, it indicates whether to open the file in
binary mode (the default) or text mode.  On some platforms, this makes
no difference.

\function{mkstemp()} returns a tuple containing an OS-level handle to
an open file (as would be returned by \function{os.open()}) and the
absolute pathname of that file, in that order.
\versionadded{2.3}
\end{funcdesc}

\begin{funcdesc}{mkdtemp}{\optional{suffix\optional{, prefix\optional{, dir}}}}
Creates a temporary directory in the most secure manner possible.
There are no race conditions in the directory's creation.  The
directory is readable, writable, and searchable only by the
creating user ID.

The user of \function{mkdtemp()} is responsible for deleting the
temporary directory and its contents when done with it.

The \var{prefix}, \var{suffix}, and \var{dir} arguments are the same
as for \function{mkstemp()}.

\function{mkdtemp()} returns the absolute pathname of the new directory.
\versionadded{2.3}
\end{funcdesc}

\begin{funcdesc}{mktemp}{\optional{suffix\optional{, prefix\optional{, dir}}}}
\deprecated{2.3}{Use \function{mkstemp()} instead.}
Return an absolute pathname of a file that did not exist at the time
the call is made.  The \var{prefix}, \var{suffix}, and \var{dir}
arguments are the same as for \function{mkstemp()}.

\warning{Use of this function may introduce a security hole in your
program.  By the time you get around to doing anything with the file
name it returns, someone else may have beaten you to the punch.}
\end{funcdesc}

The module uses two global variables that tell it how to construct a
temporary name.  They are initialized at the first call to any of the
functions above.  The caller may change them, but this is discouraged;
use the appropriate function arguments, instead.

\begin{datadesc}{tempdir}
When set to a value other than \code{None}, this variable defines the
default value for the \var{dir} argument to all the functions defined
in this module.

If \code{tempdir} is unset or \code{None} at any call to any of the
above functions, Python searches a standard list of directories and
sets \var{tempdir} to the first one which the calling user can create
files in.  The list is:

\begin{enumerate}
\item The directory named by the \envvar{TMPDIR} environment variable.
\item The directory named by the \envvar{TEMP} environment variable.
\item The directory named by the \envvar{TMP} environment variable.
\item A platform-specific location:
    \begin{itemize}
    \item On RiscOS, the directory named by the
          \envvar{Wimp\$ScrapDir} environment variable.
    \item On Windows, the directories
          \file{C:$\backslash$TEMP},
          \file{C:$\backslash$TMP},
          \file{$\backslash$TEMP}, and
          \file{$\backslash$TMP}, in that order.
    \item On all other platforms, the directories
          \file{/tmp}, \file{/var/tmp}, and \file{/usr/tmp}, in that order.
    \end{itemize}
\item As a last resort, the current working directory.
\end{enumerate}
\end{datadesc}

\begin{funcdesc}{gettempdir}{}
Return the directory currently selected to create temporary files in.
If \code{tempdir} is not \code{None}, this simply returns its contents;
otherwise, the search described above is performed, and the result
returned.
\end{funcdesc}

\begin{datadesc}{template}
\deprecated{2.0}{Use \function{gettempprefix()} instead.}
When set to a value other than \code{None}, this variable defines the
prefix of the final component of the filenames returned by
\function{mktemp()}.  A string of six random letters and digits is
appended to the prefix to make the filename unique.  On Windows,
the default prefix is \file{\textasciitilde{}T}; on all other systems
it is \file{tmp}.

Older versions of this module used to require that \code{template} be
set to \code{None} after a call to \function{os.fork()}; this has not
been necessary since version 1.5.2.
\end{datadesc}

\begin{funcdesc}{gettempprefix}{}
Return the filename prefix used to create temporary files.  This does
not contain the directory component.  Using this function is preferred
over reading the \var{template} variable directly.
\versionadded{1.5.2}
\end{funcdesc}


\chapter{Optional Operating System Services}
\label{someos}

The modules described in this chapter provide interfaces to operating
system features that are available on selected operating systems only.
The interfaces are generally modelled after the \UNIX{} or \C{}
interfaces but they are available on some other systems as well
(e.g. Windows or NT).  Here's an overview:

\begin{description}

\item[signal]
--- Set handlers for asynchronous events.

\item[socket]
--- Low-level networking interface.

\item[select]
--- Wait for I/O completion on multiple streams.

\item[thread]
--- Create multiple threads of control within one namespace.

\item[threading]
--- Higher level threading interface; use in preference of module
\module{thread}.

\item[Queue]
--- A stynchronized queue class.

\item[anydbm]
--- Generic interface to DBM-style database modules.

\item[whichdb]
--- Guess which DBM-style module created a given database.

\item[zlib]
\item[gzip]
--- Compression and decompression compatible with the
\program{gzip} program (\module{zlib} is the low-level interface,
\module{gzip} the high-level one).

\end{description}
		% Optional Operating System Services
\section{Built-in Module \sectcode{signal}}

\bimodindex{signal}
This module provides mechanisms to write signal handlers in Python.

{\bf Warning:} Some care must be taken if both signals and threads
will be used in the same program.  The fundamental thing to remember
in using signals and threads simultaneously is: always perform
\code{signal()} operations in the main thread of execution.  Any
thread can perform a \code{alarm()}, \code{getsignal()}, or
\code{pause()}; only the main thread can set a new signal handler, and
the main thread will be the only one to receive signals.  This means
that signals can't be used as a means of interthread communication.
Use locks instead.

The variables defined in the signal module are:

\renewcommand{\indexsubitem}{(in module signal)}
\begin{datadesc}{SIG_DFL}
  This is one of two standard signal handling options; it will simply
  perform the default function for the signal.  For example, on most
  systems the default action for SIGQUIT is to dump core and exit,
  while the default action for SIGCLD is to simply ignore it.
\end{datadesc}

\begin{datadesc}{SIG_IGN}
  This is another standard signal handler, which will simply ignore
  the given signal.
\end{datadesc}

\begin{datadesc}{SIG*}
  All the signal numbers are defined symbolically.  For example, the
  hangup signal is defined as \code{signal.SIGHUP}; the variable names
  are identical to the names used in C programs, as found in
  \file{signal.h}.
  The UNIX man page for \file{signal} lists the existing signals (on
  some systems this is \file{signal(2)}, on others the list is in
  \file{signal(7)}).
  Note that not all systems define the same set of signal names; only
  those names defined by the system are defined by this module.
\end{datadesc}

The signal module defines the following functions:

\begin{funcdesc}{alarm}{time}
  If \var{time} is non-zero, this function requests that a
  \code{SIGALRM} signal be sent to the process in \var{time} seconds.
  Any previously scheduled alarm is canceled (i.e. only one alarm can
  be scheduled at any time).  The returned value is then the number of
  seconds before any previously set alarm was to have been delivered.
  If \var{time} is zero, no alarm id scheduled, and any scheduled
  alarm is canceled.  The return value is the number of seconds
  remaining before a previously scheduled alarm.  If the return value
  is zero, no alarm is currently scheduled.  (See the UNIX man page
  \code{alarm(2)}.)
\end{funcdesc}

\begin{funcdesc}{getsignal}{signalnum}
  Returns the current signal handler for the signal \var{signalnum}.
  The returned value may be a callable Python object, or one of the
  special values \code{signal.SIG_IGN} or \code{signal.SIG_DFL}.
\end{funcdesc}

\begin{funcdesc}{pause}{}
  Causes the process to sleep until a signal is received; the
  appropriate handler will then be called.  Returns nothing.  (See the
  UNIX man page \code{signal(2)}.)
\end{funcdesc}

\begin{funcdesc}{signal}{signalnum\, handler}
  Sets the handler for signal \var{signalnum} to the function
  \var{handler}.  \var{handler} can be any callable Python object, or
  one of the special values \code{signal.SIG_IGN} or
  \code{signal.SIG_DFL}.  The previous signal handler will be
  returned.  (See the UNIX man page \code{signal(2)}.)

  If threads are enabled, this function can only be called from the
  main thread; attempting to call it from other threads will cause a
  \code{ValueError} exception will be raised.
\end{funcdesc}

\section{\module{socket} ---
         Low-level networking interface}

\declaremodule{builtin}{socket}
\modulesynopsis{Low-level networking interface.}


This module provides access to the BSD \emph{socket} interface.
It is available on all modern \UNIX{} systems, Windows, MacOS, BeOS,
OS/2, and probably additional platforms.

For an introduction to socket programming (in C), see the following
papers: \citetitle{An Introductory 4.3BSD Interprocess Communication
Tutorial}, by Stuart Sechrest and \citetitle{An Advanced 4.3BSD
Interprocess Communication Tutorial}, by Samuel J.  Leffler et al,
both in the \citetitle{\UNIX{} Programmer's Manual, Supplementary Documents 1}
(sections PS1:7 and PS1:8).  The platform-specific reference material
for the various socket-related system calls are also a valuable source
of information on the details of socket semantics.  For \UNIX, refer
to the manual pages; for Windows, see the WinSock (or Winsock 2)
specification.

The Python interface is a straightforward transliteration of the
\UNIX{} system call and library interface for sockets to Python's
object-oriented style: the \function{socket()} function returns a
\dfn{socket object}\obindex{socket} whose methods implement the
various socket system calls.  Parameter types are somewhat
higher-level than in the C interface: as with \method{read()} and
\method{write()} operations on Python files, buffer allocation on
receive operations is automatic, and buffer length is implicit on send
operations.

Socket addresses are represented as a single string for the
\constant{AF_UNIX} address family and as a pair
\code{(\var{host}, \var{port})} for the \constant{AF_INET} address
family, where \var{host} is a string representing
either a hostname in Internet domain notation like
\code{'daring.cwi.nl'} or an IP address like \code{'100.50.200.5'},
and \var{port} is an integral port number.  Other address families are
currently not supported.  The address format required by a particular
socket object is automatically selected based on the address family
specified when the socket object was created.

For IP addresses, two special forms are accepted instead of a host
address: the empty string represents \constant{INADDR_ANY}, and the string
\code{'<broadcast>'} represents \constant{INADDR_BROADCAST}.

All errors raise exceptions.  The normal exceptions for invalid
argument types and out-of-memory conditions can be raised; errors
related to socket or address semantics raise the error
\exception{socket.error}.

Non-blocking mode is supported through the
\method{setblocking()} method.

The module \module{socket} exports the following constants and functions:


\begin{excdesc}{error}
This exception is raised for socket- or address-related errors.
The accompanying value is either a string telling what went wrong or a
pair \code{(\var{errno}, \var{string})}
representing an error returned by a system
call, similar to the value accompanying \exception{os.error}.
See the module \refmodule{errno}\refbimodindex{errno}, which contains
names for the error codes defined by the underlying operating system.
\end{excdesc}

\begin{datadesc}{AF_UNIX}
\dataline{AF_INET}
These constants represent the address (and protocol) families,
used for the first argument to \function{socket()}.  If the
\constant{AF_UNIX} constant is not defined then this protocol is
unsupported.
\end{datadesc}

\begin{datadesc}{SOCK_STREAM}
\dataline{SOCK_DGRAM}
\dataline{SOCK_RAW}
\dataline{SOCK_RDM}
\dataline{SOCK_SEQPACKET}
These constants represent the socket types,
used for the second argument to \function{socket()}.
(Only \constant{SOCK_STREAM} and
\constant{SOCK_DGRAM} appear to be generally useful.)
\end{datadesc}

\begin{datadesc}{SO_*}
\dataline{SOMAXCONN}
\dataline{MSG_*}
\dataline{SOL_*}
\dataline{IPPROTO_*}
\dataline{IPPORT_*}
\dataline{INADDR_*}
\dataline{IP_*}
Many constants of these forms, documented in the \UNIX{} documentation on
sockets and/or the IP protocol, are also defined in the socket module.
They are generally used in arguments to the \method{setsockopt()} and
\method{getsockopt()} methods of socket objects.  In most cases, only
those symbols that are defined in the \UNIX{} header files are defined;
for a few symbols, default values are provided.
\end{datadesc}

\begin{funcdesc}{getfqdn}{\optional{name}}
Return a fully qualified domain name for \var{name}.
If \var{name} is omitted or empty, it is interpreted as the local
host.  To find the fully qualified name, the hostname returned by
\function{gethostbyaddr()} is checked, then aliases for the host, if
available.  The first name which includes a period is selected.  In
case no fully qualified domain name is available, the hostname is
returned.
\versionadded{2.0}
\end{funcdesc}

\begin{funcdesc}{gethostbyname}{hostname}
Translate a host name to IP address format.  The IP address is
returned as a string, e.g.,  \code{'100.50.200.5'}.  If the host name
is an IP address itself it is returned unchanged.  See
\function{gethostbyname_ex()} for a more complete interface.
\end{funcdesc}

\begin{funcdesc}{gethostbyname_ex}{hostname}
Translate a host name to IP address format, extended interface.
Return a triple \code{(hostname, aliaslist, ipaddrlist)} where
\code{hostname} is the primary host name responding to the given
\var{ip_address}, \code{aliaslist} is a (possibly empty) list of
alternative host names for the same address, and \code{ipaddrlist} is
a list of IP addresses for the same interface on the same
host (often but not always a single address).
\end{funcdesc}

\begin{funcdesc}{gethostname}{}
Return a string containing the hostname of the machine where 
the Python interpreter is currently executing.  If you want to know the
current machine's IP address, use \code{gethostbyname(gethostname())}.
Note: \function{gethostname()} doesn't always return the fully qualified
domain name; use \code{gethostbyaddr(gethostname())}
(see below).
\end{funcdesc}

\begin{funcdesc}{gethostbyaddr}{ip_address}
Return a triple \code{(\var{hostname}, \var{aliaslist},
\var{ipaddrlist})} where \var{hostname} is the primary host name
responding to the given \var{ip_address}, \var{aliaslist} is a
(possibly empty) list of alternative host names for the same address,
and \var{ipaddrlist} is a list of IP addresses for the same interface
on the same host (most likely containing only a single address).
To find the fully qualified domain name, use the function
\function{getfqdn()}.
\end{funcdesc}

\begin{funcdesc}{getprotobyname}{protocolname}
Translate an Internet protocol name (e.g.\ \code{'icmp'}) to a constant
suitable for passing as the (optional) third argument to the
\function{socket()} function.  This is usually only needed for sockets
opened in ``raw'' mode (\constant{SOCK_RAW}); for the normal socket
modes, the correct protocol is chosen automatically if the protocol is
omitted or zero.
\end{funcdesc}

\begin{funcdesc}{getservbyname}{servicename, protocolname}
Translate an Internet service name and protocol name to a port number
for that service.  The protocol name should be \code{'tcp'} or
\code{'udp'}.
\end{funcdesc}

\begin{funcdesc}{socket}{family, type\optional{, proto}}
Create a new socket using the given address family, socket type and
protocol number.  The address family should be \constant{AF_INET} or
\constant{AF_UNIX}.  The socket type should be \constant{SOCK_STREAM},
\constant{SOCK_DGRAM} or perhaps one of the other \samp{SOCK_} constants.
The protocol number is usually zero and may be omitted in that case.
\end{funcdesc}

\begin{funcdesc}{fromfd}{fd, family, type\optional{, proto}}
Build a socket object from an existing file descriptor (an integer as
returned by a file object's \method{fileno()} method).  Address family,
socket type and protocol number are as for the \function{socket()} function
above.  The file descriptor should refer to a socket, but this is not
checked --- subsequent operations on the object may fail if the file
descriptor is invalid.  This function is rarely needed, but can be
used to get or set socket options on a socket passed to a program as
standard input or output (e.g.\ a server started by the \UNIX{} inet
daemon).
\end{funcdesc}

\begin{funcdesc}{ntohl}{x}
Convert 32-bit integers from network to host byte order.  On machines
where the host byte order is the same as network byte order, this is a
no-op; otherwise, it performs a 4-byte swap operation.
\end{funcdesc}

\begin{funcdesc}{ntohs}{x}
Convert 16-bit integers from network to host byte order.  On machines
where the host byte order is the same as network byte order, this is a
no-op; otherwise, it performs a 2-byte swap operation.
\end{funcdesc}

\begin{funcdesc}{htonl}{x}
Convert 32-bit integers from host to network byte order.  On machines
where the host byte order is the same as network byte order, this is a
no-op; otherwise, it performs a 4-byte swap operation.
\end{funcdesc}

\begin{funcdesc}{htons}{x}
Convert 16-bit integers from host to network byte order.  On machines
where the host byte order is the same as network byte order, this is a
no-op; otherwise, it performs a 2-byte swap operation.
\end{funcdesc}

\begin{funcdesc}{inet_aton}{ip_string}
Convert an IP address from dotted-quad string format
(e.g.\ '123.45.67.89') to 32-bit packed binary format, as a string four
characters in length.

Useful when conversing with a program that uses the standard C library
and needs objects of type \ctype{struct in_addr}, which is the C type
for the 32-bit packed binary this function returns.

If the IP address string passed to this function is invalid,
\exception{socket.error} will be raised. Note that exactly what is
valid depends on the underlying C implementation of
\cfunction{inet_aton()}.
\end{funcdesc}

\begin{funcdesc}{inet_ntoa}{packed_ip}
Convert a 32-bit packed IP address (a string four characters in
length) to its standard dotted-quad string representation
(e.g. '123.45.67.89').

Useful when conversing with a program that uses the standard C library
and needs objects of type \ctype{struct in_addr}, which is the C type
for the 32-bit packed binary this function takes as an argument.

If the string passed to this function is not exactly 4 bytes in
length, \exception{socket.error} will be raised.
\end{funcdesc}

\begin{datadesc}{SocketType}
This is a Python type object that represents the socket object type.
It is the same as \code{type(socket(...))}.
\end{datadesc}


\begin{seealso}
  \seemodule{SocketServer}{Classes that simplify writing network servers.}
\end{seealso}


\subsection{Socket Objects \label{socket-objects}}

Socket objects have the following methods.  Except for
\method{makefile()} these correspond to \UNIX{} system calls
applicable to sockets.

\begin{methoddesc}[socket]{accept}{}
Accept a connection.
The socket must be bound to an address and listening for connections.
The return value is a pair \code{(\var{conn}, \var{address})}
where \var{conn} is a \emph{new} socket object usable to send and
receive data on the connection, and \var{address} is the address bound
to the socket on the other end of the connection.
\end{methoddesc}

\begin{methoddesc}[socket]{bind}{address}
Bind the socket to \var{address}.  The socket must not already be bound.
(The format of \var{address} depends on the address family --- see
above.)  \strong{Note:}  This method has historically accepted a pair
of parameters for \constant{AF_INET} addresses instead of only a
tuple.  This was never intentional and will no longer be available in
Python 1.7.
\end{methoddesc}

\begin{methoddesc}[socket]{close}{}
Close the socket.  All future operations on the socket object will fail.
The remote end will receive no more data (after queued data is flushed).
Sockets are automatically closed when they are garbage-collected.
\end{methoddesc}

\begin{methoddesc}[socket]{connect}{address}
Connect to a remote socket at \var{address}.
(The format of \var{address} depends on the address family --- see
above.)  \strong{Note:}  This method has historically accepted a pair
of parameters for \constant{AF_INET} addresses instead of only a
tuple.  This was never intentional and will no longer be available in
Python 1.7.
\end{methoddesc}

\begin{methoddesc}[socket]{connect_ex}{address}
Like \code{connect(\var{address})}, but return an error indicator
instead of raising an exception for errors returned by the C-level
\cfunction{connect()} call (other problems, such as ``host not found,''
can still raise exceptions).  The error indicator is \code{0} if the
operation succeeded, otherwise the value of the \cdata{errno}
variable.  This is useful, e.g., for asynchronous connects.
\strong{Note:}  This method has historically accepted a pair of
parameters for \constant{AF_INET} addresses instead of only a tuple.
This was never intentional and will no longer be available in Python
1.7.
\end{methoddesc}

\begin{methoddesc}[socket]{fileno}{}
Return the socket's file descriptor (a small integer).  This is useful
with \function{select.select()}.
\end{methoddesc}

\begin{methoddesc}[socket]{getpeername}{}
Return the remote address to which the socket is connected.  This is
useful to find out the port number of a remote IP socket, for instance.
(The format of the address returned depends on the address family ---
see above.)  On some systems this function is not supported.
\end{methoddesc}

\begin{methoddesc}[socket]{getsockname}{}
Return the socket's own address.  This is useful to find out the port
number of an IP socket, for instance.
(The format of the address returned depends on the address family ---
see above.)
\end{methoddesc}

\begin{methoddesc}[socket]{getsockopt}{level, optname\optional{, buflen}}
Return the value of the given socket option (see the \UNIX{} man page
\manpage{getsockopt}{2}).  The needed symbolic constants
(\constant{SO_*} etc.) are defined in this module.  If \var{buflen}
is absent, an integer option is assumed and its integer value
is returned by the function.  If \var{buflen} is present, it specifies
the maximum length of the buffer used to receive the option in, and
this buffer is returned as a string.  It is up to the caller to decode
the contents of the buffer (see the optional built-in module
\refmodule{struct} for a way to decode C structures encoded as strings).
\end{methoddesc}

\begin{methoddesc}[socket]{listen}{backlog}
Listen for connections made to the socket.  The \var{backlog} argument
specifies the maximum number of queued connections and should be at
least 1; the maximum value is system-dependent (usually 5).
\end{methoddesc}

\begin{methoddesc}[socket]{makefile}{\optional{mode\optional{, bufsize}}}
Return a \dfn{file object} associated with the socket.  (File objects
are described in \ref{bltin-file-objects}, ``File Objects.'')
The file object references a \cfunction{dup()}ped version of the
socket file descriptor, so the file object and socket object may be
closed or garbage-collected independently.
\index{I/O control!buffering}The optional \var{mode}
and \var{bufsize} arguments are interpreted the same way as by the
built-in \function{open()} function.
\end{methoddesc}

\begin{methoddesc}[socket]{recv}{bufsize\optional{, flags}}
Receive data from the socket.  The return value is a string representing
the data received.  The maximum amount of data to be received
at once is specified by \var{bufsize}.  See the \UNIX{} manual page
\manpage{recv}{2} for the meaning of the optional argument
\var{flags}; it defaults to zero.
\end{methoddesc}

\begin{methoddesc}[socket]{recvfrom}{bufsize\optional{, flags}}
Receive data from the socket.  The return value is a pair
\code{(\var{string}, \var{address})} where \var{string} is a string
representing the data received and \var{address} is the address of the
socket sending the data.  The optional \var{flags} argument has the
same meaning as for \method{recv()} above.
(The format of \var{address} depends on the address family --- see above.)
\end{methoddesc}

\begin{methoddesc}[socket]{send}{string\optional{, flags}}
Send data to the socket.  The socket must be connected to a remote
socket.  The optional \var{flags} argument has the same meaning as for
\method{recv()} above.  Returns the number of bytes sent.
\end{methoddesc}

\begin{methoddesc}[socket]{sendto}{string\optional{, flags}, address}
Send data to the socket.  The socket should not be connected to a
remote socket, since the destination socket is specified by
\var{address}.  The optional \var{flags} argument has the same
meaning as for \method{recv()} above.  Return the number of bytes sent.
(The format of \var{address} depends on the address family --- see above.)
\end{methoddesc}

\begin{methoddesc}[socket]{setblocking}{flag}
Set blocking or non-blocking mode of the socket: if \var{flag} is 0,
the socket is set to non-blocking, else to blocking mode.  Initially
all sockets are in blocking mode.  In non-blocking mode, if a
\method{recv()} call doesn't find any data, or if a
\method{send()} call can't immediately dispose of the data, a
\exception{error} exception is raised; in blocking mode, the calls
block until they can proceed.
\end{methoddesc}

\begin{methoddesc}[socket]{setsockopt}{level, optname, value}
Set the value of the given socket option (see the \UNIX{} manual page
\manpage{setsockopt}{2}).  The needed symbolic constants are defined in
the \module{socket} module (\code{SO_*} etc.).  The value can be an
integer or a string representing a buffer.  In the latter case it is
up to the caller to ensure that the string contains the proper bits
(see the optional built-in module
\refmodule{struct}\refbimodindex{struct} for a way to encode C
structures as strings). 
\end{methoddesc}

\begin{methoddesc}[socket]{shutdown}{how}
Shut down one or both halves of the connection.  If \var{how} is
\code{0}, further receives are disallowed.  If \var{how} is \code{1},
further sends are disallowed.  If \var{how} is \code{2}, further sends
and receives are disallowed.
\end{methoddesc}

Note that there are no methods \method{read()} or \method{write()};
use \method{recv()} and \method{send()} without \var{flags} argument
instead.


\subsection{Example \label{socket-example}}

Here are two minimal example programs using the TCP/IP protocol:\ a
server that echoes all data that it receives back (servicing only one
client), and a client using it.  Note that a server must perform the
sequence \function{socket()}, \method{bind()}, \method{listen()},
\method{accept()} (possibly repeating the \method{accept()} to service
more than one client), while a client only needs the sequence
\function{socket()}, \method{connect()}.  Also note that the server
does not \method{send()}/\method{recv()} on the 
socket it is listening on but on the new socket returned by
\method{accept()}.

\begin{verbatim}
# Echo server program
import socket

HOST = ''                 # Symbolic name meaning the local host
PORT = 50007              # Arbitrary non-privileged port
s = socket.socket(socket.AF_INET, socket.SOCK_STREAM)
s.bind((HOST, PORT))
s.listen(1)
conn, addr = s.accept()
print 'Connected by', addr
while 1:
    data = conn.recv(1024)
    if not data: break
    conn.send(data)
conn.close()
\end{verbatim}

\begin{verbatim}
# Echo client program
import socket

HOST = 'daring.cwi.nl'    # The remote host
PORT = 50007              # The same port as used by the server
s = socket.socket(socket.AF_INET, socket.SOCK_STREAM)
s.connect((HOST, PORT))
s.send('Hello, world')
data = s.recv(1024)
s.close()
print 'Received', `data`
\end{verbatim}

\section{Built-in Module \sectcode{select}}
\label{module-select}
\bimodindex{select}

This module provides access to the function \code{select} available in
most \UNIX{} versions.  It defines the following:

\setindexsubitem{(in module select)}
\begin{excdesc}{error}
The exception raised when an error occurs.  The accompanying value is
a pair containing the numeric error code from \code{errno} and the
corresponding string, as would be printed by the C function
\code{perror()}.
\end{excdesc}

\begin{funcdesc}{select}{iwtd, owtd, ewtd\optional{, timeout}}
This is a straightforward interface to the \UNIX{} \code{select()}
system call.  The first three arguments are lists of `waitable
objects': either integers representing \UNIX{} file descriptors or
objects with a parameterless method named \code{fileno()} returning
such an integer.  The three lists of waitable objects are for input,
output and `exceptional conditions', respectively.  Empty lists are
allowed.  The optional \var{timeout} argument specifies a time-out as a
floating point number in seconds.  When the \var{timeout} argument
is omitted the function blocks until at least one file descriptor is
ready.  A time-out value of zero specifies a poll and never blocks.

The return value is a triple of lists of objects that are ready:
subsets of the first three arguments.  When the time-out is reached
without a file descriptor becoming ready, three empty lists are
returned.

Amongst the acceptable object types in the lists are Python file
objects (e.g. \code{sys.stdin}, or objects returned by \code{open()}
or \code{posix.popen()}), socket objects returned by
\code{socket.socket()}, and the module \code{stdwin} which happens to
define a function \code{fileno()} for just this purpose.  You may
also define a \dfn{wrapper} class yourself, as long as it has an
appropriate \code{fileno()} method (that really returns a \UNIX{} file
descriptor, not just a random integer).
\end{funcdesc}
\ttindex{socket}
\ttindex{stdwin}

\section{Built-in Module \sectcode{thread}}
\label{module-thread}
\bimodindex{thread}

This module provides low-level primitives for working with multiple
threads (a.k.a.\ \dfn{light-weight processes} or \dfn{tasks}) --- multiple
threads of control sharing their global data space.  For
synchronization, simple locks (a.k.a.\ \dfn{mutexes} or \dfn{binary
semaphores}) are provided.
\index{light-weight processes}
\index{processes, light-weight}
\index{binary semaphores}
\index{semaphores, binary}

The module is optional.  It is supported on Windows NT and '95, SGI
IRIX, Solaris 2.x, as well as on systems that have a POSIX thread
(a.k.a. ``pthread'') implementation.
\index{pthreads}
\indexii{threads}{posix}

It defines the following constant and functions:

\renewcommand{\indexsubitem}{(in module thread)}
\begin{excdesc}{error}
Raised on thread-specific errors.
\end{excdesc}

\begin{funcdesc}{start_new_thread}{func\, arg}
Start a new thread.  The thread executes the function \var{func}
with the argument list \var{arg} (which must be a tuple).  When the
function returns, the thread silently exits.  When the function
terminates with an unhandled exception, a stack trace is printed and
then the thread exits (but other threads continue to run).
\end{funcdesc}

\begin{funcdesc}{exit}{}
This is a shorthand for \code{thread.exit_thread()}.
\end{funcdesc}

\begin{funcdesc}{exit_thread}{}
Raise the \code{SystemExit} exception.  When not caught, this will
cause the thread to exit silently.
\end{funcdesc}

%\begin{funcdesc}{exit_prog}{status}
%Exit all threads and report the value of the integer argument
%\var{status} as the exit status of the entire program.
%\strong{Caveat:} code in pending \code{finally} clauses, in this thread
%or in other threads, is not executed.
%\end{funcdesc}

\begin{funcdesc}{allocate_lock}{}
Return a new lock object.  Methods of locks are described below.  The
lock is initially unlocked.
\end{funcdesc}

\begin{funcdesc}{get_ident}{}
Return the `thread identifier' of the current thread.  This is a
nonzero integer.  Its value has no direct meaning; it is intended as a
magic cookie to be used e.g. to index a dictionary of thread-specific
data.  Thread identifiers may be recycled when a thread exits and
another thread is created.
\end{funcdesc}

Lock objects have the following methods:

\renewcommand{\indexsubitem}{(lock method)}
\begin{funcdesc}{acquire}{\optional{waitflag}}
Without the optional argument, this method acquires the lock
unconditionally, if necessary waiting until it is released by another
thread (only one thread at a time can acquire a lock --- that's their
reason for existence), and returns \code{None}.  If the integer
\var{waitflag} argument is present, the action depends on its value:\
if it is zero, the lock is only acquired if it can be acquired
immediately without waiting, while if it is nonzero, the lock is
acquired unconditionally as before.  If an argument is present, the
return value is 1 if the lock is acquired successfully, 0 if not.
\end{funcdesc}

\begin{funcdesc}{release}{}
Releases the lock.  The lock must have been acquired earlier, but not
necessarily by the same thread.
\end{funcdesc}

\begin{funcdesc}{locked}{}
Return the status of the lock:\ 1 if it has been acquired by some
thread, 0 if not.
\end{funcdesc}

\strong{Caveats:}

\begin{itemize}
\item
Threads interact strangely with interrupts: the
\code{KeyboardInterrupt} exception will be received by an arbitrary
thread.  (When the \code{signal}\refbimodindex{signal} module is
available, interrupts always go to the main thread.)

\item
Calling \code{sys.exit()} or raising the \code{SystemExit} exception is
equivalent to calling \code{thread.exit_thread()}.

\item
Not all built-in functions that may block waiting for I/O allow other
threads to run.  (The most popular ones (\code{sleep()}, \code{read()},
\code{select()}) work as expected.)

\item
It is not possible to interrupt the \code{acquire()} method on a lock
-- the \code{KeyboardInterrupt} exception will happen after the lock
has been acquired.

\item
When the main thread exits, it is system defined whether the other
threads survive.  On SGI IRIX using the native thread implementation,
they survive.  On most other systems, they are killed without
executing ``try-finally'' clauses or executing object destructors.
\indexii{threads}{IRIX}

\item
When the main thread exits, it doesn't do any of its usual cleanup
(except that ``try-finally'' clauses are honored), and the standard
I/O files are not flushed.

\end{itemize}


\chapter{UNIX ONLY}

The modules described in this chapter provide interfaces to features
that are unique to the \UNIX{} operating system, or in some cases to
some or many variants of it.
			% UNIX Specific Services
\section{\module{posix} ---
         The most common \POSIX{} system calls}

\declaremodule{builtin}{posix}
  \platform{Unix}
\modulesynopsis{The most common \POSIX\ system calls (normally used
                via module \refmodule{os}).}


This module provides access to operating system functionality that is
standardized by the C Standard and the \POSIX{} standard (a thinly
disguised \UNIX{} interface).

\strong{Do not import this module directly.}  Instead, import the
module \refmodule{os}, which provides a \emph{portable} version of this
interface.  On \UNIX, the \refmodule{os} module provides a superset of
the \module{posix} interface.  On non-\UNIX{} operating systems the
\module{posix} module is not available, but a subset is always
available through the \refmodule{os} interface.  Once \refmodule{os} is
imported, there is \emph{no} performance penalty in using it instead
of \module{posix}.  In addition, \refmodule{os}\refstmodindex{os}
provides some additional functionality, such as automatically calling
\function{putenv()} when an entry in \code{os.environ} is changed.

The descriptions below are very terse; refer to the corresponding
\UNIX{} manual (or \POSIX{} documentation) entry for more information.
Arguments called \var{path} refer to a pathname given as a string.

Errors are reported as exceptions; the usual exceptions are given for
type errors, while errors reported by the system calls raise
\exception{error} (a synonym for the standard exception
\exception{OSError}), described below.


\subsection{Large File Support \label{posix-large-files}}
\sectionauthor{Steve Clift}{clift@mail.anacapa.net}
\index{large files}
\index{file!large files}


Several operating systems (including AIX, HPUX, Irix and Solaris)
provide support for files that are larger than 2 Gb from a C
programming model where \ctype{int} and \ctype{long} are 32-bit
values. This is typically accomplished by defining the relevant size
and offset types as 64-bit values. Such files are sometimes referred
to as \dfn{large files}.

Large file support is enabled in Python when the size of an
\ctype{off_t} is larger than a \ctype{long} and the \ctype{long long}
type is available and is at least as large as an \ctype{off_t}. Python
longs are then used to represent file sizes, offsets and other values
that can exceed the range of a Python int. It may be necessary to
configure and compile Python with certain compiler flags to enable
this mode. For example, it is enabled by default with recent versions
of Irix, but with Solaris 2.6 and 2.7 you need to do something like:

\begin{verbatim}
CFLAGS="`getconf LFS_CFLAGS`" OPT="-g -O2 $CFLAGS" \
        ./configure
\end{verbatim} % $ <-- bow to font-lock

On large-file-capable Linux systems, this might work:

\begin{verbatim}
CFLAGS='-D_LARGEFILE64_SOURCE -D_FILE_OFFSET_BITS=64' OPT="-g -O2 $CFLAGS" \
        ./configure
\end{verbatim} % $ <-- bow to font-lock


\subsection{Module Contents \label{posix-contents}}


Module \module{posix} defines the following data item:

\begin{datadesc}{environ}
A dictionary representing the string environment at the time the
interpreter was started. For example, \code{environ['HOME']} is the
pathname of your home directory, equivalent to
\code{getenv("HOME")} in C.

Modifying this dictionary does not affect the string environment
passed on by \function{execv()}, \function{popen()} or
\function{system()}; if you need to change the environment, pass
\code{environ} to \function{execve()} or add variable assignments and
export statements to the command string for \function{system()} or
\function{popen()}.

\note{The \refmodule{os} module provides an alternate
implementation of \code{environ} which updates the environment on
modification.  Note also that updating \code{os.environ} will render
this dictionary obsolete.  Use of the \refmodule{os} module version of
this is recommended over direct access to the \module{posix} module.}
\end{datadesc}

Additional contents of this module should only be accessed via the
\refmodule{os} module; refer to the documentation for that module for
further information.

\section{Standard Module \sectcode{posixpath}}
\label{module-posixpath}
\stmodindex{posixpath}

This module implements some useful functions on \POSIX{} pathnames.

\strong{Do not import this module directly.}  Instead, import the
module \code{os} and use \code{os.path}.
\refstmodindex{os}

\setindexsubitem{(in module posixpath)}

\begin{funcdesc}{basename}{p}
Return the base name of pathname
\var{p}.
This is the second half of the pair returned by
\code{posixpath.split(\var{p})}.
\end{funcdesc}

\begin{funcdesc}{commonprefix}{list}
Return the longest string that is a prefix of all strings in
\var{list}.
If
\var{list}
is empty, return the empty string (\code{''}).
\end{funcdesc}

\begin{funcdesc}{exists}{p}
Return true if
\var{p}
refers to an existing path.
\end{funcdesc}

\begin{funcdesc}{expanduser}{p}
Return the argument with an initial component of \samp{\~} or
\samp{\~\var{user}} replaced by that \var{user}'s home directory.  An
initial \samp{\~{}} is replaced by the environment variable \code{\${}HOME};
an initial \samp{\~\var{user}} is looked up in the password directory through
the built-in module \code{pwd}.  If the expansion fails, or if the
path does not begin with a tilde, the path is returned unchanged.
\refbimodindex{pwd}
\end{funcdesc}

\begin{funcdesc}{expandvars}{p}
Return the argument with environment variables expanded.  Substrings
of the form \samp{\$\var{name}} or \samp{\$\{\var{name}\}} are
replaced by the value of environment variable \var{name}.  Malformed
variable names and references to non-existing variables are left
unchanged.
\end{funcdesc}

\begin{funcdesc}{isabs}{p}
Return true if \var{p} is an absolute pathname (begins with a slash).
\end{funcdesc}

\begin{funcdesc}{isfile}{p}
Return true if \var{p} is an existing regular file.  This follows
symbolic links, so both \code{islink()} and \code{isfile()} can be
true for the same path.
\end{funcdesc}

\begin{funcdesc}{isdir}{p}
Return true if \var{p} is an existing directory.  This follows
symbolic links, so both \code{islink()} and \code{isdir()} can be true
for the same path.
\end{funcdesc}

\begin{funcdesc}{islink}{p}
Return true if
\var{p}
refers to a directory entry that is a symbolic link.
Always false if symbolic links are not supported.
\end{funcdesc}

\begin{funcdesc}{ismount}{p}
Return true if pathname \var{p} is a \dfn{mount point}: a point in a
file system where a different file system has been mounted.  The
function checks whether \var{p}'s parent, \file{\var{p}/..}, is on a
different device than \var{p}, or whether \file{\var{p}/..} and
\var{p} point to the same i-node on the same device --- this should
detect mount points for all \UNIX{} and \POSIX{} variants.
\end{funcdesc}

\begin{funcdesc}{join}{p\optional{\, q\optional{\, ...}}}
Joins one or more path components intelligently.  If any component is
an absolute path, all previous components are thrown away, and joining
continues.  The return value is the concatenation of \var{p}, and
optionally \var{q}, etc., with exactly one slash (\code{'/'}) inserted
between components, unless \var{p} is empty.
\end{funcdesc}

\begin{funcdesc}{normcase}{p}
Normalize the case of a pathname.  This returns the path unchanged;
however, a similar function in \code{macpath} converts upper case to
lower case.
\end{funcdesc}

\begin{funcdesc}{samefile}{p\, q}
Return true if both pathname arguments refer to the same file or directory
(as indicated by device number and i-node number).
Raise an exception if a \code{stat()} call on either pathname fails.
\end{funcdesc}

\begin{funcdesc}{split}{p}
Split the pathname \var{p} in a pair \code{(\var{head}, \var{tail})},
where \var{tail} is the last pathname component and \var{head} is
everything leading up to that.  The \var{tail} part will never contain
a slash; if \var{p} ends in a slash, \var{tail} will be empty.  If
there is no slash in \var{p}, \var{head} will be empty.  If \var{p} is
empty, both \var{head} and \var{tail} are empty.  Trailing slashes are
stripped from \var{head} unless it is the root (one or more slashes
only).  In nearly all cases, \code{join(\var{head}, \var{tail})}
equals \var{p} (the only exception being when there were multiple
slashes separating \var{head} from \var{tail}).
\end{funcdesc}

\begin{funcdesc}{splitext}{p}
Split the pathname \var{p} in a pair \code{(\var{root}, \var{ext})}
such that \code{\var{root} + \var{ext} == \var{p}},
and \var{ext} is empty or begins with a period and contains
at most one period.
\end{funcdesc}

\begin{funcdesc}{walk}{p\, visit\, arg}
Calls the function \var{visit} with arguments
\code{(\var{arg}, \var{dirname}, \var{names})} for each directory in the
directory tree rooted at \var{p} (including \var{p} itself, if it is a
directory).  The argument \var{dirname} specifies the visited directory,
the argument \var{names} lists the files in the directory (gotten from
\code{posix.listdir(\var{dirname})}).
The \var{visit} function may modify \var{names} to
influence the set of directories visited below \var{dirname}, e.g., to
avoid visiting certain parts of the tree.  (The object referred to by
\var{names} must be modified in place, using \code{del} or slice
assignment.)
\end{funcdesc}
		% == posixpath
\section{\module{pwd} ---
         The password database}

\declaremodule{builtin}{pwd}
  \platform{Unix}
\modulesynopsis{The password database (\function{getpwnam()} and friends).}

This module provides access to the \UNIX{} user account and password
database.  It is available on all \UNIX{} versions.

Password database entries are reported as a tuple-like object, whose
attributes correspond to the members of the \code{passwd} structure
(Attribute field below, see \code{<pwd.h>}):

\begin{tableiii}{r|l|l}{textrm}{Index}{Attribute}{Meaning}
  \lineiii{0}{\code{pw_name}}{Login name}
  \lineiii{1}{\code{pw_passwd}}{Optional encrypted password}
  \lineiii{2}{\code{pw_uid}}{Numerical user ID}
  \lineiii{3}{\code{pw_gid}}{Numerical group ID}
  \lineiii{4}{\code{pw_gecos}}{User name or comment field}
  \lineiii{5}{\code{pw_dir}}{User home directory}
  \lineiii{6}{\code{pw_shell}}{User command interpreter}
\end{tableiii}

The uid and gid items are integers, all others are strings.
\exception{KeyError} is raised if the entry asked for cannot be found.

\note{In traditional \UNIX{} the field \code{pw_passwd} usually
contains a password encrypted with a DES derived algorithm (see module
\refmodule{crypt}\refbimodindex{crypt}).  However most modern unices 
use a so-called \emph{shadow password} system.  On those unices the
\var{pw_passwd} field only contains an asterisk (\code{'*'}) or the 
letter \character{x} where the encrypted password is stored in a file
\file{/etc/shadow} which is not world readable.  Whether the \var{pw_passwd}
field contains anything useful is system-dependent.}

It defines the following items:

\begin{funcdesc}{getpwuid}{uid}
Return the password database entry for the given numeric user ID.
\end{funcdesc}

\begin{funcdesc}{getpwnam}{name}
Return the password database entry for the given user name.
\end{funcdesc}

\begin{funcdesc}{getpwall}{}
Return a list of all available password database entries, in arbitrary order.
\end{funcdesc}


\begin{seealso}
  \seemodule{grp}{An interface to the group database, similar to this.}
\end{seealso}

\section{Built-in Module \sectcode{grp}}

\bimodindex{grp}
This module provides access to the \UNIX{} group database.
It is available on all \UNIX{} versions.

Group database entries are reported as 4-tuples containing the
following items from the group database (see \file{<grp.h>}), in order:
\code{gr_name},
\code{gr_passwd},
\code{gr_gid},
\code{gr_mem}.
The gid is an integer, name and password are strings, and the member
list is a list of strings.
(Note that most users are not explicitly listed as members of the
group they are in according to the password database.)
An exception is raised if the entry asked for cannot be found.

It defines the following items:

\renewcommand{\indexsubitem}{(in module grp)}
\begin{funcdesc}{getgrgid}{gid}
Return the group database entry for the given numeric group ID.
\end{funcdesc}

\begin{funcdesc}{getgrnam}{name}
Return the group database entry for the given group name.
\end{funcdesc}

\begin{funcdesc}{getgrall}{}
Return a list of all available group entries, in arbitrary order.
\end{funcdesc}

\section{\module{dbm} ---
         Simple ``database'' interface}

\declaremodule{builtin}{dbm}
  \platform{Unix}
\modulesynopsis{The standard ``database'' interface, based on ndbm.}


The \module{dbm} module provides an interface to the \UNIX{}
\code{(n)dbm} library.  Dbm objects behave like mappings
(dictionaries), except that keys and values are always strings.
Printing a dbm object doesn't print the keys and values, and the
\method{items()} and \method{values()} methods are not supported.

See also the \refmodule{gdbm}\refbimodindex{gdbm} module, which
provides a similar interface using the GNU GDBM library.

The module defines the following constant and functions:

\begin{excdesc}{error}
Raised on dbm-specific errors, such as I/O errors.
\exception{KeyError} is raised for general mapping errors like
specifying an incorrect key.
\end{excdesc}

\begin{funcdesc}{open}{filename, \optional{flag, \optional{mode}}}
Open a dbm database and return a dbm object.  The \var{filename}
argument is the name of the database file (without the \file{.dir} or
\file{.pag} extensions).

The optional \var{flag} argument can be
\code{'r'} (to open an existing database for reading only --- default),
\code{'w'} (to open an existing database for reading and writing),
\code{'c'} (which creates the database if it doesn't exist), or
\code{'n'} (which always creates a new empty database).

The optional \var{mode} argument is the \UNIX{} mode of the file, used
only when the database has to be created.  It defaults to octal
\code{0666}.
\end{funcdesc}


\begin{seealso}
  \seemodule{anydbm}{Generic interface to \code{dbm}-style databases.}
  \seemodule{whichdb}{Utility module used to determine the type of an
                      existing database.}
\end{seealso}

\section{\module{gdbm} ---
         GNU's reinterpretation of dbm}

\declaremodule{builtin}{gdbm}
  \platform{Unix}
\modulesynopsis{GNU's reinterpretation of dbm.}


This module is quite similar to the \refmodule{dbm}\refbimodindex{dbm}
module, but uses \code{gdbm} instead to provide some additional
functionality.  Please note that the file formats created by
\code{gdbm} and \code{dbm} are incompatible.

The \module{gdbm} module provides an interface to the GNU DBM
library.  \code{gdbm} objects behave like mappings
(dictionaries), except that keys and values are always strings.
Printing a \code{gdbm} object doesn't print the keys and values, and
the \method{items()} and \method{values()} methods are not supported.

The module defines the following constant and functions:

\begin{excdesc}{error}
Raised on \code{gdbm}-specific errors, such as I/O errors.
\exception{KeyError} is raised for general mapping errors like
specifying an incorrect key.
\end{excdesc}

\begin{funcdesc}{open}{filename, \optional{flag, \optional{mode}}}
Open a \code{gdbm} database and return a \code{gdbm} object.  The
\var{filename} argument is the name of the database file.

The optional \var{flag} argument can be
\code{'r'} (to open an existing database for reading only --- default),
\code{'w'} (to open an existing database for reading and writing),
\code{'c'} (which creates the database if it doesn't exist), or
\code{'n'} (which always creates a new empty database).

The following additional characters may be appended to the flag to
control how the database is opened:

\begin{itemize}
\item \code{'f'} --- Open the database in fast mode.  Writes to the database
                     will not be synchronized.
\item \code{'s'} --- Synchronized mode. This will cause changes to the database
                     will be immediately written to the file.
\item \code{'u'} --- Do not lock database. 
\end{itemize}

Not all flags are valid for all versions of \code{gdbm}.  The
module constant \code{open_flags} is a string of supported flag
characters.  The exception \exception{error} is raised if an invalid
flag is specified.

The optional \var{mode} argument is the \UNIX{} mode of the file, used
only when the database has to be created.  It defaults to octal
\code{0666}.
\end{funcdesc}

In addition to the dictionary-like methods, \code{gdbm} objects have the
following methods:

\begin{funcdesc}{firstkey}{}
It's possible to loop over every key in the database using this method 
and the \method{nextkey()} method.  The traversal is ordered by
\code{gdbm}'s internal hash values, and won't be sorted by the key
values.  This method returns the starting key.
\end{funcdesc}

\begin{funcdesc}{nextkey}{key}
Returns the key that follows \var{key} in the traversal.  The
following code prints every key in the database \code{db}, without
having to create a list in memory that contains them all:

\begin{verbatim}
k = db.firstkey()
while k != None:
    print k
    k = db.nextkey(k)
\end{verbatim}
\end{funcdesc}

\begin{funcdesc}{reorganize}{}
If you have carried out a lot of deletions and would like to shrink
the space used by the \code{gdbm} file, this routine will reorganize
the database.  \code{gdbm} will not shorten the length of a database
file except by using this reorganization; otherwise, deleted file
space will be kept and reused as new (key, value) pairs are added.
\end{funcdesc}

\begin{funcdesc}{sync}{}
When the database has been opened in fast mode, this method forces any 
unwritten data to be written to the disk.
\end{funcdesc}


\begin{seealso}
  \seemodule{anydbm}{Generic interface to \code{dbm}-style databases.}
  \seemodule{whichdb}{Utility module used to determine the type of an
                      existing database.}
\end{seealso}

\section{Built-in Module \sectcode{termios}}

To be provided.

% Manual text by Jaap Vermeulen
\section{Built-in Module \sectcode{fcntl}}
\bimodindex{fcntl}
\indexii{UNIX@\UNIX{}}{file control}
\indexii{UNIX@\UNIX{}}{I/O control}

This module performs file control and I/O control on file descriptors.
It is an interface to the \dfn{fcntl()} and \dfn{ioctl()} \UNIX{} routines.
File descriptors can be obtained with the \dfn{fileno()} method of a
file or socket object.

The module defines the following functions:

\renewcommand{\indexsubitem}{(in module struct)}

\begin{funcdesc}{fcntl}{fd\, op\optional{\, arg}}
  Perform the requested operation on file descriptor \code{\var{fd}}.
  The operation is defined by \code{\var{op}} and is operating system
  dependent.  Typically these codes can be retrieved from the library
  module \code{FCNTL}. The argument \code{\var{arg}} is optional, and
  defaults to the integer value \code{0}.  When
  it is present, it can either be an integer value, or a string.  With
  the argument missing or an integer value, the return value of this
  function is the integer return value of the real \code{fcntl()}
  call.  When the argument is a string it represents a binary
  structure, e.g.\ created by \code{struct.pack()}. The binary data is
  copied to a buffer whose address is passed to the real \code{fcntl()}
  call.  The return value after a successful call is the contents of
  the buffer, converted to a string object.  In case the
  \code{fcntl()} fails, an \code{IOError} will be raised.
\end{funcdesc}

\begin{funcdesc}{ioctl}{fd\, op\, arg}
  This function is identical to the \code{fcntl()} function, except
  that the operations are typically defined in the library module
  \code{IOCTL}.
\end{funcdesc}

\begin{funcdesc}{flock}{fd\, op}
Perform the lock operation \var{op} on file descriptor \var{fd}.
See the \UNIX{} manual for details.  (On some systems, this function is
emulated using \code{fcntl()}.)
\end{funcdesc}

\begin{funcdesc}{lockf}{fd\, code\, \optional{len\, \optional{start\, \optional{whence}}}}
This is a wrapper around the \code{F_SETLK} and \code{F_SETLKW}
\code{fcntl()} calls.  See the \UNIX{} manual for details.
\end{funcdesc}

If the library modules \code{FCNTL} or \code{IOCTL} are missing, you
can find the opcodes in the C include files \file{sys/fcntl.h} and
\file{sys/ioctl.h}. You can create the modules yourself with the h2py
script, found in the \file{Tools/scripts} directory.
\refstmodindex{FCNTL}
\refstmodindex{IOCTL}

Examples (all on a SVR4 compliant system):

\bcode\begin{verbatim}
import struct, FCNTL

file = open(...)
rv = fcntl(file.fileno(), FCNTL.O_NDELAY, 1)

lockdata = struct.pack('hhllhh', FCNTL.F_WRLCK, 0, 0, 0, 0, 0)
rv = fcntl(file.fileno(), FCNTL.F_SETLKW, lockdata)
\end{verbatim}\ecode
%
Note that in the first example the return value variable \code{rv} will
hold an integer value; in the second example it will hold a string
value.  The structure lay-out for the \var{lockadata} variable is
system dependent -- therefore using the \code{flock()} call may be
better.

% Manual text and implementation by Jaap Vermeulen
\section{\module{posixfile} ---
         File-like objects with locking support}

\declaremodule{builtin}{posixfile}
  \platform{Unix}
\modulesynopsis{A file-like object with support for locking.}
\moduleauthor{Jaap Vermeulen}{}
\sectionauthor{Jaap Vermeulen}{}


\indexii{\POSIX}{file object}

\deprecated{1.5}{The locking operation that this module provides is
done better and more portably by the
\function{\refmodule{fcntl}.lockf()} call.
\withsubitem{(in module fcntl)}{\ttindex{lockf()}}}

This module implements some additional functionality over the built-in
file objects.  In particular, it implements file locking, control over
the file flags, and an easy interface to duplicate the file object.
The module defines a new file object, the posixfile object.  It
has all the standard file object methods and adds the methods
described below.  This module only works for certain flavors of
\UNIX, since it uses \function{fcntl.fcntl()} for file locking.%
\withsubitem{(in module fcntl)}{\ttindex{fcntl()}}

To instantiate a posixfile object, use the \function{open()} function
in the \module{posixfile} module.  The resulting object looks and
feels roughly the same as a standard file object.

The \module{posixfile} module defines the following constants:


\begin{datadesc}{SEEK_SET}
Offset is calculated from the start of the file.
\end{datadesc}

\begin{datadesc}{SEEK_CUR}
Offset is calculated from the current position in the file.
\end{datadesc}

\begin{datadesc}{SEEK_END}
Offset is calculated from the end of the file.
\end{datadesc}

The \module{posixfile} module defines the following functions:


\begin{funcdesc}{open}{filename\optional{, mode\optional{, bufsize}}}
 Create a new posixfile object with the given filename and mode.  The
 \var{filename}, \var{mode} and \var{bufsize} arguments are
 interpreted the same way as by the built-in \function{open()}
 function.
\end{funcdesc}

\begin{funcdesc}{fileopen}{fileobject}
 Create a new posixfile object with the given standard file object.
 The resulting object has the same filename and mode as the original
 file object.
\end{funcdesc}

The posixfile object defines the following additional methods:

\setindexsubitem{(posixfile method)}
\begin{funcdesc}{lock}{fmt, \optional{len\optional{, start\optional{, whence}}}}
 Lock the specified section of the file that the file object is
 referring to.  The format is explained
 below in a table.  The \var{len} argument specifies the length of the
 section that should be locked. The default is \code{0}. \var{start}
 specifies the starting offset of the section, where the default is
 \code{0}.  The \var{whence} argument specifies where the offset is
 relative to. It accepts one of the constants \constant{SEEK_SET},
 \constant{SEEK_CUR} or \constant{SEEK_END}.  The default is
 \constant{SEEK_SET}.  For more information about the arguments refer
 to the \manpage{fcntl}{2} manual page on your system.
\end{funcdesc}

\begin{funcdesc}{flags}{\optional{flags}}
 Set the specified flags for the file that the file object is referring
 to.  The new flags are ORed with the old flags, unless specified
 otherwise.  The format is explained below in a table.  Without
 the \var{flags} argument
 a string indicating the current flags is returned (this is
 the same as the \samp{?} modifier).  For more information about the
 flags refer to the \manpage{fcntl}{2} manual page on your system.
\end{funcdesc}

\begin{funcdesc}{dup}{}
 Duplicate the file object and the underlying file pointer and file
 descriptor.  The resulting object behaves as if it were newly
 opened.
\end{funcdesc}

\begin{funcdesc}{dup2}{fd}
 Duplicate the file object and the underlying file pointer and file
 descriptor.  The new object will have the given file descriptor.
 Otherwise the resulting object behaves as if it were newly opened.
\end{funcdesc}

\begin{funcdesc}{file}{}
 Return the standard file object that the posixfile object is based
 on.  This is sometimes necessary for functions that insist on a
 standard file object.
\end{funcdesc}

All methods raise \exception{IOError} when the request fails.

Format characters for the \method{lock()} method have the following
meaning:

\begin{tableii}{c|l}{samp}{Format}{Meaning}
  \lineii{u}{unlock the specified region}
  \lineii{r}{request a read lock for the specified section}
  \lineii{w}{request a write lock for the specified section}
\end{tableii}

In addition the following modifiers can be added to the format:

\begin{tableiii}{c|l|c}{samp}{Modifier}{Meaning}{Notes}
  \lineiii{|}{wait until the lock has been granted}{}
  \lineiii{?}{return the first lock conflicting with the requested lock, or
              \code{None} if there is no conflict.}{(1)} 
\end{tableiii}

\noindent
Note:

\begin{description}
\item[(1)] The lock returned is in the format \code{(\var{mode}, \var{len},
\var{start}, \var{whence}, \var{pid})} where \var{mode} is a character
representing the type of lock ('r' or 'w').  This modifier prevents a
request from being granted; it is for query purposes only.
\end{description}

Format characters for the \method{flags()} method have the following
meanings:

\begin{tableii}{c|l}{samp}{Format}{Meaning}
  \lineii{a}{append only flag}
  \lineii{c}{close on exec flag}
  \lineii{n}{no delay flag (also called non-blocking flag)}
  \lineii{s}{synchronization flag}
\end{tableii}

In addition the following modifiers can be added to the format:

\begin{tableiii}{c|l|c}{samp}{Modifier}{Meaning}{Notes}
  \lineiii{!}{turn the specified flags 'off', instead of the default 'on'}{(1)}
  \lineiii{=}{replace the flags, instead of the default 'OR' operation}{(1)}
  \lineiii{?}{return a string in which the characters represent the flags that
  are set.}{(2)}
\end{tableiii}

\noindent
Notes:

\begin{description}
\item[(1)] The \samp{!} and \samp{=} modifiers are mutually exclusive.

\item[(2)] This string represents the flags after they may have been altered
by the same call.
\end{description}

Examples:

\begin{verbatim}
import posixfile

file = posixfile.open('/tmp/test', 'w')
file.lock('w|')
...
file.lock('u')
file.close()
\end{verbatim}


\section{Standard module \sectcode{pdb}}
\stmodindex{pdb}
\index{debugging}

This module defines an interactive source code debugger for Python
programs.  It supports breakpoints and single stepping at the source
line level, inspection of stack frames, source code listing, and
evaluation of arbitrary Python code in the context of any stack frame.
It also supports post-mortem debugging and can be called under program
control.

The debugger is extensible --- it is actually defined as a class
\code{Pdb}.  The extension interface uses the (also undocumented)
modules \code{bdb} and \code{cmd}; it is currently undocumented.
\ttindex{Pdb}
\ttindex{bdb}
\ttindex{cmd}

A primitive windowing version of the debugger also exists --- this is
module \code{wdb}, which requires STDWIN.
\index{stdwin}
\ttindex{wdb}

Typical usage to run a program under control of the debugger is:

\begin{verbatim}
>>> import pdb
>>> import mymodule
>>> pdb.run('mymodule.test()')
(Pdb)
\end{verbatim}

Typical usage to inspect a crashed program is:

\begin{verbatim}
>>> import pdb
>>> import mymodule
>>> mymodule.test()
(crashes with a stack trace)
>>> pdb.pm()
(Pdb)
\end{verbatim}

The debugger's prompt is ``\code{(Pdb) }''.

The module defines the following functions; each enters the debugger
in a slightly different way:

\begin{funcdesc}{run}{statement\optional{\, globals\optional{\, locals}}}
Execute the \var{statement} (which should be a string) under debugger
control.  The debugger prompt appears before any code is executed; you
can set breakpoint and type \code{continue}, or you can step through
the statement using \code{step} or \code{next}.  The optional
\var{globals} and \var{locals} arguments specify the environment in
which the code is executed; by default the dictionary of the module
\code{__main__} is used.  (See the explanation of the \code{exec}
statement or the \code{eval()} built-in function.)
\end{funcdesc}

\begin{funcdesc}{runeval}{expression\optional{\, globals\optional{\, locals}}}
Evaluate the \var{expression} (which should be a string) under
debugger control.  When \code{runeval()} returns, it returns the value
of the expression.  Otherwise this function is similar to
\code{run()}.
\end{funcdesc}

\begin{funcdesc}{runcall}{function\optional{\, argument\, ...}}
Call the \var{function} (which should be a callable Python object, not
a string) with the given arguments.  When \code{runcall()} returns, it
returns the return value of the function call.  The debugger prompt
appears as soon as the function is entered.
\end{funcdesc}

\begin{funcdesc}{set_trace}{}
Enter the debugger at the calling stack frame.  This is useful to
hard-code a breakpoint at a given point in code, even if the code is
not otherwise being debugged.
\end{funcdesc}

\begin{funcdesc}{post_mortem}{traceback}
Enter post-mortem debugging of the given \var{traceback} object.
\end{funcdesc}

\begin{funcdesc}{pm}{}
Enter post-mortem debugging based on the traceback found in
\code{sys.last_traceback}.
\end{funcdesc}

\subsection{Debugger Commands}

The debugger recognizes the following commands.  Most commands can be
abbreviated to one or two letters; e.g. ``\code{h(elp)}'' means that
either ``\code{h}'' or ``\code{help}'' can be used to enter the help
command (but not ``\code{he}'' or ``\code{hel}'', nor ``\code{H}'' or
``\code{Help} or ``\code{HELP}'').  Arguments to commands must be
separated by whitespace (spaces or tabs).  Optional arguments are
enclosed in square brackets (``\code{[]}'')in the command syntax; the
square brackets must not be typed.  Alternatives in the command syntax
are separated by a vertical bar (``\code{|}'').

Entering a blank line repeats the last command entered.  Exception: if
the last command was a ``\code{list}'' command, the next 11 lines are
listed.

Commands that the debugger doesn't recognize are assumed to be Python
statements and are executed in the context of the program being
debugged.  Python statements can also be prefixed with an exclamation
point (``\code{!}'').  This is a powerful way to inspect the program
being debugged; it is even possible to change variables.  When an
exception occurs in such a statement, the exception name is printed
but the debugger's state is not changed.

\begin{description}

\item[{h(elp) [\var{command}]}]

Without argument, print the list of available commands.
With a \var{command} as argument, print help about that command.
``\code{help pdb}'' displays the full documentation file; if the
environment variable \code{PAGER} is defined, the file is piped
through that command instead.  Since the var{command} argument must be
an identifier, ``\code{help exec}'' gives help on the ``\code{!}''
command.

\item[{w(here)}]

Print a stack trace, with the most recent frame at the bottom.
An arrow indicates the current frame, which determines the
context of most commands.

\item[{d(own)}]

Move the current frame one level down in the stack trace
(to an older frame).

\item[{u(p)}]

Move the current frame one level up in the stack trace
(to a newer frame).

\item[{b(reak) [\var{lineno} \code{|} \var{function}]}]

With a \var{lineno} argument, set a break there in the current
file.  With a \var{function} argument, set a break at the entry of
that function.  Without argument, list all breaks.

\item[{cl(ear) [lineno]}]

With a \var{lineno} argument, clear that break in the current file.
Without argument, clear all breaks (but first ask confirmation).

\item[{s(tep)}]

Execute the current line, stop at the first possible occasion
(either in a function that is called or on the next line in the
current function).

\item[{n(ext)}]

Continue execution until the next line in the current function
is reached or it returns.  (The difference between \code{next} and
\code{step} is that \code{step} stops inside a called function, while
\code{next} executes called functions at full speed, only stopping at
the next line in the current function.)

\item[{r(eturn)}]

Continue execution until the current function returns.

\item[{c(ont(inue))}]

Continue execution, only stop when a breakpoint is encountered.

\item[{l(ist) [\var{first} [, \var{last}]]}]

List source code for the current file.
Without arguments, list 11 lines around the current line
or continue the previous listing.
With one argument, list 11 lines around at that line.
With two arguments, list the given range;
if the second argument is less than the first, it is a count.

\item[{a(rgs)}]

Print the argument list of the current function.

\item[{p \var{expression}}]

Evaluate the \var{expression} in the current context and print its
value.

\item[{[!] \var{statement}}]

Execute the (one-line) \var{statement} in the context of
the current stack frame.
The exclamation point can be omitted unless the first word
of the statement resembles a debugger command.
To set a global variable, you can prefix the assignment
command with a ``\code{global}'' command on the same line, e.g.:
\begin{verbatim}
(Pdb) global list_options; list_options = ['-l']
(Pdb)
\end{verbatim}

\item[{q(uit)}]

Quit from the debugger.
The program being executed is aborted.

\end{description}
			% The Python Debugger

\chapter{The Python Profiler \label{profile}}

\sectionauthor{James Roskind}{}

Copyright \copyright{} 1994, by InfoSeek Corporation, all rights reserved.
\index{InfoSeek Corporation}

Written by James Roskind.\footnote{
  Updated and converted to \LaTeX\ by Guido van Rossum.  The references to
  the old profiler are left in the text, although it no longer exists.}

Permission to use, copy, modify, and distribute this Python software
and its associated documentation for any purpose (subject to the
restriction in the following sentence) without fee is hereby granted,
provided that the above copyright notice appears in all copies, and
that both that copyright notice and this permission notice appear in
supporting documentation, and that the name of InfoSeek not be used in
advertising or publicity pertaining to distribution of the software
without specific, written prior permission.  This permission is
explicitly restricted to the copying and modification of the software
to remain in Python, compiled Python, or other languages (such as C)
wherein the modified or derived code is exclusively imported into a
Python module.

INFOSEEK CORPORATION DISCLAIMS ALL WARRANTIES WITH REGARD TO THIS
SOFTWARE, INCLUDING ALL IMPLIED WARRANTIES OF MERCHANTABILITY AND
FITNESS. IN NO EVENT SHALL INFOSEEK CORPORATION BE LIABLE FOR ANY
SPECIAL, INDIRECT OR CONSEQUENTIAL DAMAGES OR ANY DAMAGES WHATSOEVER
RESULTING FROM LOSS OF USE, DATA OR PROFITS, WHETHER IN AN ACTION OF
CONTRACT, NEGLIGENCE OR OTHER TORTIOUS ACTION, ARISING OUT OF OR IN
CONNECTION WITH THE USE OR PERFORMANCE OF THIS SOFTWARE.


The profiler was written after only programming in Python for 3 weeks.
As a result, it is probably clumsy code, but I don't know for sure yet
'cause I'm a beginner :-).  I did work hard to make the code run fast,
so that profiling would be a reasonable thing to do.  I tried not to
repeat code fragments, but I'm sure I did some stuff in really awkward
ways at times.  Please send suggestions for improvements to:
\email{jar@netscape.com}.  I won't promise \emph{any} support.  ...but
I'd appreciate the feedback.


\section{Introduction to the profiler}
\nodename{Profiler Introduction}

A \dfn{profiler} is a program that describes the run time performance
of a program, providing a variety of statistics.  This documentation
describes the profiler functionality provided in the modules
\module{profile} and \module{pstats}.  This profiler provides
\dfn{deterministic profiling} of any Python programs.  It also
provides a series of report generation tools to allow users to rapidly
examine the results of a profile operation.
\index{deterministic profiling}
\index{profiling, deterministic}


\section{How Is This Profiler Different From The Old Profiler?}
\nodename{Profiler Changes}

(This section is of historical importance only; the old profiler
discussed here was last seen in Python 1.1.)

The big changes from old profiling module are that you get more
information, and you pay less CPU time.  It's not a trade-off, it's a
trade-up.

To be specific:

\begin{description}

\item[Bugs removed:]
Local stack frame is no longer molested, execution time is now charged
to correct functions.

\item[Accuracy increased:]
Profiler execution time is no longer charged to user's code,
calibration for platform is supported, file reads are not done \emph{by}
profiler \emph{during} profiling (and charged to user's code!).

\item[Speed increased:]
Overhead CPU cost was reduced by more than a factor of two (perhaps a
factor of five), lightweight profiler module is all that must be
loaded, and the report generating module (\module{pstats}) is not needed
during profiling.

\item[Recursive functions support:]
Cumulative times in recursive functions are correctly calculated;
recursive entries are counted.

\item[Large growth in report generating UI:]
Distinct profiles runs can be added together forming a comprehensive
report; functions that import statistics take arbitrary lists of
files; sorting criteria is now based on keywords (instead of 4 integer
options); reports shows what functions were profiled as well as what
profile file was referenced; output format has been improved.

\end{description}


\section{Instant Users Manual \label{profile-instant}}

This section is provided for users that ``don't want to read the
manual.'' It provides a very brief overview, and allows a user to
rapidly perform profiling on an existing application.

To profile an application with a main entry point of \function{foo()},
you would add the following to your module:

\begin{verbatim}
import profile
profile.run('foo()')
\end{verbatim}

The above action would cause \function{foo()} to be run, and a series of
informative lines (the profile) to be printed.  The above approach is
most useful when working with the interpreter.  If you would like to
save the results of a profile into a file for later examination, you
can supply a file name as the second argument to the \function{run()}
function:

\begin{verbatim}
import profile
profile.run('foo()', 'fooprof')
\end{verbatim}

The file \file{profile.py} can also be invoked as
a script to profile another script.  For example:

\begin{verbatim}
python -m profile myscript.py
\end{verbatim}

\file{profile.py} accepts two optional arguments on the command line:

\begin{verbatim}
profile.py [-o output_file] [-s sort_order]
\end{verbatim}

\programopt{-s} only applies to standard output (\programopt{-o} is
not supplied).  Look in the \class{Stats} documentation for valid sort
values.

When you wish to review the profile, you should use the methods in the
\module{pstats} module.  Typically you would load the statistics data as
follows:

\begin{verbatim}
import pstats
p = pstats.Stats('fooprof')
\end{verbatim}

The class \class{Stats} (the above code just created an instance of
this class) has a variety of methods for manipulating and printing the
data that was just read into \code{p}.  When you ran
\function{profile.run()} above, what was printed was the result of three
method calls:

\begin{verbatim}
p.strip_dirs().sort_stats(-1).print_stats()
\end{verbatim}

The first method removed the extraneous path from all the module
names. The second method sorted all the entries according to the
standard module/line/name string that is printed (this is to comply
with the semantics of the old profiler).  The third method printed out
all the statistics.  You might try the following sort calls:

\begin{verbatim}
p.sort_stats('name')
p.print_stats()
\end{verbatim}

The first call will actually sort the list by function name, and the
second call will print out the statistics.  The following are some
interesting calls to experiment with:

\begin{verbatim}
p.sort_stats('cumulative').print_stats(10)
\end{verbatim}

This sorts the profile by cumulative time in a function, and then only
prints the ten most significant lines.  If you want to understand what
algorithms are taking time, the above line is what you would use.

If you were looking to see what functions were looping a lot, and
taking a lot of time, you would do:

\begin{verbatim}
p.sort_stats('time').print_stats(10)
\end{verbatim}

to sort according to time spent within each function, and then print
the statistics for the top ten functions.

You might also try:

\begin{verbatim}
p.sort_stats('file').print_stats('__init__')
\end{verbatim}

This will sort all the statistics by file name, and then print out
statistics for only the class init methods (since they are spelled
with \code{__init__} in them).  As one final example, you could try:

\begin{verbatim}
p.sort_stats('time', 'cum').print_stats(.5, 'init')
\end{verbatim}

This line sorts statistics with a primary key of time, and a secondary
key of cumulative time, and then prints out some of the statistics.
To be specific, the list is first culled down to 50\% (re: \samp{.5})
of its original size, then only lines containing \code{init} are
maintained, and that sub-sub-list is printed.

If you wondered what functions called the above functions, you could
now (\code{p} is still sorted according to the last criteria) do:

\begin{verbatim}
p.print_callers(.5, 'init')
\end{verbatim}

and you would get a list of callers for each of the listed functions.

If you want more functionality, you're going to have to read the
manual, or guess what the following functions do:

\begin{verbatim}
p.print_callees()
p.add('fooprof')
\end{verbatim}

Invoked as a script, the \module{pstats} module is a statistics
browser for reading and examining profile dumps.  It has a simple
line-oriented interface (implemented using \refmodule{cmd}) and
interactive help.

\section{What Is Deterministic Profiling?}
\nodename{Deterministic Profiling}

\dfn{Deterministic profiling} is meant to reflect the fact that all
\emph{function call}, \emph{function return}, and \emph{exception} events
are monitored, and precise timings are made for the intervals between
these events (during which time the user's code is executing).  In
contrast, \dfn{statistical profiling} (which is not done by this
module) randomly samples the effective instruction pointer, and
deduces where time is being spent.  The latter technique traditionally
involves less overhead (as the code does not need to be instrumented),
but provides only relative indications of where time is being spent.

In Python, since there is an interpreter active during execution, the
presence of instrumented code is not required to do deterministic
profiling.  Python automatically provides a \dfn{hook} (optional
callback) for each event.  In addition, the interpreted nature of
Python tends to add so much overhead to execution, that deterministic
profiling tends to only add small processing overhead in typical
applications.  The result is that deterministic profiling is not that
expensive, yet provides extensive run time statistics about the
execution of a Python program.

Call count statistics can be used to identify bugs in code (surprising
counts), and to identify possible inline-expansion points (high call
counts).  Internal time statistics can be used to identify ``hot
loops'' that should be carefully optimized.  Cumulative time
statistics should be used to identify high level errors in the
selection of algorithms.  Note that the unusual handling of cumulative
times in this profiler allows statistics for recursive implementations
of algorithms to be directly compared to iterative implementations.


\section{Reference Manual}

\declaremodule{standard}{profile}
\modulesynopsis{Python profiler}



The primary entry point for the profiler is the global function
\function{profile.run()}.  It is typically used to create any profile
information.  The reports are formatted and printed using methods of
the class \class{pstats.Stats}.  The following is a description of all
of these standard entry points and functions.  For a more in-depth
view of some of the code, consider reading the later section on
Profiler Extensions, which includes discussion of how to derive
``better'' profilers from the classes presented, or reading the source
code for these modules.

\begin{funcdesc}{run}{command\optional{, filename}}

This function takes a single argument that has can be passed to the
\keyword{exec} statement, and an optional file name.  In all cases this
routine attempts to \keyword{exec} its first argument, and gather profiling
statistics from the execution. If no file name is present, then this
function automatically prints a simple profiling report, sorted by the
standard name string (file/line/function-name) that is presented in
each line.  The following is a typical output from such a call:

\begin{verbatim}
      main()
      2706 function calls (2004 primitive calls) in 4.504 CPU seconds

Ordered by: standard name

ncalls  tottime  percall  cumtime  percall filename:lineno(function)
     2    0.006    0.003    0.953    0.477 pobject.py:75(save_objects)
  43/3    0.533    0.012    0.749    0.250 pobject.py:99(evaluate)
 ...
\end{verbatim}

The first line indicates that this profile was generated by the call:\\
\code{profile.run('main()')}, and hence the exec'ed string is
\code{'main()'}.  The second line indicates that 2706 calls were
monitored.  Of those calls, 2004 were \dfn{primitive}.  We define
\dfn{primitive} to mean that the call was not induced via recursion.
The next line: \code{Ordered by:\ standard name}, indicates that
the text string in the far right column was used to sort the output.
The column headings include:

\begin{description}

\item[ncalls ]
for the number of calls,

\item[tottime ]
for the total time spent in the given function (and excluding time
made in calls to sub-functions),

\item[percall ]
is the quotient of \code{tottime} divided by \code{ncalls}

\item[cumtime ]
is the total time spent in this and all subfunctions (from invocation
till exit). This figure is accurate \emph{even} for recursive
functions.

\item[percall ]
is the quotient of \code{cumtime} divided by primitive calls

\item[filename:lineno(function) ]
provides the respective data of each function

\end{description}

When there are two numbers in the first column (for example,
\samp{43/3}), then the latter is the number of primitive calls, and
the former is the actual number of calls.  Note that when the function
does not recurse, these two values are the same, and only the single
figure is printed.

\end{funcdesc}

\begin{funcdesc}{runctx}{command, globals, locals\optional{, filename}}
This function is similar to \function{profile.run()}, with added
arguments to supply the globals and locals dictionaries for the
\var{command} string.
\end{funcdesc}

Analysis of the profiler data is done using this class from the
\module{pstats} module:

% now switch modules....
% (This \stmodindex use may be hard to change ;-( )
\stmodindex{pstats}

\begin{classdesc}{Stats}{filename\optional{, \moreargs}}
This class constructor creates an instance of a ``statistics object''
from a \var{filename} (or set of filenames).  \class{Stats} objects are
manipulated by methods, in order to print useful reports.

The file selected by the above constructor must have been created by
the corresponding version of \module{profile}.  To be specific, there is
\emph{no} file compatibility guaranteed with future versions of this
profiler, and there is no compatibility with files produced by other
profilers (such as the old system profiler).

If several files are provided, all the statistics for identical
functions will be coalesced, so that an overall view of several
processes can be considered in a single report.  If additional files
need to be combined with data in an existing \class{Stats} object, the
\method{add()} method can be used.
\end{classdesc}


\subsection{The \class{Stats} Class \label{profile-stats}}

\class{Stats} objects have the following methods:

\begin{methoddesc}[Stats]{strip_dirs}{}
This method for the \class{Stats} class removes all leading path
information from file names.  It is very useful in reducing the size
of the printout to fit within (close to) 80 columns.  This method
modifies the object, and the stripped information is lost.  After
performing a strip operation, the object is considered to have its
entries in a ``random'' order, as it was just after object
initialization and loading.  If \method{strip_dirs()} causes two
function names to be indistinguishable (they are on the same
line of the same filename, and have the same function name), then the
statistics for these two entries are accumulated into a single entry.
\end{methoddesc}


\begin{methoddesc}[Stats]{add}{filename\optional{, \moreargs}}
This method of the \class{Stats} class accumulates additional
profiling information into the current profiling object.  Its
arguments should refer to filenames created by the corresponding
version of \function{profile.run()}.  Statistics for identically named
(re: file, line, name) functions are automatically accumulated into
single function statistics.
\end{methoddesc}

\begin{methoddesc}[Stats]{dump_stats}{filename}
Save the data loaded into the \class{Stats} object to a file named
\var{filename}.  The file is created if it does not exist, and is
overwritten if it already exists.  This is equivalent to the method of
the same name on the \class{profile.Profile} class.
\versionadded{2.3}
\end{methoddesc}

\begin{methoddesc}[Stats]{sort_stats}{key\optional{, \moreargs}}
This method modifies the \class{Stats} object by sorting it according
to the supplied criteria.  The argument is typically a string
identifying the basis of a sort (example: \code{'time'} or
\code{'name'}).

When more than one key is provided, then additional keys are used as
secondary criteria when there is equality in all keys selected
before them.  For example, \code{sort_stats('name', 'file')} will sort
all the entries according to their function name, and resolve all ties
(identical function names) by sorting by file name.

Abbreviations can be used for any key names, as long as the
abbreviation is unambiguous.  The following are the keys currently
defined:

\begin{tableii}{l|l}{code}{Valid Arg}{Meaning}
  \lineii{'calls'}{call count}
  \lineii{'cumulative'}{cumulative time}
  \lineii{'file'}{file name}
  \lineii{'module'}{file name}
  \lineii{'pcalls'}{primitive call count}
  \lineii{'line'}{line number}
  \lineii{'name'}{function name}
  \lineii{'nfl'}{name/file/line}
  \lineii{'stdname'}{standard name}
  \lineii{'time'}{internal time}
\end{tableii}

Note that all sorts on statistics are in descending order (placing
most time consuming items first), where as name, file, and line number
searches are in ascending order (alphabetical). The subtle
distinction between \code{'nfl'} and \code{'stdname'} is that the
standard name is a sort of the name as printed, which means that the
embedded line numbers get compared in an odd way.  For example, lines
3, 20, and 40 would (if the file names were the same) appear in the
string order 20, 3 and 40.  In contrast, \code{'nfl'} does a numeric
compare of the line numbers.  In fact, \code{sort_stats('nfl')} is the
same as \code{sort_stats('name', 'file', 'line')}.

For compatibility with the old profiler, the numeric arguments
\code{-1}, \code{0}, \code{1}, and \code{2} are permitted.  They are
interpreted as \code{'stdname'}, \code{'calls'}, \code{'time'}, and
\code{'cumulative'} respectively.  If this old style format (numeric)
is used, only one sort key (the numeric key) will be used, and
additional arguments will be silently ignored.
\end{methoddesc}


\begin{methoddesc}[Stats]{reverse_order}{}
This method for the \class{Stats} class reverses the ordering of the basic
list within the object.  This method is provided primarily for
compatibility with the old profiler.  Its utility is questionable
now that ascending vs descending order is properly selected based on
the sort key of choice.
\end{methoddesc}

\begin{methoddesc}[Stats]{print_stats}{\optional{restriction, \moreargs}}
This method for the \class{Stats} class prints out a report as described
in the \function{profile.run()} definition.

The order of the printing is based on the last \method{sort_stats()}
operation done on the object (subject to caveats in \method{add()} and
\method{strip_dirs()}).

The arguments provided (if any) can be used to limit the list down to
the significant entries.  Initially, the list is taken to be the
complete set of profiled functions.  Each restriction is either an
integer (to select a count of lines), or a decimal fraction between
0.0 and 1.0 inclusive (to select a percentage of lines), or a regular
expression (to pattern match the standard name that is printed; as of
Python 1.5b1, this uses the Perl-style regular expression syntax
defined by the \refmodule{re} module).  If several restrictions are
provided, then they are applied sequentially.  For example:

\begin{verbatim}
print_stats(.1, 'foo:')
\end{verbatim}

would first limit the printing to first 10\% of list, and then only
print functions that were part of filename \file{.*foo:}.  In
contrast, the command:

\begin{verbatim}
print_stats('foo:', .1)
\end{verbatim}

would limit the list to all functions having file names \file{.*foo:},
and then proceed to only print the first 10\% of them.
\end{methoddesc}


\begin{methoddesc}[Stats]{print_callers}{\optional{restriction, \moreargs}}
This method for the \class{Stats} class prints a list of all functions
that called each function in the profiled database.  The ordering is
identical to that provided by \method{print_stats()}, and the definition
of the restricting argument is also identical.  For convenience, a
number is shown in parentheses after each caller to show how many
times this specific call was made.  A second non-parenthesized number
is the cumulative time spent in the function at the right.
\end{methoddesc}

\begin{methoddesc}[Stats]{print_callees}{\optional{restriction, \moreargs}}
This method for the \class{Stats} class prints a list of all function
that were called by the indicated function.  Aside from this reversal
of direction of calls (re: called vs was called by), the arguments and
ordering are identical to the \method{print_callers()} method.
\end{methoddesc}

\begin{methoddesc}[Stats]{ignore}{}
\deprecated{1.5.1}{This is not needed in modern versions of
Python.\footnote{
  This was once necessary, when Python would print any unused expression
  result that was not \code{None}.  The method is still defined for
  backward compatibility.}}
\end{methoddesc}


\section{Limitations \label{profile-limits}}

One limitation has to do with accuracy of timing information.
There is a fundamental problem with deterministic profilers involving
accuracy.  The most obvious restriction is that the underlying ``clock''
is only ticking at a rate (typically) of about .001 seconds.  Hence no
measurements will be more accurate than the underlying clock.  If
enough measurements are taken, then the ``error'' will tend to average
out. Unfortunately, removing this first error induces a second source
of error.

The second problem is that it ``takes a while'' from when an event is
dispatched until the profiler's call to get the time actually
\emph{gets} the state of the clock.  Similarly, there is a certain lag
when exiting the profiler event handler from the time that the clock's
value was obtained (and then squirreled away), until the user's code
is once again executing.  As a result, functions that are called many
times, or call many functions, will typically accumulate this error.
The error that accumulates in this fashion is typically less than the
accuracy of the clock (less than one clock tick), but it
\emph{can} accumulate and become very significant.  This profiler
provides a means of calibrating itself for a given platform so that
this error can be probabilistically (on the average) removed.
After the profiler is calibrated, it will be more accurate (in a least
square sense), but it will sometimes produce negative numbers (when
call counts are exceptionally low, and the gods of probability work
against you :-). )  Do \emph{not} be alarmed by negative numbers in
the profile.  They should \emph{only} appear if you have calibrated
your profiler, and the results are actually better than without
calibration.


\section{Calibration \label{profile-calibration}}

The profiler subtracts a constant from each
event handling time to compensate for the overhead of calling the time
function, and socking away the results.  By default, the constant is 0.
The following procedure can
be used to obtain a better constant for a given platform (see discussion
in section Limitations above).

\begin{verbatim}
import profile
pr = profile.Profile()
for i in range(5):
    print pr.calibrate(10000)
\end{verbatim}

The method executes the number of Python calls given by the argument,
directly and again under the profiler, measuring the time for both.
It then computes the hidden overhead per profiler event, and returns
that as a float.  For example, on an 800 MHz Pentium running
Windows 2000, and using Python's time.clock() as the timer,
the magical number is about 12.5e-6.

The object of this exercise is to get a fairly consistent result.
If your computer is \emph{very} fast, or your timer function has poor
resolution, you might have to pass 100000, or even 1000000, to get
consistent results.

When you have a consistent answer,
there are three ways you can use it:\footnote{Prior to Python 2.2, it
  was necessary to edit the profiler source code to embed the bias as
  a literal number.  You still can, but that method is no longer
  described, because no longer needed.}

\begin{verbatim}
import profile

# 1. Apply computed bias to all Profile instances created hereafter.
profile.Profile.bias = your_computed_bias

# 2. Apply computed bias to a specific Profile instance.
pr = profile.Profile()
pr.bias = your_computed_bias

# 3. Specify computed bias in instance constructor.
pr = profile.Profile(bias=your_computed_bias)
\end{verbatim}

If you have a choice, you are better off choosing a smaller constant, and
then your results will ``less often'' show up as negative in profile
statistics.


\section{Extensions --- Deriving Better Profilers}
\nodename{Profiler Extensions}

The \class{Profile} class of module \module{profile} was written so that
derived classes could be developed to extend the profiler.  The details
are not described here, as doing this successfully requires an expert
understanding of how the \class{Profile} class works internally.  Study
the source code of module \module{profile} carefully if you want to
pursue this.

If all you want to do is change how current time is determined (for
example, to force use of wall-clock time or elapsed process time),
pass the timing function you want to the \class{Profile} class
constructor:

\begin{verbatim}
pr = profile.Profile(your_time_func)
\end{verbatim}

The resulting profiler will then call \code{your_time_func()}.
The function should return a single number, or a list of
numbers whose sum is the current time (like what \function{os.times()}
returns).  If the function returns a single time number, or the list of
returned numbers has length 2, then you will get an especially fast
version of the dispatch routine.

Be warned that you should calibrate the profiler class for the
timer function that you choose.  For most machines, a timer that
returns a lone integer value will provide the best results in terms of
low overhead during profiling.  (\function{os.times()} is
\emph{pretty} bad, as it returns a tuple of floating point values).  If
you want to substitute a better timer in the cleanest fashion,
derive a class and hardwire a replacement dispatch method that best
handles your timer call, along with the appropriate calibration
constant.
		% The Python Profiler

\chapter{Internet and WWW Services}
\nodename{Internet and WWW}
\label{www}
\index{WWW}
\index{Internet}
\index{World-Wide Web}

The modules described in this chapter provide various services to
World-Wide Web (WWW) clients and/or services, and a few modules
related to news and email.  They are all implemented in Python.  Some
of these modules require the presence of the system-dependent module
\code{sockets}\refbimodindex{socket}, which is currently only fully
supported on \UNIX{} and Windows NT.  Here is an overview:

\begin{description}

\item[cgi]
--- Common Gateway Interface, used to interpret forms in server-side
scripts.

\item[urllib]
--- Open an arbitrary object given by URL (requires sockets).

\item[httplib]
--- HTTP protocol client (requires sockets).

\item[ftplib]
--- FTP protocol client (requires sockets).

\item[gopherlib]
--- Gopher protocol client (requires sockets).

\item[nntplib]
--- NNTP protocol client (requires sockets).

\item[urlparse]
--- Parse a URL string into a tuple (addressing scheme identifier, network
location, path, parameters, query string, fragment identifier).

\item[sgmllib]
--- Only as much of an SGML parser as needed to parse HTML.

\item[htmllib]
--- A parser for HTML documents.

\item[xmllib]
--- A parser for XML documents.

\item[formatter]
--- Generic output formatter and device interface.

\item[rfc822]
--- Parse \rfc{822} style mail headers.

\item[mimetools]
--- Tools for parsing MIME style message bodies.

\item[binhex]
--- Encode and decode files in binhex4 format.

\item[uu]
--- Encode and decode files in uuencode format.

\item[binascii]
--- Tools for converting between binary and various ascii-encoded binary 
representation

\item[xdrlib]
--- The External Data Representation Standard as described in \rfc{1014},
written by Sun Microsystems, Inc. June 1987.

\item[mailcap]
--- Mailcap file handling.  See \rfc{1524}.

\item[base64]
--- Encode/decode binary files using the MIME base64 encoding.

\item[quopri]
--- Encode/decode binary files using the MIME quoted-printable encoding.

\item[SocketServer]
--- A framework for network servers.

\item[mailbox]
--- Read various mailbox formats.

\item[mimify]
--- Mimification and unmimification of mail messages.

\item[BaseHTTPServer]
--- Basic HTTP server (base class for SimpleHTTPServer and CGIHTTPServer).

\end{description}
			% Internet and WWW Services
\section{\module{cgi} ---
         Common Gateway Interface support.}
\declaremodule{standard}{cgi}

\modulesynopsis{Common Gateway Interface support, used to interpret
forms in server-side scripts.}

\indexii{WWW}{server}
\indexii{CGI}{protocol}
\indexii{HTTP}{protocol}
\indexii{MIME}{headers}
\index{URL}


Support module for Common Gateway Interface (CGI) scripts.%
\index{Common Gateway Interface}

This module defines a number of utilities for use by CGI scripts
written in Python.

\subsection{Introduction}
\nodename{cgi-intro}

A CGI script is invoked by an HTTP server, usually to process user
input submitted through an HTML \code{<FORM>} or \code{<ISINDEX>} element.

Most often, CGI scripts live in the server's special \file{cgi-bin}
directory.  The HTTP server places all sorts of information about the
request (such as the client's hostname, the requested URL, the query
string, and lots of other goodies) in the script's shell environment,
executes the script, and sends the script's output back to the client.

The script's input is connected to the client too, and sometimes the
form data is read this way; at other times the form data is passed via
the ``query string'' part of the URL.  This module is intended
to take care of the different cases and provide a simpler interface to
the Python script.  It also provides a number of utilities that help
in debugging scripts, and the latest addition is support for file
uploads from a form (if your browser supports it --- Grail 0.3 and
Netscape 2.0 do).

The output of a CGI script should consist of two sections, separated
by a blank line.  The first section contains a number of headers,
telling the client what kind of data is following.  Python code to
generate a minimal header section looks like this:

\begin{verbatim}
print "Content-Type: text/html"     # HTML is following
print                               # blank line, end of headers
\end{verbatim}

The second section is usually HTML, which allows the client software
to display nicely formatted text with header, in-line images, etc.
Here's Python code that prints a simple piece of HTML:

\begin{verbatim}
print "<TITLE>CGI script output</TITLE>"
print "<H1>This is my first CGI script</H1>"
print "Hello, world!"
\end{verbatim}

\subsection{Using the cgi module}
\nodename{Using the cgi module}

Begin by writing \samp{import cgi}.  Do not use \samp{from cgi import
*} --- the module defines all sorts of names for its own use or for
backward compatibility that you don't want in your namespace.

When you write a new script, consider adding the line:

\begin{verbatim}
import cgitb; cgitb.enable()
\end{verbatim}

This activates a special exception handler that will display detailed
reports in the Web browser if any errors occur.  If you'd rather not
show the guts of your program to users of your script, you can have
the reports saved to files instead, with a line like this:

\begin{verbatim}
import cgitb; cgitb.enable(display=0, logdir="/tmp")
\end{verbatim}

It's very helpful to use this feature during script development.
The reports produced by \refmodule{cgitb} provide information that
can save you a lot of time in tracking down bugs.  You can always
remove the \code{cgitb} line later when you have tested your script
and are confident that it works correctly.

To get at submitted form data,
it's best to use the \class{FieldStorage} class.  The other classes
defined in this module are provided mostly for backward compatibility.
Instantiate it exactly once, without arguments.  This reads the form
contents from standard input or the environment (depending on the
value of various environment variables set according to the CGI
standard).  Since it may consume standard input, it should be
instantiated only once.

The \class{FieldStorage} instance can be indexed like a Python
dictionary, and also supports the standard dictionary methods
\method{has_key()} and \method{keys()}.  The built-in \function{len()}
is also supported.  Form fields containing empty strings are ignored
and do not appear in the dictionary; to keep such values, provide
a true value for the optional \var{keep_blank_values} keyword
parameter when creating the \class{FieldStorage} instance.

For instance, the following code (which assumes that the 
\mailheader{Content-Type} header and blank line have already been
printed) checks that the fields \code{name} and \code{addr} are both
set to a non-empty string:

\begin{verbatim}
form = cgi.FieldStorage()
if not (form.has_key("name") and form.has_key("addr")):
    print "<H1>Error</H1>"
    print "Please fill in the name and addr fields."
    return
print "<p>name:", form["name"].value
print "<p>addr:", form["addr"].value
...further form processing here...
\end{verbatim}

Here the fields, accessed through \samp{form[\var{key}]}, are
themselves instances of \class{FieldStorage} (or
\class{MiniFieldStorage}, depending on the form encoding).
The \member{value} attribute of the instance yields the string value
of the field.  The \method{getvalue()} method returns this string value
directly; it also accepts an optional second argument as a default to
return if the requested key is not present.

If the submitted form data contains more than one field with the same
name, the object retrieved by \samp{form[\var{key}]} is not a
\class{FieldStorage} or \class{MiniFieldStorage}
instance but a list of such instances.  Similarly, in this situation,
\samp{form.getvalue(\var{key})} would return a list of strings.
If you expect this possibility
(when your HTML form contains multiple fields with the same name), use
the \function{getlist()} function, which always returns a list of values (so that you
do not need to special-case the single item case).  For example, this
code concatenates any number of username fields, separated by
commas:

\begin{verbatim}
value = form.getlist("username")
usernames = ",".join(value)
\end{verbatim}

If a field represents an uploaded file, accessing the value via the
\member{value} attribute or the \function{getvalue()} method reads the
entire file in memory as a string.  This may not be what you want.
You can test for an uploaded file by testing either the \member{filename}
attribute or the \member{file} attribute.  You can then read the data at
leisure from the \member{file} attribute:

\begin{verbatim}
fileitem = form["userfile"]
if fileitem.file:
    # It's an uploaded file; count lines
    linecount = 0
    while 1:
        line = fileitem.file.readline()
        if not line: break
        linecount = linecount + 1
\end{verbatim}

The file upload draft standard entertains the possibility of uploading
multiple files from one field (using a recursive
\mimetype{multipart/*} encoding).  When this occurs, the item will be
a dictionary-like \class{FieldStorage} item.  This can be determined
by testing its \member{type} attribute, which should be
\mimetype{multipart/form-data} (or perhaps another MIME type matching
\mimetype{multipart/*}).  In this case, it can be iterated over
recursively just like the top-level form object.

When a form is submitted in the ``old'' format (as the query string or
as a single data part of type
\mimetype{application/x-www-form-urlencoded}), the items will actually
be instances of the class \class{MiniFieldStorage}.  In this case, the
\member{list}, \member{file}, and \member{filename} attributes are
always \code{None}.


\subsection{Higher Level Interface}

\versionadded{2.2}  % XXX: Is this true ? 

The previous section explains how to read CGI form data using the
\class{FieldStorage} class.  This section describes a higher level
interface which was added to this class to allow one to do it in a
more readable and intuitive way.  The interface doesn't make the
techniques described in previous sections obsolete --- they are still
useful to process file uploads efficiently, for example.

The interface consists of two simple methods. Using the methods
you can process form data in a generic way, without the need to worry
whether only one or more values were posted under one name.

In the previous section, you learned to write following code anytime
you expected a user to post more than one value under one name:

\begin{verbatim}
item = form.getvalue("item")
if isinstance(item, list):
    # The user is requesting more than one item.
else:
    # The user is requesting only one item.
\end{verbatim}

This situation is common for example when a form contains a group of
multiple checkboxes with the same name:

\begin{verbatim}
<input type="checkbox" name="item" value="1" />
<input type="checkbox" name="item" value="2" />
\end{verbatim}

In most situations, however, there's only one form control with a
particular name in a form and then you expect and need only one value
associated with this name.  So you write a script containing for
example this code:

\begin{verbatim}
user = form.getvalue("user").upper()
\end{verbatim}

The problem with the code is that you should never expect that a
client will provide valid input to your scripts.  For example, if a
curious user appends another \samp{user=foo} pair to the query string,
then the script would crash, because in this situation the
\code{getvalue("user")} method call returns a list instead of a
string.  Calling the \method{toupper()} method on a list is not valid
(since lists do not have a method of this name) and results in an
\exception{AttributeError} exception.

Therefore, the appropriate way to read form data values was to always
use the code which checks whether the obtained value is a single value
or a list of values.  That's annoying and leads to less readable
scripts.

A more convenient approach is to use the methods \method{getfirst()}
and \method{getlist()} provided by this higher level interface.

\begin{methoddesc}[FieldStorage]{getfirst}{name\optional{, default}}
  This method always returns only one value associated with form field
  \var{name}.  The method returns only the first value in case that
  more values were posted under such name.  Please note that the order
  in which the values are received may vary from browser to browser
  and should not be counted on.\footnote{Note that some recent
      versions of the HTML specification do state what order the
      field values should be supplied in, but knowing whether a
      request was received from a conforming browser, or even from a
      browser at all, is tedious and error-prone.}  If no such form
  field or value exists then the method returns the value specified by
  the optional parameter \var{default}.  This parameter defaults to
  \code{None} if not specified.
\end{methoddesc}

\begin{methoddesc}[FieldStorage]{getlist}{name}
  This method always returns a list of values associated with form
  field \var{name}.  The method returns an empty list if no such form
  field or value exists for \var{name}.  It returns a list consisting
  of one item if only one such value exists.
\end{methoddesc}

Using these methods you can write nice compact code:

\begin{verbatim}
import cgi
form = cgi.FieldStorage()
user = form.getfirst("user", "").upper()    # This way it's safe.
for item in form.getlist("item"):
    do_something(item)
\end{verbatim}


\subsection{Old classes}

These classes, present in earlier versions of the \module{cgi} module,
are still supported for backward compatibility.  New applications
should use the \class{FieldStorage} class.

\class{SvFormContentDict} stores single value form content as
dictionary; it assumes each field name occurs in the form only once.

\class{FormContentDict} stores multiple value form content as a
dictionary (the form items are lists of values).  Useful if your form
contains multiple fields with the same name.

Other classes (\class{FormContent}, \class{InterpFormContentDict}) are
present for backwards compatibility with really old applications only.
If you still use these and would be inconvenienced when they
disappeared from a next version of this module, drop me a note.


\subsection{Functions}
\nodename{Functions in cgi module}

These are useful if you want more control, or if you want to employ
some of the algorithms implemented in this module in other
circumstances.

\begin{funcdesc}{parse}{fp\optional{, keep_blank_values\optional{,
                        strict_parsing}}}
  Parse a query in the environment or from a file (the file defaults
  to \code{sys.stdin}).  The \var{keep_blank_values} and
  \var{strict_parsing} parameters are passed to \function{parse_qs()}
  unchanged.
\end{funcdesc}

\begin{funcdesc}{parse_qs}{qs\optional{, keep_blank_values\optional{,
                           strict_parsing}}}
Parse a query string given as a string argument (data of type 
\mimetype{application/x-www-form-urlencoded}).  Data are
returned as a dictionary.  The dictionary keys are the unique query
variable names and the values are lists of values for each name.

The optional argument \var{keep_blank_values} is
a flag indicating whether blank values in
URL encoded queries should be treated as blank strings.  
A true value indicates that blanks should be retained as 
blank strings.  The default false value indicates that
blank values are to be ignored and treated as if they were
not included.

The optional argument \var{strict_parsing} is a flag indicating what
to do with parsing errors.  If false (the default), errors
are silently ignored.  If true, errors raise a ValueError
exception.

Use the \function{\refmodule{urllib}.urlencode()} function to convert
such dictionaries into query strings.

\end{funcdesc}

\begin{funcdesc}{parse_qsl}{qs\optional{, keep_blank_values\optional{,
                            strict_parsing}}}
Parse a query string given as a string argument (data of type 
\mimetype{application/x-www-form-urlencoded}).  Data are
returned as a list of name, value pairs.

The optional argument \var{keep_blank_values} is
a flag indicating whether blank values in
URL encoded queries should be treated as blank strings.  
A true value indicates that blanks should be retained as 
blank strings.  The default false value indicates that
blank values are to be ignored and treated as if they were
not included.

The optional argument \var{strict_parsing} is a flag indicating what
to do with parsing errors.  If false (the default), errors
are silently ignored.  If true, errors raise a ValueError
exception.

Use the \function{\refmodule{urllib}.urlencode()} function to convert
such lists of pairs into query strings.
\end{funcdesc}

\begin{funcdesc}{parse_multipart}{fp, pdict}
Parse input of type \mimetype{multipart/form-data} (for 
file uploads).  Arguments are \var{fp} for the input file and
\var{pdict} for a dictionary containing other parameters in
the \mailheader{Content-Type} header.

Returns a dictionary just like \function{parse_qs()} keys are the
field names, each value is a list of values for that field.  This is
easy to use but not much good if you are expecting megabytes to be
uploaded --- in that case, use the \class{FieldStorage} class instead
which is much more flexible.

Note that this does not parse nested multipart parts --- use
\class{FieldStorage} for that.
\end{funcdesc}

\begin{funcdesc}{parse_header}{string}
Parse a MIME header (such as \mailheader{Content-Type}) into a main
value and a dictionary of parameters.
\end{funcdesc}

\begin{funcdesc}{test}{}
Robust test CGI script, usable as main program.
Writes minimal HTTP headers and formats all information provided to
the script in HTML form.
\end{funcdesc}

\begin{funcdesc}{print_environ}{}
Format the shell environment in HTML.
\end{funcdesc}

\begin{funcdesc}{print_form}{form}
Format a form in HTML.
\end{funcdesc}

\begin{funcdesc}{print_directory}{}
Format the current directory in HTML.
\end{funcdesc}

\begin{funcdesc}{print_environ_usage}{}
Print a list of useful (used by CGI) environment variables in
HTML.
\end{funcdesc}

\begin{funcdesc}{escape}{s\optional{, quote}}
Convert the characters
\character{\&}, \character{<} and \character{>} in string \var{s} to
HTML-safe sequences.  Use this if you need to display text that might
contain such characters in HTML.  If the optional flag \var{quote} is
true, the double-quote character (\character{"}) is also translated;
this helps for inclusion in an HTML attribute value, as in \code{<A
HREF="...">}.  If the value to be quoted might include single- or
double-quote characters, or both, consider using the
\function{quoteattr()} function in the \refmodule{xml.sax.saxutils}
module instead.
\end{funcdesc}


\subsection{Caring about security \label{cgi-security}}

\indexii{CGI}{security}

There's one important rule: if you invoke an external program (via the
\function{os.system()} or \function{os.popen()} functions. or others
with similar functionality), make very sure you don't pass arbitrary
strings received from the client to the shell.  This is a well-known
security hole whereby clever hackers anywhere on the Web can exploit a
gullible CGI script to invoke arbitrary shell commands.  Even parts of
the URL or field names cannot be trusted, since the request doesn't
have to come from your form!

To be on the safe side, if you must pass a string gotten from a form
to a shell command, you should make sure the string contains only
alphanumeric characters, dashes, underscores, and periods.


\subsection{Installing your CGI script on a \UNIX\ system}

Read the documentation for your HTTP server and check with your local
system administrator to find the directory where CGI scripts should be
installed; usually this is in a directory \file{cgi-bin} in the server tree.

Make sure that your script is readable and executable by ``others''; the
\UNIX{} file mode should be \code{0755} octal (use \samp{chmod 0755
\var{filename}}).  Make sure that the first line of the script contains
\code{\#!} starting in column 1 followed by the pathname of the Python
interpreter, for instance:

\begin{verbatim}
#!/usr/local/bin/python
\end{verbatim}

Make sure the Python interpreter exists and is executable by ``others''.

Make sure that any files your script needs to read or write are
readable or writable, respectively, by ``others'' --- their mode
should be \code{0644} for readable and \code{0666} for writable.  This
is because, for security reasons, the HTTP server executes your script
as user ``nobody'', without any special privileges.  It can only read
(write, execute) files that everybody can read (write, execute).  The
current directory at execution time is also different (it is usually
the server's cgi-bin directory) and the set of environment variables
is also different from what you get when you log in.  In particular, don't
count on the shell's search path for executables (\envvar{PATH}) or
the Python module search path (\envvar{PYTHONPATH}) to be set to
anything interesting.

If you need to load modules from a directory which is not on Python's
default module search path, you can change the path in your script,
before importing other modules.  For example:

\begin{verbatim}
import sys
sys.path.insert(0, "/usr/home/joe/lib/python")
sys.path.insert(0, "/usr/local/lib/python")
\end{verbatim}

(This way, the directory inserted last will be searched first!)

Instructions for non-\UNIX{} systems will vary; check your HTTP server's
documentation (it will usually have a section on CGI scripts).


\subsection{Testing your CGI script}

Unfortunately, a CGI script will generally not run when you try it
from the command line, and a script that works perfectly from the
command line may fail mysteriously when run from the server.  There's
one reason why you should still test your script from the command
line: if it contains a syntax error, the Python interpreter won't
execute it at all, and the HTTP server will most likely send a cryptic
error to the client.

Assuming your script has no syntax errors, yet it does not work, you
have no choice but to read the next section.


\subsection{Debugging CGI scripts} \indexii{CGI}{debugging}

First of all, check for trivial installation errors --- reading the
section above on installing your CGI script carefully can save you a
lot of time.  If you wonder whether you have understood the
installation procedure correctly, try installing a copy of this module
file (\file{cgi.py}) as a CGI script.  When invoked as a script, the file
will dump its environment and the contents of the form in HTML form.
Give it the right mode etc, and send it a request.  If it's installed
in the standard \file{cgi-bin} directory, it should be possible to send it a
request by entering a URL into your browser of the form:

\begin{verbatim}
http://yourhostname/cgi-bin/cgi.py?name=Joe+Blow&addr=At+Home
\end{verbatim}

If this gives an error of type 404, the server cannot find the script
-- perhaps you need to install it in a different directory.  If it
gives another error, there's an installation problem that
you should fix before trying to go any further.  If you get a nicely
formatted listing of the environment and form content (in this
example, the fields should be listed as ``addr'' with value ``At Home''
and ``name'' with value ``Joe Blow''), the \file{cgi.py} script has been
installed correctly.  If you follow the same procedure for your own
script, you should now be able to debug it.

The next step could be to call the \module{cgi} module's
\function{test()} function from your script: replace its main code
with the single statement

\begin{verbatim}
cgi.test()
\end{verbatim}

This should produce the same results as those gotten from installing
the \file{cgi.py} file itself.

When an ordinary Python script raises an unhandled exception (for
whatever reason: of a typo in a module name, a file that can't be
opened, etc.), the Python interpreter prints a nice traceback and
exits.  While the Python interpreter will still do this when your CGI
script raises an exception, most likely the traceback will end up in
one of the HTTP server's log files, or be discarded altogether.

Fortunately, once you have managed to get your script to execute
\emph{some} code, you can easily send tracebacks to the Web browser
using the \refmodule{cgitb} module.  If you haven't done so already,
just add the line:

\begin{verbatim}
import cgitb; cgitb.enable()
\end{verbatim}

to the top of your script.  Then try running it again; when a
problem occurs, you should see a detailed report that will
likely make apparent the cause of the crash.

If you suspect that there may be a problem in importing the
\refmodule{cgitb} module, you can use an even more robust approach
(which only uses built-in modules):

\begin{verbatim}
import sys
sys.stderr = sys.stdout
print "Content-Type: text/plain"
print
...your code here...
\end{verbatim}

This relies on the Python interpreter to print the traceback.  The
content type of the output is set to plain text, which disables all
HTML processing.  If your script works, the raw HTML will be displayed
by your client.  If it raises an exception, most likely after the
first two lines have been printed, a traceback will be displayed.
Because no HTML interpretation is going on, the traceback will be
readable.


\subsection{Common problems and solutions}

\begin{itemize}
\item Most HTTP servers buffer the output from CGI scripts until the
script is completed.  This means that it is not possible to display a
progress report on the client's display while the script is running.

\item Check the installation instructions above.

\item Check the HTTP server's log files.  (\samp{tail -f logfile} in a
separate window may be useful!)

\item Always check a script for syntax errors first, by doing something
like \samp{python script.py}.

\item If your script does not have any syntax errors, try adding
\samp{import cgitb; cgitb.enable()} to the top of the script.

\item When invoking external programs, make sure they can be found.
Usually, this means using absolute path names --- \envvar{PATH} is
usually not set to a very useful value in a CGI script.

\item When reading or writing external files, make sure they can be read
or written by the userid under which your CGI script will be running:
this is typically the userid under which the web server is running, or some
explicitly specified userid for a web server's \samp{suexec} feature.

\item Don't try to give a CGI script a set-uid mode.  This doesn't work on
most systems, and is a security liability as well.
\end{itemize}


\section{\module{urllib} ---
         Open arbitrary resources by URL}

\declaremodule{standard}{urllib}
\modulesynopsis{Open an arbitrary network resource by URL (requires sockets).}

\index{WWW}
\index{World-Wide Web}
\index{URL}


This module provides a high-level interface for fetching data across
the World-Wide Web.  In particular, the \function{urlopen()} function
is similar to the built-in function \function{open()}, but accepts
Universal Resource Locators (URLs) instead of filenames.  Some
restrictions apply --- it can only open URLs for reading, and no seek
operations are available.

It defines the following public functions:

\begin{funcdesc}{urlopen}{url\optional{, data}}
Open a network object denoted by a URL for reading.  If the URL does
not have a scheme identifier, or if it has \file{file:} as its scheme
identifier, this opens a local file; otherwise it opens a socket to a
server somewhere on the network.  If the connection cannot be made, or
if the server returns an error code, the \exception{IOError} exception
is raised.  If all went well, a file-like object is returned.  This
supports the following methods: \method{read()}, \method{readline()},
\method{readlines()}, \method{fileno()}, \method{close()},
\method{info()} and \method{geturl()}.

Except for the \method{info()} and \method{geturl()} methods,
these methods have the same interface as for
file objects --- see section \ref{bltin-file-objects} in this
manual.  (It is not a built-in file object, however, so it can't be
used at those few places where a true built-in file object is
required.)

The \method{info()} method returns an instance of the class
\class{mimetools.Message} containing meta-information associated
with the URL.  When the method is HTTP, these headers are those
returned by the server at the head of the retrieved HTML page
(including Content-Length and Content-Type).  When the method is FTP,
a Content-Length header will be present if (as is now usual) the
server passed back a file length in response to the FTP retrieval
request.  When the method is local-file, returned headers will include
a Date representing the file's last-modified time, a Content-Length
giving file size, and a Content-Type containing a guess at the file's
type. See also the description of the
\refmodule{mimetools}\refstmodindex{mimetools} module.

The \method{geturl()} method returns the real URL of the page.  In
some cases, the HTTP server redirects a client to another URL.  The
\function{urlopen()} function handles this transparently, but in some
cases the caller needs to know which URL the client was redirected
to.  The \method{geturl()} method can be used to get at this
redirected URL.

If the \var{url} uses the \file{http:} scheme identifier, the optional
\var{data} argument may be given to specify a \code{POST} request
(normally the request type is \code{GET}).  The \var{data} argument
must in standard \file{application/x-www-form-urlencoded} format;
see the \function{urlencode()} function below.

The \function{urlopen()} function works transparently with proxies
which do not require authentication.  In a \UNIX{} or Windows
environment, set the \envvar{http_proxy}, \envvar{ftp_proxy} or
\envvar{gopher_proxy} environment variables to a URL that identifies
the proxy server before starting the Python interpreter.  For example
(the \character{\%} is the command prompt):

\begin{verbatim}
% http_proxy="http://www.someproxy.com:3128"
% export http_proxy
% python
...
\end{verbatim}

In a Macintosh environment, \function{urlopen()} will retrieve proxy
information from Internet\index{Internet Config} Config.

Proxies which require authentication for use are not currently
supported; this is considered an implementation limitation.
\end{funcdesc}

\begin{funcdesc}{urlretrieve}{url\optional{, filename\optional{, hook}}}
Copy a network object denoted by a URL to a local file, if necessary.
If the URL points to a local file, or a valid cached copy of the
object exists, the object is not copied.  Return a tuple
\code{(\var{filename}, \var{headers})} where \var{filename} is the
local file name under which the object can be found, and \var{headers}
is either \code{None} (for a local object) or whatever the
\method{info()} method of the object returned by \function{urlopen()}
returned (for a remote object, possibly cached).  Exceptions are the
same as for \function{urlopen()}.

The second argument, if present, specifies the file location to copy
to (if absent, the location will be a tempfile with a generated name).
The third argument, if present, is a hook function that will be called
once on establishment of the network connection and once after each
block read thereafter.  The hook will be passed three arguments; a
count of blocks transferred so far, a block size in bytes, and the
total size of the file.  The third argument may be \code{-1} on older
FTP servers which do not return a file size in response to a retrieval 
request.

If the \var{url} uses the \file{http:} scheme identifier, the optional
\var{data} argument may be given to specify a \code{POST} request
(normally the request type is \code{GET}).  The \var{data} argument
must in standard \file{application/x-www-form-urlencoded} format;
see the \function{urlencode()} function below.
\end{funcdesc}

\begin{funcdesc}{urlcleanup}{}
Clear the cache that may have been built up by previous calls to
\function{urlretrieve()}.
\end{funcdesc}

\begin{funcdesc}{quote}{string\optional{, safe}}
Replace special characters in \var{string} using the \samp{\%xx} escape.
Letters, digits, and the characters \character{_,.-} are never quoted.
The optional \var{safe} parameter specifies additional characters
that should not be quoted --- its default value is \code{'/'}.

Example: \code{quote('/\~{}connolly/')} yields \code{'/\%7econnolly/'}.
\end{funcdesc}

\begin{funcdesc}{quote_plus}{string\optional{, safe}}
Like \function{quote()}, but also replaces spaces by plus signs, as
required for quoting HTML form values.  Plus signs in the original
string are escaped unless they are included in \var{safe}.
\end{funcdesc}

\begin{funcdesc}{unquote}{string}
Replace \samp{\%xx} escapes by their single-character equivalent.

Example: \code{unquote('/\%7Econnolly/')} yields \code{'/\~{}connolly/'}.
\end{funcdesc}

\begin{funcdesc}{unquote_plus}{string}
Like \function{unquote()}, but also replaces plus signs by spaces, as
required for unquoting HTML form values.
\end{funcdesc}

\begin{funcdesc}{urlencode}{dict}
Convert a dictionary to a ``url-encoded'' string, suitable to pass to
\function{urlopen()} above as the optional \var{data} argument.  This
is useful to pass a dictionary of form fields to a \code{POST}
request.  The resulting string is a series of
\code{\var{key}=\var{value}} pairs separated by \character{\&}
characters, where both \var{key} and \var{value} are quoted using
\function{quote_plus()} above.
\end{funcdesc}

The public functions \function{urlopen()} and
\function{urlretrieve()} create an instance of the
\class{FancyURLopener} class and use it to perform their requested
actions.  To override this functionality, programmers can create a
subclass of \class{URLopener} or \class{FancyURLopener}, then assign
that an instance of that class to the
\code{urllib._urlopener} variable before calling the desired function.
For example, applications may want to specify a different
\code{user-agent} header than \class{URLopener} defines.  This can be
accomplished with the following code:

\begin{verbatim}
class AppURLopener(urllib.FancyURLopener):
    def __init__(self, *args):
        self.version = "App/1.7"
        apply(urllib.FancyURLopener.__init__, (self,) + args)

urllib._urlopener = AppURLopener()
\end{verbatim}

\begin{classdesc}{URLopener}{\optional{proxies\optional{, **x509}}}
Base class for opening and reading URLs.  Unless you need to support
opening objects using schemes other than \file{http:}, \file{ftp:},
\file{gopher:} or \file{file:}, you probably want to use
\class{FancyURLopener}.

By default, the \class{URLopener} class sends a
\code{user-agent} header of \samp{urllib/\var{VVV}}, where
\var{VVV} is the \module{urllib} version number.  Applications can
define their own \code{user-agent} header by subclassing
\class{URLopener} or \class{FancyURLopener} and setting the instance
attribute \member{version} to an appropriate string value before the
\method{open()} method is called.

Additional keyword parameters, collected in \var{x509}, are used for
authentication with the \file{https:} scheme.  The keywords
\var{key_file} and \var{cert_file} are supported; both are needed to
actually retrieve a resource at an \file{https:} URL.
\end{classdesc}

\begin{classdesc}{FancyURLopener}{...}
\class{FancyURLopener} subclasses \class{URLopener} providing default
handling for the following HTTP response codes: 301, 302 or 401.  For
301 and 302 response codes, the \code{location} header is used to
fetch the actual URL.  For 401 response codes (authentication
required), basic HTTP authentication is performed.

The parameters to the constructor are the same as those for
\class{URLopener}.
\end{classdesc}

Restrictions:

\begin{itemize}

\item
Currently, only the following protocols are supported: HTTP, (versions
0.9 and 1.0), Gopher (but not Gopher-+), FTP, and local files.
\indexii{HTTP}{protocol}
\indexii{Gopher}{protocol}
\indexii{FTP}{protocol}

\item
The caching feature of \function{urlretrieve()} has been disabled
until I find the time to hack proper processing of Expiration time
headers.

\item
There should be a function to query whether a particular URL is in
the cache.

\item
For backward compatibility, if a URL appears to point to a local file
but the file can't be opened, the URL is re-interpreted using the FTP
protocol.  This can sometimes cause confusing error messages.

\item
The \function{urlopen()} and \function{urlretrieve()} functions can
cause arbitrarily long delays while waiting for a network connection
to be set up.  This means that it is difficult to build an interactive
web client using these functions without using threads.

\item
The data returned by \function{urlopen()} or \function{urlretrieve()}
is the raw data returned by the server.  This may be binary data
(e.g. an image), plain text or (for example) HTML\index{HTML}.  The
HTTP\indexii{HTTP}{protocol} protocol provides type information in the
reply header, which can be inspected by looking at the
\code{content-type} header.  For the Gopher\indexii{Gopher}{protocol}
protocol, type information is encoded in the URL; there is currently
no easy way to extract it.  If the returned data is HTML, you can use
the module \refmodule{htmllib}\refstmodindex{htmllib} to parse it.

\item
This module does not support the use of proxies which require
authentication.  This may be implemented in the future.

\item
Although the \module{urllib} module contains (undocumented) routines
to parse and unparse URL strings, the recommended interface for URL
manipulation is in module \refmodule{urlparse}\refstmodindex{urlparse}.

\end{itemize}


\subsection{URLopener Objects \label{urlopener-objs}}
\sectionauthor{Skip Montanaro}{skip@mojam.com}

\class{URLopener} and \class{FancyURLopener} objects have the
following attributes.

\begin{methoddesc}[URLopener]{open}{fullurl\optional{, data}}
Open \var{fullurl} using the appropriate protocol.  This method sets 
up cache and proxy information, then calls the appropriate open method with
its input arguments.  If the scheme is not recognized,
\method{open_unknown()} is called.  The \var{data} argument 
has the same meaning as the \var{data} argument of \function{urlopen()}.
\end{methoddesc}

\begin{methoddesc}[URLopener]{open_unknown}{fullurl\optional{, data}}
Overridable interface to open unknown URL types.
\end{methoddesc}

\begin{methoddesc}[URLopener]{retrieve}{url\optional{,
                                        filename\optional{,
                                        reporthook\optional{, data}}}}
Retrieves the contents of \var{url} and places it in \var{filename}.  The
return value is a tuple consisting of a local filename and either a
\class{mimetools.Message} object containing the response headers (for remote
URLs) or None (for local URLs).  The caller must then open and read the
contents of \var{filename}.  If \var{filename} is not given and the URL
refers to a local file, the input filename is returned.  If the URL is
non-local and \var{filename} is not given, the filename is the output of
\function{tempfile.mktemp()} with a suffix that matches the suffix of the last
path component of the input URL.  If \var{reporthook} is given, it must be
a function accepting three numeric parameters.  It will be called after each
chunk of data is read from the network.  \var{reporthook} is ignored for
local URLs.

If the \var{url} uses the \file{http:} scheme identifier, the optional
\var{data} argument may be given to specify a \code{POST} request
(normally the request type is \code{GET}).  The \var{data} argument
must in standard \file{application/x-www-form-urlencoded} format;
see the \function{urlencode()} function below.
\end{methoddesc}

\begin{memberdesc}[URLopener]{version}
Variable that specifies the user agent of the opener object.  To get
\refmodule{urllib} to tell servers that it is a particular user agent,
set this in a subclass as a class variable or in the constructor
before calling the base constructor.
\end{memberdesc}


\subsection{Examples}
\nodename{Urllib Examples}

Here is an example session that uses the \samp{GET} method to retrieve
a URL containing parameters:

\begin{verbatim}
>>> import urllib
>>> params = urllib.urlencode({'spam': 1, 'eggs': 2, 'bacon': 0})
>>> f = urllib.urlopen("http://www.musi-cal.com/cgi-bin/query?%s" % params)
>>> print f.read()
\end{verbatim}

The following example uses the \samp{POST} method instead:

\begin{verbatim}
>>> import urllib
>>> params = urllib.urlencode({'spam': 1, 'eggs': 2, 'bacon': 0})
>>> f = urllib.urlopen("http://www.musi-cal.com/cgi-bin/query", params)
>>> print f.read()
\end{verbatim}

\section{\module{httplib} ---
         HTTP protocol client}

\declaremodule{standard}{httplib}
\modulesynopsis{HTTP and HTTPS protocol client (requires sockets).}

\indexii{HTTP}{protocol}
\index{HTTP!\module{httplib} (standard module)}

This module defines classes which implement the client side of the
HTTP and HTTPS protocols.  It is normally not used directly --- the
module \refmodule{urllib}\refstmodindex{urllib} uses it to handle URLs
that use HTTP and HTTPS.  \note{HTTPS support is only
available if the \refmodule{socket} module was compiled with SSL
support.}

The constants defined in this module are:

\begin{datadesc}{HTTP_PORT}
  The default port for the HTTP protocol (always \code{80}).
\end{datadesc}

\begin{datadesc}{HTTPS_PORT}
  The default port for the HTTPS protocol (always \code{443}).
\end{datadesc}

The module provides the following classes:

\begin{classdesc}{HTTPConnection}{host\optional{, port}}
An \class{HTTPConnection} instance represents one transaction with an HTTP
server.  It should be instantiated passing it a host and optional port number.
If no port number is passed, the port is extracted from the host string if it
has the form \code{\var{host}:\var{port}}, else the default HTTP port (80) is
used.  For example, the following calls all create instances that connect to
the server at the same host and port:

\begin{verbatim}
>>> h1 = httplib.HTTPConnection('www.cwi.nl')
>>> h2 = httplib.HTTPConnection('www.cwi.nl:80')
>>> h3 = httplib.HTTPConnection('www.cwi.nl', 80)
\end{verbatim}
\end{classdesc}

\begin{classdesc}{HTTPSConnection}{host\optional{, port}}
A subclass of \class{HTTPConnection} that uses SSL for communication with
secure servers.  Default port is \code{443}.
\end{classdesc}

The following exceptions are raised as appropriate:

\begin{excdesc}{HTTPException}
The base class of the other exceptions in this module.  It is a
subclass of \exception{Exception}.
\end{excdesc}

\begin{excdesc}{NotConnected}
A subclass of \exception{HTTPException}.
\end{excdesc}

\begin{excdesc}{InvalidURL}
A subclass of \exception{HTTPException}, raised if a port is given and is
either non-numeric or empty.
\end{excdesc}

\begin{excdesc}{UnknownProtocol}
A subclass of \exception{HTTPException}.
\end{excdesc}

\begin{excdesc}{UnknownTransferEncoding}
A subclass of \exception{HTTPException}.
\end{excdesc}

\begin{excdesc}{IllegalKeywordArgument}
A subclass of \exception{HTTPException}.
\end{excdesc}

\begin{excdesc}{UnimplementedFileMode}
A subclass of \exception{HTTPException}.
\end{excdesc}

\begin{excdesc}{IncompleteRead}
A subclass of \exception{HTTPException}.
\end{excdesc}

\begin{excdesc}{ImproperConnectionState}
A subclass of \exception{HTTPException}.
\end{excdesc}

\begin{excdesc}{CannotSendRequest}
A subclass of \exception{ImproperConnectionState}.
\end{excdesc}

\begin{excdesc}{CannotSendHeader}
A subclass of \exception{ImproperConnectionState}.
\end{excdesc}

\begin{excdesc}{ResponseNotReady}
A subclass of \exception{ImproperConnectionState}.
\end{excdesc}

\begin{excdesc}{BadStatusLine}
A subclass of \exception{HTTPException}.  Raised if a server responds with a
HTTP status code that we don't understand.
\end{excdesc}


\subsection{HTTPConnection Objects \label{httpconnection-objects}}

\class{HTTPConnection} instances have the following methods:

\begin{methoddesc}{request}{method, url\optional{, body\optional{, headers}}}
This will send a request to the server using the HTTP request method
\var{method} and the selector \var{url}.  If the \var{body} argument is
present, it should be a string of data to send after the headers are finished.
The header Content-Length is automatically set to the correct value.
The \var{headers} argument should be a mapping of extra HTTP headers to send
with the request.
\end{methoddesc}

\begin{methoddesc}{getresponse}{}
Should be called after a request is sent to get the response from the server.
Returns an \class{HTTPResponse} instance.
\end{methoddesc}

\begin{methoddesc}{set_debuglevel}{level}
Set the debugging level (the amount of debugging output printed).
The default debug level is \code{0}, meaning no debugging output is
printed.
\end{methoddesc}

\begin{methoddesc}{connect}{}
Connect to the server specified when the object was created.
\end{methoddesc}

\begin{methoddesc}{close}{}
Close the connection to the server.
\end{methoddesc}

\begin{methoddesc}{send}{data}
Send data to the server.  This should be used directly only after the
\method{endheaders()} method has been called and before
\method{getreply()} has been called.
\end{methoddesc}

\begin{methoddesc}{putrequest}{request, selector}
This should be the first call after the connection to the server has
been made.  It sends a line to the server consisting of the
\var{request} string, the \var{selector} string, and the HTTP version
(\code{HTTP/1.1}).
\end{methoddesc}

\begin{methoddesc}{putheader}{header, argument\optional{, ...}}
Send an \rfc{822}-style header to the server.  It sends a line to the
server consisting of the header, a colon and a space, and the first
argument.  If more arguments are given, continuation lines are sent,
each consisting of a tab and an argument.
\end{methoddesc}

\begin{methoddesc}{endheaders}{}
Send a blank line to the server, signalling the end of the headers.
\end{methoddesc}


\subsection{HTTPResponse Objects \label{httpresponse-objects}}

\class{HTTPResponse} instances have the following methods and attributes:

\begin{methoddesc}{read}{}
Reads and returns the response body.
\end{methoddesc}

\begin{methoddesc}{getheader}{name\optional{, default}}
Get the contents of the header \var{name}, or \var{default} if there is no
matching header.
\end{methoddesc}

\begin{datadesc}{msg}
  A \class{mimetools.Message} instance containing the response headers.
\end{datadesc}

\begin{datadesc}{version}
  HTTP protocol version used by server.  10 for HTTP/1.0, 11 for HTTP/1.1.
\end{datadesc}

\begin{datadesc}{status}
  Status code returned by server.
\end{datadesc}

\begin{datadesc}{reason}
  Reason phrase returned by server.
\end{datadesc}


\subsection{Examples \label{httplib-examples}}

Here is an example session that uses the \samp{GET} method:

\begin{verbatim}
>>> import httplib
>>> conn = httplib.HTTPConnection("www.python.org")
>>> conn.request("GET", "/index.html")
>>> r1 = conn.getresponse()
>>> print r1.status, r1.reason
200 OK
>>> data1 = r1.read()
>>> conn.request("GET", "/parrot.spam")
>>> r2 = conn.getresponse()
>>> print r2.status, r2.reason
404 Not Found
>>> data2 = r2.read()
>>> conn.close()
\end{verbatim}

Here is an example session that shows how to \samp{POST} requests:

\begin{verbatim}
>>> import httplib, urllib
>>> params = urllib.urlencode({'spam': 1, 'eggs': 2, 'bacon': 0})
>>> headers = {"Content-type": "application/x-www-form-urlencoded",
...            "Accept": "text/plain"}
>>> conn = httplib.HTTPConnection("musi-cal.mojam.com:80")
>>> conn.request("POST", "/cgi-bin/query", params, headers)
>>> response = conn.getresponse()
>>> print response.status, response.reason
200 OK
>>> data = response.read()
>>> conn.close()
\end{verbatim}

\section{Built-in module \sectcode{ftplib}}
\stmodindex{ftplib}

\renewcommand{\indexsubitem}{(in module ftplib)}

To be provided.

\section{Built-in module \sectcode{gopherlib}}
\stmodindex{gopherlib}

\renewcommand{\indexsubitem}{(in module gopherlib)}

To be provided.

\section{Standard Module \sectcode{nntplib}}
\label{module-nntplib}
\stmodindex{nntplib}

\renewcommand{\indexsubitem}{(in module nntplib)}

This module defines the class \code{NNTP} which implements the client
side of the NNTP protocol.  It can be used to implement a news reader
or poster, or automated news processors.  For more information on NNTP
(Network News Transfer Protocol), see Internet RFC 977.

Here are two small examples of how it can be used.  To list some
statistics about a newsgroup and print the subjects of the last 10
articles:

\bcode\begin{verbatim}
>>> s = NNTP('news.cwi.nl')
>>> resp, count, first, last, name = s.group('comp.lang.python')
>>> print 'Group', name, 'has', count, 'articles, range', first, 'to', last
Group comp.lang.python has 59 articles, range 3742 to 3803
>>> resp, subs = s.xhdr('subject', first + '-' + last)
>>> for id, sub in subs[-10:]: print id, sub
... 
3792 Re: Removing elements from a list while iterating...
3793 Re: Who likes Info files?
3794 Emacs and doc strings
3795 a few questions about the Mac implementation
3796 Re: executable python scripts
3797 Re: executable python scripts
3798 Re: a few questions about the Mac implementation 
3799 Re: PROPOSAL: A Generic Python Object Interface for Python C Modules
3802 Re: executable python scripts 
3803 Re: POSIX wait and SIGCHLD
>>> s.quit()
'205 news.cwi.nl closing connection.  Goodbye.'
>>> 
\end{verbatim}\ecode

To post an article from a file (this assumes that the article has
valid headers):

\bcode\begin{verbatim}
>>> s = NNTP('news.cwi.nl')
>>> f = open('/tmp/article')
>>> s.post(f)
'240 Article posted successfully.'
>>> s.quit()
'205 news.cwi.nl closing connection.  Goodbye.'
>>> 
\end{verbatim}\ecode
%
The module itself defines the following items:

\begin{funcdesc}{NNTP}{host\optional{\, port}}
Return a new instance of the \code{NNTP} class, representing a
connection to the NNTP server running on host \var{host}, listening at
port \var{port}.  The default \var{port} is 119.
\end{funcdesc}

\begin{excdesc}{error_reply}
Exception raised when an unexpected reply is received from the server.
\end{excdesc}

\begin{excdesc}{error_temp}
Exception raised when an error code in the range 400--499 is received.
\end{excdesc}

\begin{excdesc}{error_perm}
Exception raised when an error code in the range 500--599 is received.
\end{excdesc}

\begin{excdesc}{error_proto}
Exception raised when a reply is received from the server that does
not begin with a digit in the range 1--5.
\end{excdesc}

\subsection{NNTP Objects}

NNTP instances have the following methods.  The \var{response} that is
returned as the first item in the return tuple of almost all methods
is the server's response: a string beginning with a three-digit code.
If the server's response indicates an error, the method raises one of
the above exceptions.

\renewcommand{\indexsubitem}{(NNTP object method)}

\begin{funcdesc}{getwelcome}{}
Return the welcome message sent by the server in reply to the initial
connection.  (This message sometimes contains disclaimers or help
information that may be relevant to the user.)
\end{funcdesc}

\begin{funcdesc}{set_debuglevel}{level}
Set the instance's debugging level.  This controls the amount of
debugging output printed.  The default, 0, produces no debugging
output.  A value of 1 produces a moderate amount of debugging output,
generally a single line per request or response.  A value of 2 or
higher produces the maximum amount of debugging output, logging each
line sent and received on the connection (including message text).
\end{funcdesc}

\begin{funcdesc}{newgroups}{date\, time}
Send a \samp{NEWGROUPS} command.  The \var{date} argument should be a
string of the form \code{"\var{yy}\var{mm}\var{dd}"} indicating the
date, and \var{time} should be a string of the form
\code{"\var{hh}\var{mm}\var{ss}"} indicating the time.  Return a pair
\code{(\var{response}, \var{groups})} where \var{groups} is a list of
group names that are new since the given date and time.
\end{funcdesc}

\begin{funcdesc}{newnews}{group\, date\, time}
Send a \samp{NEWNEWS} command.  Here, \var{group} is a group name or
\code{"*"}, and \var{date} and \var{time} have the same meaning as for
\code{newgroups()}.  Return a pair \code{(\var{response},
\var{articles})} where \var{articles} is a list of article ids.
\end{funcdesc}

\begin{funcdesc}{list}{}
Send a \samp{LIST} command.  Return a pair \code{(\var{response},
\var{list})} where \var{list} is a list of tuples.  Each tuple has the
form \code{(\var{group}, \var{last}, \var{first}, \var{flag})}, where
\var{group} is a group name, \var{last} and \var{first} are the last
and first article numbers (as strings), and \var{flag} is \code{'y'}
if posting is allowed, \code{'n'} if not, and \code{'m'} if the
newsgroup is moderated.  (Note the ordering: \var{last}, \var{first}.)
\end{funcdesc}

\begin{funcdesc}{group}{name}
Send a \samp{GROUP} command, where \var{name} is the group name.
Return a tuple \code{(\var{response}, \var{count}, \var{first},
\var{last}, \var{name})} where \var{count} is the (estimated) number
of articles in the group, \var{first} is the first article number in
the group, \var{last} is the last article number in the group, and
\var{name} is the group name.  The numbers are returned as strings.
\end{funcdesc}

\begin{funcdesc}{help}{}
Send a \samp{HELP} command.  Return a pair \code{(\var{response},
\var{list})} where \var{list} is a list of help strings.
\end{funcdesc}

\begin{funcdesc}{stat}{id}
Send a \samp{STAT} command, where \var{id} is the message id (enclosed
in \samp{<} and \samp{>}) or an article number (as a string).
Return a triple \code{(\var{response}, \var{number}, \var{id})} where
\var{number} is the article number (as a string) and \var{id} is the
article id  (enclosed in \samp{<} and \samp{>}).
\end{funcdesc}

\begin{funcdesc}{next}{}
Send a \samp{NEXT} command.  Return as for \code{stat()}.
\end{funcdesc}

\begin{funcdesc}{last}{}
Send a \samp{LAST} command.  Return as for \code{stat()}.
\end{funcdesc}

\begin{funcdesc}{head}{id}
Send a \samp{HEAD} command, where \var{id} has the same meaning as for
\code{stat()}.  Return a pair \code{(\var{response}, \var{list})}
where \var{list} is a list of the article's headers (an uninterpreted
list of lines, without trailing newlines).
\end{funcdesc}

\begin{funcdesc}{body}{id}
Send a \samp{BODY} command, where \var{id} has the same meaning as for
\code{stat()}.  Return a pair \code{(\var{response}, \var{list})}
where \var{list} is a list of the article's body text (an
uninterpreted list of lines, without trailing newlines).
\end{funcdesc}

\begin{funcdesc}{article}{id}
Send a \samp{ARTICLE} command, where \var{id} has the same meaning as
for \code{stat()}.  Return a pair \code{(\var{response}, \var{list})}
where \var{list} is a list of the article's header and body text (an
uninterpreted list of lines, without trailing newlines).
\end{funcdesc}

\begin{funcdesc}{slave}{}
Send a \samp{SLAVE} command.  Return the server's \var{response}.
\end{funcdesc}

\begin{funcdesc}{xhdr}{header\, string}
Send an \samp{XHDR} command.  This command is not defined in the RFC
but is a common extension.  The \var{header} argument is a header
keyword, e.g. \code{"subject"}.  The \var{string} argument should have
the form \code{"\var{first}-\var{last}"} where \var{first} and
\var{last} are the first and last article numbers to search.  Return a
pair \code{(\var{response}, \var{list})}, where \var{list} is a list of
pairs \code{(\var{id}, \var{text})}, where \var{id} is an article id
(as a string) and \var{text} is the text of the requested header for
that article.
\end{funcdesc}

\begin{funcdesc}{post}{file}
Post an article using the \samp{POST} command.  The \var{file}
argument is an open file object which is read until EOF using its
\code{readline()} method.  It should be a well-formed news article,
including the required headers.  The \code{post()} method
automatically escapes lines beginning with \samp{.}.
\end{funcdesc}

\begin{funcdesc}{ihave}{id\, file}
Send an \samp{IHAVE} command.  If the response is not an error, treat
\var{file} exactly as for the \code{post()} method.
\end{funcdesc}

\begin{funcdesc}{date}{}
Return a triple \code{(\var{response}, \var{date}, \var{time})},
containing the current date and time in a form suitable for the
\code{newnews} and \code{newgroups} methods.
This is an optional NNTP extension, and may not be supported by all
servers.
\end{funcdesc}

\begin{funcdesc}{xgtitle}{name}
Process an XGTITLE command, returning a pair \code{(\var{response},
\var{list}}, where \var{list} is a list of tuples containing
\code{(\var{name}, \var{title})}.
% XXX huh?  Should that be name, description?
This is an optional NNTP extension, and may not be supported by all
servers.
\end{funcdesc}

\begin{funcdesc}{xover}{start\, end}
Return a pair \code{(\var{resp}, \var{list})}.  \var{list} is a list
of tuples, one for each article in the range delimited by the \var{start}
and \var{end} article numbers.  Each tuple is of the form
\code{(\var{article number}, \var{subject}, \var{poster}, \var{date}, \var{id}, \var{references}, \var{size}, \var{lines})}.
This is an optional NNTP extension, and may not be supported by all
servers.
\end{funcdesc}

\begin{funcdesc}{xpath}{id}
Return a pair \code{(\var{resp}, \var{path})}, where \var{path} is the
directory path to the article with message ID \var{id}.  This is an
optional NNTP extension, and may not be supported by all servers.
\end{funcdesc}

\begin{funcdesc}{quit}{}
Send a \samp{QUIT} command and close the connection.  Once this method
has been called, no other methods of the NNTP object should be called.
\end{funcdesc}

\section{Built-in module \sectcode{urlparse}}
\stmodindex{urlparse}
\index{WWW}
\indexii{World-Wide}{Web}
\index{URL}
\indexii{URL}{parsing}
\indexii{relative}{URL}

\renewcommand{\indexsubitem}{(in module urlparse)}

This module defines a standard interface to break URL strings up in
components (addessing scheme, network location, path etc.), to combine
the components back into a URL string, and to convert a ``relative
URL'' to an absolute URL given a ``base URL''.

The module has been designed to match the current Internet draft on
Relative Uniform Resource Locators (and discovered a bug in an earlier
draft!).

It defines the following functions:

\begin{funcdesc}{urlparse}{urlstring\optional{\,
default_scheme\optional{\, allow_fragments}}}
Parse a URL into 6 components, returning a 6-tuple: (addressing
scheme, network location, path, parameters, query, fragment
identifier).  This corresponds to the general structure of a URL:
\code{\var{scheme}://\var{netloc}/\var{path};\var{parameters}?\var{query}\#\var{fragment}}.
Each tuple item is a string, possibly empty.
The components are not broken up in smaller parts (e.g. the network
location is a single string), and \% escapes are not expanded.
The delimiters as shown above are not part of the tuple items, {\em
except} for a leading slash in the \var{path} component, which is
kept if present.

Example:
\code{urlparse('http://www.cwi.nl:80/\%7eguido/Python.html')}
yields the tuple
\code{('http', 'www.cwi.nl:80', '/\%e7guido/Python.html', '', '', '')}.

If the \var{default_scheme} argument is specified, it gives the
default addressing scheme, to be used only if the URL string does not
specify one.  The default value for this argument is the empty string.

If the \var{allow_fragments} argument is zero, fragment identifiers
are not allowed, even if the URL's addressing scheme normally does
support them.  The default value for this argument is \code{1}.
\end{funcdesc}

\begin{funcdesc}{urlunparse}{tuple}
Construct a URL string from a tuple as returned by \code{urlparse}.
This may result in a slightly different, but equivalent URL, if the
URL that was parsed originally had redundant delimiters, e.g. a ? with
an empty query (the draft states that these are equivalent).
\end{funcdesc}

\begin{funcdesc}{urljoin}{base\, url\optional{\, allow_fragments}}
Construct a full (``absolute'') URL by combining a ``base URL''
(\var{base}) with a ``relative URL'' (\var{url}).  Informally, this
uses components of the base URL, in particular the addressing scheme,
the network location and (part of) the path, to provide missing
components in the relative URL.

Example:
\code{urljoin('http://www.cwi.nl/\%7eguido/Python.html',}
\code{'FAQ.html')} yields the string
\code{'http://www.cwi.nl/\%7eguido/FAQ.html'}.

The \var{allow_fragments} argument has the same meaning as for
\code{urlparse}.
\end{funcdesc}

\section{\module{htmllib} ---
         A parser for HTML documents}

\declaremodule{standard}{htmllib}
\modulesynopsis{A parser for HTML documents.}

\index{HTML}
\index{hypertext}


This module defines a class which can serve as a base for parsing text
files formatted in the HyperText Mark-up Language (HTML).  The class
is not directly concerned with I/O --- it must be provided with input
in string form via a method, and makes calls to methods of a
``formatter'' object in order to produce output.  The
\class{HTMLParser} class is designed to be used as a base class for
other classes in order to add functionality, and allows most of its
methods to be extended or overridden.  In turn, this class is derived
from and extends the \class{SGMLParser} class defined in module
\refmodule{sgmllib}\refstmodindex{sgmllib}.  The \class{HTMLParser}
implementation supports the HTML 2.0 language as described in
\rfc{1866}.  Two implementations of formatter objects are provided in
the \refmodule{formatter}\refstmodindex{formatter}\ module; refer to the
documentation for that module for information on the formatter
interface.
\withsubitem{(in module sgmllib)}{\ttindex{SGMLParser}}

The following is a summary of the interface defined by
\class{sgmllib.SGMLParser}:

\begin{itemize}

\item
The interface to feed data to an instance is through the \method{feed()}
method, which takes a string argument.  This can be called with as
little or as much text at a time as desired; \samp{p.feed(a);
p.feed(b)} has the same effect as \samp{p.feed(a+b)}.  When the data
contains complete HTML markup constructs, these are processed immediately;
incomplete constructs are saved in a buffer.  To force processing of all
unprocessed data, call the \method{close()} method.

For example, to parse the entire contents of a file, use:
\begin{verbatim}
parser.feed(open('myfile.html').read())
parser.close()
\end{verbatim}

\item
The interface to define semantics for HTML tags is very simple: derive
a class and define methods called \method{start_\var{tag}()},
\method{end_\var{tag}()}, or \method{do_\var{tag}()}.  The parser will
call these at appropriate moments: \method{start_\var{tag}} or
\method{do_\var{tag}()} is called when an opening tag of the form
\code{<\var{tag} ...>} is encountered; \method{end_\var{tag}()} is called
when a closing tag of the form \code{<\var{tag}>} is encountered.  If
an opening tag requires a corresponding closing tag, like \code{<H1>}
... \code{</H1>}, the class should define the \method{start_\var{tag}()}
method; if a tag requires no closing tag, like \code{<P>}, the class
should define the \method{do_\var{tag}()} method.

\end{itemize}

The module defines a parser class and an exception:

\begin{classdesc}{HTMLParser}{formatter}
This is the basic HTML parser class.  It supports all entity names
required by the XHTML 1.0 Recommendation (\url{http://www.w3.org/TR/xhtml1}).  
It also defines handlers for all HTML 2.0 and many HTML 3.0 and 3.2 elements.
\end{classdesc}

\begin{excdesc}{HTMLParseError}
Exception raised by the \class{HTMLParser} class when it encounters an
error while parsing.
\versionadded{2.4}
\end{excdesc}


\begin{seealso}
  \seemodule{formatter}{Interface definition for transforming an
                        abstract flow of formatting events into
                        specific output events on writer objects.}
  \seemodule{HTMLParser}{Alternate HTML parser that offers a slightly
                         lower-level view of the input, but is
                         designed to work with XHTML, and does not
                         implement some of the SGML syntax not used in
                         ``HTML as deployed'' and which isn't legal
                         for XHTML.}
  \seemodule{htmlentitydefs}{Definition of replacement text for XHTML 1.0 
                             entities.}
  \seemodule{sgmllib}{Base class for \class{HTMLParser}.}
\end{seealso}


\subsection{HTMLParser Objects \label{html-parser-objects}}

In addition to tag methods, the \class{HTMLParser} class provides some
additional methods and instance variables for use within tag methods.

\begin{memberdesc}{formatter}
This is the formatter instance associated with the parser.
\end{memberdesc}

\begin{memberdesc}{nofill}
Boolean flag which should be true when whitespace should not be
collapsed, or false when it should be.  In general, this should only
be true when character data is to be treated as ``preformatted'' text,
as within a \code{<PRE>} element.  The default value is false.  This
affects the operation of \method{handle_data()} and \method{save_end()}.
\end{memberdesc}


\begin{methoddesc}{anchor_bgn}{href, name, type}
This method is called at the start of an anchor region.  The arguments
correspond to the attributes of the \code{<A>} tag with the same
names.  The default implementation maintains a list of hyperlinks
(defined by the \code{HREF} attribute for \code{<A>} tags) within the
document.  The list of hyperlinks is available as the data attribute
\member{anchorlist}.
\end{methoddesc}

\begin{methoddesc}{anchor_end}{}
This method is called at the end of an anchor region.  The default
implementation adds a textual footnote marker using an index into the
list of hyperlinks created by \method{anchor_bgn()}.
\end{methoddesc}

\begin{methoddesc}{handle_image}{source, alt\optional{, ismap\optional{,
                                 align\optional{, width\optional{, height}}}}}
This method is called to handle images.  The default implementation
simply passes the \var{alt} value to the \method{handle_data()}
method.
\end{methoddesc}

\begin{methoddesc}{save_bgn}{}
Begins saving character data in a buffer instead of sending it to the
formatter object.  Retrieve the stored data via \method{save_end()}.
Use of the \method{save_bgn()} / \method{save_end()} pair may not be
nested.
\end{methoddesc}

\begin{methoddesc}{save_end}{}
Ends buffering character data and returns all data saved since the
preceding call to \method{save_bgn()}.  If the \member{nofill} flag is
false, whitespace is collapsed to single spaces.  A call to this
method without a preceding call to \method{save_bgn()} will raise a
\exception{TypeError} exception.
\end{methoddesc}



\section{\module{htmlentitydefs} ---
         Definitions of HTML general entities}

\declaremodule{standard}{htmlentitydefs}
\modulesynopsis{Definitions of HTML general entities.}
\sectionauthor{Fred L. Drake, Jr.}{fdrake@acm.org}

This module defines three dictionaries, \code{name2codepoint},
\code{codepoint2name}, and \code{entitydefs}. \code{entitydefs} is
used by the \refmodule{htmllib} module to provide the
\member{entitydefs} member of the \class{HTMLParser} class.  The
definition provided here contains all the entities defined by XHTML 1.0 
that can be handled using simple textual substitution in the Latin-1
character set (ISO-8859-1).


\begin{datadesc}{entitydefs}
  A dictionary mapping XHTML 1.0 entity definitions to their
  replacement text in ISO Latin-1.

\end{datadesc}

\begin{datadesc}{name2codepoint}
  A dictionary that maps HTML entity names to the Unicode codepoints.
  \versionadded{2.3}
\end{datadesc}

\begin{datadesc}{codepoint2name}
  A dictionary that maps Unicode codepoints to HTML entity names.
  \versionadded{2.3}
\end{datadesc}

\section{Standard Module \sectcode{sgmllib}}
\stmodindex{sgmllib}
\index{SGML}

\renewcommand{\indexsubitem}{(in module sgmllib)}

This module defines a class \code{SGMLParser} which serves as the
basis for parsing text files formatted in SGML (Standard Generalized
Mark-up Language).  In fact, it does not provide a full SGML parser
--- it only parses SGML insofar as it is used by HTML, and the module only
exists as a basis for the \code{htmllib} module.
\stmodindex{htmllib}

In particular, the parser is hardcoded to recognize the following
elements:

\begin{itemize}

\item
Opening and closing tags of the form
``\code{<\var{tag} \var{attr}="\var{value}" ...>}'' and
``\code{</\var{tag}>}'', respectively.

\item
Character references of the form ``\code{\&\#\var{name};}''.

\item
Entity references of the form ``\code{\&\var{name};}''.

\item
SGML comments of the form ``\code{<!--\var{text}>}''.

\end{itemize}

The \code{SGMLParser} class must be instantiated without arguments.
It has the following interface methods:

\begin{funcdesc}{reset}{}
Reset the instance.  Loses all unprocessed data.  This is called
implicitly at instantiation time.
\end{funcdesc}

\begin{funcdesc}{setnomoretags}{}
Stop processing tags.  Treat all following input as literal input
(CDATA).  (This is only provided so the HTML tag \code{<PLAINTEXT>}
can be implemented.)
\end{funcdesc}

\begin{funcdesc}{setliteral}{}
Enter literal mode (CDATA mode).
\end{funcdesc}

\begin{funcdesc}{feed}{data}
Feed some text to the parser.  It is processed insofar as it consists
of complete elements; incomplete data is buffered until more data is
fed or \code{close()} is called.
\end{funcdesc}

\begin{funcdesc}{close}{}
Force processing of all buffered data as if it were followed by an
end-of-file mark.  This method may be redefined by a derived class to
define additional processing at the end of the input, but the
redefined version should always call \code{SGMLParser.close()}.
\end{funcdesc}

\begin{funcdesc}{handle_charref}{ref}
This method is called to process a character reference of the form
``\code{\&\#\var{ref};}'' where \var{ref} is a decimal number in the
range 0-255.  It translates the character to \ASCII{} and calls the
method \code{handle_data()} with the character as argument.  If
\var{ref} is invalid or out of range, the method
\code{unknown_charref(\var{ref})} is called instead.
\end{funcdesc}

\begin{funcdesc}{handle_entityref}{ref}
This method is called to process an entity reference of the form
``\code{\&\var{ref};}'' where \var{ref} is an alphabetic entity
reference.  It looks for \var{ref} in the instance (or class)
variable \code{entitydefs} which should give the entity's translation.
If a translation is found, it calls the method \code{handle_data()}
with the translation; otherwise, it calls the method
\code{unknown_entityref(\var{ref})}.
\end{funcdesc}

\begin{funcdesc}{handle_data}{data}
This method is called to process arbitrary data.  It is intended to be
overridden by a derived class; the base class implementation does
nothing.
\end{funcdesc}

\begin{funcdesc}{unknown_starttag}{tag\, attributes}
This method is called to process an unknown start tag.  It is intended
to be overridden by a derived class; the base class implementation
does nothing.  The \var{attributes} argument is a list of
(\var{name}, \var{value}) pairs containing the attributes found inside
the tag's \code{<>} brackets.  The \var{name} has been translated to
lower case and double quotes and backslashes in the \var{value} have
been interpreted.  For instance, for the tag
\code{<A HREF="http://www.cwi.nl/">}, this method would be
called as \code{unknown_starttag('a', [('href', 'http://www.cwi.nl/')])}.
\end{funcdesc}

\begin{funcdesc}{unknown_endtag}{tag}
This method is called to process an unknown end tag.  It is intended
to be overridden by a derived class; the base class implementation
does nothing.
\end{funcdesc}

\begin{funcdesc}{unknown_charref}{ref}
This method is called to process an unknown character reference.  It
is intended to be overridden by a derived class; the base class
implementation does nothing.
\end{funcdesc}

\begin{funcdesc}{unknown_entityref}{ref}
This method is called to process an unknown entity reference.  It is
intended to be overridden by a derived class; the base class
implementation does nothing.
\end{funcdesc}

Apart from overriding or extending the methods listed above, derived
classes may also define methods of the following form to define
processing of specific tags.  Tag names in the input stream are case
independent; the \var{tag} occurring in method names must be in lower
case:

\begin{funcdesc}{start_\var{tag}}{attributes}
This method is called to process an opening tag \var{tag}.  It has
preference over \code{do_\var{tag}()}.  The \var{attributes} argument
has the same meaning as described for \code{unknown_tag()} above.
\end{funcdesc}

\begin{funcdesc}{do_\var{tag}}{attributes}
This method is called to process an opening tag \var{tag} that does
not come with a matching closing tag.  The \var{attributes} argument
has the same meaning as described for \code{unknown_tag()} above.
\end{funcdesc}

\begin{funcdesc}{end_\var{tag}}{}
This method is called to process a closing tag \var{tag}.
\end{funcdesc}

Note that the parser maintains a stack of opening tags for which no
matching closing tag has been found yet.  Only tags processed by
\code{start_\var{tag}()} are pushed on this stack.  Definition of a
\code{end_\var{tag}()} method is optional for these tags.  For tags
processed by \code{do_\var{tag}()} or by \code{unknown_tag()}, no
\code{end_\var{tag}()} method must be defined.

\section{Standard Module \module{rfc822}}
\label{module-rfc822}
\stmodindex{rfc822}


This module defines a class, \class{Message}, which represents a
collection of ``email headers'' as defined by the Internet standard
\rfc{822}.  It is used in various contexts, usually to read such
headers from a file.

Note that there's a separate module to read \UNIX{}, MH, and MMDF
style mailbox files: \module{mailbox}\refstmodindex{mailbox}.

\begin{classdesc}{Message}{file\optional{, seekable}}
A \class{Message} instance is instantiated with an open file object as
parameter.  The optional \var{seekable} parameter indicates if the
file object is seekable; the default value is \code{1} for true.
Instantiation reads headers from the file up to a blank line and
stores them in the instance; after instantiation, the file is
positioned directly after the blank line that terminates the headers.

Input lines as read from the file may either be terminated by CR-LF or
by a single linefeed; a terminating CR-LF is replaced by a single
linefeed before the line is stored.

All header matching is done independent of upper or lower case;
e.g. \code{\var{m}['From']}, \code{\var{m}['from']} and
\code{\var{m}['FROM']} all yield the same result.
\end{classdesc}

\begin{funcdesc}{parsedate}{date}
Attempts to parse a date according to the rules in \rfc{822}.
however, some mailers don't follow that format as specified, so
\function{parsedate()} tries to guess correctly in such cases. 
\var{date} is a string containing an \rfc{822} date, such as 
\code{'Mon, 20 Nov 1995 19:12:08 -0500'}.  If it succeeds in parsing
the date, \function{parsedate()} returns a 9-tuple that can be passed
directly to \function{time.mktime()}; otherwise \code{None} will be
returned.  
\end{funcdesc}

\begin{funcdesc}{parsedate_tz}{date}
Performs the same function as \function{parsedate()}, but returns
either \code{None} or a 10-tuple; the first 9 elements make up a tuple
that can be passed directly to \function{time.mktime()}, and the tenth
is the offset of the date's timezone from UTC (which is the official
term for Greenwich Mean Time).  (Note that the sign of the timezone
offset is the opposite of the sign of the \code{time.timezone}
variable for the same timezone; the latter variable follows the
\POSIX{} standard while this module follows \rfc{822}.)  If the input
string has no timezone, the last element of the tuple returned is
\code{None}.
\end{funcdesc}

\begin{funcdesc}{mktime_tz}{tuple}
Turn a 10-tuple as returned by \function{parsedate_tz()} into a UTC
timestamp.  It the timezone item in the tuple is \code{None}, assume
local time.  Minor deficiency: this first interprets the first 8
elements as a local time and then compensates for the timezone
difference; this may yield a slight error around daylight savings time
switch dates.  Not enough to worry about for common use.
\end{funcdesc}

\subsection{Message Objects}
\label{message-objects}

A \class{Message} instance has the following methods:

\begin{methoddesc}{rewindbody}{}
Seek to the start of the message body.  This only works if the file
object is seekable.
\end{methoddesc}

\begin{methoddesc}{getallmatchingheaders}{name}
Return a list of lines consisting of all headers matching
\var{name}, if any.  Each physical line, whether it is a continuation
line or not, is a separate list item.  Return the empty list if no
header matches \var{name}.
\end{methoddesc}

\begin{methoddesc}{getfirstmatchingheader}{name}
Return a list of lines comprising the first header matching
\var{name}, and its continuation line(s), if any.  Return \code{None}
if there is no header matching \var{name}.
\end{methoddesc}

\begin{methoddesc}{getrawheader}{name}
Return a single string consisting of the text after the colon in the
first header matching \var{name}.  This includes leading whitespace,
the trailing linefeed, and internal linefeeds and whitespace if there
any continuation line(s) were present.  Return \code{None} if there is
no header matching \var{name}.
\end{methoddesc}

\begin{methoddesc}{getheader}{name}
Like \code{getrawheader(\var{name})}, but strip leading and trailing
whitespace.  Internal whitespace is not stripped.
\end{methoddesc}

\begin{methoddesc}{getaddr}{name}
Return a pair \code{(\var{full name}, \var{email address})} parsed
from the string returned by \code{getheader(\var{name})}.  If no
header matching \var{name} exists, return \code{(None, None)};
otherwise both the full name and the address are (possibly empty)
strings.

Example: If \var{m}'s first \code{From} header contains the string
\code{'jack@cwi.nl (Jack Jansen)'}, then
\code{m.getaddr('From')} will yield the pair
\code{('Jack Jansen', 'jack@cwi.nl')}.
If the header contained
\code{'Jack Jansen <jack@cwi.nl>'} instead, it would yield the
exact same result.
\end{methoddesc}

\begin{methoddesc}{getaddrlist}{name}
This is similar to \code{getaddr(\var{list})}, but parses a header
containing a list of email addresses (e.g. a \code{To} header) and
returns a list of \code{(\var{full name}, \var{email address})} pairs
(even if there was only one address in the header).  If there is no
header matching \var{name}, return an empty list.

XXX The current version of this function is not really correct.  It
yields bogus results if a full name contains a comma.
\end{methoddesc}

\begin{methoddesc}{getdate}{name}
Retrieve a header using \method{getheader()} and parse it into a 9-tuple
compatible with \function{time.mktime()}.  If there is no header matching
\var{name}, or it is unparsable, return \code{None}.

Date parsing appears to be a black art, and not all mailers adhere to
the standard.  While it has been tested and found correct on a large
collection of email from many sources, it is still possible that this
function may occasionally yield an incorrect result.
\end{methoddesc}

\begin{methoddesc}{getdate_tz}{name}
Retrieve a header using \method{getheader()} and parse it into a
10-tuple; the first 9 elements will make a tuple compatible with
\function{time.mktime()}, and the 10th is a number giving the offset
of the date's timezone from UTC.  Similarly to \method{getdate()}, if
there is no header matching \var{name}, or it is unparsable, return
\code{None}. 
\end{methoddesc}

\class{Message} instances also support a read-only mapping interface.
In particular: \code{\var{m}[name]} is like
\code{\var{m}.getheader(name)} but raises \exception{KeyError} if
there is no matching header; and \code{len(\var{m})},
\code{\var{m}.has_key(name)}, \code{\var{m}.keys()},
\code{\var{m}.values()} and \code{\var{m}.items()} act as expected
(and consistently).

Finally, \class{Message} instances have two public instance variables:

\begin{memberdesc}{headers}
A list containing the entire set of header lines, in the order in
which they were read.  Each line contains a trailing newline.  The
blank line terminating the headers is not contained in the list.
\end{memberdesc}

\begin{memberdesc}{fp}
The file object passed at instantiation time.
\end{memberdesc}

\section{Standard Module \sectcode{mimetools}}
\stmodindex{mimetools}

\renewcommand{\indexsubitem}{(in module mimetools)}

To be provided.


\chapter{Multimedia Services}
\label{mmedia}

The modules described in this chapter implement various algorithms or
interfaces that are mainly useful for multimedia applications.  They
are available at the discretion of the installation.  Here's an overview:

\localmoduletable
			% Multimedia Services
\section{\module{audioop} ---
         Manipulate raw audio data}

\declaremodule{builtin}{audioop}
\modulesynopsis{Manipulate raw audio data.}


The \module{audioop} module contains some useful operations on sound
fragments.  It operates on sound fragments consisting of signed
integer samples 8, 16 or 32 bits wide, stored in Python strings.  This
is the same format as used by the \refmodule{al} and \refmodule{sunaudiodev}
modules.  All scalar items are integers, unless specified otherwise.

% This para is mostly here to provide an excuse for the index entries...
This module provides support for u-LAW and Intel/DVI ADPCM encodings.
\index{Intel/DVI ADPCM}
\index{ADPCM, Intel/DVI}
\index{u-LAW}

A few of the more complicated operations only take 16-bit samples,
otherwise the sample size (in bytes) is always a parameter of the
operation.

The module defines the following variables and functions:

\begin{excdesc}{error}
This exception is raised on all errors, such as unknown number of bytes
per sample, etc.
\end{excdesc}

\begin{funcdesc}{add}{fragment1, fragment2, width}
Return a fragment which is the addition of the two samples passed as
parameters.  \var{width} is the sample width in bytes, either
\code{1}, \code{2} or \code{4}.  Both fragments should have the same
length.
\end{funcdesc}

\begin{funcdesc}{adpcm2lin}{adpcmfragment, width, state}
Decode an Intel/DVI ADPCM coded fragment to a linear fragment.  See
the description of \function{lin2adpcm()} for details on ADPCM coding.
Return a tuple \code{(\var{sample}, \var{newstate})} where the sample
has the width specified in \var{width}.
\end{funcdesc}

\begin{funcdesc}{adpcm32lin}{adpcmfragment, width, state}
Decode an alternative 3-bit ADPCM code.  See \function{lin2adpcm3()}
for details.
\end{funcdesc}

\begin{funcdesc}{avg}{fragment, width}
Return the average over all samples in the fragment.
\end{funcdesc}

\begin{funcdesc}{avgpp}{fragment, width}
Return the average peak-peak value over all samples in the fragment.
No filtering is done, so the usefulness of this routine is
questionable.
\end{funcdesc}

\begin{funcdesc}{bias}{fragment, width, bias}
Return a fragment that is the original fragment with a bias added to
each sample.
\end{funcdesc}

\begin{funcdesc}{cross}{fragment, width}
Return the number of zero crossings in the fragment passed as an
argument.
\end{funcdesc}

\begin{funcdesc}{findfactor}{fragment, reference}
Return a factor \var{F} such that
\code{rms(add(\var{fragment}, mul(\var{reference}, -\var{F})))} is
minimal, i.e., return the factor with which you should multiply
\var{reference} to make it match as well as possible to
\var{fragment}.  The fragments should both contain 2-byte samples.

The time taken by this routine is proportional to
\code{len(\var{fragment})}.
\end{funcdesc}

\begin{funcdesc}{findfit}{fragment, reference}
Try to match \var{reference} as well as possible to a portion of
\var{fragment} (which should be the longer fragment).  This is
(conceptually) done by taking slices out of \var{fragment}, using
\function{findfactor()} to compute the best match, and minimizing the
result.  The fragments should both contain 2-byte samples.  Return a
tuple \code{(\var{offset}, \var{factor})} where \var{offset} is the
(integer) offset into \var{fragment} where the optimal match started
and \var{factor} is the (floating-point) factor as per
\function{findfactor()}.
\end{funcdesc}

\begin{funcdesc}{findmax}{fragment, length}
Search \var{fragment} for a slice of length \var{length} samples (not
bytes!)\ with maximum energy, i.e., return \var{i} for which
\code{rms(fragment[i*2:(i+length)*2])} is maximal.  The fragments
should both contain 2-byte samples.

The routine takes time proportional to \code{len(\var{fragment})}.
\end{funcdesc}

\begin{funcdesc}{getsample}{fragment, width, index}
Return the value of sample \var{index} from the fragment.
\end{funcdesc}

\begin{funcdesc}{lin2lin}{fragment, width, newwidth}
Convert samples between 1-, 2- and 4-byte formats.
\end{funcdesc}

\begin{funcdesc}{lin2adpcm}{fragment, width, state}
Convert samples to 4 bit Intel/DVI ADPCM encoding.  ADPCM coding is an
adaptive coding scheme, whereby each 4 bit number is the difference
between one sample and the next, divided by a (varying) step.  The
Intel/DVI ADPCM algorithm has been selected for use by the IMA, so it
may well become a standard.

\var{state} is a tuple containing the state of the coder.  The coder
returns a tuple \code{(\var{adpcmfrag}, \var{newstate})}, and the
\var{newstate} should be passed to the next call of
\function{lin2adpcm()}.  In the initial call, \code{None} can be
passed as the state.  \var{adpcmfrag} is the ADPCM coded fragment
packed 2 4-bit values per byte.
\end{funcdesc}

\begin{funcdesc}{lin2adpcm3}{fragment, width, state}
This is an alternative ADPCM coder that uses only 3 bits per sample.
It is not compatible with the Intel/DVI ADPCM coder and its output is
not packed (due to laziness on the side of the author).  Its use is
discouraged.
\end{funcdesc}

\begin{funcdesc}{lin2ulaw}{fragment, width}
Convert samples in the audio fragment to u-LAW encoding and return
this as a Python string.  u-LAW is an audio encoding format whereby
you get a dynamic range of about 14 bits using only 8 bit samples.  It
is used by the Sun audio hardware, among others.
\end{funcdesc}

\begin{funcdesc}{minmax}{fragment, width}
Return a tuple consisting of the minimum and maximum values of all
samples in the sound fragment.
\end{funcdesc}

\begin{funcdesc}{max}{fragment, width}
Return the maximum of the \emph{absolute value} of all samples in a
fragment.
\end{funcdesc}

\begin{funcdesc}{maxpp}{fragment, width}
Return the maximum peak-peak value in the sound fragment.
\end{funcdesc}

\begin{funcdesc}{mul}{fragment, width, factor}
Return a fragment that has all samples in the original fragment
multiplied by the floating-point value \var{factor}.  Overflow is
silently ignored.
\end{funcdesc}

\begin{funcdesc}{ratecv}{fragment, width, nchannels, inrate, outrate,
                         state\optional{, weightA\optional{, weightB}}}
Convert the frame rate of the input fragment.

\var{state} is a tuple containing the state of the converter.  The
converter returns a tuple \code{(\var{newfragment}, \var{newstate})},
and \var{newstate} should be passed to the next call of
\function{ratecv()}.  The initial call should pass \code{None}
as the state.

The \var{weightA} and \var{weightB} arguments are parameters for a
simple digital filter and default to \code{1} and \code{0} respectively.
\end{funcdesc}

\begin{funcdesc}{reverse}{fragment, width}
Reverse the samples in a fragment and returns the modified fragment.
\end{funcdesc}

\begin{funcdesc}{rms}{fragment, width}
Return the root-mean-square of the fragment, i.e.
\begin{displaymath}
\catcode`_=8
\sqrt{\frac{\sum{{S_{i}}^{2}}}{n}}
\end{displaymath}
This is a measure of the power in an audio signal.
\end{funcdesc}

\begin{funcdesc}{tomono}{fragment, width, lfactor, rfactor} 
Convert a stereo fragment to a mono fragment.  The left channel is
multiplied by \var{lfactor} and the right channel by \var{rfactor}
before adding the two channels to give a mono signal.
\end{funcdesc}

\begin{funcdesc}{tostereo}{fragment, width, lfactor, rfactor}
Generate a stereo fragment from a mono fragment.  Each pair of samples
in the stereo fragment are computed from the mono sample, whereby left
channel samples are multiplied by \var{lfactor} and right channel
samples by \var{rfactor}.
\end{funcdesc}

\begin{funcdesc}{ulaw2lin}{fragment, width}
Convert sound fragments in u-LAW encoding to linearly encoded sound
fragments.  u-LAW encoding always uses 8 bits samples, so \var{width}
refers only to the sample width of the output fragment here.
\end{funcdesc}

Note that operations such as \function{mul()} or \function{max()} make
no distinction between mono and stereo fragments, i.e.\ all samples
are treated equal.  If this is a problem the stereo fragment should be
split into two mono fragments first and recombined later.  Here is an
example of how to do that:

\begin{verbatim}
def mul_stereo(sample, width, lfactor, rfactor):
    lsample = audioop.tomono(sample, width, 1, 0)
    rsample = audioop.tomono(sample, width, 0, 1)
    lsample = audioop.mul(sample, width, lfactor)
    rsample = audioop.mul(sample, width, rfactor)
    lsample = audioop.tostereo(lsample, width, 1, 0)
    rsample = audioop.tostereo(rsample, width, 0, 1)
    return audioop.add(lsample, rsample, width)
\end{verbatim}

If you use the ADPCM coder to build network packets and you want your
protocol to be stateless (i.e.\ to be able to tolerate packet loss)
you should not only transmit the data but also the state.  Note that
you should send the \var{initial} state (the one you passed to
\function{lin2adpcm()}) along to the decoder, not the final state (as
returned by the coder).  If you want to use \function{struct.struct()}
to store the state in binary you can code the first element (the
predicted value) in 16 bits and the second (the delta index) in 8.

The ADPCM coders have never been tried against other ADPCM coders,
only against themselves.  It could well be that I misinterpreted the
standards in which case they will not be interoperable with the
respective standards.

The \function{find*()} routines might look a bit funny at first sight.
They are primarily meant to do echo cancellation.  A reasonably
fast way to do this is to pick the most energetic piece of the output
sample, locate that in the input sample and subtract the whole output
sample from the input sample:

\begin{verbatim}
def echocancel(outputdata, inputdata):
    pos = audioop.findmax(outputdata, 800)    # one tenth second
    out_test = outputdata[pos*2:]
    in_test = inputdata[pos*2:]
    ipos, factor = audioop.findfit(in_test, out_test)
    # Optional (for better cancellation):
    # factor = audioop.findfactor(in_test[ipos*2:ipos*2+len(out_test)], 
    #              out_test)
    prefill = '\0'*(pos+ipos)*2
    postfill = '\0'*(len(inputdata)-len(prefill)-len(outputdata))
    outputdata = prefill + audioop.mul(outputdata,2,-factor) + postfill
    return audioop.add(inputdata, outputdata, 2)
\end{verbatim}

\section{\module{imageop} ---
         Manipulate raw image data}

\declaremodule{builtin}{imageop}
\modulesynopsis{Manipulate raw image data.}


The \module{imageop} module contains some useful operations on images.
It operates on images consisting of 8 or 32 bit pixels stored in
Python strings.  This is the same format as used by
\function{gl.lrectwrite()} and the \refmodule{imgfile} module.

The module defines the following variables and functions:

\begin{excdesc}{error}
This exception is raised on all errors, such as unknown number of bits
per pixel, etc.
\end{excdesc}


\begin{funcdesc}{crop}{image, psize, width, height, x0, y0, x1, y1}
Return the selected part of \var{image}, which should by
\var{width} by \var{height} in size and consist of pixels of
\var{psize} bytes. \var{x0}, \var{y0}, \var{x1} and \var{y1} are like
the \function{gl.lrectread()} parameters, i.e.\ the boundary is
included in the new image.  The new boundaries need not be inside the
picture.  Pixels that fall outside the old image will have their value
set to zero.  If \var{x0} is bigger than \var{x1} the new image is
mirrored.  The same holds for the y coordinates.
\end{funcdesc}

\begin{funcdesc}{scale}{image, psize, width, height, newwidth, newheight}
Return \var{image} scaled to size \var{newwidth} by \var{newheight}.
No interpolation is done, scaling is done by simple-minded pixel
duplication or removal.  Therefore, computer-generated images or
dithered images will not look nice after scaling.
\end{funcdesc}

\begin{funcdesc}{tovideo}{image, psize, width, height}
Run a vertical low-pass filter over an image.  It does so by computing
each destination pixel as the average of two vertically-aligned source
pixels.  The main use of this routine is to forestall excessive
flicker if the image is displayed on a video device that uses
interlacing, hence the name.
\end{funcdesc}

\begin{funcdesc}{grey2mono}{image, width, height, threshold}
Convert a 8-bit deep greyscale image to a 1-bit deep image by
thresholding all the pixels.  The resulting image is tightly packed and
is probably only useful as an argument to \function{mono2grey()}.
\end{funcdesc}

\begin{funcdesc}{dither2mono}{image, width, height}
Convert an 8-bit greyscale image to a 1-bit monochrome image using a
(simple-minded) dithering algorithm.
\end{funcdesc}

\begin{funcdesc}{mono2grey}{image, width, height, p0, p1}
Convert a 1-bit monochrome image to an 8 bit greyscale or color image.
All pixels that are zero-valued on input get value \var{p0} on output
and all one-value input pixels get value \var{p1} on output.  To
convert a monochrome black-and-white image to greyscale pass the
values \code{0} and \code{255} respectively.
\end{funcdesc}

\begin{funcdesc}{grey2grey4}{image, width, height}
Convert an 8-bit greyscale image to a 4-bit greyscale image without
dithering.
\end{funcdesc}

\begin{funcdesc}{grey2grey2}{image, width, height}
Convert an 8-bit greyscale image to a 2-bit greyscale image without
dithering.
\end{funcdesc}

\begin{funcdesc}{dither2grey2}{image, width, height}
Convert an 8-bit greyscale image to a 2-bit greyscale image with
dithering.  As for \function{dither2mono()}, the dithering algorithm
is currently very simple.
\end{funcdesc}

\begin{funcdesc}{grey42grey}{image, width, height}
Convert a 4-bit greyscale image to an 8-bit greyscale image.
\end{funcdesc}

\begin{funcdesc}{grey22grey}{image, width, height}
Convert a 2-bit greyscale image to an 8-bit greyscale image.
\end{funcdesc}

\begin{datadesc}{backward_compatible}
If set to 0, the functions in this module use a non-backward
compatible way of representing multi-byte pixels on little-endian
systems.  The SGI for which this module was originally written is a
big-endian system, so setting this variable will have no effect.
However, the code wasn't originally intended to run on anything else,
so it made assumptions about byte order which are not universal.
Setting this variable to 0 will cause the byte order to be reversed on
little-endian systems, so that it then is the same as on big-endian
systems.
\end{datadesc}

\section{\module{aifc} ---
         Read and write AIFF and AIFC files}

\declaremodule{standard}{aifc}
\modulesynopsis{Read and write audio files in AIFF or AIFC format.}


This module provides support for reading and writing AIFF and AIFF-C
files.  AIFF is Audio Interchange File Format, a format for storing
digital audio samples in a file.  AIFF-C is a newer version of the
format that includes the ability to compress the audio data.
\index{Audio Interchange File Format}
\index{AIFF}
\index{AIFF-C}

\strong{Caveat:}  Some operations may only work under IRIX; these will
raise \exception{ImportError} when attempting to import the
\module{cl} module, which is only available on IRIX.

Audio files have a number of parameters that describe the audio data.
The sampling rate or frame rate is the number of times per second the
sound is sampled.  The number of channels indicate if the audio is
mono, stereo, or quadro.  Each frame consists of one sample per
channel.  The sample size is the size in bytes of each sample.  Thus a
frame consists of \var{nchannels}*\var{samplesize} bytes, and a
second's worth of audio consists of
\var{nchannels}*\var{samplesize}*\var{framerate} bytes.

For example, CD quality audio has a sample size of two bytes (16
bits), uses two channels (stereo) and has a frame rate of 44,100
frames/second.  This gives a frame size of 4 bytes (2*2), and a
second's worth occupies 2*2*44100 bytes (176,400 bytes).

Module \module{aifc} defines the following function:

\begin{funcdesc}{open}{file\optional{, mode}}
Open an AIFF or AIFF-C file and return an object instance with
methods that are described below.  The argument \var{file} is either a
string naming a file or a file object.  \var{mode} must be \code{'r'}
or \code{'rb'} when the file must be opened for reading, or \code{'w'} 
or \code{'wb'} when the file must be opened for writing.  If omitted,
\code{\var{file}.mode} is used if it exists, otherwise \code{'rb'} is
used.  When used for writing, the file object should be seekable,
unless you know ahead of time how many samples you are going to write
in total and use \method{writeframesraw()} and \method{setnframes()}.
\end{funcdesc}

Objects returned by \function{open()} when a file is opened for
reading have the following methods:

\begin{methoddesc}[aifc]{getnchannels}{}
Return the number of audio channels (1 for mono, 2 for stereo).
\end{methoddesc}

\begin{methoddesc}[aifc]{getsampwidth}{}
Return the size in bytes of individual samples.
\end{methoddesc}

\begin{methoddesc}[aifc]{getframerate}{}
Return the sampling rate (number of audio frames per second).
\end{methoddesc}

\begin{methoddesc}[aifc]{getnframes}{}
Return the number of audio frames in the file.
\end{methoddesc}

\begin{methoddesc}[aifc]{getcomptype}{}
Return a four-character string describing the type of compression used
in the audio file.  For AIFF files, the returned value is
\code{'NONE'}.
\end{methoddesc}

\begin{methoddesc}[aifc]{getcompname}{}
Return a human-readable description of the type of compression used in
the audio file.  For AIFF files, the returned value is \code{'not
compressed'}.
\end{methoddesc}

\begin{methoddesc}[aifc]{getparams}{}
Return a tuple consisting of all of the above values in the above
order.
\end{methoddesc}

\begin{methoddesc}[aifc]{getmarkers}{}
Return a list of markers in the audio file.  A marker consists of a
tuple of three elements.  The first is the mark ID (an integer), the
second is the mark position in frames from the beginning of the data
(an integer), the third is the name of the mark (a string).
\end{methoddesc}

\begin{methoddesc}[aifc]{getmark}{id}
Return the tuple as described in \method{getmarkers()} for the mark
with the given \var{id}.
\end{methoddesc}

\begin{methoddesc}[aifc]{readframes}{nframes}
Read and return the next \var{nframes} frames from the audio file.  The
returned data is a string containing for each frame the uncompressed
samples of all channels.
\end{methoddesc}

\begin{methoddesc}[aifc]{rewind}{}
Rewind the read pointer.  The next \method{readframes()} will start from
the beginning.
\end{methoddesc}

\begin{methoddesc}[aifc]{setpos}{pos}
Seek to the specified frame number.
\end{methoddesc}

\begin{methoddesc}[aifc]{tell}{}
Return the current frame number.
\end{methoddesc}

\begin{methoddesc}[aifc]{close}{}
Close the AIFF file.  After calling this method, the object can no
longer be used.
\end{methoddesc}

Objects returned by \function{open()} when a file is opened for
writing have all the above methods, except for \method{readframes()} and
\method{setpos()}.  In addition the following methods exist.  The
\method{get*()} methods can only be called after the corresponding
\method{set*()} methods have been called.  Before the first
\method{writeframes()} or \method{writeframesraw()}, all parameters
except for the number of frames must be filled in.

\begin{methoddesc}[aifc]{aiff}{}
Create an AIFF file.  The default is that an AIFF-C file is created,
unless the name of the file ends in \code{'.aiff'} in which case the
default is an AIFF file.
\end{methoddesc}

\begin{methoddesc}[aifc]{aifc}{}
Create an AIFF-C file.  The default is that an AIFF-C file is created,
unless the name of the file ends in \code{'.aiff'} in which case the
default is an AIFF file.
\end{methoddesc}

\begin{methoddesc}[aifc]{setnchannels}{nchannels}
Specify the number of channels in the audio file.
\end{methoddesc}

\begin{methoddesc}[aifc]{setsampwidth}{width}
Specify the size in bytes of audio samples.
\end{methoddesc}

\begin{methoddesc}[aifc]{setframerate}{rate}
Specify the sampling frequency in frames per second.
\end{methoddesc}

\begin{methoddesc}[aifc]{setnframes}{nframes}
Specify the number of frames that are to be written to the audio file.
If this parameter is not set, or not set correctly, the file needs to
support seeking.
\end{methoddesc}

\begin{methoddesc}[aifc]{setcomptype}{type, name}
Specify the compression type.  If not specified, the audio data will
not be compressed.  In AIFF files, compression is not possible.  The
name parameter should be a human-readable description of the
compression type, the type parameter should be a four-character
string.  Currently the following compression types are supported:
NONE, ULAW, ALAW, G722.
\index{u-LAW}
\index{A-LAW}
\index{G.722}
\end{methoddesc}

\begin{methoddesc}[aifc]{setparams}{nchannels, sampwidth, framerate, comptype, compname}
Set all the above parameters at once.  The argument is a tuple
consisting of the various parameters.  This means that it is possible
to use the result of a \method{getparams()} call as argument to
\method{setparams()}.
\end{methoddesc}

\begin{methoddesc}[aifc]{setmark}{id, pos, name}
Add a mark with the given id (larger than 0), and the given name at
the given position.  This method can be called at any time before
\method{close()}.
\end{methoddesc}

\begin{methoddesc}[aifc]{tell}{}
Return the current write position in the output file.  Useful in
combination with \method{setmark()}.
\end{methoddesc}

\begin{methoddesc}[aifc]{writeframes}{data}
Write data to the output file.  This method can only be called after
the audio file parameters have been set.
\end{methoddesc}

\begin{methoddesc}[aifc]{writeframesraw}{data}
Like \method{writeframes()}, except that the header of the audio file
is not updated.
\end{methoddesc}

\begin{methoddesc}[aifc]{close}{}
Close the AIFF file.  The header of the file is updated to reflect the
actual size of the audio data. After calling this method, the object
can no longer be used.
\end{methoddesc}

\section{\module{jpeg} ---
         Read and write JPEG files}

\declaremodule{builtin}{jpeg}
  \platform{IRIX}
\modulesynopsis{Read and write image files in compressed JPEG format.}


The module \module{jpeg} provides access to the jpeg compressor and
decompressor written by the Independent JPEG Group
\index{Independent JPEG Group}(IJG). JPEG is a standard for
compressing pictures; it is defined in ISO 10918.  For details on JPEG
or the Independent JPEG Group software refer to the JPEG standard or
the documentation provided with the software.

A portable interface to JPEG image files is available with the Python
Imaging Library (PIL) by Fredrik Lundh.  Information on PIL is
available at \url{http://www.pythonware.com/products/pil/}.
\index{Python Imaging Library}
\index{PIL (the Python Imaging Library)}
\index{Lundh, Fredrik}

The \module{jpeg} module defines an exception and some functions.

\begin{excdesc}{error}
Exception raised by \function{compress()} and \function{decompress()}
in case of errors.
\end{excdesc}

\begin{funcdesc}{compress}{data, w, h, b}
Treat data as a pixmap of width \var{w} and height \var{h}, with
\var{b} bytes per pixel.  The data is in SGI GL order, so the first
pixel is in the lower-left corner. This means that \function{gl.lrectread()}
return data can immediately be passed to \function{compress()}.
Currently only 1 byte and 4 byte pixels are allowed, the former being
treated as greyscale and the latter as RGB color.
\function{compress()} returns a string that contains the compressed
picture, in JFIF\index{JFIF} format.
\end{funcdesc}

\begin{funcdesc}{decompress}{data}
Data is a string containing a picture in JFIF\index{JFIF} format. It
returns a tuple \code{(\var{data}, \var{width}, \var{height},
\var{bytesperpixel})}.  Again, the data is suitable to pass to
\function{gl.lrectwrite()}.
\end{funcdesc}

\begin{funcdesc}{setoption}{name, value}
Set various options.  Subsequent \function{compress()} and
\function{decompress()} calls will use these options.  The following
options are available:

\begin{tableii}{l|p{3in}}{code}{Option}{Effect}
  \lineii{'forcegray'}{%
    Force output to be grayscale, even if input is RGB.}
  \lineii{'quality'}{%
    Set the quality of the compressed image to a value between
    \code{0} and \code{100} (default is \code{75}).  This only affects
    compression.}
  \lineii{'optimize'}{%
    Perform Huffman table optimization.  Takes longer, but results in
    smaller compressed image.  This only affects compression.}
  \lineii{'smooth'}{%
    Perform inter-block smoothing on uncompressed image.  Only useful
    for low-quality images.  This only affects decompression.}
\end{tableii}
\end{funcdesc}


\begin{seealso}
  \seetitle{JPEG Still Image Data Compression Standard}{The 
            canonical reference for the JPEG image format, by
            Pennebaker and Mitchell.}

  \seetitle[http://www.w3.org/Graphics/JPEG/itu-t81.pdf]{Information
            Technology - Digital Compression and Coding of
            Continuous-tone Still Images - Requirements and
            Guidelines}{The ISO standard for JPEG is also published as
            ITU T.81.  This is available online in PDF form.}
\end{seealso}

\section{\module{rgbimg} ---
         Read and write ``SGI RGB'' files}

\declaremodule{builtin}{rgbimg}
\modulesynopsis{Read and write image files in ``SGI RGB'' format (the module is
\emph{not} SGI specific though!).}


The \module{rgbimg} module allows Python programs to access SGI imglib image
files (also known as \file{.rgb} files).  The module is far from
complete, but is provided anyway since the functionality that there is
enough in some cases.  Currently, colormap files are not supported.

The module defines the following variables and functions:

\begin{excdesc}{error}
This exception is raised on all errors, such as unsupported file type, etc.
\end{excdesc}

\begin{funcdesc}{sizeofimage}{file}
This function returns a tuple \code{(\var{x}, \var{y})} where
\var{x} and \var{y} are the size of the image in pixels.
Only 4 byte RGBA pixels, 3 byte RGB pixels, and 1 byte greyscale pixels
are currently supported.
\end{funcdesc}

\begin{funcdesc}{longimagedata}{file}
This function reads and decodes the image on the specified file, and
returns it as a Python string. The string has 4 byte RGBA pixels.
The bottom left pixel is the first in
the string. This format is suitable to pass to \function{gl.lrectwrite()},
for instance.
\end{funcdesc}

\begin{funcdesc}{longstoimage}{data, x, y, z, file}
This function writes the RGBA data in \var{data} to image
file \var{file}. \var{x} and \var{y} give the size of the image.
\var{z} is 1 if the saved image should be 1 byte greyscale, 3 if the
saved image should be 3 byte RGB data, or 4 if the saved images should
be 4 byte RGBA data.  The input data always contains 4 bytes per pixel.
These are the formats returned by \function{gl.lrectread()}.
\end{funcdesc}

\begin{funcdesc}{ttob}{flag}
This function sets a global flag which defines whether the scan lines
of the image are read or written from bottom to top (flag is zero,
compatible with SGI GL) or from top to bottom(flag is one,
compatible with X).  The default is zero.
\end{funcdesc}


\chapter{Cryptographic Services}
\label{crypto}
\index{cryptography}

The modules described in this chapter implement various algorithms of
a cryptographic nature.  They are available at the discretion of the
installation.  Here's an overview:

\localmoduletable

Hardcore cypherpunks will probably find the cryptographic modules
written by A.M. Kuchling of further interest; the package adds
built-in modules for DES and IDEA encryption, provides a Python module
for reading and decrypting PGP files, and then some.  These modules
are not distributed with Python but available separately.  See the URL
\url{http://www.amk.ca/python/code/crypto.html} 
for more information.
\index{PGP}
\index{Pretty Good Privacy}
\indexii{DES}{cipher}
\indexii{IDEA}{cipher}
\index{cryptography}
\index{Kuchling, Andrew}
		% Cryptographic Services
\section{Built-in module \sectcode{md5}}
\bimodindex{md5}

This module implements the interface to RSA's MD5 message digest
algorithm (see also the file \file{md5.doc}). Its use is quite
straightforward:\ use the function \code{new} to create an
\dfn{md5}-object. You can now ``feed'' this object with arbitrary
strings.

At any time you can ask for the ``final'' digest of the object. Internally,
a temporary copy of the object is made and the digest is computed and
returned. Because of the copy, the digest operation is not destructive
for the object. Before a more exact description of the module's use, a small
example will be helpful: 
to obtain the digest of the string \code{'abc'}, use \ldots

\bcode\begin{verbatim}
>>> import md5
>>> m = md5.new()
>>> m.update('abc')
>>> m.digest()
'\220\001P\230<\322O\260\326\226?}(\341\177r'
\end{verbatim}\ecode

More condensed:

\bcode\begin{verbatim}
>>> md5.new('abc').digest()
'\220\001P\230<\322O\260\326\226?}(\341\177r'
\end{verbatim}\ecode

\renewcommand{\indexsubitem}{(in module md5)}

\begin{funcdesc}{new}{\optional{arg}}
  Create a new md5-object. If \var{arg} is present, an initial
  \code{update} method is called with \var{arg} as argument.
\end{funcdesc}

\begin{funcdesc}{md5}{\optional{arg}}
For backward compatibility reasons, this is an alternative name for the
\code{new} function.
\end{funcdesc}

An md5-object has the following methods:

\renewcommand{\indexsubitem}{(md5 method)}
\begin{funcdesc}{update}{arg}
  Update this md5-object with the string \var{arg}.
\end{funcdesc}

\begin{funcdesc}{digest}{}
% XXX The following is not quite clear; what does MD5Final do?
  Return the \dfn{digest} of this md5-object. Internally, a copy is made
  and the \C-function \code{MD5Final} is called. Finally the digest is
  returned.
\end{funcdesc}

\begin{funcdesc}{copy}{}
  Return a separate copy of this md5-object.  An \code{update} to this
  copy won't affect the original object.
\end{funcdesc}

\section{Built-in Module \sectcode{mpz}}
\label{module-mpz}
\bimodindex{mpz}

This is an optional module.  It is only available when Python is
configured to include it, which requires that the GNU MP software is
installed.

This module implements the interface to part of the GNU MP library,
which defines arbitrary precision integer and rational number
arithmetic routines.  Only the interfaces to the \emph{integer}
(\samp{mpz_{\rm \ldots}}) routines are provided. If not stated
otherwise, the description in the GNU MP documentation can be applied.

In general, \dfn{mpz}-numbers can be used just like other standard
Python numbers, e.g.\ you can use the built-in operators like \code{+},
\code{*}, etc., as well as the standard built-in functions like
\code{abs}, \code{int}, \ldots, \code{divmod}, \code{pow}.
\strong{Please note:} the {\it bitwise-xor} operation has been implemented as
a bunch of {\it and}s, {\it invert}s and {\it or}s, because the library
lacks an \code{mpz_xor} function, and I didn't need one.

You create an mpz-number by calling the function called \code{mpz} (see
below for an exact description). An mpz-number is printed like this:
\code{mpz(\var{value})}.

\renewcommand{\indexsubitem}{(in module mpz)}
\begin{funcdesc}{mpz}{value}
  Create a new mpz-number. \var{value} can be an integer, a long,
  another mpz-number, or even a string. If it is a string, it is
  interpreted as an array of radix-256 digits, least significant digit
  first, resulting in a positive number. See also the \code{binary}
  method, described below.
\end{funcdesc}

A number of \emph{extra} functions are defined in this module. Non
mpz-arguments are converted to mpz-values first, and the functions
return mpz-numbers.

\begin{funcdesc}{powm}{base\, exponent\, modulus}
  Return \code{pow(\var{base}, \var{exponent}) \%{} \var{modulus}}. If
  \code{\var{exponent} == 0}, return \code{mpz(1)}. In contrast to the
  \C-library function, this version can handle negative exponents.
\end{funcdesc}

\begin{funcdesc}{gcd}{op1\, op2}
  Return the greatest common divisor of \var{op1} and \var{op2}.
\end{funcdesc}

\begin{funcdesc}{gcdext}{a\, b}
  Return a tuple \code{(\var{g}, \var{s}, \var{t})}, such that
  \code{\var{a}*\var{s} + \var{b}*\var{t} == \var{g} == gcd(\var{a}, \var{b})}.
\end{funcdesc}

\begin{funcdesc}{sqrt}{op}
  Return the square root of \var{op}. The result is rounded towards zero.
\end{funcdesc}

\begin{funcdesc}{sqrtrem}{op}
  Return a tuple \code{(\var{root}, \var{remainder})}, such that
  \code{\var{root}*\var{root} + \var{remainder} == \var{op}}.
\end{funcdesc}

\begin{funcdesc}{divm}{numerator\, denominator\, modulus}
  Returns a number \var{q}. such that
  \code{\var{q} * \var{denominator} \%{} \var{modulus} == \var{numerator}}.
  One could also implement this function in Python, using \code{gcdext}.
\end{funcdesc}

An mpz-number has one method:

\renewcommand{\indexsubitem}{(mpz method)}
\begin{funcdesc}{binary}{}
  Convert this mpz-number to a binary string, where the number has been
  stored as an array of radix-256 digits, least significant digit first.

  The mpz-number must have a value greater than or equal to zero,
  otherwise a \code{ValueError}-exception will be raised.
\end{funcdesc}

\section{Built-in module \sectcode{rotor}}
\bimodindex{rotor}

This module implements a rotor-based encryption algorithm, contributed
by Lance Ellinghouse.  Currently no further documentation is available
--- you are kindly advised to read the source...


%\chapter{Amoeba Specific Services}

\section{\module{amoeba} ---
         Amoeba system support}

\declaremodule{builtin}{amoeba}
  \platform{Amoeba}
\modulesynopsis{Functions for the Amoeba operating system.}


This module provides some object types and operations useful for
Amoeba applications.  It is only available on systems that support
Amoeba operations.  RPC errors and other Amoeba errors are reported as
the exception \code{amoeba.error = 'amoeba.error'}.

The module \module{amoeba} defines the following items:

\begin{funcdesc}{name_append}{path, cap}
Stores a capability in the Amoeba directory tree.
Arguments are the pathname (a string) and the capability (a capability
object as returned by
\function{name_lookup()}).
\end{funcdesc}

\begin{funcdesc}{name_delete}{path}
Deletes a capability from the Amoeba directory tree.
Argument is the pathname.
\end{funcdesc}

\begin{funcdesc}{name_lookup}{path}
Looks up a capability.
Argument is the pathname.
Returns a
\dfn{capability}
object, to which various interesting operations apply, described below.
\end{funcdesc}

\begin{funcdesc}{name_replace}{path, cap}
Replaces a capability in the Amoeba directory tree.
Arguments are the pathname and the new capability.
(This differs from
\function{name_append()}
in the behavior when the pathname already exists:
\function{name_append()}
finds this an error while
\function{name_replace()}
allows it, as its name suggests.)
\end{funcdesc}

\begin{datadesc}{capv}
A table representing the capability environment at the time the
interpreter was started.
(Alas, modifying this table does not affect the capability environment
of the interpreter.)
For example,
\code{amoeba.capv['ROOT']}
is the capability of your root directory, similar to
\code{getcap("ROOT")}
in C.
\end{datadesc}

\begin{excdesc}{error}
The exception raised when an Amoeba function returns an error.
The value accompanying this exception is a pair containing the numeric
error code and the corresponding string, as returned by the C function
\cfunction{err_why()}.
\end{excdesc}

\begin{funcdesc}{timeout}{msecs}
Sets the transaction timeout, in milliseconds.
Returns the previous timeout.
Initially, the timeout is set to 2 seconds by the Python interpreter.
\end{funcdesc}

\subsection{Capability Operations}

Capabilities are written in a convenient \ASCII{} format, also used by the
Amoeba utilities
\emph{c2a}(U)
and
\emph{a2c}(U).
For example:

\begin{verbatim}
>>> amoeba.name_lookup('/profile/cap')
aa:1c:95:52:6a:fa/14(ff)/8e:ba:5b:8:11:1a
>>> 
\end{verbatim}
%
The following methods are defined for capability objects.

\setindexsubitem{(capability method)}
\begin{funcdesc}{dir_list}{}
Returns a list of the names of the entries in an Amoeba directory.
\end{funcdesc}

\begin{funcdesc}{b_read}{offset, maxsize}
Reads (at most)
\var{maxsize}
bytes from a bullet file at offset
\var{offset.}
The data is returned as a string.
EOF is reported as an empty string.
\end{funcdesc}

\begin{funcdesc}{b_size}{}
Returns the size of a bullet file.
\end{funcdesc}

\begin{funcdesc}{dir_append}{}
\funcline{dir_delete}{}
\funcline{dir_lookup}{}
\funcline{dir_replace}{}
Like the corresponding
\samp{name_}*
functions, but with a path relative to the capability.
(For paths beginning with a slash the capability is ignored, since this
is the defined semantics for Amoeba.)
\end{funcdesc}

\begin{funcdesc}{std_info}{}
Returns the standard info string of the object.
\end{funcdesc}

\begin{funcdesc}{tod_gettime}{}
Returns the time (in seconds since the Epoch, in UCT, as for \POSIX{}) from
a time server.
\end{funcdesc}

\begin{funcdesc}{tod_settime}{t}
Sets the time kept by a time server.
\end{funcdesc}
		% AMOEBA ONLY

\section{Introduction}
\label{intro}

The modules in this manual are available on the Apple Macintosh only.

Aside from the modules described here there are also interfaces to
various MacOS toolboxes, which are currently not extensively
described. The toolboxes for which modules exist are:
\module{AE} (Apple Events),
\module{Cm} (Component Manager),
\module{Ctl} (Control Manager),
\module{Dlg} (Dialog Manager),
\module{Evt} (Event Manager),
\module{Fm} (Font Manager),
\module{List} (List Manager),
\module{Menu} (Moenu Manager),
\module{Qd} (QuickDraw),
\module{Qt} (QuickTime),
\module{Res} (Resource Manager and Handles),
\module{Scrap} (Scrap Manager),
\module{Snd} (Sound Manager),
\module{TE} (TextEdit),
\module{Waste} (non-Apple \program{TextEdit} replacement) and
\module{Win} (Window Manager).

If applicable the module will define a number of Python objects for
the various structures declared by the toolbox, and operations will be
implemented as methods of the object. Other operations will be
implemented as functions in the module. Not all operations possible in
\C{} will also be possible in Python (callbacks are often a problem), and
parameters will occasionally be different in Python (input and output
buffers, especially). All methods and functions have a \code{__doc__}
string describing their arguments and return values, and for
additional description you are referred to \citetitle{Inside
Macintosh} or similar works.

The following modules are documented here:

\localmoduletable


\section{\module{mac} ---
         Implementations for the \module{os} module}

\declaremodule{builtin}{mac}
  \platform{Mac}
\modulesynopsis{Implementations for the \module{os} module.}


This module implements the operating system dependent functionality
provided by the standard module \module{os}\refstmodindex{os}.  It is
best accessed through the \module{os} module.

The following functions are available in this module:
\function{chdir()},
\function{close()},
\function{dup()},
\function{fdopen()},
\function{getcwd()},
\function{lseek()},
\function{listdir()},
\function{mkdir()},
\function{open()},
\function{read()},
\function{rename()},
\function{rmdir()},
\function{stat()},
\function{sync()},
\function{unlink()},
\function{write()},
as well as the exception \exception{error}. Note that the times
returned by \function{stat()} are floating-point values, like all time
values in MacPython.

One additional function is available:

\begin{funcdesc}{xstat}{path}
  This function returns the same information as \function{stat()}, but
  with three additional values appended: the size of the resource fork
  of the file and its 4-character creator and type.
\end{funcdesc}


\section{\module{macpath} ---
         MacOS path manipulation functions}

\declaremodule{standard}{macpath}
% Could be labeled \platform{Mac}, but the module should work anywhere and
% is distributed with the standard library.
\modulesynopsis{MacOS path manipulation functions.}


This module is the Macintosh implementation of the \module{os.path}
module.  It is most portably accessed as
\module{os.path}\refstmodindex{os.path}.  Refer to the
\citetitle[../lib/lib.html]{Python Library Reference} for
documentation of \module{os.path}.

The following functions are available in this module:
\function{normcase()},
\function{normpath()},
\function{isabs()},
\function{join()},
\function{split()},
\function{isdir()},
\function{isfile()},
\function{walk()},
\function{exists()}.
For other functions available in \module{os.path} dummy counterparts
are available.
			% MACINTOSH ONLY
\section{Built-in Module \sectcode{ctb}}
\bimodindex{ctb}
\renewcommand{\indexsubitem}{(in module ctb)}

This module provides a partial interface to the Macintosh
Communications Toolbox. Currently, only Connection Manager tools are
supported. 

\begin{datadesc}{error}
The exception raised on errors.
\end{datadesc}

\begin{datadesc}{cmData}
\dataline{cmCntl}
\dataline{cmAttn}
Flags for the \var{channel} argument of the \var{Read} and \var{Write}
methods.
\end{datadesc}

\begin{datadesc}{cmFlagsEOM}
End-of-message flag for \var{Read} and \var{Write}.
\end{datadesc}

\begin{datadesc}{choose*}
Values returned by \var{Choose}.
\end{datadesc}

\begin{datadesc}{cmStatus*}
Bits in the status as returned by \var{Status}.
\end{datadesc}

\begin{funcdesc}{available}{}
Return 1 if the communication toolbox is available, zero otherwise.
\end{funcdesc}

\begin{funcdesc}{CMNew}{name\, sizes}
Create a connection object using the connection tool named
\var{name}. \var{sizes} is a 6-tuple given buffer sizes for data in,
data out, control in, control out, attention in and attention out.
Alternatively, passing \code{None} will result in default buffer sizes.
\end{funcdesc}

\subsection{connection object}
For all connection methods that take a \var{timeout} argument, a value
of \code{-1} is indefinite, meaning that the command runs to completion.

\renewcommand{\indexsubitem}{(connection object attribute)}

\begin{datadesc}{callback}
If this member is set to a value other than \code{None} it should point
to a function accepting a single argument (the connection
object). This will make all connection object methods work
asynchronously, with the callback routine being called upon
completion.

{\em Note:} for reasons beyond my understanding the callback routine
is currently never called. You are advised against using asynchronous
calls for the time being.
\end{datadesc}


\renewcommand{\indexsubitem}{(connection object method)}

\begin{funcdesc}{Open}{timeout}
Open an outgoing connection, waiting at most \var{timeout} seconds for
the connection to be established.
\end{funcdesc}

\begin{funcdesc}{Listen}{timeout}
Wait for an incoming connection. Stop waiting after \var{timeout}
seconds. This call is only meaningful to some tools.
\end{funcdesc}

\begin{funcdesc}{accept}{yesno}
Accept (when \var{yesno} is non-zero) or reject an incoming call after
\var{Listen} returned.
\end{funcdesc}

\begin{funcdesc}{Close}{timeout\, now}
Close a connection. When \var{now} is zero, the close is orderly
(i.e.\ outstanding output is flushed, etc.)\ with a timeout of
\var{timeout} seconds. When \var{now} is non-zero the close is
immediate, discarding output.
\end{funcdesc}

\begin{funcdesc}{Read}{len\, chan\, timeout}
Read \var{len} bytes, or until \var{timeout} seconds have passed, from
the channel \var{chan} (which is one of \var{cmData}, \var{cmCntl} or
\var{cmAttn}). Return a 2-tuple:\ the data read and the end-of-message
flag.
\end{funcdesc}

\begin{funcdesc}{Write}{buf\, chan\, timeout\, eom}
Write \var{buf} to channel \var{chan}, aborting after \var{timeout}
seconds. When \var{eom} has the value \var{cmFlagsEOM} an
end-of-message indicator will be written after the data (if this
concept has a meaning for this communication tool). The method returns
the number of bytes written.
\end{funcdesc}

\begin{funcdesc}{Status}{}
Return connection status as the 2-tuple \code{(\var{sizes},
\var{flags})}. \var{sizes} is a 6-tuple giving the actual buffer sizes used
(see \var{CMNew}), \var{flags} is a set of bits describing the state
of the connection.
\end{funcdesc}

\begin{funcdesc}{GetConfig}{}
Return the configuration string of the communication tool. These
configuration strings are tool-dependent, but usually easily parsed
and modified.
\end{funcdesc}

\begin{funcdesc}{SetConfig}{str}
Set the configuration string for the tool. The strings are parsed
left-to-right, with later values taking precedence. This means
individual configuration parameters can be modified by simply appending
something like \code{'baud 4800'} to the end of the string returned by
\var{GetConfig} and passing that to this method. The method returns
the number of characters actually parsed by the tool before it
encountered an error (or completed successfully).
\end{funcdesc}

\begin{funcdesc}{Choose}{}
Present the user with a dialog to choose a communication tool and
configure it. If there is an outstanding connection some choices (like
selecting a different tool) may cause the connection to be
aborted. The return value (one of the \var{choose*} constants) will
indicate this.
\end{funcdesc}

\begin{funcdesc}{Idle}{}
Give the tool a chance to use the processor. You should call this
method regularly.
\end{funcdesc}

\begin{funcdesc}{Abort}{}
Abort an outstanding asynchronous \var{Open} or \var{Listen}.
\end{funcdesc}

\begin{funcdesc}{Reset}{}
Reset a connection. Exact meaning depends on the tool.
\end{funcdesc}

\begin{funcdesc}{Break}{length}
Send a break. Whether this means anything, what it means and
interpretation of the \var{length} parameter depend on the tool in
use.
\end{funcdesc}

\section{\module{macconsole} ---
         Think C's console package}

\declaremodule{builtin}{macconsole}
  \platform{Mac}
\modulesynopsis{Think C's console package.}


This module is available on the Macintosh, provided Python has been
built using the Think C compiler. It provides an interface to the
Think console package, with which basic text windows can be created.

\begin{datadesc}{options}
An object allowing you to set various options when creating windows,
see below.
\end{datadesc}

\begin{datadesc}{C_ECHO}
\dataline{C_NOECHO}
\dataline{C_CBREAK}
\dataline{C_RAW}
Options for the \code{setmode} method. \constant{C_ECHO} and
\constant{C_CBREAK} enable character echo, the other two disable it,
\constant{C_ECHO} and \constant{C_NOECHO} enable line-oriented input
(erase/kill processing, etc).
\end{datadesc}

\begin{funcdesc}{copen}{}
Open a new console window. Return a console window object.
\end{funcdesc}

\begin{funcdesc}{fopen}{fp}
Return the console window object corresponding with the given file
object. \var{fp} should be one of \code{sys.stdin}, \code{sys.stdout} or
\code{sys.stderr}.
\end{funcdesc}

\subsection{macconsole options object}
These options are examined when a window is created:

\setindexsubitem{(macconsole option)}
\begin{datadesc}{top}
\dataline{left}
The origin of the window.
\end{datadesc}

\begin{datadesc}{nrows}
\dataline{ncols}
The size of the window.
\end{datadesc}

\begin{datadesc}{txFont}
\dataline{txSize}
\dataline{txStyle}
The font, fontsize and fontstyle to be used in the window.
\end{datadesc}

\begin{datadesc}{title}
The title of the window.
\end{datadesc}

\begin{datadesc}{pause_atexit}
If set non-zero, the window will wait for user action before closing.
\end{datadesc}

\subsection{console window object}

\setindexsubitem{(console window attribute)}

\begin{datadesc}{file}
The file object corresponding to this console window. If the file is
buffered, you should call \code{\var{file}.flush()} between
\code{write()} and \code{read()} calls.
\end{datadesc}

\setindexsubitem{(console window method)}

\begin{funcdesc}{setmode}{mode}
Set the input mode of the console to \constant{C_ECHO}, etc.
\end{funcdesc}

\begin{funcdesc}{settabs}{n}
Set the tabsize to \var{n} spaces.
\end{funcdesc}

\begin{funcdesc}{cleos}{}
Clear to end-of-screen.
\end{funcdesc}

\begin{funcdesc}{cleol}{}
Clear to end-of-line.
\end{funcdesc}

\begin{funcdesc}{inverse}{onoff}
Enable inverse-video mode:\ characters with the high bit set are
displayed in inverse video (this disables the upper half of a
non-\ASCII{} character set).
\end{funcdesc}

\begin{funcdesc}{gotoxy}{x, y}
Set the cursor to position \code{(\var{x}, \var{y})}.
\end{funcdesc}

\begin{funcdesc}{hide}{}
Hide the window, remembering the contents.
\end{funcdesc}

\begin{funcdesc}{show}{}
Show the window again.
\end{funcdesc}

\begin{funcdesc}{echo2printer}{}
Copy everything written to the window to the printer as well.
\end{funcdesc}


\section{Built-in Module \sectcode{macdnr}}
\bimodindex{macdnr}

This module provides an interface to the Macintosh Domain Name
Resolver. It is usually used in conjunction with the \var{mactcp} module, to
map hostnames to IP-addresses.

The \code{macdnr} module defines the following functions:

\renewcommand{\indexsubitem}{(in module macdnr)}

\begin{funcdesc}{Open}{\optional{filename}}
Open the domain name resolver extension. If \var{filename} is given it
should be the pathname of the extension, otherwise a default is
used. Normally, this call is not needed since the other calls will
open the extension automatically.
\end{funcdesc}

\begin{funcdesc}{Close}{}
Close the resolver extension. Again, not needed for normal use.
\end{funcdesc}

\begin{funcdesc}{StrToAddr}{hostname}
Look up the IP address for \var{hostname}. This call returns a dnr
result object of the ``address'' variation.
\end{funcdesc}

\begin{funcdesc}{AddrToName}{addr}
Do a reverse lookup on the 32-bit integer IP-address
\var{addr}. Returns a dnr result object of the ``address'' variation.
\end{funcdesc}

\begin{funcdesc}{AddrToStr}{addr}
Convert the 32-bit integer IP-address \var{addr} to a dotted-decimal
string. Returns the string.
\end{funcdesc}

\begin{funcdesc}{HInfo}{hostname}
Query the nameservers for a \code{HInfo} record for host
\var{hostname}. These records contain hardware and software
information about the machine in question (if they are available in
the first place). Returns a dnr result object of the ``hinfo''
variety.
\end{funcdesc}

\begin{funcdesc}{MXInfo}{domain}
Query the nameservers for a mail exchanger for \var{domain}. This is
the hostname of a host willing to accept SMTP mail for the given
domain. Returns a dnr result object of the ``mx'' variety.
\end{funcdesc}

\subsection{dnr result object}

Since the DNR calls all execute asynchronously you do not get the
results back immedeately. In stead, you get a dnr result object. You
can check this object to see whether the query is complete, and access
its attributes to obtain the information when it is.

Alternatively, you can also reference the result attributes directly,
this will result in an implicit wait for the query to complete.

The \var{rtnCode} and \var{cname} attributes are always available, the
others depend on the type of query (address, hinfo or mx).

\renewcommand{\indexsubitem}{(dnr result object method)}

% Add args, as in {arg1\, arg2 \optional{\, arg3}}
\begin{funcdesc}{wait}{}
Wait for the query to complete.
\end{funcdesc}

% Add args, as in {arg1\, arg2 \optional{\, arg3}}
\begin{funcdesc}{isdone}{}
Return 1 if the query is complete.
\end{funcdesc}

\renewcommand{\indexsubitem}{(dnr result object attribute)}

\begin{datadesc}{rtnCode}
The error code returned by the query.
\end{datadesc}

\begin{datadesc}{cname}
The canonical name of the host that was queried.
\end{datadesc}

\begin{datadesc}{ip0}
\dataline{ip1}
\dataline{ip2}
\dataline{ip3}
At most four integer IP addresses for this host. Unused entries are
zero. Valid only for address queries.
\end{datadesc}

\begin{datadesc}{cpuType}
\dataline{osType}
Textual strings giving the machine type an OS name. Valid for hinfo
queries.
\end{datadesc}

\begin{datadesc}{exchange}
The name of a mail-exchanger host. Valid for mx queries.
\end{datadesc}

\begin{datadesc}{preference}
The preference of this mx record. Not too useful, since the Macintosh
will only return a single mx record. Mx queries only.
\end{datadesc}

The simplest way to use the module to convert names to dotted-decimal
strings, without worrying about idle time, etc:
\begin{verbatim}
>>> def gethostname(name):
...     import macdnr
...     dnrr = macdnr.StrToAddr(name)
...     return macdnr.AddrToStr(dnrr.ip0)
\end{verbatim}

\section{Built-in Module \sectcode{macfs}}
\bimodindex{macfs}

\renewcommand{\indexsubitem}{(in module macfs)}

This module provides access to macintosh FSSpec handling, the Alias
Manager, finder aliases and the Standard File package.

Whenever a function or method expects a \var{file} argument, this
argument can be one of three things:\ (1) a full or partial Macintosh
pathname, (2) an FSSpec object or (3) a 3-tuple \code{(wdRefNum,
parID, name)} as described in Inside Mac VI\@. A description of aliases
and the standard file package can also be found there.

\begin{funcdesc}{FSSpec}{file}
Create an FSSpec object for the specified file.
\end{funcdesc}

\begin{funcdesc}{RawFSSpec}{data}
Create an FSSpec object given the raw data for the C structure for the
FSSpec as a string.  This is mainly useful if you have obtained an
FSSpec structure over a network.
\end{funcdesc}

\begin{funcdesc}{RawAlias}{data}
Create an Alias object given the raw data for the C structure for the
alias as a string.  This is mainly useful if you have obtained an
FSSpec structure over a network.
\end{funcdesc}

\begin{funcdesc}{FInfo}{}
Create a zero-filled FInfo object.
\end{funcdesc}

\begin{funcdesc}{ResolveAliasFile}{file}
Resolve an alias file. Returns a 3-tuple \code{(\var{fsspec}, \var{isfolder},
\var{aliased})} where \var{fsspec} is the resulting FSSpec object,
\var{isfolder} is true if \var{fsspec} points to a folder and
\var{aliased} is true if the file was an alias in the first place
(otherwise the FSSpec object for the file itself is returned).
\end{funcdesc}

\begin{funcdesc}{StandardGetFile}{\optional{type\, ...}}
Present the user with a standard ``open input file''
dialog. Optionally, you can pass up to four 4-char file types to limit
the files the user can choose from. The function returns an FSSpec
object and a flag indicating that the user completed the dialog
without cancelling.
\end{funcdesc}

\begin{funcdesc}{StandardPutFile}{prompt\, \optional{default}}
Present the user with a standard ``open output file''
dialog. \var{prompt} is the prompt string, and the optional
\var{default} argument initializes the output file name. The function
returns an FSSpec object and a flag indicating that the user completed
the dialog without cancelling.
\end{funcdesc}

\begin{funcdesc}{GetDirectory}{}
Present the user with a non-standard ``select a directory''
dialog. Return an FSSpec object and a success-indicator.
\end{funcdesc}

\begin{funcdesc}{FindFolder}{where\, which\, create}
Locates one of the ``special'' folders that MacOS knows about, such as
the trash or the Preferences folder. \var{Where} is the disk to search
(\code{0x8000} for the boot disk), \var{which} is the 4-char string
specifying which folder to locate. Setting \var{create} causes the
folder to be created if it does not exist. Returns a \code{(vrefnum,
dirid)} tuple. See Inside Mac VI for a complete description, including
4-char names.
\end{funcdesc}

\subsection{FSSpec objects}

\renewcommand{\indexsubitem}{(FSSpec object attribute)}
\begin{datadesc}{data}
The raw data from the FSSpec object, suitable for passing
to other applications, for instance.
\end{datadesc}

\renewcommand{\indexsubitem}{(FSSpec object method)}
\begin{funcdesc}{as_pathname}{}
Return the full pathname of the file described by the FSSpec object.
\end{funcdesc}

\begin{funcdesc}{as_tuple}{}
Return the \code{(\var{wdRefNum}, \var{parID}, \var{name})} tuple of the file described
by the FSSpec object.
\end{funcdesc}

\begin{funcdesc}{NewAlias}{\optional{file}}
Create an Alias object pointing to the file described by this
FSSpec. If the optional \var{file} parameter is present the alias
will be relative to that file, otherwise it will be absolute.
\end{funcdesc}

\begin{funcdesc}{NewAliasMinimal}{}
Create a minimal alias pointing to this file.
\end{funcdesc}

\begin{funcdesc}{GetCreatorType}{}
Return the 4-char creator and type of the file.
\end{funcdesc}

\begin{funcdesc}{SetCreatorType}{creator\, type}
Set the 4-char creator and type of the file.
\end{funcdesc}

\begin{funcdesc}{GetFInfo}{}
Return a FInfo object describing the finder info for the file.
\end{funcdesc}

\begin{funcdesc}{SetFInfo}{finfo}
Set the finder info for the file to the values specified in the
\var{finfo} object.
\end{funcdesc}

\subsection{alias objects}

\renewcommand{\indexsubitem}{(alias object attribute)}
\begin{datadesc}{data}
The raw data for the Alias record, suitable for storing in a resource
or transmitting to other programs.
\end{datadesc}

\renewcommand{\indexsubitem}{(alias object method)}
\begin{funcdesc}{Resolve}{\optional{file}}
Resolve the alias. If the alias was created as a relative alias you
should pass the file relative to which it is. Return the FSSpec for
the file pointed to and a flag indicating whether the alias object
itself was modified during the search process. 
\end{funcdesc}

\begin{funcdesc}{GetInfo}{num}
An interface to the C routine \code{GetAliasInfo()}.
\end{funcdesc}

\begin{funcdesc}{Update}{file\, \optional{file2}}
Update the alias to point to the \var{file} given. If \var{file2} is
present a relative alias will be created.
\end{funcdesc}

Note that it is currently not possible to directly manipulate a resource
as an alias object. Hence, after calling \var{Update} or after
\var{Resolve} indicates that the alias has changed the Python program
is responsible for getting the \var{data} from the alias object and
modifying the resource.


\subsection{FInfo objects}

See Inside Mac for a complete description of what the various fields
mean.

\renewcommand{\indexsubitem}{(FInfo object attribute)}
\begin{datadesc}{Creator}
The 4-char creator code of the file.
\end{datadesc}

\begin{datadesc}{Type}
The 4-char type code of the file.
\end{datadesc}

\begin{datadesc}{Flags}
The finder flags for the file as 16-bit integer.
\end{datadesc}

\begin{datadesc}{Location}
A Point giving the position of the file's icon in its folder.
\end{datadesc}

\begin{datadesc}{Fldr}
The folder the file is in (as an integer).
\end{datadesc}

\section{Built-in Module \sectcode{mactcp}}
\label{module-mactcp}
\bimodindex{mactcp}

\setindexsubitem{(in module mactcp)}

This module provides an interface to the Macintosh TCP/IP driver
MacTCP\@. There is an accompanying module \code{macdnr} which provides an
interface to the name-server (allowing you to translate hostnames to
ip-addresses), a module \code{MACTCPconst} which has symbolic names for
constants constants used by MacTCP. Since the builtin module
\code{socket} is also available on the mac it is usually easier to use
sockets in stead of the mac-specific MacTCP API.

A complete description of the MacTCP interface can be found in the
Apple MacTCP API documentation.

\begin{funcdesc}{MTU}{}
Return the Maximum Transmit Unit (the packet size) of the network
interface.
\end{funcdesc}

\begin{funcdesc}{IPAddr}{}
Return the 32-bit integer IP address of the network interface.
\end{funcdesc}

\begin{funcdesc}{NetMask}{}
Return the 32-bit integer network mask of the interface.
\end{funcdesc}

\begin{funcdesc}{TCPCreate}{size}
Create a TCP Stream object. \var{size} is the size of the receive
buffer, \code{4096} is suggested by various sources.
\end{funcdesc}

\begin{funcdesc}{UDPCreate}{size, port}
Create a UDP stream object. \var{size} is the size of the receive
buffer (and, hence, the size of the biggest datagram you can receive
on this port). \var{port} is the UDP port number you want to receive
datagrams on, a value of zero will make MacTCP select a free port.
\end{funcdesc}

\subsection{TCP Stream Objects}

\setindexsubitem{(TCP stream attribute)}

\begin{datadesc}{asr}
When set to a value different than \code{None} this should point to a
function with two integer parameters:\ an event code and a detail. This
function will be called upon network-generated events such as urgent
data arrival. In addition, it is called with eventcode
\code{MACTCP.PassiveOpenDone} when a \code{PassiveOpen} completes. This
is a Python addition to the MacTCP semantics.
It is safe to do further calls from the \code{asr}.
\end{datadesc}

\setindexsubitem{(TCP stream method)}

\begin{funcdesc}{PassiveOpen}{port}
Wait for an incoming connection on TCP port \var{port} (zero makes the
system pick a free port). The call returns immediately, and you should
use \var{wait} to wait for completion. You should not issue any method
calls other than
\code{wait}, \code{isdone} or \code{GetSockName} before the call
completes.
\end{funcdesc}

\begin{funcdesc}{wait}{}
Wait for \code{PassiveOpen} to complete.
\end{funcdesc}

\begin{funcdesc}{isdone}{}
Return 1 if a \code{PassiveOpen} has completed.
\end{funcdesc}

\begin{funcdesc}{GetSockName}{}
Return the TCP address of this side of a connection as a 2-tuple
\code{(host, port)}, both integers.
\end{funcdesc}

\begin{funcdesc}{ActiveOpen}{lport, host, rport}
Open an outgoing connection to TCP address \code{(\var{host}, \var{rport})}. Use
local port \var{lport} (zero makes the system pick a free port). This
call blocks until the connection has been established.
\end{funcdesc}

\begin{funcdesc}{Send}{buf, push, urgent}
Send data \var{buf} over the connection. \var{Push} and \var{urgent}
are flags as specified by the TCP standard.
\end{funcdesc}

\begin{funcdesc}{Rcv}{timeout}
Receive data. The call returns when \var{timeout} seconds have passed
or when (according to the MacTCP documentation) ``a reasonable amount
of data has been received''. The return value is a 3-tuple
\code{(\var{data}, \var{urgent}, \var{mark})}. If urgent data is outstanding \code{Rcv}
will always return that before looking at any normal data. The first
call returning urgent data will have the \var{urgent} flag set, the
last will have the \var{mark} flag set.
\end{funcdesc}

\begin{funcdesc}{Close}{}
Tell MacTCP that no more data will be transmitted on this
connection. The call returns when all data has been acknowledged by
the receiving side.
\end{funcdesc}

\begin{funcdesc}{Abort}{}
Forcibly close both sides of a connection, ignoring outstanding data.
\end{funcdesc}

\begin{funcdesc}{Status}{}
Return a TCP status object for this stream giving the current status
(see below).
\end{funcdesc}

\subsection{TCP Status Objects}
This object has no methods, only some members holding information on
the connection. A complete description of all fields in this objects
can be found in the Apple documentation. The most interesting ones are:

\setindexsubitem{(TCP status attribute)}

\begin{datadesc}{localHost}
\dataline{localPort}
\dataline{remoteHost}
\dataline{remotePort}
The integer IP-addresses and port numbers of both endpoints of the
connection. 
\end{datadesc}

\begin{datadesc}{sendWindow}
The current window size.
\end{datadesc}

\begin{datadesc}{amtUnackedData}
The number of bytes sent but not yet acknowledged. \code{sendWindow -
amtUnackedData} is what you can pass to \code{Send} without blocking.
\end{datadesc}

\begin{datadesc}{amtUnreadData}
The number of bytes received but not yet read (what you can \code{Recv}
without blocking).
\end{datadesc}



\subsection{UDP Stream Objects}
Note that, unlike the name suggests, there is nothing stream-like
about UDP.

\setindexsubitem{(UDP stream attribute)}

\begin{datadesc}{asr}
The asynchronous service routine to be called on events such as
datagram arrival without outstanding \code{Read} call. The \code{asr} has a
single argument, the event code.
\end{datadesc}

\begin{datadesc}{port}
A read-only member giving the port number of this UDP stream.
\end{datadesc}

\setindexsubitem{(UDP stream method)}

\begin{funcdesc}{Read}{timeout}
Read a datagram, waiting at most \var{timeout} seconds (-1 is
infinite).  Return the data.
\end{funcdesc}

\begin{funcdesc}{Write}{host, port, buf}
Send \var{buf} as a datagram to IP-address \var{host}, port
\var{port}.
\end{funcdesc}

\section{Built-in Module \sectcode{macspeech}}
\label{module-macspeech}
\bimodindex{macspeech}

\renewcommand{\indexsubitem}{(in module macspeech)}

This module provides an interface to the Macintosh Speech Manager,
allowing you to let the Macintosh utter phrases. You need a version of
the speech manager extension (version 1 and 2 have been tested) in
your \code{Extensions} folder for this to work. The module does not
provide full access to all features of the Speech Manager yet.  It may
not be available in all Mac Python versions.

\begin{funcdesc}{Available}{}
Test availability of the Speech Manager extension (and, on the
PowerPC, the Speech Manager shared library). Return 0 or 1. 
\end{funcdesc}

\begin{funcdesc}{Version}{}
Return the (integer) version number of the Speech Manager.
\end{funcdesc}

\begin{funcdesc}{SpeakString}{str}
Utter the string \var{str} using the default voice,
asynchronously. This aborts any speech that may still be active from
prior \code{SpeakString} invocations.
\end{funcdesc}

\begin{funcdesc}{Busy}{}
Return the number of speech channels busy, system-wide.
\end{funcdesc}

\begin{funcdesc}{CountVoices}{}
Return the number of different voices available.
\end{funcdesc}

\begin{funcdesc}{GetIndVoice}{num}
Return a voice object for voice number \var{num}.
\end{funcdesc}

\subsection{voice objects}
Voice objects contain the description of a voice. It is currently not
yet possible to access the parameters of a voice.

\renewcommand{\indexsubitem}{(voice object method)}

\begin{funcdesc}{GetGender}{}
Return the gender of the voice: 0 for male, 1 for female and -1 for neuter.
\end{funcdesc}

\begin{funcdesc}{NewChannel}{}
Return a new speech channel object using this voice.
\end{funcdesc}

\subsection{speech channel objects}
A speech channel object allows you to speak strings with slightly more
control than \code{SpeakString()}, and allows you to use multiple
speakers at the same time. Please note that channel pitch and rate are
interrelated in some way, so that to make your Macintosh sing you will
have to adjust both.

\renewcommand{\indexsubitem}{(speech channel object method)}
\begin{funcdesc}{SpeakText}{str}
Start uttering the given string.
\end{funcdesc}

\begin{funcdesc}{Stop}{}
Stop babbling.
\end{funcdesc}

\begin{funcdesc}{GetPitch}{}
Return the current pitch of the channel, as a floating-point number.
\end{funcdesc}

\begin{funcdesc}{SetPitch}{pitch}
Set the pitch of the channel.
\end{funcdesc}

\begin{funcdesc}{GetRate}{}
Get the speech rate (utterances per minute) of the channel as a
floating point number.
\end{funcdesc}

\begin{funcdesc}{SetRate}{rate}
Set the speech rate of the channel.
\end{funcdesc}



\chapter{Standard Windowing Interface}

The modules in this chapter are available only on those systems where
the STDWIN library is available.  STDWIN runs on \UNIX{} under X11 and
on the Macintosh.  See CWI report CS-R8817.

\strong{Warning:} Using STDWIN is not recommended for new
applications.  It has never been ported to Microsoft Windows or
Windows NT, and for X11 or the Macintosh it lacks important
functionality --- in particular, it has no tools for the construction
of dialogs.  For most platforms, alternative, native solutions exist
(though none are currently documented in this manual): Tkinter for
\UNIX{} under X11, native Xt with Motif or Athena widgets for \UNIX{}
under X11, Win32 for Windows and Windows NT, and a collection of
native toolkit interfaces for the Macintosh.

\section{Built-in Module \sectcode{stdwin}}
\bimodindex{stdwin}

This module defines several new object types and functions that
provide access to the functionality of STDWIN.

On Unix running X11, it can only be used if the \code{DISPLAY}
environment variable is set or an explicit \samp{-display
\var{displayname}} argument is passed to the Python interpreter.

Functions have names that usually resemble their C STDWIN counterparts
with the initial `w' dropped.
Points are represented by pairs of integers; rectangles
by pairs of points.
For a complete description of STDWIN please refer to the documentation
of STDWIN for C programmers (aforementioned CWI report).

\subsection{Functions Defined in Module \sectcode{stdwin}}
\nodename{STDWIN Functions}

The following functions are defined in the \code{stdwin} module:

\renewcommand{\indexsubitem}{(in module stdwin)}
\begin{funcdesc}{open}{title}
Open a new window whose initial title is given by the string argument.
Return a window object; window object methods are described below.%
\footnote{The Python version of STDWIN does not support draw procedures; all
	drawing requests are reported as draw events.}
\end{funcdesc}

\begin{funcdesc}{getevent}{}
Wait for and return the next event.
An event is returned as a triple: the first element is the event
type, a small integer; the second element is the window object to which
the event applies, or
\code{None}
if it applies to no window in particular;
the third element is type-dependent.
Names for event types and command codes are defined in the standard
module
\code{stdwinevent}.
\end{funcdesc}

\begin{funcdesc}{pollevent}{}
Return the next event, if one is immediately available.
If no event is available, return \code{()}.
\end{funcdesc}

\begin{funcdesc}{getactive}{}
Return the window that is currently active, or \code{None} if no
window is currently active.  (This can be emulated by monitoring
WE_ACTIVATE and WE_DEACTIVATE events.)
\end{funcdesc}

\begin{funcdesc}{listfontnames}{pattern}
Return the list of font names in the system that match the pattern (a
string).  The pattern should normally be \code{'*'}; returns all
available fonts.  If the underlying window system is X11, other
patterns follow the standard X11 font selection syntax (as used e.g.
in resource definitions), i.e. the wildcard character \code{'*'}
matches any sequence of characters (including none) and \code{'?'}
matches any single character.
On the Macintosh this function currently returns an empty list.
\end{funcdesc}

\begin{funcdesc}{setdefscrollbars}{hflag\, vflag}
Set the flags controlling whether subsequently opened windows will
have horizontal and/or vertical scroll bars.
\end{funcdesc}

\begin{funcdesc}{setdefwinpos}{h\, v}
Set the default window position for windows opened subsequently.
\end{funcdesc}

\begin{funcdesc}{setdefwinsize}{width\, height}
Set the default window size for windows opened subsequently.
\end{funcdesc}

\begin{funcdesc}{getdefscrollbars}{}
Return the flags controlling whether subsequently opened windows will
have horizontal and/or vertical scroll bars.
\end{funcdesc}

\begin{funcdesc}{getdefwinpos}{}
Return the default window position for windows opened subsequently.
\end{funcdesc}

\begin{funcdesc}{getdefwinsize}{}
Return the default window size for windows opened subsequently.
\end{funcdesc}

\begin{funcdesc}{getscrsize}{}
Return the screen size in pixels.
\end{funcdesc}

\begin{funcdesc}{getscrmm}{}
Return the screen size in millimeters.
\end{funcdesc}

\begin{funcdesc}{fetchcolor}{colorname}
Return the pixel value corresponding to the given color name.
Return the default foreground color for unknown color names.
Hint: the following code tests whether you are on a machine that
supports more than two colors:
\bcode\begin{verbatim}
if stdwin.fetchcolor('black') <> \
          stdwin.fetchcolor('red') <> \
          stdwin.fetchcolor('white'):
    print 'color machine'
else:
    print 'monochrome machine'
\end{verbatim}\ecode
\end{funcdesc}

\begin{funcdesc}{setfgcolor}{pixel}
Set the default foreground color.
This will become the default foreground color of windows opened
subsequently, including dialogs.
\end{funcdesc}

\begin{funcdesc}{setbgcolor}{pixel}
Set the default background color.
This will become the default background color of windows opened
subsequently, including dialogs.
\end{funcdesc}

\begin{funcdesc}{getfgcolor}{}
Return the pixel value of the current default foreground color.
\end{funcdesc}

\begin{funcdesc}{getbgcolor}{}
Return the pixel value of the current default background color.
\end{funcdesc}

\begin{funcdesc}{setfont}{fontname}
Set the current default font.
This will become the default font for windows opened subsequently,
and is also used by the text measuring functions \code{textwidth},
\code{textbreak}, \code{lineheight} and \code{baseline} below.
This accepts two more optional parameters, size and style:
Size is the font size (in `points').
Style is a single character specifying the style, as follows:
\code{'b'} = bold,
\code{'i'} = italic,
\code{'o'} = bold + italic,
\code{'u'} = underline;
default style is roman.
Size and style are ignored under X11 but used on the Macintosh.
(Sorry for all this complexity --- a more uniform interface is being designed.)
\end{funcdesc}

\begin{funcdesc}{menucreate}{title}
Create a menu object referring to a global menu (a menu that appears in
all windows).
Methods of menu objects are described below.
Note: normally, menus are created locally; see the window method
\code{menucreate} below.
\strong{Warning:} the menu only appears in a window as long as the object
returned by this call exists.
\end{funcdesc}

\begin{funcdesc}{newbitmap}{width\, height}
Create a new bitmap object of the given dimensions.
Methods of bitmap objects are described below.
Not available on the Macintosh.
\end{funcdesc}

\begin{funcdesc}{fleep}{}
Cause a beep or bell (or perhaps a `visual bell' or flash, hence the
name).
\end{funcdesc}

\begin{funcdesc}{message}{string}
Display a dialog box containing the string.
The user must click OK before the function returns.
\end{funcdesc}

\begin{funcdesc}{askync}{prompt\, default}
Display a dialog that prompts the user to answer a question with yes or
no.
Return 0 for no, 1 for yes.
If the user hits the Return key, the default (which must be 0 or 1) is
returned.
If the user cancels the dialog, the
\code{KeyboardInterrupt}
exception is raised.
\end{funcdesc}

\begin{funcdesc}{askstr}{prompt\, default}
Display a dialog that prompts the user for a string.
If the user hits the Return key, the default string is returned.
If the user cancels the dialog, the
\code{KeyboardInterrupt}
exception is raised.
\end{funcdesc}

\begin{funcdesc}{askfile}{prompt\, default\, new}
Ask the user to specify a filename.
If
\var{new}
is zero it must be an existing file; otherwise, it must be a new file.
If the user cancels the dialog, the
\code{KeyboardInterrupt}
exception is raised.
\end{funcdesc}

\begin{funcdesc}{setcutbuffer}{i\, string}
Store the string in the system's cut buffer number
\var{i},
where it can be found (for pasting) by other applications.
On X11, there are 8 cut buffers (numbered 0..7).
Cut buffer number 0 is the `clipboard' on the Macintosh.
\end{funcdesc}

\begin{funcdesc}{getcutbuffer}{i}
Return the contents of the system's cut buffer number
\var{i}.
\end{funcdesc}

\begin{funcdesc}{rotatecutbuffers}{n}
On X11, rotate the 8 cut buffers by
\var{n}.
Ignored on the Macintosh.
\end{funcdesc}

\begin{funcdesc}{getselection}{i}
Return X11 selection number
\var{i.}
Selections are not cut buffers.
Selection numbers are defined in module
\code{stdwinevents}.
Selection \code{WS_PRIMARY} is the
\dfn{primary}
selection (used by
xterm,
for instance);
selection \code{WS_SECONDARY} is the
\dfn{secondary}
selection; selection \code{WS_CLIPBOARD} is the
\dfn{clipboard}
selection (used by
xclipboard).
On the Macintosh, this always returns an empty string.
\end{funcdesc}

\begin{funcdesc}{resetselection}{i}
Reset selection number
\var{i},
if this process owns it.
(See window method
\code{setselection()}).
\end{funcdesc}

\begin{funcdesc}{baseline}{}
Return the baseline of the current font (defined by STDWIN as the
vertical distance between the baseline and the top of the
characters).
\end{funcdesc}

\begin{funcdesc}{lineheight}{}
Return the total line height of the current font.
\end{funcdesc}

\begin{funcdesc}{textbreak}{str\, width}
Return the number of characters of the string that fit into a space of
\var{width}
bits wide when drawn in the curent font.
\end{funcdesc}

\begin{funcdesc}{textwidth}{str}
Return the width in bits of the string when drawn in the current font.
\end{funcdesc}

\begin{funcdesc}{connectionnumber}{}
\funcline{fileno}{}
(X11 under \UNIX{} only) Return the ``connection number'' used by the
underlying X11 implementation.  (This is normally the file number of
the socket.)  Both functions return the same value;
\code{connectionnumber()} is named after the corresponding function in
X11 and STDWIN, while \code{fileno()} makes it possible to use the
\code{stdwin} module as a ``file'' object parameter to
\code{select.select()}.  Note that if \code{select()} implies that
input is possible on \code{stdwin}, this does not guarantee that an
event is ready --- it may be some internal communication going on
between the X server and the client library.  Thus, you should call
\code{stdwin.pollevent()} until it returns \code{None} to check for
events if you don't want your program to block.  Because of internal
buffering in X11, it is also possible that \code{stdwin.pollevent()}
returns an event while \code{select()} does not find \code{stdwin} to
be ready, so you should read any pending events with
\code{stdwin.pollevent()} until it returns \code{None} before entering
a blocking \code{select()} call.
\ttindex{select}
\end{funcdesc}

\subsection{Window Objects}

Window objects are created by \code{stdwin.open()}.  They are closed
by their \code{close()} method or when they are garbage-collected.
Window objects have the following methods:

\renewcommand{\indexsubitem}{(window method)}

\begin{funcdesc}{begindrawing}{}
Return a drawing object, whose methods (described below) allow drawing
in the window.
\end{funcdesc}

\begin{funcdesc}{change}{rect}
Invalidate the given rectangle; this may cause a draw event.
\end{funcdesc}

\begin{funcdesc}{gettitle}{}
Returns the window's title string.
\end{funcdesc}

\begin{funcdesc}{getdocsize}{}
\begin{sloppypar}
Return a pair of integers giving the size of the document as set by
\code{setdocsize()}.
\end{sloppypar}
\end{funcdesc}

\begin{funcdesc}{getorigin}{}
Return a pair of integers giving the origin of the window with respect
to the document.
\end{funcdesc}

\begin{funcdesc}{gettitle}{}
Return the window's title string.
\end{funcdesc}

\begin{funcdesc}{getwinsize}{}
Return a pair of integers giving the size of the window.
\end{funcdesc}

\begin{funcdesc}{getwinpos}{}
Return a pair of integers giving the position of the window's upper
left corner (relative to the upper left corner of the screen).
\end{funcdesc}

\begin{funcdesc}{menucreate}{title}
Create a menu object referring to a local menu (a menu that appears
only in this window).
Methods of menu objects are described below.
{\bf Warning:} the menu only appears as long as the object
returned by this call exists.
\end{funcdesc}

\begin{funcdesc}{scroll}{rect\, point}
Scroll the given rectangle by the vector given by the point.
\end{funcdesc}

\begin{funcdesc}{setdocsize}{point}
Set the size of the drawing document.
\end{funcdesc}

\begin{funcdesc}{setorigin}{point}
Move the origin of the window (its upper left corner)
to the given point in the document.
\end{funcdesc}

\begin{funcdesc}{setselection}{i\, str}
Attempt to set X11 selection number
\var{i}
to the string
\var{str}.
(See stdwin method
\code{getselection()}
for the meaning of
\var{i}.)
Return true if it succeeds.
If  succeeds, the window ``owns'' the selection until
(a) another application takes ownership of the selection; or
(b) the window is deleted; or
(c) the application clears ownership by calling
\code{stdwin.resetselection(\var{i})}.
When another application takes ownership of the selection, a
\code{WE_LOST_SEL}
event is received for no particular window and with the selection number
as detail.
Ignored on the Macintosh.
\end{funcdesc}

\begin{funcdesc}{settimer}{dsecs}
Schedule a timer event for the window in
\code{\var{dsecs}/10}
seconds.
\end{funcdesc}

\begin{funcdesc}{settitle}{title}
Set the window's title string.
\end{funcdesc}

\begin{funcdesc}{setwincursor}{name}
\begin{sloppypar}
Set the window cursor to a cursor of the given name.
It raises the
\code{RuntimeError}
exception if no cursor of the given name exists.
Suitable names include
\code{'ibeam'},
\code{'arrow'},
\code{'cross'},
\code{'watch'}
and
\code{'plus'}.
On X11, there are many more (see
\file{<X11/cursorfont.h>}).
\end{sloppypar}
\end{funcdesc}

\begin{funcdesc}{setwinpos}{h\, v}
Set the the position of the window's upper left corner (relative to
the upper left corner of the screen).
\end{funcdesc}

\begin{funcdesc}{setwinsize}{width\, height}
Set the window's size.
\end{funcdesc}

\begin{funcdesc}{show}{rect}
Try to ensure that the given rectangle of the document is visible in
the window.
\end{funcdesc}

\begin{funcdesc}{textcreate}{rect}
Create a text-edit object in the document at the given rectangle.
Methods of text-edit objects are described below.
\end{funcdesc}

\begin{funcdesc}{setactive}{}
Attempt to make this window the active window.  If successful, this
will generate a WE_ACTIVATE event (and a WE_DEACTIVATE event in case
another window in this application became inactive).
\end{funcdesc}

\begin{funcdesc}{close}{}
Discard the window object.  It should not be used again.
\end{funcdesc}

\subsection{Drawing Objects}

Drawing objects are created exclusively by the window method
\code{begindrawing()}.
Only one drawing object can exist at any given time; the drawing object
must be deleted to finish drawing.
No drawing object may exist when
\code{stdwin.getevent()}
is called.
Drawing objects have the following methods:

\renewcommand{\indexsubitem}{(drawing method)}

\begin{funcdesc}{box}{rect}
Draw a box just inside a rectangle.
\end{funcdesc}

\begin{funcdesc}{circle}{center\, radius}
Draw a circle with given center point and radius.
\end{funcdesc}

\begin{funcdesc}{elarc}{center\, \(rh\, rv\)\, \(a1\, a2\)}
Draw an elliptical arc with given center point.
\code{(\var{rh}, \var{rv})}
gives the half sizes of the horizontal and vertical radii.
\code{(\var{a1}, \var{a2})}
gives the angles (in degrees) of the begin and end points.
0 degrees is at 3 o'clock, 90 degrees is at 12 o'clock.
\end{funcdesc}

\begin{funcdesc}{erase}{rect}
Erase a rectangle.
\end{funcdesc}

\begin{funcdesc}{fillcircle}{center\, radius}
Draw a filled circle with given center point and radius.
\end{funcdesc}

\begin{funcdesc}{fillelarc}{center\, \(rh\, rv\)\, \(a1\, a2\)}
Draw a filled elliptical arc; arguments as for \code{elarc}.
\end{funcdesc}

\begin{funcdesc}{fillpoly}{points}
Draw a filled polygon given by a list (or tuple) of points.
\end{funcdesc}

\begin{funcdesc}{invert}{rect}
Invert a rectangle.
\end{funcdesc}

\begin{funcdesc}{line}{p1\, p2}
Draw a line from point
\var{p1}
to
\var{p2}.
\end{funcdesc}

\begin{funcdesc}{paint}{rect}
Fill a rectangle.
\end{funcdesc}

\begin{funcdesc}{poly}{points}
Draw the lines connecting the given list (or tuple) of points.
\end{funcdesc}

\begin{funcdesc}{shade}{rect\, percent}
Fill a rectangle with a shading pattern that is about
\var{percent}
percent filled.
\end{funcdesc}

\begin{funcdesc}{text}{p\, str}
Draw a string starting at point p (the point specifies the
top left coordinate of the string).
\end{funcdesc}

\begin{funcdesc}{xorcircle}{center\, radius}
\funcline{xorelarc}{center\, \(rh\, rv\)\, \(a1\, a2\)}
\funcline{xorline}{p1\, p2}
\funcline{xorpoly}{points}
Draw a circle, an elliptical arc, a line or a polygon, respectively,
in XOR mode.
\end{funcdesc}

\begin{funcdesc}{setfgcolor}{}
\funcline{setbgcolor}{}
\funcline{getfgcolor}{}
\funcline{getbgcolor}{}
These functions are similar to the corresponding functions described
above for the
\code{stdwin}
module, but affect or return the colors currently used for drawing
instead of the global default colors.
When a drawing object is created, its colors are set to the window's
default colors, which are in turn initialized from the global default
colors when the window is created.
\end{funcdesc}

\begin{funcdesc}{setfont}{}
\funcline{baseline}{}
\funcline{lineheight}{}
\funcline{textbreak}{}
\funcline{textwidth}{}
These functions are similar to the corresponding functions described
above for the
\code{stdwin}
module, but affect or use the current drawing font instead of
the global default font.
When a drawing object is created, its font is set to the window's
default font, which is in turn initialized from the global default
font when the window is created.
\end{funcdesc}

\begin{funcdesc}{bitmap}{point\, bitmap\, mask}
Draw the \var{bitmap} with its top left corner at \var{point}.
If the optional \var{mask} argument is present, it should be either
the same object as \var{bitmap}, to draw only those bits that are set
in the bitmap, in the foreground color, or \code{None}, to draw all
bits (ones are drawn in the foreground color, zeros in the background
color).
Not available on the Macintosh.
\end{funcdesc}

\begin{funcdesc}{cliprect}{rect}
Set the ``clipping region'' to a rectangle.
The clipping region limits the effect of all drawing operations, until
it is changed again or until the drawing object is closed.  When a
drawing object is created the clipping region is set to the entire
window.  When an object to be drawn falls partly outside the clipping
region, the set of pixels drawn is the intersection of the clipping
region and the set of pixels that would be drawn by the same operation
in the absence of a clipping region.
\end{funcdesc}

\begin{funcdesc}{noclip}{}
Reset the clipping region to the entire window.
\end{funcdesc}

\begin{funcdesc}{close}{}
\funcline{enddrawing}{}
Discard the drawing object.  It should not be used again.
\end{funcdesc}

\subsection{Menu Objects}

A menu object represents a menu.
The menu is destroyed when the menu object is deleted.
The following methods are defined:

\renewcommand{\indexsubitem}{(menu method)}

\begin{funcdesc}{additem}{text\, shortcut}
Add a menu item with given text.
The shortcut must be a string of length 1, or omitted (to specify no
shortcut).
\end{funcdesc}

\begin{funcdesc}{setitem}{i\, text}
Set the text of item number
\var{i}.
\end{funcdesc}

\begin{funcdesc}{enable}{i\, flag}
Enable or disables item
\var{i}.
\end{funcdesc}

\begin{funcdesc}{check}{i\, flag}
Set or clear the
\dfn{check mark}
for item
\var{i}.
\end{funcdesc}

\begin{funcdesc}{close}{}
Discard the menu object.  It should not be used again.
\end{funcdesc}

\subsection{Bitmap Objects}

A bitmap represents a rectangular array of bits.
The top left bit has coordinate (0, 0).
A bitmap can be drawn with the \code{bitmap} method of a drawing object.
Bitmaps are currently not available on the Macintosh.

The following methods are defined:

\renewcommand{\indexsubitem}{(bitmap method)}

\begin{funcdesc}{getsize}{}
Return a tuple representing the width and height of the bitmap.
(This returns the values that have been passed to the \code{newbitmap}
function.)
\end{funcdesc}

\begin{funcdesc}{setbit}{point\, bit}
Set the value of the bit indicated by \var{point} to \var{bit}.
\end{funcdesc}

\begin{funcdesc}{getbit}{point}
Return the value of the bit indicated by \var{point}.
\end{funcdesc}

\begin{funcdesc}{close}{}
Discard the bitmap object.  It should not be used again.
\end{funcdesc}

\subsection{Text-edit Objects}

A text-edit object represents a text-edit block.
For semantics, see the STDWIN documentation for C programmers.
The following methods exist:

\renewcommand{\indexsubitem}{(text-edit method)}

\begin{funcdesc}{arrow}{code}
Pass an arrow event to the text-edit block.
The
\var{code}
must be one of
\code{WC_LEFT},
\code{WC_RIGHT},
\code{WC_UP}
or
\code{WC_DOWN}
(see module
\code{stdwinevents}).
\end{funcdesc}

\begin{funcdesc}{draw}{rect}
Pass a draw event to the text-edit block.
The rectangle specifies the redraw area.
\end{funcdesc}

\begin{funcdesc}{event}{type\, window\, detail}
Pass an event gotten from
\code{stdwin.getevent()}
to the text-edit block.
Return true if the event was handled.
\end{funcdesc}

\begin{funcdesc}{getfocus}{}
Return 2 integers representing the start and end positions of the
focus, usable as slice indices on the string returned by
\code{gettext()}.
\end{funcdesc}

\begin{funcdesc}{getfocustext}{}
Return the text in the focus.
\end{funcdesc}

\begin{funcdesc}{getrect}{}
Return a rectangle giving the actual position of the text-edit block.
(The bottom coordinate may differ from the initial position because
the block automatically shrinks or grows to fit.)
\end{funcdesc}

\begin{funcdesc}{gettext}{}
Return the entire text buffer.
\end{funcdesc}

\begin{funcdesc}{move}{rect}
Specify a new position for the text-edit block in the document.
\end{funcdesc}

\begin{funcdesc}{replace}{str}
Replace the text in the focus by the given string.
The new focus is an insert point at the end of the string.
\end{funcdesc}

\begin{funcdesc}{setfocus}{i\, j}
Specify the new focus.
Out-of-bounds values are silently clipped.
\end{funcdesc}

\begin{funcdesc}{settext}{str}
Replace the entire text buffer by the given string and set the focus
to \code{(0, 0)}.
\end{funcdesc}

\begin{funcdesc}{setview}{rect}
Set the view rectangle to \var{rect}.  If \var{rect} is \code{None},
viewing mode is reset.  In viewing mode, all output from the text-edit
object is clipped to the viewing rectangle.  This may be useful to
implement your own scrolling text subwindow.
\end{funcdesc}

\begin{funcdesc}{close}{}
Discard the text-edit object.  It should not be used again.
\end{funcdesc}

\subsection{Example}
\nodename{STDWIN Example}

Here is a minimal example of using STDWIN in Python.
It creates a window and draws the string ``Hello world'' in the top
left corner of the window.
The window will be correctly redrawn when covered and re-exposed.
The program quits when the close icon or menu item is requested.

\bcode\begin{verbatim}
import stdwin
from stdwinevents import *

def main():
    mywin = stdwin.open('Hello')
    #
    while 1:
        (type, win, detail) = stdwin.getevent()
        if type == WE_DRAW:
            draw = win.begindrawing()
            draw.text((0, 0), 'Hello, world')
            del draw
        elif type == WE_CLOSE:
            break

main()
\end{verbatim}\ecode

\section{Standard Module \sectcode{stdwinevents}}
\stmodindex{stdwinevents}

This module defines constants used by STDWIN for event types
(\code{WE_ACTIVATE} etc.), command codes (\code{WC_LEFT} etc.)
and selection types (\code{WS_PRIMARY} etc.).
Read the file for details.
Suggested usage is

\bcode\begin{verbatim}
>>> from stdwinevents import *
>>> 
\end{verbatim}\ecode

\section{Standard Module \sectcode{rect}}
\stmodindex{rect}

This module contains useful operations on rectangles.
A rectangle is defined as in module
\code{stdwin}:
a pair of points, where a point is a pair of integers.
For example, the rectangle

\bcode\begin{verbatim}
(10, 20), (90, 80)
\end{verbatim}\ecode

is a rectangle whose left, top, right and bottom edges are 10, 20, 90
and 80, respectively.
Note that the positive vertical axis points down (as in
\code{stdwin}).

The module defines the following objects:

\renewcommand{\indexsubitem}{(in module rect)}
\begin{excdesc}{error}
The exception raised by functions in this module when they detect an
error.
The exception argument is a string describing the problem in more
detail.
\end{excdesc}

\begin{datadesc}{empty}
The rectangle returned when some operations return an empty result.
This makes it possible to quickly check whether a result is empty:

\bcode\begin{verbatim}
>>> import rect
>>> r1 = (10, 20), (90, 80)
>>> r2 = (0, 0), (10, 20)
>>> r3 = rect.intersect([r1, r2])
>>> if r3 is rect.empty: print 'Empty intersection'
Empty intersection
>>> 
\end{verbatim}\ecode
\end{datadesc}

\begin{funcdesc}{is_empty}{r}
Returns true if the given rectangle is empty.
A rectangle
\code{(\var{left}, \var{top}), (\var{right}, \var{bottom})}
is empty if
\iftexi
\code{\var{left} >= \var{right}} or \code{\var{top} => \var{bottom}}.
\else
$\var{left} \geq \var{right}$ or $\var{top} \geq \var{bottom}$.
%%JHXXX{\em left~$\geq$~right} or {\em top~$\leq$~bottom}.
\fi
\end{funcdesc}

\begin{funcdesc}{intersect}{list}
Returns the intersection of all rectangles in the list argument.
It may also be called with a tuple argument.
Raises
\code{rect.error}
if the list is empty.
Returns
\code{rect.empty}
if the intersection of the rectangles is empty.
\end{funcdesc}

\begin{funcdesc}{union}{list}
Returns the smallest rectangle that contains all non-empty rectangles in
the list argument.
It may also be called with a tuple argument or with two or more
rectangles as arguments.
Returns
\code{rect.empty}
if the list is empty or all its rectangles are empty.
\end{funcdesc}

\begin{funcdesc}{pointinrect}{point\, rect}
Returns true if the point is inside the rectangle.
By definition, a point
\code{(\var{h}, \var{v})}
is inside a rectangle
\code{(\var{left}, \var{top}), (\var{right}, \var{bottom})} if
\iftexi
\code{\var{left} <= \var{h} < \var{right}} and
\code{\var{top} <= \var{v} < \var{bottom}}.
\else
$\var{left} \leq \var{h} < \var{right}$ and
$\var{top} \leq \var{v} < \var{bottom}$.
\fi
\end{funcdesc}

\begin{funcdesc}{inset}{rect\, \(dh\, dv\)}
Returns a rectangle that lies inside the
\code{rect}
argument by
\var{dh}
pixels horizontally
and
\var{dv}
pixels
vertically.
If
\var{dh}
or
\var{dv}
is negative, the result lies outside
\var{rect}.
\end{funcdesc}

\begin{funcdesc}{rect2geom}{rect}
Converts a rectangle to geometry representation:
\code{(\var{left}, \var{top}), (\var{width}, \var{height})}.
\end{funcdesc}

\begin{funcdesc}{geom2rect}{geom}
Converts a rectangle given in geometry representation back to the
standard rectangle representation
\code{(\var{left}, \var{top}), (\var{right}, \var{bottom})}.
\end{funcdesc}
		% STDWIN ONLY

\chapter{SGI IRIX Specific Services}
\label{sgi}

The modules described in this chapter provide interfaces to features
that are unique to SGI's IRIX operating system (versions 4 and 5).

\localmoduletable
			% SGI IRIX ONLY
\section{\module{al} ---
         Audio functions on the SGI}

\declaremodule{builtin}{al}
  \platform{IRIX}
\modulesynopsis{Audio functions on the SGI.}


This module provides access to the audio facilities of the SGI Indy
and Indigo workstations.  See section 3A of the IRIX man pages for
details.  You'll need to read those man pages to understand what these
functions do!  Some of the functions are not available in IRIX
releases before 4.0.5.  Again, see the manual to check whether a
specific function is available on your platform.

All functions and methods defined in this module are equivalent to
the C functions with \samp{AL} prefixed to their name.

Symbolic constants from the C header file \code{<audio.h>} are
defined in the standard module
\refmodule[al-constants]{AL}\refstmodindex{AL}, see below.

\strong{Warning:} the current version of the audio library may dump core
when bad argument values are passed rather than returning an error
status.  Unfortunately, since the precise circumstances under which
this may happen are undocumented and hard to check, the Python
interface can provide no protection against this kind of problems.
(One example is specifying an excessive queue size --- there is no
documented upper limit.)

The module defines the following functions:


\begin{funcdesc}{openport}{name, direction\optional{, config}}
The name and direction arguments are strings.  The optional
\var{config} argument is a configuration object as returned by
\function{newconfig()}.  The return value is an \dfn{audio port
object}; methods of audio port objects are described below.
\end{funcdesc}

\begin{funcdesc}{newconfig}{}
The return value is a new \dfn{audio configuration object}; methods of
audio configuration objects are described below.
\end{funcdesc}

\begin{funcdesc}{queryparams}{device}
The device argument is an integer.  The return value is a list of
integers containing the data returned by \cfunction{ALqueryparams()}.
\end{funcdesc}

\begin{funcdesc}{getparams}{device, list}
The \var{device} argument is an integer.  The list argument is a list
such as returned by \function{queryparams()}; it is modified in place
(!).
\end{funcdesc}

\begin{funcdesc}{setparams}{device, list}
The \var{device} argument is an integer.  The \var{list} argument is a
list such as returned by \function{queryparams()}.
\end{funcdesc}


\subsection{Configuration Objects \label{al-config-objects}}

Configuration objects (returned by \function{newconfig()} have the
following methods:

\begin{methoddesc}[audio configuration]{getqueuesize}{}
Return the queue size.
\end{methoddesc}

\begin{methoddesc}[audio configuration]{setqueuesize}{size}
Set the queue size.
\end{methoddesc}

\begin{methoddesc}[audio configuration]{getwidth}{}
Get the sample width.
\end{methoddesc}

\begin{methoddesc}[audio configuration]{setwidth}{width}
Set the sample width.
\end{methoddesc}

\begin{methoddesc}[audio configuration]{getchannels}{}
Get the channel count.
\end{methoddesc}

\begin{methoddesc}[audio configuration]{setchannels}{nchannels}
Set the channel count.
\end{methoddesc}

\begin{methoddesc}[audio configuration]{getsampfmt}{}
Get the sample format.
\end{methoddesc}

\begin{methoddesc}[audio configuration]{setsampfmt}{sampfmt}
Set the sample format.
\end{methoddesc}

\begin{methoddesc}[audio configuration]{getfloatmax}{}
Get the maximum value for floating sample formats.
\end{methoddesc}

\begin{methoddesc}[audio configuration]{setfloatmax}{floatmax}
Set the maximum value for floating sample formats.
\end{methoddesc}


\subsection{Port Objects \label{al-port-objects}}

Port objects, as returned by \function{openport()}, have the following
methods:

\begin{methoddesc}[audio port]{closeport}{}
Close the port.
\end{methoddesc}

\begin{methoddesc}[audio port]{getfd}{}
Return the file descriptor as an int.
\end{methoddesc}

\begin{methoddesc}[audio port]{getfilled}{}
Return the number of filled samples.
\end{methoddesc}

\begin{methoddesc}[audio port]{getfillable}{}
Return the number of fillable samples.
\end{methoddesc}

\begin{methoddesc}[audio port]{readsamps}{nsamples}
Read a number of samples from the queue, blocking if necessary.
Return the data as a string containing the raw data, (e.g., 2 bytes per
sample in big-endian byte order (high byte, low byte) if you have set
the sample width to 2 bytes).
\end{methoddesc}

\begin{methoddesc}[audio port]{writesamps}{samples}
Write samples into the queue, blocking if necessary.  The samples are
encoded as described for the \method{readsamps()} return value.
\end{methoddesc}

\begin{methoddesc}[audio port]{getfillpoint}{}
Return the `fill point'.
\end{methoddesc}

\begin{methoddesc}[audio port]{setfillpoint}{fillpoint}
Set the `fill point'.
\end{methoddesc}

\begin{methoddesc}[audio port]{getconfig}{}
Return a configuration object containing the current configuration of
the port.
\end{methoddesc}

\begin{methoddesc}[audio port]{setconfig}{config}
Set the configuration from the argument, a configuration object.
\end{methoddesc}

\begin{methoddesc}[audio port]{getstatus}{list}
Get status information on last error.
\end{methoddesc}


\section{\module{AL} ---
         Constants used with the \module{al} module}

\declaremodule[al-constants]{standard}{AL}
  \platform{IRIX}
\modulesynopsis{Constants used with the \module{al} module.}


This module defines symbolic constants needed to use the built-in
module \refmodule{al} (see above); they are equivalent to those defined
in the C header file \code{<audio.h>} except that the name prefix
\samp{AL_} is omitted.  Read the module source for a complete list of
the defined names.  Suggested use:

\begin{verbatim}
import al
from AL import *
\end{verbatim}

%\section{Built-in Module \sectcode{audio}}
\bimodindex{audio}

\strong{Note:} This module is obsolete, since the hardware to which it
interfaces is obsolete.  For audio on the Indigo or 4D/35, see
built-in module \code{al} above.

This module provides rudimentary access to the audio I/O device
\file{/dev/audio} on the Silicon Graphics Personal IRIS 4D/25;
see {\it audio}(7). It supports the following operations:

\renewcommand{\indexsubitem}{(in module audio)}
\begin{funcdesc}{setoutgain}{n}
Sets the output gain.
\iftexi
\code{0 <= \var{n} < 256}.
\else
$0 \leq \var{n} < 256$.
%%JHXXX Sets the output gain (0-255).
\fi
\end{funcdesc}

\begin{funcdesc}{getoutgain}{}
Returns the output gain.
\end{funcdesc}

\begin{funcdesc}{setrate}{n}
Sets the sampling rate: \code{1} = 32K/sec, \code{2} = 16K/sec,
\code{3} = 8K/sec.
\end{funcdesc}

\begin{funcdesc}{setduration}{n}
Sets the `sound duration' in units of 1/100 seconds.
\end{funcdesc}

\begin{funcdesc}{read}{n}
Reads a chunk of
\var{n}
sampled bytes from the audio input (line in or microphone).
The chunk is returned as a string of length n.
Each byte encodes one sample as a signed 8-bit quantity using linear
encoding.
This string can be converted to numbers using \code{chr2num()} described
below.
\end{funcdesc}

\begin{funcdesc}{write}{buf}
Writes a chunk of samples to the audio output (speaker).
\end{funcdesc}

These operations support asynchronous audio I/O:

\renewcommand{\indexsubitem}{(in module audio)}
\begin{funcdesc}{start_recording}{n}
Starts a second thread (a process with shared memory) that begins reading
\var{n}
bytes from the audio device.
The main thread immediately continues.
\end{funcdesc}

\begin{funcdesc}{wait_recording}{}
Waits for the second thread to finish and returns the data read.
\end{funcdesc}

\begin{funcdesc}{stop_recording}{}
Makes the second thread stop reading as soon as possible.
Returns the data read so far.
\end{funcdesc}

\begin{funcdesc}{poll_recording}{}
Returns true if the second thread has finished reading (so
\code{wait_recording()} would return the data without delay).
\end{funcdesc}

\begin{funcdesc}{start_playing}{}
\funcline{wait_playing}{}
\funcline{stop_playing}{}
\funcline{poll_playing}{}
\begin{sloppypar}
Similar but for output.
\code{stop_playing()}
returns a lower bound for the number of bytes actually played (not very
accurate).
\end{sloppypar}
\end{funcdesc}

The following operations do not affect the audio device but are
implemented in C for efficiency:

\renewcommand{\indexsubitem}{(in module audio)}
\begin{funcdesc}{amplify}{buf\, f1\, f2}
Amplifies a chunk of samples by a variable factor changing from
\code{\var{f1}/256} to \code{\var{f2}/256.}
Negative factors are allowed.
Resulting values that are to large to fit in a byte are clipped.         
\end{funcdesc}

\begin{funcdesc}{reverse}{buf}
Returns a chunk of samples backwards.
\end{funcdesc}

\begin{funcdesc}{add}{buf1\, buf2}
Bytewise adds two chunks of samples.
Bytes that exceed the range are clipped.
If one buffer is shorter, it is assumed to be padded with zeros.
\end{funcdesc}

\begin{funcdesc}{chr2num}{buf}
Converts a string of sampled bytes as returned by \code{read()} into
a list containing the numeric values of the samples.
\end{funcdesc}

\begin{funcdesc}{num2chr}{list}
\begin{sloppypar}
Converts a list as returned by
\code{chr2num()}
back to a buffer acceptable by
\code{write()}.
\end{sloppypar}
\end{funcdesc}

\section{\module{cd} ---
         CD-ROM access on SGI systems}

\declaremodule{builtin}{cd}
  \platform{IRIX}
\modulesynopsis{Interface to the CD-ROM on Silicon Graphics systems.}


This module provides an interface to the Silicon Graphics CD library.
It is available only on Silicon Graphics systems.

The way the library works is as follows.  A program opens the CD-ROM
device with \function{open()} and creates a parser to parse the data
from the CD with \function{createparser()}.  The object returned by
\function{open()} can be used to read data from the CD, but also to get
status information for the CD-ROM device, and to get information about
the CD, such as the table of contents.  Data from the CD is passed to
the parser, which parses the frames, and calls any callback
functions that have previously been added.

An audio CD is divided into \dfn{tracks} or \dfn{programs} (the terms
are used interchangeably).  Tracks can be subdivided into
\dfn{indices}.  An audio CD contains a \dfn{table of contents} which
gives the starts of the tracks on the CD.  Index 0 is usually the
pause before the start of a track.  The start of the track as given by
the table of contents is normally the start of index 1.

Positions on a CD can be represented in two ways.  Either a frame
number or a tuple of three values, minutes, seconds and frames.  Most
functions use the latter representation.  Positions can be both
relative to the beginning of the CD, and to the beginning of the
track.

Module \module{cd} defines the following functions and constants:


\begin{funcdesc}{createparser}{}
Create and return an opaque parser object.  The methods of the parser
object are described below.
\end{funcdesc}

\begin{funcdesc}{msftoframe}{minutes, seconds, frames}
Converts a \code{(\var{minutes}, \var{seconds}, \var{frames})} triple
representing time in absolute time code into the corresponding CD
frame number.
\end{funcdesc}

\begin{funcdesc}{open}{\optional{device\optional{, mode}}}
Open the CD-ROM device.  The return value is an opaque player object;
methods of the player object are described below.  The device is the
name of the SCSI device file, e.g. \code{'/dev/scsi/sc0d4l0'}, or
\code{None}.  If omitted or \code{None}, the hardware inventory is
consulted to locate a CD-ROM drive.  The \var{mode}, if not omited,
should be the string \code{'r'}.
\end{funcdesc}

The module defines the following variables:

\begin{excdesc}{error}
Exception raised on various errors.
\end{excdesc}

\begin{datadesc}{DATASIZE}
The size of one frame's worth of audio data.  This is the size of the
audio data as passed to the callback of type \code{audio}.
\end{datadesc}

\begin{datadesc}{BLOCKSIZE}
The size of one uninterpreted frame of audio data.
\end{datadesc}

The following variables are states as returned by
\function{getstatus()}:

\begin{datadesc}{READY}
The drive is ready for operation loaded with an audio CD.
\end{datadesc}

\begin{datadesc}{NODISC}
The drive does not have a CD loaded.
\end{datadesc}

\begin{datadesc}{CDROM}
The drive is loaded with a CD-ROM.  Subsequent play or read operations
will return I/O errors.
\end{datadesc}

\begin{datadesc}{ERROR}
An error occurred while trying to read the disc or its table of
contents.
\end{datadesc}

\begin{datadesc}{PLAYING}
The drive is in CD player mode playing an audio CD through its audio
jacks.
\end{datadesc}

\begin{datadesc}{PAUSED}
The drive is in CD layer mode with play paused.
\end{datadesc}

\begin{datadesc}{STILL}
The equivalent of \constant{PAUSED} on older (non 3301) model Toshiba
CD-ROM drives.  Such drives have never been shipped by SGI.
\end{datadesc}

\begin{datadesc}{audio}
\dataline{pnum}
\dataline{index}
\dataline{ptime}
\dataline{atime}
\dataline{catalog}
\dataline{ident}
\dataline{control}
Integer constants describing the various types of parser callbacks
that can be set by the \method{addcallback()} method of CD parser
objects (see below).
\end{datadesc}


\subsection{Player Objects}
\label{player-objects}

Player objects (returned by \function{open()}) have the following
methods:

\begin{methoddesc}[CD player]{allowremoval}{}
Unlocks the eject button on the CD-ROM drive permitting the user to
eject the caddy if desired.
\end{methoddesc}

\begin{methoddesc}[CD player]{bestreadsize}{}
Returns the best value to use for the \var{num_frames} parameter of
the \method{readda()} method.  Best is defined as the value that
permits a continuous flow of data from the CD-ROM drive.
\end{methoddesc}

\begin{methoddesc}[CD player]{close}{}
Frees the resources associated with the player object.  After calling
\method{close()}, the methods of the object should no longer be used.
\end{methoddesc}

\begin{methoddesc}[CD player]{eject}{}
Ejects the caddy from the CD-ROM drive.
\end{methoddesc}

\begin{methoddesc}[CD player]{getstatus}{}
Returns information pertaining to the current state of the CD-ROM
drive.  The returned information is a tuple with the following values:
\var{state}, \var{track}, \var{rtime}, \var{atime}, \var{ttime},
\var{first}, \var{last}, \var{scsi_audio}, \var{cur_block}.
\var{rtime} is the time relative to the start of the current track;
\var{atime} is the time relative to the beginning of the disc;
\var{ttime} is the total time on the disc.  For more information on
the meaning of the values, see the man page \manpage{CDgetstatus}{3dm}.
The value of \var{state} is one of the following: \constant{ERROR},
\constant{NODISC}, \constant{READY}, \constant{PLAYING},
\constant{PAUSED}, \constant{STILL}, or \constant{CDROM}.
\end{methoddesc}

\begin{methoddesc}[CD player]{gettrackinfo}{track}
Returns information about the specified track.  The returned
information is a tuple consisting of two elements, the start time of
the track and the duration of the track.
\end{methoddesc}

\begin{methoddesc}[CD player]{msftoblock}{min, sec, frame}
Converts a minutes, seconds, frames triple representing a time in
absolute time code into the corresponding logical block number for the
given CD-ROM drive.  You should use \function{msftoframe()} rather than
\method{msftoblock()} for comparing times.  The logical block number
differs from the frame number by an offset required by certain CD-ROM
drives.
\end{methoddesc}

\begin{methoddesc}[CD player]{play}{start, play}
Starts playback of an audio CD in the CD-ROM drive at the specified
track.  The audio output appears on the CD-ROM drive's headphone and
audio jacks (if fitted).  Play stops at the end of the disc.
\var{start} is the number of the track at which to start playing the
CD; if \var{play} is 0, the CD will be set to an initial paused
state.  The method \method{togglepause()} can then be used to commence
play.
\end{methoddesc}

\begin{methoddesc}[CD player]{playabs}{minutes, seconds, frames, play}
Like \method{play()}, except that the start is given in minutes,
seconds, and frames instead of a track number.
\end{methoddesc}

\begin{methoddesc}[CD player]{playtrack}{start, play}
Like \method{play()}, except that playing stops at the end of the
track.
\end{methoddesc}

\begin{methoddesc}[CD player]{playtrackabs}{track, minutes, seconds, frames, play}
Like \method{play()}, except that playing begins at the specified
absolute time and ends at the end of the specified track.
\end{methoddesc}

\begin{methoddesc}[CD player]{preventremoval}{}
Locks the eject button on the CD-ROM drive thus preventing the user
from arbitrarily ejecting the caddy.
\end{methoddesc}

\begin{methoddesc}[CD player]{readda}{num_frames}
Reads the specified number of frames from an audio CD mounted in the
CD-ROM drive.  The return value is a string representing the audio
frames.  This string can be passed unaltered to the
\method{parseframe()} method of the parser object.
\end{methoddesc}

\begin{methoddesc}[CD player]{seek}{minutes, seconds, frames}
Sets the pointer that indicates the starting point of the next read of
digital audio data from a CD-ROM.  The pointer is set to an absolute
time code location specified in \var{minutes}, \var{seconds}, and
\var{frames}.  The return value is the logical block number to which
the pointer has been set.
\end{methoddesc}

\begin{methoddesc}[CD player]{seekblock}{block}
Sets the pointer that indicates the starting point of the next read of
digital audio data from a CD-ROM.  The pointer is set to the specified
logical block number.  The return value is the logical block number to
which the pointer has been set.
\end{methoddesc}

\begin{methoddesc}[CD player]{seektrack}{track}
Sets the pointer that indicates the starting point of the next read of
digital audio data from a CD-ROM.  The pointer is set to the specified
track.  The return value is the logical block number to which the
pointer has been set.
\end{methoddesc}

\begin{methoddesc}[CD player]{stop}{}
Stops the current playing operation.
\end{methoddesc}

\begin{methoddesc}[CD player]{togglepause}{}
Pauses the CD if it is playing, and makes it play if it is paused.
\end{methoddesc}


\subsection{Parser Objects}
\label{cd-parser-objects}

Parser objects (returned by \function{createparser()}) have the
following methods:

\begin{methoddesc}[CD parser]{addcallback}{type, func, arg}
Adds a callback for the parser.  The parser has callbacks for eight
different types of data in the digital audio data stream.  Constants
for these types are defined at the \module{cd} module level (see above).
The callback is called as follows: \code{\var{func}(\var{arg}, type,
data)}, where \var{arg} is the user supplied argument, \var{type} is
the particular type of callback, and \var{data} is the data returned
for this \var{type} of callback.  The type of the data depends on the
\var{type} of callback as follows:

\begin{tableii}{l|p{4in}}{code}{Type}{Value}
  \lineii{audio}{String which can be passed unmodified to
\function{al.writesamps()}.}
  \lineii{pnum}{Integer giving the program (track) number.}
  \lineii{index}{Integer giving the index number.}
  \lineii{ptime}{Tuple consisting of the program time in minutes,
seconds, and frames.}
  \lineii{atime}{Tuple consisting of the absolute time in minutes,
seconds, and frames.}
  \lineii{catalog}{String of 13 characters, giving the catalog number
of the CD.}
  \lineii{ident}{String of 12 characters, giving the ISRC
identification number of the recording.  The string consists of two
characters country code, three characters owner code, two characters
giving the year, and five characters giving a serial number.}
  \lineii{control}{Integer giving the control bits from the CD
subcode data}
\end{tableii}
\end{methoddesc}

\begin{methoddesc}[CD parser]{deleteparser}{}
Deletes the parser and frees the memory it was using.  The object
should not be used after this call.  This call is done automatically
when the last reference to the object is removed.
\end{methoddesc}

\begin{methoddesc}[CD parser]{parseframe}{frame}
Parses one or more frames of digital audio data from a CD such as
returned by \method{readda()}.  It determines which subcodes are
present in the data.  If these subcodes have changed since the last
frame, then \method{parseframe()} executes a callback of the
appropriate type passing to it the subcode data found in the frame.
Unlike the \C{} function, more than one frame of digital audio data
can be passed to this method.
\end{methoddesc}

\begin{methoddesc}[CD parser]{removecallback}{type}
Removes the callback for the given \var{type}.
\end{methoddesc}

\begin{methoddesc}[CD parser]{resetparser}{}
Resets the fields of the parser used for tracking subcodes to an
initial state.  \method{resetparser()} should be called after the disc
has been changed.
\end{methoddesc}

\section{\module{fl} ---
         FORMS library interface for GUI applications}

\declaremodule{builtin}{fl}
  \platform{IRIX}
\modulesynopsis{FORMS library interface for GUI applications.}


This module provides an interface to the FORMS Library\index{FORMS
Library} by Mark Overmars\index{Overmars, Mark}.  The source for the
library can be retrieved by anonymous ftp from host
\samp{ftp.cs.ruu.nl}, directory \file{SGI/FORMS}.  It was last tested
with version 2.0b.

Most functions are literal translations of their C equivalents,
dropping the initial \samp{fl_} from their name.  Constants used by
the library are defined in module \refmodule[fl-constants]{FL}
described below.

The creation of objects is a little different in Python than in C:
instead of the `current form' maintained by the library to which new
FORMS objects are added, all functions that add a FORMS object to a
form are methods of the Python object representing the form.
Consequently, there are no Python equivalents for the C functions
\cfunction{fl_addto_form()} and \cfunction{fl_end_form()}, and the
equivalent of \cfunction{fl_bgn_form()} is called
\function{fl.make_form()}.

Watch out for the somewhat confusing terminology: FORMS uses the word
\dfn{object} for the buttons, sliders etc. that you can place in a form.
In Python, `object' means any value.  The Python interface to FORMS
introduces two new Python object types: form objects (representing an
entire form) and FORMS objects (representing one button, slider etc.).
Hopefully this isn't too confusing.

There are no `free objects' in the Python interface to FORMS, nor is
there an easy way to add object classes written in Python.  The FORMS
interface to GL event handling is available, though, so you can mix
FORMS with pure GL windows.

\strong{Please note:} importing \module{fl} implies a call to the GL
function \cfunction{foreground()} and to the FORMS routine
\cfunction{fl_init()}.

\subsection{Functions Defined in Module \module{fl}}
\nodename{FL Functions}

Module \module{fl} defines the following functions.  For more
information about what they do, see the description of the equivalent
C function in the FORMS documentation:

\begin{funcdesc}{make_form}{type, width, height}
Create a form with given type, width and height.  This returns a
\dfn{form} object, whose methods are described below.
\end{funcdesc}

\begin{funcdesc}{do_forms}{}
The standard FORMS main loop.  Returns a Python object representing
the FORMS object needing interaction, or the special value
\constant{FL.EVENT}.
\end{funcdesc}

\begin{funcdesc}{check_forms}{}
Check for FORMS events.  Returns what \function{do_forms()} above
returns, or \code{None} if there is no event that immediately needs
interaction.
\end{funcdesc}

\begin{funcdesc}{set_event_call_back}{function}
Set the event callback function.
\end{funcdesc}

\begin{funcdesc}{set_graphics_mode}{rgbmode, doublebuffering}
Set the graphics modes.
\end{funcdesc}

\begin{funcdesc}{get_rgbmode}{}
Return the current rgb mode.  This is the value of the C global
variable \cdata{fl_rgbmode}.
\end{funcdesc}

\begin{funcdesc}{show_message}{str1, str2, str3}
Show a dialog box with a three-line message and an OK button.
\end{funcdesc}

\begin{funcdesc}{show_question}{str1, str2, str3}
Show a dialog box with a three-line message and YES and NO buttons.
It returns \code{1} if the user pressed YES, \code{0} if NO.
\end{funcdesc}

\begin{funcdesc}{show_choice}{str1, str2, str3, but1\optional{,
                              but2\optional{, but3}}}
Show a dialog box with a three-line message and up to three buttons.
It returns the number of the button clicked by the user
(\code{1}, \code{2} or \code{3}).
\end{funcdesc}

\begin{funcdesc}{show_input}{prompt, default}
Show a dialog box with a one-line prompt message and text field in
which the user can enter a string.  The second argument is the default
input string.  It returns the string value as edited by the user.
\end{funcdesc}

\begin{funcdesc}{show_file_selector}{message, directory, pattern, default}
Show a dialog box in which the user can select a file.  It returns
the absolute filename selected by the user, or \code{None} if the user
presses Cancel.
\end{funcdesc}

\begin{funcdesc}{get_directory}{}
\funcline{get_pattern}{}
\funcline{get_filename}{}
These functions return the directory, pattern and filename (the tail
part only) selected by the user in the last
\function{show_file_selector()} call.
\end{funcdesc}

\begin{funcdesc}{qdevice}{dev}
\funcline{unqdevice}{dev}
\funcline{isqueued}{dev}
\funcline{qtest}{}
\funcline{qread}{}
%\funcline{blkqread}{?}
\funcline{qreset}{}
\funcline{qenter}{dev, val}
\funcline{get_mouse}{}
\funcline{tie}{button, valuator1, valuator2}
These functions are the FORMS interfaces to the corresponding GL
functions.  Use these if you want to handle some GL events yourself
when using \function{fl.do_events()}.  When a GL event is detected that
FORMS cannot handle, \function{fl.do_forms()} returns the special value
\constant{FL.EVENT} and you should call \function{fl.qread()} to read
the event from the queue.  Don't use the equivalent GL functions!
\end{funcdesc}

\begin{funcdesc}{color}{}
\funcline{mapcolor}{}
\funcline{getmcolor}{}
See the description in the FORMS documentation of
\cfunction{fl_color()}, \cfunction{fl_mapcolor()} and
\cfunction{fl_getmcolor()}.
\end{funcdesc}

\subsection{Form Objects}
\label{form-objects}

Form objects (returned by \function{make_form()} above) have the
following methods.  Each method corresponds to a C function whose
name is prefixed with \samp{fl_}; and whose first argument is a form
pointer; please refer to the official FORMS documentation for
descriptions.

All the \method{add_*()} methods return a Python object representing
the FORMS object.  Methods of FORMS objects are described below.  Most
kinds of FORMS object also have some methods specific to that kind;
these methods are listed here.

\begin{flushleft}

\begin{methoddesc}[form]{show_form}{placement, bordertype, name}
  Show the form.
\end{methoddesc}

\begin{methoddesc}[form]{hide_form}{}
  Hide the form.
\end{methoddesc}

\begin{methoddesc}[form]{redraw_form}{}
  Redraw the form.
\end{methoddesc}

\begin{methoddesc}[form]{set_form_position}{x, y}
Set the form's position.
\end{methoddesc}

\begin{methoddesc}[form]{freeze_form}{}
Freeze the form.
\end{methoddesc}

\begin{methoddesc}[form]{unfreeze_form}{}
  Unfreeze the form.
\end{methoddesc}

\begin{methoddesc}[form]{activate_form}{}
  Activate the form.
\end{methoddesc}

\begin{methoddesc}[form]{deactivate_form}{}
  Deactivate the form.
\end{methoddesc}

\begin{methoddesc}[form]{bgn_group}{}
  Begin a new group of objects; return a group object.
\end{methoddesc}

\begin{methoddesc}[form]{end_group}{}
  End the current group of objects.
\end{methoddesc}

\begin{methoddesc}[form]{find_first}{}
  Find the first object in the form.
\end{methoddesc}

\begin{methoddesc}[form]{find_last}{}
  Find the last object in the form.
\end{methoddesc}

%---

\begin{methoddesc}[form]{add_box}{type, x, y, w, h, name}
Add a box object to the form.
No extra methods.
\end{methoddesc}

\begin{methoddesc}[form]{add_text}{type, x, y, w, h, name}
Add a text object to the form.
No extra methods.
\end{methoddesc}

%\begin{methoddesc}[form]{add_bitmap}{type, x, y, w, h, name}
%Add a bitmap object to the form.
%\end{methoddesc}

\begin{methoddesc}[form]{add_clock}{type, x, y, w, h, name}
Add a clock object to the form. \\
Method:
\method{get_clock()}.
\end{methoddesc}

%---

\begin{methoddesc}[form]{add_button}{type, x, y, w, h,  name}
Add a button object to the form. \\
Methods:
\method{get_button()},
\method{set_button()}.
\end{methoddesc}

\begin{methoddesc}[form]{add_lightbutton}{type, x, y, w, h, name}
Add a lightbutton object to the form. \\
Methods:
\method{get_button()},
\method{set_button()}.
\end{methoddesc}

\begin{methoddesc}[form]{add_roundbutton}{type, x, y, w, h, name}
Add a roundbutton object to the form. \\
Methods:
\method{get_button()},
\method{set_button()}.
\end{methoddesc}

%---

\begin{methoddesc}[form]{add_slider}{type, x, y, w, h, name}
Add a slider object to the form. \\
Methods:
\method{set_slider_value()},
\method{get_slider_value()},
\method{set_slider_bounds()},
\method{get_slider_bounds()},
\method{set_slider_return()},
\method{set_slider_size()},
\method{set_slider_precision()},
\method{set_slider_step()}.
\end{methoddesc}

\begin{methoddesc}[form]{add_valslider}{type, x, y, w, h, name}
Add a valslider object to the form. \\
Methods:
\method{set_slider_value()},
\method{get_slider_value()},
\method{set_slider_bounds()},
\method{get_slider_bounds()},
\method{set_slider_return()},
\method{set_slider_size()},
\method{set_slider_precision()},
\method{set_slider_step()}.
\end{methoddesc}

\begin{methoddesc}[form]{add_dial}{type, x, y, w, h, name}
Add a dial object to the form. \\
Methods:
\method{set_dial_value()},
\method{get_dial_value()},
\method{set_dial_bounds()},
\method{get_dial_bounds()}.
\end{methoddesc}

\begin{methoddesc}[form]{add_positioner}{type, x, y, w, h, name}
Add a positioner object to the form. \\
Methods:
\method{set_positioner_xvalue()},
\method{set_positioner_yvalue()},
\method{set_positioner_xbounds()},
\method{set_positioner_ybounds()},
\method{get_positioner_xvalue()},
\method{get_positioner_yvalue()},
\method{get_positioner_xbounds()},
\method{get_positioner_ybounds()}.
\end{methoddesc}

\begin{methoddesc}[form]{add_counter}{type, x, y, w, h, name}
Add a counter object to the form. \\
Methods:
\method{set_counter_value()},
\method{get_counter_value()},
\method{set_counter_bounds()},
\method{set_counter_step()},
\method{set_counter_precision()},
\method{set_counter_return()}.
\end{methoddesc}

%---

\begin{methoddesc}[form]{add_input}{type, x, y, w, h, name}
Add a input object to the form. \\
Methods:
\method{set_input()},
\method{get_input()},
\method{set_input_color()},
\method{set_input_return()}.
\end{methoddesc}

%---

\begin{methoddesc}[form]{add_menu}{type, x, y, w, h, name}
Add a menu object to the form. \\
Methods:
\method{set_menu()},
\method{get_menu()},
\method{addto_menu()}.
\end{methoddesc}

\begin{methoddesc}[form]{add_choice}{type, x, y, w, h, name}
Add a choice object to the form. \\
Methods:
\method{set_choice()},
\method{get_choice()},
\method{clear_choice()},
\method{addto_choice()},
\method{replace_choice()},
\method{delete_choice()},
\method{get_choice_text()},
\method{set_choice_fontsize()},
\method{set_choice_fontstyle()}.
\end{methoddesc}

\begin{methoddesc}[form]{add_browser}{type, x, y, w, h, name}
Add a browser object to the form. \\
Methods:
\method{set_browser_topline()},
\method{clear_browser()},
\method{add_browser_line()},
\method{addto_browser()},
\method{insert_browser_line()},
\method{delete_browser_line()},
\method{replace_browser_line()},
\method{get_browser_line()},
\method{load_browser()},
\method{get_browser_maxline()},
\method{select_browser_line()},
\method{deselect_browser_line()},
\method{deselect_browser()},
\method{isselected_browser_line()},
\method{get_browser()},
\method{set_browser_fontsize()},
\method{set_browser_fontstyle()},
\method{set_browser_specialkey()}.
\end{methoddesc}

%---

\begin{methoddesc}[form]{add_timer}{type, x, y, w, h, name}
Add a timer object to the form. \\
Methods:
\method{set_timer()},
\method{get_timer()}.
\end{methoddesc}
\end{flushleft}

Form objects have the following data attributes; see the FORMS
documentation:

\begin{tableiii}{l|l|l}{member}{Name}{C Type}{Meaning}
  \lineiii{window}{int (read-only)}{GL window id}
  \lineiii{w}{float}{form width}
  \lineiii{h}{float}{form height}
  \lineiii{x}{float}{form x origin}
  \lineiii{y}{float}{form y origin}
  \lineiii{deactivated}{int}{nonzero if form is deactivated}
  \lineiii{visible}{int}{nonzero if form is visible}
  \lineiii{frozen}{int}{nonzero if form is frozen}
  \lineiii{doublebuf}{int}{nonzero if double buffering on}
\end{tableiii}

\subsection{FORMS Objects}
\label{forms-objects}

Besides methods specific to particular kinds of FORMS objects, all
FORMS objects also have the following methods:

\begin{methoddesc}[FORMS object]{set_call_back}{function, argument}
Set the object's callback function and argument.  When the object
needs interaction, the callback function will be called with two
arguments: the object, and the callback argument.  (FORMS objects
without a callback function are returned by \function{fl.do_forms()}
or \function{fl.check_forms()} when they need interaction.)  Call this
method without arguments to remove the callback function.
\end{methoddesc}

\begin{methoddesc}[FORMS object]{delete_object}{}
  Delete the object.
\end{methoddesc}

\begin{methoddesc}[FORMS object]{show_object}{}
  Show the object.
\end{methoddesc}

\begin{methoddesc}[FORMS object]{hide_object}{}
  Hide the object.
\end{methoddesc}

\begin{methoddesc}[FORMS object]{redraw_object}{}
  Redraw the object.
\end{methoddesc}

\begin{methoddesc}[FORMS object]{freeze_object}{}
  Freeze the object.
\end{methoddesc}

\begin{methoddesc}[FORMS object]{unfreeze_object}{}
  Unfreeze the object.
\end{methoddesc}

%\begin{methoddesc}[FORMS object]{handle_object}{} XXX
%\end{methoddesc}

%\begin{methoddesc}[FORMS object]{handle_object_direct}{} XXX
%\end{methoddesc}

FORMS objects have these data attributes; see the FORMS documentation:

\begin{tableiii}{l|l|l}{member}{Name}{C Type}{Meaning}
  \lineiii{objclass}{int (read-only)}{object class}
  \lineiii{type}{int (read-only)}{object type}
  \lineiii{boxtype}{int}{box type}
  \lineiii{x}{float}{x origin}
  \lineiii{y}{float}{y origin}
  \lineiii{w}{float}{width}
  \lineiii{h}{float}{height}
  \lineiii{col1}{int}{primary color}
  \lineiii{col2}{int}{secondary color}
  \lineiii{align}{int}{alignment}
  \lineiii{lcol}{int}{label color}
  \lineiii{lsize}{float}{label font size}
  \lineiii{label}{string}{label string}
  \lineiii{lstyle}{int}{label style}
  \lineiii{pushed}{int (read-only)}{(see FORMS docs)}
  \lineiii{focus}{int (read-only)}{(see FORMS docs)}
  \lineiii{belowmouse}{int (read-only)}{(see FORMS docs)}
  \lineiii{frozen}{int (read-only)}{(see FORMS docs)}
  \lineiii{active}{int (read-only)}{(see FORMS docs)}
  \lineiii{input}{int (read-only)}{(see FORMS docs)}
  \lineiii{visible}{int (read-only)}{(see FORMS docs)}
  \lineiii{radio}{int (read-only)}{(see FORMS docs)}
  \lineiii{automatic}{int (read-only)}{(see FORMS docs)}
\end{tableiii}


\section{\module{FL} ---
         Constants used with the \module{fl} module}

\declaremodule[fl-constants]{standard}{FL}
  \platform{IRIX}
\modulesynopsis{Constants used with the \module{fl} module.}


This module defines symbolic constants needed to use the built-in
module \refmodule{fl} (see above); they are equivalent to those defined in
the C header file \code{<forms.h>} except that the name prefix
\samp{FL_} is omitted.  Read the module source for a complete list of
the defined names.  Suggested use:

\begin{verbatim}
import fl
from FL import *
\end{verbatim}


\section{\module{flp} ---
         Functions for loading stored FORMS designs}

\declaremodule{standard}{flp}
  \platform{IRIX}
\modulesynopsis{Functions for loading stored FORMS designs.}


This module defines functions that can read form definitions created
by the `form designer' (\program{fdesign}) program that comes with the
FORMS library (see module \refmodule{fl} above).

For now, see the file \file{flp.doc} in the Python library source
directory for a description.

XXX A complete description should be inserted here!

\section{\module{fm} ---
         \emph{Font Manager} interface}

\declaremodule{builtin}{fm}
  \platform{IRIX}
\modulesynopsis{\emph{Font Manager} interface for SGI workstations.}


This module provides access to the IRIS \emph{Font Manager} library.
\index{Font Manager, IRIS}
\index{IRIS Font Manager}
It is available only on Silicon Graphics machines.
See also: \emph{4Sight User's Guide}, section 1, chapter 5: ``Using
the IRIS Font Manager.''

This is not yet a full interface to the IRIS Font Manager.
Among the unsupported features are: matrix operations; cache
operations; character operations (use string operations instead); some
details of font info; individual glyph metrics; and printer matching.

It supports the following operations:

\begin{funcdesc}{init}{}
Initialization function.
Calls \cfunction{fminit()}.
It is normally not necessary to call this function, since it is called
automatically the first time the \module{fm} module is imported.
\end{funcdesc}

\begin{funcdesc}{findfont}{fontname}
Return a font handle object.
Calls \code{fmfindfont(\var{fontname})}.
\end{funcdesc}

\begin{funcdesc}{enumerate}{}
Returns a list of available font names.
This is an interface to \cfunction{fmenumerate()}.
\end{funcdesc}

\begin{funcdesc}{prstr}{string}
Render a string using the current font (see the \function{setfont()} font
handle method below).
Calls \code{fmprstr(\var{string})}.
\end{funcdesc}

\begin{funcdesc}{setpath}{string}
Sets the font search path.
Calls \code{fmsetpath(\var{string})}.
(XXX Does not work!?!)
\end{funcdesc}

\begin{funcdesc}{fontpath}{}
Returns the current font search path.
\end{funcdesc}

Font handle objects support the following operations:

\setindexsubitem{(font handle method)}
\begin{funcdesc}{scalefont}{factor}
Returns a handle for a scaled version of this font.
Calls \code{fmscalefont(\var{fh}, \var{factor})}.
\end{funcdesc}

\begin{funcdesc}{setfont}{}
Makes this font the current font.
Note: the effect is undone silently when the font handle object is
deleted.
Calls \code{fmsetfont(\var{fh})}.
\end{funcdesc}

\begin{funcdesc}{getfontname}{}
Returns this font's name.
Calls \code{fmgetfontname(\var{fh})}.
\end{funcdesc}

\begin{funcdesc}{getcomment}{}
Returns the comment string associated with this font.
Raises an exception if there is none.
Calls \code{fmgetcomment(\var{fh})}.
\end{funcdesc}

\begin{funcdesc}{getfontinfo}{}
Returns a tuple giving some pertinent data about this font.
This is an interface to \code{fmgetfontinfo()}.
The returned tuple contains the following numbers:
\code{(}\var{printermatched}, \var{fixed_width}, \var{xorig},
\var{yorig}, \var{xsize}, \var{ysize}, \var{height},
\var{nglyphs}\code{)}.
\end{funcdesc}

\begin{funcdesc}{getstrwidth}{string}
Returns the width, in pixels, of \var{string} when drawn in this font.
Calls \code{fmgetstrwidth(\var{fh}, \var{string})}.
\end{funcdesc}

\section{\module{gl} ---
         \emph{Graphics Library} interface}

\declaremodule{builtin}{gl}
  \platform{IRIX}
\modulesynopsis{Functions from the Silicon Graphics \emph{Graphics Library}.}


This module provides access to the Silicon Graphics
\emph{Graphics Library}.
It is available only on Silicon Graphics machines.

\warning{Some illegal calls to the GL library cause the Python
interpreter to dump core.
In particular, the use of most GL calls is unsafe before the first
window is opened.}

The module is too large to document here in its entirety, but the
following should help you to get started.
The parameter conventions for the C functions are translated to Python as
follows:

\begin{itemize}
\item
All (short, long, unsigned) int values are represented by Python
integers.
\item
All float and double values are represented by Python floating point
numbers.
In most cases, Python integers are also allowed.
\item
All arrays are represented by one-dimensional Python lists.
In most cases, tuples are also allowed.
\item
\begin{sloppypar}
All string and character arguments are represented by Python strings,
for instance,
\code{winopen('Hi There!')}
and
\code{rotate(900, 'z')}.
\end{sloppypar}
\item
All (short, long, unsigned) integer arguments or return values that are
only used to specify the length of an array argument are omitted.
For example, the C call

\begin{verbatim}
lmdef(deftype, index, np, props)
\end{verbatim}

is translated to Python as

\begin{verbatim}
lmdef(deftype, index, props)
\end{verbatim}

\item
Output arguments are omitted from the argument list; they are
transmitted as function return values instead.
If more than one value must be returned, the return value is a tuple.
If the C function has both a regular return value (that is not omitted
because of the previous rule) and an output argument, the return value
comes first in the tuple.
Examples: the C call

\begin{verbatim}
getmcolor(i, &red, &green, &blue)
\end{verbatim}

is translated to Python as

\begin{verbatim}
red, green, blue = getmcolor(i)
\end{verbatim}

\end{itemize}

The following functions are non-standard or have special argument
conventions:

\begin{funcdesc}{varray}{argument}
%JHXXX the argument-argument added
Equivalent to but faster than a number of
\code{v3d()}
calls.
The \var{argument} is a list (or tuple) of points.
Each point must be a tuple of coordinates
\code{(\var{x}, \var{y}, \var{z})} or \code{(\var{x}, \var{y})}.
The points may be 2- or 3-dimensional but must all have the
same dimension.
Float and int values may be mixed however.
The points are always converted to 3D double precision points
by assuming \code{\var{z} = 0.0} if necessary (as indicated in the man page),
and for each point
\code{v3d()}
is called.
\end{funcdesc}

\begin{funcdesc}{nvarray}{}
Equivalent to but faster than a number of
\code{n3f}
and
\code{v3f}
calls.
The argument is an array (list or tuple) of pairs of normals and points.
Each pair is a tuple of a point and a normal for that point.
Each point or normal must be a tuple of coordinates
\code{(\var{x}, \var{y}, \var{z})}.
Three coordinates must be given.
Float and int values may be mixed.
For each pair,
\code{n3f()}
is called for the normal, and then
\code{v3f()}
is called for the point.
\end{funcdesc}

\begin{funcdesc}{vnarray}{}
Similar to 
\code{nvarray()}
but the pairs have the point first and the normal second.
\end{funcdesc}

\begin{funcdesc}{nurbssurface}{s_k, t_k, ctl, s_ord, t_ord, type}
% XXX s_k[], t_k[], ctl[][]
Defines a nurbs surface.
The dimensions of
\code{\var{ctl}[][]}
are computed as follows:
\code{[len(\var{s_k}) - \var{s_ord}]},
\code{[len(\var{t_k}) - \var{t_ord}]}.
\end{funcdesc}

\begin{funcdesc}{nurbscurve}{knots, ctlpoints, order, type}
Defines a nurbs curve.
The length of ctlpoints is
\code{len(\var{knots}) - \var{order}}.
\end{funcdesc}

\begin{funcdesc}{pwlcurve}{points, type}
Defines a piecewise-linear curve.
\var{points}
is a list of points.
\var{type}
must be
\code{N_ST}.
\end{funcdesc}

\begin{funcdesc}{pick}{n}
\funcline{select}{n}
The only argument to these functions specifies the desired size of the
pick or select buffer.
\end{funcdesc}

\begin{funcdesc}{endpick}{}
\funcline{endselect}{}
These functions have no arguments.
They return a list of integers representing the used part of the
pick/select buffer.
No method is provided to detect buffer overrun.
\end{funcdesc}

Here is a tiny but complete example GL program in Python:

\begin{verbatim}
import gl, GL, time

def main():
    gl.foreground()
    gl.prefposition(500, 900, 500, 900)
    w = gl.winopen('CrissCross')
    gl.ortho2(0.0, 400.0, 0.0, 400.0)
    gl.color(GL.WHITE)
    gl.clear()
    gl.color(GL.RED)
    gl.bgnline()
    gl.v2f(0.0, 0.0)
    gl.v2f(400.0, 400.0)
    gl.endline()
    gl.bgnline()
    gl.v2f(400.0, 0.0)
    gl.v2f(0.0, 400.0)
    gl.endline()
    time.sleep(5)

main()
\end{verbatim}


\begin{seealso}
  \seetitle[http://pyopengl.sourceforge.net/]
           {PyOpenGL: The Python OpenGL Binding}
           {An interface to OpenGL\index{OpenGL} is also available;
            see information about the
            \strong{PyOpenGL}\index{PyOpenGL} project online at
            \url{http://pyopengl.sourceforge.net/}.  This may be a
            better option if support for SGI hardware from before
            about 1996 is not required.}
\end{seealso}


\section{\module{DEVICE} ---
         Constants used with the \module{gl} module}

\declaremodule{standard}{DEVICE}
  \platform{IRIX}
\modulesynopsis{Constants used with the \module{gl} module.}

This modules defines the constants used by the Silicon Graphics
\emph{Graphics Library} that C programmers find in the header file
\code{<gl/device.h>}.
Read the module source file for details.


\section{\module{GL} ---
         Constants used with the \module{gl} module}

\declaremodule[gl-constants]{standard}{GL}
  \platform{IRIX}
\modulesynopsis{Constants used with the \module{gl} module.}

This module contains constants used by the Silicon Graphics
\emph{Graphics Library} from the C header file \code{<gl/gl.h>}.
Read the module source file for details.

\section{\module{imgfile} ---
         Support for SGI imglib files}

\declaremodule{builtin}{imgfile}
  \platform{IRIX}
\modulesynopsis{Support for SGI imglib files.}


The \module{imgfile} module allows Python programs to access SGI imglib image
files (also known as \file{.rgb} files).  The module is far from
complete, but is provided anyway since the functionality that there is
is enough in some cases.  Currently, colormap files are not supported.

The module defines the following variables and functions:

\begin{excdesc}{error}
This exception is raised on all errors, such as unsupported file type, etc.
\end{excdesc}

\begin{funcdesc}{getsizes}{file}
This function returns a tuple \code{(\var{x}, \var{y}, \var{z})} where
\var{x} and \var{y} are the size of the image in pixels and
\var{z} is the number of
bytes per pixel. Only 3 byte RGB pixels and 1 byte greyscale pixels
are currently supported.
\end{funcdesc}

\begin{funcdesc}{read}{file}
This function reads and decodes the image on the specified file, and
returns it as a Python string. The string has either 1 byte greyscale
pixels or 4 byte RGBA pixels. The bottom left pixel is the first in
the string. This format is suitable to pass to \function{gl.lrectwrite()},
for instance.
\end{funcdesc}

\begin{funcdesc}{readscaled}{file, x, y, filter\optional{, blur}}
This function is identical to read but it returns an image that is
scaled to the given \var{x} and \var{y} sizes. If the \var{filter} and
\var{blur} parameters are omitted scaling is done by
simply dropping or duplicating pixels, so the result will be less than
perfect, especially for computer-generated images.

Alternatively, you can specify a filter to use to smoothen the image
after scaling. The filter forms supported are \code{'impulse'},
\code{'box'}, \code{'triangle'}, \code{'quadratic'} and
\code{'gaussian'}. If a filter is specified \var{blur} is an optional
parameter specifying the blurriness of the filter. It defaults to \code{1.0}.

\function{readscaled()} makes no attempt to keep the aspect ratio
correct, so that is the users' responsibility.
\end{funcdesc}

\begin{funcdesc}{ttob}{flag}
This function sets a global flag which defines whether the scan lines
of the image are read or written from bottom to top (flag is zero,
compatible with SGI GL) or from top to bottom(flag is one,
compatible with X).  The default is zero.
\end{funcdesc}

\begin{funcdesc}{write}{file, data, x, y, z}
This function writes the RGB or greyscale data in \var{data} to image
file \var{file}. \var{x} and \var{y} give the size of the image,
\var{z} is 1 for 1 byte greyscale images or 3 for RGB images (which are
stored as 4 byte values of which only the lower three bytes are used).
These are the formats returned by \function{gl.lrectread()}.
\end{funcdesc}

%\section{\module{panel} ---
         None}
\declaremodule{standard}{panel}

\modulesynopsis{None}


\strong{Please note:} The FORMS library, to which the
\code{fl}\refbimodindex{fl} module described above interfaces, is a
simpler and more accessible user interface library for use with GL
than the \code{panel} module (besides also being by a Dutch author).

This module should be used instead of the built-in module
\code{pnl}\refbimodindex{pnl}
to interface with the
\emph{Panel Library}.

The module is too large to document here in its entirety.
One interesting function:

\begin{funcdesc}{defpanellist}{filename}
Parses a panel description file containing S-expressions written by the
\emph{Panel Editor}
that accompanies the Panel Library and creates the described panels.
It returns a list of panel objects.
\end{funcdesc}

\strong{Warning:}
the Python interpreter will dump core if you don't create a GL window
before calling
\code{panel.mkpanel()}
or
\code{panel.defpanellist()}.

\section{\module{panelparser} ---
         None}
\declaremodule{standard}{panelparser}

\modulesynopsis{None}


This module defines a self-contained parser for S-expressions as output
by the Panel Editor (which is written in Scheme so it can't help writing
S-expressions).
The relevant function is
\code{panelparser.parse_file(\var{file})}
which has a file object (not a filename!) as argument and returns a list
of parsed S-expressions.
Each S-expression is converted into a Python list, with atoms converted
to Python strings and sub-expressions (recursively) to Python lists.
For more details, read the module file.
% XXXXJH should be funcdesc, I think

\section{\module{pnl} ---
         None}
\declaremodule{builtin}{pnl}

\modulesynopsis{None}


This module provides access to the
\emph{Panel Library}
built by NASA Ames\index{NASA} (to get it, send e-mail to
\code{panel-request@nas.nasa.gov}).
All access to it should be done through the standard module
\code{panel}\refstmodindex{panel},
which transparently exports most functions from
\code{pnl}
but redefines
\code{pnl.dopanel()}.

\strong{Warning:}
the Python interpreter will dump core if you don't create a GL window
before calling
\code{pnl.mkpanel()}.

The module is too large to document here in its entirety.


\chapter{SunOS Specific Services}
\label{sunos}

The modules described in this chapter provide interfaces to features
that are unique to SunOS 5 (also known as Solaris version 2).
			% SUNOS ONLY

\documentstyle[twoside,11pt,myformat]{report}

% NOTE: this file controls which chapters/sections of the library
% manual are actually printed.  It is easy to customize your manual
% by commenting out sections that you're not interested in.

\title{Python Library Reference}

\author{Guido van Rossum\\
	Fred L. Drake, Jr., editor}
\authoraddress{
	BeOpen PythonLabs\\
	E-mail: \email{python-docs@python.org}
}

\date{September 5, 2000}			% XXX update before release!
\release{2.0b1}


\makeindex			% tell \index to actually write the .idx file


\begin{document}

\pagenumbering{roman}

\maketitle

\begin{small}
Copyright \copyright{} 2001 Python Software Foundation.
All rights reserved.

Copyright \copyright{} 2000 BeOpen.com.
All rights reserved.

Copyright \copyright{} 1995-2000 Corporation for National Research Initiatives.
All rights reserved.

Copyright \copyright{} 1991-1995 Stichting Mathematisch Centrum.
All rights reserved.

%%begin{latexonly}
\vskip 4mm
%%end{latexonly}

\centerline{\strong{BEOPEN.COM TERMS AND CONDITIONS FOR PYTHON 2.0}}

\centerline{\strong{BEOPEN PYTHON OPEN SOURCE LICENSE AGREEMENT VERSION 1}}

\begin{enumerate}

\item
This LICENSE AGREEMENT is between BeOpen.com (``BeOpen''), having an
office at 160 Saratoga Avenue, Santa Clara, CA 95051, and the
Individual or Organization (``Licensee'') accessing and otherwise
using this software in source or binary form and its associated
documentation (``the Software'').

\item
Subject to the terms and conditions of this BeOpen Python License
Agreement, BeOpen hereby grants Licensee a non-exclusive,
royalty-free, world-wide license to reproduce, analyze, test, perform
and/or display publicly, prepare derivative works, distribute, and
otherwise use the Software alone or in any derivative version,
provided, however, that the BeOpen Python License is retained in the
Software, alone or in any derivative version prepared by Licensee.

\item
BeOpen is making the Software available to Licensee on an ``AS IS''
basis.  BEOPEN MAKES NO REPRESENTATIONS OR WARRANTIES, EXPRESS OR
IMPLIED.  BY WAY OF EXAMPLE, BUT NOT LIMITATION, BEOPEN MAKES NO AND
DISCLAIMS ANY REPRESENTATION OR WARRANTY OF MERCHANTABILITY OR FITNESS
FOR ANY PARTICULAR PURPOSE OR THAT THE USE OF THE SOFTWARE WILL NOT
INFRINGE ANY THIRD PARTY RIGHTS.

\item
BEOPEN SHALL NOT BE LIABLE TO LICENSEE OR ANY OTHER USERS OF THE
SOFTWARE FOR ANY INCIDENTAL, SPECIAL, OR CONSEQUENTIAL DAMAGES OR LOSS
AS A RESULT OF USING, MODIFYING OR DISTRIBUTING THE SOFTWARE, OR ANY
DERIVATIVE THEREOF, EVEN IF ADVISED OF THE POSSIBILITY THEREOF.

\item
This License Agreement will automatically terminate upon a material
breach of its terms and conditions.

\item
This License Agreement shall be governed by and interpreted in all
respects by the law of the State of California, excluding conflict of
law provisions.  Nothing in this License Agreement shall be deemed to
create any relationship of agency, partnership, or joint venture
between BeOpen and Licensee.  This License Agreement does not grant
permission to use BeOpen trademarks or trade names in a trademark
sense to endorse or promote products or services of Licensee, or any
third party.  As an exception, the ``BeOpen Python'' logos available
at http://www.pythonlabs.com/logos.html may be used according to the
permissions granted on that web page.

\item
By copying, installing or otherwise using the software, Licensee
agrees to be bound by the terms and conditions of this License
Agreement.
\end{enumerate}


\centerline{\strong{CNRI OPEN SOURCE GPL-COMPATIBLE LICENSE AGREEMENT}}

Python 1.6.1 is made available subject to the terms and conditions in
CNRI's License Agreement.  This Agreement together with Python 1.6.1 may
be located on the Internet using the following unique, persistent
identifier (known as a handle): 1895.22/1013.  This Agreement may also
be obtained from a proxy server on the Internet using the following
URL: \url{http://hdl.handle.net/1895.22/1013}.


\centerline{\strong{CWI PERMISSIONS STATEMENT AND DISCLAIMER}}

Copyright \copyright{} 1991 - 1995, Stichting Mathematisch Centrum
Amsterdam, The Netherlands.  All rights reserved.

Permission to use, copy, modify, and distribute this software and its
documentation for any purpose and without fee is hereby granted,
provided that the above copyright notice appear in all copies and that
both that copyright notice and this permission notice appear in
supporting documentation, and that the name of Stichting Mathematisch
Centrum or CWI not be used in advertising or publicity pertaining to
distribution of the software without specific, written prior
permission.

STICHTING MATHEMATISCH CENTRUM DISCLAIMS ALL WARRANTIES WITH REGARD TO
THIS SOFTWARE, INCLUDING ALL IMPLIED WARRANTIES OF MERCHANTABILITY AND
FITNESS, IN NO EVENT SHALL STICHTING MATHEMATISCH CENTRUM BE LIABLE
FOR ANY SPECIAL, INDIRECT OR CONSEQUENTIAL DAMAGES OR ANY DAMAGES
WHATSOEVER RESULTING FROM LOSS OF USE, DATA OR PROFITS, WHETHER IN AN
ACTION OF CONTRACT, NEGLIGENCE OR OTHER TORTIOUS ACTION, ARISING OUT
OF OR IN CONNECTION WITH THE USE OR PERFORMANCE OF THIS SOFTWARE.
\end{small}


\begin{abstract}

\noindent
This document describes the built-in and standard types, exceptions,
functions and modules that come with the Python system.  It assumes
basic knowledge about the Python language.  For an informal
introduction to the language, see the {\em Python Tutorial}.  The {\em
Python Reference Manual} gives a more formal definition of the
language.

\end{abstract}

\pagebreak

{
\parskip = 0mm
\tableofcontents
}

\pagebreak

\pagenumbering{arabic}

				% Chapter title:

\chapter{Introduction}

The Python library consists of three parts, with different levels of
integration with the interpreter.
Closest to the interpreter are built-in types, exceptions and functions.
Next are built-in modules, which are written in \C{} and linked statically
with the interpreter.
Finally there are standard modules that are implemented entirely in
Python, but are always available.
For efficiency, some standard modules may become built-in modules in
future versions of the interpreter.
\indexii{built-in}{types}
\indexii{built-in}{exceptions}
\indexii{built-in}{functions}
\indexii{built-in}{modules}
\indexii{standard}{modules}
\indexii{\C{}}{language}
		% Introduction

\chapter{Built-In Objects \label{builtin}}

Names for built-in exceptions and functions and a number of constants are
found in a separate 
symbol table.  This table is searched last when the interpreter looks
up the meaning of a name, so local and global
user-defined names can override built-in names.  Built-in types are
described together here for easy reference.\footnote{
	Most descriptions sorely lack explanations of the exceptions
	that may be raised --- this will be fixed in a future version of
	this manual.}
\indexii{built-in}{types}
\indexii{built-in}{exceptions}
\indexii{built-in}{functions}
\indexii{built-in}{constants}
\index{symbol table}

The tables in this chapter document the priorities of operators by
listing them in order of ascending priority (within a table) and
grouping operators that have the same priority in the same box.
Binary operators of the same priority group from left to right.
(Unary operators group from right to left, but there you have no real
choice.)  See chapter 5 of the \citetitle[../ref/ref.html]{Python
Reference Manual} for the complete picture on operator priorities.
			% Built-in Types, Exceptions and Functions
\section{\module{types} ---
         Names for all built-in types}

\declaremodule{standard}{types}
\modulesynopsis{Names for all built-in types.}


This module defines names for all object types that are used by the
standard Python interpreter, but not for the types defined by various
extension modules.  It is safe to use \samp{from types import *} ---
the module does not export any names besides the ones listed here.
New names exported by future versions of this module will all end in
\samp{Type}.

Typical use is for functions that do different things depending on
their argument types, like the following:

\begin{verbatim}
from types import *
def delete(list, item):
    if type(item) is IntType:
       del list[item]
    else:
       list.remove(item)
\end{verbatim}

The module defines the following names:

\begin{datadesc}{NoneType}
The type of \code{None}.
\end{datadesc}

\begin{datadesc}{TypeType}
The type of type objects (such as returned by
\function{type()}\bifuncindex{type}).
\end{datadesc}

\begin{datadesc}{IntType}
The type of integers (e.g. \code{1}).
\end{datadesc}

\begin{datadesc}{LongType}
The type of long integers (e.g. \code{1L}).
\end{datadesc}

\begin{datadesc}{FloatType}
The type of floating point numbers (e.g. \code{1.0}).
\end{datadesc}

\begin{datadesc}{ComplexType}
The type of complex numbers (e.g. \code{1.0j}).
\end{datadesc}

\begin{datadesc}{StringType}
The type of character strings (e.g. \code{'Spam'}).
\end{datadesc}

\begin{datadesc}{UnicodeType}
The type of Unicode character strings (e.g. \code{u'Spam'}).
\end{datadesc}

\begin{datadesc}{TupleType}
The type of tuples (e.g. \code{(1, 2, 3, 'Spam')}).
\end{datadesc}

\begin{datadesc}{ListType}
The type of lists (e.g. \code{[0, 1, 2, 3]}).
\end{datadesc}

\begin{datadesc}{DictType}
The type of dictionaries (e.g. \code{\{'Bacon': 1, 'Ham': 0\}}).
\end{datadesc}

\begin{datadesc}{DictionaryType}
An alternate name for \code{DictType}.
\end{datadesc}

\begin{datadesc}{FunctionType}
The type of user-defined functions and lambdas.
\end{datadesc}

\begin{datadesc}{LambdaType}
An alternate name for \code{FunctionType}.
\end{datadesc}

\begin{datadesc}{CodeType}
The type for code objects such as returned by
\function{compile()}\bifuncindex{compile}.
\end{datadesc}

\begin{datadesc}{ClassType}
The type of user-defined classes.
\end{datadesc}

\begin{datadesc}{InstanceType}
The type of instances of user-defined classes.
\end{datadesc}

\begin{datadesc}{MethodType}
The type of methods of user-defined class instances.
\end{datadesc}

\begin{datadesc}{UnboundMethodType}
An alternate name for \code{MethodType}.
\end{datadesc}

\begin{datadesc}{BuiltinFunctionType}
The type of built-in functions like \function{len()} or
\function{sys.exit()}.
\end{datadesc}

\begin{datadesc}{BuiltinMethodType}
An alternate name for \code{BuiltinFunction}.
\end{datadesc}

\begin{datadesc}{ModuleType}
The type of modules.
\end{datadesc}

\begin{datadesc}{FileType}
The type of open file objects such as \code{sys.stdout}.
\end{datadesc}

\begin{datadesc}{XRangeType}
The type of range objects returned by
\function{xrange()}\bifuncindex{xrange}.
\end{datadesc}

\begin{datadesc}{SliceType}
The type of objects returned by
\function{slice()}\bifuncindex{slice}.
\end{datadesc}

\begin{datadesc}{EllipsisType}
The type of \code{Ellipsis}.
\end{datadesc}

\begin{datadesc}{TracebackType}
The type of traceback objects such as found in
\code{sys.exc_traceback}.
\end{datadesc}

\begin{datadesc}{FrameType}
The type of frame objects such as found in \code{tb.tb_frame} if
\code{tb} is a traceback object.
\end{datadesc}

\begin{datadesc}{BufferType}
The type of buffer objects created by the
\function{buffer()}\bifuncindex{buffer} function.
\end{datadesc}

\section{Built-in Exceptions}
\label{module-exceptions}
\stmodindex{exceptions}

Exceptions can be class objects or string objects.  While
traditionally, most exceptions have been string objects, in Python
1.5, all standard exceptions have been converted to class objects,
and users are encouraged to the the same.  The source code for those
exceptions is present in the standard library module
\code{exceptions}; this module never needs to be imported explicitly.

For backward compatibility, when Python is invoked with the \code{-X}
option, the standard exceptions are strings.  This may be needed to
run some code that breaks because of the different semantics of class
based exceptions.  The \code{-X} option will become obsolete in future
Python versions, so the recommended solution is to fix the code.

Two distinct string objects with the same value are considered different
exceptions.  This is done to force programmers to use exception names
rather than their string value when specifying exception handlers.
The string value of all built-in exceptions is their name, but this is
not a requirement for user-defined exceptions or exceptions defined by
library modules.

For class exceptions, in a \code{try} statement with an \code{except}
clause that mentions a particular class, that clause also handles
any exception classes derived from that class (but not exception
classes from which \emph{it} is derived).  Two exception classes
that are not related via subclassing are never equivalent, even if
they have the same name.
\stindex{try}
\stindex{except}

The built-in exceptions listed below can be generated by the
interpreter or built-in functions.  Except where mentioned, they have
an ``associated value'' indicating the detailed cause of the error.
This may be a string or a tuple containing several items of
information (e.g., an error code and a string explaining the code).
The associated value is the second argument to the \code{raise}
statement.  For string exceptions, the associated value itself will be
stored in the variable named as the second argument of the
\code{except} clause (if any).  For class exceptions derived from
the root class \code{Exception}, that variable receives the exception
instance, and the associated value is present as the exception
instance's \code{args} attribute; this is a tuple even if the second
argument to \code{raise} was not (then it is a singleton tuple).
\stindex{raise}

User code can raise built-in exceptions.  This can be used to test an
exception handler or to report an error condition ``just like'' the
situation in which the interpreter raises the same exception; but
beware that there is nothing to prevent user code from raising an
inappropriate error.

\setindexsubitem{(built-in exception base class)}

The following exceptions are only used as base classes for other
exceptions.  When string-based standard exceptions are used, they
are tuples containing the directly derived classes.

\begin{excdesc}{Exception}
The root class for exceptions.  All built-in exceptions are derived
from this class.  All user-defined exceptions should also be derived
from this class, but this is not (yet) enforced.  The \code{str()}
function, when applied to an instance of this class (or most derived
classes) returns the string value of the argument or arguments, or an
empty string if no arguments were given to the constructor.  When used
as a sequence, this accesses the arguments given to the constructor
(handy for backward compatibility with old code).
\end{excdesc}

\begin{excdesc}{StandardError}
The base class for built-in exceptions.  All built-in exceptions are
derived from this class, which is itself derived from the root class
\code{Exception}.
\end{excdesc}

\begin{excdesc}{ArithmeticError}
The base class for those built-in exceptions that are raised for
various arithmetic errors: \code{OverflowError},
\code{ZeroDivisionError}, \code{FloatingPointError}.
\end{excdesc}

\begin{excdesc}{LookupError}
The base class for thise exceptions that are raised when a key or
index used on a mapping or sequence is invalid: \code{IndexError},
\code{KeyError}.
\end{excdesc}

\setindexsubitem{(built-in exception)}

The following exceptions are the exceptions that are actually raised.
They are class objects, except when the \code{-X} option is used to
revert back to string-based standard exceptions.

\begin{excdesc}{AssertionError}
Raised when an \code{assert} statement fails.
\stindex{assert}
\end{excdesc}

\begin{excdesc}{AttributeError}
% xref to attribute reference?
  Raised when an attribute reference or assignment fails.  (When an
  object does not support attribute references or attribute assignments
  at all, \code{TypeError} is raised.)
\end{excdesc}

\begin{excdesc}{EOFError}
% XXXJH xrefs here
  Raised when one of the built-in functions (\code{input()} or
  \code{raw_input()}) hits an end-of-file condition (\EOF{}) without
  reading any data.
% XXXJH xrefs here
  (N.B.: the \code{read()} and \code{readline()} methods of file
  objects return an empty string when they hit \EOF{}.)  No associated value.
\end{excdesc}

\begin{excdesc}{FloatingPointError}
Raised when a floating point operation fails.  This exception is
always defined, but can only be raised when Python is configured with
the \code{--with-fpectl} option, or the \code{WANT_SIGFPE_HANDLER}
symbol is defined in the \file{config.h} file.
\end{excdesc}

\begin{excdesc}{IOError}
% XXXJH xrefs here
  Raised when an I/O operation (such as a \code{print} statement, the
  built-in \code{open()} function or a method of a file object) fails
  for an I/O-related reason, e.g., ``file not found'' or ``disk full''.

When class exceptions are used, and this exception is instantiated as
\code{IOError(errno, strerror)}, the instance has two additional
attributes \code{errno} and \code{strerror} set to the error code and
the error message, respectively.  These attributes default to
\code{None}.
\end{excdesc}

\begin{excdesc}{ImportError}
% XXXJH xref to import statement?
  Raised when an \code{import} statement fails to find the module
  definition or when a \code{from {\rm \ldots} import} fails to find a
  name that is to be imported.
\end{excdesc}

\begin{excdesc}{IndexError}
% XXXJH xref to sequences
  Raised when a sequence subscript is out of range.  (Slice indices are
  silently truncated to fall in the allowed range; if an index is not a
  plain integer, \code{TypeError} is raised.)
\end{excdesc}

\begin{excdesc}{KeyError}
% XXXJH xref to mapping objects?
  Raised when a mapping (dictionary) key is not found in the set of
  existing keys.
\end{excdesc}

\begin{excdesc}{KeyboardInterrupt}
  Raised when the user hits the interrupt key (normally
  \kbd{Control-C} or \kbd{DEL}).  During execution, a check for
  interrupts is made regularly.
% XXXJH xrefs here
  Interrupts typed when a built-in function \function{input()} or
  \function{raw_input()}) is waiting for input also raise this
  exception.  This exception has no associated value.
\end{excdesc}

\begin{excdesc}{MemoryError}
  Raised when an operation runs out of memory but the situation may
  still be rescued (by deleting some objects).  The associated value is
  a string indicating what kind of (internal) operation ran out of memory.
  Note that because of the underlying memory management architecture
  (\C{}'s \code{malloc()} function), the interpreter may not always be able
  to completely recover from this situation; it nevertheless raises an
  exception so that a stack traceback can be printed, in case a run-away
  program was the cause.
\end{excdesc}

\begin{excdesc}{NameError}
  Raised when a local or global name is not found.  This applies only
  to unqualified names.  The associated value is the name that could
  not be found.
\end{excdesc}

\begin{excdesc}{OverflowError}
% XXXJH reference to long's and/or int's?
  Raised when the result of an arithmetic operation is too large to be
  represented.  This cannot occur for long integers (which would rather
  raise \code{MemoryError} than give up).  Because of the lack of
  standardization of floating point exception handling in \C{}, most
  floating point operations also aren't checked.  For plain integers,
  all operations that can overflow are checked except left shift, where
  typical applications prefer to drop bits than raise an exception.
\end{excdesc}

\begin{excdesc}{RuntimeError}
  Raised when an error is detected that doesn't fall in any of the
  other categories.  The associated value is a string indicating what
  precisely went wrong.  (This exception is mostly a relic from a
  previous version of the interpreter; it is not used very much any
  more.)
\end{excdesc}

\begin{excdesc}{SyntaxError}
% XXXJH xref to these functions?
  Raised when the parser encounters a syntax error.  This may occur in
  an \code{import} statement, in an \code{exec} statement, in a call
  to the built-in function \code{eval()} or \code{input()}, or
  when reading the initial script or standard input (also
  interactively).

When class exceptions are used, instances of this class have
atttributes \code{filename}, \code{lineno}, \code{offset} and
\code{text} for easier access to the details; for string exceptions,
the associated value is usually a tuple of the form
\code{(message, (filename, lineno, offset, text))}.
For class exceptions, \code{str()} returns only the message.
\end{excdesc}

\begin{excdesc}{SystemError}
  Raised when the interpreter finds an internal error, but the
  situation does not look so serious to cause it to abandon all hope.
  The associated value is a string indicating what went wrong (in
  low-level terms).
  
  You should report this to the author or maintainer of your Python
  interpreter.  Be sure to report the version string of the Python
  interpreter (\code{sys.version}; it is also printed at the start of an
  interactive Python session), the exact error message (the exception's
  associated value) and if possible the source of the program that
  triggered the error.
\end{excdesc}

\begin{excdesc}{SystemExit}
% XXXJH xref to module sys?
  This exception is raised by the \code{sys.exit()} function.  When it
  is not handled, the Python interpreter exits; no stack traceback is
  printed.  If the associated value is a plain integer, it specifies the
  system exit status (passed to \C{}'s \code{exit()} function); if it is
  \code{None}, the exit status is zero; if it has another type (such as
  a string), the object's value is printed and the exit status is one.

When class exceptions are used, the instance has an attribute
\code{code} which is set to the proposed exit status or error message
(defaulting to \code{None}).
  
  A call to \code{sys.exit()} is translated into an exception so that
  clean-up handlers (\code{finally} clauses of \code{try} statements)
  can be executed, and so that a debugger can execute a script without
  running the risk of losing control.  The \code{os._exit()} function
  can be used if it is absolutely positively necessary to exit
  immediately (e.g., after a \code{fork()} in the child process).
\end{excdesc}

\begin{excdesc}{TypeError}
  Raised when a built-in operation or function is applied to an object
  of inappropriate type.  The associated value is a string giving
  details about the type mismatch.
\end{excdesc}

\begin{excdesc}{ValueError}
  Raised when a built-in operation or function receives an argument
  that has the right type but an inappropriate value, and the
  situation is not described by a more precise exception such as
  \code{IndexError}.
\end{excdesc}

\begin{excdesc}{ZeroDivisionError}
  Raised when the second argument of a division or modulo operation is
  zero.  The associated value is a string indicating the type of the
  operands and the operation.
\end{excdesc}

\section{Built-in Functions}

The Python interpreter has a number of functions built into it that
are always available.  They are listed here in alphabetical order.


\renewcommand{\indexsubitem}{(built-in function)}
\begin{funcdesc}{abs}{x}
  Return the absolute value of a number.  The argument may be a plain
  or long integer or a floating point number.
\end{funcdesc}

\begin{funcdesc}{apply}{function\, args}
The \var{function} argument must be a callable object (a user-defined or
built-in function or method, or a class object) and the \var{args}
argument must be a tuple.  The \var{function} is called with
\var{args} as argument list; the number of arguments is the the length
of the tuple.  (This is different from just calling
\code{\var{func}(\var{args})}, since in that case there is always
exactly one argument.)
\end{funcdesc}

\begin{funcdesc}{chr}{i}
  Return a string of one character whose \ASCII{} code is the integer
  \var{i}, e.g., \code{chr(97)} returns the string \code{'a'}.  This is the
  inverse of \code{ord()}.  The argument must be in the range [0..255],
  inclusive.
\end{funcdesc}

\begin{funcdesc}{cmp}{x\, y}
  Compare the two objects \var{x} and \var{y} and return an integer
  according to the outcome.  The return value is negative if \code{\var{x}
  < \var{y}}, zero if \code{\var{x} == \var{y}} and strictly positive if
  \code{\var{x} > \var{y}}.
\end{funcdesc}

\begin{funcdesc}{coerce}{x\, y}
  Return a tuple consisting of the two numeric arguments converted to
  a common type, using the same rules as used by arithmetic
  operations.
\end{funcdesc}

\begin{funcdesc}{compile}{string\, filename\, kind}
  Compile the \var{string} into a code object.  Code objects can be
  executed by a \code{exec()} statement or evaluated by a call to
  \code{eval()}.  The \var{filename} argument should
  give the file from which the code was read; pass e.g. \code{'<string>'}
  if it wasn't read from a file.  The \var{kind} argument specifies
  what kind of code must be compiled; it can be \code{'exec'} if
  \var{string} consists of a sequence of statements, or \code{'eval'}
  if it consists of a single expression.
\end{funcdesc}

\begin{funcdesc}{delattr}{object\, name}
  This is a relative of \code{setattr}.  The arguments are an
  object and a string.  The string must be the name
  of one of the object's attributes.  The function deletes
  the named attribute, provided the object allows it.  For example,
  \code{setattr(\var{x}, '\var{foobar}')} is equivalent to
  \code{del \var{x}.\var{foobar}}.
\end{funcdesc}

\begin{funcdesc}{dir}{}
  Without arguments, return the list of names in the current local
  symbol table.  With a module, class or class instance object as
  argument (or anything else that has a \code{__dict__} attribute),
  returns the list of names in that object's attribute dictionary.
  The resulting list is sorted.  For example:

\bcode\begin{verbatim}
>>> import sys
>>> dir()
['sys']
>>> dir(sys)
['argv', 'exit', 'modules', 'path', 'stderr', 'stdin', 'stdout']
>>> 
\end{verbatim}\ecode
\end{funcdesc}

\begin{funcdesc}{divmod}{a\, b}
  Take two numbers as arguments and return a pair of integers
  consisting of their integer quotient and remainder.  With mixed
  operand types, the rules for binary arithmetic operators apply.  For
  plain and long integers, the result is the same as
  \code{(\var{a} / \var{b}, \var{a} \%{} \var{b})}.
  For floating point numbers the result is the same as
  \code{(math.floor(\var{a} / \var{b}), \var{a} \%{} \var{b})}.
\end{funcdesc}

\begin{funcdesc}{eval}{expression\optional{\, globals\optional{\, locals}}}
  The arguments are a string and two optional dictionaries.  The
  \var{expression} argument is parsed and evaluated as a Python
  expression (technically speaking, a condition list) using the
  \var{globals} and \var{locals} dictionaries as global and local name
  space.  If the \var{globals} dictionary is omitted it defaults to
  the \var{locals} dictionary.  If both dictionaries are omitted, the
  expression is executed in the environment where \code{eval} is
  called.  The return value is the result of the evaluated expression.
  Syntax errors are reported as exceptions.  Example:

\bcode\begin{verbatim}
>>> x = 1
>>> print eval('x+1')
2
>>> 
\end{verbatim}\ecode

  This function can also be used to execute arbitrary code objects
  (e.g. created by \code{compile()}).  In this case pass a code
  object instead of a string.  The code object must have been compiled
  passing \code{'eval'} to the \var{kind} argument.

  Note: dynamic execution of statements is supported by the
  \code{exec} statement.  Execution of statements from a file is
  supported by the \code{execfile()} function.

\end{funcdesc}

\begin{funcdesc}{execfile}{file\optional{\, globals\optional{\, locals}}}
  This function is similar to the \code{eval()} function or the
  \code{exec} statement, but parses a file instead of a string.  It is
  different from the \code{import} statement in that it does not use
  the module administration -- it reads the file unconditionally and
  does not create a new module.

  The arguments are a file name and two optional dictionaries.  The
  file is parsed and evaluated as a sequence of Python statements
  (similarly to a module) using the \var{globals} and \var{locals}
  dictionaries as global and local name space.  If the \var{globals}
  dictionary is omitted it defaults to the \var{locals} dictionary.
  If both dictionaries are omitted, the expression is executed in the
  environment where \code{execfile} is called.  The return value is
  None.
\end{funcdesc}

\begin{funcdesc}{filter}{function\, list}
Construct a list from those elements of \var{list} for which
\var{function} returns true.  If \var{list} is a string or a tuple,
the result also has that type; otherwise it is always a list.  If
\var{function} is \code{None}, the identity function is assumed,
i.e. all elements of \var{list} that are false (zero or empty) are
removed.
\end{funcdesc}

\begin{funcdesc}{float}{x}
  Convert a number to floating point.  The argument may be a plain or
  long integer or a floating point number.
\end{funcdesc}

\begin{funcdesc}{getattr}{object\, name}
  The arguments are an object and a string.  The string must be the
  name
  of one of the object's attributes.  The result is the value of that
  attribute.  For example, \code{getattr(\var{x}, '\var{foobar}')} is equivalent to
  \code{\var{x}.\var{foobar}}.
\end{funcdesc}

\begin{funcdesc}{hasattr}{object\, name}
  The arguments are an object and a string.  The result is 1 if the
  string is the name of one of the object's attributes, 0 if not.
  (This is implemented by calling \code{getattr(object, name)} and
  seeing whether it raises an exception or not.)
\end{funcdesc}

\begin{funcdesc}{hash}{object}
  Return the hash value of the object (if it has one).  Hash values
  are 32-bit integers.  They are used to quickly compare dictionary
  keys during a dictionary lookup.  Numeric values that compare equal
  have the same hash value (even if they are of different types, e.g.
  1 and 1.0).
\end{funcdesc}

\begin{funcdesc}{hex}{x}
  Convert a number to a hexadecimal string.  The result is a valid
  Python expression.
\end{funcdesc}

\begin{funcdesc}{id}{object}
  Return the `identity' of an object.  This is an integer which is
  guaranteed to be unique and constant for this object during its
  lifetime.  (Two objects whose lifetimes are disjunct may have the
  same id() value.)  (Implementation note: this is the address of the
  object.)
\end{funcdesc}

\begin{funcdesc}{input}{\optional{prompt}}
  Almost equivalent to \code{eval(raw_input(\var{prompt}))}.  Like
  \code{raw_input()}, the \var{prompt} argument is optional.  The difference
  is that a long input expression may be broken over multiple lines using
  the backslash convention.
\end{funcdesc}

\begin{funcdesc}{int}{x}
  Convert a number to a plain integer.  The argument may be a plain or
  long integer or a floating point number.
\end{funcdesc}

\begin{funcdesc}{len}{s}
  Return the length (the number of items) of an object.  The argument
  may be a sequence (string, tuple or list) or a mapping (dictionary).
\end{funcdesc}

\begin{funcdesc}{long}{x}
  Convert a number to a long integer.  The argument may be a plain or
  long integer or a floating point number.
\end{funcdesc}

\begin{funcdesc}{map}{function\, list\, ...}
Apply \var{function} to every item of \var{list} and return a list
of the results.  If additional \var{list} arguments are passed, 
\var{function} must take that many arguments and is applied to
the items of all lists in parallel; if a list is shorter than another
it is assumed to be extended with \code{None} items.  If
\var{function} is \code{None}, the identity function is assumed; if
there are multiple list arguments, \code{map} returns a list
consisting of tuples containing the corresponding items from all lists
(i.e. a kind of transpose operation).  The \var{list} arguments may be
any kind of sequence; the result is always a list.
\end{funcdesc}

\begin{funcdesc}{max}{s}
  Return the largest item of a non-empty sequence (string, tuple or
  list).
\end{funcdesc}

\begin{funcdesc}{min}{s}
  Return the smallest item of a non-empty sequence (string, tuple or
  list).
\end{funcdesc}

\begin{funcdesc}{oct}{x}
  Convert a number to an octal string.  The result is a valid Python
  expression.
\end{funcdesc}

\begin{funcdesc}{open}{filename\, \optional{mode\optional{\, bufsize}}}
  Return a new file object (described earlier under Built-in Types).
  The first two arguments are the same as for \code{stdio}'s
  \code{fopen()}: \var{filename} is the file name to be opened,
  \var{mode} indicates how the file is to be opened: \code{'r'} for
  reading, \code{'w'} for writing (truncating an existing file), and
  \code{'a'} opens it for appending.  Modes \code{'r+'}, \code{'w+'} and
  \code{'a+'} open the file for updating, provided the underlying
  \code{stdio} library understands this.  On systems that differentiate
  between binary and text files, \code{'b'} appended to the mode opens
  the file in binary mode.  If the file cannot be opened, \code{IOError}
  is raised.
If \var{mode} is omitted, it defaults to \code{'r'}.
The optional \var{bufsize} argument specifies the file's desired
buffer size: 0 means unbuffered, 1 means line buffered, any other
positive value means use a buffer of (approximately) that size.  A
negative \var{bufsize} means to use the system default, which is
usually line buffered for for tty devices and fully buffered for other
files.%
\footnote{Specifying a buffer size currently has no effect on systems
that don't have \code{setvbuf()}.  The interface to specify the buffer
size is not done using a method that calls \code{setvbuf()}, because
that may dump core when called after any I/O has been performed, and
there's no reliable way to determine whether this is the case.}
\end{funcdesc}

\begin{funcdesc}{ord}{c}
  Return the \ASCII{} value of a string of one character.  E.g.,
  \code{ord('a')} returns the integer \code{97}.  This is the inverse of
  \code{chr()}.
\end{funcdesc}

\begin{funcdesc}{pow}{x\, y\optional{\, z}}
  Return \var{x} to the power \var{y}; if \var{z} is present, return
  \var{x} to the power \var{y}, modulo \var{z} (computed more
  efficiently that \code{pow(\var{x}, \var{y}) \% \var{z}}).
  The arguments must have
  numeric types.  With mixed operand types, the rules for binary
  arithmetic operators apply.  The effective operand type is also the
  type of the result; if the result is not expressible in this type, the
  function raises an exception; e.g., \code{pow(2, -1)} or \code{pow(2,
  35000)} is not allowed.
\end{funcdesc}

\begin{funcdesc}{range}{\optional{start\,} end\optional{\, step}}
  This is a versatile function to create lists containing arithmetic
  progressions.  It is most often used in \code{for} loops.  The
  arguments must be plain integers.  If the \var{step} argument is
  omitted, it defaults to \code{1}.  If the \var{start} argument is
  omitted, it defaults to \code{0}.  The full form returns a list of
  plain integers \code{[\var{start}, \var{start} + \var{step},
  \var{start} + 2 * \var{step}, \ldots]}.  If \var{step} is positive,
  the last element is the largest \code{\var{start} + \var{i} *
  \var{step}} less than \var{end}; if \var{step} is negative, the last
  element is the largest \code{\var{start} + \var{i} * \var{step}}
  greater than \var{end}.  \var{step} must not be zero.  Example:

\bcode\begin{verbatim}
>>> range(10)
[0, 1, 2, 3, 4, 5, 6, 7, 8, 9]
>>> range(1, 11)
[1, 2, 3, 4, 5, 6, 7, 8, 9, 10]
>>> range(0, 30, 5)
[0, 5, 10, 15, 20, 25]
>>> range(0, 10, 3)
[0, 3, 6, 9]
>>> range(0, -10, -1)
[0, -1, -2, -3, -4, -5, -6, -7, -8, -9]
>>> range(0)
[]
>>> range(1, 0)
[]
>>> 
\end{verbatim}\ecode
\end{funcdesc}

\begin{funcdesc}{raw_input}{\optional{prompt}}
  If the \var{prompt} argument is present, it is written to standard output
  without a trailing newline.  The function then reads a line from input,
  converts it to a string (stripping a trailing newline), and returns that.
  When \EOF{} is read, \code{EOFError} is raised. Example:

\bcode\begin{verbatim}
>>> s = raw_input('--> ')
--> Monty Python's Flying Circus
>>> s
'Monty Python\'s Flying Circus'
>>> 
\end{verbatim}\ecode
\end{funcdesc}

\begin{funcdesc}{reduce}{function\, list\optional{\, initializer}}
Apply the binary \var{function} to the items of \var{list} so as to
reduce the list to a single value.  E.g.,
\code{reduce(lambda x, y: x*y, \var{list}, 1)} returns the product of
the elements of \var{list}.  The optional \var{initializer} can be
thought of as being prepended to \var{list} so as to allow reduction
of an empty \var{list}.  The \var{list} arguments may be any kind of
sequence.
\end{funcdesc}

\begin{funcdesc}{reload}{module}
  Re-parse and re-initialize an already imported \var{module}.  The
  argument must be a module object, so it must have been successfully
  imported before.  This is useful if you have edited the module source
  file using an external editor and want to try out the new version
  without leaving the Python interpreter.  Note that if a module is
  syntactically correct but its initialization fails, the first
  \code{import} statement for it does not import the name, but does
  create a (partially initialized) module object; to reload the module
  you must first \code{import} it again (this will just make the
  partially initialized module object available) before you can
  \code{reload()} it.
\end{funcdesc}

\begin{funcdesc}{repr}{object}
Return a string containing a printable representation of an object.
This is the same value yielded by conversions (reverse quotes).
It is sometimes useful to be able to access this operation as an
ordinary function.  For many types, this function makes an attempt
to return a string that would yield an object with the same value
when passed to \code{eval()}.
\end{funcdesc}

\begin{funcdesc}{round}{x\, n}
  Return the floating point value \var{x} rounded to \var{n} digits
  after the decimal point.  If \var{n} is omitted, it defaults to zero.
  The result is a floating point number.  Values are rounded to the
  closest multiple of 10 to the power minus \var{n}; if two multiples
  are equally close, rounding is done away from 0 (so e.g.
  \code{round(0.5)} is \code{1.0} and \code{round(-0.5)} is \code{-1.0}).
\end{funcdesc}

\begin{funcdesc}{setattr}{object\, name\, value}
  This is the counterpart of \code{getattr}.  The arguments are an
  object, a string and an arbitrary value.  The string must be the name
  of one of the object's attributes.  The function assigns the value to
  the attribute, provided the object allows it.  For example,
  \code{setattr(\var{x}, '\var{foobar}', 123)} is equivalent to
  \code{\var{x}.\var{foobar} = 123}.
\end{funcdesc}

\begin{funcdesc}{str}{object}
Return a string containing a nicely printable representation of an
object.  For strings, this returns the string itself.  The difference
with \code{repr(\var{object}} is that \code{str(\var{object}} does not
always attempt to return a string that is acceptable to \code{eval()};
its goal is to return a printable string.
\end{funcdesc}

\begin{funcdesc}{tuple}{object}
Return a tuple whose items are the same and in the same order as
\var{object}'s items.  If \var{object} is alread a tuple, it
is returned unchanged.  For instance, \code{tuple('abc')} returns
returns \code{('a', 'b', 'c')} and \code{tuple([1, 2, 3])} returns
\code{(1, 2, 3)}.
\end{funcdesc}

\begin{funcdesc}{type}{object}
% XXXJH xref to buil-in objects here?
  Return the type of an \var{object}.  The return value is a type
  object.  There is not much you can do with type objects except compare
  them to other type objects; e.g., the following checks if a variable
  is a string:

\bcode\begin{verbatim}
>>> if type(x) == type(''): print 'It is a string'
\end{verbatim}\ecode
\end{funcdesc}

\begin{funcdesc}{vars}{}
Without arguments, return a dictionary corresponding to the current
local symbol table.  With a module, class or class instance object as
argument (or anything else that has a \code{__dict__} attribute),
returns a dictionary corresponding to the object's symbol table.
The returned dictionary should not be modified: the effects on the
corresponding symbol table are undefined.%
\footnote{In the current implementation, local variable bindings
cannot normally be affected this way, but variables retrieved from
other scopes can be.  This may change.}
\end{funcdesc}

\begin{funcdesc}{xrange}{\optional{start\,} end\optional{\, step}}
This function is very similar to \code{range()}, but returns an
``xrange object'' instead of a list.  This is an opaque sequence type
which yields the same values as the corresponding list, without
actually storing them all simultaneously.  The advantage of
\code{xrange()} over \code{range()} is minimal (since \code{xrange()}
still has to create the values when asked for them) except when a very
large range is used on a memory-starved machine (e.g. DOS) or when all
of the range's elements are never used (e.g. when the loop is usually
terminated with \code{break}).
\end{funcdesc}


\chapter{Python Services}
\label{python}

The modules described in this chapter provide a wide range of services
related to the Python interpreter and its interaction with its
environment.  Here's an overview:

\begin{description}

\item[sys]
--- Access system specific parameters and functions.

\item[types]
--- Names for all built-in types.

\item[UserDict]
--- Class wrapper for dictionary objects.

\item[UserList]
--- Class wrapper for list objects.

\item[operator]
--- All Python's standard operators as built-in functions.

\item[traceback]
--- Print or retrieve a stack traceback.

\item[pickle]
--- Convert Python objects to streams of bytes and back.

\item[cPickle]
--- Faster version of \module{pickle}, but not subclassable.

\item[copy_reg]
--- Register \module{pickle} support functions.

\item[shelve]
--- Python object persistency.

\item[copy]
--- Shallow and deep copy operations.

\item[marshal]
--- Convert Python objects to streams of bytes and back (with
different constraints).

\item[imp]
--- Access the implementation of the \keyword{import} statement.

\item[parser]
--- Retrieve and submit parse trees from and to the runtime support
environment.

\item[symbol]
--- Constants representing internal nodes of the parse tree.

\item[token]
--- Constants representing terminal nodes of the parse tree.

\item[keyword]
--- Test whether a string is a keyword in the Python language.

\item[code]
--- Code object services.

\item[pprint]
--- Data pretty printer.

\item[dis]
--- Disassembler.

\item[site]
--- A standard way to reference site-specific modules.

\item[user]
--- A standard way to reference user-specific modules.

\item[__builtin__]
--- The set of built-in functions.

\item[__main__]
--- The environment where the top-level script is run.

\end{description}
		% Python Services
\section{Built-in Module \sectcode{sys}}
\label{module-sys}

\bimodindex{sys}
This module provides access to some variables used or maintained by the
interpreter and to functions that interact strongly with the interpreter.
It is always available.

\setindexsubitem{(in module sys)}

\begin{datadesc}{argv}
  The list of command line arguments passed to a Python script.
  \code{argv[0]} is the script name (it is operating system
  dependent whether this is a full pathname or not).
  If the command was executed using the \samp{-c} command line option
  to the interpreter, \code{argv[0]} is set to the string
  \code{"-c"}.
  If no script name was passed to the Python interpreter,
  \code{argv} has zero length.
\end{datadesc}

\begin{datadesc}{builtin_module_names}
  A tuple of strings giving the names of all modules that are compiled
  into this Python interpreter.  (This information is not available in
  any other way --- \code{modules.keys()} only lists the imported
  modules.)
\end{datadesc}

\begin{funcdesc}{exc_info}{}
This function returns a tuple of three values that give information
about the exception that is currently being handled.  The information
returned is specific both to the current thread and to the current
stack frame.  If the current stack frame is not handling an exception,
the information is taken from the calling stack frame, or its caller,
and so on until a stack frame is found that is handling an exception.
Here, ``handling an exception'' is defined as ``executing or having
executed an except clause.''  For any stack frame, only
information about the most recently handled exception is accessible.

If no exception is being handled anywhere on the stack, a tuple
containing three \code{None} values is returned.  Otherwise, the
values returned are
\code{(\var{type}, \var{value}, \var{traceback})}.
Their meaning is: \var{type} gets the exception type of the exception
being handled (a string or class object); \var{value} gets the
exception parameter (its \dfn{associated value} or the second argument
to \keyword{raise}, which is always a class instance if the exception
type is a class object); \var{traceback} gets a traceback object (see
the Reference Manual) which encapsulates the call stack at the point
where the exception originally occurred.
\obindex{traceback}

\strong{Warning:} assigning the \var{traceback} return value to a
local variable in a function that is handling an exception will cause
a circular reference. This will prevent anything referenced by a local
variable in the same function or by the traceback from being garbage
collected.  Since most functions don't need access to the traceback,
the best solution is to use something like
\code{type, value = sys.exc_info()[:2]}
to extract only the exception type and value.  If you do need the
traceback, make sure to delete it after use (best done with a
\keyword{try} ... \keyword{finally} statement) or to call
\function{exc_info()} in a function that does not itself handle an
exception.
\end{funcdesc}

\begin{datadesc}{exc_type}
\dataline{exc_value}
\dataline{exc_traceback}
\deprecated {1.5}
            {Use \function{exc_info()} instead.}
Since they are global variables, they are not specific to the current
thread, so their use is not safe in a multi-threaded program.  When no
exception is being handled, \code{exc_type} is set to \code{None} and
the other two are undefined.
\end{datadesc}

\begin{datadesc}{exec_prefix}
A string giving the site-specific
directory prefix where the platform-dependent Python files are
installed; by default, this is also \code{"/usr/local"}.  This can be
set at build time with the \code{-}\code{-exec-prefix} argument to the
\program{configure} script.  Specifically, all configuration files
(e.g. the \file{config.h} header file) are installed in the directory
\code{exec_prefix + "/lib/python\var{version}/config"}, and shared library
modules are installed in
\code{exec_prefix + "/lib/python\var{version}/lib-dynload"},
where \var{version} is equal to \code{version[:3]}.
\end{datadesc}

\begin{funcdesc}{exit}{n}
  Exit from Python with numeric exit status \var{n}.  This is
  implemented by raising the \exception{SystemExit} exception, so cleanup
  actions specified by finally clauses of \keyword{try} statements
  are honored, and it is possible to catch the exit attempt at an outer
  level.
\end{funcdesc}

\begin{datadesc}{exitfunc}
  This value is not actually defined by the module, but can be set by
  the user (or by a program) to specify a clean-up action at program
  exit.  When set, it should be a parameterless function.  This function
  will be called when the interpreter exits in any way (except when a
  fatal error occurs: in that case the interpreter's internal state
  cannot be trusted).
\end{datadesc}

\begin{funcdesc}{getrefcount}{object}
Return the reference count of the \var{object}.  The count returned is
generally one higher than you might expect, because it includes the
(temporary) reference as an argument to \code{getrefcount()}.
\end{funcdesc}

\begin{datadesc}{last_type}
\dataline{last_value}
\dataline{last_traceback}
These three variables are not always defined; they are set when an
exception is not handled and the interpreter prints an error message
and a stack traceback.  Their intended use is to allow an interactive
user to import a debugger module and engage in post-mortem debugging
without having to re-execute the command that caused the error.
(Typical use is \samp{import pdb; pdb.pm()} to enter the post-mortem
debugger; see the chapter ``The Python Debugger'' for more
information.)
\refstmodindex{pdb}

The meaning of the variables is the same
as that of the return values from \function{exc_info()} above.
(Since there is only one interactive thread, thread-safety is not a
concern for these variables, unlike for \code{exc_type} etc.)
\end{datadesc}

\begin{datadesc}{modules}
  This is a dictionary that maps module names to modules which have
  already been loaded.  This can be manipulated to force reloading of
  modules and other tricks.  Note that removing a module from this
  dictionary is \emph{not} the same as calling
  \function{reload()}\bifuncindex{reload} on the corresponding module
  object.
\end{datadesc}

\begin{datadesc}{path}
\indexiii{module}{search}{path}
  A list of strings that specifies the search path for modules.
  Initialized from the environment variable \code{\$PYTHONPATH}, or an
  installation-dependent default.  

The first item of this list, \code{path[0]}, is the 
directory containing the script that was used to invoke the Python 
interpreter.  If the script directory is not available (e.g.  if the 
interpreter is invoked interactively or if the script is read from 
standard input), \code{path[0]} is the empty string, which directs 
Python to search modules in the current directory first.  Notice that 
the script directory is inserted \emph{before} the entries inserted as 
a result of \code{\$PYTHONPATH}.  
\end{datadesc}

\begin{datadesc}{platform}
This string contains a platform identifier, e.g. \code{'sunos5'} or
\code{'linux1'}.  This can be used to append platform-specific
components to \code{path}, for instance. 
\end{datadesc}

\begin{datadesc}{prefix}
A string giving the site-specific directory prefix where the platform
independent Python files are installed; by default, this is the string
\code{"/usr/local"}.  This can be set at build time with the
\code{-}\code{-prefix} argument to the \program{configure} script.  The main
collection of Python library modules is installed in the directory
\code{prefix + "/lib/python\var{version}"} while the platform
independent header files (all except \file{config.h}) are stored in
\code{prefix + "/include/python\var{version}"},
where \var{version} is equal to \code{version[:3]}.

\end{datadesc}

\begin{datadesc}{ps1}
\dataline{ps2}
\index{interpreter prompts}
\index{prompts, interpreter}
  Strings specifying the primary and secondary prompt of the
  interpreter.  These are only defined if the interpreter is in
  interactive mode.  Their initial values in this case are
  \code{'>>> '} and \code{'... '}.  If a non-string object is assigned
  to either variable, its \function{str()} is re-evaluated each time
  the interpreter prepares to read a new interactive command; this can
  be used to implement a dynamic prompt.
\end{datadesc}

\begin{funcdesc}{setcheckinterval}{interval}
Set the interpreter's ``check interval''.  This integer value
determines how often the interpreter checks for periodic things such
as thread switches and signal handlers.  The default is \code{10}, meaning
the check is performed every 10 Python virtual instructions.  Setting
it to a larger value may increase performance for programs using
threads.  Setting it to a value \code{<=} 0 checks every virtual instruction,
maximizing responsiveness as well as overhead.
\end{funcdesc}

\begin{funcdesc}{settrace}{tracefunc}
  Set the system's trace function, which allows you to implement a
  Python source code debugger in Python.  See section ``How It Works''
  in the chapter on the Python Debugger.
\end{funcdesc}
\index{trace function}
\index{debugger}

\begin{funcdesc}{setprofile}{profilefunc}
  Set the system's profile function, which allows you to implement a
  Python source code profiler in Python.  See the chapter on the
  Python Profiler.  The system's profile function
  is called similarly to the system's trace function (see
  \function{settrace()}), but it isn't called for each executed line of
  code (only on call and return and when an exception occurs).  Also,
  its return value is not used, so it can just return \code{None}.
\end{funcdesc}
\index{profile function}
\index{profiler}

\begin{datadesc}{stdin}
\dataline{stdout}
\dataline{stderr}
  File objects corresponding to the interpreter's standard input,
  output and error streams.  \code{stdin} is used for all
  interpreter input except for scripts but including calls to
  \function{input()}\bifuncindex{input} and
  \function{raw_input()}\bifuncindex{raw_input}.  \code{stdout} is used
  for the output of \keyword{print} and expression statements and for the
  prompts of \function{input()} and \function{raw_input()}.  The interpreter's
  own prompts and (almost all of) its error messages go to
  \code{stderr}.  \code{stdout} and \code{stderr} needn't
  be built-in file objects: any object is acceptable as long as it has
  a \method{write()} method that takes a string argument.  (Changing these
  objects doesn't affect the standard I/O streams of processes
  executed by \function{os.popen()}, \function{os.system()} or the
  \function{exec*()} family of functions in the \module{os} module.)
\refstmodindex{os}
\end{datadesc}

\begin{datadesc}{tracebacklimit}
When this variable is set to an integer value, it determines the
maximum number of levels of traceback information printed when an
unhandled exception occurs.  The default is \code{1000}.  When set to
0 or less, all traceback information is suppressed and only the
exception type and value are printed.
\end{datadesc}

\begin{datadesc}{version}
A string containing the version number of the Python interpreter.  
\end{datadesc}

\input{libtypes2}		% types is already taken :-(
\section{\module{traceback} ---
         Print or retrieve a stack traceback}

\declaremodule{standard}{traceback}
\modulesynopsis{Print or retrieve a stack traceback.}


This module provides a standard interface to extract, format and print
stack traces of Python programs.  It exactly mimics the behavior of
the Python interpreter when it prints a stack trace.  This is useful
when you want to print stack traces under program control, e.g. in a
``wrapper'' around the interpreter.

The module uses traceback objects --- this is the object type
that is stored in the variables \code{sys.exc_traceback} and
\code{sys.last_traceback} and returned as the third item from
\function{sys.exc_info()}.
\obindex{traceback}

The module defines the following functions:

\begin{funcdesc}{print_tb}{traceback\optional{, limit\optional{, file}}}
Print up to \var{limit} stack trace entries from \var{traceback}.  If
\var{limit} is omitted or \code{None}, all entries are printed.
If \var{file} is omitted or \code{None}, the output goes to
\code{sys.stderr}; otherwise it should be an open file or file-like
object to receive the output.
\end{funcdesc}

\begin{funcdesc}{print_exception}{type, value, traceback\optional{,
                                  limit\optional{, file}}}
Print exception information and up to \var{limit} stack trace entries
from \var{traceback} to \var{file}.
This differs from \function{print_tb()} in the
following ways: (1) if \var{traceback} is not \code{None}, it prints a
header \samp{Traceback (innermost last):}; (2) it prints the
exception \var{type} and \var{value} after the stack trace; (3) if
\var{type} is \exception{SyntaxError} and \var{value} has the appropriate
format, it prints the line where the syntax error occurred with a
caret indicating the approximate position of the error.
\end{funcdesc}

\begin{funcdesc}{print_exc}{\optional{limit\optional{, file}}}
This is a shorthand for `\code{print_exception(sys.exc_type,}
\code{sys.exc_value,} \code{sys.exc_traceback,} \var{limit}\code{,}
\var{file}\code{)}'.  (In fact, it uses \code{sys.exc_info()} to
retrieve the same information in a thread-safe way.)
\end{funcdesc}

\begin{funcdesc}{print_last}{\optional{limit\optional{, file}}}
This is a shorthand for `\code{print_exception(sys.last_type,}
\code{sys.last_value,} \code{sys.last_traceback,} \var{limit}\code{,}
\var{file}\code{)}'.
\end{funcdesc}

\begin{funcdesc}{print_stack}{\optional{f\optional{, limit\optional{, file}}}}
This function prints a stack trace from its invocation point.  The
optional \var{f} argument can be used to specify an alternate stack
frame to start.  The optional \var{limit} and \var{file} arguments have the
same meaning as for \function{print_exception()}.
\end{funcdesc}

\begin{funcdesc}{extract_tb}{traceback\optional{, limit}}
Return a list of up to \var{limit} ``pre-processed'' stack trace
entries extracted from the traceback object \var{traceback}.  It is
useful for alternate formatting of stack traces.  If \var{limit} is
omitted or \code{None}, all entries are extracted.  A
``pre-processed'' stack trace entry is a quadruple (\var{filename},
\var{line number}, \var{function name}, \var{text}) representing
the information that is usually printed for a stack trace.  The
\var{text} is a string with leading and trailing whitespace
stripped; if the source is not available it is \code{None}.
\end{funcdesc}

\begin{funcdesc}{extract_stack}{\optional{f\optional{, limit}}}
Extract the raw traceback from the current stack frame.  The return
value has the same format as for \function{extract_tb()}.  The
optional \var{f} and \var{limit} arguments have the same meaning as
for \function{print_stack()}.
\end{funcdesc}

\begin{funcdesc}{format_list}{list}
Given a list of tuples as returned by \function{extract_tb()} or
\function{extract_stack()}, return a list of strings ready for
printing.  Each string in the resulting list corresponds to the item
with the same index in the argument list.  Each string ends in a
newline; the strings may contain internal newlines as well, for those
items whose source text line is not \code{None}.
\end{funcdesc}

\begin{funcdesc}{format_exception_only}{type, value}
Format the exception part of a traceback.  The arguments are the
exception type and value such as given by \code{sys.last_type} and
\code{sys.last_value}.  The return value is a list of strings, each
ending in a newline.  Normally, the list contains a single string;
however, for \code{SyntaxError} exceptions, it contains several lines
that (when printed) display detailed information about where the
syntax error occurred.  The message indicating which exception
occurred is the always last string in the list.
\end{funcdesc}

\begin{funcdesc}{format_exception}{type, value, tb\optional{, limit}}
Format a stack trace and the exception information.  The arguments 
have the same meaning as the corresponding arguments to
\function{print_exception()}.  The return value is a list of strings,
each ending in a newline and some containing internal newlines.  When
these lines are concatenated and printed, exactly the same text is
printed as does \function{print_exception()}.
\end{funcdesc}

\begin{funcdesc}{format_tb}{tb\optional{, limit}}
A shorthand for \code{format_list(extract_tb(\var{tb}, \var{limit}))}.
\end{funcdesc}

\begin{funcdesc}{format_stack}{\optional{f\optional{, limit}}}
A shorthand for \code{format_list(extract_stack(\var{f}, \var{limit}))}.
\end{funcdesc}

\begin{funcdesc}{tb_lineno}{tb}
This function returns the current line number set in the traceback
object.  This is normally the same as the \code{\var{tb}.tb_lineno}
field of the object, but when optimization is used (the -O flag) this
field is not updated correctly; this function calculates the correct
value.
\end{funcdesc}


\subsection{Traceback Example \label{traceback-example}}

This simple example implements a basic read-eval-print loop, similar
to (but less useful than) the standard Python interactive interpreter
loop.  For a more complete implementation of the interpreter loop,
refer to the \refmodule{code} module.

\begin{verbatim}
import sys, traceback

def run_user_code(envdir):
    source = raw_input(">>> ")
    try:
        exec source in envdir
    except:
        print "Exception in user code:"
        print '-'*60
        traceback.print_exc(file=sys.stdout)
        print '-'*60

envdir = {}
while 1:
    run_user_code(envdir)
\end{verbatim}

\section{\module{pickle} --- Python object serialization}

\declaremodule{standard}{pickle}
\modulesynopsis{Convert Python objects to streams of bytes and back.}
% Substantial improvements by Jim Kerr <jbkerr@sr.hp.com>.
% Rewritten by Barry Warsaw <barry@zope.com>

\index{persistence}
\indexii{persistent}{objects}
\indexii{serializing}{objects}
\indexii{marshalling}{objects}
\indexii{flattening}{objects}
\indexii{pickling}{objects}

The \module{pickle} module implements a fundamental, but powerful
algorithm for serializing and de-serializing a Python object
structure.  ``Pickling'' is the process whereby a Python object
hierarchy is converted into a byte stream, and ``unpickling'' is the
inverse operation, whereby a byte stream is converted back into an
object hierarchy.  Pickling (and unpickling) is alternatively known as
``serialization'', ``marshalling,''\footnote{Don't confuse this with
the \refmodule{marshal} module} or ``flattening'',
however, to avoid confusion, the terms used here are ``pickling'' and
``unpickling''.

This documentation describes both the \module{pickle} module and the 
\refmodule{cPickle} module.

\subsection{Relationship to other Python modules}

The \module{pickle} module has an optimized cousin called the
\module{cPickle} module.  As its name implies, \module{cPickle} is
written in C, so it can be up to 1000 times faster than
\module{pickle}.  However it does not support subclassing of the
\function{Pickler()} and \function{Unpickler()} classes, because in
\module{cPickle} these are functions, not classes.  Most applications
have no need for this functionality, and can benefit from the improved
performance of \module{cPickle}.  Other than that, the interfaces of
the two modules are nearly identical; the common interface is
described in this manual and differences are pointed out where
necessary.  In the following discussions, we use the term ``pickle''
to collectively describe the \module{pickle} and
\module{cPickle} modules.

The data streams the two modules produce are guaranteed to be
interchangeable.

Python has a more primitive serialization module called
\refmodule{marshal}, but in general
\module{pickle} should always be the preferred way to serialize Python
objects.  \module{marshal} exists primarily to support Python's
\file{.pyc} files.

The \module{pickle} module differs from \refmodule{marshal} several
significant ways:

\begin{itemize}

\item The \module{pickle} module keeps track of the objects it has
      already serialized, so that later references to the same object
      won't be serialized again.  \module{marshal} doesn't do this.

      This has implications both for recursive objects and object
      sharing.  Recursive objects are objects that contain references
      to themselves.  These are not handled by marshal, and in fact,
      attempting to marshal recursive objects will crash your Python
      interpreter.  Object sharing happens when there are multiple
      references to the same object in different places in the object
      hierarchy being serialized.  \module{pickle} stores such objects
      only once, and ensures that all other references point to the
      master copy.  Shared objects remain shared, which can be very
      important for mutable objects.

\item \module{marshal} cannot be used to serialize user-defined
      classes and their instances.  \module{pickle} can save and
      restore class instances transparently, however the class
      definition must be importable and live in the same module as
      when the object was stored.

\item The \module{marshal} serialization format is not guaranteed to
      be portable across Python versions.  Because its primary job in
      life is to support \file{.pyc} files, the Python implementers
      reserve the right to change the serialization format in
      non-backwards compatible ways should the need arise.  The
      \module{pickle} serialization format is guaranteed to be
      backwards compatible across Python releases.

\end{itemize}

\begin{notice}[warning]
The \module{pickle} module is not intended to be secure against
erroneous or maliciously constructed data.  Never unpickle data
received from an untrusted or unauthenticated source.
\end{notice}

Note that serialization is a more primitive notion than persistence;
although
\module{pickle} reads and writes file objects, it does not handle the
issue of naming persistent objects, nor the (even more complicated)
issue of concurrent access to persistent objects.  The \module{pickle}
module can transform a complex object into a byte stream and it can
transform the byte stream into an object with the same internal
structure.  Perhaps the most obvious thing to do with these byte
streams is to write them onto a file, but it is also conceivable to
send them across a network or store them in a database.  The module
\refmodule{shelve} provides a simple interface
to pickle and unpickle objects on DBM-style database files.

\subsection{Data stream format}

The data format used by \module{pickle} is Python-specific.  This has
the advantage that there are no restrictions imposed by external
standards such as XDR\index{XDR}\index{External Data Representation}
(which can't represent pointer sharing); however it means that
non-Python programs may not be able to reconstruct pickled Python
objects.

By default, the \module{pickle} data format uses a printable \ASCII{}
representation.  This is slightly more voluminous than a binary
representation.  The big advantage of using printable \ASCII{} (and of
some other characteristics of \module{pickle}'s representation) is that
for debugging or recovery purposes it is possible for a human to read
the pickled file with a standard text editor.

There are currently 3 different protocols which can be used for pickling.

\begin{itemize}

\item Protocol version 0 is the original ASCII protocol and is backwards
compatible with earlier versions of Python.

\item Protocol version 1 is the old binary format which is also compatible
with earlier versions of Python.

\item Protocol version 2 was introduced in Python 2.3.  It provides
much more efficient pickling of new-style classes.

\end{itemize}

Refer to PEP 307 for more information.

If a \var{protocol} is not specified, protocol 0 is used.
If \var{protocol} is specified as a negative value
or \constant{HIGHEST_PROTOCOL},
the highest protocol version available will be used.

\versionchanged[The \var{bin} parameter is deprecated and only provided
for backwards compatibility.  You should use the \var{protocol}
parameter instead]{2.3}

A binary format, which is slightly more efficient, can be chosen by
specifying a true value for the \var{bin} argument to the
\class{Pickler} constructor or the \function{dump()} and \function{dumps()}
functions.  A \var{protocol} version >= 1 implies use of a binary format.

\subsection{Usage}

To serialize an object hierarchy, you first create a pickler, then you
call the pickler's \method{dump()} method.  To de-serialize a data
stream, you first create an unpickler, then you call the unpickler's
\method{load()} method.  The \module{pickle} module provides the
following constant:

\begin{datadesc}{HIGHEST_PROTOCOL}
The highest protocol version available.  This value can be passed
as a \var{protocol} value.
\versionadded{2.3}
\end{datadesc}

The \module{pickle} module provides the
following functions to make this process more convenient:

\begin{funcdesc}{dump}{obj, file\optional{, protocol\optional{, bin}}}
Write a pickled representation of \var{obj} to the open file object
\var{file}.  This is equivalent to
\code{Pickler(\var{file}, \var{protocol}, \var{bin}).dump(\var{obj})}.

If the \var{protocol} parameter is omitted, protocol 0 is used.
If \var{protocol} is specified as a negative value
or \constant{HIGHEST_PROTOCOL},
the highest protocol version will be used.

\versionchanged[The \var{protocol} parameter was added.
The \var{bin} parameter is deprecated and only provided
for backwards compatibility.  You should use the \var{protocol}
parameter instead]{2.3}

If the optional \var{bin} argument is true, the binary pickle format
is used; otherwise the (less efficient) text pickle format is used
(for backwards compatibility, this is the default).

\var{file} must have a \method{write()} method that accepts a single
string argument.  It can thus be a file object opened for writing, a
\refmodule{StringIO} object, or any other custom
object that meets this interface.
\end{funcdesc}

\begin{funcdesc}{load}{file}
Read a string from the open file object \var{file} and interpret it as
a pickle data stream, reconstructing and returning the original object
hierarchy.  This is equivalent to \code{Unpickler(\var{file}).load()}.

\var{file} must have two methods, a \method{read()} method that takes
an integer argument, and a \method{readline()} method that requires no
arguments.  Both methods should return a string.  Thus \var{file} can
be a file object opened for reading, a
\module{StringIO} object, or any other custom
object that meets this interface.

This function automatically determines whether the data stream was
written in binary mode or not.
\end{funcdesc}

\begin{funcdesc}{dumps}{obj\optional{, protocol\optional{, bin}}}
Return the pickled representation of the object as a string, instead
of writing it to a file.

If the \var{protocol} parameter is omitted, protocol 0 is used.
If \var{protocol} is specified as a negative value
or \constant{HIGHEST_PROTOCOL},
the highest protocol version will be used.

\versionchanged[The \var{protocol} parameter was added.
The \var{bin} parameter is deprecated and only provided
for backwards compatibility.  You should use the \var{protocol}
parameter instead]{2.3}

If the optional \var{bin} argument is
true, the binary pickle format is used; otherwise the (less efficient)
text pickle format is used (this is the default).
\end{funcdesc}

\begin{funcdesc}{loads}{string}
Read a pickled object hierarchy from a string.  Characters in the
string past the pickled object's representation are ignored.
\end{funcdesc}

The \module{pickle} module also defines three exceptions:

\begin{excdesc}{PickleError}
A common base class for the other exceptions defined below.  This
inherits from \exception{Exception}.
\end{excdesc}

\begin{excdesc}{PicklingError}
This exception is raised when an unpicklable object is passed to
the \method{dump()} method.
\end{excdesc}

\begin{excdesc}{UnpicklingError}
This exception is raised when there is a problem unpickling an object.
Note that other exceptions may also be raised during unpickling,
including (but not necessarily limited to) \exception{AttributeError},
\exception{EOFError}, \exception{ImportError}, and \exception{IndexError}.
\end{excdesc}

The \module{pickle} module also exports two callables\footnote{In the
\module{pickle} module these callables are classes, which you could
subclass to customize the behavior.  However, in the \refmodule{cPickle}
module these callables are factory functions and so cannot be
subclassed.  One common reason to subclass is to control what
objects can actually be unpickled.  See section~\ref{pickle-sub} for
more details.}, \class{Pickler} and \class{Unpickler}:

\begin{classdesc}{Pickler}{file\optional{, protocol\optional{, bin}}}
This takes a file-like object to which it will write a pickle data
stream.  

If the \var{protocol} parameter is omitted, protocol 0 is used.
If \var{protocol} is specified as a negative value,
the highest protocol version will be used.

\versionchanged[The \var{bin} parameter is deprecated and only provided
for backwards compatibility.  You should use the \var{protocol}
parameter instead]{2.3}

Optional \var{bin} if true, tells the pickler to use the more
efficient binary pickle format, otherwise the \ASCII{} format is used
(this is the default).

\var{file} must have a \method{write()} method that accepts a single
string argument.  It can thus be an open file object, a
\module{StringIO} object, or any other custom
object that meets this interface.
\end{classdesc}

\class{Pickler} objects define one (or two) public methods:

\begin{methoddesc}[Pickler]{dump}{obj}
Write a pickled representation of \var{obj} to the open file object
given in the constructor.  Either the binary or \ASCII{} format will
be used, depending on the value of the \var{bin} flag passed to the
constructor.
\end{methoddesc}

\begin{methoddesc}[Pickler]{clear_memo}{}
Clears the pickler's ``memo''.  The memo is the data structure that
remembers which objects the pickler has already seen, so that shared
or recursive objects pickled by reference and not by value.  This
method is useful when re-using picklers.

\begin{notice}
Prior to Python 2.3, \method{clear_memo()} was only available on the
picklers created by \refmodule{cPickle}.  In the \module{pickle} module,
picklers have an instance variable called \member{memo} which is a
Python dictionary.  So to clear the memo for a \module{pickle} module
pickler, you could do the following:

\begin{verbatim}
mypickler.memo.clear()
\end{verbatim}

Code that does not need to support older versions of Python should
simply use \method{clear_memo()}.
\end{notice}
\end{methoddesc}

It is possible to make multiple calls to the \method{dump()} method of
the same \class{Pickler} instance.  These must then be matched to the
same number of calls to the \method{load()} method of the
corresponding \class{Unpickler} instance.  If the same object is
pickled by multiple \method{dump()} calls, the \method{load()} will
all yield references to the same object.\footnote{\emph{Warning}: this
is intended for pickling multiple objects without intervening
modifications to the objects or their parts.  If you modify an object
and then pickle it again using the same \class{Pickler} instance, the
object is not pickled again --- a reference to it is pickled and the
\class{Unpickler} will return the old value, not the modified one.
There are two problems here: (1) detecting changes, and (2)
marshalling a minimal set of changes.  Garbage Collection may also
become a problem here.}

\class{Unpickler} objects are defined as:

\begin{classdesc}{Unpickler}{file}
This takes a file-like object from which it will read a pickle data
stream.  This class automatically determines whether the data stream
was written in binary mode or not, so it does not need a flag as in
the \class{Pickler} factory.

\var{file} must have two methods, a \method{read()} method that takes
an integer argument, and a \method{readline()} method that requires no
arguments.  Both methods should return a string.  Thus \var{file} can
be a file object opened for reading, a
\module{StringIO} object, or any other custom
object that meets this interface.
\end{classdesc}

\class{Unpickler} objects have one (or two) public methods:

\begin{methoddesc}[Unpickler]{load}{}
Read a pickled object representation from the open file object given
in the constructor, and return the reconstituted object hierarchy
specified therein.
\end{methoddesc}

\begin{methoddesc}[Unpickler]{noload}{}
This is just like \method{load()} except that it doesn't actually
create any objects.  This is useful primarily for finding what's
called ``persistent ids'' that may be referenced in a pickle data
stream.  See section~\ref{pickle-protocol} below for more details.

\strong{Note:} the \method{noload()} method is currently only
available on \class{Unpickler} objects created with the
\module{cPickle} module.  \module{pickle} module \class{Unpickler}s do
not have the \method{noload()} method.
\end{methoddesc}

\subsection{What can be pickled and unpickled?}

The following types can be pickled:

\begin{itemize}

\item \code{None}, \code{True}, and \code{False}

\item integers, long integers, floating point numbers, complex numbers

\item normal and Unicode strings

\item tuples, lists, sets, and dictionaries containing only picklable objects

\item functions defined at the top level of a module

\item built-in functions defined at the top level of a module

\item classes that are defined at the top level of a module

\item instances of such classes whose \member{__dict__} or
\method{__setstate__()} is picklable  (see
section~\ref{pickle-protocol} for details)

\end{itemize}

Attempts to pickle unpicklable objects will raise the
\exception{PicklingError} exception; when this happens, an unspecified
number of bytes may have already been written to the underlying file.

Note that functions (built-in and user-defined) are pickled by ``fully
qualified'' name reference, not by value.  This means that only the
function name is pickled, along with the name of module the function
is defined in.  Neither the function's code, nor any of its function
attributes are pickled.  Thus the defining module must be importable
in the unpickling environment, and the module must contain the named
object, otherwise an exception will be raised.\footnote{The exception
raised will likely be an \exception{ImportError} or an
\exception{AttributeError} but it could be something else.}

Similarly, classes are pickled by named reference, so the same
restrictions in the unpickling environment apply.  Note that none of
the class's code or data is pickled, so in the following example the
class attribute \code{attr} is not restored in the unpickling
environment:

\begin{verbatim}
class Foo:
    attr = 'a class attr'

picklestring = pickle.dumps(Foo)
\end{verbatim}

These restrictions are why picklable functions and classes must be
defined in the top level of a module.

Similarly, when class instances are pickled, their class's code and
data are not pickled along with them.  Only the instance data are
pickled.  This is done on purpose, so you can fix bugs in a class or
add methods to the class and still load objects that were created with
an earlier version of the class.  If you plan to have long-lived
objects that will see many versions of a class, it may be worthwhile
to put a version number in the objects so that suitable conversions
can be made by the class's \method{__setstate__()} method.

\subsection{The pickle protocol
\label{pickle-protocol}}\setindexsubitem{(pickle protocol)}

This section describes the ``pickling protocol'' that defines the
interface between the pickler/unpickler and the objects that are being
serialized.  This protocol provides a standard way for you to define,
customize, and control how your objects are serialized and
de-serialized.  The description in this section doesn't cover specific
customizations that you can employ to make the unpickling environment
slightly safer from untrusted pickle data streams; see section~\ref{pickle-sub}
for more details.

\subsubsection{Pickling and unpickling normal class
    instances\label{pickle-inst}}

When a pickled class instance is unpickled, its \method{__init__()}
method is normally \emph{not} invoked.  If it is desirable that the
\method{__init__()} method be called on unpickling, an old-style class
can define a method \method{__getinitargs__()}, which should return a
\emph{tuple} containing the arguments to be passed to the class
constructor (i.e. \method{__init__()}).  The
\method{__getinitargs__()} method is called at
pickle time; the tuple it returns is incorporated in the pickle for
the instance.
\withsubitem{(copy protocol)}{\ttindex{__getinitargs__()}}
\withsubitem{(instance constructor)}{\ttindex{__init__()}}

\withsubitem{(copy protocol)}{\ttindex{__getnewargs__()}}

New-style types can provide a \method{__getnewargs__()} method that is
used for protocol 2.  Implementing this method is needed if the type
establishes some internal invariants when the instance is created, or
if the memory allocation is affected by the values passed to the
\method{__new__()} method for the type (as it is for tuples and
strings).  Instances of a new-style type \class{C} are created using

\begin{alltt}
obj = C.__new__(C, *\var{args})
\end{alltt}

where \var{args} is the result of calling \method{__getnewargs__()} on
the original object; if there is no \method{__getnewargs__()}, an
empty tuple is assumed.

\withsubitem{(copy protocol)}{
  \ttindex{__getstate__()}\ttindex{__setstate__()}}
\withsubitem{(instance attribute)}{
  \ttindex{__dict__}}

Classes can further influence how their instances are pickled; if the
class defines the method \method{__getstate__()}, it is called and the
return state is pickled as the contents for the instance, instead of
the contents of the instance's dictionary.  If there is no
\method{__getstate__()} method, the instance's \member{__dict__} is
pickled.

Upon unpickling, if the class also defines the method
\method{__setstate__()}, it is called with the unpickled
state.\footnote{These methods can also be used to implement copying
class instances.}  If there is no \method{__setstate__()} method, the
pickled state must be a dictionary and its items are assigned to the
new instance's dictionary.  If a class defines both
\method{__getstate__()} and \method{__setstate__()}, the state object
needn't be a dictionary and these methods can do what they
want.\footnote{This protocol is also used by the shallow and deep
copying operations defined in the
\refmodule{copy} module.}

\begin{notice}[warning]
  For new-style classes, if \method{__getstate__()} returns a false
  value, the \method{__setstate__()} method will not be called.
\end{notice}


\subsubsection{Pickling and unpickling extension types}

When the \class{Pickler} encounters an object of a type it knows
nothing about --- such as an extension type --- it looks in two places
for a hint of how to pickle it.  One alternative is for the object to
implement a \method{__reduce__()} method.  If provided, at pickling
time \method{__reduce__()} will be called with no arguments, and it
must return either a string or a tuple.

If a string is returned, it names a global variable whose contents are
pickled as normal.  The string returned by \method{__reduce__} should
be the object's local name relative to its module; the pickle module
searches the module namespace to determine the object's module.

When a tuple is returned, it must be between two and five elements
long. Optional elements can either be omitted, or \code{None} can be provided 
as their value.  The semantics of each element are:

\begin{itemize}

\item A callable object that will be called to create the initial
version of the object.  The next element of the tuple will provide
arguments for this callable, and later elements provide additional
state information that will subsequently be used to fully reconstruct
the pickled date.

In the unpickling environment this object must be either a class, a
callable registered as a ``safe constructor'' (see below), or it must
have an attribute \member{__safe_for_unpickling__} with a true value.
Otherwise, an \exception{UnpicklingError} will be raised in the
unpickling environment.  Note that as usual, the callable itself is
pickled by name.

\item A tuple of arguments for the callable object, or \code{None}.
\deprecated{2.3}{If this item is \code{None}, then instead of calling
the callable directly, its \method{__basicnew__()} method is called
without arguments; this method should also return the unpickled
object.  Providing \code{None} is deprecated, however; return a
tuple of arguments instead.}

\item Optionally, the object's state, which will be passed to
      the object's \method{__setstate__()} method as described in
      section~\ref{pickle-inst}.  If the object has no
      \method{__setstate__()} method, then, as above, the value must
      be a dictionary and it will be added to the object's
      \member{__dict__}.

\item Optionally, an iterator (and not a sequence) yielding successive
list items.  These list items will be pickled, and appended to the
object using either \code{obj.append(\var{item})} or
\code{obj.extend(\var{list_of_items})}.  This is primarily used for
list subclasses, but may be used by other classes as long as they have
\method{append()} and \method{extend()} methods with the appropriate
signature.  (Whether \method{append()} or \method{extend()} is used
depends on which pickle protocol version is used as well as the number
of items to append, so both must be supported.)

\item Optionally, an iterator (not a sequence)
yielding successive dictionary items, which should be tuples of the
form \code{(\var{key}, \var{value})}.  These items will be pickled
and stored to the object using \code{obj[\var{key}] = \var{value}}.
This is primarily used for dictionary subclasses, but may be used by
other classes as long as they implement \method{__setitem__}.

\end{itemize}

It is sometimes useful to know the protocol version when implementing
\method{__reduce__}.  This can be done by implementing a method named
\method{__reduce_ex__} instead of \method{__reduce__}.
\method{__reduce_ex__}, when it exists, is called in preference over
\method{__reduce__} (you may still provide \method{__reduce__} for
backwards compatibility).  The \method{__reduce_ex__} method will be
called with a single integer argument, the protocol version.

The \class{object} class implements both \method{__reduce__} and
\method{__reduce_ex__}; however, if a subclass overrides
\method{__reduce__} but not \method{__reduce_ex__}, the
\method{__reduce_ex__} implementation detects this and calls
\method{__reduce__}.

An alternative to implementing a \method{__reduce__()} method on the
object to be pickled, is to register the callable with the
\refmodule[copyreg]{copy_reg} module.  This module provides a way
for programs to register ``reduction functions'' and constructors for
user-defined types.   Reduction functions have the same semantics and
interface as the \method{__reduce__()} method described above, except
that they are called with a single argument, the object to be pickled.

The registered constructor is deemed a ``safe constructor'' for purposes
of unpickling as described above.


\subsubsection{Pickling and unpickling external objects}

For the benefit of object persistence, the \module{pickle} module
supports the notion of a reference to an object outside the pickled
data stream.  Such objects are referenced by a ``persistent id'',
which is just an arbitrary string of printable \ASCII{} characters.
The resolution of such names is not defined by the \module{pickle}
module; it will delegate this resolution to user defined functions on
the pickler and unpickler.\footnote{The actual mechanism for
associating these user defined functions is slightly different for
\module{pickle} and \module{cPickle}.  The description given here
works the same for both implementations.  Users of the \module{pickle}
module could also use subclassing to effect the same results,
overriding the \method{persistent_id()} and \method{persistent_load()}
methods in the derived classes.}

To define external persistent id resolution, you need to set the
\member{persistent_id} attribute of the pickler object and the
\member{persistent_load} attribute of the unpickler object.

To pickle objects that have an external persistent id, the pickler
must have a custom \function{persistent_id()} method that takes an
object as an argument and returns either \code{None} or the persistent
id for that object.  When \code{None} is returned, the pickler simply
pickles the object as normal.  When a persistent id string is
returned, the pickler will pickle that string, along with a marker
so that the unpickler will recognize the string as a persistent id.

To unpickle external objects, the unpickler must have a custom
\function{persistent_load()} function that takes a persistent id
string and returns the referenced object.

Here's a silly example that \emph{might} shed more light:

\begin{verbatim}
import pickle
from cStringIO import StringIO

src = StringIO()
p = pickle.Pickler(src)

def persistent_id(obj):
    if hasattr(obj, 'x'):
        return 'the value %d' % obj.x
    else:
        return None

p.persistent_id = persistent_id

class Integer:
    def __init__(self, x):
        self.x = x
    def __str__(self):
        return 'My name is integer %d' % self.x

i = Integer(7)
print i
p.dump(i)

datastream = src.getvalue()
print repr(datastream)
dst = StringIO(datastream)

up = pickle.Unpickler(dst)

class FancyInteger(Integer):
    def __str__(self):
        return 'I am the integer %d' % self.x

def persistent_load(persid):
    if persid.startswith('the value '):
        value = int(persid.split()[2])
        return FancyInteger(value)
    else:
        raise pickle.UnpicklingError, 'Invalid persistent id'

up.persistent_load = persistent_load

j = up.load()
print j
\end{verbatim}

In the \module{cPickle} module, the unpickler's
\member{persistent_load} attribute can also be set to a Python
list, in which case, when the unpickler reaches a persistent id, the
persistent id string will simply be appended to this list.  This
functionality exists so that a pickle data stream can be ``sniffed''
for object references without actually instantiating all the objects
in a pickle.\footnote{We'll leave you with the image of Guido and Jim
sitting around sniffing pickles in their living rooms.}  Setting
\member{persistent_load} to a list is usually used in conjunction with
the \method{noload()} method on the Unpickler.

% BAW: Both pickle and cPickle support something called
% inst_persistent_id() which appears to give unknown types a second
% shot at producing a persistent id.  Since Jim Fulton can't remember
% why it was added or what it's for, I'm leaving it undocumented.

\subsection{Subclassing Unpicklers \label{pickle-sub}}

By default, unpickling will import any class that it finds in the
pickle data.  You can control exactly what gets unpickled and what
gets called by customizing your unpickler.  Unfortunately, exactly how
you do this is different depending on whether you're using
\module{pickle} or \module{cPickle}.\footnote{A word of caution: the
mechanisms described here use internal attributes and methods, which
are subject to change in future versions of Python.  We intend to
someday provide a common interface for controlling this behavior,
which will work in either \module{pickle} or \module{cPickle}.}

In the \module{pickle} module, you need to derive a subclass from
\class{Unpickler}, overriding the \method{load_global()}
method.  \method{load_global()} should read two lines from the pickle
data stream where the first line will the name of the module
containing the class and the second line will be the name of the
instance's class.  It then looks up the class, possibly importing the
module and digging out the attribute, then it appends what it finds to
the unpickler's stack.  Later on, this class will be assigned to the
\member{__class__} attribute of an empty class, as a way of magically
creating an instance without calling its class's \method{__init__()}.
Your job (should you choose to accept it), would be to have
\method{load_global()} push onto the unpickler's stack, a known safe
version of any class you deem safe to unpickle.  It is up to you to
produce such a class.  Or you could raise an error if you want to
disallow all unpickling of instances.  If this sounds like a hack,
you're right.  Refer to the source code to make this work.

Things are a little cleaner with \module{cPickle}, but not by much.
To control what gets unpickled, you can set the unpickler's
\member{find_global} attribute to a function or \code{None}.  If it is
\code{None} then any attempts to unpickle instances will raise an
\exception{UnpicklingError}.  If it is a function,
then it should accept a module name and a class name, and return the
corresponding class object.  It is responsible for looking up the
class and performing any necessary imports, and it may raise an
error to prevent instances of the class from being unpickled.

The moral of the story is that you should be really careful about the
source of the strings your application unpickles.

\subsection{Example \label{pickle-example}}

Here's a simple example of how to modify pickling behavior for a
class.  The \class{TextReader} class opens a text file, and returns
the line number and line contents each time its \method{readline()}
method is called. If a \class{TextReader} instance is pickled, all
attributes \emph{except} the file object member are saved. When the
instance is unpickled, the file is reopened, and reading resumes from
the last location. The \method{__setstate__()} and
\method{__getstate__()} methods are used to implement this behavior.

\begin{verbatim}
class TextReader:
    """Print and number lines in a text file."""
    def __init__(self, file):
        self.file = file
        self.fh = open(file)
        self.lineno = 0

    def readline(self):
        self.lineno = self.lineno + 1
        line = self.fh.readline()
        if not line:
            return None
        if line.endswith("\n"):
            line = line[:-1]
        return "%d: %s" % (self.lineno, line)

    def __getstate__(self):
        odict = self.__dict__.copy() # copy the dict since we change it
        del odict['fh']              # remove filehandle entry
        return odict

    def __setstate__(self,dict):
        fh = open(dict['file'])      # reopen file
        count = dict['lineno']       # read from file...
        while count:                 # until line count is restored
            fh.readline()
            count = count - 1
        self.__dict__.update(dict)   # update attributes
        self.fh = fh                 # save the file object
\end{verbatim}

A sample usage might be something like this:

\begin{verbatim}
>>> import TextReader
>>> obj = TextReader.TextReader("TextReader.py")
>>> obj.readline()
'1: #!/usr/local/bin/python'
>>> # (more invocations of obj.readline() here)
... obj.readline()
'7: class TextReader:'
>>> import pickle
>>> pickle.dump(obj,open('save.p','w'))
\end{verbatim}

If you want to see that \refmodule{pickle} works across Python
processes, start another Python session, before continuing.  What
follows can happen from either the same process or a new process.

\begin{verbatim}
>>> import pickle
>>> reader = pickle.load(open('save.p'))
>>> reader.readline()
'8:     "Print and number lines in a text file."'
\end{verbatim}


\begin{seealso}
  \seemodule[copyreg]{copy_reg}{Pickle interface constructor
                                registration for extension types.}

  \seemodule{shelve}{Indexed databases of objects; uses \module{pickle}.}

  \seemodule{copy}{Shallow and deep object copying.}

  \seemodule{marshal}{High-performance serialization of built-in types.}
\end{seealso}


\section{\module{cPickle} --- A faster \module{pickle}}

\declaremodule{builtin}{cPickle}
\modulesynopsis{Faster version of \refmodule{pickle}, but not subclassable.}
\moduleauthor{Jim Fulton}{jim@zope.com}
\sectionauthor{Fred L. Drake, Jr.}{fdrake@acm.org}

The \module{cPickle} module supports serialization and
de-serialization of Python objects, providing an interface and
functionality nearly identical to the
\refmodule{pickle}\refstmodindex{pickle} module.  There are several
differences, the most important being performance and subclassability.

First, \module{cPickle} can be up to 1000 times faster than
\module{pickle} because the former is implemented in C.  Second, in
the \module{cPickle} module the callables \function{Pickler()} and
\function{Unpickler()} are functions, not classes.  This means that
you cannot use them to derive custom pickling and unpickling
subclasses.  Most applications have no need for this functionality and
should benefit from the greatly improved performance of the
\module{cPickle} module.

The pickle data stream produced by \module{pickle} and
\module{cPickle} are identical, so it is possible to use
\module{pickle} and \module{cPickle} interchangeably with existing
pickles.\footnote{Since the pickle data format is actually a tiny
stack-oriented programming language, and some freedom is taken in the
encodings of certain objects, it is possible that the two modules
produce different data streams for the same input objects.  However it
is guaranteed that they will always be able to read each other's
data streams.}

There are additional minor differences in API between \module{cPickle}
and \module{pickle}, however for most applications, they are
interchangeable.  More documentation is provided in the
\module{pickle} module documentation, which
includes a list of the documented differences.



\section{\module{shelve} ---
         Python object persistence}

\declaremodule{standard}{shelve}
\modulesynopsis{Python object persistence.}


A ``shelf'' is a persistent, dictionary-like object.  The difference
with ``dbm'' databases is that the values (not the keys!) in a shelf
can be essentially arbitrary Python objects --- anything that the
\refmodule{pickle} module can handle.  This includes most class
instances, recursive data types, and objects containing lots of shared 
sub-objects.  The keys are ordinary strings.
\refstmodindex{pickle}

To summarize the interface (\code{key} is a string, \code{data} is an
arbitrary object):

\begin{verbatim}
import shelve

d = shelve.open(filename) # open -- file may get suffix added by low-level
                          # library

d[key] = data   # store data at key (overwrites old data if
                # using an existing key)
data = d[key]   # retrieve data at key (raise KeyError if no
                # such key)
del d[key]      # delete data stored at key (raises KeyError
                # if no such key)
flag = d.has_key(key)   # true if the key exists
list = d.keys() # a list of all existing keys (slow!)

d.close()       # close it
\end{verbatim}

In addition to the above, shelve supports all methods that are
supported by dictionaries.  This eases the transition from dictionary
based scripts to those requiring persistent storage.

Restrictions:

\begin{itemize}

\item
The choice of which database package will be used
(e.g. \refmodule{dbm} or \refmodule{gdbm}) depends on which interface
is available.  Therefore it is not safe to open the database directly
using \refmodule{dbm}.  The database is also (unfortunately) subject
to the limitations of \refmodule{dbm}, if it is used --- this means
that (the pickled representation of) the objects stored in the
database should be fairly small, and in rare cases key collisions may
cause the database to refuse updates.
\refbimodindex{dbm}
\refbimodindex{gdbm}

\item
Depending on the implementation, closing a persistent dictionary may
or may not be necessary to flush changes to disk.  The \method{__del__}
method of the \class{Shelf} class calls the \method{close} method, so the
programmer generally need not do this explicitly.

\item
The \module{shelve} module does not support \emph{concurrent} read/write
access to shelved objects.  (Multiple simultaneous read accesses are
safe.)  When a program has a shelf open for writing, no other program
should have it open for reading or writing.  \UNIX{} file locking can
be used to solve this, but this differs across \UNIX{} versions and
requires knowledge about the database implementation used.

\end{itemize}

\begin{classdesc}{Shelf}{dict\optional{, binary=False}}
A subclass of \class{UserDict.DictMixin} which stores pickled values in the
\var{dict} object.  If the \var{binary} parameter is \constant{True}, binary
pickles will be used.  This can provide much more compact storage than plain
text pickles, depending on the nature of the objects stored in the databse.
\end{classdesc}

\begin{classdesc}{BsdDbShelf}{dict\optional{, binary=False}}
A subclass of \class{Shelf} which exposes \method{first}, \method{next},
{}\method{previous}, \method{last} and \method{set_location} which are
available in the \module{bsddb} module but not in other database modules.
The \var{dict} object passed to the constructor must support those methods.
This is generally accomplished by calling one of \function{bsddb.hashopen},
\function{bsddb.btopen} or \function{bsddb.rnopen}.  The optional
\var{binary} parameter has the same interpretation as for the \class{Shelf}
class. 
\end{classdesc}

\begin{classdesc}{DbfilenameShelf}{dict\optional{, flag='c'}\optional{, binary=False}}
A subclass of \class{Shelf} which accepts a filename instead of a dict-like
object.  The underlying file will be opened using \function{anydbm.open}.
By default, the file will be created and opened for both read and write.
The optional \var{binary} parameter has the same interpretation as for the
\class{Shelf} class.
\end{classdesc}

\begin{seealso}
  \seemodule{anydbm}{Generic interface to \code{dbm}-style databases.}
  \seemodule{bsddb}{BSD \code{db} database interface.}
  \seemodule{dbhash}{Thin layer around the \module{bsddb} which provides an
  \function{open} function like the other database modules.}
  \seemodule{dbm}{Standard \UNIX{} database interface.}
  \seemodule{dumbdbm}{Portable implementation of the \code{dbm} interface.}
  \seemodule{gdbm}{GNU database interface, based on the \code{dbm} interface.}
  \seemodule{pickle}{Object serialization used by \module{shelve}.}
  \seemodule{cPickle}{High-performance version of \refmodule{pickle}.}
\end{seealso}

\section{Built-in module \sectcode{copy}}
\stmodindex{copy}
\ttindex{copy}
\ttindex{deepcopy}

This module provides generic (shallow and deep) copying operations.

Interface summary:

\begin{verbatim}
import copy

x = copy.copy(y)	# make a shallow copy of y
x = copy.deepcopy(y)	# make a deep copy of y
\end{verbatim}

For module specific errors, \code{copy.Error} is raised.

The difference between shallow and deep copying is only relevant for
compound objects (objects that contain other objects, like lists or
class instances):

\begin{itemize}

\item
A {\em shallow copy} constructs a new compound object and then (to the
extent possible) inserts {\em references} into it to the objects found
in the original.

\item
A {\em deep copy} constructs a new compound object and then,
recursively, inserts {\em copies} into it of the objects found in the
original.

\end{itemize}

Two problems often exist with deep copy operations that don't exist
with shallow copy operations:

\begin{itemize}

\item
Recursive objects (compound objects that, directly or indirectly,
contain a reference to themselves) may cause a recursive loop.

\item
Because deep copy copies {\em everything} it may copy too much, e.g.
administrative data structures that should be shared even between
copies.

\end{itemize}

Python's \code{deepcopy()} operation avoids these problems by:

\begin{itemize}

\item
keeping a table of objects already copied during the current
copying pass; and

\item
letting user-defined classes override the copying operation or the
set of components copied.

\end{itemize}

This version does not copy types like module, class, function, method,
nor stack trace, stack frame, nor file, socket, window, nor array, nor
any similar types.

Classes can use the same interfaces to control copying that they use
to control pickling: they can define methods called
\code{__getinitargs__()}, \code{__getstate__()} and
\code{__setstate__()}.  See the description of module \code{pickle}
for information on these methods.
\stmodindex{pickle}
\ttindex{__getinitargs__}
\ttindex{__getstate__}
\ttindex{__setstate__}

\section{Built-in Module \sectcode{marshal}}
\label{module-marshal}

\bimodindex{marshal}
This module contains functions that can read and write Python
values in a binary format.  The format is specific to Python, but
independent of machine architecture issues (e.g., you can write a
Python value to a file on a PC, transport the file to a Sun, and read
it back there).  Details of the format are undocumented on purpose;
it may change between Python versions (although it rarely does).%
\footnote{The name of this module stems from a bit of terminology used
by the designers of Modula-3 (amongst others), who use the term
``marshalling'' for shipping of data around in a self-contained form.
Strictly speaking, ``to marshal'' means to convert some data from
internal to external form (in an RPC buffer for instance) and
``unmarshalling'' for the reverse process.}

This is not a general ``persistency'' module.  For general persistency
and transfer of Python objects through RPC calls, see the modules
\code{pickle} and \code{shelve}.  The \code{marshal} module exists
mainly to support reading and writing the ``pseudo-compiled'' code for
Python modules of \samp{.pyc} files.
\refstmodindex{pickle}
\refstmodindex{shelve}
\obindex{code}

Not all Python object types are supported; in general, only objects
whose value is independent from a particular invocation of Python can
be written and read by this module.  The following types are supported:
\code{None}, integers, long integers, floating point numbers,
strings, tuples, lists, dictionaries, and code objects, where it
should be understood that tuples, lists and dictionaries are only
supported as long as the values contained therein are themselves
supported; and recursive lists and dictionaries should not be written
(they will cause infinite loops).

{\bf Caveat:} On machines where C's \code{long int} type has more than
32 bits (such as the DEC Alpha), it
is possible to create plain Python integers that are longer than 32
bits.  Since the current \code{marshal} module uses 32 bits to
transfer plain Python integers, such values are silently truncated.
This particularly affects the use of very long integer literals in
Python modules --- these will be accepted by the parser on such
machines, but will be silently be truncated when the module is read
from the \code{.pyc} instead.%
\footnote{A solution would be to refuse such literals in the parser,
since they are inherently non-portable.  Another solution would be to
let the \code{marshal} module raise an exception when an integer value
would be truncated.  At least one of these solutions will be
implemented in a future version.}

There are functions that read/write files as well as functions
operating on strings.

The module defines these functions:

\renewcommand{\indexsubitem}{(in module marshal)}

\begin{funcdesc}{dump}{value\, file}
  Write the value on the open file.  The value must be a supported
  type.  The file must be an open file object such as
  \code{sys.stdout} or returned by \code{open()} or
  \code{posix.popen()}.
  
  If the value has (or contains an object that has) an unsupported type,
  a \code{ValueError} exception is raised -- but garbage data will also
  be written to the file.  The object will not be properly read back by
  \code{load()}.
\end{funcdesc}

\begin{funcdesc}{load}{file}
  Read one value from the open file and return it.  If no valid value
  is read, raise \code{EOFError}, \code{ValueError} or
  \code{TypeError}.  The file must be an open file object.

  Warning: If an object containing an unsupported type was marshalled
  with \code{dump()}, \code{load()} will substitute \code{None} for the
  unmarshallable type.
\end{funcdesc}

\begin{funcdesc}{dumps}{value}
  Return the string that would be written to a file by
  \code{dump(value, file)}.  The value must be a supported type.
  Raise a \code{ValueError} exception if value has (or contains an
  object that has) an unsupported type.
\end{funcdesc}

\begin{funcdesc}{loads}{string}
  Convert the string to a value.  If no valid value is found, raise
  \code{EOFError}, \code{ValueError} or \code{TypeError}.  Extra
  characters in the string are ignored.
\end{funcdesc}

\section{\module{imp} ---
         Access the \keyword{import} internals}

\declaremodule{builtin}{imp}
\modulesynopsis{Access the implementation of the \keyword{import} statement.}


This\stindex{import} module provides an interface to the mechanisms
used to implement the \keyword{import} statement.  It defines the
following constants and functions:


\begin{funcdesc}{get_magic}{}
\indexii{file}{byte-code}
Return the magic string value used to recognize byte-compiled code
files (\file{.pyc} files).  (This value may be different for each
Python version.)
\end{funcdesc}

\begin{funcdesc}{get_suffixes}{}
Return a list of triples, each describing a particular type of module.
Each triple has the form \code{(\var{suffix}, \var{mode},
\var{type})}, where \var{suffix} is a string to be appended to the
module name to form the filename to search for, \var{mode} is the mode
string to pass to the built-in \function{open()} function to open the
file (this can be \code{'r'} for text files or \code{'rb'} for binary
files), and \var{type} is the file type, which has one of the values
\constant{PY_SOURCE}, \constant{PY_COMPILED}, or
\constant{C_EXTENSION}, described below.
\end{funcdesc}

\begin{funcdesc}{find_module}{name\optional{, path}}
Try to find the module \var{name} on the search path \var{path}.  If
\var{path} is a list of directory names, each directory is searched
for files with any of the suffixes returned by \function{get_suffixes()}
above.  Invalid names in the list are silently ignored (but all list
items must be strings).  If \var{path} is omitted or \code{None}, the
list of directory names given by \code{sys.path} is searched, but
first it searches a few special places: it tries to find a built-in
module with the given name (\constant{C_BUILTIN}), then a frozen module
(\constant{PY_FROZEN}), and on some systems some other places are looked
in as well (on the Mac, it looks for a resource (\constant{PY_RESOURCE});
on Windows, it looks in the registry which may point to a specific
file).

If search is successful, the return value is a triple
\code{(\var{file}, \var{pathname}, \var{description})} where
\var{file} is an open file object positioned at the beginning,
\var{pathname} is the pathname of the
file found, and \var{description} is a triple as contained in the list
returned by \function{get_suffixes()} describing the kind of module found.
If the module does not live in a file, the returned \var{file} is
\code{None}, \var{filename} is the empty string, and the
\var{description} tuple contains empty strings for its suffix and
mode; the module type is as indicate in parentheses above.  If the
search is unsuccessful, \exception{ImportError} is raised.  Other
exceptions indicate problems with the arguments or environment.

This function does not handle hierarchical module names (names
containing dots).  In order to find \var{P}.\var{M}, that is, submodule
\var{M} of package \var{P}, use \function{find_module()} and
\function{load_module()} to find and load package \var{P}, and then use
\function{find_module()} with the \var{path} argument set to
\code{\var{P}.__path__}.  When \var{P} itself has a dotted name, apply
this recipe recursively.
\end{funcdesc}

\begin{funcdesc}{load_module}{name, file, filename, description}
Load a module that was previously found by \function{find_module()} (or by
an otherwise conducted search yielding compatible results).  This
function does more than importing the module: if the module was
already imported, it is equivalent to a
\function{reload()}\bifuncindex{reload}!  The \var{name} argument
indicates the full module name (including the package name, if this is
a submodule of a package).  The \var{file} argument is an open file,
and \var{filename} is the corresponding file name; these can be
\code{None} and \code{''}, respectively, when the module is not being
loaded from a file.  The \var{description} argument is a tuple, as
would be returned by \function{get_suffixes()}, describing what kind
of module must be loaded.

If the load is successful, the return value is the module object;
otherwise, an exception (usually \exception{ImportError}) is raised.

\strong{Important:} the caller is responsible for closing the
\var{file} argument, if it was not \code{None}, even when an exception
is raised.  This is best done using a \keyword{try}
... \keyword{finally} statement.
\end{funcdesc}

\begin{funcdesc}{new_module}{name}
Return a new empty module object called \var{name}.  This object is
\emph{not} inserted in \code{sys.modules}.
\end{funcdesc}

\begin{funcdesc}{lock_held}{}
Return \code{True} if the import lock is currently held, else \code{False}.
On platforms without threads, always return \code{False}.

On platforms with threads, a thread executing an import holds an internal
lock until the import is complete.
This lock blocks other threads from doing an import until the original
import completes, which in turn prevents other threads from seeing
incomplete module objects constructed by the original thread while in
the process of completing its import (and the imports, if any,
triggered by that).
\end{funcdesc}

\begin{funcdesc}{acquire_lock}{}
Acquires the interpreter's import lock for the current thread.  This lock
should be used by import hooks to ensure thread-safety when importing modules.
On platforms without threads, this function does nothing.
\versionadded{2.3}
\end{funcdesc}

\begin{funcdesc}{release_lock}{}
Release the interpreter's import lock.
On platforms without threads, this function does nothing.
\versionadded{2.3}
\end{funcdesc}

The following constants with integer values, defined in this module,
are used to indicate the search result of \function{find_module()}.

\begin{datadesc}{PY_SOURCE}
The module was found as a source file.
\end{datadesc}

\begin{datadesc}{PY_COMPILED}
The module was found as a compiled code object file.
\end{datadesc}

\begin{datadesc}{C_EXTENSION}
The module was found as dynamically loadable shared library.
\end{datadesc}

\begin{datadesc}{PY_RESOURCE}
The module was found as a Mac OS 9 resource.  This value can only be
returned on a Mac OS 9 or earlier Macintosh.
\end{datadesc}

\begin{datadesc}{PKG_DIRECTORY}
The module was found as a package directory.
\end{datadesc}

\begin{datadesc}{C_BUILTIN}
The module was found as a built-in module.
\end{datadesc}

\begin{datadesc}{PY_FROZEN}
The module was found as a frozen module (see \function{init_frozen()}).
\end{datadesc}

The following constant and functions are obsolete; their functionality
is available through \function{find_module()} or \function{load_module()}.
They are kept around for backward compatibility:

\begin{datadesc}{SEARCH_ERROR}
Unused.
\end{datadesc}

\begin{funcdesc}{init_builtin}{name}
Initialize the built-in module called \var{name} and return its module
object.  If the module was already initialized, it will be initialized
\emph{again}.  A few modules cannot be initialized twice --- attempting
to initialize these again will raise an \exception{ImportError}
exception.  If there is no
built-in module called \var{name}, \code{None} is returned.
\end{funcdesc}

\begin{funcdesc}{init_frozen}{name}
Initialize the frozen module called \var{name} and return its module
object.  If the module was already initialized, it will be initialized
\emph{again}.  If there is no frozen module called \var{name},
\code{None} is returned.  (Frozen modules are modules written in
Python whose compiled byte-code object is incorporated into a
custom-built Python interpreter by Python's \program{freeze} utility.
See \file{Tools/freeze/} for now.)
\end{funcdesc}

\begin{funcdesc}{is_builtin}{name}
Return \code{1} if there is a built-in module called \var{name} which
can be initialized again.  Return \code{-1} if there is a built-in
module called \var{name} which cannot be initialized again (see
\function{init_builtin()}).  Return \code{0} if there is no built-in
module called \var{name}.
\end{funcdesc}

\begin{funcdesc}{is_frozen}{name}
Return \code{True} if there is a frozen module (see
\function{init_frozen()}) called \var{name}, or \code{False} if there is
no such module.
\end{funcdesc}

\begin{funcdesc}{load_compiled}{name, pathname, \optional{file}}
\indexii{file}{byte-code}
Load and initialize a module implemented as a byte-compiled code file
and return its module object.  If the module was already initialized,
it will be initialized \emph{again}.  The \var{name} argument is used
to create or access a module object.  The \var{pathname} argument
points to the byte-compiled code file.  The \var{file}
argument is the byte-compiled code file, open for reading in binary
mode, from the beginning.
It must currently be a real file object, not a
user-defined class emulating a file.
\end{funcdesc}

\begin{funcdesc}{load_dynamic}{name, pathname\optional{, file}}
Load and initialize a module implemented as a dynamically loadable
shared library and return its module object.  If the module was
already initialized, it will be initialized \emph{again}.  Some modules
don't like that and may raise an exception.  The \var{pathname}
argument must point to the shared library.  The \var{name} argument is
used to construct the name of the initialization function: an external
C function called \samp{init\var{name}()} in the shared library is
called.  The optional \var{file} argument is ignored.  (Note: using
shared libraries is highly system dependent, and not all systems
support it.)
\end{funcdesc}

\begin{funcdesc}{load_source}{name, pathname\optional{, file}}
Load and initialize a module implemented as a Python source file and
return its module object.  If the module was already initialized, it
will be initialized \emph{again}.  The \var{name} argument is used to
create or access a module object.  The \var{pathname} argument points
to the source file.  The \var{file} argument is the source
file, open for reading as text, from the beginning.
It must currently be a real file
object, not a user-defined class emulating a file.  Note that if a
properly matching byte-compiled file (with suffix \file{.pyc} or
\file{.pyo}) exists, it will be used instead of parsing the given
source file.
\end{funcdesc}


\subsection{Examples}
\label{examples-imp}

The following function emulates what was the standard import statement
up to Python 1.4 (no hierarchical module names).  (This
\emph{implementation} wouldn't work in that version, since
\function{find_module()} has been extended and
\function{load_module()} has been added in 1.4.)

\begin{verbatim}
import imp
import sys

def __import__(name, globals=None, locals=None, fromlist=None):
    # Fast path: see if the module has already been imported.
    try:
        return sys.modules[name]
    except KeyError:
        pass

    # If any of the following calls raises an exception,
    # there's a problem we can't handle -- let the caller handle it.

    fp, pathname, description = imp.find_module(name)
    
    try:
        return imp.load_module(name, fp, pathname, description)
    finally:
        # Since we may exit via an exception, close fp explicitly.
        if fp:
            fp.close()
\end{verbatim}

A more complete example that implements hierarchical module names and
includes a \function{reload()}\bifuncindex{reload} function can be
found in the module \module{knee}\refmodindex{knee}.  The
\module{knee} module can be found in \file{Demo/imputil/} in the
Python source distribution.

\section{Built-in Module \sectcode{__builtin__}}
\bimodindex{__builtin__}

This module provides direct access to all `built-in' identifier of
Python; e.g. \code{__builtin__.open} is the full name for the built-in
function \code{open}.
		% really __builtin__
\section{Built-in Module \module{__main__}}
\label{module-main}
\bimodindex{__main__}
This module represents the (otherwise anonymous) scope in which the
interpreter's main program executes --- commands read either from
standard input or from a script file.
			% really __main__

\chapter{String Services}
\label{strings}

The modules described in this chapter provide a wide range of string
manipulation operations.  Here's an overview:

\begin{description}

\item[string]
--- Common string operations.

\item[re]
--- New Perl-style regular expression search and match operations.

\item[regex]
--- Regular expression search and match operations.

\item[regsub]
--- Substitution and splitting operations that use regular expressions.

\item[struct]
--- Interpret strings as packed binary data.

\item[StringIO]
--- Read and write strings as if they were files.

\end{description}
		% String Services
\section{Standard Module \sectcode{string}}
\label{module-string}
\stmodindex{string}

This module defines some constants useful for checking character
classes and some useful string functions.  See the module
\module{re}\refstmodindex{re} for string functions based on regular
expressions.

The constants defined in this module are are:

\setindexsubitem{(data in module string)}
\begin{datadesc}{digits}
  The string \code{'0123456789'}.
\end{datadesc}

\begin{datadesc}{hexdigits}
  The string \code{'0123456789abcdefABCDEF'}.
\end{datadesc}

\begin{datadesc}{letters}
  The concatenation of the strings \function{lowercase()} and
  \function{uppercase()} described below.
\end{datadesc}

\begin{datadesc}{lowercase}
  A string containing all the characters that are considered lowercase
  letters.  On most systems this is the string
  \code{'abcdefghijklmnopqrstuvwxyz'}.  Do not change its definition ---
  the effect on the routines \function{upper()} and
  \function{swapcase()} is undefined.
\end{datadesc}

\begin{datadesc}{octdigits}
  The string \code{'01234567'}.
\end{datadesc}

\begin{datadesc}{uppercase}
  A string containing all the characters that are considered uppercase
  letters.  On most systems this is the string
  \code{'ABCDEFGHIJKLMNOPQRSTUVWXYZ'}.  Do not change its definition ---
  the effect on the routines \function{lower()} and
  \function{swapcase()} is undefined.
\end{datadesc}

\begin{datadesc}{whitespace}
  A string containing all characters that are considered whitespace.
  On most systems this includes the characters space, tab, linefeed,
  return, formfeed, and vertical tab.  Do not change its definition ---
  the effect on the routines \function{strip()} and \function{split()}
  is undefined.
\end{datadesc}

The functions defined in this module are:


\begin{funcdesc}{atof}{s}
Convert a string to a floating point number.  The string must have
the standard syntax for a floating point literal in Python, optionally
preceded by a sign (\samp{+} or \samp{-}).  Note that this behaves
identical to the built-in function
\function{float()}\bifuncindex{float} when passed a string.
\end{funcdesc}

\begin{funcdesc}{atoi}{s\optional{, base}}
Convert string \var{s} to an integer in the given \var{base}.  The
string must consist of one or more digits, optionally preceded by a
sign (\samp{+} or \samp{-}).  The \var{base} defaults to 10.  If it is
0, a default base is chosen depending on the leading characters of the
string (after stripping the sign): \samp{0x} or \samp{0X} means 16,
\samp{0} means 8, anything else means 10.  If \var{base} is 16, a
leading \samp{0x} or \samp{0X} is always accepted.  Note that when
invoked without \var{base} or with \var{base} set to 10, this behaves
identical to the built-in function \function{int()} when passed a string.
(Also note: for a more flexible interpretation of numeric literals,
use the built-in function \function{eval()}\bifuncindex{eval}.)
\end{funcdesc}

\begin{funcdesc}{atol}{s\optional{, base}}
Convert string \var{s} to a long integer in the given \var{base}.  The 
string must consist of one or more digits, optionally preceded by a
sign (\samp{+} or \samp{-}).  The \var{base} argument has the same
meaning as for \function{atoi()}.  A trailing \samp{l} or \samp{L} is
not allowed, except if the base is 0.  Note that when invoked without
\var{base} or with \var{base} set to 10, this behaves identical to the
built-in function \function{long()}\bifuncindex{long} when passed a
string.
\end{funcdesc}

\begin{funcdesc}{capitalize}{word}
Capitalize the first character of the argument.
\end{funcdesc}

\begin{funcdesc}{capwords}{s}
Split the argument into words using \function{split()}, capitalize
each word using \function{capitalize()}, and join the capitalized
words using \function{join()}.  Note that this replaces runs of
whitespace characters by a single space, and removes leading and
trailing whitespace.
\end{funcdesc}

\begin{funcdesc}{expandtabs}{s, tabsize}
Expand tabs in a string, i.e.\ replace them by one or more spaces,
depending on the current column and the given tab size.  The column
number is reset to zero after each newline occurring in the string.
This doesn't understand other non-printing characters or escape
sequences.
\end{funcdesc}

\begin{funcdesc}{find}{s, sub\optional{, start\optional{,end}}}
Return the lowest index in \var{s} where the substring \var{sub} is
found such that \var{sub} is wholly contained in
\code{\var{s}[\var{start}:\var{end}]}.  Return \code{-1} on failure.
Defaults for \var{start} and \var{end} and interpretation of negative
values is the same as for slices.
\end{funcdesc}

\begin{funcdesc}{rfind}{s, sub\optional{, start\optional{, end}}}
Like \function{find()} but find the highest index.
\end{funcdesc}

\begin{funcdesc}{index}{s, sub\optional{, start\optional{, end}}}
Like \function{find()} but raise \exception{ValueError} when the
substring is not found.
\end{funcdesc}

\begin{funcdesc}{rindex}{s, sub\optional{, start\optional{, end}}}
Like \function{rfind()} but raise \exception{ValueError} when the
substring is not found.
\end{funcdesc}

\begin{funcdesc}{count}{s, sub\optional{, start\optional{, end}}}
Return the number of (non-overlapping) occurrences of substring
\var{sub} in string \code{\var{s}[\var{start}:\var{end}]}.
Defaults for \var{start} and \var{end} and interpretation of negative
values is the same as for slices.
\end{funcdesc}

\begin{funcdesc}{lower}{s}
Convert letters to lower case.
\end{funcdesc}

\begin{funcdesc}{maketrans}{from, to}
Return a translation table suitable for passing to
\function{translate()} or \function{regex.compile()}, that will map
each character in \var{from} into the character at the same position
in \var{to}; \var{from} and \var{to} must have the same length. 
\end{funcdesc}

\begin{funcdesc}{split}{s\optional{, sep\optional{, maxsplit}}}
Return a list of the words of the string \var{s}.  If the optional
second argument \var{sep} is absent or \code{None}, the words are
separated by arbitrary strings of whitespace characters (space, tab,
newline, return, formfeed).  If the second argument \var{sep} is
present and not \code{None}, it specifies a string to be used as the
word separator.  The returned list will then have one more items than
the number of non-overlapping occurrences of the separator in the
string.  The optional third argument \var{maxsplit} defaults to 0.  If
it is nonzero, at most \var{maxsplit} number of splits occur, and the
remainder of the string is returned as the final element of the list
(thus, the list will have at most \code{\var{maxsplit}+1} elements).
\end{funcdesc}

\begin{funcdesc}{splitfields}{s\optional{, sep\optional{, maxsplit}}}
This function behaves identically to \function{split()}.  (In the
past, \function{split()} was only used with one argument, while
\function{splitfields()} was only used with two arguments.)
\end{funcdesc}

\begin{funcdesc}{join}{words\optional{, sep}}
Concatenate a list or tuple of words with intervening occurrences of
\var{sep}.  The default value for \var{sep} is a single space
character.  It is always true that
\samp{string.join(string.split(\var{s}, \var{sep}), \var{sep})}
equals \var{s}.
\end{funcdesc}

\begin{funcdesc}{joinfields}{words\optional{, sep}}
This function behaves identical to \function{join()}.  (In the past,
\function{join()} was only used with one argument, while
\function{joinfields()} was only used with two arguments.)
\end{funcdesc}

\begin{funcdesc}{lstrip}{s}
Remove leading whitespace from the string \var{s}.
\end{funcdesc}

\begin{funcdesc}{rstrip}{s}
Remove trailing whitespace from the string \var{s}.
\end{funcdesc}

\begin{funcdesc}{strip}{s}
Remove leading and trailing whitespace from the string \var{s}.
\end{funcdesc}

\begin{funcdesc}{swapcase}{s}
Convert lower case letters to upper case and vice versa.
\end{funcdesc}

\begin{funcdesc}{translate}{s, table\optional{, deletechars}}
Delete all characters from \var{s} that are in \var{deletechars} (if
present), and then translate the characters using \var{table}, which
must be a 256-character string giving the translation for each
character value, indexed by its ordinal.  
\end{funcdesc}

\begin{funcdesc}{upper}{s}
Convert letters to upper case.
\end{funcdesc}

\begin{funcdesc}{ljust}{s, width}
\funcline{rjust}{s, width}
\funcline{center}{s, width}
These functions respectively left-justify, right-justify and center a
string in a field of given width.
They return a string that is at least
\var{width}
characters wide, created by padding the string
\var{s}
with spaces until the given width on the right, left or both sides.
The string is never truncated.
\end{funcdesc}

\begin{funcdesc}{zfill}{s, width}
Pad a numeric string on the left with zero digits until the given
width is reached.  Strings starting with a sign are handled correctly.
\end{funcdesc}

\begin{funcdesc}{replace}{str, old, new\optional{, maxsplit}}
Return a copy of string \var{str} with all occurrences of substring
\var{old} replaced by \var{new}.  If the optional argument
\var{maxsplit} is given, the first \var{maxsplit} occurrences are
replaced.
\end{funcdesc}

This module is implemented in Python.  Much of its functionality has
been reimplemented in the built-in module
\module{strop}\refbimodindex{strop}.  However, you
should \emph{never} import the latter module directly.  When
\module{string} discovers that \module{strop} exists, it transparently
replaces parts of itself with the implementation from \module{strop}.
After initialization, there is \emph{no} overhead in using
\module{string} instead of \module{strop}.

\section{Built-in Module \sectcode{regex}}

\bimodindex{regex}
This module provides regular expression matching operations similar to
those found in Emacs.  It is always available.

By default the patterns are Emacs-style regular expressions; there is
a way to change the syntax to match that of several well-known
\UNIX{} utilities.

This module is 8-bit clean: both patterns and strings may contain null
bytes and characters whose high bit is set.

\strong{Please note:} There is a little-known fact about Python string
literals which means that you don't usually have to worry about
doubling backslashes, even though they are used to escape special
characters in string literals as well as in regular expressions.  This
is because Python doesn't remove backslashes from string literals if
they are followed by an unrecognized escape character.
\emph{However}, if you want to include a literal \dfn{backslash} in a
regular expression represented as a string literal, you have to
\emph{quadruple} it.  E.g.  to extract LaTeX \samp{\e section\{{\rm
\ldots}\}} headers from a document, you can use this pattern:
\code{'\e \e \e\e section\{\e (.*\e )\}'}.

The module defines these functions, and an exception:

\renewcommand{\indexsubitem}{(in module regex)}

\begin{funcdesc}{match}{pattern\, string}
  Return how many characters at the beginning of \var{string} match
  the regular expression \var{pattern}.  Return \code{-1} if the
  string does not match the pattern (this is different from a
  zero-length match!).
\end{funcdesc}

\begin{funcdesc}{search}{pattern\, string}
  Return the first position in \var{string} that matches the regular
  expression \var{pattern}.  Return -1 if no position in the string
  matches the pattern (this is different from a zero-length match
  anywhere!).
\end{funcdesc}

\begin{funcdesc}{compile}{pattern\optional{\, translate}}
  Compile a regular expression pattern into a regular expression
  object, which can be used for matching using its \code{match} and
  \code{search} methods, described below.  The optional
  \var{translate}, if present, must be a 256-character string
  indicating how characters (both of the pattern and of the strings to
  be matched) are translated before comparing them; the \code{i}-th
  element of the string gives the translation for the character with
  ASCII code \code{i}.

  The sequence

\bcode\begin{verbatim}
prog = regex.compile(pat)
result = prog.match(str)
\end{verbatim}\ecode

is equivalent to

\bcode\begin{verbatim}
result = regex.match(pat, str)
\end{verbatim}\ecode

but the version using \code{compile()} is more efficient when multiple
regular expressions are used concurrently in a single program.  (The
compiled version of the last pattern passed to \code{regex.match()} or
\code{regex.search()} is cached, so programs that use only a single
regular expression at a time needn't worry about compiling regular
expressions.)
\end{funcdesc}

\begin{funcdesc}{set_syntax}{flags}
  Set the syntax to be used by future calls to \code{compile},
  \code{match} and \code{search}.  (Already compiled expression objects
  are not affected.)  The argument is an integer which is the OR of
  several flag bits.  The return value is the previous value of
  the syntax flags.  Names for the flags are defined in the standard
  module \code{regex_syntax}; read the file \file{regex_syntax.py} for
  more information.
\end{funcdesc}

\begin{funcdesc}{symcomp}{pattern\optional{\, translate}}
This is like \code{compile}, but supports symbolic group names: if a
parentheses-enclosed group begins with a group name in angular
brackets, e.g. \code{'\e(<id>[a-z][a-z0-9]*\e)'}, the group can
be referenced by its name in arguments to the \code{group} method of
the resulting compiled regular expression object, like this:
\code{p.group('id')}.
\end{funcdesc}

\begin{excdesc}{error}
  Exception raised when a string passed to one of the functions here
  is not a valid regular expression (e.g., unmatched parentheses) or
  when some other error occurs during compilation or matching.  (It is
  never an error if a string contains no match for a pattern.)
\end{excdesc}

\begin{datadesc}{casefold}
A string suitable to pass as \var{translate} argument to
\code{compile} to map all upper case characters to their lowercase
equivalents.
\end{datadesc}

\noindent
Compiled regular expression objects support these methods:

\renewcommand{\indexsubitem}{(regex method)}
\begin{funcdesc}{match}{string\optional{\, pos}}
  Return how many characters at the beginning of \var{string} match
  the compiled regular expression.  Return \code{-1} if the string
  does not match the pattern (this is different from a zero-length
  match!).
  
  The optional second parameter \var{pos} gives an index in the string
  where the search is to start; it defaults to \code{0}.  This is not
  completely equivalent to slicing the string; the \code{'\^'} pattern
  character matches at the real begin of the string and at positions
  just after a newline, not necessarily at the index where the search
  is to start.
\end{funcdesc}

\begin{funcdesc}{search}{string\optional{\, pos}}
  Return the first position in \var{string} that matches the regular
  expression \code{pattern}.  Return \code{-1} if no position in the
  string matches the pattern (this is different from a zero-length
  match anywhere!).
  
  The optional second parameter has the same meaning as for the
  \code{match} method.
\end{funcdesc}

\begin{funcdesc}{group}{index\, index\, ...}
This method is only valid when the last call to the \code{match}
or \code{search} method found a match.  It returns one or more
groups of the match.  If there is a single \var{index} argument,
the result is a single string; if there are multiple arguments, the
result is a tuple with one item per argument.  If the \var{index} is
zero, the corresponding return value is the entire matching string; if
it is in the inclusive range [1..99], it is the string matching the
the corresponding parenthesized group (using the default syntax,
groups are parenthesized using \code{\\(} and \code{\\)}).  If no
such group exists, the corresponding result is \code{None}.

If the regular expression was compiled by \code{symcomp} instead of
\code{compile}, the \var{index} arguments may also be strings
identifying groups by their group name.
\end{funcdesc}

\noindent
Compiled regular expressions support these data attributes:

\renewcommand{\indexsubitem}{(regex attribute)}

\begin{datadesc}{regs}
When the last call to the \code{match} or \code{search} method found a
match, this is a tuple of pairs of indices corresponding to the
beginning and end of all parenthesized groups in the pattern.  Indices
are relative to the string argument passed to \code{match} or
\code{search}.  The 0-th tuple gives the beginning and end or the
whole pattern.  When the last match or search failed, this is
\code{None}.
\end{datadesc}

\begin{datadesc}{last}
When the last call to the \code{match} or \code{search} method found a
match, this is the string argument passed to that method.  When the
last match or search failed, this is \code{None}.
\end{datadesc}

\begin{datadesc}{translate}
This is the value of the \var{translate} argument to
\code{regex.compile} that created this regular expression object.  If
the \var{translate} argument was omitted in the \code{regex.compile}
call, this is \code{None}.
\end{datadesc}

\begin{datadesc}{givenpat}
The regular expression pattern as passed to \code{compile} or
\code{symcomp}.
\end{datadesc}

\begin{datadesc}{realpat}
The regular expression after stripping the group names for regular
expressions compiled with \code{symcomp}.  Same as \code{givenpat}
otherwise.
\end{datadesc}

\begin{datadesc}{groupindex}
A dictionary giving the mapping from symbolic group names to numerical
group indices for regular expressions compiled with \code{symcomp}.
\code{None} otherwise.
\end{datadesc}

\section{\module{regsub} ---
         String operations using regular expressions}

\declaremodule{standard}{regsub}
\modulesynopsis{Substitution and splitting operations that use
                regular expressions.  \strong{Obsolete!}}


This module defines a number of functions useful for working with
regular expressions (see built-in module \refmodule{regex}).

Warning: these functions are not thread-safe.

\strong{Obsolescence note:}
This module is obsolete as of Python version 1.5; it is still being
maintained because much existing code still uses it.  All new code in
need of regular expressions should use the new \refmodule{re} module, which
supports the more powerful and regular Perl-style regular expressions.
Existing code should be converted.  The standard library module
\module{reconvert} helps in converting \refmodule{regex} style regular
expressions to \refmodule{re} style regular expressions.  (For more
conversion help, see Andrew Kuchling's\index{Kuchling, Andrew}
``regex-to-re HOWTO'' at
\url{http://www.python.org/doc/howto/regex-to-re/}.)


\begin{funcdesc}{sub}{pat, repl, str}
Replace the first occurrence of pattern \var{pat} in string
\var{str} by replacement \var{repl}.  If the pattern isn't found,
the string is returned unchanged.  The pattern may be a string or an
already compiled pattern.  The replacement may contain references
\samp{\e \var{digit}} to subpatterns and escaped backslashes.
\end{funcdesc}

\begin{funcdesc}{gsub}{pat, repl, str}
Replace all (non-overlapping) occurrences of pattern \var{pat} in
string \var{str} by replacement \var{repl}.  The same rules as for
\code{sub()} apply.  Empty matches for the pattern are replaced only
when not adjacent to a previous match, so e.g.
\code{gsub('', '-', 'abc')} returns \code{'-a-b-c-'}.
\end{funcdesc}

\begin{funcdesc}{split}{str, pat\optional{, maxsplit}}
Split the string \var{str} in fields separated by delimiters matching
the pattern \var{pat}, and return a list containing the fields.  Only
non-empty matches for the pattern are considered, so e.g.
\code{split('a:b', ':*')} returns \code{['a', 'b']} and
\code{split('abc', '')} returns \code{['abc']}.  The \var{maxsplit}
defaults to 0. If it is nonzero, only \var{maxsplit} number of splits
occur, and the remainder of the string is returned as the final
element of the list.
\end{funcdesc}

\begin{funcdesc}{splitx}{str, pat\optional{, maxsplit}}
Split the string \var{str} in fields separated by delimiters matching
the pattern \var{pat}, and return a list containing the fields as well
as the separators.  For example, \code{splitx('a:::b', ':*')} returns
\code{['a', ':::', 'b']}.  Otherwise, this function behaves the same
as \code{split}.
\end{funcdesc}

\begin{funcdesc}{capwords}{s\optional{, pat}}
Capitalize words separated by optional pattern \var{pat}.  The default
pattern uses any characters except letters, digits and underscores as
word delimiters.  Capitalization is done by changing the first
character of each word to upper case.
\end{funcdesc}

\begin{funcdesc}{clear_cache}{}
The regsub module maintains a cache of compiled regular expressions,
keyed on the regular expression string and the syntax of the regex
module at the time the expression was compiled.  This function clears
that cache.
\end{funcdesc}

\section{Built-in Module \module{struct}}
\declaremodule{builtin}{struct}

\modulesynopsis{Interpret strings as packed binary data.}

\indexii{C@\C{}}{structures}

This module performs conversions between Python values and C
structs represented as Python strings.  It uses \dfn{format strings}
(explained below) as compact descriptions of the lay-out of the C
structs and the intended conversion to/from Python values.

The module defines the following exception and functions:


\begin{excdesc}{error}
  Exception raised on various occasions; argument is a string
  describing what is wrong.
\end{excdesc}

\begin{funcdesc}{pack}{fmt, v1, v2, {\rm \ldots}}
  Return a string containing the values
  \code{\var{v1}, \var{v2}, {\rm \ldots}} packed according to the given
  format.  The arguments must match the values required by the format
  exactly.
\end{funcdesc}

\begin{funcdesc}{unpack}{fmt, string}
  Unpack the string (presumably packed by \code{pack(\var{fmt}, {\rm \ldots})})
  according to the given format.  The result is a tuple even if it
  contains exactly one item.  The string must contain exactly the
  amount of data required by the format (i.e.  \code{len(\var{string})} must
  equal \code{calcsize(\var{fmt})}).
\end{funcdesc}

\begin{funcdesc}{calcsize}{fmt}
  Return the size of the struct (and hence of the string)
  corresponding to the given format.
\end{funcdesc}

Format characters have the following meaning; the conversion between C
and Python values should be obvious given their types:

\begin{tableiii}{c|l|l}{samp}{Format}{C Type}{Python}
  \lineiii{x}{pad byte}{no value}
  \lineiii{c}{char}{string of length 1}
  \lineiii{b}{signed char}{integer}
  \lineiii{B}{unsigned char}{integer}
  \lineiii{h}{short}{integer}
  \lineiii{H}{unsigned short}{integer}
  \lineiii{i}{int}{integer}
  \lineiii{I}{unsigned int}{integer}
  \lineiii{l}{long}{integer}
  \lineiii{L}{unsigned long}{integer}
  \lineiii{f}{float}{float}
  \lineiii{d}{double}{float}
  \lineiii{s}{char[]}{string}
\end{tableiii}

A format character may be preceded by an integral repeat count; e.g.\
the format string \code{'4h'} means exactly the same as \code{'hhhh'}.

Whitespace characters between formats are ignored; a count and its
format must not contain whitespace though.

For the \code{'s'} format character, the count is interpreted as the
size of the string, not a repeat count like for the other format
characters; e.g. \code{'10s'} means a single 10-byte string, while
\code{'10c'} means 10 characters.  For packing, the string is
truncated or padded with null bytes as appropriate to make it fit.
For unpacking, the resulting string always has exactly the specified
number of bytes.  As a special case, \code{'0s'} means a single, empty
string (while \code{'0c'} means 0 characters).

For the \code{'I'} and \code{'L'} format characters, the return
value is a Python long integer.

By default, C numbers are represented in the machine's native format
and byte order, and properly aligned by skipping pad bytes if
necessary (according to the rules used by the C compiler).

Alternatively, the first character of the format string can be used to
indicate the byte order, size and alignment of the packed data,
according to the following table:

\begin{tableiii}{c|l|l}{samp}{Character}{Byte order}{Size and alignment}
  \lineiii{@}{native}{native}
  \lineiii{=}{native}{standard}
  \lineiii{<}{little-endian}{standard}
  \lineiii{>}{big-endian}{standard}
  \lineiii{!}{network (= big-endian)}{standard}
\end{tableiii}

If the first character is not one of these, \code{'@'} is assumed.

Native byte order is big-endian or little-endian, depending on the
host system (e.g. Motorola and Sun are big-endian; Intel and DEC are
little-endian).

Native size and alignment are determined using the C compiler's sizeof
expression.  This is always combined with native byte order.

Standard size and alignment are as follows: no alignment is required
for any type (so you have to use pad bytes); short is 2 bytes; int and
long are 4 bytes.  Float and double are 32-bit and 64-bit IEEE floating
point numbers, respectively.

Note the difference between \code{'@'} and \code{'='}: both use native
byte order, but the size and alignment of the latter is standardized.

The form \code{'!'} is available for those poor souls who claim they
can't remember whether network byte order is big-endian or
little-endian.

There is no way to indicate non-native byte order (i.e. force
byte-swapping); use the appropriate choice of \code{'<'} or
\code{'>'}.

Examples (all using native byte order, size and alignment, on a
big-endian machine):

\begin{verbatim}
>>> from struct import *
>>> pack('hhl', 1, 2, 3)
'\000\001\000\002\000\000\000\003'
>>> unpack('hhl', '\000\001\000\002\000\000\000\003')
(1, 2, 3)
>>> calcsize('hhl')
8
>>> 
\end{verbatim}
%
Hint: to align the end of a structure to the alignment requirement of
a particular type, end the format with the code for that type with a
repeat count of zero, e.g.\ the format \code{'llh0l'} specifies two
pad bytes at the end, assuming longs are aligned on 4-byte boundaries.
This only works when native size and alignment are in effect;
standard size and alignment does not enforce any alignment.

\begin{seealso}
\seemodule{array}{packed binary storage of homogeneous data}
\end{seealso}


\chapter{Miscellaneous Services}
\label{misc}

The modules described in this chapter provide miscellaneous services
that are available in all Python versions.  Here's an overview:

\localmoduletable
			% Miscellaneous Services
\section{Built-in Module \sectcode{math}}
\label{module-math}

\bimodindex{math}
\renewcommand{\indexsubitem}{(in module math)}
This module is always available.
It provides access to the mathematical functions defined by the C
standard.
They are:

\begin{funcdesc}{acos}{x}
Return the arc cosine of \var{x}.
\end{funcdesc}

\begin{funcdesc}{asin}{x}
Return the arc sine of \var{x}.
\end{funcdesc}

\begin{funcdesc}{atan}{x}
Return the arc tangent of \var{x}.
\end{funcdesc}

\begin{funcdesc}{atan2}{x, y}
Return \code{atan(x / y)}.
\end{funcdesc}

\begin{funcdesc}{ceil}{x}
Return the ceiling of \var{x}.
\end{funcdesc}

\begin{funcdesc}{cos}{x}
Return the cosine of \var{x}.
\end{funcdesc}

\begin{funcdesc}{cosh}{x}
Return the hyperbolic cosine of \var{x}.
\end{funcdesc}

\begin{funcdesc}{exp}{x}
Return the exponential value $\mbox{e}^x$.
\end{funcdesc}

\begin{funcdesc}{fabs}{x}
Return the absolute value of the real \var{x}.
\end{funcdesc}

\begin{funcdesc}{floor}{x}
Return the floor of \var{x}.
\end{funcdesc}

\begin{funcdesc}{fmod}{x, y}
Return \code{x \% y}.
\end{funcdesc}

\begin{funcdesc}{frexp}{x}
Return the matissa and exponent for \var{x}.  The mantissa is
positive.
\end{funcdesc}

\begin{funcdesc}{hypot}{x, y}
Return the Euclidean distance, \code{sqrt(x*x + y*y)}.
\end{funcdesc}

\begin{funcdesc}{ldexp}{x, i}
Return $x {\times} 2^i$.
\end{funcdesc}

\begin{funcdesc}{modf}{x}
Return the fractional and integer parts of \var{x}.  Both results
carry the sign of \var{x}.
\end{funcdesc}

\begin{funcdesc}{pow}{x, y}
Return $x^y$.
\end{funcdesc}

\begin{funcdesc}{sin}{x}
Return the sine of \var{x}.
\end{funcdesc}

\begin{funcdesc}{sinh}{x}
Return the hyperbolic sine of \var{x}.
\end{funcdesc}

\begin{funcdesc}{sqrt}{x}
Return the square root of \var{x}.
\end{funcdesc}

\begin{funcdesc}{tan}{x}
Return the tangent of \var{x}.
\end{funcdesc}

\begin{funcdesc}{tanh}{x}
Return the hyperbolic tangent of \var{x}.
\end{funcdesc}

Note that \code{frexp} and \code{modf} have a different call/return
pattern than their C equivalents: they take a single argument and
return a pair of values, rather than returning their second return
value through an `output parameter' (there is no such thing in Python).

The module also defines two mathematical constants:

\begin{datadesc}{pi}
The mathematical constant \emph{pi}.
\end{datadesc}

\begin{datadesc}{e}
The mathematical constant \emph{e}.
\end{datadesc}

\begin{seealso}
  \seemodule{cmath}{Complex number versions of many of these functions.}
\end{seealso}

\section{Standard Module \sectcode{rand}}

\stmodindex{rand} This module implements a pseudo-random number
generator with an interface similar to \code{rand()} in C\@.  It defines
the following functions:

\renewcommand{\indexsubitem}{(in module rand)}
\begin{funcdesc}{rand}{}
Returns an integer random number in the range [0 ... 32768).
\end{funcdesc}

\begin{funcdesc}{choice}{s}
Returns a random element from the sequence (string, tuple or list)
\var{s}.
\end{funcdesc}

\begin{funcdesc}{srand}{seed}
Initializes the random number generator with the given integral seed.
When the module is first imported, the random number is initialized with
the current time.
\end{funcdesc}

\section{Standard Module \sectcode{whrandom}}

\stmodindex{whrandom}
This module implements a Wichmann-Hill pseudo-random number generator.
It defines the following functions:

\renewcommand{\indexsubitem}{(in module whrandom)}
\begin{funcdesc}{random}{}
Returns the next random floating point number in the range [0.0 ... 1.0).
\end{funcdesc}

\begin{funcdesc}{seed}{x\, y\, z}
Initializes the random number generator from the integers
\var{x},
\var{y}
and
\var{z}.
When the module is first imported, the random number is initialized
using values derived from the current time.
\end{funcdesc}

\section{\module{array} ---
         Efficient arrays of numeric values}

\declaremodule{builtin}{array}
\modulesynopsis{Efficient arrays of uniformly typed numeric values.}


This module defines an object type which can efficiently represent
an array of basic values: characters, integers, floating point
numbers.  Arrays\index{arrays} are sequence types and behave very much
like lists, except that the type of objects stored in them is
constrained.  The type is specified at object creation time by using a
\dfn{type code}, which is a single character.  The following type
codes are defined:

\begin{tableiv}{c|l|l|c}{code}{Type code}{C Type}{Python Type}{Minimum size in bytes}
  \lineiv{'c'}{char}          {character}        {1}
  \lineiv{'b'}{signed char}   {int}              {1}
  \lineiv{'B'}{unsigned char} {int}              {1}
  \lineiv{'u'}{Py_UNICODE}    {Unicode character}{2}
  \lineiv{'h'}{signed short}  {int}              {2}
  \lineiv{'H'}{unsigned short}{int}              {2}
  \lineiv{'i'}{signed int}    {int}              {2}
  \lineiv{'I'}{unsigned int}  {long}             {2}
  \lineiv{'l'}{signed long}   {int}              {4}
  \lineiv{'L'}{unsigned long} {long}             {4}
  \lineiv{'f'}{float}         {float}            {4}
  \lineiv{'d'}{double}        {float}            {8}
\end{tableiv}

The actual representation of values is determined by the machine
architecture (strictly speaking, by the C implementation).  The actual
size can be accessed through the \member{itemsize} attribute.  The values
stored  for \code{'L'} and \code{'I'} items will be represented as
Python long integers when retrieved, because Python's plain integer
type cannot represent the full range of C's unsigned (long) integers.


The module defines the following type:

\begin{funcdesc}{array}{typecode\optional{, initializer}}
Return a new array whose items are restricted by \var{typecode},
and initialized from the optional \var{initializer} value, which
must be a list, string, or iterable over elements of the
appropriate type.
\versionchanged[Formerly, only lists or strings were accepted]{2.4}
If given a list or string, the initializer is passed to the
new array's \method{fromlist()}, \method{fromstring()}, or
\method{fromunicode()} method (see below) to add initial items to
the array.  Otherwise, the iterable initializer is passed to the
\method{extend()} method.
\end{funcdesc}

\begin{datadesc}{ArrayType}
Obsolete alias for \function{array}.
\end{datadesc}


Array objects support the ordinary sequence operations of
indexing, slicing, concatenation, and multiplication.  When using
slice assignment, the assigned value must be an array object with the
same type code; in all other cases, \exception{TypeError} is raised.
Array objects also implement the buffer interface, and may be used
wherever buffer objects are supported.

The following data items and methods are also supported:

\begin{memberdesc}[array]{typecode}
The typecode character used to create the array.
\end{memberdesc}

\begin{memberdesc}[array]{itemsize}
The length in bytes of one array item in the internal representation.
\end{memberdesc}


\begin{methoddesc}[array]{append}{x}
Append a new item with value \var{x} to the end of the array.
\end{methoddesc}

\begin{methoddesc}[array]{buffer_info}{}
Return a tuple \code{(\var{address}, \var{length})} giving the current
memory address and the length in elements of the buffer used to hold
array's contents.  The size of the memory buffer in bytes can be
computed as \code{\var{array}.buffer_info()[1] *
\var{array}.itemsize}.  This is occasionally useful when working with
low-level (and inherently unsafe) I/O interfaces that require memory
addresses, such as certain \cfunction{ioctl()} operations.  The
returned numbers are valid as long as the array exists and no
length-changing operations are applied to it.

\note{When using array objects from code written in C or
\Cpp{} (the only way to effectively make use of this information), it
makes more sense to use the buffer interface supported by array
objects.  This method is maintained for backward compatibility and
should be avoided in new code.  The buffer interface is documented in
the \citetitle[../api/newTypes.html]{Python/C API Reference Manual}.}
\end{methoddesc}

\begin{methoddesc}[array]{byteswap}{}
``Byteswap'' all items of the array.  This is only supported for
values which are 1, 2, 4, or 8 bytes in size; for other types of
values, \exception{RuntimeError} is raised.  It is useful when reading
data from a file written on a machine with a different byte order.
\end{methoddesc}

\begin{methoddesc}[array]{count}{x}
Return the number of occurrences of \var{x} in the array.
\end{methoddesc}

\begin{methoddesc}[array]{extend}{iterable}
Append items from \var{iterable} to the end of the array.  If
\var{iterable} is another array, it must have \emph{exactly} the same
type code; if not, \exception{TypeError} will be raised.  If
\var{iterable} is not an array, it must be iterable and its
elements must be the right type to be appended to the array.
\versionchanged[Formerly, the argument could only be another array]{2.4}
\end{methoddesc}

\begin{methoddesc}[array]{fromfile}{f, n}
Read \var{n} items (as machine values) from the file object \var{f}
and append them to the end of the array.  If less than \var{n} items
are available, \exception{EOFError} is raised, but the items that were
available are still inserted into the array.  \var{f} must be a real
built-in file object; something else with a \method{read()} method won't
do.
\end{methoddesc}

\begin{methoddesc}[array]{fromlist}{list}
Append items from the list.  This is equivalent to
\samp{for x in \var{list}:\ a.append(x)}
except that if there is a type error, the array is unchanged.
\end{methoddesc}

\begin{methoddesc}[array]{fromstring}{s}
Appends items from the string, interpreting the string as an
array of machine values (as if it had been read from a
file using the \method{fromfile()} method).
\end{methoddesc}

\begin{methoddesc}[array]{fromunicode}{s}
Extends this array with data from the given unicode string.
The array must be a type 'u' array; otherwise a ValueError
is raised.  Use \samp{array.fromstring(ustr.decode(enc))} to
append Unicode data to an array of some other type.
\end{methoddesc}

\begin{methoddesc}[array]{index}{x}
Return the smallest \var{i} such that \var{i} is the index of
the first occurrence of \var{x} in the array.
\end{methoddesc}

\begin{methoddesc}[array]{insert}{i, x}
Insert a new item with value \var{x} in the array before position
\var{i}. Negative values are treated as being relative to the end
of the array.
\end{methoddesc}

\begin{methoddesc}[array]{pop}{\optional{i}}
Removes the item with the index \var{i} from the array and returns
it. The optional argument defaults to \code{-1}, so that by default
the last item is removed and returned.
\end{methoddesc}

\begin{methoddesc}[array]{read}{f, n}
\deprecated {1.5.1}
  {Use the \method{fromfile()} method.}
Read \var{n} items (as machine values) from the file object \var{f}
and append them to the end of the array.  If less than \var{n} items
are available, \exception{EOFError} is raised, but the items that were
available are still inserted into the array.  \var{f} must be a real
built-in file object; something else with a \method{read()} method won't
do.
\end{methoddesc}

\begin{methoddesc}[array]{remove}{x}
Remove the first occurrence of \var{x} from the array.
\end{methoddesc}

\begin{methoddesc}[array]{reverse}{}
Reverse the order of the items in the array.
\end{methoddesc}

\begin{methoddesc}[array]{tofile}{f}
Write all items (as machine values) to the file object \var{f}.
\end{methoddesc}

\begin{methoddesc}[array]{tolist}{}
Convert the array to an ordinary list with the same items.
\end{methoddesc}

\begin{methoddesc}[array]{tostring}{}
Convert the array to an array of machine values and return the
string representation (the same sequence of bytes that would
be written to a file by the \method{tofile()} method.)
\end{methoddesc}

\begin{methoddesc}[array]{tounicode}{}
Convert the array to a unicode string.  The array must be
a type 'u' array; otherwise a ValueError is raised.  Use
array.tostring().decode(enc) to obtain a unicode string
from an array of some other type.
\end{methoddesc}

\begin{methoddesc}[array]{write}{f}
\deprecated {1.5.1}
  {Use the \method{tofile()} method.}
Write all items (as machine values) to the file object \var{f}.
\end{methoddesc}

When an array object is printed or converted to a string, it is
represented as \code{array(\var{typecode}, \var{initializer})}.  The
\var{initializer} is omitted if the array is empty, otherwise it is a
string if the \var{typecode} is \code{'c'}, otherwise it is a list of
numbers.  The string is guaranteed to be able to be converted back to
an array with the same type and value using reverse quotes
(\code{``}), so long as the \function{array()} function has been
imported using \code{from array import array}.  Examples:

\begin{verbatim}
array('l')
array('c', 'hello world')
array('u', u'hello \textbackslash u2641')
array('l', [1, 2, 3, 4, 5])
array('d', [1.0, 2.0, 3.14])
\end{verbatim}


\begin{seealso}
  \seemodule{struct}{Packing and unpacking of heterogeneous binary data.}
  \seemodule{xdrlib}{Packing and unpacking of External Data
                     Representation (XDR) data as used in some remote
                     procedure call systems.}
  \seetitle[http://numpy.sourceforge.net/numdoc/HTML/numdoc.htm]{The
           Numerical Python Manual}{The Numeric Python extension
           (NumPy) defines another array type; see
           \url{http://numpy.sourceforge.net/} for further information
           about Numerical Python.  (A PDF version of the NumPy manual
           is available at
           \url{http://numpy.sourceforge.net/numdoc/numdoc.pdf}).}
\end{seealso}


\chapter{Generic Operating System Services}

The modules described in this chapter provide interfaces to operating
system features that are available on (almost) all operating systems,
such as files and a clock.  The interfaces are generally modelled
after the \UNIX{} or C interfaces but they are available on most other
systems as well.  Here's an overview:

\begin{description}

\item[os]
--- Miscellaneous OS interfaces.

\item[time]
--- Time access and conversions.

\item[getopt]
--- Parser for command line options.

\item[tempfile]
--- Generate temporary file names.

\item[errno]
--- Standard errno system symbols.

\item[glob]
--- \UNIX{} shell style pathname pattern expansion.

\item[fnmatch]
--- \UNIX{} shell style pathname pattern matching.

\item[locale]
--- Internationalization services.

\end{description}
		% Generic Operating System Services
\section{\module{os} ---
         Miscellaneous operating system interfaces}

\declaremodule{standard}{os}
\modulesynopsis{Miscellaneous operating system interfaces.}


This module provides a more portable way of using operating system
dependent functionality than importing a operating system dependent
built-in module like \refmodule{posix} or \module{nt}.

This module searches for an operating system dependent built-in module like
\module{mac} or \refmodule{posix} and exports the same functions and data
as found there.  The design of all Python's built-in operating system dependent
modules is such that as long as the same functionality is available,
it uses the same interface; for example, the function
\code{os.stat(\var{path})} returns stat information about \var{path} in
the same format (which happens to have originated with the
\POSIX{} interface).

Extensions peculiar to a particular operating system are also
available through the \module{os} module, but using them is of course a
threat to portability!

Note that after the first time \module{os} is imported, there is
\emph{no} performance penalty in using functions from \module{os}
instead of directly from the operating system dependent built-in module,
so there should be \emph{no} reason not to use \module{os}!


% Frank Stajano <fstajano@uk.research.att.com> complained that it
% wasn't clear that the entries described in the subsections were all
% available at the module level (most uses of subsections are
% different); I think this is only a problem for the HTML version,
% where the relationship may not be as clear.
%
\ifhtml
The \module{os} module contains many functions and data values.
The items below and in the following sub-sections are all available
directly from the \module{os} module.
\fi


\begin{excdesc}{error}
This exception is raised when a function returns a system-related
error (not for illegal argument types or other incidental errors).
This is also known as the built-in exception \exception{OSError}.  The
accompanying value is a pair containing the numeric error code from
\cdata{errno} and the corresponding string, as would be printed by the
C function \cfunction{perror()}.  See the module
\refmodule{errno}\refbimodindex{errno}, which contains names for the
error codes defined by the underlying operating system.

When exceptions are classes, this exception carries two attributes,
\member{errno} and \member{strerror}.  The first holds the value of
the C \cdata{errno} variable, and the latter holds the corresponding
error message from \cfunction{strerror()}.  For exceptions that
involve a file system path (such as \function{chdir()} or
\function{unlink()}), the exception instance will contain a third
attribute, \member{filename}, which is the file name passed to the
function.
\end{excdesc}

\begin{datadesc}{name}
The name of the operating system dependent module imported.  The
following names have currently been registered: \code{'posix'},
\code{'nt'}, \code{'mac'}, \code{'os2'}, \code{'ce'},
\code{'java'}, \code{'riscos'}.
\end{datadesc}

\begin{datadesc}{path}
The corresponding operating system dependent standard module for pathname
operations, such as \module{posixpath} or \module{macpath}.  Thus,
given the proper imports, \code{os.path.split(\var{file})} is
equivalent to but more portable than
\code{posixpath.split(\var{file})}.  Note that this is also an
importable module: it may be imported directly as
\refmodule{os.path}.
\end{datadesc}



\subsection{Process Parameters \label{os-procinfo}}

These functions and data items provide information and operate on the
current process and user.

\begin{datadesc}{environ}
A mapping object representing the string environment. For example,
\code{environ['HOME']} is the pathname of your home directory (on some
platforms), and is equivalent to \code{getenv("HOME")} in C.

This mapping is captured the first time the \module{os} module is
imported, typically during Python startup as part of processing
\file{site.py}.  Changes to the environment made after this time are
not reflected in \code{os.environ}, except for changes made by modifying
\code{os.environ} directly.

If the platform supports the \function{putenv()} function, this
mapping may be used to modify the environment as well as query the
environment.  \function{putenv()} will be called automatically when
the mapping is modified.
\note{Calling \function{putenv()} directly does not change
\code{os.environ}, so it's better to modify \code{os.environ}.}
\note{On some platforms, including FreeBSD and Mac OS X, setting
\code{environ} may cause memory leaks.  Refer to the system documentation
for \cfunction{putenv()}.}

If \function{putenv()} is not provided, this mapping may be passed to
the appropriate process-creation functions to cause child processes to
use a modified environment.
\end{datadesc}

\begin{funcdescni}{chdir}{path}
\funclineni{fchdir}{fd}
\funclineni{getcwd}{}
These functions are described in ``Files and Directories'' (section
\ref{os-file-dir}).
\end{funcdescni}

\begin{funcdesc}{ctermid}{}
Return the filename corresponding to the controlling terminal of the
process.
Availability: \UNIX.
\end{funcdesc}

\begin{funcdesc}{getegid}{}
Return the effective group id of the current process.  This
corresponds to the `set id' bit on the file being executed in the
current process.
Availability: \UNIX.
\end{funcdesc}

\begin{funcdesc}{geteuid}{}
\index{user!effective id}
Return the current process' effective user id.
Availability: \UNIX.
\end{funcdesc}

\begin{funcdesc}{getgid}{}
\index{process!group}
Return the real group id of the current process.
Availability: \UNIX.
\end{funcdesc}

\begin{funcdesc}{getgroups}{}
Return list of supplemental group ids associated with the current
process.
Availability: \UNIX.
\end{funcdesc}

\begin{funcdesc}{getlogin}{}
Return the name of the user logged in on the controlling terminal of
the process.  For most purposes, it is more useful to use the
environment variable \envvar{LOGNAME} to find out who the user is,
or \code{pwd.getpwuid(os.getuid())[0]} to get the login name
of the currently effective user ID.
Availability: \UNIX.
\end{funcdesc}

\begin{funcdesc}{getpgid}{pid}
Return the process group id of the process with process id \var{pid}.
If \var{pid} is 0, the process group id of the current process is
returned. Availability: \UNIX.
\versionadded{2.3}
\end{funcdesc}

\begin{funcdesc}{getpgrp}{}
\index{process!group}
Return the id of the current process group.
Availability: \UNIX.
\end{funcdesc}

\begin{funcdesc}{getpid}{}
\index{process!id}
Return the current process id.
Availability: \UNIX, Windows.
\end{funcdesc}

\begin{funcdesc}{getppid}{}
\index{process!id of parent}
Return the parent's process id.
Availability: \UNIX.
\end{funcdesc}

\begin{funcdesc}{getuid}{}
\index{user!id}
Return the current process' user id.
Availability: \UNIX.
\end{funcdesc}

\begin{funcdesc}{getenv}{varname\optional{, value}}
Return the value of the environment variable \var{varname} if it
exists, or \var{value} if it doesn't.  \var{value} defaults to
\code{None}.
Availability: most flavors of \UNIX, Windows.
\end{funcdesc}

\begin{funcdesc}{putenv}{varname, value}
\index{environment variables!setting}
Set the environment variable named \var{varname} to the string
\var{value}.  Such changes to the environment affect subprocesses
started with \function{os.system()}, \function{popen()} or
\function{fork()} and \function{execv()}.
Availability: most flavors of \UNIX, Windows.

\note{On some platforms, including FreeBSD and Mac OS X,
setting \code{environ} may cause memory leaks.
Refer to the system documentation for putenv.}

When \function{putenv()} is
supported, assignments to items in \code{os.environ} are automatically
translated into corresponding calls to \function{putenv()}; however,
calls to \function{putenv()} don't update \code{os.environ}, so it is
actually preferable to assign to items of \code{os.environ}.
\end{funcdesc}

\begin{funcdesc}{setegid}{egid}
Set the current process's effective group id.
Availability: \UNIX.
\end{funcdesc}

\begin{funcdesc}{seteuid}{euid}
Set the current process's effective user id.
Availability: \UNIX.
\end{funcdesc}

\begin{funcdesc}{setgid}{gid}
Set the current process' group id.
Availability: \UNIX.
\end{funcdesc}

\begin{funcdesc}{setgroups}{groups}
Set the list of supplemental group ids associated with the current
process to \var{groups}. \var{groups} must be a sequence, and each
element must be an integer identifying a group. This operation is
typical available only to the superuser.
Availability: \UNIX.
\versionadded{2.2}
\end{funcdesc}

\begin{funcdesc}{setpgrp}{}
Calls the system call \cfunction{setpgrp()} or \cfunction{setpgrp(0,
0)} depending on which version is implemented (if any).  See the
\UNIX{} manual for the semantics.
Availability: \UNIX.
\end{funcdesc}

\begin{funcdesc}{setpgid}{pid, pgrp} Calls the system call
\cfunction{setpgid()} to set the process group id of the process with
id \var{pid} to the process group with id \var{pgrp}.  See the \UNIX{}
manual for the semantics.
Availability: \UNIX.
\end{funcdesc}

\begin{funcdesc}{setreuid}{ruid, euid}
Set the current process's real and effective user ids.
Availability: \UNIX.
\end{funcdesc}

\begin{funcdesc}{setregid}{rgid, egid}
Set the current process's real and effective group ids.
Availability: \UNIX.
\end{funcdesc}

\begin{funcdesc}{getsid}{pid}
Calls the system call \cfunction{getsid()}.  See the \UNIX{} manual
for the semantics.
Availability: \UNIX. \versionadded{2.4}
\end{funcdesc}

\begin{funcdesc}{setsid}{}
Calls the system call \cfunction{setsid()}.  See the \UNIX{} manual
for the semantics.
Availability: \UNIX.
\end{funcdesc}

\begin{funcdesc}{setuid}{uid}
\index{user!id, setting}
Set the current process' user id.
Availability: \UNIX.
\end{funcdesc}

% placed in this section since it relates to errno.... a little weak
\begin{funcdesc}{strerror}{code}
Return the error message corresponding to the error code in
\var{code}.
Availability: \UNIX, Windows.
\end{funcdesc}

\begin{funcdesc}{umask}{mask}
Set the current numeric umask and returns the previous umask.
Availability: \UNIX, Windows.
\end{funcdesc}

\begin{funcdesc}{uname}{}
Return a 5-tuple containing information identifying the current
operating system.  The tuple contains 5 strings:
\code{(\var{sysname}, \var{nodename}, \var{release}, \var{version},
\var{machine})}.  Some systems truncate the nodename to 8
characters or to the leading component; a better way to get the
hostname is \function{socket.gethostname()}
\withsubitem{(in module socket)}{\ttindex{gethostname()}}
or even
\withsubitem{(in module socket)}{\ttindex{gethostbyaddr()}}
\code{socket.gethostbyaddr(socket.gethostname())}.
Availability: recent flavors of \UNIX.
\end{funcdesc}



\subsection{File Object Creation \label{os-newstreams}}

These functions create new file objects.


\begin{funcdesc}{fdopen}{fd\optional{, mode\optional{, bufsize}}}
Return an open file object connected to the file descriptor \var{fd}.
\index{I/O control!buffering}
The \var{mode} and \var{bufsize} arguments have the same meaning as
the corresponding arguments to the built-in \function{open()}
function.
Availability: Macintosh, \UNIX, Windows.

\versionchanged[When specified, the \var{mode} argument must now start
  with one of the letters \character{r}, \character{w}, or \character{a},
  otherwise a \exception{ValueError} is raised]{2.3}
\end{funcdesc}

\begin{funcdesc}{popen}{command\optional{, mode\optional{, bufsize}}}
Open a pipe to or from \var{command}.  The return value is an open
file object connected to the pipe, which can be read or written
depending on whether \var{mode} is \code{'r'} (default) or \code{'w'}.
The \var{bufsize} argument has the same meaning as the corresponding
argument to the built-in \function{open()} function.  The exit status of
the command (encoded in the format specified for \function{wait()}) is
available as the return value of the \method{close()} method of the file
object, except that when the exit status is zero (termination without
errors), \code{None} is returned.
Availability: Macintosh, \UNIX, Windows.

\versionchanged[This function worked unreliably under Windows in
  earlier versions of Python.  This was due to the use of the
  \cfunction{_popen()} function from the libraries provided with
  Windows.  Newer versions of Python do not use the broken
  implementation from the Windows libraries]{2.0}
\end{funcdesc}

\begin{funcdesc}{tmpfile}{}
Return a new file object opened in update mode (\samp{w+b}).  The file
has no directory entries associated with it and will be automatically
deleted once there are no file descriptors for the file.
Availability: Macintosh, \UNIX, Windows.
\end{funcdesc}


For each of these \function{popen()} variants, if \var{bufsize} is
specified, it specifies the buffer size for the I/O pipes.
\var{mode}, if provided, should be the string \code{'b'} or
\code{'t'}; on Windows this is needed to determine whether the file
objects should be opened in binary or text mode.  The default value
for \var{mode} is \code{'t'}.

Also, for each of these variants, on \UNIX, \var{cmd} may be a sequence, in
which case arguments will be passed directly to the program without shell
intervention (as with \function{os.spawnv()}). If \var{cmd} is a string it will
be passed to the shell (as with \function{os.system()}).

These methods do not make it possible to retrieve the return code from
the child processes.  The only way to control the input and output
streams and also retrieve the return codes is to use the
\class{Popen3} and \class{Popen4} classes from the \refmodule{popen2}
module; these are only available on \UNIX.

For a discussion of possible deadlock conditions related to the use
of these functions, see ``\ulink{Flow Control
Issues}{popen2-flow-control.html}''
(section~\ref{popen2-flow-control}).

\begin{funcdesc}{popen2}{cmd\optional{, mode\optional{, bufsize}}}
Executes \var{cmd} as a sub-process.  Returns the file objects
\code{(\var{child_stdin}, \var{child_stdout})}.
Availability: Macintosh, \UNIX, Windows.
\versionadded{2.0}
\end{funcdesc}

\begin{funcdesc}{popen3}{cmd\optional{, mode\optional{, bufsize}}}
Executes \var{cmd} as a sub-process.  Returns the file objects
\code{(\var{child_stdin}, \var{child_stdout}, \var{child_stderr})}.
Availability: Macintosh, \UNIX, Windows.
\versionadded{2.0}
\end{funcdesc}

\begin{funcdesc}{popen4}{cmd\optional{, mode\optional{, bufsize}}}
Executes \var{cmd} as a sub-process.  Returns the file objects
\code{(\var{child_stdin}, \var{child_stdout_and_stderr})}.
Availability: Macintosh, \UNIX, Windows.
\versionadded{2.0}
\end{funcdesc}

(Note that \code{\var{child_stdin}, \var{child_stdout}, and
\var{child_stderr}} are named from the point of view of the child
process, i.e. \var{child_stdin} is the child's standard input.)

This functionality is also available in the \refmodule{popen2} module
using functions of the same names, but the return values of those
functions have a different order.


\subsection{File Descriptor Operations \label{os-fd-ops}}

These functions operate on I/O streams referred to
using file descriptors.


\begin{funcdesc}{close}{fd}
Close file descriptor \var{fd}.
Availability: Macintosh, \UNIX, Windows.

\begin{notice}
This function is intended for low-level I/O and must be applied
to a file descriptor as returned by \function{open()} or
\function{pipe()}.  To close a ``file object'' returned by the
built-in function \function{open()} or by \function{popen()} or
\function{fdopen()}, use its \method{close()} method.
\end{notice}
\end{funcdesc}

\begin{funcdesc}{dup}{fd}
Return a duplicate of file descriptor \var{fd}.
Availability: Macintosh, \UNIX, Windows.
\end{funcdesc}

\begin{funcdesc}{dup2}{fd, fd2}
Duplicate file descriptor \var{fd} to \var{fd2}, closing the latter
first if necessary.
Availability: Macintosh, \UNIX, Windows.
\end{funcdesc}

\begin{funcdesc}{fdatasync}{fd}
Force write of file with filedescriptor \var{fd} to disk.
Does not force update of metadata.
Availability: \UNIX.
\end{funcdesc}

\begin{funcdesc}{fpathconf}{fd, name}
Return system configuration information relevant to an open file.
\var{name} specifies the configuration value to retrieve; it may be a
string which is the name of a defined system value; these names are
specified in a number of standards (\POSIX.1, \UNIX{} 95, \UNIX{} 98, and
others).  Some platforms define additional names as well.  The names
known to the host operating system are given in the
\code{pathconf_names} dictionary.  For configuration variables not
included in that mapping, passing an integer for \var{name} is also
accepted.
Availability: Macintosh, \UNIX.

If \var{name} is a string and is not known, \exception{ValueError} is
raised.  If a specific value for \var{name} is not supported by the
host system, even if it is included in \code{pathconf_names}, an
\exception{OSError} is raised with \constant{errno.EINVAL} for the
error number.
\end{funcdesc}

\begin{funcdesc}{fstat}{fd}
Return status for file descriptor \var{fd}, like \function{stat()}.
Availability: Macintosh, \UNIX, Windows.
\end{funcdesc}

\begin{funcdesc}{fstatvfs}{fd}
Return information about the filesystem containing the file associated
with file descriptor \var{fd}, like \function{statvfs()}.
Availability: \UNIX.
\end{funcdesc}

\begin{funcdesc}{fsync}{fd}
Force write of file with filedescriptor \var{fd} to disk.  On \UNIX,
this calls the native \cfunction{fsync()} function; on Windows, the
MS \cfunction{_commit()} function.

If you're starting with a Python file object \var{f}, first do
\code{\var{f}.flush()}, and then do \code{os.fsync(\var{f}.fileno())},
to ensure that all internal buffers associated with \var{f} are written
to disk.
Availability: Macintosh, \UNIX, and Windows starting in 2.2.3.
\end{funcdesc}

\begin{funcdesc}{ftruncate}{fd, length}
Truncate the file corresponding to file descriptor \var{fd},
so that it is at most \var{length} bytes in size.
Availability: Macintosh, \UNIX.
\end{funcdesc}

\begin{funcdesc}{isatty}{fd}
Return \code{True} if the file descriptor \var{fd} is open and
connected to a tty(-like) device, else \code{False}.
Availability: Macintosh, \UNIX.
\end{funcdesc}

\begin{funcdesc}{lseek}{fd, pos, how}
Set the current position of file descriptor \var{fd} to position
\var{pos}, modified by \var{how}: \code{0} to set the position
relative to the beginning of the file; \code{1} to set it relative to
the current position; \code{2} to set it relative to the end of the
file.
Availability: Macintosh, \UNIX, Windows.
\end{funcdesc}

\begin{funcdesc}{open}{file, flags\optional{, mode}}
Open the file \var{file} and set various flags according to
\var{flags} and possibly its mode according to \var{mode}.
The default \var{mode} is \code{0777} (octal), and the current umask
value is first masked out.  Return the file descriptor for the newly
opened file.
Availability: Macintosh, \UNIX, Windows.

For a description of the flag and mode values, see the C run-time
documentation; flag constants (like \constant{O_RDONLY} and
\constant{O_WRONLY}) are defined in this module too (see below).

\begin{notice}
This function is intended for low-level I/O.  For normal usage,
use the built-in function \function{open()}, which returns a ``file
object'' with \method{read()} and \method{write()} methods (and many
more).
\end{notice}
\end{funcdesc}

\begin{funcdesc}{openpty}{}
Open a new pseudo-terminal pair. Return a pair of file descriptors
\code{(\var{master}, \var{slave})} for the pty and the tty,
respectively. For a (slightly) more portable approach, use the
\refmodule{pty}\refstmodindex{pty} module.
Availability: Macintosh, Some flavors of \UNIX.
\end{funcdesc}

\begin{funcdesc}{pipe}{}
Create a pipe.  Return a pair of file descriptors \code{(\var{r},
\var{w})} usable for reading and writing, respectively.
Availability: Macintosh, \UNIX, Windows.
\end{funcdesc}

\begin{funcdesc}{read}{fd, n}
Read at most \var{n} bytes from file descriptor \var{fd}.
Return a string containing the bytes read.  If the end of the file
referred to by \var{fd} has been reached, an empty string is
returned.
Availability: Macintosh, \UNIX, Windows.

\begin{notice}
This function is intended for low-level I/O and must be applied
to a file descriptor as returned by \function{open()} or
\function{pipe()}.  To read a ``file object'' returned by the
built-in function \function{open()} or by \function{popen()} or
\function{fdopen()}, or \code{sys.stdin}, use its
\method{read()} or \method{readline()} methods.
\end{notice}
\end{funcdesc}

\begin{funcdesc}{tcgetpgrp}{fd}
Return the process group associated with the terminal given by
\var{fd} (an open file descriptor as returned by \function{open()}).
Availability: Macintosh, \UNIX.
\end{funcdesc}

\begin{funcdesc}{tcsetpgrp}{fd, pg}
Set the process group associated with the terminal given by
\var{fd} (an open file descriptor as returned by \function{open()})
to \var{pg}.
Availability: Macintosh, \UNIX.
\end{funcdesc}

\begin{funcdesc}{ttyname}{fd}
Return a string which specifies the terminal device associated with
file-descriptor \var{fd}.  If \var{fd} is not associated with a terminal
device, an exception is raised.
Availability:Macintosh,  \UNIX.
\end{funcdesc}

\begin{funcdesc}{write}{fd, str}
Write the string \var{str} to file descriptor \var{fd}.
Return the number of bytes actually written.
Availability: Macintosh, \UNIX, Windows.

\begin{notice}
This function is intended for low-level I/O and must be applied
to a file descriptor as returned by \function{open()} or
\function{pipe()}.  To write a ``file object'' returned by the
built-in function \function{open()} or by \function{popen()} or
\function{fdopen()}, or \code{sys.stdout} or \code{sys.stderr}, use
its \method{write()} method.
\end{notice}
\end{funcdesc}


The following data items are available for use in constructing the
\var{flags} parameter to the \function{open()} function.

\begin{datadesc}{O_RDONLY}
\dataline{O_WRONLY}
\dataline{O_RDWR}
\dataline{O_APPEND}
\dataline{O_CREAT}
\dataline{O_EXCL}
\dataline{O_TRUNC}
Options for the \var{flag} argument to the \function{open()} function.
These can be bit-wise OR'd together.
Availability: Macintosh, \UNIX, Windows.
\end{datadesc}

\begin{datadesc}{O_DSYNC}
\dataline{O_RSYNC}
\dataline{O_SYNC}
\dataline{O_NDELAY}
\dataline{O_NONBLOCK}
\dataline{O_NOCTTY}
More options for the \var{flag} argument to the \function{open()} function.
Availability: Macintosh, \UNIX.
\end{datadesc}

\begin{datadesc}{O_BINARY}
Option for the \var{flag} argument to the \function{open()} function.
This can be bit-wise OR'd together with those listed above.
Availability: Windows.
% XXX need to check on the availability of this one.
\end{datadesc}

\begin{datadesc}{O_NOINHERIT}
\dataline{O_SHORT_LIVED}
\dataline{O_TEMPORARY}
\dataline{O_RANDOM}
\dataline{O_SEQUENTIAL}
\dataline{O_TEXT}
Options for the \var{flag} argument to the \function{open()} function.
These can be bit-wise OR'd together.
Availability: Windows.
\end{datadesc}

\subsection{Files and Directories \label{os-file-dir}}

\begin{funcdesc}{access}{path, mode}
Use the real uid/gid to test for access to \var{path}.  Note that most
operations will use the effective uid/gid, therefore this routine can
be used in a suid/sgid environment to test if the invoking user has the
specified access to \var{path}.  \var{mode} should be \constant{F_OK}
to test the existence of \var{path}, or it can be the inclusive OR of
one or more of \constant{R_OK}, \constant{W_OK}, and \constant{X_OK} to
test permissions.  Return \constant{True} if access is allowed,
\constant{False} if not.
See the \UNIX{} man page \manpage{access}{2} for more information.
Availability: Macintosh, \UNIX, Windows.
\end{funcdesc}

\begin{datadesc}{F_OK}
  Value to pass as the \var{mode} parameter of \function{access()} to
  test the existence of \var{path}.
\end{datadesc}

\begin{datadesc}{R_OK}
  Value to include in the \var{mode} parameter of \function{access()}
  to test the readability of \var{path}.
\end{datadesc}

\begin{datadesc}{W_OK}
  Value to include in the \var{mode} parameter of \function{access()}
  to test the writability of \var{path}.
\end{datadesc}

\begin{datadesc}{X_OK}
  Value to include in the \var{mode} parameter of \function{access()}
  to determine if \var{path} can be executed.
\end{datadesc}

\begin{funcdesc}{chdir}{path}
\index{directory!changing}
Change the current working directory to \var{path}.
Availability: Macintosh, \UNIX, Windows.
\end{funcdesc}

\begin{funcdesc}{fchdir}{fd}
Change the current working directory to the directory represented by
the file descriptor \var{fd}.  The descriptor must refer to an opened
directory, not an open file.
Availability: \UNIX.
\versionadded{2.3}
\end{funcdesc}

\begin{funcdesc}{getcwd}{}
Return a string representing the current working directory.
Availability: Macintosh, \UNIX, Windows.
\end{funcdesc}

\begin{funcdesc}{getcwdu}{}
Return a Unicode object representing the current working directory.
Availability: Macintosh, \UNIX, Windows.
\versionadded{2.3}
\end{funcdesc}

\begin{funcdesc}{chroot}{path}
Change the root directory of the current process to \var{path}.
Availability: Macintosh, \UNIX.
\versionadded{2.2}
\end{funcdesc}

\begin{funcdesc}{chmod}{path, mode}
Change the mode of \var{path} to the numeric \var{mode}.
\var{mode} may take one of the following values
(as defined in the \module{stat} module):
\begin{itemize}
  \item \code{S_ISUID}
  \item \code{S_ISGID}
  \item \code{S_ENFMT}
  \item \code{S_ISVTX}
  \item \code{S_IREAD}
  \item \code{S_IWRITE}
  \item \code{S_IEXEC}
  \item \code{S_IRWXU}
  \item \code{S_IRUSR}
  \item \code{S_IWUSR}
  \item \code{S_IXUSR}
  \item \code{S_IRWXG}
  \item \code{S_IRGRP}
  \item \code{S_IWGRP}
  \item \code{S_IXGRP}
  \item \code{S_IRWXO}
  \item \code{S_IROTH}
  \item \code{S_IWOTH}
  \item \code{S_IXOTH}
\end{itemize}
Availability: Macintosh, \UNIX, Windows.
\end{funcdesc}

\begin{funcdesc}{chown}{path, uid, gid}
Change the owner and group id of \var{path} to the numeric \var{uid}
and \var{gid}.
Availability: Macintosh, \UNIX.
\end{funcdesc}

\begin{funcdesc}{lchown}{path, uid, gid}
Change the owner and group id of \var{path} to the numeric \var{uid}
and gid. This function will not follow symbolic links.
Availability: Macintosh, \UNIX.
\versionadded{2.3}
\end{funcdesc}

\begin{funcdesc}{link}{src, dst}
Create a hard link pointing to \var{src} named \var{dst}.
Availability: Macintosh, \UNIX.
\end{funcdesc}

\begin{funcdesc}{listdir}{path}
Return a list containing the names of the entries in the directory.
The list is in arbitrary order.  It does not include the special
entries \code{'.'} and \code{'..'} even if they are present in the
directory.
Availability: Macintosh, \UNIX, Windows.

\versionchanged[On Windows NT/2k/XP and Unix, if \var{path} is a Unicode
object, the result will be a list of Unicode objects.]{2.3}
\end{funcdesc}

\begin{funcdesc}{lstat}{path}
Like \function{stat()}, but do not follow symbolic links.
Availability: Macintosh, \UNIX.
\end{funcdesc}

\begin{funcdesc}{mkfifo}{path\optional{, mode}}
Create a FIFO (a named pipe) named \var{path} with numeric mode
\var{mode}.  The default \var{mode} is \code{0666} (octal).  The current
umask value is first masked out from the mode.
Availability: Macintosh, \UNIX.

FIFOs are pipes that can be accessed like regular files.  FIFOs exist
until they are deleted (for example with \function{os.unlink()}).
Generally, FIFOs are used as rendezvous between ``client'' and
``server'' type processes: the server opens the FIFO for reading, and
the client opens it for writing.  Note that \function{mkfifo()}
doesn't open the FIFO --- it just creates the rendezvous point.
\end{funcdesc}

\begin{funcdesc}{mknod}{path\optional{, mode=0600, device}}
Create a filesystem node (file, device special file or named pipe)
named filename. \var{mode} specifies both the permissions to use and
the type of node to be created, being combined (bitwise OR) with one
of S_IFREG, S_IFCHR, S_IFBLK, and S_IFIFO (those constants are
available in \module{stat}). For S_IFCHR and S_IFBLK, \var{device}
defines the newly created device special file (probably using
\function{os.makedev()}), otherwise it is ignored.
\versionadded{2.3}
\end{funcdesc}

\begin{funcdesc}{major}{device}
Extracts a device major number from a raw device number.
\versionadded{2.3}
\end{funcdesc}

\begin{funcdesc}{minor}{device}
Extracts a device minor number from a raw device number.
\versionadded{2.3}
\end{funcdesc}

\begin{funcdesc}{makedev}{major, minor}
Composes a raw device number from the major and minor device numbers.
\versionadded{2.3}
\end{funcdesc}

\begin{funcdesc}{mkdir}{path\optional{, mode}}
Create a directory named \var{path} with numeric mode \var{mode}.
The default \var{mode} is \code{0777} (octal).  On some systems,
\var{mode} is ignored.  Where it is used, the current umask value is
first masked out.
Availability: Macintosh, \UNIX, Windows.
\end{funcdesc}

\begin{funcdesc}{makedirs}{path\optional{, mode}}
Recursive directory creation function.\index{directory!creating}
\index{UNC paths!and \function{os.makedirs()}}
Like \function{mkdir()},
but makes all intermediate-level directories needed to contain the
leaf directory.  Throws an \exception{error} exception if the leaf
directory already exists or cannot be created.  The default \var{mode}
is \code{0777} (octal).  This function does not properly handle UNC
paths (only relevant on Windows systems; Universal Naming Convention
paths are those that use the `\code{\e\e host\e path}' syntax).
\versionadded{1.5.2}
\end{funcdesc}

\begin{funcdesc}{pathconf}{path, name}
Return system configuration information relevant to a named file.
\var{name} specifies the configuration value to retrieve; it may be a
string which is the name of a defined system value; these names are
specified in a number of standards (\POSIX.1, \UNIX{} 95, \UNIX{} 98, and
others).  Some platforms define additional names as well.  The names
known to the host operating system are given in the
\code{pathconf_names} dictionary.  For configuration variables not
included in that mapping, passing an integer for \var{name} is also
accepted.
Availability: Macintosh, \UNIX.

If \var{name} is a string and is not known, \exception{ValueError} is
raised.  If a specific value for \var{name} is not supported by the
host system, even if it is included in \code{pathconf_names}, an
\exception{OSError} is raised with \constant{errno.EINVAL} for the
error number.
\end{funcdesc}

\begin{datadesc}{pathconf_names}
Dictionary mapping names accepted by \function{pathconf()} and
\function{fpathconf()} to the integer values defined for those names
by the host operating system.  This can be used to determine the set
of names known to the system.
Availability: Macintosh, \UNIX.
\end{datadesc}

\begin{funcdesc}{readlink}{path}
Return a string representing the path to which the symbolic link
points.  The result may be either an absolute or relative pathname; if
it is relative, it may be converted to an absolute pathname using
\code{os.path.join(os.path.dirname(\var{path}), \var{result})}.
Availability: Macintosh, \UNIX.
\end{funcdesc}

\begin{funcdesc}{remove}{path}
Remove the file \var{path}.  If \var{path} is a directory,
\exception{OSError} is raised; see \function{rmdir()} below to remove
a directory.  This is identical to the \function{unlink()} function
documented below.  On Windows, attempting to remove a file that is in
use causes an exception to be raised; on \UNIX, the directory entry is
removed but the storage allocated to the file is not made available
until the original file is no longer in use.
Availability: Macintosh, \UNIX, Windows.
\end{funcdesc}

\begin{funcdesc}{removedirs}{path}
\index{directory!deleting}
Removes directories recursively.  Works like
\function{rmdir()} except that, if the leaf directory is
successfully removed, directories corresponding to rightmost path
segments will be pruned way until either the whole path is consumed or
an error is raised (which is ignored, because it generally means that
a parent directory is not empty).  Throws an \exception{error}
exception if the leaf directory could not be successfully removed.
\versionadded{1.5.2}
\end{funcdesc}

\begin{funcdesc}{rename}{src, dst}
Rename the file or directory \var{src} to \var{dst}.  If \var{dst} is
a directory, \exception{OSError} will be raised.  On \UNIX, if
\var{dst} exists and is a file, it will be removed silently if the
user has permission.  The operation may fail on some \UNIX{} flavors
if \var{src} and \var{dst} are on different filesystems.  If
successful, the renaming will be an atomic operation (this is a
\POSIX{} requirement).  On Windows, if \var{dst} already exists,
\exception{OSError} will be raised even if it is a file; there may be
no way to implement an atomic rename when \var{dst} names an existing
file.
Availability: Macintosh, \UNIX, Windows.
\end{funcdesc}

\begin{funcdesc}{renames}{old, new}
Recursive directory or file renaming function.
Works like \function{rename()}, except creation of any intermediate
directories needed to make the new pathname good is attempted first.
After the rename, directories corresponding to rightmost path segments
of the old name will be pruned away using \function{removedirs()}.
\versionadded{1.5.2}

\begin{notice}
This function can fail with the new directory structure made if
you lack permissions needed to remove the leaf directory or file.
\end{notice}
\end{funcdesc}

\begin{funcdesc}{rmdir}{path}
Remove the directory \var{path}.
Availability: Macintosh, \UNIX, Windows.
\end{funcdesc}

\begin{funcdesc}{stat}{path}
Perform a \cfunction{stat()} system call on the given path.  The
return value is an object whose attributes correspond to the members of
the \ctype{stat} structure, namely:
\member{st_mode} (protection bits),
\member{st_ino} (inode number),
\member{st_dev} (device),
\member{st_nlink} (number of hard links),
\member{st_uid} (user ID of owner),
\member{st_gid} (group ID of owner),
\member{st_size} (size of file, in bytes),
\member{st_atime} (time of most recent access),
\member{st_mtime} (time of most recent content modification),
\member{st_ctime}
(platform dependent; time of most recent metadata change on \UNIX, or
the time of creation on Windows).

\versionchanged [If \function{stat_float_times} returns true, the time
values are floats, measuring seconds. Fractions of a second may be
reported if the system supports that. On Mac OS, the times are always
floats. See \function{stat_float_times} for further discussion. ]{2.3}

On some Unix systems (such as Linux), the following attributes may
also be available:
\member{st_blocks} (number of blocks allocated for file),
\member{st_blksize} (filesystem blocksize),
\member{st_rdev} (type of device if an inode device).

On Mac OS systems, the following attributes may also be available:
\member{st_rsize},
\member{st_creator},
\member{st_type}.

On RISCOS systems, the following attributes are also available:
\member{st_ftype} (file type),
\member{st_attrs} (attributes),
\member{st_obtype} (object type).

For backward compatibility, the return value of \function{stat()} is
also accessible as a tuple of at least 10 integers giving the most
important (and portable) members of the \ctype{stat} structure, in the
order
\member{st_mode},
\member{st_ino},
\member{st_dev},
\member{st_nlink},
\member{st_uid},
\member{st_gid},
\member{st_size},
\member{st_atime},
\member{st_mtime},
\member{st_ctime}.
More items may be added at the end by some implementations.
The standard module \refmodule{stat}\refstmodindex{stat} defines
functions and constants that are useful for extracting information
from a \ctype{stat} structure.
(On Windows, some items are filled with dummy values.)

\note{The exact meaning and resolution of the \member{st_atime},
 \member{st_mtime}, and \member{st_ctime} members depends on the
 operating system and the file system.  For example, on Windows systems
 using the FAT or FAT32 file systems, \member{st_mtime} has 2-second
 resolution, and \member{st_atime} has only 1-day resolution.  See
 your operating system documentation for details.}

Availability: Macintosh, \UNIX, Windows.

\versionchanged
[Added access to values as attributes of the returned object]{2.2}
\end{funcdesc}

\begin{funcdesc}{stat_float_times}{\optional{newvalue}}
Determine whether \class{stat_result} represents time stamps as float
objects.  If newval is True, future calls to stat() return floats, if
it is False, future calls return ints.  If newval is omitted, return
the current setting.

For compatibility with older Python versions, accessing
\class{stat_result} as a tuple always returns integers. For
compatibility with Python 2.2, accessing the time stamps by field name
also returns integers. Applications that want to determine the
fractions of a second in a time stamp can use this function to have
time stamps represented as floats. Whether they will actually observe
non-zero fractions depends on the system.

Future Python releases will change the default of this setting;
applications that cannot deal with floating point time stamps can then
use this function to turn the feature off.

It is recommended that this setting is only changed at program startup
time in the \var{__main__} module; libraries should never change this
setting. If an application uses a library that works incorrectly if
floating point time stamps are processed, this application should turn
the feature off until the library has been corrected.

\end{funcdesc}

\begin{funcdesc}{statvfs}{path}
Perform a \cfunction{statvfs()} system call on the given path.  The
return value is an object whose attributes describe the filesystem on
the given path, and correspond to the members of the
\ctype{statvfs} structure, namely:
\member{f_frsize},
\member{f_blocks},
\member{f_bfree},
\member{f_bavail},
\member{f_files},
\member{f_ffree},
\member{f_favail},
\member{f_flag},
\member{f_namemax}.
Availability: \UNIX.

For backward compatibility, the return value is also accessible as a
tuple whose values correspond to the attributes, in the order given above.
The standard module \refmodule{statvfs}\refstmodindex{statvfs}
defines constants that are useful for extracting information
from a \ctype{statvfs} structure when accessing it as a sequence; this
remains useful when writing code that needs to work with versions of
Python that don't support accessing the fields as attributes.

\versionchanged
[Added access to values as attributes of the returned object]{2.2}
\end{funcdesc}

\begin{funcdesc}{symlink}{src, dst}
Create a symbolic link pointing to \var{src} named \var{dst}.
Availability: \UNIX.
\end{funcdesc}

\begin{funcdesc}{tempnam}{\optional{dir\optional{, prefix}}}
Return a unique path name that is reasonable for creating a temporary
file.  This will be an absolute path that names a potential directory
entry in the directory \var{dir} or a common location for temporary
files if \var{dir} is omitted or \code{None}.  If given and not
\code{None}, \var{prefix} is used to provide a short prefix to the
filename.  Applications are responsible for properly creating and
managing files created using paths returned by \function{tempnam()};
no automatic cleanup is provided.
On \UNIX, the environment variable \envvar{TMPDIR} overrides
\var{dir}, while on Windows the \envvar{TMP} is used.  The specific
behavior of this function depends on the C library implementation;
some aspects are underspecified in system documentation.
\warning{Use of \function{tempnam()} is vulnerable to symlink attacks;
consider using \function{tmpfile()} instead.}
Availability: Macintosh, \UNIX, Windows.
\end{funcdesc}

\begin{funcdesc}{tmpnam}{}
Return a unique path name that is reasonable for creating a temporary
file.  This will be an absolute path that names a potential directory
entry in a common location for temporary files.  Applications are
responsible for properly creating and managing files created using
paths returned by \function{tmpnam()}; no automatic cleanup is
provided.
\warning{Use of \function{tmpnam()} is vulnerable to symlink attacks;
consider using \function{tmpfile()} instead.}
Availability: \UNIX, Windows.  This function probably shouldn't be used
on Windows, though:  Microsoft's implementation of \function{tmpnam()}
always creates a name in the root directory of the current drive, and
that's generally a poor location for a temp file (depending on
privileges, you may not even be able to open a file using this name).
\end{funcdesc}

\begin{datadesc}{TMP_MAX}
The maximum number of unique names that \function{tmpnam()} will
generate before reusing names.
\end{datadesc}

\begin{funcdesc}{unlink}{path}
Remove the file \var{path}.  This is the same function as
\function{remove()}; the \function{unlink()} name is its traditional
\UNIX{} name.
Availability: Macintosh, \UNIX, Windows.
\end{funcdesc}

\begin{funcdesc}{utime}{path, times}
Set the access and modified times of the file specified by \var{path}.
If \var{times} is \code{None}, then the file's access and modified
times are set to the current time.  Otherwise, \var{times} must be a
2-tuple of numbers, of the form \code{(\var{atime}, \var{mtime})}
which is used to set the access and modified times, respectively.
Whether a directory can be given for \var{path} depends on whether the
operating system implements directories as files (for example, Windows
does not).  Note that the exact times you set here may not be returned
by a subsequent \function{stat()} call, depending on the resolution
with which your operating system records access and modification times;
see \function{stat()}.
\versionchanged[Added support for \code{None} for \var{times}]{2.0}
Availability: Macintosh, \UNIX, Windows.
\end{funcdesc}

\begin{funcdesc}{walk}{top\optional{, topdown\code{=True}
                       \optional{, onerror\code{=None}}}}
\index{directory!walking}
\index{directory!traversal}
\function{walk()} generates the file names in a directory tree, by
walking the tree either top down or bottom up.
For each directory in the tree rooted at directory \var{top} (including
\var{top} itself), it yields a 3-tuple
\code{(\var{dirpath}, \var{dirnames}, \var{filenames})}.

\var{dirpath} is a string, the path to the directory.  \var{dirnames} is
a list of the names of the subdirectories in \var{dirpath}
(excluding \code{'.'} and \code{'..'}).  \var{filenames} is a list of
the names of the non-directory files in \var{dirpath}.  Note that the
names in the lists contain no path components.  To get a full
path (which begins with \var{top}) to a file or directory in
\var{dirpath}, do \code{os.path.join(\var{dirpath}, \var{name})}.

If optional argument \var{topdown} is true or not specified, the triple
for a directory is generated before the triples for any of its
subdirectories (directories are generated top down).  If \var{topdown} is
false, the triple for a directory is generated after the triples for all
of its subdirectories (directories are generated bottom up).

When \var{topdown} is true, the caller can modify the \var{dirnames} list
in-place (perhaps using \keyword{del} or slice assignment), and
\function{walk()} will only recurse into the subdirectories whose names
remain in \var{dirnames}; this can be used to prune the search,
impose a specific order of visiting, or even to inform \function{walk()}
about directories the caller creates or renames before it resumes
\function{walk()} again.  Modifying \var{dirnames} when \var{topdown} is
false is ineffective, because in bottom-up mode the directories in
\var{dirnames} are generated before \var{dirnames} itself is generated.

By default errors from the \code{os.listdir()} call are ignored.  If
optional argument \var{onerror} is specified, it should be a function;
it will be called with one argument, an os.error instance.  It can
report the error to continue with the walk, or raise the exception
to abort the walk.  Note that the filename is available as the
\code{filename} attribute of the exception object.

\begin{notice}
If you pass a relative pathname, don't change the current working
directory between resumptions of \function{walk()}.  \function{walk()}
never changes the current directory, and assumes that its caller
doesn't either.
\end{notice}

\begin{notice}
On systems that support symbolic links, links to subdirectories appear
in \var{dirnames} lists, but \function{walk()} will not visit them
(infinite loops are hard to avoid when following symbolic links).
To visit linked directories, you can identify them with
\code{os.path.islink(\var{path})}, and invoke \code{walk(\var{path})}
on each directly.
\end{notice}

This example displays the number of bytes taken by non-directory files
in each directory under the starting directory, except that it doesn't
look under any CVS subdirectory:

\begin{verbatim}
import os
from os.path import join, getsize
for root, dirs, files in os.walk('python/Lib/email'):
    print root, "consumes",
    print sum(getsize(join(root, name)) for name in files),
    print "bytes in", len(files), "non-directory files"
    if 'CVS' in dirs:
        dirs.remove('CVS')  # don't visit CVS directories
\end{verbatim}

In the next example, walking the tree bottom up is essential:
\function{rmdir()} doesn't allow deleting a directory before the
directory is empty:

\begin{verbatim}
# Delete everything reachable from the directory named in 'top',
# assuming there are no symbolic links.
# CAUTION:  This is dangerous!  For example, if top == '/', it
# could delete all your disk files.
import os
for root, dirs, files in os.walk(top, topdown=False):
    for name in files:
        os.remove(os.path.join(root, name))
    for name in dirs:
        os.rmdir(os.path.join(root, name))
\end{verbatim}

\versionadded{2.3}
\end{funcdesc}

\subsection{Process Management \label{os-process}}

These functions may be used to create and manage processes.

The various \function{exec*()} functions take a list of arguments for
the new program loaded into the process.  In each case, the first of
these arguments is passed to the new program as its own name rather
than as an argument a user may have typed on a command line.  For the
C programmer, this is the \code{argv[0]} passed to a program's
\cfunction{main()}.  For example, \samp{os.execv('/bin/echo', ['foo',
'bar'])} will only print \samp{bar} on standard output; \samp{foo}
will seem to be ignored.


\begin{funcdesc}{abort}{}
Generate a \constant{SIGABRT} signal to the current process.  On
\UNIX, the default behavior is to produce a core dump; on Windows, the
process immediately returns an exit code of \code{3}.  Be aware that
programs which use \function{signal.signal()} to register a handler
for \constant{SIGABRT} will behave differently.
Availability: Macintosh, \UNIX, Windows.
\end{funcdesc}

\begin{funcdesc}{execl}{path, arg0, arg1, \moreargs}
\funcline{execle}{path, arg0, arg1, \moreargs, env}
\funcline{execlp}{file, arg0, arg1, \moreargs}
\funcline{execlpe}{file, arg0, arg1, \moreargs, env}
\funcline{execv}{path, args}
\funcline{execve}{path, args, env}
\funcline{execvp}{file, args}
\funcline{execvpe}{file, args, env}
These functions all execute a new program, replacing the current
process; they do not return.  On \UNIX, the new executable is loaded
into the current process, and will have the same process ID as the
caller.  Errors will be reported as \exception{OSError} exceptions.

The \character{l} and \character{v} variants of the
\function{exec*()} functions differ in how command-line arguments are
passed.  The \character{l} variants are perhaps the easiest to work
with if the number of parameters is fixed when the code is written;
the individual parameters simply become additional parameters to the
\function{execl*()} functions.  The \character{v} variants are good
when the number of parameters is variable, with the arguments being
passed in a list or tuple as the \var{args} parameter.  In either
case, the arguments to the child process should start with the name of
the command being run, but this is not enforced.

The variants which include a \character{p} near the end
(\function{execlp()}, \function{execlpe()}, \function{execvp()},
and \function{execvpe()}) will use the \envvar{PATH} environment
variable to locate the program \var{file}.  When the environment is
being replaced (using one of the \function{exec*e()} variants,
discussed in the next paragraph), the
new environment is used as the source of the \envvar{PATH} variable.
The other variants, \function{execl()}, \function{execle()},
\function{execv()}, and \function{execve()}, will not use the
\envvar{PATH} variable to locate the executable; \var{path} must
contain an appropriate absolute or relative path.

For \function{execle()}, \function{execlpe()}, \function{execve()},
and \function{execvpe()} (note that these all end in \character{e}),
the \var{env} parameter must be a mapping which is used to define the
environment variables for the new process; the \function{execl()},
\function{execlp()}, \function{execv()}, and \function{execvp()}
all cause the new process to inherit the environment of the current
process.
Availability: Macintosh, \UNIX, Windows.
\end{funcdesc}

\begin{funcdesc}{_exit}{n}
Exit to the system with status \var{n}, without calling cleanup
handlers, flushing stdio buffers, etc.
Availability: Macintosh, \UNIX, Windows.

\begin{notice}
The standard way to exit is \code{sys.exit(\var{n})}.
\function{_exit()} should normally only be used in the child process
after a \function{fork()}.
\end{notice}
\end{funcdesc}

The following exit codes are a defined, and can be used with
\function{_exit()}, although they are not required.  These are
typically used for system programs written in Python, such as a
mail server's external command delivery program.

\begin{datadesc}{EX_OK}
Exit code that means no error occurred.
Availability: Macintosh, \UNIX.
\versionadded{2.3}
\end{datadesc}

\begin{datadesc}{EX_USAGE}
Exit code that means the command was used incorrectly, such as when
the wrong number of arguments are given.
Availability: Macintosh, \UNIX.
\versionadded{2.3}
\end{datadesc}

\begin{datadesc}{EX_DATAERR}
Exit code that means the input data was incorrect.
Availability: Macintosh, \UNIX.
\versionadded{2.3}
\end{datadesc}

\begin{datadesc}{EX_NOINPUT}
Exit code that means an input file did not exist or was not readable.
Availability: Macintosh, \UNIX.
\versionadded{2.3}
\end{datadesc}

\begin{datadesc}{EX_NOUSER}
Exit code that means a specified user did not exist.
Availability: Macintosh, \UNIX.
\versionadded{2.3}
\end{datadesc}

\begin{datadesc}{EX_NOHOST}
Exit code that means a specified host did not exist.
Availability: Macintosh, \UNIX.
\versionadded{2.3}
\end{datadesc}

\begin{datadesc}{EX_UNAVAILABLE}
Exit code that means that a required service is unavailable.
Availability: Macintosh, \UNIX.
\versionadded{2.3}
\end{datadesc}

\begin{datadesc}{EX_SOFTWARE}
Exit code that means an internal software error was detected.
Availability: Macintosh, \UNIX.
\versionadded{2.3}
\end{datadesc}

\begin{datadesc}{EX_OSERR}
Exit code that means an operating system error was detected, such as
the inability to fork or create a pipe.
Availability: Macintosh, \UNIX.
\versionadded{2.3}
\end{datadesc}

\begin{datadesc}{EX_OSFILE}
Exit code that means some system file did not exist, could not be
opened, or had some other kind of error.
Availability: Macintosh, \UNIX.
\versionadded{2.3}
\end{datadesc}

\begin{datadesc}{EX_CANTCREAT}
Exit code that means a user specified output file could not be created.
Availability: Macintosh, \UNIX.
\versionadded{2.3}
\end{datadesc}

\begin{datadesc}{EX_IOERR}
Exit code that means that an error occurred while doing I/O on some file.
Availability: Macintosh, \UNIX.
\versionadded{2.3}
\end{datadesc}

\begin{datadesc}{EX_TEMPFAIL}
Exit code that means a temporary failure occurred.  This indicates
something that may not really be an error, such as a network
connection that couldn't be made during a retryable operation.
Availability: Macintosh, \UNIX.
\versionadded{2.3}
\end{datadesc}

\begin{datadesc}{EX_PROTOCOL}
Exit code that means that a protocol exchange was illegal, invalid, or
not understood.
Availability: Macintosh, \UNIX.
\versionadded{2.3}
\end{datadesc}

\begin{datadesc}{EX_NOPERM}
Exit code that means that there were insufficient permissions to
perform the operation (but not intended for file system problems).
Availability: Macintosh, \UNIX.
\versionadded{2.3}
\end{datadesc}

\begin{datadesc}{EX_CONFIG}
Exit code that means that some kind of configuration error occurred.
Availability: Macintosh, \UNIX.
\versionadded{2.3}
\end{datadesc}

\begin{datadesc}{EX_NOTFOUND}
Exit code that means something like ``an entry was not found''.
Availability: Macintosh, \UNIX.
\versionadded{2.3}
\end{datadesc}

\begin{funcdesc}{fork}{}
Fork a child process.  Return \code{0} in the child, the child's
process id in the parent.
Availability: Macintosh, \UNIX.
\end{funcdesc}

\begin{funcdesc}{forkpty}{}
Fork a child process, using a new pseudo-terminal as the child's
controlling terminal. Return a pair of \code{(\var{pid}, \var{fd})},
where \var{pid} is \code{0} in the child, the new child's process id
in the parent, and \var{fd} is the file descriptor of the master end
of the pseudo-terminal.  For a more portable approach, use the
\refmodule{pty} module.
Availability: Macintosh, Some flavors of \UNIX.
\end{funcdesc}

\begin{funcdesc}{kill}{pid, sig}
\index{process!killing}
\index{process!signalling}
Kill the process \var{pid} with signal \var{sig}.  Constants for the
specific signals available on the host platform are defined in the
\refmodule{signal} module.
Availability: Macintosh, \UNIX.
\end{funcdesc}

\begin{funcdesc}{killpg}{pgid, sig}
\index{process!killing}
\index{process!signalling}
Kill the process group \var{pgid} with the signal \var{sig}.
Availability: Macintosh, \UNIX.
\versionadded{2.3}
\end{funcdesc}

\begin{funcdesc}{nice}{increment}
Add \var{increment} to the process's ``niceness''.  Return the new
niceness.
Availability: Macintosh, \UNIX.
\end{funcdesc}

\begin{funcdesc}{plock}{op}
Lock program segments into memory.  The value of \var{op}
(defined in \code{<sys/lock.h>}) determines which segments are locked.
Availability: Macintosh, \UNIX.
\end{funcdesc}

\begin{funcdescni}{popen}{\unspecified}
\funclineni{popen2}{\unspecified}
\funclineni{popen3}{\unspecified}
\funclineni{popen4}{\unspecified}
Run child processes, returning opened pipes for communications.  These
functions are described in section \ref{os-newstreams}.
\end{funcdescni}

\begin{funcdesc}{spawnl}{mode, path, \moreargs}
\funcline{spawnle}{mode, path, \moreargs, env}
\funcline{spawnlp}{mode, file, \moreargs}
\funcline{spawnlpe}{mode, file, \moreargs, env}
\funcline{spawnv}{mode, path, args}
\funcline{spawnve}{mode, path, args, env}
\funcline{spawnvp}{mode, file, args}
\funcline{spawnvpe}{mode, file, args, env}
Execute the program \var{path} in a new process.  If \var{mode} is
\constant{P_NOWAIT}, this function returns the process ID of the new
process; if \var{mode} is \constant{P_WAIT}, returns the process's
exit code if it exits normally, or \code{-\var{signal}}, where
\var{signal} is the signal that killed the process.  On Windows, the
process ID will actually be the process handle, so can be used with
the \function{waitpid()} function.

The \character{l} and \character{v} variants of the
\function{spawn*()} functions differ in how command-line arguments are
passed.  The \character{l} variants are perhaps the easiest to work
with if the number of parameters is fixed when the code is written;
the individual parameters simply become additional parameters to the
\function{spawnl*()} functions.  The \character{v} variants are good
when the number of parameters is variable, with the arguments being
passed in a list or tuple as the \var{args} parameter.  In either
case, the arguments to the child process must start with the name of
the command being run.

The variants which include a second \character{p} near the end
(\function{spawnlp()}, \function{spawnlpe()}, \function{spawnvp()},
and \function{spawnvpe()}) will use the \envvar{PATH} environment
variable to locate the program \var{file}.  When the environment is
being replaced (using one of the \function{spawn*e()} variants,
discussed in the next paragraph), the new environment is used as the
source of the \envvar{PATH} variable.  The other variants,
\function{spawnl()}, \function{spawnle()}, \function{spawnv()}, and
\function{spawnve()}, will not use the \envvar{PATH} variable to
locate the executable; \var{path} must contain an appropriate absolute
or relative path.

For \function{spawnle()}, \function{spawnlpe()}, \function{spawnve()},
and \function{spawnvpe()} (note that these all end in \character{e}),
the \var{env} parameter must be a mapping which is used to define the
environment variables for the new process; the \function{spawnl()},
\function{spawnlp()}, \function{spawnv()}, and \function{spawnvp()}
all cause the new process to inherit the environment of the current
process.

As an example, the following calls to \function{spawnlp()} and
\function{spawnvpe()} are equivalent:

\begin{verbatim}
import os
os.spawnlp(os.P_WAIT, 'cp', 'cp', 'index.html', '/dev/null')

L = ['cp', 'index.html', '/dev/null']
os.spawnvpe(os.P_WAIT, 'cp', L, os.environ)
\end{verbatim}

Availability: \UNIX, Windows.  \function{spawnlp()},
\function{spawnlpe()}, \function{spawnvp()} and \function{spawnvpe()}
are not available on Windows.
\versionadded{1.6}
\end{funcdesc}

\begin{datadesc}{P_NOWAIT}
\dataline{P_NOWAITO}
Possible values for the \var{mode} parameter to the \function{spawn*()}
family of functions.  If either of these values is given, the
\function{spawn*()} functions will return as soon as the new process
has been created, with the process ID as the return value.
Availability: Macintosh, \UNIX, Windows.
\versionadded{1.6}
\end{datadesc}

\begin{datadesc}{P_WAIT}
Possible value for the \var{mode} parameter to the \function{spawn*()}
family of functions.  If this is given as \var{mode}, the
\function{spawn*()} functions will not return until the new process
has run to completion and will return the exit code of the process the
run is successful, or \code{-\var{signal}} if a signal kills the
process.
Availability: Macintosh, \UNIX, Windows.
\versionadded{1.6}
\end{datadesc}

\begin{datadesc}{P_DETACH}
\dataline{P_OVERLAY}
Possible values for the \var{mode} parameter to the
\function{spawn*()} family of functions.  These are less portable than
those listed above.
\constant{P_DETACH} is similar to \constant{P_NOWAIT}, but the new
process is detached from the console of the calling process.
If \constant{P_OVERLAY} is used, the current process will be replaced;
the \function{spawn*()} function will not return.
Availability: Windows.
\versionadded{1.6}
\end{datadesc}

\begin{funcdesc}{startfile}{path}
Start a file with its associated application.  This acts like
double-clicking the file in Windows Explorer, or giving the file name
as an argument to the \program{start} command from the interactive
command shell: the file is opened with whatever application (if any)
its extension is associated.

\function{startfile()} returns as soon as the associated application
is launched.  There is no option to wait for the application to close,
and no way to retrieve the application's exit status.  The \var{path}
parameter is relative to the current directory.  If you want to use an
absolute path, make sure the first character is not a slash
(\character{/}); the underlying Win32 \cfunction{ShellExecute()}
function doesn't work if it is.  Use the \function{os.path.normpath()}
function to ensure that the path is properly encoded for Win32.
Availability: Windows.
\versionadded{2.0}
\end{funcdesc}

\begin{funcdesc}{system}{command}
Execute the command (a string) in a subshell.  This is implemented by
calling the Standard C function \cfunction{system()}, and has the
same limitations.  Changes to \code{posix.environ}, \code{sys.stdin},
etc.\ are not reflected in the environment of the executed command.

On \UNIX, the return value is the exit status of the process encoded in the
format specified for \function{wait()}.  Note that \POSIX{} does not
specify the meaning of the return value of the C \cfunction{system()}
function, so the return value of the Python function is system-dependent.

On Windows, the return value is that returned by the system shell after
running \var{command}, given by the Windows environment variable
\envvar{COMSPEC}: on \program{command.com} systems (Windows 95, 98 and ME)
this is always \code{0}; on \program{cmd.exe} systems (Windows NT, 2000
and XP) this is the exit status of the command run; on systems using
a non-native shell, consult your shell documentation.

Availability: Macintosh, \UNIX, Windows.
\end{funcdesc}

\begin{funcdesc}{times}{}
Return a 5-tuple of floating point numbers indicating accumulated
(processor or other)
times, in seconds.  The items are: user time, system time, children's
user time, children's system time, and elapsed real time since a fixed
point in the past, in that order.  See the \UNIX{} manual page
\manpage{times}{2} or the corresponding Windows Platform API
documentation.
Availability: Macintosh, \UNIX, Windows.
\end{funcdesc}

\begin{funcdesc}{wait}{}
Wait for completion of a child process, and return a tuple containing
its pid and exit status indication: a 16-bit number, whose low byte is
the signal number that killed the process, and whose high byte is the
exit status (if the signal number is zero); the high bit of the low
byte is set if a core file was produced.
Availability: Macintosh, \UNIX.
\end{funcdesc}

\begin{funcdesc}{waitpid}{pid, options}
The details of this function differ on \UNIX{} and Windows.

On \UNIX:
Wait for completion of a child process given by process id \var{pid},
and return a tuple containing its process id and exit status
indication (encoded as for \function{wait()}).  The semantics of the
call are affected by the value of the integer \var{options}, which
should be \code{0} for normal operation.

If \var{pid} is greater than \code{0}, \function{waitpid()} requests
status information for that specific process.  If \var{pid} is
\code{0}, the request is for the status of any child in the process
group of the current process.  If \var{pid} is \code{-1}, the request
pertains to any child of the current process.  If \var{pid} is less
than \code{-1}, status is requested for any process in the process
group \code{-\var{pid}} (the absolute value of \var{pid}).

On Windows:
Wait for completion of a process given by process handle \var{pid},
and return a tuple containing \var{pid},
and its exit status shifted left by 8 bits (shifting makes cross-platform
use of the function easier).
A \var{pid} less than or equal to \code{0} has no special meaning on
Windows, and raises an exception.
The value of integer \var{options} has no effect.
\var{pid} can refer to any process whose id is known, not necessarily a
child process.
The \function{spawn()} functions called with \constant{P_NOWAIT}
return suitable process handles.
\end{funcdesc}

\begin{datadesc}{WNOHANG}
The option for \function{waitpid()} to avoid hanging if no child
process status is available immediately.
Availability: Macintosh, \UNIX.
\end{datadesc}

\begin{datadesc}{WCONTINUED}
This option causes child processes to be reported if they have been
continued from a job control stop since their status was last
reported.
Availability: Some \UNIX{} systems.
\versionadded{2.3}
\end{datadesc}

\begin{datadesc}{WUNTRACED}
This option causes child processes to be reported if they have been
stopped but their current state has not been reported since they were
stopped.
Availability: Macintosh, \UNIX.
\versionadded{2.3}
\end{datadesc}

The following functions take a process status code as returned by
\function{system()}, \function{wait()}, or \function{waitpid()} as a
parameter.  They may be used to determine the disposition of a
process.

\begin{funcdesc}{WCOREDUMP}{status}
Returns \code{True} if a core dump was generated for the process,
otherwise it returns \code{False}.
Availability: Macintosh, \UNIX.
\versionadded{2.3}
\end{funcdesc}

\begin{funcdesc}{WIFCONTINUED}{status}
Returns \code{True} if the process has been continued from a job
control stop, otherwise it returns \code{False}.
Availability: \UNIX.
\versionadded{2.3}
\end{funcdesc}

\begin{funcdesc}{WIFSTOPPED}{status}
Returns \code{True} if the process has been stopped, otherwise it
returns \code{False}.
Availability: \UNIX.
\end{funcdesc}

\begin{funcdesc}{WIFSIGNALED}{status}
Returns \code{True} if the process exited due to a signal, otherwise
it returns \code{False}.
Availability: Macintosh, \UNIX.
\end{funcdesc}

\begin{funcdesc}{WIFEXITED}{status}
Returns \code{True} if the process exited using the \manpage{exit}{2}
system call, otherwise it returns \code{False}.
Availability: Macintosh, \UNIX.
\end{funcdesc}

\begin{funcdesc}{WEXITSTATUS}{status}
If \code{WIFEXITED(\var{status})} is true, return the integer
parameter to the \manpage{exit}{2} system call.  Otherwise, the return
value is meaningless.
Availability: Macintosh, \UNIX.
\end{funcdesc}

\begin{funcdesc}{WSTOPSIG}{status}
Return the signal which caused the process to stop.
Availability: Macintosh, \UNIX.
\end{funcdesc}

\begin{funcdesc}{WTERMSIG}{status}
Return the signal which caused the process to exit.
Availability: Macintosh, \UNIX.
\end{funcdesc}


\subsection{Miscellaneous System Information \label{os-path}}


\begin{funcdesc}{confstr}{name}
Return string-valued system configuration values.
\var{name} specifies the configuration value to retrieve; it may be a
string which is the name of a defined system value; these names are
specified in a number of standards (\POSIX, \UNIX{} 95, \UNIX{} 98, and
others).  Some platforms define additional names as well.  The names
known to the host operating system are given in the
\code{confstr_names} dictionary.  For configuration variables not
included in that mapping, passing an integer for \var{name} is also
accepted.
Availability: Macintosh, \UNIX.

If the configuration value specified by \var{name} isn't defined, the
empty string is returned.

If \var{name} is a string and is not known, \exception{ValueError} is
raised.  If a specific value for \var{name} is not supported by the
host system, even if it is included in \code{confstr_names}, an
\exception{OSError} is raised with \constant{errno.EINVAL} for the
error number.
\end{funcdesc}

\begin{datadesc}{confstr_names}
Dictionary mapping names accepted by \function{confstr()} to the
integer values defined for those names by the host operating system.
This can be used to determine the set of names known to the system.
Availability: Macintosh, \UNIX.
\end{datadesc}

\begin{funcdesc}{getloadavg}{}
Return the number of processes in the system run queue averaged over
the last 1, 5, and 15 minutes or raises OSError if the load average
was unobtainable.

\versionadded{2.3}
\end{funcdesc}

\begin{funcdesc}{sysconf}{name}
Return integer-valued system configuration values.
If the configuration value specified by \var{name} isn't defined,
\code{-1} is returned.  The comments regarding the \var{name}
parameter for \function{confstr()} apply here as well; the dictionary
that provides information on the known names is given by
\code{sysconf_names}.
Availability: Macintosh, \UNIX.
\end{funcdesc}

\begin{datadesc}{sysconf_names}
Dictionary mapping names accepted by \function{sysconf()} to the
integer values defined for those names by the host operating system.
This can be used to determine the set of names known to the system.
Availability: Macintosh, \UNIX.
\end{datadesc}


The follow data values are used to support path manipulation
operations.  These are defined for all platforms.

Higher-level operations on pathnames are defined in the
\refmodule{os.path} module.


\begin{datadesc}{curdir}
The constant string used by the operating system to refer to the current
directory.
For example: \code{'.'} for \POSIX{} or \code{':'} for Mac OS 9.
Also available via \module{os.path}.
\end{datadesc}

\begin{datadesc}{pardir}
The constant string used by the operating system to refer to the parent
directory.
For example: \code{'..'} for \POSIX{} or \code{'::'} for Mac OS 9.
Also available via \module{os.path}.
\end{datadesc}

\begin{datadesc}{sep}
The character used by the operating system to separate pathname components,
for example, \character{/} for \POSIX{} or \character{:} for
Mac OS 9.  Note that knowing this is not sufficient to be able to
parse or concatenate pathnames --- use \function{os.path.split()} and
\function{os.path.join()} --- but it is occasionally useful.
Also available via \module{os.path}.
\end{datadesc}

\begin{datadesc}{altsep}
An alternative character used by the operating system to separate pathname
components, or \code{None} if only one separator character exists.  This is
set to \character{/} on Windows systems where \code{sep} is a
backslash.
Also available via \module{os.path}.
\end{datadesc}

\begin{datadesc}{extsep}
The character which separates the base filename from the extension;
for example, the \character{.} in \file{os.py}.
Also available via \module{os.path}.
\versionadded{2.2}
\end{datadesc}

\begin{datadesc}{pathsep}
The character conventionally used by the operating system to separate
search path components (as in \envvar{PATH}), such as \character{:} for
\POSIX{} or \character{;} for Windows.
Also available via \module{os.path}.
\end{datadesc}

\begin{datadesc}{defpath}
The default search path used by \function{exec*p*()} and
\function{spawn*p*()} if the environment doesn't have a \code{'PATH'}
key.
Also available via \module{os.path}.
\end{datadesc}

\begin{datadesc}{linesep}
The string used to separate (or, rather, terminate) lines on the
current platform.  This may be a single character, such as \code{'\e
n'} for \POSIX{} or \code{'\e r'} for Mac OS, or multiple characters,
for example, \code{'\e r\e n'} for Windows.
\end{datadesc}

\begin{datadesc}{devnull}
The file path of the null device.
For example: \code{'/dev/null'} for \POSIX{} or \code{'Dev:Nul'} for
Mac OS 9.
Also available via \module{os.path}.
\versionadded{2.4}
\end{datadesc}


\subsection{Miscellaneous Functions \label{os-miscfunc}}

\begin{funcdesc}{urandom}{n}
Return a string of \var{n} random bytes suitable for cryptographic use.

This function returns random bytes from an OS-specific
randomness source.  The returned data should be unpredictable enough for
cryptographic applications, though its exact quality depends on the OS
implementation.  On a UNIX-like system this will query /dev/urandom, and
on Windows it will use CryptGenRandom.  If a randomness source is not
found, \exception{NotImplementedError} will be raised.
\versionadded{2.4}
\end{funcdesc}





\section{\module{time} ---
         Time access and conversions}

\declaremodule{builtin}{time}
\modulesynopsis{Time access and conversions.}


This module provides various time-related functions.
It is always available, but not all functions are available
on all platforms.

An explanation of some terminology and conventions is in order.

\begin{itemize}

\item
The \dfn{epoch}\index{epoch} is the point where the time starts.  On
January 1st of that year, at 0 hours, the ``time since the epoch'' is
zero.  For \UNIX{}, the epoch is 1970.  To find out what the epoch is,
look at \code{gmtime(0)}.

\item
The functions in this module do not handle dates and times before the
epoch or far in the future.  The cut-off point in the future is
determined by the C library; for \UNIX{}, it is typically in
2038\index{Year 2038}.

\item
\strong{Year 2000 (Y2K) issues}:\index{Year 2000}\index{Y2K}  Python
depends on the platform's C library, which generally doesn't have year
2000 issues, since all dates and times are represented internally as
seconds since the epoch.  Functions accepting a time tuple (see below)
generally require a 4-digit year.  For backward compatibility, 2-digit
years are supported if the module variable \code{accept2dyear} is a
non-zero integer; this variable is initialized to \code{1} unless the
environment variable \envvar{PYTHONY2K} is set to a non-empty string,
in which case it is initialized to \code{0}.  Thus, you can set
\envvar{PYTHONY2K} to a non-empty string in the environment to require 4-digit
years for all year input.  When 2-digit years are accepted, they are
converted according to the \POSIX{} or X/Open standard: values 69-99
are mapped to 1969-1999, and values 0--68 are mapped to 2000--2068.
Values 100--1899 are always illegal.  Note that this is new as of
Python 1.5.2(a2); earlier versions, up to Python 1.5.1 and 1.5.2a1,
would add 1900 to year values below 1900.

\item
UTC\index{UTC} is Coordinated Universal Time\index{Coordinated
Universal Time} (formerly known as Greenwich Mean
Time,\index{Greenwich Mean Time} or GMT).  The acronym UTC is not a
mistake but a compromise between English and French.

\item
DST is Daylight Saving Time,\index{Daylight Saving Time} an adjustment
of the timezone by (usually) one hour during part of the year.  DST
rules are magic (determined by local law) and can change from year to
year.  The C library has a table containing the local rules (often it
is read from a system file for flexibility) and is the only source of
True Wisdom in this respect.

\item
The precision of the various real-time functions may be less than
suggested by the units in which their value or argument is expressed.
E.g.\ on most \UNIX{} systems, the clock ``ticks'' only 50 or 100 times a
second, and on the Mac, times are only accurate to whole seconds.

\item
On the other hand, the precision of \function{time()} and
\function{sleep()} is better than their \UNIX{} equivalents: times are
expressed as floating point numbers, \function{time()} returns the
most accurate time available (using \UNIX{} \cfunction{gettimeofday()}
where available), and \function{sleep()} will accept a time with a
nonzero fraction (\UNIX{} \cfunction{select()} is used to implement
this, where available).

\item

The time tuple as returned by \function{gmtime()},
\function{localtime()}, and \function{strptime()}, and accepted by
\function{asctime()}, \function{mktime()} and \function{strftime()},
is a tuple of 9 integers:

\begin{tableiii}{r|l|l}{textrm}{Index}{Field}{Values}
  \lineiii{0}{year}{(e.g.\ 1993)}
  \lineiii{1}{month}{range [1,12]}
  \lineiii{2}{day}{range [1,31]}
  \lineiii{3}{hour}{range [0,23]}
  \lineiii{4}{minute}{range [0,59]}
  \lineiii{5}{second}{range [0,61]; see \strong{(1)} in \function{strftime()} description}
  \lineiii{6}{weekday}{range [0,6], Monday is 0}
  \lineiii{7}{Julian day}{range [1,366]}
  \lineiii{8}{daylight savings flag}{0, 1 or -1; see below}
\end{tableiii}

Note that unlike the C structure, the month value is a
range of 1-12, not 0-11.  A year value will be handled as described
under ``Year 2000 (Y2K) issues'' above.  A \code{-1} argument as
daylight savings flag, passed to \function{mktime()} will usually
result in the correct daylight savings state to be filled in.

\end{itemize}

The module defines the following functions and data items:


\begin{datadesc}{accept2dyear}
Boolean value indicating whether two-digit year values will be
accepted.  This is true by default, but will be set to false if the
environment variable \envvar{PYTHONY2K} has been set to a non-empty
string.  It may also be modified at run time.
\end{datadesc}

\begin{datadesc}{altzone}
The offset of the local DST timezone, in seconds west of UTC, if one
is defined.  This is negative if the local DST timezone is east of UTC
(as in Western Europe, including the UK).  Only use this if
\code{daylight} is nonzero.
\end{datadesc}

\begin{funcdesc}{asctime}{\optional{tuple}}
Convert a tuple representing a time as returned by \function{gmtime()}
or \function{localtime()} to a 24-character string of the following form:
\code{'Sun Jun 20 23:21:05 1993'}.  If \var{tuple} is not provided, the
current time as returned by \function{localtime()} is used.  Note: unlike
the C function of the same name, there is no trailing newline.
\versionchanged[Allowed \var{tuple} to be omitted]{2.1}
\end{funcdesc}

\begin{funcdesc}{clock}{}
On \UNIX, return
the current processor time as a floating point number expressed in
seconds.  The precision, and in fact the very definition of the meaning
of ``processor time''\index{CPU time}\index{processor time}, depends
on that of the C function of the same name, but in any case, this is
the function to use for benchmarking\index{benchmarking} Python or
timing algorithms.

On Windows, this function returns the nearest approximation to
wall-clock time since the first call to this function, based on the
Win32 function \cfunction{QueryPerformanceCounter()}.  The resolution
is typically better than one microsecond.
\end{funcdesc}

\begin{funcdesc}{ctime}{\optional{secs}}
Convert a time expressed in seconds since the epoch to a string
representing local time. If \var{secs} is not provided, the current time
as returned by \function{time()} is used.  \code{ctime(\var{secs})}
is equivalent to \code{asctime(localtime(\var{secs}))}.
\versionchanged[Allowed \var{secs} to be omitted]{2.1}
\end{funcdesc}

\begin{datadesc}{daylight}
Nonzero if a DST timezone is defined.
\end{datadesc}

\begin{funcdesc}{gmtime}{\optional{secs}}
Convert a time expressed in seconds since the epoch to a time tuple
in UTC in which the dst flag is always zero.  If \var{secs} is not
provided, the current time as returned by \function{time()} is used.
Fractions of a second are ignored.  See above for a description of the
tuple lay-out.
\versionchanged[Allowed \var{secs} to be omitted]{2.1}
\end{funcdesc}

\begin{funcdesc}{localtime}{\optional{secs}}
Like \function{gmtime()} but converts to local time.  The dst flag is
set to \code{1} when DST applies to the given time.
\versionchanged[Allowed \var{secs} to be omitted]{2.1}
\end{funcdesc}

\begin{funcdesc}{mktime}{tuple}
This is the inverse function of \function{localtime()}.  Its argument
is the full 9-tuple (since the dst flag is needed; use \code{-1} as
the dst flag if it is unknown) which expresses the time in
\emph{local} time, not UTC.  It returns a floating point number, for
compatibility with \function{time()}.  If the input value cannot be
represented as a valid time, \exception{OverflowError} is raised.
\end{funcdesc}

\begin{funcdesc}{sleep}{secs}
Suspend execution for the given number of seconds.  The argument may
be a floating point number to indicate a more precise sleep time.
The actual suspension time may be less than that requested because any
caught signal will terminate the \function{sleep()} following
execution of that signal's catching routine.  Also, the suspension
time may be longer than requested by an arbitrary amount because of
the scheduling of other activity in the system.
\end{funcdesc}

\begin{funcdesc}{strftime}{format\optional{, tuple}}
Convert a tuple representing a time as returned by \function{gmtime()}
or \function{localtime()} to a string as specified by the \var{format}
argument.  If \var{tuple} is not provided, the current time as returned by
\function{localtime()} is used.  \var{format} must be a string.
\versionchanged[Allowed \var{tuple} to be omitted]{2.1}

The following directives can be embedded in the \var{format} string.
They are shown without the optional field width and precision
specification, and are replaced by the indicated characters in the
\function{strftime()} result:

\begin{tableiii}{c|p{24em}|c}{code}{Directive}{Meaning}{Notes}
  \lineiii{\%a}{Locale's abbreviated weekday name.}{}
  \lineiii{\%A}{Locale's full weekday name.}{}
  \lineiii{\%b}{Locale's abbreviated month name.}{}
  \lineiii{\%B}{Locale's full month name.}{}
  \lineiii{\%c}{Locale's appropriate date and time representation.}{}
  \lineiii{\%d}{Day of the month as a decimal number [01,31].}{}
  \lineiii{\%H}{Hour (24-hour clock) as a decimal number [00,23].}{}
  \lineiii{\%I}{Hour (12-hour clock) as a decimal number [01,12].}{}
  \lineiii{\%j}{Day of the year as a decimal number [001,366].}{}
  \lineiii{\%m}{Month as a decimal number [01,12].}{}
  \lineiii{\%M}{Minute as a decimal number [00,59].}{}
  \lineiii{\%p}{Locale's equivalent of either AM or PM.}{}
  \lineiii{\%S}{Second as a decimal number [00,61].}{(1)}
  \lineiii{\%U}{Week number of the year (Sunday as the first day of the
                week) as a decimal number [00,53].  All days in a new year
                preceding the first Sunday are considered to be in week 0.}{}
  \lineiii{\%w}{Weekday as a decimal number [0(Sunday),6].}{}
  \lineiii{\%W}{Week number of the year (Monday as the first day of the
                week) as a decimal number [00,53].  All days in a new year
                preceding the first Sunday are considered to be in week 0.}{}
  \lineiii{\%x}{Locale's appropriate date representation.}{}
  \lineiii{\%X}{Locale's appropriate time representation.}{}
  \lineiii{\%y}{Year without century as a decimal number [00,99].}{}
  \lineiii{\%Y}{Year with century as a decimal number.}{}
  \lineiii{\%Z}{Time zone name (or by no characters if no time zone exists).}{}
  \lineiii{\%\%}{A literal \character{\%} character.}{}
\end{tableiii}

\noindent
Notes:

\begin{description}
  \item[(1)]
    The range really is \code{0} to \code{61}; this accounts for leap
    seconds and the (very rare) double leap seconds.
\end{description}

Here is an example, a format for dates compatible with that specified 
in the \rfc{2822} Internet email standard.
	\footnote{The use of \code{\%Z} is now
	deprecated, but the \code{\%z} escape that expands to the preferred 
	hour/minute offset is not supported by all ANSI C libraries. Also,
	a strict reading of the original 1982 \rfc{822} standard calls for
	a two-digit year (\%y rather than \%Y), but practice moved to
	4-digit years long before the year 2000.  The 4-digit year has
        been mandated by \rfc{2822}, which obsoletes \rfc{822}.}

\begin{verbatim}
>>> from time import gmtime, strftime
>>> strftime("%a, %d %b %Y %H:%M:%S +0000", gmtime())
'Thu, 28 Jun 2001 14:17:15 +0000'
\end{verbatim}

Additional directives may be supported on certain platforms, but
only the ones listed here have a meaning standardized by ANSI C.

On some platforms, an optional field width and precision
specification can immediately follow the initial \character{\%} of a
directive in the following order; this is also not portable.
The field width is normally 2 except for \code{\%j} where it is 3.
\end{funcdesc}

\begin{funcdesc}{strptime}{string\optional{, format}}
Parse a string representing a time according to a format.  The return 
value is a tuple as returned by \function{gmtime()} or
\function{localtime()}.  The \var{format} parameter uses the same
directives as those used by \function{strftime()}; it defaults to
\code{"\%a \%b \%d \%H:\%M:\%S \%Y"} which matches the formatting
returned by \function{ctime()}.  The same platform caveats apply; see
the local \UNIX{} documentation for restrictions or additional
supported directives.  If \var{string} cannot be parsed according to
\var{format}, \exception{ValueError} is raised.  Values which are not
provided as part of the input string are filled in with default
values; the specific values are platform-dependent as the XPG standard
does not provide sufficient information to constrain the result.

\strong{Note:} This function relies entirely on the underlying
platform's C library for the date parsing, and some of these libraries
are buggy.  There's nothing to be done about this short of a new,
portable implementation of \cfunction{strptime()}.

Availability: Most modern \UNIX{} systems.
\end{funcdesc}

\begin{funcdesc}{time}{}
Return the time as a floating point number expressed in seconds since
the epoch, in UTC.  Note that even though the time is always returned
as a floating point number, not all systems provide time with a better
precision than 1 second.
\end{funcdesc}

\begin{datadesc}{timezone}
The offset of the local (non-DST) timezone, in seconds west of UTC
(i.e. negative in most of Western Europe, positive in the US, zero in
the UK).
\end{datadesc}

\begin{datadesc}{tzname}
A tuple of two strings: the first is the name of the local non-DST
timezone, the second is the name of the local DST timezone.  If no DST
timezone is defined, the second string should not be used.
\end{datadesc}


\begin{seealso}
  \seemodule{locale}{Internationalization services.  The locale
                     settings can affect the return values for some of 
                     the functions in the \module{time} module.}
\end{seealso}

\section{Standard Module \sectcode{getopt}}

\stmodindex{getopt}
This module helps scripts to parse the command line arguments in
\code{sys.argv}.
It uses the same conventions as the \UNIX{}
\code{getopt()}
function (including the special meanings of arguments of the form
\samp{-} and \samp{--}).
It defines the function
\code{getopt.getopt(args, options)}
and the exception
\code{getopt.error}.

The first argument to
\code{getopt()}
is the argument list passed to the script with its first element
chopped off (i.e.,
\code{sys.argv[1:]}).
The second argument is the string of option letters that the
script wants to recognize, with options that require an argument
followed by a colon (i.e., the same format that \UNIX{}
\code{getopt()}
uses).
The return value consists of two elements: the first is a list of
option-and-value pairs; the second is the list of program arguments
left after the option list was stripped (this is a trailing slice of the
first argument).
Each option-and-value pair returned has the option as its first element,
prefixed with a hyphen (e.g.,
\code{'-x'}),
and the option argument as its second element, or an empty string if the
option has no argument.
The options occur in the list in the same order in which they were
found, thus allowing multiple occurrences.
Example:

\bcode\begin{verbatim}
>>> import getopt, string
>>> args = string.split('-a -b -cfoo -d bar a1 a2')
>>> args
['-a', '-b', '-cfoo', '-d', 'bar', 'a1', 'a2']
>>> optlist, args = getopt.getopt(args, 'abc:d:')
>>> optlist
[('-a', ''), ('-b', ''), ('-c', 'foo'), ('-d', 'bar')]
>>> args
['a1', 'a2']
>>> 
\end{verbatim}\ecode

The exception
\code{getopt.error = 'getopt error'}
is raised when an unrecognized option is found in the argument list or
when an option requiring an argument is given none.
The argument to the exception is a string indicating the cause of the
error.

\section{\module{tempfile} ---
         Generate temporary files and directories}
\sectionauthor{Zack Weinberg}{zack@codesourcery.com}

\declaremodule{standard}{tempfile}
\modulesynopsis{Generate temporary files and directories.}

\indexii{temporary}{file name}
\indexii{temporary}{file}

This module generates temporary files and directories.  It works on
all supported platforms.

In version 2.3 of Python, this module was overhauled for enhanced
security.  It now provides three new functions,
\function{NamedTemporaryFile()}, \function{mkstemp()}, and
\function{mkdtemp()}, which should eliminate all remaining need to use
the insecure \function{mktemp()} function.  Temporary file names created
by this module no longer contain the process ID; instead a string of
six random characters is used.

Also, all the user-callable functions now take additional arguments
which allow direct control over the location and name of temporary
files.  It is no longer necessary to use the global \var{tempdir} and
\var{template} variables.  To maintain backward compatibility, the
argument order is somewhat odd; it is recommended to use keyword
arguments for clarity.

The module defines the following user-callable functions:

\begin{funcdesc}{TemporaryFile}{\optional{mode=\code{'w+b'}\optional{,
                                bufsize=\code{-1}\optional{,
                                suffix\optional{, prefix\optional{, dir}}}}}}
Return a file (or file-like) object that can be used as a temporary
storage area.  The file is created using \function{mkstemp}. It will
be destroyed as soon as it is closed (including an implicit close when
the object is garbage collected).  Under \UNIX, the directory entry
for the file is removed immediately after the file is created.  Other
platforms do not support this; your code should not rely on a
temporary file created using this function having or not having a
visible name in the file system.

The \var{mode} parameter defaults to \code{'w+b'} so that the file
created can be read and written without being closed.  Binary mode is
used so that it behaves consistently on all platforms without regard
for the data that is stored.  \var{bufsize} defaults to \code{-1},
meaning that the operating system default is used.

The \var{dir}, \var{prefix} and \var{suffix} parameters are passed to
\function{mkstemp()}.
\end{funcdesc}

\begin{funcdesc}{NamedTemporaryFile}{\optional{mode=\code{'w+b'}\optional{,
                                     bufsize=\code{-1}\optional{,
                                     suffix\optional{, prefix\optional{,
                                     dir}}}}}}
This function operates exactly as \function{TemporaryFile()} does,
except that the file is guaranteed to have a visible name in the file
system (on \UNIX, the directory entry is not unlinked).  That name can
be retrieved from the \member{name} member of the file object.  Whether
the name can be used to open the file a second time, while the
named temporary file is still open, varies across platforms (it can
be so used on \UNIX; it cannot on Windows NT or later).
\versionadded{2.3}
\end{funcdesc}

\begin{funcdesc}{mkstemp}{\optional{suffix\optional{,
                          prefix\optional{, dir\optional{, text}}}}}
Creates a temporary file in the most secure manner possible.  There
are no race conditions in the file's creation, assuming that the
platform properly implements the \constant{O_EXCL} flag for
\function{os.open()}.  The file is readable and writable only by the
creating user ID.  If the platform uses permission bits to indicate
whether a file is executable, the file is executable by no one.  The
file descriptor is not inherited by child processes.

Unlike \function{TemporaryFile()}, the user of \function{mkstemp()} is
responsible for deleting the temporary file when done with it.

If \var{suffix} is specified, the file name will end with that suffix,
otherwise there will be no suffix.  \function{mkstemp()} does not put a
dot between the file name and the suffix; if you need one, put it at
the beginning of \var{suffix}.

If \var{prefix} is specified, the file name will begin with that
prefix; otherwise, a default prefix is used.

If \var{dir} is specified, the file will be created in that directory;
otherwise, a default directory is used.

If \var{text} is specified, it indicates whether to open the file in
binary mode (the default) or text mode.  On some platforms, this makes
no difference.

\function{mkstemp()} returns a tuple containing an OS-level handle to
an open file (as would be returned by \function{os.open()}) and the
absolute pathname of that file, in that order.
\versionadded{2.3}
\end{funcdesc}

\begin{funcdesc}{mkdtemp}{\optional{suffix\optional{, prefix\optional{, dir}}}}
Creates a temporary directory in the most secure manner possible.
There are no race conditions in the directory's creation.  The
directory is readable, writable, and searchable only by the
creating user ID.

The user of \function{mkdtemp()} is responsible for deleting the
temporary directory and its contents when done with it.

The \var{prefix}, \var{suffix}, and \var{dir} arguments are the same
as for \function{mkstemp()}.

\function{mkdtemp()} returns the absolute pathname of the new directory.
\versionadded{2.3}
\end{funcdesc}

\begin{funcdesc}{mktemp}{\optional{suffix\optional{, prefix\optional{, dir}}}}
\deprecated{2.3}{Use \function{mkstemp()} instead.}
Return an absolute pathname of a file that did not exist at the time
the call is made.  The \var{prefix}, \var{suffix}, and \var{dir}
arguments are the same as for \function{mkstemp()}.

\warning{Use of this function may introduce a security hole in your
program.  By the time you get around to doing anything with the file
name it returns, someone else may have beaten you to the punch.}
\end{funcdesc}

The module uses two global variables that tell it how to construct a
temporary name.  They are initialized at the first call to any of the
functions above.  The caller may change them, but this is discouraged;
use the appropriate function arguments, instead.

\begin{datadesc}{tempdir}
When set to a value other than \code{None}, this variable defines the
default value for the \var{dir} argument to all the functions defined
in this module.

If \code{tempdir} is unset or \code{None} at any call to any of the
above functions, Python searches a standard list of directories and
sets \var{tempdir} to the first one which the calling user can create
files in.  The list is:

\begin{enumerate}
\item The directory named by the \envvar{TMPDIR} environment variable.
\item The directory named by the \envvar{TEMP} environment variable.
\item The directory named by the \envvar{TMP} environment variable.
\item A platform-specific location:
    \begin{itemize}
    \item On RiscOS, the directory named by the
          \envvar{Wimp\$ScrapDir} environment variable.
    \item On Windows, the directories
          \file{C:$\backslash$TEMP},
          \file{C:$\backslash$TMP},
          \file{$\backslash$TEMP}, and
          \file{$\backslash$TMP}, in that order.
    \item On all other platforms, the directories
          \file{/tmp}, \file{/var/tmp}, and \file{/usr/tmp}, in that order.
    \end{itemize}
\item As a last resort, the current working directory.
\end{enumerate}
\end{datadesc}

\begin{funcdesc}{gettempdir}{}
Return the directory currently selected to create temporary files in.
If \code{tempdir} is not \code{None}, this simply returns its contents;
otherwise, the search described above is performed, and the result
returned.
\end{funcdesc}

\begin{datadesc}{template}
\deprecated{2.0}{Use \function{gettempprefix()} instead.}
When set to a value other than \code{None}, this variable defines the
prefix of the final component of the filenames returned by
\function{mktemp()}.  A string of six random letters and digits is
appended to the prefix to make the filename unique.  On Windows,
the default prefix is \file{\textasciitilde{}T}; on all other systems
it is \file{tmp}.

Older versions of this module used to require that \code{template} be
set to \code{None} after a call to \function{os.fork()}; this has not
been necessary since version 1.5.2.
\end{datadesc}

\begin{funcdesc}{gettempprefix}{}
Return the filename prefix used to create temporary files.  This does
not contain the directory component.  Using this function is preferred
over reading the \var{template} variable directly.
\versionadded{1.5.2}
\end{funcdesc}


\chapter{Optional Operating System Services}
\label{someos}

The modules described in this chapter provide interfaces to operating
system features that are available on selected operating systems only.
The interfaces are generally modelled after the \UNIX{} or \C{}
interfaces but they are available on some other systems as well
(e.g. Windows or NT).  Here's an overview:

\begin{description}

\item[signal]
--- Set handlers for asynchronous events.

\item[socket]
--- Low-level networking interface.

\item[select]
--- Wait for I/O completion on multiple streams.

\item[thread]
--- Create multiple threads of control within one namespace.

\item[threading]
--- Higher level threading interface; use in preference of module
\module{thread}.

\item[Queue]
--- A stynchronized queue class.

\item[anydbm]
--- Generic interface to DBM-style database modules.

\item[whichdb]
--- Guess which DBM-style module created a given database.

\item[zlib]
\item[gzip]
--- Compression and decompression compatible with the
\program{gzip} program (\module{zlib} is the low-level interface,
\module{gzip} the high-level one).

\end{description}
		% Optional Operating System Services
\section{Built-in Module \sectcode{signal}}

\bimodindex{signal}
This module provides mechanisms to write signal handlers in Python.

{\bf Warning:} Some care must be taken if both signals and threads
will be used in the same program.  The fundamental thing to remember
in using signals and threads simultaneously is: always perform
\code{signal()} operations in the main thread of execution.  Any
thread can perform a \code{alarm()}, \code{getsignal()}, or
\code{pause()}; only the main thread can set a new signal handler, and
the main thread will be the only one to receive signals.  This means
that signals can't be used as a means of interthread communication.
Use locks instead.

The variables defined in the signal module are:

\renewcommand{\indexsubitem}{(in module signal)}
\begin{datadesc}{SIG_DFL}
  This is one of two standard signal handling options; it will simply
  perform the default function for the signal.  For example, on most
  systems the default action for SIGQUIT is to dump core and exit,
  while the default action for SIGCLD is to simply ignore it.
\end{datadesc}

\begin{datadesc}{SIG_IGN}
  This is another standard signal handler, which will simply ignore
  the given signal.
\end{datadesc}

\begin{datadesc}{SIG*}
  All the signal numbers are defined symbolically.  For example, the
  hangup signal is defined as \code{signal.SIGHUP}; the variable names
  are identical to the names used in C programs, as found in
  \file{signal.h}.
  The UNIX man page for \file{signal} lists the existing signals (on
  some systems this is \file{signal(2)}, on others the list is in
  \file{signal(7)}).
  Note that not all systems define the same set of signal names; only
  those names defined by the system are defined by this module.
\end{datadesc}

The signal module defines the following functions:

\begin{funcdesc}{alarm}{time}
  If \var{time} is non-zero, this function requests that a
  \code{SIGALRM} signal be sent to the process in \var{time} seconds.
  Any previously scheduled alarm is canceled (i.e. only one alarm can
  be scheduled at any time).  The returned value is then the number of
  seconds before any previously set alarm was to have been delivered.
  If \var{time} is zero, no alarm id scheduled, and any scheduled
  alarm is canceled.  The return value is the number of seconds
  remaining before a previously scheduled alarm.  If the return value
  is zero, no alarm is currently scheduled.  (See the UNIX man page
  \code{alarm(2)}.)
\end{funcdesc}

\begin{funcdesc}{getsignal}{signalnum}
  Returns the current signal handler for the signal \var{signalnum}.
  The returned value may be a callable Python object, or one of the
  special values \code{signal.SIG_IGN} or \code{signal.SIG_DFL}.
\end{funcdesc}

\begin{funcdesc}{pause}{}
  Causes the process to sleep until a signal is received; the
  appropriate handler will then be called.  Returns nothing.  (See the
  UNIX man page \code{signal(2)}.)
\end{funcdesc}

\begin{funcdesc}{signal}{signalnum\, handler}
  Sets the handler for signal \var{signalnum} to the function
  \var{handler}.  \var{handler} can be any callable Python object, or
  one of the special values \code{signal.SIG_IGN} or
  \code{signal.SIG_DFL}.  The previous signal handler will be
  returned.  (See the UNIX man page \code{signal(2)}.)

  If threads are enabled, this function can only be called from the
  main thread; attempting to call it from other threads will cause a
  \code{ValueError} exception will be raised.
\end{funcdesc}

\section{\module{socket} ---
         Low-level networking interface}

\declaremodule{builtin}{socket}
\modulesynopsis{Low-level networking interface.}


This module provides access to the BSD \emph{socket} interface.
It is available on all modern \UNIX{} systems, Windows, MacOS, BeOS,
OS/2, and probably additional platforms.

For an introduction to socket programming (in C), see the following
papers: \citetitle{An Introductory 4.3BSD Interprocess Communication
Tutorial}, by Stuart Sechrest and \citetitle{An Advanced 4.3BSD
Interprocess Communication Tutorial}, by Samuel J.  Leffler et al,
both in the \citetitle{\UNIX{} Programmer's Manual, Supplementary Documents 1}
(sections PS1:7 and PS1:8).  The platform-specific reference material
for the various socket-related system calls are also a valuable source
of information on the details of socket semantics.  For \UNIX, refer
to the manual pages; for Windows, see the WinSock (or Winsock 2)
specification.

The Python interface is a straightforward transliteration of the
\UNIX{} system call and library interface for sockets to Python's
object-oriented style: the \function{socket()} function returns a
\dfn{socket object}\obindex{socket} whose methods implement the
various socket system calls.  Parameter types are somewhat
higher-level than in the C interface: as with \method{read()} and
\method{write()} operations on Python files, buffer allocation on
receive operations is automatic, and buffer length is implicit on send
operations.

Socket addresses are represented as a single string for the
\constant{AF_UNIX} address family and as a pair
\code{(\var{host}, \var{port})} for the \constant{AF_INET} address
family, where \var{host} is a string representing
either a hostname in Internet domain notation like
\code{'daring.cwi.nl'} or an IP address like \code{'100.50.200.5'},
and \var{port} is an integral port number.  Other address families are
currently not supported.  The address format required by a particular
socket object is automatically selected based on the address family
specified when the socket object was created.

For IP addresses, two special forms are accepted instead of a host
address: the empty string represents \constant{INADDR_ANY}, and the string
\code{'<broadcast>'} represents \constant{INADDR_BROADCAST}.

All errors raise exceptions.  The normal exceptions for invalid
argument types and out-of-memory conditions can be raised; errors
related to socket or address semantics raise the error
\exception{socket.error}.

Non-blocking mode is supported through the
\method{setblocking()} method.

The module \module{socket} exports the following constants and functions:


\begin{excdesc}{error}
This exception is raised for socket- or address-related errors.
The accompanying value is either a string telling what went wrong or a
pair \code{(\var{errno}, \var{string})}
representing an error returned by a system
call, similar to the value accompanying \exception{os.error}.
See the module \refmodule{errno}\refbimodindex{errno}, which contains
names for the error codes defined by the underlying operating system.
\end{excdesc}

\begin{datadesc}{AF_UNIX}
\dataline{AF_INET}
These constants represent the address (and protocol) families,
used for the first argument to \function{socket()}.  If the
\constant{AF_UNIX} constant is not defined then this protocol is
unsupported.
\end{datadesc}

\begin{datadesc}{SOCK_STREAM}
\dataline{SOCK_DGRAM}
\dataline{SOCK_RAW}
\dataline{SOCK_RDM}
\dataline{SOCK_SEQPACKET}
These constants represent the socket types,
used for the second argument to \function{socket()}.
(Only \constant{SOCK_STREAM} and
\constant{SOCK_DGRAM} appear to be generally useful.)
\end{datadesc}

\begin{datadesc}{SO_*}
\dataline{SOMAXCONN}
\dataline{MSG_*}
\dataline{SOL_*}
\dataline{IPPROTO_*}
\dataline{IPPORT_*}
\dataline{INADDR_*}
\dataline{IP_*}
Many constants of these forms, documented in the \UNIX{} documentation on
sockets and/or the IP protocol, are also defined in the socket module.
They are generally used in arguments to the \method{setsockopt()} and
\method{getsockopt()} methods of socket objects.  In most cases, only
those symbols that are defined in the \UNIX{} header files are defined;
for a few symbols, default values are provided.
\end{datadesc}

\begin{funcdesc}{getfqdn}{\optional{name}}
Return a fully qualified domain name for \var{name}.
If \var{name} is omitted or empty, it is interpreted as the local
host.  To find the fully qualified name, the hostname returned by
\function{gethostbyaddr()} is checked, then aliases for the host, if
available.  The first name which includes a period is selected.  In
case no fully qualified domain name is available, the hostname is
returned.
\versionadded{2.0}
\end{funcdesc}

\begin{funcdesc}{gethostbyname}{hostname}
Translate a host name to IP address format.  The IP address is
returned as a string, e.g.,  \code{'100.50.200.5'}.  If the host name
is an IP address itself it is returned unchanged.  See
\function{gethostbyname_ex()} for a more complete interface.
\end{funcdesc}

\begin{funcdesc}{gethostbyname_ex}{hostname}
Translate a host name to IP address format, extended interface.
Return a triple \code{(hostname, aliaslist, ipaddrlist)} where
\code{hostname} is the primary host name responding to the given
\var{ip_address}, \code{aliaslist} is a (possibly empty) list of
alternative host names for the same address, and \code{ipaddrlist} is
a list of IP addresses for the same interface on the same
host (often but not always a single address).
\end{funcdesc}

\begin{funcdesc}{gethostname}{}
Return a string containing the hostname of the machine where 
the Python interpreter is currently executing.  If you want to know the
current machine's IP address, use \code{gethostbyname(gethostname())}.
Note: \function{gethostname()} doesn't always return the fully qualified
domain name; use \code{gethostbyaddr(gethostname())}
(see below).
\end{funcdesc}

\begin{funcdesc}{gethostbyaddr}{ip_address}
Return a triple \code{(\var{hostname}, \var{aliaslist},
\var{ipaddrlist})} where \var{hostname} is the primary host name
responding to the given \var{ip_address}, \var{aliaslist} is a
(possibly empty) list of alternative host names for the same address,
and \var{ipaddrlist} is a list of IP addresses for the same interface
on the same host (most likely containing only a single address).
To find the fully qualified domain name, use the function
\function{getfqdn()}.
\end{funcdesc}

\begin{funcdesc}{getprotobyname}{protocolname}
Translate an Internet protocol name (e.g.\ \code{'icmp'}) to a constant
suitable for passing as the (optional) third argument to the
\function{socket()} function.  This is usually only needed for sockets
opened in ``raw'' mode (\constant{SOCK_RAW}); for the normal socket
modes, the correct protocol is chosen automatically if the protocol is
omitted or zero.
\end{funcdesc}

\begin{funcdesc}{getservbyname}{servicename, protocolname}
Translate an Internet service name and protocol name to a port number
for that service.  The protocol name should be \code{'tcp'} or
\code{'udp'}.
\end{funcdesc}

\begin{funcdesc}{socket}{family, type\optional{, proto}}
Create a new socket using the given address family, socket type and
protocol number.  The address family should be \constant{AF_INET} or
\constant{AF_UNIX}.  The socket type should be \constant{SOCK_STREAM},
\constant{SOCK_DGRAM} or perhaps one of the other \samp{SOCK_} constants.
The protocol number is usually zero and may be omitted in that case.
\end{funcdesc}

\begin{funcdesc}{fromfd}{fd, family, type\optional{, proto}}
Build a socket object from an existing file descriptor (an integer as
returned by a file object's \method{fileno()} method).  Address family,
socket type and protocol number are as for the \function{socket()} function
above.  The file descriptor should refer to a socket, but this is not
checked --- subsequent operations on the object may fail if the file
descriptor is invalid.  This function is rarely needed, but can be
used to get or set socket options on a socket passed to a program as
standard input or output (e.g.\ a server started by the \UNIX{} inet
daemon).
\end{funcdesc}

\begin{funcdesc}{ntohl}{x}
Convert 32-bit integers from network to host byte order.  On machines
where the host byte order is the same as network byte order, this is a
no-op; otherwise, it performs a 4-byte swap operation.
\end{funcdesc}

\begin{funcdesc}{ntohs}{x}
Convert 16-bit integers from network to host byte order.  On machines
where the host byte order is the same as network byte order, this is a
no-op; otherwise, it performs a 2-byte swap operation.
\end{funcdesc}

\begin{funcdesc}{htonl}{x}
Convert 32-bit integers from host to network byte order.  On machines
where the host byte order is the same as network byte order, this is a
no-op; otherwise, it performs a 4-byte swap operation.
\end{funcdesc}

\begin{funcdesc}{htons}{x}
Convert 16-bit integers from host to network byte order.  On machines
where the host byte order is the same as network byte order, this is a
no-op; otherwise, it performs a 2-byte swap operation.
\end{funcdesc}

\begin{funcdesc}{inet_aton}{ip_string}
Convert an IP address from dotted-quad string format
(e.g.\ '123.45.67.89') to 32-bit packed binary format, as a string four
characters in length.

Useful when conversing with a program that uses the standard C library
and needs objects of type \ctype{struct in_addr}, which is the C type
for the 32-bit packed binary this function returns.

If the IP address string passed to this function is invalid,
\exception{socket.error} will be raised. Note that exactly what is
valid depends on the underlying C implementation of
\cfunction{inet_aton()}.
\end{funcdesc}

\begin{funcdesc}{inet_ntoa}{packed_ip}
Convert a 32-bit packed IP address (a string four characters in
length) to its standard dotted-quad string representation
(e.g. '123.45.67.89').

Useful when conversing with a program that uses the standard C library
and needs objects of type \ctype{struct in_addr}, which is the C type
for the 32-bit packed binary this function takes as an argument.

If the string passed to this function is not exactly 4 bytes in
length, \exception{socket.error} will be raised.
\end{funcdesc}

\begin{datadesc}{SocketType}
This is a Python type object that represents the socket object type.
It is the same as \code{type(socket(...))}.
\end{datadesc}


\begin{seealso}
  \seemodule{SocketServer}{Classes that simplify writing network servers.}
\end{seealso}


\subsection{Socket Objects \label{socket-objects}}

Socket objects have the following methods.  Except for
\method{makefile()} these correspond to \UNIX{} system calls
applicable to sockets.

\begin{methoddesc}[socket]{accept}{}
Accept a connection.
The socket must be bound to an address and listening for connections.
The return value is a pair \code{(\var{conn}, \var{address})}
where \var{conn} is a \emph{new} socket object usable to send and
receive data on the connection, and \var{address} is the address bound
to the socket on the other end of the connection.
\end{methoddesc}

\begin{methoddesc}[socket]{bind}{address}
Bind the socket to \var{address}.  The socket must not already be bound.
(The format of \var{address} depends on the address family --- see
above.)  \strong{Note:}  This method has historically accepted a pair
of parameters for \constant{AF_INET} addresses instead of only a
tuple.  This was never intentional and will no longer be available in
Python 1.7.
\end{methoddesc}

\begin{methoddesc}[socket]{close}{}
Close the socket.  All future operations on the socket object will fail.
The remote end will receive no more data (after queued data is flushed).
Sockets are automatically closed when they are garbage-collected.
\end{methoddesc}

\begin{methoddesc}[socket]{connect}{address}
Connect to a remote socket at \var{address}.
(The format of \var{address} depends on the address family --- see
above.)  \strong{Note:}  This method has historically accepted a pair
of parameters for \constant{AF_INET} addresses instead of only a
tuple.  This was never intentional and will no longer be available in
Python 1.7.
\end{methoddesc}

\begin{methoddesc}[socket]{connect_ex}{address}
Like \code{connect(\var{address})}, but return an error indicator
instead of raising an exception for errors returned by the C-level
\cfunction{connect()} call (other problems, such as ``host not found,''
can still raise exceptions).  The error indicator is \code{0} if the
operation succeeded, otherwise the value of the \cdata{errno}
variable.  This is useful, e.g., for asynchronous connects.
\strong{Note:}  This method has historically accepted a pair of
parameters for \constant{AF_INET} addresses instead of only a tuple.
This was never intentional and will no longer be available in Python
1.7.
\end{methoddesc}

\begin{methoddesc}[socket]{fileno}{}
Return the socket's file descriptor (a small integer).  This is useful
with \function{select.select()}.
\end{methoddesc}

\begin{methoddesc}[socket]{getpeername}{}
Return the remote address to which the socket is connected.  This is
useful to find out the port number of a remote IP socket, for instance.
(The format of the address returned depends on the address family ---
see above.)  On some systems this function is not supported.
\end{methoddesc}

\begin{methoddesc}[socket]{getsockname}{}
Return the socket's own address.  This is useful to find out the port
number of an IP socket, for instance.
(The format of the address returned depends on the address family ---
see above.)
\end{methoddesc}

\begin{methoddesc}[socket]{getsockopt}{level, optname\optional{, buflen}}
Return the value of the given socket option (see the \UNIX{} man page
\manpage{getsockopt}{2}).  The needed symbolic constants
(\constant{SO_*} etc.) are defined in this module.  If \var{buflen}
is absent, an integer option is assumed and its integer value
is returned by the function.  If \var{buflen} is present, it specifies
the maximum length of the buffer used to receive the option in, and
this buffer is returned as a string.  It is up to the caller to decode
the contents of the buffer (see the optional built-in module
\refmodule{struct} for a way to decode C structures encoded as strings).
\end{methoddesc}

\begin{methoddesc}[socket]{listen}{backlog}
Listen for connections made to the socket.  The \var{backlog} argument
specifies the maximum number of queued connections and should be at
least 1; the maximum value is system-dependent (usually 5).
\end{methoddesc}

\begin{methoddesc}[socket]{makefile}{\optional{mode\optional{, bufsize}}}
Return a \dfn{file object} associated with the socket.  (File objects
are described in \ref{bltin-file-objects}, ``File Objects.'')
The file object references a \cfunction{dup()}ped version of the
socket file descriptor, so the file object and socket object may be
closed or garbage-collected independently.
\index{I/O control!buffering}The optional \var{mode}
and \var{bufsize} arguments are interpreted the same way as by the
built-in \function{open()} function.
\end{methoddesc}

\begin{methoddesc}[socket]{recv}{bufsize\optional{, flags}}
Receive data from the socket.  The return value is a string representing
the data received.  The maximum amount of data to be received
at once is specified by \var{bufsize}.  See the \UNIX{} manual page
\manpage{recv}{2} for the meaning of the optional argument
\var{flags}; it defaults to zero.
\end{methoddesc}

\begin{methoddesc}[socket]{recvfrom}{bufsize\optional{, flags}}
Receive data from the socket.  The return value is a pair
\code{(\var{string}, \var{address})} where \var{string} is a string
representing the data received and \var{address} is the address of the
socket sending the data.  The optional \var{flags} argument has the
same meaning as for \method{recv()} above.
(The format of \var{address} depends on the address family --- see above.)
\end{methoddesc}

\begin{methoddesc}[socket]{send}{string\optional{, flags}}
Send data to the socket.  The socket must be connected to a remote
socket.  The optional \var{flags} argument has the same meaning as for
\method{recv()} above.  Returns the number of bytes sent.
\end{methoddesc}

\begin{methoddesc}[socket]{sendto}{string\optional{, flags}, address}
Send data to the socket.  The socket should not be connected to a
remote socket, since the destination socket is specified by
\var{address}.  The optional \var{flags} argument has the same
meaning as for \method{recv()} above.  Return the number of bytes sent.
(The format of \var{address} depends on the address family --- see above.)
\end{methoddesc}

\begin{methoddesc}[socket]{setblocking}{flag}
Set blocking or non-blocking mode of the socket: if \var{flag} is 0,
the socket is set to non-blocking, else to blocking mode.  Initially
all sockets are in blocking mode.  In non-blocking mode, if a
\method{recv()} call doesn't find any data, or if a
\method{send()} call can't immediately dispose of the data, a
\exception{error} exception is raised; in blocking mode, the calls
block until they can proceed.
\end{methoddesc}

\begin{methoddesc}[socket]{setsockopt}{level, optname, value}
Set the value of the given socket option (see the \UNIX{} manual page
\manpage{setsockopt}{2}).  The needed symbolic constants are defined in
the \module{socket} module (\code{SO_*} etc.).  The value can be an
integer or a string representing a buffer.  In the latter case it is
up to the caller to ensure that the string contains the proper bits
(see the optional built-in module
\refmodule{struct}\refbimodindex{struct} for a way to encode C
structures as strings). 
\end{methoddesc}

\begin{methoddesc}[socket]{shutdown}{how}
Shut down one or both halves of the connection.  If \var{how} is
\code{0}, further receives are disallowed.  If \var{how} is \code{1},
further sends are disallowed.  If \var{how} is \code{2}, further sends
and receives are disallowed.
\end{methoddesc}

Note that there are no methods \method{read()} or \method{write()};
use \method{recv()} and \method{send()} without \var{flags} argument
instead.


\subsection{Example \label{socket-example}}

Here are two minimal example programs using the TCP/IP protocol:\ a
server that echoes all data that it receives back (servicing only one
client), and a client using it.  Note that a server must perform the
sequence \function{socket()}, \method{bind()}, \method{listen()},
\method{accept()} (possibly repeating the \method{accept()} to service
more than one client), while a client only needs the sequence
\function{socket()}, \method{connect()}.  Also note that the server
does not \method{send()}/\method{recv()} on the 
socket it is listening on but on the new socket returned by
\method{accept()}.

\begin{verbatim}
# Echo server program
import socket

HOST = ''                 # Symbolic name meaning the local host
PORT = 50007              # Arbitrary non-privileged port
s = socket.socket(socket.AF_INET, socket.SOCK_STREAM)
s.bind((HOST, PORT))
s.listen(1)
conn, addr = s.accept()
print 'Connected by', addr
while 1:
    data = conn.recv(1024)
    if not data: break
    conn.send(data)
conn.close()
\end{verbatim}

\begin{verbatim}
# Echo client program
import socket

HOST = 'daring.cwi.nl'    # The remote host
PORT = 50007              # The same port as used by the server
s = socket.socket(socket.AF_INET, socket.SOCK_STREAM)
s.connect((HOST, PORT))
s.send('Hello, world')
data = s.recv(1024)
s.close()
print 'Received', `data`
\end{verbatim}

\section{Built-in Module \sectcode{select}}
\label{module-select}
\bimodindex{select}

This module provides access to the function \code{select} available in
most \UNIX{} versions.  It defines the following:

\setindexsubitem{(in module select)}
\begin{excdesc}{error}
The exception raised when an error occurs.  The accompanying value is
a pair containing the numeric error code from \code{errno} and the
corresponding string, as would be printed by the C function
\code{perror()}.
\end{excdesc}

\begin{funcdesc}{select}{iwtd, owtd, ewtd\optional{, timeout}}
This is a straightforward interface to the \UNIX{} \code{select()}
system call.  The first three arguments are lists of `waitable
objects': either integers representing \UNIX{} file descriptors or
objects with a parameterless method named \code{fileno()} returning
such an integer.  The three lists of waitable objects are for input,
output and `exceptional conditions', respectively.  Empty lists are
allowed.  The optional \var{timeout} argument specifies a time-out as a
floating point number in seconds.  When the \var{timeout} argument
is omitted the function blocks until at least one file descriptor is
ready.  A time-out value of zero specifies a poll and never blocks.

The return value is a triple of lists of objects that are ready:
subsets of the first three arguments.  When the time-out is reached
without a file descriptor becoming ready, three empty lists are
returned.

Amongst the acceptable object types in the lists are Python file
objects (e.g. \code{sys.stdin}, or objects returned by \code{open()}
or \code{posix.popen()}), socket objects returned by
\code{socket.socket()}, and the module \code{stdwin} which happens to
define a function \code{fileno()} for just this purpose.  You may
also define a \dfn{wrapper} class yourself, as long as it has an
appropriate \code{fileno()} method (that really returns a \UNIX{} file
descriptor, not just a random integer).
\end{funcdesc}
\ttindex{socket}
\ttindex{stdwin}

\section{Built-in Module \sectcode{thread}}
\label{module-thread}
\bimodindex{thread}

This module provides low-level primitives for working with multiple
threads (a.k.a.\ \dfn{light-weight processes} or \dfn{tasks}) --- multiple
threads of control sharing their global data space.  For
synchronization, simple locks (a.k.a.\ \dfn{mutexes} or \dfn{binary
semaphores}) are provided.
\index{light-weight processes}
\index{processes, light-weight}
\index{binary semaphores}
\index{semaphores, binary}

The module is optional.  It is supported on Windows NT and '95, SGI
IRIX, Solaris 2.x, as well as on systems that have a POSIX thread
(a.k.a. ``pthread'') implementation.
\index{pthreads}
\indexii{threads}{posix}

It defines the following constant and functions:

\renewcommand{\indexsubitem}{(in module thread)}
\begin{excdesc}{error}
Raised on thread-specific errors.
\end{excdesc}

\begin{funcdesc}{start_new_thread}{func\, arg}
Start a new thread.  The thread executes the function \var{func}
with the argument list \var{arg} (which must be a tuple).  When the
function returns, the thread silently exits.  When the function
terminates with an unhandled exception, a stack trace is printed and
then the thread exits (but other threads continue to run).
\end{funcdesc}

\begin{funcdesc}{exit}{}
This is a shorthand for \code{thread.exit_thread()}.
\end{funcdesc}

\begin{funcdesc}{exit_thread}{}
Raise the \code{SystemExit} exception.  When not caught, this will
cause the thread to exit silently.
\end{funcdesc}

%\begin{funcdesc}{exit_prog}{status}
%Exit all threads and report the value of the integer argument
%\var{status} as the exit status of the entire program.
%\strong{Caveat:} code in pending \code{finally} clauses, in this thread
%or in other threads, is not executed.
%\end{funcdesc}

\begin{funcdesc}{allocate_lock}{}
Return a new lock object.  Methods of locks are described below.  The
lock is initially unlocked.
\end{funcdesc}

\begin{funcdesc}{get_ident}{}
Return the `thread identifier' of the current thread.  This is a
nonzero integer.  Its value has no direct meaning; it is intended as a
magic cookie to be used e.g. to index a dictionary of thread-specific
data.  Thread identifiers may be recycled when a thread exits and
another thread is created.
\end{funcdesc}

Lock objects have the following methods:

\renewcommand{\indexsubitem}{(lock method)}
\begin{funcdesc}{acquire}{\optional{waitflag}}
Without the optional argument, this method acquires the lock
unconditionally, if necessary waiting until it is released by another
thread (only one thread at a time can acquire a lock --- that's their
reason for existence), and returns \code{None}.  If the integer
\var{waitflag} argument is present, the action depends on its value:\
if it is zero, the lock is only acquired if it can be acquired
immediately without waiting, while if it is nonzero, the lock is
acquired unconditionally as before.  If an argument is present, the
return value is 1 if the lock is acquired successfully, 0 if not.
\end{funcdesc}

\begin{funcdesc}{release}{}
Releases the lock.  The lock must have been acquired earlier, but not
necessarily by the same thread.
\end{funcdesc}

\begin{funcdesc}{locked}{}
Return the status of the lock:\ 1 if it has been acquired by some
thread, 0 if not.
\end{funcdesc}

\strong{Caveats:}

\begin{itemize}
\item
Threads interact strangely with interrupts: the
\code{KeyboardInterrupt} exception will be received by an arbitrary
thread.  (When the \code{signal}\refbimodindex{signal} module is
available, interrupts always go to the main thread.)

\item
Calling \code{sys.exit()} or raising the \code{SystemExit} exception is
equivalent to calling \code{thread.exit_thread()}.

\item
Not all built-in functions that may block waiting for I/O allow other
threads to run.  (The most popular ones (\code{sleep()}, \code{read()},
\code{select()}) work as expected.)

\item
It is not possible to interrupt the \code{acquire()} method on a lock
-- the \code{KeyboardInterrupt} exception will happen after the lock
has been acquired.

\item
When the main thread exits, it is system defined whether the other
threads survive.  On SGI IRIX using the native thread implementation,
they survive.  On most other systems, they are killed without
executing ``try-finally'' clauses or executing object destructors.
\indexii{threads}{IRIX}

\item
When the main thread exits, it doesn't do any of its usual cleanup
(except that ``try-finally'' clauses are honored), and the standard
I/O files are not flushed.

\end{itemize}


\chapter{UNIX ONLY}

The modules described in this chapter provide interfaces to features
that are unique to the \UNIX{} operating system, or in some cases to
some or many variants of it.
			% UNIX Specific Services
\section{\module{posix} ---
         The most common \POSIX{} system calls}

\declaremodule{builtin}{posix}
  \platform{Unix}
\modulesynopsis{The most common \POSIX\ system calls (normally used
                via module \refmodule{os}).}


This module provides access to operating system functionality that is
standardized by the C Standard and the \POSIX{} standard (a thinly
disguised \UNIX{} interface).

\strong{Do not import this module directly.}  Instead, import the
module \refmodule{os}, which provides a \emph{portable} version of this
interface.  On \UNIX, the \refmodule{os} module provides a superset of
the \module{posix} interface.  On non-\UNIX{} operating systems the
\module{posix} module is not available, but a subset is always
available through the \refmodule{os} interface.  Once \refmodule{os} is
imported, there is \emph{no} performance penalty in using it instead
of \module{posix}.  In addition, \refmodule{os}\refstmodindex{os}
provides some additional functionality, such as automatically calling
\function{putenv()} when an entry in \code{os.environ} is changed.

The descriptions below are very terse; refer to the corresponding
\UNIX{} manual (or \POSIX{} documentation) entry for more information.
Arguments called \var{path} refer to a pathname given as a string.

Errors are reported as exceptions; the usual exceptions are given for
type errors, while errors reported by the system calls raise
\exception{error} (a synonym for the standard exception
\exception{OSError}), described below.


\subsection{Large File Support \label{posix-large-files}}
\sectionauthor{Steve Clift}{clift@mail.anacapa.net}
\index{large files}
\index{file!large files}


Several operating systems (including AIX, HPUX, Irix and Solaris)
provide support for files that are larger than 2 Gb from a C
programming model where \ctype{int} and \ctype{long} are 32-bit
values. This is typically accomplished by defining the relevant size
and offset types as 64-bit values. Such files are sometimes referred
to as \dfn{large files}.

Large file support is enabled in Python when the size of an
\ctype{off_t} is larger than a \ctype{long} and the \ctype{long long}
type is available and is at least as large as an \ctype{off_t}. Python
longs are then used to represent file sizes, offsets and other values
that can exceed the range of a Python int. It may be necessary to
configure and compile Python with certain compiler flags to enable
this mode. For example, it is enabled by default with recent versions
of Irix, but with Solaris 2.6 and 2.7 you need to do something like:

\begin{verbatim}
CFLAGS="`getconf LFS_CFLAGS`" OPT="-g -O2 $CFLAGS" \
        ./configure
\end{verbatim} % $ <-- bow to font-lock

On large-file-capable Linux systems, this might work:

\begin{verbatim}
CFLAGS='-D_LARGEFILE64_SOURCE -D_FILE_OFFSET_BITS=64' OPT="-g -O2 $CFLAGS" \
        ./configure
\end{verbatim} % $ <-- bow to font-lock


\subsection{Module Contents \label{posix-contents}}


Module \module{posix} defines the following data item:

\begin{datadesc}{environ}
A dictionary representing the string environment at the time the
interpreter was started. For example, \code{environ['HOME']} is the
pathname of your home directory, equivalent to
\code{getenv("HOME")} in C.

Modifying this dictionary does not affect the string environment
passed on by \function{execv()}, \function{popen()} or
\function{system()}; if you need to change the environment, pass
\code{environ} to \function{execve()} or add variable assignments and
export statements to the command string for \function{system()} or
\function{popen()}.

\note{The \refmodule{os} module provides an alternate
implementation of \code{environ} which updates the environment on
modification.  Note also that updating \code{os.environ} will render
this dictionary obsolete.  Use of the \refmodule{os} module version of
this is recommended over direct access to the \module{posix} module.}
\end{datadesc}

Additional contents of this module should only be accessed via the
\refmodule{os} module; refer to the documentation for that module for
further information.

\section{Standard Module \sectcode{posixpath}}
\label{module-posixpath}
\stmodindex{posixpath}

This module implements some useful functions on \POSIX{} pathnames.

\strong{Do not import this module directly.}  Instead, import the
module \code{os} and use \code{os.path}.
\refstmodindex{os}

\setindexsubitem{(in module posixpath)}

\begin{funcdesc}{basename}{p}
Return the base name of pathname
\var{p}.
This is the second half of the pair returned by
\code{posixpath.split(\var{p})}.
\end{funcdesc}

\begin{funcdesc}{commonprefix}{list}
Return the longest string that is a prefix of all strings in
\var{list}.
If
\var{list}
is empty, return the empty string (\code{''}).
\end{funcdesc}

\begin{funcdesc}{exists}{p}
Return true if
\var{p}
refers to an existing path.
\end{funcdesc}

\begin{funcdesc}{expanduser}{p}
Return the argument with an initial component of \samp{\~} or
\samp{\~\var{user}} replaced by that \var{user}'s home directory.  An
initial \samp{\~{}} is replaced by the environment variable \code{\${}HOME};
an initial \samp{\~\var{user}} is looked up in the password directory through
the built-in module \code{pwd}.  If the expansion fails, or if the
path does not begin with a tilde, the path is returned unchanged.
\refbimodindex{pwd}
\end{funcdesc}

\begin{funcdesc}{expandvars}{p}
Return the argument with environment variables expanded.  Substrings
of the form \samp{\$\var{name}} or \samp{\$\{\var{name}\}} are
replaced by the value of environment variable \var{name}.  Malformed
variable names and references to non-existing variables are left
unchanged.
\end{funcdesc}

\begin{funcdesc}{isabs}{p}
Return true if \var{p} is an absolute pathname (begins with a slash).
\end{funcdesc}

\begin{funcdesc}{isfile}{p}
Return true if \var{p} is an existing regular file.  This follows
symbolic links, so both \code{islink()} and \code{isfile()} can be
true for the same path.
\end{funcdesc}

\begin{funcdesc}{isdir}{p}
Return true if \var{p} is an existing directory.  This follows
symbolic links, so both \code{islink()} and \code{isdir()} can be true
for the same path.
\end{funcdesc}

\begin{funcdesc}{islink}{p}
Return true if
\var{p}
refers to a directory entry that is a symbolic link.
Always false if symbolic links are not supported.
\end{funcdesc}

\begin{funcdesc}{ismount}{p}
Return true if pathname \var{p} is a \dfn{mount point}: a point in a
file system where a different file system has been mounted.  The
function checks whether \var{p}'s parent, \file{\var{p}/..}, is on a
different device than \var{p}, or whether \file{\var{p}/..} and
\var{p} point to the same i-node on the same device --- this should
detect mount points for all \UNIX{} and \POSIX{} variants.
\end{funcdesc}

\begin{funcdesc}{join}{p\optional{\, q\optional{\, ...}}}
Joins one or more path components intelligently.  If any component is
an absolute path, all previous components are thrown away, and joining
continues.  The return value is the concatenation of \var{p}, and
optionally \var{q}, etc., with exactly one slash (\code{'/'}) inserted
between components, unless \var{p} is empty.
\end{funcdesc}

\begin{funcdesc}{normcase}{p}
Normalize the case of a pathname.  This returns the path unchanged;
however, a similar function in \code{macpath} converts upper case to
lower case.
\end{funcdesc}

\begin{funcdesc}{samefile}{p\, q}
Return true if both pathname arguments refer to the same file or directory
(as indicated by device number and i-node number).
Raise an exception if a \code{stat()} call on either pathname fails.
\end{funcdesc}

\begin{funcdesc}{split}{p}
Split the pathname \var{p} in a pair \code{(\var{head}, \var{tail})},
where \var{tail} is the last pathname component and \var{head} is
everything leading up to that.  The \var{tail} part will never contain
a slash; if \var{p} ends in a slash, \var{tail} will be empty.  If
there is no slash in \var{p}, \var{head} will be empty.  If \var{p} is
empty, both \var{head} and \var{tail} are empty.  Trailing slashes are
stripped from \var{head} unless it is the root (one or more slashes
only).  In nearly all cases, \code{join(\var{head}, \var{tail})}
equals \var{p} (the only exception being when there were multiple
slashes separating \var{head} from \var{tail}).
\end{funcdesc}

\begin{funcdesc}{splitext}{p}
Split the pathname \var{p} in a pair \code{(\var{root}, \var{ext})}
such that \code{\var{root} + \var{ext} == \var{p}},
and \var{ext} is empty or begins with a period and contains
at most one period.
\end{funcdesc}

\begin{funcdesc}{walk}{p\, visit\, arg}
Calls the function \var{visit} with arguments
\code{(\var{arg}, \var{dirname}, \var{names})} for each directory in the
directory tree rooted at \var{p} (including \var{p} itself, if it is a
directory).  The argument \var{dirname} specifies the visited directory,
the argument \var{names} lists the files in the directory (gotten from
\code{posix.listdir(\var{dirname})}).
The \var{visit} function may modify \var{names} to
influence the set of directories visited below \var{dirname}, e.g., to
avoid visiting certain parts of the tree.  (The object referred to by
\var{names} must be modified in place, using \code{del} or slice
assignment.)
\end{funcdesc}
		% == posixpath
\section{\module{pwd} ---
         The password database}

\declaremodule{builtin}{pwd}
  \platform{Unix}
\modulesynopsis{The password database (\function{getpwnam()} and friends).}

This module provides access to the \UNIX{} user account and password
database.  It is available on all \UNIX{} versions.

Password database entries are reported as a tuple-like object, whose
attributes correspond to the members of the \code{passwd} structure
(Attribute field below, see \code{<pwd.h>}):

\begin{tableiii}{r|l|l}{textrm}{Index}{Attribute}{Meaning}
  \lineiii{0}{\code{pw_name}}{Login name}
  \lineiii{1}{\code{pw_passwd}}{Optional encrypted password}
  \lineiii{2}{\code{pw_uid}}{Numerical user ID}
  \lineiii{3}{\code{pw_gid}}{Numerical group ID}
  \lineiii{4}{\code{pw_gecos}}{User name or comment field}
  \lineiii{5}{\code{pw_dir}}{User home directory}
  \lineiii{6}{\code{pw_shell}}{User command interpreter}
\end{tableiii}

The uid and gid items are integers, all others are strings.
\exception{KeyError} is raised if the entry asked for cannot be found.

\note{In traditional \UNIX{} the field \code{pw_passwd} usually
contains a password encrypted with a DES derived algorithm (see module
\refmodule{crypt}\refbimodindex{crypt}).  However most modern unices 
use a so-called \emph{shadow password} system.  On those unices the
\var{pw_passwd} field only contains an asterisk (\code{'*'}) or the 
letter \character{x} where the encrypted password is stored in a file
\file{/etc/shadow} which is not world readable.  Whether the \var{pw_passwd}
field contains anything useful is system-dependent.}

It defines the following items:

\begin{funcdesc}{getpwuid}{uid}
Return the password database entry for the given numeric user ID.
\end{funcdesc}

\begin{funcdesc}{getpwnam}{name}
Return the password database entry for the given user name.
\end{funcdesc}

\begin{funcdesc}{getpwall}{}
Return a list of all available password database entries, in arbitrary order.
\end{funcdesc}


\begin{seealso}
  \seemodule{grp}{An interface to the group database, similar to this.}
\end{seealso}

\section{Built-in Module \sectcode{grp}}

\bimodindex{grp}
This module provides access to the \UNIX{} group database.
It is available on all \UNIX{} versions.

Group database entries are reported as 4-tuples containing the
following items from the group database (see \file{<grp.h>}), in order:
\code{gr_name},
\code{gr_passwd},
\code{gr_gid},
\code{gr_mem}.
The gid is an integer, name and password are strings, and the member
list is a list of strings.
(Note that most users are not explicitly listed as members of the
group they are in according to the password database.)
An exception is raised if the entry asked for cannot be found.

It defines the following items:

\renewcommand{\indexsubitem}{(in module grp)}
\begin{funcdesc}{getgrgid}{gid}
Return the group database entry for the given numeric group ID.
\end{funcdesc}

\begin{funcdesc}{getgrnam}{name}
Return the group database entry for the given group name.
\end{funcdesc}

\begin{funcdesc}{getgrall}{}
Return a list of all available group entries, in arbitrary order.
\end{funcdesc}

\section{\module{dbm} ---
         Simple ``database'' interface}

\declaremodule{builtin}{dbm}
  \platform{Unix}
\modulesynopsis{The standard ``database'' interface, based on ndbm.}


The \module{dbm} module provides an interface to the \UNIX{}
\code{(n)dbm} library.  Dbm objects behave like mappings
(dictionaries), except that keys and values are always strings.
Printing a dbm object doesn't print the keys and values, and the
\method{items()} and \method{values()} methods are not supported.

See also the \refmodule{gdbm}\refbimodindex{gdbm} module, which
provides a similar interface using the GNU GDBM library.

The module defines the following constant and functions:

\begin{excdesc}{error}
Raised on dbm-specific errors, such as I/O errors.
\exception{KeyError} is raised for general mapping errors like
specifying an incorrect key.
\end{excdesc}

\begin{funcdesc}{open}{filename, \optional{flag, \optional{mode}}}
Open a dbm database and return a dbm object.  The \var{filename}
argument is the name of the database file (without the \file{.dir} or
\file{.pag} extensions).

The optional \var{flag} argument can be
\code{'r'} (to open an existing database for reading only --- default),
\code{'w'} (to open an existing database for reading and writing),
\code{'c'} (which creates the database if it doesn't exist), or
\code{'n'} (which always creates a new empty database).

The optional \var{mode} argument is the \UNIX{} mode of the file, used
only when the database has to be created.  It defaults to octal
\code{0666}.
\end{funcdesc}


\begin{seealso}
  \seemodule{anydbm}{Generic interface to \code{dbm}-style databases.}
  \seemodule{whichdb}{Utility module used to determine the type of an
                      existing database.}
\end{seealso}

\section{\module{gdbm} ---
         GNU's reinterpretation of dbm}

\declaremodule{builtin}{gdbm}
  \platform{Unix}
\modulesynopsis{GNU's reinterpretation of dbm.}


This module is quite similar to the \refmodule{dbm}\refbimodindex{dbm}
module, but uses \code{gdbm} instead to provide some additional
functionality.  Please note that the file formats created by
\code{gdbm} and \code{dbm} are incompatible.

The \module{gdbm} module provides an interface to the GNU DBM
library.  \code{gdbm} objects behave like mappings
(dictionaries), except that keys and values are always strings.
Printing a \code{gdbm} object doesn't print the keys and values, and
the \method{items()} and \method{values()} methods are not supported.

The module defines the following constant and functions:

\begin{excdesc}{error}
Raised on \code{gdbm}-specific errors, such as I/O errors.
\exception{KeyError} is raised for general mapping errors like
specifying an incorrect key.
\end{excdesc}

\begin{funcdesc}{open}{filename, \optional{flag, \optional{mode}}}
Open a \code{gdbm} database and return a \code{gdbm} object.  The
\var{filename} argument is the name of the database file.

The optional \var{flag} argument can be
\code{'r'} (to open an existing database for reading only --- default),
\code{'w'} (to open an existing database for reading and writing),
\code{'c'} (which creates the database if it doesn't exist), or
\code{'n'} (which always creates a new empty database).

The following additional characters may be appended to the flag to
control how the database is opened:

\begin{itemize}
\item \code{'f'} --- Open the database in fast mode.  Writes to the database
                     will not be synchronized.
\item \code{'s'} --- Synchronized mode. This will cause changes to the database
                     will be immediately written to the file.
\item \code{'u'} --- Do not lock database. 
\end{itemize}

Not all flags are valid for all versions of \code{gdbm}.  The
module constant \code{open_flags} is a string of supported flag
characters.  The exception \exception{error} is raised if an invalid
flag is specified.

The optional \var{mode} argument is the \UNIX{} mode of the file, used
only when the database has to be created.  It defaults to octal
\code{0666}.
\end{funcdesc}

In addition to the dictionary-like methods, \code{gdbm} objects have the
following methods:

\begin{funcdesc}{firstkey}{}
It's possible to loop over every key in the database using this method 
and the \method{nextkey()} method.  The traversal is ordered by
\code{gdbm}'s internal hash values, and won't be sorted by the key
values.  This method returns the starting key.
\end{funcdesc}

\begin{funcdesc}{nextkey}{key}
Returns the key that follows \var{key} in the traversal.  The
following code prints every key in the database \code{db}, without
having to create a list in memory that contains them all:

\begin{verbatim}
k = db.firstkey()
while k != None:
    print k
    k = db.nextkey(k)
\end{verbatim}
\end{funcdesc}

\begin{funcdesc}{reorganize}{}
If you have carried out a lot of deletions and would like to shrink
the space used by the \code{gdbm} file, this routine will reorganize
the database.  \code{gdbm} will not shorten the length of a database
file except by using this reorganization; otherwise, deleted file
space will be kept and reused as new (key, value) pairs are added.
\end{funcdesc}

\begin{funcdesc}{sync}{}
When the database has been opened in fast mode, this method forces any 
unwritten data to be written to the disk.
\end{funcdesc}


\begin{seealso}
  \seemodule{anydbm}{Generic interface to \code{dbm}-style databases.}
  \seemodule{whichdb}{Utility module used to determine the type of an
                      existing database.}
\end{seealso}

\section{Built-in Module \sectcode{termios}}

To be provided.

% Manual text by Jaap Vermeulen
\section{Built-in Module \sectcode{fcntl}}
\bimodindex{fcntl}
\indexii{UNIX@\UNIX{}}{file control}
\indexii{UNIX@\UNIX{}}{I/O control}

This module performs file control and I/O control on file descriptors.
It is an interface to the \dfn{fcntl()} and \dfn{ioctl()} \UNIX{} routines.
File descriptors can be obtained with the \dfn{fileno()} method of a
file or socket object.

The module defines the following functions:

\renewcommand{\indexsubitem}{(in module struct)}

\begin{funcdesc}{fcntl}{fd\, op\optional{\, arg}}
  Perform the requested operation on file descriptor \code{\var{fd}}.
  The operation is defined by \code{\var{op}} and is operating system
  dependent.  Typically these codes can be retrieved from the library
  module \code{FCNTL}. The argument \code{\var{arg}} is optional, and
  defaults to the integer value \code{0}.  When
  it is present, it can either be an integer value, or a string.  With
  the argument missing or an integer value, the return value of this
  function is the integer return value of the real \code{fcntl()}
  call.  When the argument is a string it represents a binary
  structure, e.g.\ created by \code{struct.pack()}. The binary data is
  copied to a buffer whose address is passed to the real \code{fcntl()}
  call.  The return value after a successful call is the contents of
  the buffer, converted to a string object.  In case the
  \code{fcntl()} fails, an \code{IOError} will be raised.
\end{funcdesc}

\begin{funcdesc}{ioctl}{fd\, op\, arg}
  This function is identical to the \code{fcntl()} function, except
  that the operations are typically defined in the library module
  \code{IOCTL}.
\end{funcdesc}

\begin{funcdesc}{flock}{fd\, op}
Perform the lock operation \var{op} on file descriptor \var{fd}.
See the \UNIX{} manual for details.  (On some systems, this function is
emulated using \code{fcntl()}.)
\end{funcdesc}

\begin{funcdesc}{lockf}{fd\, code\, \optional{len\, \optional{start\, \optional{whence}}}}
This is a wrapper around the \code{F_SETLK} and \code{F_SETLKW}
\code{fcntl()} calls.  See the \UNIX{} manual for details.
\end{funcdesc}

If the library modules \code{FCNTL} or \code{IOCTL} are missing, you
can find the opcodes in the C include files \file{sys/fcntl.h} and
\file{sys/ioctl.h}. You can create the modules yourself with the h2py
script, found in the \file{Tools/scripts} directory.
\refstmodindex{FCNTL}
\refstmodindex{IOCTL}

Examples (all on a SVR4 compliant system):

\bcode\begin{verbatim}
import struct, FCNTL

file = open(...)
rv = fcntl(file.fileno(), FCNTL.O_NDELAY, 1)

lockdata = struct.pack('hhllhh', FCNTL.F_WRLCK, 0, 0, 0, 0, 0)
rv = fcntl(file.fileno(), FCNTL.F_SETLKW, lockdata)
\end{verbatim}\ecode
%
Note that in the first example the return value variable \code{rv} will
hold an integer value; in the second example it will hold a string
value.  The structure lay-out for the \var{lockadata} variable is
system dependent -- therefore using the \code{flock()} call may be
better.

% Manual text and implementation by Jaap Vermeulen
\section{\module{posixfile} ---
         File-like objects with locking support}

\declaremodule{builtin}{posixfile}
  \platform{Unix}
\modulesynopsis{A file-like object with support for locking.}
\moduleauthor{Jaap Vermeulen}{}
\sectionauthor{Jaap Vermeulen}{}


\indexii{\POSIX}{file object}

\deprecated{1.5}{The locking operation that this module provides is
done better and more portably by the
\function{\refmodule{fcntl}.lockf()} call.
\withsubitem{(in module fcntl)}{\ttindex{lockf()}}}

This module implements some additional functionality over the built-in
file objects.  In particular, it implements file locking, control over
the file flags, and an easy interface to duplicate the file object.
The module defines a new file object, the posixfile object.  It
has all the standard file object methods and adds the methods
described below.  This module only works for certain flavors of
\UNIX, since it uses \function{fcntl.fcntl()} for file locking.%
\withsubitem{(in module fcntl)}{\ttindex{fcntl()}}

To instantiate a posixfile object, use the \function{open()} function
in the \module{posixfile} module.  The resulting object looks and
feels roughly the same as a standard file object.

The \module{posixfile} module defines the following constants:


\begin{datadesc}{SEEK_SET}
Offset is calculated from the start of the file.
\end{datadesc}

\begin{datadesc}{SEEK_CUR}
Offset is calculated from the current position in the file.
\end{datadesc}

\begin{datadesc}{SEEK_END}
Offset is calculated from the end of the file.
\end{datadesc}

The \module{posixfile} module defines the following functions:


\begin{funcdesc}{open}{filename\optional{, mode\optional{, bufsize}}}
 Create a new posixfile object with the given filename and mode.  The
 \var{filename}, \var{mode} and \var{bufsize} arguments are
 interpreted the same way as by the built-in \function{open()}
 function.
\end{funcdesc}

\begin{funcdesc}{fileopen}{fileobject}
 Create a new posixfile object with the given standard file object.
 The resulting object has the same filename and mode as the original
 file object.
\end{funcdesc}

The posixfile object defines the following additional methods:

\setindexsubitem{(posixfile method)}
\begin{funcdesc}{lock}{fmt, \optional{len\optional{, start\optional{, whence}}}}
 Lock the specified section of the file that the file object is
 referring to.  The format is explained
 below in a table.  The \var{len} argument specifies the length of the
 section that should be locked. The default is \code{0}. \var{start}
 specifies the starting offset of the section, where the default is
 \code{0}.  The \var{whence} argument specifies where the offset is
 relative to. It accepts one of the constants \constant{SEEK_SET},
 \constant{SEEK_CUR} or \constant{SEEK_END}.  The default is
 \constant{SEEK_SET}.  For more information about the arguments refer
 to the \manpage{fcntl}{2} manual page on your system.
\end{funcdesc}

\begin{funcdesc}{flags}{\optional{flags}}
 Set the specified flags for the file that the file object is referring
 to.  The new flags are ORed with the old flags, unless specified
 otherwise.  The format is explained below in a table.  Without
 the \var{flags} argument
 a string indicating the current flags is returned (this is
 the same as the \samp{?} modifier).  For more information about the
 flags refer to the \manpage{fcntl}{2} manual page on your system.
\end{funcdesc}

\begin{funcdesc}{dup}{}
 Duplicate the file object and the underlying file pointer and file
 descriptor.  The resulting object behaves as if it were newly
 opened.
\end{funcdesc}

\begin{funcdesc}{dup2}{fd}
 Duplicate the file object and the underlying file pointer and file
 descriptor.  The new object will have the given file descriptor.
 Otherwise the resulting object behaves as if it were newly opened.
\end{funcdesc}

\begin{funcdesc}{file}{}
 Return the standard file object that the posixfile object is based
 on.  This is sometimes necessary for functions that insist on a
 standard file object.
\end{funcdesc}

All methods raise \exception{IOError} when the request fails.

Format characters for the \method{lock()} method have the following
meaning:

\begin{tableii}{c|l}{samp}{Format}{Meaning}
  \lineii{u}{unlock the specified region}
  \lineii{r}{request a read lock for the specified section}
  \lineii{w}{request a write lock for the specified section}
\end{tableii}

In addition the following modifiers can be added to the format:

\begin{tableiii}{c|l|c}{samp}{Modifier}{Meaning}{Notes}
  \lineiii{|}{wait until the lock has been granted}{}
  \lineiii{?}{return the first lock conflicting with the requested lock, or
              \code{None} if there is no conflict.}{(1)} 
\end{tableiii}

\noindent
Note:

\begin{description}
\item[(1)] The lock returned is in the format \code{(\var{mode}, \var{len},
\var{start}, \var{whence}, \var{pid})} where \var{mode} is a character
representing the type of lock ('r' or 'w').  This modifier prevents a
request from being granted; it is for query purposes only.
\end{description}

Format characters for the \method{flags()} method have the following
meanings:

\begin{tableii}{c|l}{samp}{Format}{Meaning}
  \lineii{a}{append only flag}
  \lineii{c}{close on exec flag}
  \lineii{n}{no delay flag (also called non-blocking flag)}
  \lineii{s}{synchronization flag}
\end{tableii}

In addition the following modifiers can be added to the format:

\begin{tableiii}{c|l|c}{samp}{Modifier}{Meaning}{Notes}
  \lineiii{!}{turn the specified flags 'off', instead of the default 'on'}{(1)}
  \lineiii{=}{replace the flags, instead of the default 'OR' operation}{(1)}
  \lineiii{?}{return a string in which the characters represent the flags that
  are set.}{(2)}
\end{tableiii}

\noindent
Notes:

\begin{description}
\item[(1)] The \samp{!} and \samp{=} modifiers are mutually exclusive.

\item[(2)] This string represents the flags after they may have been altered
by the same call.
\end{description}

Examples:

\begin{verbatim}
import posixfile

file = posixfile.open('/tmp/test', 'w')
file.lock('w|')
...
file.lock('u')
file.close()
\end{verbatim}


\section{Standard module \sectcode{pdb}}
\stmodindex{pdb}
\index{debugging}

This module defines an interactive source code debugger for Python
programs.  It supports breakpoints and single stepping at the source
line level, inspection of stack frames, source code listing, and
evaluation of arbitrary Python code in the context of any stack frame.
It also supports post-mortem debugging and can be called under program
control.

The debugger is extensible --- it is actually defined as a class
\code{Pdb}.  The extension interface uses the (also undocumented)
modules \code{bdb} and \code{cmd}; it is currently undocumented.
\ttindex{Pdb}
\ttindex{bdb}
\ttindex{cmd}

A primitive windowing version of the debugger also exists --- this is
module \code{wdb}, which requires STDWIN.
\index{stdwin}
\ttindex{wdb}

Typical usage to run a program under control of the debugger is:

\begin{verbatim}
>>> import pdb
>>> import mymodule
>>> pdb.run('mymodule.test()')
(Pdb)
\end{verbatim}

Typical usage to inspect a crashed program is:

\begin{verbatim}
>>> import pdb
>>> import mymodule
>>> mymodule.test()
(crashes with a stack trace)
>>> pdb.pm()
(Pdb)
\end{verbatim}

The debugger's prompt is ``\code{(Pdb) }''.

The module defines the following functions; each enters the debugger
in a slightly different way:

\begin{funcdesc}{run}{statement\optional{\, globals\optional{\, locals}}}
Execute the \var{statement} (which should be a string) under debugger
control.  The debugger prompt appears before any code is executed; you
can set breakpoint and type \code{continue}, or you can step through
the statement using \code{step} or \code{next}.  The optional
\var{globals} and \var{locals} arguments specify the environment in
which the code is executed; by default the dictionary of the module
\code{__main__} is used.  (See the explanation of the \code{exec}
statement or the \code{eval()} built-in function.)
\end{funcdesc}

\begin{funcdesc}{runeval}{expression\optional{\, globals\optional{\, locals}}}
Evaluate the \var{expression} (which should be a string) under
debugger control.  When \code{runeval()} returns, it returns the value
of the expression.  Otherwise this function is similar to
\code{run()}.
\end{funcdesc}

\begin{funcdesc}{runcall}{function\optional{\, argument\, ...}}
Call the \var{function} (which should be a callable Python object, not
a string) with the given arguments.  When \code{runcall()} returns, it
returns the return value of the function call.  The debugger prompt
appears as soon as the function is entered.
\end{funcdesc}

\begin{funcdesc}{set_trace}{}
Enter the debugger at the calling stack frame.  This is useful to
hard-code a breakpoint at a given point in code, even if the code is
not otherwise being debugged.
\end{funcdesc}

\begin{funcdesc}{post_mortem}{traceback}
Enter post-mortem debugging of the given \var{traceback} object.
\end{funcdesc}

\begin{funcdesc}{pm}{}
Enter post-mortem debugging based on the traceback found in
\code{sys.last_traceback}.
\end{funcdesc}

\subsection{Debugger Commands}

The debugger recognizes the following commands.  Most commands can be
abbreviated to one or two letters; e.g. ``\code{h(elp)}'' means that
either ``\code{h}'' or ``\code{help}'' can be used to enter the help
command (but not ``\code{he}'' or ``\code{hel}'', nor ``\code{H}'' or
``\code{Help} or ``\code{HELP}'').  Arguments to commands must be
separated by whitespace (spaces or tabs).  Optional arguments are
enclosed in square brackets (``\code{[]}'')in the command syntax; the
square brackets must not be typed.  Alternatives in the command syntax
are separated by a vertical bar (``\code{|}'').

Entering a blank line repeats the last command entered.  Exception: if
the last command was a ``\code{list}'' command, the next 11 lines are
listed.

Commands that the debugger doesn't recognize are assumed to be Python
statements and are executed in the context of the program being
debugged.  Python statements can also be prefixed with an exclamation
point (``\code{!}'').  This is a powerful way to inspect the program
being debugged; it is even possible to change variables.  When an
exception occurs in such a statement, the exception name is printed
but the debugger's state is not changed.

\begin{description}

\item[{h(elp) [\var{command}]}]

Without argument, print the list of available commands.
With a \var{command} as argument, print help about that command.
``\code{help pdb}'' displays the full documentation file; if the
environment variable \code{PAGER} is defined, the file is piped
through that command instead.  Since the var{command} argument must be
an identifier, ``\code{help exec}'' gives help on the ``\code{!}''
command.

\item[{w(here)}]

Print a stack trace, with the most recent frame at the bottom.
An arrow indicates the current frame, which determines the
context of most commands.

\item[{d(own)}]

Move the current frame one level down in the stack trace
(to an older frame).

\item[{u(p)}]

Move the current frame one level up in the stack trace
(to a newer frame).

\item[{b(reak) [\var{lineno} \code{|} \var{function}]}]

With a \var{lineno} argument, set a break there in the current
file.  With a \var{function} argument, set a break at the entry of
that function.  Without argument, list all breaks.

\item[{cl(ear) [lineno]}]

With a \var{lineno} argument, clear that break in the current file.
Without argument, clear all breaks (but first ask confirmation).

\item[{s(tep)}]

Execute the current line, stop at the first possible occasion
(either in a function that is called or on the next line in the
current function).

\item[{n(ext)}]

Continue execution until the next line in the current function
is reached or it returns.  (The difference between \code{next} and
\code{step} is that \code{step} stops inside a called function, while
\code{next} executes called functions at full speed, only stopping at
the next line in the current function.)

\item[{r(eturn)}]

Continue execution until the current function returns.

\item[{c(ont(inue))}]

Continue execution, only stop when a breakpoint is encountered.

\item[{l(ist) [\var{first} [, \var{last}]]}]

List source code for the current file.
Without arguments, list 11 lines around the current line
or continue the previous listing.
With one argument, list 11 lines around at that line.
With two arguments, list the given range;
if the second argument is less than the first, it is a count.

\item[{a(rgs)}]

Print the argument list of the current function.

\item[{p \var{expression}}]

Evaluate the \var{expression} in the current context and print its
value.

\item[{[!] \var{statement}}]

Execute the (one-line) \var{statement} in the context of
the current stack frame.
The exclamation point can be omitted unless the first word
of the statement resembles a debugger command.
To set a global variable, you can prefix the assignment
command with a ``\code{global}'' command on the same line, e.g.:
\begin{verbatim}
(Pdb) global list_options; list_options = ['-l']
(Pdb)
\end{verbatim}

\item[{q(uit)}]

Quit from the debugger.
The program being executed is aborted.

\end{description}
			% The Python Debugger

\chapter{The Python Profiler \label{profile}}

\sectionauthor{James Roskind}{}

Copyright \copyright{} 1994, by InfoSeek Corporation, all rights reserved.
\index{InfoSeek Corporation}

Written by James Roskind.\footnote{
  Updated and converted to \LaTeX\ by Guido van Rossum.  The references to
  the old profiler are left in the text, although it no longer exists.}

Permission to use, copy, modify, and distribute this Python software
and its associated documentation for any purpose (subject to the
restriction in the following sentence) without fee is hereby granted,
provided that the above copyright notice appears in all copies, and
that both that copyright notice and this permission notice appear in
supporting documentation, and that the name of InfoSeek not be used in
advertising or publicity pertaining to distribution of the software
without specific, written prior permission.  This permission is
explicitly restricted to the copying and modification of the software
to remain in Python, compiled Python, or other languages (such as C)
wherein the modified or derived code is exclusively imported into a
Python module.

INFOSEEK CORPORATION DISCLAIMS ALL WARRANTIES WITH REGARD TO THIS
SOFTWARE, INCLUDING ALL IMPLIED WARRANTIES OF MERCHANTABILITY AND
FITNESS. IN NO EVENT SHALL INFOSEEK CORPORATION BE LIABLE FOR ANY
SPECIAL, INDIRECT OR CONSEQUENTIAL DAMAGES OR ANY DAMAGES WHATSOEVER
RESULTING FROM LOSS OF USE, DATA OR PROFITS, WHETHER IN AN ACTION OF
CONTRACT, NEGLIGENCE OR OTHER TORTIOUS ACTION, ARISING OUT OF OR IN
CONNECTION WITH THE USE OR PERFORMANCE OF THIS SOFTWARE.


The profiler was written after only programming in Python for 3 weeks.
As a result, it is probably clumsy code, but I don't know for sure yet
'cause I'm a beginner :-).  I did work hard to make the code run fast,
so that profiling would be a reasonable thing to do.  I tried not to
repeat code fragments, but I'm sure I did some stuff in really awkward
ways at times.  Please send suggestions for improvements to:
\email{jar@netscape.com}.  I won't promise \emph{any} support.  ...but
I'd appreciate the feedback.


\section{Introduction to the profiler}
\nodename{Profiler Introduction}

A \dfn{profiler} is a program that describes the run time performance
of a program, providing a variety of statistics.  This documentation
describes the profiler functionality provided in the modules
\module{profile} and \module{pstats}.  This profiler provides
\dfn{deterministic profiling} of any Python programs.  It also
provides a series of report generation tools to allow users to rapidly
examine the results of a profile operation.
\index{deterministic profiling}
\index{profiling, deterministic}


\section{How Is This Profiler Different From The Old Profiler?}
\nodename{Profiler Changes}

(This section is of historical importance only; the old profiler
discussed here was last seen in Python 1.1.)

The big changes from old profiling module are that you get more
information, and you pay less CPU time.  It's not a trade-off, it's a
trade-up.

To be specific:

\begin{description}

\item[Bugs removed:]
Local stack frame is no longer molested, execution time is now charged
to correct functions.

\item[Accuracy increased:]
Profiler execution time is no longer charged to user's code,
calibration for platform is supported, file reads are not done \emph{by}
profiler \emph{during} profiling (and charged to user's code!).

\item[Speed increased:]
Overhead CPU cost was reduced by more than a factor of two (perhaps a
factor of five), lightweight profiler module is all that must be
loaded, and the report generating module (\module{pstats}) is not needed
during profiling.

\item[Recursive functions support:]
Cumulative times in recursive functions are correctly calculated;
recursive entries are counted.

\item[Large growth in report generating UI:]
Distinct profiles runs can be added together forming a comprehensive
report; functions that import statistics take arbitrary lists of
files; sorting criteria is now based on keywords (instead of 4 integer
options); reports shows what functions were profiled as well as what
profile file was referenced; output format has been improved.

\end{description}


\section{Instant Users Manual \label{profile-instant}}

This section is provided for users that ``don't want to read the
manual.'' It provides a very brief overview, and allows a user to
rapidly perform profiling on an existing application.

To profile an application with a main entry point of \function{foo()},
you would add the following to your module:

\begin{verbatim}
import profile
profile.run('foo()')
\end{verbatim}

The above action would cause \function{foo()} to be run, and a series of
informative lines (the profile) to be printed.  The above approach is
most useful when working with the interpreter.  If you would like to
save the results of a profile into a file for later examination, you
can supply a file name as the second argument to the \function{run()}
function:

\begin{verbatim}
import profile
profile.run('foo()', 'fooprof')
\end{verbatim}

The file \file{profile.py} can also be invoked as
a script to profile another script.  For example:

\begin{verbatim}
python -m profile myscript.py
\end{verbatim}

\file{profile.py} accepts two optional arguments on the command line:

\begin{verbatim}
profile.py [-o output_file] [-s sort_order]
\end{verbatim}

\programopt{-s} only applies to standard output (\programopt{-o} is
not supplied).  Look in the \class{Stats} documentation for valid sort
values.

When you wish to review the profile, you should use the methods in the
\module{pstats} module.  Typically you would load the statistics data as
follows:

\begin{verbatim}
import pstats
p = pstats.Stats('fooprof')
\end{verbatim}

The class \class{Stats} (the above code just created an instance of
this class) has a variety of methods for manipulating and printing the
data that was just read into \code{p}.  When you ran
\function{profile.run()} above, what was printed was the result of three
method calls:

\begin{verbatim}
p.strip_dirs().sort_stats(-1).print_stats()
\end{verbatim}

The first method removed the extraneous path from all the module
names. The second method sorted all the entries according to the
standard module/line/name string that is printed (this is to comply
with the semantics of the old profiler).  The third method printed out
all the statistics.  You might try the following sort calls:

\begin{verbatim}
p.sort_stats('name')
p.print_stats()
\end{verbatim}

The first call will actually sort the list by function name, and the
second call will print out the statistics.  The following are some
interesting calls to experiment with:

\begin{verbatim}
p.sort_stats('cumulative').print_stats(10)
\end{verbatim}

This sorts the profile by cumulative time in a function, and then only
prints the ten most significant lines.  If you want to understand what
algorithms are taking time, the above line is what you would use.

If you were looking to see what functions were looping a lot, and
taking a lot of time, you would do:

\begin{verbatim}
p.sort_stats('time').print_stats(10)
\end{verbatim}

to sort according to time spent within each function, and then print
the statistics for the top ten functions.

You might also try:

\begin{verbatim}
p.sort_stats('file').print_stats('__init__')
\end{verbatim}

This will sort all the statistics by file name, and then print out
statistics for only the class init methods (since they are spelled
with \code{__init__} in them).  As one final example, you could try:

\begin{verbatim}
p.sort_stats('time', 'cum').print_stats(.5, 'init')
\end{verbatim}

This line sorts statistics with a primary key of time, and a secondary
key of cumulative time, and then prints out some of the statistics.
To be specific, the list is first culled down to 50\% (re: \samp{.5})
of its original size, then only lines containing \code{init} are
maintained, and that sub-sub-list is printed.

If you wondered what functions called the above functions, you could
now (\code{p} is still sorted according to the last criteria) do:

\begin{verbatim}
p.print_callers(.5, 'init')
\end{verbatim}

and you would get a list of callers for each of the listed functions.

If you want more functionality, you're going to have to read the
manual, or guess what the following functions do:

\begin{verbatim}
p.print_callees()
p.add('fooprof')
\end{verbatim}

Invoked as a script, the \module{pstats} module is a statistics
browser for reading and examining profile dumps.  It has a simple
line-oriented interface (implemented using \refmodule{cmd}) and
interactive help.

\section{What Is Deterministic Profiling?}
\nodename{Deterministic Profiling}

\dfn{Deterministic profiling} is meant to reflect the fact that all
\emph{function call}, \emph{function return}, and \emph{exception} events
are monitored, and precise timings are made for the intervals between
these events (during which time the user's code is executing).  In
contrast, \dfn{statistical profiling} (which is not done by this
module) randomly samples the effective instruction pointer, and
deduces where time is being spent.  The latter technique traditionally
involves less overhead (as the code does not need to be instrumented),
but provides only relative indications of where time is being spent.

In Python, since there is an interpreter active during execution, the
presence of instrumented code is not required to do deterministic
profiling.  Python automatically provides a \dfn{hook} (optional
callback) for each event.  In addition, the interpreted nature of
Python tends to add so much overhead to execution, that deterministic
profiling tends to only add small processing overhead in typical
applications.  The result is that deterministic profiling is not that
expensive, yet provides extensive run time statistics about the
execution of a Python program.

Call count statistics can be used to identify bugs in code (surprising
counts), and to identify possible inline-expansion points (high call
counts).  Internal time statistics can be used to identify ``hot
loops'' that should be carefully optimized.  Cumulative time
statistics should be used to identify high level errors in the
selection of algorithms.  Note that the unusual handling of cumulative
times in this profiler allows statistics for recursive implementations
of algorithms to be directly compared to iterative implementations.


\section{Reference Manual}

\declaremodule{standard}{profile}
\modulesynopsis{Python profiler}



The primary entry point for the profiler is the global function
\function{profile.run()}.  It is typically used to create any profile
information.  The reports are formatted and printed using methods of
the class \class{pstats.Stats}.  The following is a description of all
of these standard entry points and functions.  For a more in-depth
view of some of the code, consider reading the later section on
Profiler Extensions, which includes discussion of how to derive
``better'' profilers from the classes presented, or reading the source
code for these modules.

\begin{funcdesc}{run}{command\optional{, filename}}

This function takes a single argument that has can be passed to the
\keyword{exec} statement, and an optional file name.  In all cases this
routine attempts to \keyword{exec} its first argument, and gather profiling
statistics from the execution. If no file name is present, then this
function automatically prints a simple profiling report, sorted by the
standard name string (file/line/function-name) that is presented in
each line.  The following is a typical output from such a call:

\begin{verbatim}
      main()
      2706 function calls (2004 primitive calls) in 4.504 CPU seconds

Ordered by: standard name

ncalls  tottime  percall  cumtime  percall filename:lineno(function)
     2    0.006    0.003    0.953    0.477 pobject.py:75(save_objects)
  43/3    0.533    0.012    0.749    0.250 pobject.py:99(evaluate)
 ...
\end{verbatim}

The first line indicates that this profile was generated by the call:\\
\code{profile.run('main()')}, and hence the exec'ed string is
\code{'main()'}.  The second line indicates that 2706 calls were
monitored.  Of those calls, 2004 were \dfn{primitive}.  We define
\dfn{primitive} to mean that the call was not induced via recursion.
The next line: \code{Ordered by:\ standard name}, indicates that
the text string in the far right column was used to sort the output.
The column headings include:

\begin{description}

\item[ncalls ]
for the number of calls,

\item[tottime ]
for the total time spent in the given function (and excluding time
made in calls to sub-functions),

\item[percall ]
is the quotient of \code{tottime} divided by \code{ncalls}

\item[cumtime ]
is the total time spent in this and all subfunctions (from invocation
till exit). This figure is accurate \emph{even} for recursive
functions.

\item[percall ]
is the quotient of \code{cumtime} divided by primitive calls

\item[filename:lineno(function) ]
provides the respective data of each function

\end{description}

When there are two numbers in the first column (for example,
\samp{43/3}), then the latter is the number of primitive calls, and
the former is the actual number of calls.  Note that when the function
does not recurse, these two values are the same, and only the single
figure is printed.

\end{funcdesc}

\begin{funcdesc}{runctx}{command, globals, locals\optional{, filename}}
This function is similar to \function{profile.run()}, with added
arguments to supply the globals and locals dictionaries for the
\var{command} string.
\end{funcdesc}

Analysis of the profiler data is done using this class from the
\module{pstats} module:

% now switch modules....
% (This \stmodindex use may be hard to change ;-( )
\stmodindex{pstats}

\begin{classdesc}{Stats}{filename\optional{, \moreargs}}
This class constructor creates an instance of a ``statistics object''
from a \var{filename} (or set of filenames).  \class{Stats} objects are
manipulated by methods, in order to print useful reports.

The file selected by the above constructor must have been created by
the corresponding version of \module{profile}.  To be specific, there is
\emph{no} file compatibility guaranteed with future versions of this
profiler, and there is no compatibility with files produced by other
profilers (such as the old system profiler).

If several files are provided, all the statistics for identical
functions will be coalesced, so that an overall view of several
processes can be considered in a single report.  If additional files
need to be combined with data in an existing \class{Stats} object, the
\method{add()} method can be used.
\end{classdesc}


\subsection{The \class{Stats} Class \label{profile-stats}}

\class{Stats} objects have the following methods:

\begin{methoddesc}[Stats]{strip_dirs}{}
This method for the \class{Stats} class removes all leading path
information from file names.  It is very useful in reducing the size
of the printout to fit within (close to) 80 columns.  This method
modifies the object, and the stripped information is lost.  After
performing a strip operation, the object is considered to have its
entries in a ``random'' order, as it was just after object
initialization and loading.  If \method{strip_dirs()} causes two
function names to be indistinguishable (they are on the same
line of the same filename, and have the same function name), then the
statistics for these two entries are accumulated into a single entry.
\end{methoddesc}


\begin{methoddesc}[Stats]{add}{filename\optional{, \moreargs}}
This method of the \class{Stats} class accumulates additional
profiling information into the current profiling object.  Its
arguments should refer to filenames created by the corresponding
version of \function{profile.run()}.  Statistics for identically named
(re: file, line, name) functions are automatically accumulated into
single function statistics.
\end{methoddesc}

\begin{methoddesc}[Stats]{dump_stats}{filename}
Save the data loaded into the \class{Stats} object to a file named
\var{filename}.  The file is created if it does not exist, and is
overwritten if it already exists.  This is equivalent to the method of
the same name on the \class{profile.Profile} class.
\versionadded{2.3}
\end{methoddesc}

\begin{methoddesc}[Stats]{sort_stats}{key\optional{, \moreargs}}
This method modifies the \class{Stats} object by sorting it according
to the supplied criteria.  The argument is typically a string
identifying the basis of a sort (example: \code{'time'} or
\code{'name'}).

When more than one key is provided, then additional keys are used as
secondary criteria when there is equality in all keys selected
before them.  For example, \code{sort_stats('name', 'file')} will sort
all the entries according to their function name, and resolve all ties
(identical function names) by sorting by file name.

Abbreviations can be used for any key names, as long as the
abbreviation is unambiguous.  The following are the keys currently
defined:

\begin{tableii}{l|l}{code}{Valid Arg}{Meaning}
  \lineii{'calls'}{call count}
  \lineii{'cumulative'}{cumulative time}
  \lineii{'file'}{file name}
  \lineii{'module'}{file name}
  \lineii{'pcalls'}{primitive call count}
  \lineii{'line'}{line number}
  \lineii{'name'}{function name}
  \lineii{'nfl'}{name/file/line}
  \lineii{'stdname'}{standard name}
  \lineii{'time'}{internal time}
\end{tableii}

Note that all sorts on statistics are in descending order (placing
most time consuming items first), where as name, file, and line number
searches are in ascending order (alphabetical). The subtle
distinction between \code{'nfl'} and \code{'stdname'} is that the
standard name is a sort of the name as printed, which means that the
embedded line numbers get compared in an odd way.  For example, lines
3, 20, and 40 would (if the file names were the same) appear in the
string order 20, 3 and 40.  In contrast, \code{'nfl'} does a numeric
compare of the line numbers.  In fact, \code{sort_stats('nfl')} is the
same as \code{sort_stats('name', 'file', 'line')}.

For compatibility with the old profiler, the numeric arguments
\code{-1}, \code{0}, \code{1}, and \code{2} are permitted.  They are
interpreted as \code{'stdname'}, \code{'calls'}, \code{'time'}, and
\code{'cumulative'} respectively.  If this old style format (numeric)
is used, only one sort key (the numeric key) will be used, and
additional arguments will be silently ignored.
\end{methoddesc}


\begin{methoddesc}[Stats]{reverse_order}{}
This method for the \class{Stats} class reverses the ordering of the basic
list within the object.  This method is provided primarily for
compatibility with the old profiler.  Its utility is questionable
now that ascending vs descending order is properly selected based on
the sort key of choice.
\end{methoddesc}

\begin{methoddesc}[Stats]{print_stats}{\optional{restriction, \moreargs}}
This method for the \class{Stats} class prints out a report as described
in the \function{profile.run()} definition.

The order of the printing is based on the last \method{sort_stats()}
operation done on the object (subject to caveats in \method{add()} and
\method{strip_dirs()}).

The arguments provided (if any) can be used to limit the list down to
the significant entries.  Initially, the list is taken to be the
complete set of profiled functions.  Each restriction is either an
integer (to select a count of lines), or a decimal fraction between
0.0 and 1.0 inclusive (to select a percentage of lines), or a regular
expression (to pattern match the standard name that is printed; as of
Python 1.5b1, this uses the Perl-style regular expression syntax
defined by the \refmodule{re} module).  If several restrictions are
provided, then they are applied sequentially.  For example:

\begin{verbatim}
print_stats(.1, 'foo:')
\end{verbatim}

would first limit the printing to first 10\% of list, and then only
print functions that were part of filename \file{.*foo:}.  In
contrast, the command:

\begin{verbatim}
print_stats('foo:', .1)
\end{verbatim}

would limit the list to all functions having file names \file{.*foo:},
and then proceed to only print the first 10\% of them.
\end{methoddesc}


\begin{methoddesc}[Stats]{print_callers}{\optional{restriction, \moreargs}}
This method for the \class{Stats} class prints a list of all functions
that called each function in the profiled database.  The ordering is
identical to that provided by \method{print_stats()}, and the definition
of the restricting argument is also identical.  For convenience, a
number is shown in parentheses after each caller to show how many
times this specific call was made.  A second non-parenthesized number
is the cumulative time spent in the function at the right.
\end{methoddesc}

\begin{methoddesc}[Stats]{print_callees}{\optional{restriction, \moreargs}}
This method for the \class{Stats} class prints a list of all function
that were called by the indicated function.  Aside from this reversal
of direction of calls (re: called vs was called by), the arguments and
ordering are identical to the \method{print_callers()} method.
\end{methoddesc}

\begin{methoddesc}[Stats]{ignore}{}
\deprecated{1.5.1}{This is not needed in modern versions of
Python.\footnote{
  This was once necessary, when Python would print any unused expression
  result that was not \code{None}.  The method is still defined for
  backward compatibility.}}
\end{methoddesc}


\section{Limitations \label{profile-limits}}

One limitation has to do with accuracy of timing information.
There is a fundamental problem with deterministic profilers involving
accuracy.  The most obvious restriction is that the underlying ``clock''
is only ticking at a rate (typically) of about .001 seconds.  Hence no
measurements will be more accurate than the underlying clock.  If
enough measurements are taken, then the ``error'' will tend to average
out. Unfortunately, removing this first error induces a second source
of error.

The second problem is that it ``takes a while'' from when an event is
dispatched until the profiler's call to get the time actually
\emph{gets} the state of the clock.  Similarly, there is a certain lag
when exiting the profiler event handler from the time that the clock's
value was obtained (and then squirreled away), until the user's code
is once again executing.  As a result, functions that are called many
times, or call many functions, will typically accumulate this error.
The error that accumulates in this fashion is typically less than the
accuracy of the clock (less than one clock tick), but it
\emph{can} accumulate and become very significant.  This profiler
provides a means of calibrating itself for a given platform so that
this error can be probabilistically (on the average) removed.
After the profiler is calibrated, it will be more accurate (in a least
square sense), but it will sometimes produce negative numbers (when
call counts are exceptionally low, and the gods of probability work
against you :-). )  Do \emph{not} be alarmed by negative numbers in
the profile.  They should \emph{only} appear if you have calibrated
your profiler, and the results are actually better than without
calibration.


\section{Calibration \label{profile-calibration}}

The profiler subtracts a constant from each
event handling time to compensate for the overhead of calling the time
function, and socking away the results.  By default, the constant is 0.
The following procedure can
be used to obtain a better constant for a given platform (see discussion
in section Limitations above).

\begin{verbatim}
import profile
pr = profile.Profile()
for i in range(5):
    print pr.calibrate(10000)
\end{verbatim}

The method executes the number of Python calls given by the argument,
directly and again under the profiler, measuring the time for both.
It then computes the hidden overhead per profiler event, and returns
that as a float.  For example, on an 800 MHz Pentium running
Windows 2000, and using Python's time.clock() as the timer,
the magical number is about 12.5e-6.

The object of this exercise is to get a fairly consistent result.
If your computer is \emph{very} fast, or your timer function has poor
resolution, you might have to pass 100000, or even 1000000, to get
consistent results.

When you have a consistent answer,
there are three ways you can use it:\footnote{Prior to Python 2.2, it
  was necessary to edit the profiler source code to embed the bias as
  a literal number.  You still can, but that method is no longer
  described, because no longer needed.}

\begin{verbatim}
import profile

# 1. Apply computed bias to all Profile instances created hereafter.
profile.Profile.bias = your_computed_bias

# 2. Apply computed bias to a specific Profile instance.
pr = profile.Profile()
pr.bias = your_computed_bias

# 3. Specify computed bias in instance constructor.
pr = profile.Profile(bias=your_computed_bias)
\end{verbatim}

If you have a choice, you are better off choosing a smaller constant, and
then your results will ``less often'' show up as negative in profile
statistics.


\section{Extensions --- Deriving Better Profilers}
\nodename{Profiler Extensions}

The \class{Profile} class of module \module{profile} was written so that
derived classes could be developed to extend the profiler.  The details
are not described here, as doing this successfully requires an expert
understanding of how the \class{Profile} class works internally.  Study
the source code of module \module{profile} carefully if you want to
pursue this.

If all you want to do is change how current time is determined (for
example, to force use of wall-clock time or elapsed process time),
pass the timing function you want to the \class{Profile} class
constructor:

\begin{verbatim}
pr = profile.Profile(your_time_func)
\end{verbatim}

The resulting profiler will then call \code{your_time_func()}.
The function should return a single number, or a list of
numbers whose sum is the current time (like what \function{os.times()}
returns).  If the function returns a single time number, or the list of
returned numbers has length 2, then you will get an especially fast
version of the dispatch routine.

Be warned that you should calibrate the profiler class for the
timer function that you choose.  For most machines, a timer that
returns a lone integer value will provide the best results in terms of
low overhead during profiling.  (\function{os.times()} is
\emph{pretty} bad, as it returns a tuple of floating point values).  If
you want to substitute a better timer in the cleanest fashion,
derive a class and hardwire a replacement dispatch method that best
handles your timer call, along with the appropriate calibration
constant.
		% The Python Profiler

\chapter{Internet and WWW Services}
\nodename{Internet and WWW}
\label{www}
\index{WWW}
\index{Internet}
\index{World-Wide Web}

The modules described in this chapter provide various services to
World-Wide Web (WWW) clients and/or services, and a few modules
related to news and email.  They are all implemented in Python.  Some
of these modules require the presence of the system-dependent module
\code{sockets}\refbimodindex{socket}, which is currently only fully
supported on \UNIX{} and Windows NT.  Here is an overview:

\begin{description}

\item[cgi]
--- Common Gateway Interface, used to interpret forms in server-side
scripts.

\item[urllib]
--- Open an arbitrary object given by URL (requires sockets).

\item[httplib]
--- HTTP protocol client (requires sockets).

\item[ftplib]
--- FTP protocol client (requires sockets).

\item[gopherlib]
--- Gopher protocol client (requires sockets).

\item[nntplib]
--- NNTP protocol client (requires sockets).

\item[urlparse]
--- Parse a URL string into a tuple (addressing scheme identifier, network
location, path, parameters, query string, fragment identifier).

\item[sgmllib]
--- Only as much of an SGML parser as needed to parse HTML.

\item[htmllib]
--- A parser for HTML documents.

\item[xmllib]
--- A parser for XML documents.

\item[formatter]
--- Generic output formatter and device interface.

\item[rfc822]
--- Parse \rfc{822} style mail headers.

\item[mimetools]
--- Tools for parsing MIME style message bodies.

\item[binhex]
--- Encode and decode files in binhex4 format.

\item[uu]
--- Encode and decode files in uuencode format.

\item[binascii]
--- Tools for converting between binary and various ascii-encoded binary 
representation

\item[xdrlib]
--- The External Data Representation Standard as described in \rfc{1014},
written by Sun Microsystems, Inc. June 1987.

\item[mailcap]
--- Mailcap file handling.  See \rfc{1524}.

\item[base64]
--- Encode/decode binary files using the MIME base64 encoding.

\item[quopri]
--- Encode/decode binary files using the MIME quoted-printable encoding.

\item[SocketServer]
--- A framework for network servers.

\item[mailbox]
--- Read various mailbox formats.

\item[mimify]
--- Mimification and unmimification of mail messages.

\item[BaseHTTPServer]
--- Basic HTTP server (base class for SimpleHTTPServer and CGIHTTPServer).

\end{description}
			% Internet and WWW Services
\section{\module{cgi} ---
         Common Gateway Interface support.}
\declaremodule{standard}{cgi}

\modulesynopsis{Common Gateway Interface support, used to interpret
forms in server-side scripts.}

\indexii{WWW}{server}
\indexii{CGI}{protocol}
\indexii{HTTP}{protocol}
\indexii{MIME}{headers}
\index{URL}


Support module for Common Gateway Interface (CGI) scripts.%
\index{Common Gateway Interface}

This module defines a number of utilities for use by CGI scripts
written in Python.

\subsection{Introduction}
\nodename{cgi-intro}

A CGI script is invoked by an HTTP server, usually to process user
input submitted through an HTML \code{<FORM>} or \code{<ISINDEX>} element.

Most often, CGI scripts live in the server's special \file{cgi-bin}
directory.  The HTTP server places all sorts of information about the
request (such as the client's hostname, the requested URL, the query
string, and lots of other goodies) in the script's shell environment,
executes the script, and sends the script's output back to the client.

The script's input is connected to the client too, and sometimes the
form data is read this way; at other times the form data is passed via
the ``query string'' part of the URL.  This module is intended
to take care of the different cases and provide a simpler interface to
the Python script.  It also provides a number of utilities that help
in debugging scripts, and the latest addition is support for file
uploads from a form (if your browser supports it --- Grail 0.3 and
Netscape 2.0 do).

The output of a CGI script should consist of two sections, separated
by a blank line.  The first section contains a number of headers,
telling the client what kind of data is following.  Python code to
generate a minimal header section looks like this:

\begin{verbatim}
print "Content-Type: text/html"     # HTML is following
print                               # blank line, end of headers
\end{verbatim}

The second section is usually HTML, which allows the client software
to display nicely formatted text with header, in-line images, etc.
Here's Python code that prints a simple piece of HTML:

\begin{verbatim}
print "<TITLE>CGI script output</TITLE>"
print "<H1>This is my first CGI script</H1>"
print "Hello, world!"
\end{verbatim}

\subsection{Using the cgi module}
\nodename{Using the cgi module}

Begin by writing \samp{import cgi}.  Do not use \samp{from cgi import
*} --- the module defines all sorts of names for its own use or for
backward compatibility that you don't want in your namespace.

When you write a new script, consider adding the line:

\begin{verbatim}
import cgitb; cgitb.enable()
\end{verbatim}

This activates a special exception handler that will display detailed
reports in the Web browser if any errors occur.  If you'd rather not
show the guts of your program to users of your script, you can have
the reports saved to files instead, with a line like this:

\begin{verbatim}
import cgitb; cgitb.enable(display=0, logdir="/tmp")
\end{verbatim}

It's very helpful to use this feature during script development.
The reports produced by \refmodule{cgitb} provide information that
can save you a lot of time in tracking down bugs.  You can always
remove the \code{cgitb} line later when you have tested your script
and are confident that it works correctly.

To get at submitted form data,
it's best to use the \class{FieldStorage} class.  The other classes
defined in this module are provided mostly for backward compatibility.
Instantiate it exactly once, without arguments.  This reads the form
contents from standard input or the environment (depending on the
value of various environment variables set according to the CGI
standard).  Since it may consume standard input, it should be
instantiated only once.

The \class{FieldStorage} instance can be indexed like a Python
dictionary, and also supports the standard dictionary methods
\method{has_key()} and \method{keys()}.  The built-in \function{len()}
is also supported.  Form fields containing empty strings are ignored
and do not appear in the dictionary; to keep such values, provide
a true value for the optional \var{keep_blank_values} keyword
parameter when creating the \class{FieldStorage} instance.

For instance, the following code (which assumes that the 
\mailheader{Content-Type} header and blank line have already been
printed) checks that the fields \code{name} and \code{addr} are both
set to a non-empty string:

\begin{verbatim}
form = cgi.FieldStorage()
if not (form.has_key("name") and form.has_key("addr")):
    print "<H1>Error</H1>"
    print "Please fill in the name and addr fields."
    return
print "<p>name:", form["name"].value
print "<p>addr:", form["addr"].value
...further form processing here...
\end{verbatim}

Here the fields, accessed through \samp{form[\var{key}]}, are
themselves instances of \class{FieldStorage} (or
\class{MiniFieldStorage}, depending on the form encoding).
The \member{value} attribute of the instance yields the string value
of the field.  The \method{getvalue()} method returns this string value
directly; it also accepts an optional second argument as a default to
return if the requested key is not present.

If the submitted form data contains more than one field with the same
name, the object retrieved by \samp{form[\var{key}]} is not a
\class{FieldStorage} or \class{MiniFieldStorage}
instance but a list of such instances.  Similarly, in this situation,
\samp{form.getvalue(\var{key})} would return a list of strings.
If you expect this possibility
(when your HTML form contains multiple fields with the same name), use
the \function{getlist()} function, which always returns a list of values (so that you
do not need to special-case the single item case).  For example, this
code concatenates any number of username fields, separated by
commas:

\begin{verbatim}
value = form.getlist("username")
usernames = ",".join(value)
\end{verbatim}

If a field represents an uploaded file, accessing the value via the
\member{value} attribute or the \function{getvalue()} method reads the
entire file in memory as a string.  This may not be what you want.
You can test for an uploaded file by testing either the \member{filename}
attribute or the \member{file} attribute.  You can then read the data at
leisure from the \member{file} attribute:

\begin{verbatim}
fileitem = form["userfile"]
if fileitem.file:
    # It's an uploaded file; count lines
    linecount = 0
    while 1:
        line = fileitem.file.readline()
        if not line: break
        linecount = linecount + 1
\end{verbatim}

The file upload draft standard entertains the possibility of uploading
multiple files from one field (using a recursive
\mimetype{multipart/*} encoding).  When this occurs, the item will be
a dictionary-like \class{FieldStorage} item.  This can be determined
by testing its \member{type} attribute, which should be
\mimetype{multipart/form-data} (or perhaps another MIME type matching
\mimetype{multipart/*}).  In this case, it can be iterated over
recursively just like the top-level form object.

When a form is submitted in the ``old'' format (as the query string or
as a single data part of type
\mimetype{application/x-www-form-urlencoded}), the items will actually
be instances of the class \class{MiniFieldStorage}.  In this case, the
\member{list}, \member{file}, and \member{filename} attributes are
always \code{None}.


\subsection{Higher Level Interface}

\versionadded{2.2}  % XXX: Is this true ? 

The previous section explains how to read CGI form data using the
\class{FieldStorage} class.  This section describes a higher level
interface which was added to this class to allow one to do it in a
more readable and intuitive way.  The interface doesn't make the
techniques described in previous sections obsolete --- they are still
useful to process file uploads efficiently, for example.

The interface consists of two simple methods. Using the methods
you can process form data in a generic way, without the need to worry
whether only one or more values were posted under one name.

In the previous section, you learned to write following code anytime
you expected a user to post more than one value under one name:

\begin{verbatim}
item = form.getvalue("item")
if isinstance(item, list):
    # The user is requesting more than one item.
else:
    # The user is requesting only one item.
\end{verbatim}

This situation is common for example when a form contains a group of
multiple checkboxes with the same name:

\begin{verbatim}
<input type="checkbox" name="item" value="1" />
<input type="checkbox" name="item" value="2" />
\end{verbatim}

In most situations, however, there's only one form control with a
particular name in a form and then you expect and need only one value
associated with this name.  So you write a script containing for
example this code:

\begin{verbatim}
user = form.getvalue("user").upper()
\end{verbatim}

The problem with the code is that you should never expect that a
client will provide valid input to your scripts.  For example, if a
curious user appends another \samp{user=foo} pair to the query string,
then the script would crash, because in this situation the
\code{getvalue("user")} method call returns a list instead of a
string.  Calling the \method{toupper()} method on a list is not valid
(since lists do not have a method of this name) and results in an
\exception{AttributeError} exception.

Therefore, the appropriate way to read form data values was to always
use the code which checks whether the obtained value is a single value
or a list of values.  That's annoying and leads to less readable
scripts.

A more convenient approach is to use the methods \method{getfirst()}
and \method{getlist()} provided by this higher level interface.

\begin{methoddesc}[FieldStorage]{getfirst}{name\optional{, default}}
  This method always returns only one value associated with form field
  \var{name}.  The method returns only the first value in case that
  more values were posted under such name.  Please note that the order
  in which the values are received may vary from browser to browser
  and should not be counted on.\footnote{Note that some recent
      versions of the HTML specification do state what order the
      field values should be supplied in, but knowing whether a
      request was received from a conforming browser, or even from a
      browser at all, is tedious and error-prone.}  If no such form
  field or value exists then the method returns the value specified by
  the optional parameter \var{default}.  This parameter defaults to
  \code{None} if not specified.
\end{methoddesc}

\begin{methoddesc}[FieldStorage]{getlist}{name}
  This method always returns a list of values associated with form
  field \var{name}.  The method returns an empty list if no such form
  field or value exists for \var{name}.  It returns a list consisting
  of one item if only one such value exists.
\end{methoddesc}

Using these methods you can write nice compact code:

\begin{verbatim}
import cgi
form = cgi.FieldStorage()
user = form.getfirst("user", "").upper()    # This way it's safe.
for item in form.getlist("item"):
    do_something(item)
\end{verbatim}


\subsection{Old classes}

These classes, present in earlier versions of the \module{cgi} module,
are still supported for backward compatibility.  New applications
should use the \class{FieldStorage} class.

\class{SvFormContentDict} stores single value form content as
dictionary; it assumes each field name occurs in the form only once.

\class{FormContentDict} stores multiple value form content as a
dictionary (the form items are lists of values).  Useful if your form
contains multiple fields with the same name.

Other classes (\class{FormContent}, \class{InterpFormContentDict}) are
present for backwards compatibility with really old applications only.
If you still use these and would be inconvenienced when they
disappeared from a next version of this module, drop me a note.


\subsection{Functions}
\nodename{Functions in cgi module}

These are useful if you want more control, or if you want to employ
some of the algorithms implemented in this module in other
circumstances.

\begin{funcdesc}{parse}{fp\optional{, keep_blank_values\optional{,
                        strict_parsing}}}
  Parse a query in the environment or from a file (the file defaults
  to \code{sys.stdin}).  The \var{keep_blank_values} and
  \var{strict_parsing} parameters are passed to \function{parse_qs()}
  unchanged.
\end{funcdesc}

\begin{funcdesc}{parse_qs}{qs\optional{, keep_blank_values\optional{,
                           strict_parsing}}}
Parse a query string given as a string argument (data of type 
\mimetype{application/x-www-form-urlencoded}).  Data are
returned as a dictionary.  The dictionary keys are the unique query
variable names and the values are lists of values for each name.

The optional argument \var{keep_blank_values} is
a flag indicating whether blank values in
URL encoded queries should be treated as blank strings.  
A true value indicates that blanks should be retained as 
blank strings.  The default false value indicates that
blank values are to be ignored and treated as if they were
not included.

The optional argument \var{strict_parsing} is a flag indicating what
to do with parsing errors.  If false (the default), errors
are silently ignored.  If true, errors raise a ValueError
exception.

Use the \function{\refmodule{urllib}.urlencode()} function to convert
such dictionaries into query strings.

\end{funcdesc}

\begin{funcdesc}{parse_qsl}{qs\optional{, keep_blank_values\optional{,
                            strict_parsing}}}
Parse a query string given as a string argument (data of type 
\mimetype{application/x-www-form-urlencoded}).  Data are
returned as a list of name, value pairs.

The optional argument \var{keep_blank_values} is
a flag indicating whether blank values in
URL encoded queries should be treated as blank strings.  
A true value indicates that blanks should be retained as 
blank strings.  The default false value indicates that
blank values are to be ignored and treated as if they were
not included.

The optional argument \var{strict_parsing} is a flag indicating what
to do with parsing errors.  If false (the default), errors
are silently ignored.  If true, errors raise a ValueError
exception.

Use the \function{\refmodule{urllib}.urlencode()} function to convert
such lists of pairs into query strings.
\end{funcdesc}

\begin{funcdesc}{parse_multipart}{fp, pdict}
Parse input of type \mimetype{multipart/form-data} (for 
file uploads).  Arguments are \var{fp} for the input file and
\var{pdict} for a dictionary containing other parameters in
the \mailheader{Content-Type} header.

Returns a dictionary just like \function{parse_qs()} keys are the
field names, each value is a list of values for that field.  This is
easy to use but not much good if you are expecting megabytes to be
uploaded --- in that case, use the \class{FieldStorage} class instead
which is much more flexible.

Note that this does not parse nested multipart parts --- use
\class{FieldStorage} for that.
\end{funcdesc}

\begin{funcdesc}{parse_header}{string}
Parse a MIME header (such as \mailheader{Content-Type}) into a main
value and a dictionary of parameters.
\end{funcdesc}

\begin{funcdesc}{test}{}
Robust test CGI script, usable as main program.
Writes minimal HTTP headers and formats all information provided to
the script in HTML form.
\end{funcdesc}

\begin{funcdesc}{print_environ}{}
Format the shell environment in HTML.
\end{funcdesc}

\begin{funcdesc}{print_form}{form}
Format a form in HTML.
\end{funcdesc}

\begin{funcdesc}{print_directory}{}
Format the current directory in HTML.
\end{funcdesc}

\begin{funcdesc}{print_environ_usage}{}
Print a list of useful (used by CGI) environment variables in
HTML.
\end{funcdesc}

\begin{funcdesc}{escape}{s\optional{, quote}}
Convert the characters
\character{\&}, \character{<} and \character{>} in string \var{s} to
HTML-safe sequences.  Use this if you need to display text that might
contain such characters in HTML.  If the optional flag \var{quote} is
true, the double-quote character (\character{"}) is also translated;
this helps for inclusion in an HTML attribute value, as in \code{<A
HREF="...">}.  If the value to be quoted might include single- or
double-quote characters, or both, consider using the
\function{quoteattr()} function in the \refmodule{xml.sax.saxutils}
module instead.
\end{funcdesc}


\subsection{Caring about security \label{cgi-security}}

\indexii{CGI}{security}

There's one important rule: if you invoke an external program (via the
\function{os.system()} or \function{os.popen()} functions. or others
with similar functionality), make very sure you don't pass arbitrary
strings received from the client to the shell.  This is a well-known
security hole whereby clever hackers anywhere on the Web can exploit a
gullible CGI script to invoke arbitrary shell commands.  Even parts of
the URL or field names cannot be trusted, since the request doesn't
have to come from your form!

To be on the safe side, if you must pass a string gotten from a form
to a shell command, you should make sure the string contains only
alphanumeric characters, dashes, underscores, and periods.


\subsection{Installing your CGI script on a \UNIX\ system}

Read the documentation for your HTTP server and check with your local
system administrator to find the directory where CGI scripts should be
installed; usually this is in a directory \file{cgi-bin} in the server tree.

Make sure that your script is readable and executable by ``others''; the
\UNIX{} file mode should be \code{0755} octal (use \samp{chmod 0755
\var{filename}}).  Make sure that the first line of the script contains
\code{\#!} starting in column 1 followed by the pathname of the Python
interpreter, for instance:

\begin{verbatim}
#!/usr/local/bin/python
\end{verbatim}

Make sure the Python interpreter exists and is executable by ``others''.

Make sure that any files your script needs to read or write are
readable or writable, respectively, by ``others'' --- their mode
should be \code{0644} for readable and \code{0666} for writable.  This
is because, for security reasons, the HTTP server executes your script
as user ``nobody'', without any special privileges.  It can only read
(write, execute) files that everybody can read (write, execute).  The
current directory at execution time is also different (it is usually
the server's cgi-bin directory) and the set of environment variables
is also different from what you get when you log in.  In particular, don't
count on the shell's search path for executables (\envvar{PATH}) or
the Python module search path (\envvar{PYTHONPATH}) to be set to
anything interesting.

If you need to load modules from a directory which is not on Python's
default module search path, you can change the path in your script,
before importing other modules.  For example:

\begin{verbatim}
import sys
sys.path.insert(0, "/usr/home/joe/lib/python")
sys.path.insert(0, "/usr/local/lib/python")
\end{verbatim}

(This way, the directory inserted last will be searched first!)

Instructions for non-\UNIX{} systems will vary; check your HTTP server's
documentation (it will usually have a section on CGI scripts).


\subsection{Testing your CGI script}

Unfortunately, a CGI script will generally not run when you try it
from the command line, and a script that works perfectly from the
command line may fail mysteriously when run from the server.  There's
one reason why you should still test your script from the command
line: if it contains a syntax error, the Python interpreter won't
execute it at all, and the HTTP server will most likely send a cryptic
error to the client.

Assuming your script has no syntax errors, yet it does not work, you
have no choice but to read the next section.


\subsection{Debugging CGI scripts} \indexii{CGI}{debugging}

First of all, check for trivial installation errors --- reading the
section above on installing your CGI script carefully can save you a
lot of time.  If you wonder whether you have understood the
installation procedure correctly, try installing a copy of this module
file (\file{cgi.py}) as a CGI script.  When invoked as a script, the file
will dump its environment and the contents of the form in HTML form.
Give it the right mode etc, and send it a request.  If it's installed
in the standard \file{cgi-bin} directory, it should be possible to send it a
request by entering a URL into your browser of the form:

\begin{verbatim}
http://yourhostname/cgi-bin/cgi.py?name=Joe+Blow&addr=At+Home
\end{verbatim}

If this gives an error of type 404, the server cannot find the script
-- perhaps you need to install it in a different directory.  If it
gives another error, there's an installation problem that
you should fix before trying to go any further.  If you get a nicely
formatted listing of the environment and form content (in this
example, the fields should be listed as ``addr'' with value ``At Home''
and ``name'' with value ``Joe Blow''), the \file{cgi.py} script has been
installed correctly.  If you follow the same procedure for your own
script, you should now be able to debug it.

The next step could be to call the \module{cgi} module's
\function{test()} function from your script: replace its main code
with the single statement

\begin{verbatim}
cgi.test()
\end{verbatim}

This should produce the same results as those gotten from installing
the \file{cgi.py} file itself.

When an ordinary Python script raises an unhandled exception (for
whatever reason: of a typo in a module name, a file that can't be
opened, etc.), the Python interpreter prints a nice traceback and
exits.  While the Python interpreter will still do this when your CGI
script raises an exception, most likely the traceback will end up in
one of the HTTP server's log files, or be discarded altogether.

Fortunately, once you have managed to get your script to execute
\emph{some} code, you can easily send tracebacks to the Web browser
using the \refmodule{cgitb} module.  If you haven't done so already,
just add the line:

\begin{verbatim}
import cgitb; cgitb.enable()
\end{verbatim}

to the top of your script.  Then try running it again; when a
problem occurs, you should see a detailed report that will
likely make apparent the cause of the crash.

If you suspect that there may be a problem in importing the
\refmodule{cgitb} module, you can use an even more robust approach
(which only uses built-in modules):

\begin{verbatim}
import sys
sys.stderr = sys.stdout
print "Content-Type: text/plain"
print
...your code here...
\end{verbatim}

This relies on the Python interpreter to print the traceback.  The
content type of the output is set to plain text, which disables all
HTML processing.  If your script works, the raw HTML will be displayed
by your client.  If it raises an exception, most likely after the
first two lines have been printed, a traceback will be displayed.
Because no HTML interpretation is going on, the traceback will be
readable.


\subsection{Common problems and solutions}

\begin{itemize}
\item Most HTTP servers buffer the output from CGI scripts until the
script is completed.  This means that it is not possible to display a
progress report on the client's display while the script is running.

\item Check the installation instructions above.

\item Check the HTTP server's log files.  (\samp{tail -f logfile} in a
separate window may be useful!)

\item Always check a script for syntax errors first, by doing something
like \samp{python script.py}.

\item If your script does not have any syntax errors, try adding
\samp{import cgitb; cgitb.enable()} to the top of the script.

\item When invoking external programs, make sure they can be found.
Usually, this means using absolute path names --- \envvar{PATH} is
usually not set to a very useful value in a CGI script.

\item When reading or writing external files, make sure they can be read
or written by the userid under which your CGI script will be running:
this is typically the userid under which the web server is running, or some
explicitly specified userid for a web server's \samp{suexec} feature.

\item Don't try to give a CGI script a set-uid mode.  This doesn't work on
most systems, and is a security liability as well.
\end{itemize}


\section{\module{urllib} ---
         Open arbitrary resources by URL}

\declaremodule{standard}{urllib}
\modulesynopsis{Open an arbitrary network resource by URL (requires sockets).}

\index{WWW}
\index{World-Wide Web}
\index{URL}


This module provides a high-level interface for fetching data across
the World-Wide Web.  In particular, the \function{urlopen()} function
is similar to the built-in function \function{open()}, but accepts
Universal Resource Locators (URLs) instead of filenames.  Some
restrictions apply --- it can only open URLs for reading, and no seek
operations are available.

It defines the following public functions:

\begin{funcdesc}{urlopen}{url\optional{, data}}
Open a network object denoted by a URL for reading.  If the URL does
not have a scheme identifier, or if it has \file{file:} as its scheme
identifier, this opens a local file; otherwise it opens a socket to a
server somewhere on the network.  If the connection cannot be made, or
if the server returns an error code, the \exception{IOError} exception
is raised.  If all went well, a file-like object is returned.  This
supports the following methods: \method{read()}, \method{readline()},
\method{readlines()}, \method{fileno()}, \method{close()},
\method{info()} and \method{geturl()}.

Except for the \method{info()} and \method{geturl()} methods,
these methods have the same interface as for
file objects --- see section \ref{bltin-file-objects} in this
manual.  (It is not a built-in file object, however, so it can't be
used at those few places where a true built-in file object is
required.)

The \method{info()} method returns an instance of the class
\class{mimetools.Message} containing meta-information associated
with the URL.  When the method is HTTP, these headers are those
returned by the server at the head of the retrieved HTML page
(including Content-Length and Content-Type).  When the method is FTP,
a Content-Length header will be present if (as is now usual) the
server passed back a file length in response to the FTP retrieval
request.  When the method is local-file, returned headers will include
a Date representing the file's last-modified time, a Content-Length
giving file size, and a Content-Type containing a guess at the file's
type. See also the description of the
\refmodule{mimetools}\refstmodindex{mimetools} module.

The \method{geturl()} method returns the real URL of the page.  In
some cases, the HTTP server redirects a client to another URL.  The
\function{urlopen()} function handles this transparently, but in some
cases the caller needs to know which URL the client was redirected
to.  The \method{geturl()} method can be used to get at this
redirected URL.

If the \var{url} uses the \file{http:} scheme identifier, the optional
\var{data} argument may be given to specify a \code{POST} request
(normally the request type is \code{GET}).  The \var{data} argument
must in standard \file{application/x-www-form-urlencoded} format;
see the \function{urlencode()} function below.

The \function{urlopen()} function works transparently with proxies
which do not require authentication.  In a \UNIX{} or Windows
environment, set the \envvar{http_proxy}, \envvar{ftp_proxy} or
\envvar{gopher_proxy} environment variables to a URL that identifies
the proxy server before starting the Python interpreter.  For example
(the \character{\%} is the command prompt):

\begin{verbatim}
% http_proxy="http://www.someproxy.com:3128"
% export http_proxy
% python
...
\end{verbatim}

In a Macintosh environment, \function{urlopen()} will retrieve proxy
information from Internet\index{Internet Config} Config.

Proxies which require authentication for use are not currently
supported; this is considered an implementation limitation.
\end{funcdesc}

\begin{funcdesc}{urlretrieve}{url\optional{, filename\optional{, hook}}}
Copy a network object denoted by a URL to a local file, if necessary.
If the URL points to a local file, or a valid cached copy of the
object exists, the object is not copied.  Return a tuple
\code{(\var{filename}, \var{headers})} where \var{filename} is the
local file name under which the object can be found, and \var{headers}
is either \code{None} (for a local object) or whatever the
\method{info()} method of the object returned by \function{urlopen()}
returned (for a remote object, possibly cached).  Exceptions are the
same as for \function{urlopen()}.

The second argument, if present, specifies the file location to copy
to (if absent, the location will be a tempfile with a generated name).
The third argument, if present, is a hook function that will be called
once on establishment of the network connection and once after each
block read thereafter.  The hook will be passed three arguments; a
count of blocks transferred so far, a block size in bytes, and the
total size of the file.  The third argument may be \code{-1} on older
FTP servers which do not return a file size in response to a retrieval 
request.

If the \var{url} uses the \file{http:} scheme identifier, the optional
\var{data} argument may be given to specify a \code{POST} request
(normally the request type is \code{GET}).  The \var{data} argument
must in standard \file{application/x-www-form-urlencoded} format;
see the \function{urlencode()} function below.
\end{funcdesc}

\begin{funcdesc}{urlcleanup}{}
Clear the cache that may have been built up by previous calls to
\function{urlretrieve()}.
\end{funcdesc}

\begin{funcdesc}{quote}{string\optional{, safe}}
Replace special characters in \var{string} using the \samp{\%xx} escape.
Letters, digits, and the characters \character{_,.-} are never quoted.
The optional \var{safe} parameter specifies additional characters
that should not be quoted --- its default value is \code{'/'}.

Example: \code{quote('/\~{}connolly/')} yields \code{'/\%7econnolly/'}.
\end{funcdesc}

\begin{funcdesc}{quote_plus}{string\optional{, safe}}
Like \function{quote()}, but also replaces spaces by plus signs, as
required for quoting HTML form values.  Plus signs in the original
string are escaped unless they are included in \var{safe}.
\end{funcdesc}

\begin{funcdesc}{unquote}{string}
Replace \samp{\%xx} escapes by their single-character equivalent.

Example: \code{unquote('/\%7Econnolly/')} yields \code{'/\~{}connolly/'}.
\end{funcdesc}

\begin{funcdesc}{unquote_plus}{string}
Like \function{unquote()}, but also replaces plus signs by spaces, as
required for unquoting HTML form values.
\end{funcdesc}

\begin{funcdesc}{urlencode}{dict}
Convert a dictionary to a ``url-encoded'' string, suitable to pass to
\function{urlopen()} above as the optional \var{data} argument.  This
is useful to pass a dictionary of form fields to a \code{POST}
request.  The resulting string is a series of
\code{\var{key}=\var{value}} pairs separated by \character{\&}
characters, where both \var{key} and \var{value} are quoted using
\function{quote_plus()} above.
\end{funcdesc}

The public functions \function{urlopen()} and
\function{urlretrieve()} create an instance of the
\class{FancyURLopener} class and use it to perform their requested
actions.  To override this functionality, programmers can create a
subclass of \class{URLopener} or \class{FancyURLopener}, then assign
that an instance of that class to the
\code{urllib._urlopener} variable before calling the desired function.
For example, applications may want to specify a different
\code{user-agent} header than \class{URLopener} defines.  This can be
accomplished with the following code:

\begin{verbatim}
class AppURLopener(urllib.FancyURLopener):
    def __init__(self, *args):
        self.version = "App/1.7"
        apply(urllib.FancyURLopener.__init__, (self,) + args)

urllib._urlopener = AppURLopener()
\end{verbatim}

\begin{classdesc}{URLopener}{\optional{proxies\optional{, **x509}}}
Base class for opening and reading URLs.  Unless you need to support
opening objects using schemes other than \file{http:}, \file{ftp:},
\file{gopher:} or \file{file:}, you probably want to use
\class{FancyURLopener}.

By default, the \class{URLopener} class sends a
\code{user-agent} header of \samp{urllib/\var{VVV}}, where
\var{VVV} is the \module{urllib} version number.  Applications can
define their own \code{user-agent} header by subclassing
\class{URLopener} or \class{FancyURLopener} and setting the instance
attribute \member{version} to an appropriate string value before the
\method{open()} method is called.

Additional keyword parameters, collected in \var{x509}, are used for
authentication with the \file{https:} scheme.  The keywords
\var{key_file} and \var{cert_file} are supported; both are needed to
actually retrieve a resource at an \file{https:} URL.
\end{classdesc}

\begin{classdesc}{FancyURLopener}{...}
\class{FancyURLopener} subclasses \class{URLopener} providing default
handling for the following HTTP response codes: 301, 302 or 401.  For
301 and 302 response codes, the \code{location} header is used to
fetch the actual URL.  For 401 response codes (authentication
required), basic HTTP authentication is performed.

The parameters to the constructor are the same as those for
\class{URLopener}.
\end{classdesc}

Restrictions:

\begin{itemize}

\item
Currently, only the following protocols are supported: HTTP, (versions
0.9 and 1.0), Gopher (but not Gopher-+), FTP, and local files.
\indexii{HTTP}{protocol}
\indexii{Gopher}{protocol}
\indexii{FTP}{protocol}

\item
The caching feature of \function{urlretrieve()} has been disabled
until I find the time to hack proper processing of Expiration time
headers.

\item
There should be a function to query whether a particular URL is in
the cache.

\item
For backward compatibility, if a URL appears to point to a local file
but the file can't be opened, the URL is re-interpreted using the FTP
protocol.  This can sometimes cause confusing error messages.

\item
The \function{urlopen()} and \function{urlretrieve()} functions can
cause arbitrarily long delays while waiting for a network connection
to be set up.  This means that it is difficult to build an interactive
web client using these functions without using threads.

\item
The data returned by \function{urlopen()} or \function{urlretrieve()}
is the raw data returned by the server.  This may be binary data
(e.g. an image), plain text or (for example) HTML\index{HTML}.  The
HTTP\indexii{HTTP}{protocol} protocol provides type information in the
reply header, which can be inspected by looking at the
\code{content-type} header.  For the Gopher\indexii{Gopher}{protocol}
protocol, type information is encoded in the URL; there is currently
no easy way to extract it.  If the returned data is HTML, you can use
the module \refmodule{htmllib}\refstmodindex{htmllib} to parse it.

\item
This module does not support the use of proxies which require
authentication.  This may be implemented in the future.

\item
Although the \module{urllib} module contains (undocumented) routines
to parse and unparse URL strings, the recommended interface for URL
manipulation is in module \refmodule{urlparse}\refstmodindex{urlparse}.

\end{itemize}


\subsection{URLopener Objects \label{urlopener-objs}}
\sectionauthor{Skip Montanaro}{skip@mojam.com}

\class{URLopener} and \class{FancyURLopener} objects have the
following attributes.

\begin{methoddesc}[URLopener]{open}{fullurl\optional{, data}}
Open \var{fullurl} using the appropriate protocol.  This method sets 
up cache and proxy information, then calls the appropriate open method with
its input arguments.  If the scheme is not recognized,
\method{open_unknown()} is called.  The \var{data} argument 
has the same meaning as the \var{data} argument of \function{urlopen()}.
\end{methoddesc}

\begin{methoddesc}[URLopener]{open_unknown}{fullurl\optional{, data}}
Overridable interface to open unknown URL types.
\end{methoddesc}

\begin{methoddesc}[URLopener]{retrieve}{url\optional{,
                                        filename\optional{,
                                        reporthook\optional{, data}}}}
Retrieves the contents of \var{url} and places it in \var{filename}.  The
return value is a tuple consisting of a local filename and either a
\class{mimetools.Message} object containing the response headers (for remote
URLs) or None (for local URLs).  The caller must then open and read the
contents of \var{filename}.  If \var{filename} is not given and the URL
refers to a local file, the input filename is returned.  If the URL is
non-local and \var{filename} is not given, the filename is the output of
\function{tempfile.mktemp()} with a suffix that matches the suffix of the last
path component of the input URL.  If \var{reporthook} is given, it must be
a function accepting three numeric parameters.  It will be called after each
chunk of data is read from the network.  \var{reporthook} is ignored for
local URLs.

If the \var{url} uses the \file{http:} scheme identifier, the optional
\var{data} argument may be given to specify a \code{POST} request
(normally the request type is \code{GET}).  The \var{data} argument
must in standard \file{application/x-www-form-urlencoded} format;
see the \function{urlencode()} function below.
\end{methoddesc}

\begin{memberdesc}[URLopener]{version}
Variable that specifies the user agent of the opener object.  To get
\refmodule{urllib} to tell servers that it is a particular user agent,
set this in a subclass as a class variable or in the constructor
before calling the base constructor.
\end{memberdesc}


\subsection{Examples}
\nodename{Urllib Examples}

Here is an example session that uses the \samp{GET} method to retrieve
a URL containing parameters:

\begin{verbatim}
>>> import urllib
>>> params = urllib.urlencode({'spam': 1, 'eggs': 2, 'bacon': 0})
>>> f = urllib.urlopen("http://www.musi-cal.com/cgi-bin/query?%s" % params)
>>> print f.read()
\end{verbatim}

The following example uses the \samp{POST} method instead:

\begin{verbatim}
>>> import urllib
>>> params = urllib.urlencode({'spam': 1, 'eggs': 2, 'bacon': 0})
>>> f = urllib.urlopen("http://www.musi-cal.com/cgi-bin/query", params)
>>> print f.read()
\end{verbatim}

\section{\module{httplib} ---
         HTTP protocol client}

\declaremodule{standard}{httplib}
\modulesynopsis{HTTP and HTTPS protocol client (requires sockets).}

\indexii{HTTP}{protocol}
\index{HTTP!\module{httplib} (standard module)}

This module defines classes which implement the client side of the
HTTP and HTTPS protocols.  It is normally not used directly --- the
module \refmodule{urllib}\refstmodindex{urllib} uses it to handle URLs
that use HTTP and HTTPS.  \note{HTTPS support is only
available if the \refmodule{socket} module was compiled with SSL
support.}

The constants defined in this module are:

\begin{datadesc}{HTTP_PORT}
  The default port for the HTTP protocol (always \code{80}).
\end{datadesc}

\begin{datadesc}{HTTPS_PORT}
  The default port for the HTTPS protocol (always \code{443}).
\end{datadesc}

The module provides the following classes:

\begin{classdesc}{HTTPConnection}{host\optional{, port}}
An \class{HTTPConnection} instance represents one transaction with an HTTP
server.  It should be instantiated passing it a host and optional port number.
If no port number is passed, the port is extracted from the host string if it
has the form \code{\var{host}:\var{port}}, else the default HTTP port (80) is
used.  For example, the following calls all create instances that connect to
the server at the same host and port:

\begin{verbatim}
>>> h1 = httplib.HTTPConnection('www.cwi.nl')
>>> h2 = httplib.HTTPConnection('www.cwi.nl:80')
>>> h3 = httplib.HTTPConnection('www.cwi.nl', 80)
\end{verbatim}
\end{classdesc}

\begin{classdesc}{HTTPSConnection}{host\optional{, port}}
A subclass of \class{HTTPConnection} that uses SSL for communication with
secure servers.  Default port is \code{443}.
\end{classdesc}

The following exceptions are raised as appropriate:

\begin{excdesc}{HTTPException}
The base class of the other exceptions in this module.  It is a
subclass of \exception{Exception}.
\end{excdesc}

\begin{excdesc}{NotConnected}
A subclass of \exception{HTTPException}.
\end{excdesc}

\begin{excdesc}{InvalidURL}
A subclass of \exception{HTTPException}, raised if a port is given and is
either non-numeric or empty.
\end{excdesc}

\begin{excdesc}{UnknownProtocol}
A subclass of \exception{HTTPException}.
\end{excdesc}

\begin{excdesc}{UnknownTransferEncoding}
A subclass of \exception{HTTPException}.
\end{excdesc}

\begin{excdesc}{IllegalKeywordArgument}
A subclass of \exception{HTTPException}.
\end{excdesc}

\begin{excdesc}{UnimplementedFileMode}
A subclass of \exception{HTTPException}.
\end{excdesc}

\begin{excdesc}{IncompleteRead}
A subclass of \exception{HTTPException}.
\end{excdesc}

\begin{excdesc}{ImproperConnectionState}
A subclass of \exception{HTTPException}.
\end{excdesc}

\begin{excdesc}{CannotSendRequest}
A subclass of \exception{ImproperConnectionState}.
\end{excdesc}

\begin{excdesc}{CannotSendHeader}
A subclass of \exception{ImproperConnectionState}.
\end{excdesc}

\begin{excdesc}{ResponseNotReady}
A subclass of \exception{ImproperConnectionState}.
\end{excdesc}

\begin{excdesc}{BadStatusLine}
A subclass of \exception{HTTPException}.  Raised if a server responds with a
HTTP status code that we don't understand.
\end{excdesc}


\subsection{HTTPConnection Objects \label{httpconnection-objects}}

\class{HTTPConnection} instances have the following methods:

\begin{methoddesc}{request}{method, url\optional{, body\optional{, headers}}}
This will send a request to the server using the HTTP request method
\var{method} and the selector \var{url}.  If the \var{body} argument is
present, it should be a string of data to send after the headers are finished.
The header Content-Length is automatically set to the correct value.
The \var{headers} argument should be a mapping of extra HTTP headers to send
with the request.
\end{methoddesc}

\begin{methoddesc}{getresponse}{}
Should be called after a request is sent to get the response from the server.
Returns an \class{HTTPResponse} instance.
\end{methoddesc}

\begin{methoddesc}{set_debuglevel}{level}
Set the debugging level (the amount of debugging output printed).
The default debug level is \code{0}, meaning no debugging output is
printed.
\end{methoddesc}

\begin{methoddesc}{connect}{}
Connect to the server specified when the object was created.
\end{methoddesc}

\begin{methoddesc}{close}{}
Close the connection to the server.
\end{methoddesc}

\begin{methoddesc}{send}{data}
Send data to the server.  This should be used directly only after the
\method{endheaders()} method has been called and before
\method{getreply()} has been called.
\end{methoddesc}

\begin{methoddesc}{putrequest}{request, selector}
This should be the first call after the connection to the server has
been made.  It sends a line to the server consisting of the
\var{request} string, the \var{selector} string, and the HTTP version
(\code{HTTP/1.1}).
\end{methoddesc}

\begin{methoddesc}{putheader}{header, argument\optional{, ...}}
Send an \rfc{822}-style header to the server.  It sends a line to the
server consisting of the header, a colon and a space, and the first
argument.  If more arguments are given, continuation lines are sent,
each consisting of a tab and an argument.
\end{methoddesc}

\begin{methoddesc}{endheaders}{}
Send a blank line to the server, signalling the end of the headers.
\end{methoddesc}


\subsection{HTTPResponse Objects \label{httpresponse-objects}}

\class{HTTPResponse} instances have the following methods and attributes:

\begin{methoddesc}{read}{}
Reads and returns the response body.
\end{methoddesc}

\begin{methoddesc}{getheader}{name\optional{, default}}
Get the contents of the header \var{name}, or \var{default} if there is no
matching header.
\end{methoddesc}

\begin{datadesc}{msg}
  A \class{mimetools.Message} instance containing the response headers.
\end{datadesc}

\begin{datadesc}{version}
  HTTP protocol version used by server.  10 for HTTP/1.0, 11 for HTTP/1.1.
\end{datadesc}

\begin{datadesc}{status}
  Status code returned by server.
\end{datadesc}

\begin{datadesc}{reason}
  Reason phrase returned by server.
\end{datadesc}


\subsection{Examples \label{httplib-examples}}

Here is an example session that uses the \samp{GET} method:

\begin{verbatim}
>>> import httplib
>>> conn = httplib.HTTPConnection("www.python.org")
>>> conn.request("GET", "/index.html")
>>> r1 = conn.getresponse()
>>> print r1.status, r1.reason
200 OK
>>> data1 = r1.read()
>>> conn.request("GET", "/parrot.spam")
>>> r2 = conn.getresponse()
>>> print r2.status, r2.reason
404 Not Found
>>> data2 = r2.read()
>>> conn.close()
\end{verbatim}

Here is an example session that shows how to \samp{POST} requests:

\begin{verbatim}
>>> import httplib, urllib
>>> params = urllib.urlencode({'spam': 1, 'eggs': 2, 'bacon': 0})
>>> headers = {"Content-type": "application/x-www-form-urlencoded",
...            "Accept": "text/plain"}
>>> conn = httplib.HTTPConnection("musi-cal.mojam.com:80")
>>> conn.request("POST", "/cgi-bin/query", params, headers)
>>> response = conn.getresponse()
>>> print response.status, response.reason
200 OK
>>> data = response.read()
>>> conn.close()
\end{verbatim}

\section{Built-in module \sectcode{ftplib}}
\stmodindex{ftplib}

\renewcommand{\indexsubitem}{(in module ftplib)}

To be provided.

\section{Built-in module \sectcode{gopherlib}}
\stmodindex{gopherlib}

\renewcommand{\indexsubitem}{(in module gopherlib)}

To be provided.

\section{Standard Module \sectcode{nntplib}}
\label{module-nntplib}
\stmodindex{nntplib}

\renewcommand{\indexsubitem}{(in module nntplib)}

This module defines the class \code{NNTP} which implements the client
side of the NNTP protocol.  It can be used to implement a news reader
or poster, or automated news processors.  For more information on NNTP
(Network News Transfer Protocol), see Internet RFC 977.

Here are two small examples of how it can be used.  To list some
statistics about a newsgroup and print the subjects of the last 10
articles:

\bcode\begin{verbatim}
>>> s = NNTP('news.cwi.nl')
>>> resp, count, first, last, name = s.group('comp.lang.python')
>>> print 'Group', name, 'has', count, 'articles, range', first, 'to', last
Group comp.lang.python has 59 articles, range 3742 to 3803
>>> resp, subs = s.xhdr('subject', first + '-' + last)
>>> for id, sub in subs[-10:]: print id, sub
... 
3792 Re: Removing elements from a list while iterating...
3793 Re: Who likes Info files?
3794 Emacs and doc strings
3795 a few questions about the Mac implementation
3796 Re: executable python scripts
3797 Re: executable python scripts
3798 Re: a few questions about the Mac implementation 
3799 Re: PROPOSAL: A Generic Python Object Interface for Python C Modules
3802 Re: executable python scripts 
3803 Re: POSIX wait and SIGCHLD
>>> s.quit()
'205 news.cwi.nl closing connection.  Goodbye.'
>>> 
\end{verbatim}\ecode

To post an article from a file (this assumes that the article has
valid headers):

\bcode\begin{verbatim}
>>> s = NNTP('news.cwi.nl')
>>> f = open('/tmp/article')
>>> s.post(f)
'240 Article posted successfully.'
>>> s.quit()
'205 news.cwi.nl closing connection.  Goodbye.'
>>> 
\end{verbatim}\ecode
%
The module itself defines the following items:

\begin{funcdesc}{NNTP}{host\optional{\, port}}
Return a new instance of the \code{NNTP} class, representing a
connection to the NNTP server running on host \var{host}, listening at
port \var{port}.  The default \var{port} is 119.
\end{funcdesc}

\begin{excdesc}{error_reply}
Exception raised when an unexpected reply is received from the server.
\end{excdesc}

\begin{excdesc}{error_temp}
Exception raised when an error code in the range 400--499 is received.
\end{excdesc}

\begin{excdesc}{error_perm}
Exception raised when an error code in the range 500--599 is received.
\end{excdesc}

\begin{excdesc}{error_proto}
Exception raised when a reply is received from the server that does
not begin with a digit in the range 1--5.
\end{excdesc}

\subsection{NNTP Objects}

NNTP instances have the following methods.  The \var{response} that is
returned as the first item in the return tuple of almost all methods
is the server's response: a string beginning with a three-digit code.
If the server's response indicates an error, the method raises one of
the above exceptions.

\renewcommand{\indexsubitem}{(NNTP object method)}

\begin{funcdesc}{getwelcome}{}
Return the welcome message sent by the server in reply to the initial
connection.  (This message sometimes contains disclaimers or help
information that may be relevant to the user.)
\end{funcdesc}

\begin{funcdesc}{set_debuglevel}{level}
Set the instance's debugging level.  This controls the amount of
debugging output printed.  The default, 0, produces no debugging
output.  A value of 1 produces a moderate amount of debugging output,
generally a single line per request or response.  A value of 2 or
higher produces the maximum amount of debugging output, logging each
line sent and received on the connection (including message text).
\end{funcdesc}

\begin{funcdesc}{newgroups}{date\, time}
Send a \samp{NEWGROUPS} command.  The \var{date} argument should be a
string of the form \code{"\var{yy}\var{mm}\var{dd}"} indicating the
date, and \var{time} should be a string of the form
\code{"\var{hh}\var{mm}\var{ss}"} indicating the time.  Return a pair
\code{(\var{response}, \var{groups})} where \var{groups} is a list of
group names that are new since the given date and time.
\end{funcdesc}

\begin{funcdesc}{newnews}{group\, date\, time}
Send a \samp{NEWNEWS} command.  Here, \var{group} is a group name or
\code{"*"}, and \var{date} and \var{time} have the same meaning as for
\code{newgroups()}.  Return a pair \code{(\var{response},
\var{articles})} where \var{articles} is a list of article ids.
\end{funcdesc}

\begin{funcdesc}{list}{}
Send a \samp{LIST} command.  Return a pair \code{(\var{response},
\var{list})} where \var{list} is a list of tuples.  Each tuple has the
form \code{(\var{group}, \var{last}, \var{first}, \var{flag})}, where
\var{group} is a group name, \var{last} and \var{first} are the last
and first article numbers (as strings), and \var{flag} is \code{'y'}
if posting is allowed, \code{'n'} if not, and \code{'m'} if the
newsgroup is moderated.  (Note the ordering: \var{last}, \var{first}.)
\end{funcdesc}

\begin{funcdesc}{group}{name}
Send a \samp{GROUP} command, where \var{name} is the group name.
Return a tuple \code{(\var{response}, \var{count}, \var{first},
\var{last}, \var{name})} where \var{count} is the (estimated) number
of articles in the group, \var{first} is the first article number in
the group, \var{last} is the last article number in the group, and
\var{name} is the group name.  The numbers are returned as strings.
\end{funcdesc}

\begin{funcdesc}{help}{}
Send a \samp{HELP} command.  Return a pair \code{(\var{response},
\var{list})} where \var{list} is a list of help strings.
\end{funcdesc}

\begin{funcdesc}{stat}{id}
Send a \samp{STAT} command, where \var{id} is the message id (enclosed
in \samp{<} and \samp{>}) or an article number (as a string).
Return a triple \code{(\var{response}, \var{number}, \var{id})} where
\var{number} is the article number (as a string) and \var{id} is the
article id  (enclosed in \samp{<} and \samp{>}).
\end{funcdesc}

\begin{funcdesc}{next}{}
Send a \samp{NEXT} command.  Return as for \code{stat()}.
\end{funcdesc}

\begin{funcdesc}{last}{}
Send a \samp{LAST} command.  Return as for \code{stat()}.
\end{funcdesc}

\begin{funcdesc}{head}{id}
Send a \samp{HEAD} command, where \var{id} has the same meaning as for
\code{stat()}.  Return a pair \code{(\var{response}, \var{list})}
where \var{list} is a list of the article's headers (an uninterpreted
list of lines, without trailing newlines).
\end{funcdesc}

\begin{funcdesc}{body}{id}
Send a \samp{BODY} command, where \var{id} has the same meaning as for
\code{stat()}.  Return a pair \code{(\var{response}, \var{list})}
where \var{list} is a list of the article's body text (an
uninterpreted list of lines, without trailing newlines).
\end{funcdesc}

\begin{funcdesc}{article}{id}
Send a \samp{ARTICLE} command, where \var{id} has the same meaning as
for \code{stat()}.  Return a pair \code{(\var{response}, \var{list})}
where \var{list} is a list of the article's header and body text (an
uninterpreted list of lines, without trailing newlines).
\end{funcdesc}

\begin{funcdesc}{slave}{}
Send a \samp{SLAVE} command.  Return the server's \var{response}.
\end{funcdesc}

\begin{funcdesc}{xhdr}{header\, string}
Send an \samp{XHDR} command.  This command is not defined in the RFC
but is a common extension.  The \var{header} argument is a header
keyword, e.g. \code{"subject"}.  The \var{string} argument should have
the form \code{"\var{first}-\var{last}"} where \var{first} and
\var{last} are the first and last article numbers to search.  Return a
pair \code{(\var{response}, \var{list})}, where \var{list} is a list of
pairs \code{(\var{id}, \var{text})}, where \var{id} is an article id
(as a string) and \var{text} is the text of the requested header for
that article.
\end{funcdesc}

\begin{funcdesc}{post}{file}
Post an article using the \samp{POST} command.  The \var{file}
argument is an open file object which is read until EOF using its
\code{readline()} method.  It should be a well-formed news article,
including the required headers.  The \code{post()} method
automatically escapes lines beginning with \samp{.}.
\end{funcdesc}

\begin{funcdesc}{ihave}{id\, file}
Send an \samp{IHAVE} command.  If the response is not an error, treat
\var{file} exactly as for the \code{post()} method.
\end{funcdesc}

\begin{funcdesc}{date}{}
Return a triple \code{(\var{response}, \var{date}, \var{time})},
containing the current date and time in a form suitable for the
\code{newnews} and \code{newgroups} methods.
This is an optional NNTP extension, and may not be supported by all
servers.
\end{funcdesc}

\begin{funcdesc}{xgtitle}{name}
Process an XGTITLE command, returning a pair \code{(\var{response},
\var{list}}, where \var{list} is a list of tuples containing
\code{(\var{name}, \var{title})}.
% XXX huh?  Should that be name, description?
This is an optional NNTP extension, and may not be supported by all
servers.
\end{funcdesc}

\begin{funcdesc}{xover}{start\, end}
Return a pair \code{(\var{resp}, \var{list})}.  \var{list} is a list
of tuples, one for each article in the range delimited by the \var{start}
and \var{end} article numbers.  Each tuple is of the form
\code{(\var{article number}, \var{subject}, \var{poster}, \var{date}, \var{id}, \var{references}, \var{size}, \var{lines})}.
This is an optional NNTP extension, and may not be supported by all
servers.
\end{funcdesc}

\begin{funcdesc}{xpath}{id}
Return a pair \code{(\var{resp}, \var{path})}, where \var{path} is the
directory path to the article with message ID \var{id}.  This is an
optional NNTP extension, and may not be supported by all servers.
\end{funcdesc}

\begin{funcdesc}{quit}{}
Send a \samp{QUIT} command and close the connection.  Once this method
has been called, no other methods of the NNTP object should be called.
\end{funcdesc}

\section{Built-in module \sectcode{urlparse}}
\stmodindex{urlparse}
\index{WWW}
\indexii{World-Wide}{Web}
\index{URL}
\indexii{URL}{parsing}
\indexii{relative}{URL}

\renewcommand{\indexsubitem}{(in module urlparse)}

This module defines a standard interface to break URL strings up in
components (addessing scheme, network location, path etc.), to combine
the components back into a URL string, and to convert a ``relative
URL'' to an absolute URL given a ``base URL''.

The module has been designed to match the current Internet draft on
Relative Uniform Resource Locators (and discovered a bug in an earlier
draft!).

It defines the following functions:

\begin{funcdesc}{urlparse}{urlstring\optional{\,
default_scheme\optional{\, allow_fragments}}}
Parse a URL into 6 components, returning a 6-tuple: (addressing
scheme, network location, path, parameters, query, fragment
identifier).  This corresponds to the general structure of a URL:
\code{\var{scheme}://\var{netloc}/\var{path};\var{parameters}?\var{query}\#\var{fragment}}.
Each tuple item is a string, possibly empty.
The components are not broken up in smaller parts (e.g. the network
location is a single string), and \% escapes are not expanded.
The delimiters as shown above are not part of the tuple items, {\em
except} for a leading slash in the \var{path} component, which is
kept if present.

Example:
\code{urlparse('http://www.cwi.nl:80/\%7eguido/Python.html')}
yields the tuple
\code{('http', 'www.cwi.nl:80', '/\%e7guido/Python.html', '', '', '')}.

If the \var{default_scheme} argument is specified, it gives the
default addressing scheme, to be used only if the URL string does not
specify one.  The default value for this argument is the empty string.

If the \var{allow_fragments} argument is zero, fragment identifiers
are not allowed, even if the URL's addressing scheme normally does
support them.  The default value for this argument is \code{1}.
\end{funcdesc}

\begin{funcdesc}{urlunparse}{tuple}
Construct a URL string from a tuple as returned by \code{urlparse}.
This may result in a slightly different, but equivalent URL, if the
URL that was parsed originally had redundant delimiters, e.g. a ? with
an empty query (the draft states that these are equivalent).
\end{funcdesc}

\begin{funcdesc}{urljoin}{base\, url\optional{\, allow_fragments}}
Construct a full (``absolute'') URL by combining a ``base URL''
(\var{base}) with a ``relative URL'' (\var{url}).  Informally, this
uses components of the base URL, in particular the addressing scheme,
the network location and (part of) the path, to provide missing
components in the relative URL.

Example:
\code{urljoin('http://www.cwi.nl/\%7eguido/Python.html',}
\code{'FAQ.html')} yields the string
\code{'http://www.cwi.nl/\%7eguido/FAQ.html'}.

The \var{allow_fragments} argument has the same meaning as for
\code{urlparse}.
\end{funcdesc}

\section{\module{htmllib} ---
         A parser for HTML documents}

\declaremodule{standard}{htmllib}
\modulesynopsis{A parser for HTML documents.}

\index{HTML}
\index{hypertext}


This module defines a class which can serve as a base for parsing text
files formatted in the HyperText Mark-up Language (HTML).  The class
is not directly concerned with I/O --- it must be provided with input
in string form via a method, and makes calls to methods of a
``formatter'' object in order to produce output.  The
\class{HTMLParser} class is designed to be used as a base class for
other classes in order to add functionality, and allows most of its
methods to be extended or overridden.  In turn, this class is derived
from and extends the \class{SGMLParser} class defined in module
\refmodule{sgmllib}\refstmodindex{sgmllib}.  The \class{HTMLParser}
implementation supports the HTML 2.0 language as described in
\rfc{1866}.  Two implementations of formatter objects are provided in
the \refmodule{formatter}\refstmodindex{formatter}\ module; refer to the
documentation for that module for information on the formatter
interface.
\withsubitem{(in module sgmllib)}{\ttindex{SGMLParser}}

The following is a summary of the interface defined by
\class{sgmllib.SGMLParser}:

\begin{itemize}

\item
The interface to feed data to an instance is through the \method{feed()}
method, which takes a string argument.  This can be called with as
little or as much text at a time as desired; \samp{p.feed(a);
p.feed(b)} has the same effect as \samp{p.feed(a+b)}.  When the data
contains complete HTML markup constructs, these are processed immediately;
incomplete constructs are saved in a buffer.  To force processing of all
unprocessed data, call the \method{close()} method.

For example, to parse the entire contents of a file, use:
\begin{verbatim}
parser.feed(open('myfile.html').read())
parser.close()
\end{verbatim}

\item
The interface to define semantics for HTML tags is very simple: derive
a class and define methods called \method{start_\var{tag}()},
\method{end_\var{tag}()}, or \method{do_\var{tag}()}.  The parser will
call these at appropriate moments: \method{start_\var{tag}} or
\method{do_\var{tag}()} is called when an opening tag of the form
\code{<\var{tag} ...>} is encountered; \method{end_\var{tag}()} is called
when a closing tag of the form \code{<\var{tag}>} is encountered.  If
an opening tag requires a corresponding closing tag, like \code{<H1>}
... \code{</H1>}, the class should define the \method{start_\var{tag}()}
method; if a tag requires no closing tag, like \code{<P>}, the class
should define the \method{do_\var{tag}()} method.

\end{itemize}

The module defines a parser class and an exception:

\begin{classdesc}{HTMLParser}{formatter}
This is the basic HTML parser class.  It supports all entity names
required by the XHTML 1.0 Recommendation (\url{http://www.w3.org/TR/xhtml1}).  
It also defines handlers for all HTML 2.0 and many HTML 3.0 and 3.2 elements.
\end{classdesc}

\begin{excdesc}{HTMLParseError}
Exception raised by the \class{HTMLParser} class when it encounters an
error while parsing.
\versionadded{2.4}
\end{excdesc}


\begin{seealso}
  \seemodule{formatter}{Interface definition for transforming an
                        abstract flow of formatting events into
                        specific output events on writer objects.}
  \seemodule{HTMLParser}{Alternate HTML parser that offers a slightly
                         lower-level view of the input, but is
                         designed to work with XHTML, and does not
                         implement some of the SGML syntax not used in
                         ``HTML as deployed'' and which isn't legal
                         for XHTML.}
  \seemodule{htmlentitydefs}{Definition of replacement text for XHTML 1.0 
                             entities.}
  \seemodule{sgmllib}{Base class for \class{HTMLParser}.}
\end{seealso}


\subsection{HTMLParser Objects \label{html-parser-objects}}

In addition to tag methods, the \class{HTMLParser} class provides some
additional methods and instance variables for use within tag methods.

\begin{memberdesc}{formatter}
This is the formatter instance associated with the parser.
\end{memberdesc}

\begin{memberdesc}{nofill}
Boolean flag which should be true when whitespace should not be
collapsed, or false when it should be.  In general, this should only
be true when character data is to be treated as ``preformatted'' text,
as within a \code{<PRE>} element.  The default value is false.  This
affects the operation of \method{handle_data()} and \method{save_end()}.
\end{memberdesc}


\begin{methoddesc}{anchor_bgn}{href, name, type}
This method is called at the start of an anchor region.  The arguments
correspond to the attributes of the \code{<A>} tag with the same
names.  The default implementation maintains a list of hyperlinks
(defined by the \code{HREF} attribute for \code{<A>} tags) within the
document.  The list of hyperlinks is available as the data attribute
\member{anchorlist}.
\end{methoddesc}

\begin{methoddesc}{anchor_end}{}
This method is called at the end of an anchor region.  The default
implementation adds a textual footnote marker using an index into the
list of hyperlinks created by \method{anchor_bgn()}.
\end{methoddesc}

\begin{methoddesc}{handle_image}{source, alt\optional{, ismap\optional{,
                                 align\optional{, width\optional{, height}}}}}
This method is called to handle images.  The default implementation
simply passes the \var{alt} value to the \method{handle_data()}
method.
\end{methoddesc}

\begin{methoddesc}{save_bgn}{}
Begins saving character data in a buffer instead of sending it to the
formatter object.  Retrieve the stored data via \method{save_end()}.
Use of the \method{save_bgn()} / \method{save_end()} pair may not be
nested.
\end{methoddesc}

\begin{methoddesc}{save_end}{}
Ends buffering character data and returns all data saved since the
preceding call to \method{save_bgn()}.  If the \member{nofill} flag is
false, whitespace is collapsed to single spaces.  A call to this
method without a preceding call to \method{save_bgn()} will raise a
\exception{TypeError} exception.
\end{methoddesc}



\section{\module{htmlentitydefs} ---
         Definitions of HTML general entities}

\declaremodule{standard}{htmlentitydefs}
\modulesynopsis{Definitions of HTML general entities.}
\sectionauthor{Fred L. Drake, Jr.}{fdrake@acm.org}

This module defines three dictionaries, \code{name2codepoint},
\code{codepoint2name}, and \code{entitydefs}. \code{entitydefs} is
used by the \refmodule{htmllib} module to provide the
\member{entitydefs} member of the \class{HTMLParser} class.  The
definition provided here contains all the entities defined by XHTML 1.0 
that can be handled using simple textual substitution in the Latin-1
character set (ISO-8859-1).


\begin{datadesc}{entitydefs}
  A dictionary mapping XHTML 1.0 entity definitions to their
  replacement text in ISO Latin-1.

\end{datadesc}

\begin{datadesc}{name2codepoint}
  A dictionary that maps HTML entity names to the Unicode codepoints.
  \versionadded{2.3}
\end{datadesc}

\begin{datadesc}{codepoint2name}
  A dictionary that maps Unicode codepoints to HTML entity names.
  \versionadded{2.3}
\end{datadesc}

\section{Standard Module \sectcode{sgmllib}}
\stmodindex{sgmllib}
\index{SGML}

\renewcommand{\indexsubitem}{(in module sgmllib)}

This module defines a class \code{SGMLParser} which serves as the
basis for parsing text files formatted in SGML (Standard Generalized
Mark-up Language).  In fact, it does not provide a full SGML parser
--- it only parses SGML insofar as it is used by HTML, and the module only
exists as a basis for the \code{htmllib} module.
\stmodindex{htmllib}

In particular, the parser is hardcoded to recognize the following
elements:

\begin{itemize}

\item
Opening and closing tags of the form
``\code{<\var{tag} \var{attr}="\var{value}" ...>}'' and
``\code{</\var{tag}>}'', respectively.

\item
Character references of the form ``\code{\&\#\var{name};}''.

\item
Entity references of the form ``\code{\&\var{name};}''.

\item
SGML comments of the form ``\code{<!--\var{text}>}''.

\end{itemize}

The \code{SGMLParser} class must be instantiated without arguments.
It has the following interface methods:

\begin{funcdesc}{reset}{}
Reset the instance.  Loses all unprocessed data.  This is called
implicitly at instantiation time.
\end{funcdesc}

\begin{funcdesc}{setnomoretags}{}
Stop processing tags.  Treat all following input as literal input
(CDATA).  (This is only provided so the HTML tag \code{<PLAINTEXT>}
can be implemented.)
\end{funcdesc}

\begin{funcdesc}{setliteral}{}
Enter literal mode (CDATA mode).
\end{funcdesc}

\begin{funcdesc}{feed}{data}
Feed some text to the parser.  It is processed insofar as it consists
of complete elements; incomplete data is buffered until more data is
fed or \code{close()} is called.
\end{funcdesc}

\begin{funcdesc}{close}{}
Force processing of all buffered data as if it were followed by an
end-of-file mark.  This method may be redefined by a derived class to
define additional processing at the end of the input, but the
redefined version should always call \code{SGMLParser.close()}.
\end{funcdesc}

\begin{funcdesc}{handle_charref}{ref}
This method is called to process a character reference of the form
``\code{\&\#\var{ref};}'' where \var{ref} is a decimal number in the
range 0-255.  It translates the character to \ASCII{} and calls the
method \code{handle_data()} with the character as argument.  If
\var{ref} is invalid or out of range, the method
\code{unknown_charref(\var{ref})} is called instead.
\end{funcdesc}

\begin{funcdesc}{handle_entityref}{ref}
This method is called to process an entity reference of the form
``\code{\&\var{ref};}'' where \var{ref} is an alphabetic entity
reference.  It looks for \var{ref} in the instance (or class)
variable \code{entitydefs} which should give the entity's translation.
If a translation is found, it calls the method \code{handle_data()}
with the translation; otherwise, it calls the method
\code{unknown_entityref(\var{ref})}.
\end{funcdesc}

\begin{funcdesc}{handle_data}{data}
This method is called to process arbitrary data.  It is intended to be
overridden by a derived class; the base class implementation does
nothing.
\end{funcdesc}

\begin{funcdesc}{unknown_starttag}{tag\, attributes}
This method is called to process an unknown start tag.  It is intended
to be overridden by a derived class; the base class implementation
does nothing.  The \var{attributes} argument is a list of
(\var{name}, \var{value}) pairs containing the attributes found inside
the tag's \code{<>} brackets.  The \var{name} has been translated to
lower case and double quotes and backslashes in the \var{value} have
been interpreted.  For instance, for the tag
\code{<A HREF="http://www.cwi.nl/">}, this method would be
called as \code{unknown_starttag('a', [('href', 'http://www.cwi.nl/')])}.
\end{funcdesc}

\begin{funcdesc}{unknown_endtag}{tag}
This method is called to process an unknown end tag.  It is intended
to be overridden by a derived class; the base class implementation
does nothing.
\end{funcdesc}

\begin{funcdesc}{unknown_charref}{ref}
This method is called to process an unknown character reference.  It
is intended to be overridden by a derived class; the base class
implementation does nothing.
\end{funcdesc}

\begin{funcdesc}{unknown_entityref}{ref}
This method is called to process an unknown entity reference.  It is
intended to be overridden by a derived class; the base class
implementation does nothing.
\end{funcdesc}

Apart from overriding or extending the methods listed above, derived
classes may also define methods of the following form to define
processing of specific tags.  Tag names in the input stream are case
independent; the \var{tag} occurring in method names must be in lower
case:

\begin{funcdesc}{start_\var{tag}}{attributes}
This method is called to process an opening tag \var{tag}.  It has
preference over \code{do_\var{tag}()}.  The \var{attributes} argument
has the same meaning as described for \code{unknown_tag()} above.
\end{funcdesc}

\begin{funcdesc}{do_\var{tag}}{attributes}
This method is called to process an opening tag \var{tag} that does
not come with a matching closing tag.  The \var{attributes} argument
has the same meaning as described for \code{unknown_tag()} above.
\end{funcdesc}

\begin{funcdesc}{end_\var{tag}}{}
This method is called to process a closing tag \var{tag}.
\end{funcdesc}

Note that the parser maintains a stack of opening tags for which no
matching closing tag has been found yet.  Only tags processed by
\code{start_\var{tag}()} are pushed on this stack.  Definition of a
\code{end_\var{tag}()} method is optional for these tags.  For tags
processed by \code{do_\var{tag}()} or by \code{unknown_tag()}, no
\code{end_\var{tag}()} method must be defined.

\section{Standard Module \module{rfc822}}
\label{module-rfc822}
\stmodindex{rfc822}


This module defines a class, \class{Message}, which represents a
collection of ``email headers'' as defined by the Internet standard
\rfc{822}.  It is used in various contexts, usually to read such
headers from a file.

Note that there's a separate module to read \UNIX{}, MH, and MMDF
style mailbox files: \module{mailbox}\refstmodindex{mailbox}.

\begin{classdesc}{Message}{file\optional{, seekable}}
A \class{Message} instance is instantiated with an open file object as
parameter.  The optional \var{seekable} parameter indicates if the
file object is seekable; the default value is \code{1} for true.
Instantiation reads headers from the file up to a blank line and
stores them in the instance; after instantiation, the file is
positioned directly after the blank line that terminates the headers.

Input lines as read from the file may either be terminated by CR-LF or
by a single linefeed; a terminating CR-LF is replaced by a single
linefeed before the line is stored.

All header matching is done independent of upper or lower case;
e.g. \code{\var{m}['From']}, \code{\var{m}['from']} and
\code{\var{m}['FROM']} all yield the same result.
\end{classdesc}

\begin{funcdesc}{parsedate}{date}
Attempts to parse a date according to the rules in \rfc{822}.
however, some mailers don't follow that format as specified, so
\function{parsedate()} tries to guess correctly in such cases. 
\var{date} is a string containing an \rfc{822} date, such as 
\code{'Mon, 20 Nov 1995 19:12:08 -0500'}.  If it succeeds in parsing
the date, \function{parsedate()} returns a 9-tuple that can be passed
directly to \function{time.mktime()}; otherwise \code{None} will be
returned.  
\end{funcdesc}

\begin{funcdesc}{parsedate_tz}{date}
Performs the same function as \function{parsedate()}, but returns
either \code{None} or a 10-tuple; the first 9 elements make up a tuple
that can be passed directly to \function{time.mktime()}, and the tenth
is the offset of the date's timezone from UTC (which is the official
term for Greenwich Mean Time).  (Note that the sign of the timezone
offset is the opposite of the sign of the \code{time.timezone}
variable for the same timezone; the latter variable follows the
\POSIX{} standard while this module follows \rfc{822}.)  If the input
string has no timezone, the last element of the tuple returned is
\code{None}.
\end{funcdesc}

\begin{funcdesc}{mktime_tz}{tuple}
Turn a 10-tuple as returned by \function{parsedate_tz()} into a UTC
timestamp.  It the timezone item in the tuple is \code{None}, assume
local time.  Minor deficiency: this first interprets the first 8
elements as a local time and then compensates for the timezone
difference; this may yield a slight error around daylight savings time
switch dates.  Not enough to worry about for common use.
\end{funcdesc}

\subsection{Message Objects}
\label{message-objects}

A \class{Message} instance has the following methods:

\begin{methoddesc}{rewindbody}{}
Seek to the start of the message body.  This only works if the file
object is seekable.
\end{methoddesc}

\begin{methoddesc}{getallmatchingheaders}{name}
Return a list of lines consisting of all headers matching
\var{name}, if any.  Each physical line, whether it is a continuation
line or not, is a separate list item.  Return the empty list if no
header matches \var{name}.
\end{methoddesc}

\begin{methoddesc}{getfirstmatchingheader}{name}
Return a list of lines comprising the first header matching
\var{name}, and its continuation line(s), if any.  Return \code{None}
if there is no header matching \var{name}.
\end{methoddesc}

\begin{methoddesc}{getrawheader}{name}
Return a single string consisting of the text after the colon in the
first header matching \var{name}.  This includes leading whitespace,
the trailing linefeed, and internal linefeeds and whitespace if there
any continuation line(s) were present.  Return \code{None} if there is
no header matching \var{name}.
\end{methoddesc}

\begin{methoddesc}{getheader}{name}
Like \code{getrawheader(\var{name})}, but strip leading and trailing
whitespace.  Internal whitespace is not stripped.
\end{methoddesc}

\begin{methoddesc}{getaddr}{name}
Return a pair \code{(\var{full name}, \var{email address})} parsed
from the string returned by \code{getheader(\var{name})}.  If no
header matching \var{name} exists, return \code{(None, None)};
otherwise both the full name and the address are (possibly empty)
strings.

Example: If \var{m}'s first \code{From} header contains the string
\code{'jack@cwi.nl (Jack Jansen)'}, then
\code{m.getaddr('From')} will yield the pair
\code{('Jack Jansen', 'jack@cwi.nl')}.
If the header contained
\code{'Jack Jansen <jack@cwi.nl>'} instead, it would yield the
exact same result.
\end{methoddesc}

\begin{methoddesc}{getaddrlist}{name}
This is similar to \code{getaddr(\var{list})}, but parses a header
containing a list of email addresses (e.g. a \code{To} header) and
returns a list of \code{(\var{full name}, \var{email address})} pairs
(even if there was only one address in the header).  If there is no
header matching \var{name}, return an empty list.

XXX The current version of this function is not really correct.  It
yields bogus results if a full name contains a comma.
\end{methoddesc}

\begin{methoddesc}{getdate}{name}
Retrieve a header using \method{getheader()} and parse it into a 9-tuple
compatible with \function{time.mktime()}.  If there is no header matching
\var{name}, or it is unparsable, return \code{None}.

Date parsing appears to be a black art, and not all mailers adhere to
the standard.  While it has been tested and found correct on a large
collection of email from many sources, it is still possible that this
function may occasionally yield an incorrect result.
\end{methoddesc}

\begin{methoddesc}{getdate_tz}{name}
Retrieve a header using \method{getheader()} and parse it into a
10-tuple; the first 9 elements will make a tuple compatible with
\function{time.mktime()}, and the 10th is a number giving the offset
of the date's timezone from UTC.  Similarly to \method{getdate()}, if
there is no header matching \var{name}, or it is unparsable, return
\code{None}. 
\end{methoddesc}

\class{Message} instances also support a read-only mapping interface.
In particular: \code{\var{m}[name]} is like
\code{\var{m}.getheader(name)} but raises \exception{KeyError} if
there is no matching header; and \code{len(\var{m})},
\code{\var{m}.has_key(name)}, \code{\var{m}.keys()},
\code{\var{m}.values()} and \code{\var{m}.items()} act as expected
(and consistently).

Finally, \class{Message} instances have two public instance variables:

\begin{memberdesc}{headers}
A list containing the entire set of header lines, in the order in
which they were read.  Each line contains a trailing newline.  The
blank line terminating the headers is not contained in the list.
\end{memberdesc}

\begin{memberdesc}{fp}
The file object passed at instantiation time.
\end{memberdesc}

\section{Standard Module \sectcode{mimetools}}
\stmodindex{mimetools}

\renewcommand{\indexsubitem}{(in module mimetools)}

To be provided.


\chapter{Multimedia Services}
\label{mmedia}

The modules described in this chapter implement various algorithms or
interfaces that are mainly useful for multimedia applications.  They
are available at the discretion of the installation.  Here's an overview:

\localmoduletable
			% Multimedia Services
\section{\module{audioop} ---
         Manipulate raw audio data}

\declaremodule{builtin}{audioop}
\modulesynopsis{Manipulate raw audio data.}


The \module{audioop} module contains some useful operations on sound
fragments.  It operates on sound fragments consisting of signed
integer samples 8, 16 or 32 bits wide, stored in Python strings.  This
is the same format as used by the \refmodule{al} and \refmodule{sunaudiodev}
modules.  All scalar items are integers, unless specified otherwise.

% This para is mostly here to provide an excuse for the index entries...
This module provides support for u-LAW and Intel/DVI ADPCM encodings.
\index{Intel/DVI ADPCM}
\index{ADPCM, Intel/DVI}
\index{u-LAW}

A few of the more complicated operations only take 16-bit samples,
otherwise the sample size (in bytes) is always a parameter of the
operation.

The module defines the following variables and functions:

\begin{excdesc}{error}
This exception is raised on all errors, such as unknown number of bytes
per sample, etc.
\end{excdesc}

\begin{funcdesc}{add}{fragment1, fragment2, width}
Return a fragment which is the addition of the two samples passed as
parameters.  \var{width} is the sample width in bytes, either
\code{1}, \code{2} or \code{4}.  Both fragments should have the same
length.
\end{funcdesc}

\begin{funcdesc}{adpcm2lin}{adpcmfragment, width, state}
Decode an Intel/DVI ADPCM coded fragment to a linear fragment.  See
the description of \function{lin2adpcm()} for details on ADPCM coding.
Return a tuple \code{(\var{sample}, \var{newstate})} where the sample
has the width specified in \var{width}.
\end{funcdesc}

\begin{funcdesc}{adpcm32lin}{adpcmfragment, width, state}
Decode an alternative 3-bit ADPCM code.  See \function{lin2adpcm3()}
for details.
\end{funcdesc}

\begin{funcdesc}{avg}{fragment, width}
Return the average over all samples in the fragment.
\end{funcdesc}

\begin{funcdesc}{avgpp}{fragment, width}
Return the average peak-peak value over all samples in the fragment.
No filtering is done, so the usefulness of this routine is
questionable.
\end{funcdesc}

\begin{funcdesc}{bias}{fragment, width, bias}
Return a fragment that is the original fragment with a bias added to
each sample.
\end{funcdesc}

\begin{funcdesc}{cross}{fragment, width}
Return the number of zero crossings in the fragment passed as an
argument.
\end{funcdesc}

\begin{funcdesc}{findfactor}{fragment, reference}
Return a factor \var{F} such that
\code{rms(add(\var{fragment}, mul(\var{reference}, -\var{F})))} is
minimal, i.e., return the factor with which you should multiply
\var{reference} to make it match as well as possible to
\var{fragment}.  The fragments should both contain 2-byte samples.

The time taken by this routine is proportional to
\code{len(\var{fragment})}.
\end{funcdesc}

\begin{funcdesc}{findfit}{fragment, reference}
Try to match \var{reference} as well as possible to a portion of
\var{fragment} (which should be the longer fragment).  This is
(conceptually) done by taking slices out of \var{fragment}, using
\function{findfactor()} to compute the best match, and minimizing the
result.  The fragments should both contain 2-byte samples.  Return a
tuple \code{(\var{offset}, \var{factor})} where \var{offset} is the
(integer) offset into \var{fragment} where the optimal match started
and \var{factor} is the (floating-point) factor as per
\function{findfactor()}.
\end{funcdesc}

\begin{funcdesc}{findmax}{fragment, length}
Search \var{fragment} for a slice of length \var{length} samples (not
bytes!)\ with maximum energy, i.e., return \var{i} for which
\code{rms(fragment[i*2:(i+length)*2])} is maximal.  The fragments
should both contain 2-byte samples.

The routine takes time proportional to \code{len(\var{fragment})}.
\end{funcdesc}

\begin{funcdesc}{getsample}{fragment, width, index}
Return the value of sample \var{index} from the fragment.
\end{funcdesc}

\begin{funcdesc}{lin2lin}{fragment, width, newwidth}
Convert samples between 1-, 2- and 4-byte formats.
\end{funcdesc}

\begin{funcdesc}{lin2adpcm}{fragment, width, state}
Convert samples to 4 bit Intel/DVI ADPCM encoding.  ADPCM coding is an
adaptive coding scheme, whereby each 4 bit number is the difference
between one sample and the next, divided by a (varying) step.  The
Intel/DVI ADPCM algorithm has been selected for use by the IMA, so it
may well become a standard.

\var{state} is a tuple containing the state of the coder.  The coder
returns a tuple \code{(\var{adpcmfrag}, \var{newstate})}, and the
\var{newstate} should be passed to the next call of
\function{lin2adpcm()}.  In the initial call, \code{None} can be
passed as the state.  \var{adpcmfrag} is the ADPCM coded fragment
packed 2 4-bit values per byte.
\end{funcdesc}

\begin{funcdesc}{lin2adpcm3}{fragment, width, state}
This is an alternative ADPCM coder that uses only 3 bits per sample.
It is not compatible with the Intel/DVI ADPCM coder and its output is
not packed (due to laziness on the side of the author).  Its use is
discouraged.
\end{funcdesc}

\begin{funcdesc}{lin2ulaw}{fragment, width}
Convert samples in the audio fragment to u-LAW encoding and return
this as a Python string.  u-LAW is an audio encoding format whereby
you get a dynamic range of about 14 bits using only 8 bit samples.  It
is used by the Sun audio hardware, among others.
\end{funcdesc}

\begin{funcdesc}{minmax}{fragment, width}
Return a tuple consisting of the minimum and maximum values of all
samples in the sound fragment.
\end{funcdesc}

\begin{funcdesc}{max}{fragment, width}
Return the maximum of the \emph{absolute value} of all samples in a
fragment.
\end{funcdesc}

\begin{funcdesc}{maxpp}{fragment, width}
Return the maximum peak-peak value in the sound fragment.
\end{funcdesc}

\begin{funcdesc}{mul}{fragment, width, factor}
Return a fragment that has all samples in the original fragment
multiplied by the floating-point value \var{factor}.  Overflow is
silently ignored.
\end{funcdesc}

\begin{funcdesc}{ratecv}{fragment, width, nchannels, inrate, outrate,
                         state\optional{, weightA\optional{, weightB}}}
Convert the frame rate of the input fragment.

\var{state} is a tuple containing the state of the converter.  The
converter returns a tuple \code{(\var{newfragment}, \var{newstate})},
and \var{newstate} should be passed to the next call of
\function{ratecv()}.  The initial call should pass \code{None}
as the state.

The \var{weightA} and \var{weightB} arguments are parameters for a
simple digital filter and default to \code{1} and \code{0} respectively.
\end{funcdesc}

\begin{funcdesc}{reverse}{fragment, width}
Reverse the samples in a fragment and returns the modified fragment.
\end{funcdesc}

\begin{funcdesc}{rms}{fragment, width}
Return the root-mean-square of the fragment, i.e.
\begin{displaymath}
\catcode`_=8
\sqrt{\frac{\sum{{S_{i}}^{2}}}{n}}
\end{displaymath}
This is a measure of the power in an audio signal.
\end{funcdesc}

\begin{funcdesc}{tomono}{fragment, width, lfactor, rfactor} 
Convert a stereo fragment to a mono fragment.  The left channel is
multiplied by \var{lfactor} and the right channel by \var{rfactor}
before adding the two channels to give a mono signal.
\end{funcdesc}

\begin{funcdesc}{tostereo}{fragment, width, lfactor, rfactor}
Generate a stereo fragment from a mono fragment.  Each pair of samples
in the stereo fragment are computed from the mono sample, whereby left
channel samples are multiplied by \var{lfactor} and right channel
samples by \var{rfactor}.
\end{funcdesc}

\begin{funcdesc}{ulaw2lin}{fragment, width}
Convert sound fragments in u-LAW encoding to linearly encoded sound
fragments.  u-LAW encoding always uses 8 bits samples, so \var{width}
refers only to the sample width of the output fragment here.
\end{funcdesc}

Note that operations such as \function{mul()} or \function{max()} make
no distinction between mono and stereo fragments, i.e.\ all samples
are treated equal.  If this is a problem the stereo fragment should be
split into two mono fragments first and recombined later.  Here is an
example of how to do that:

\begin{verbatim}
def mul_stereo(sample, width, lfactor, rfactor):
    lsample = audioop.tomono(sample, width, 1, 0)
    rsample = audioop.tomono(sample, width, 0, 1)
    lsample = audioop.mul(sample, width, lfactor)
    rsample = audioop.mul(sample, width, rfactor)
    lsample = audioop.tostereo(lsample, width, 1, 0)
    rsample = audioop.tostereo(rsample, width, 0, 1)
    return audioop.add(lsample, rsample, width)
\end{verbatim}

If you use the ADPCM coder to build network packets and you want your
protocol to be stateless (i.e.\ to be able to tolerate packet loss)
you should not only transmit the data but also the state.  Note that
you should send the \var{initial} state (the one you passed to
\function{lin2adpcm()}) along to the decoder, not the final state (as
returned by the coder).  If you want to use \function{struct.struct()}
to store the state in binary you can code the first element (the
predicted value) in 16 bits and the second (the delta index) in 8.

The ADPCM coders have never been tried against other ADPCM coders,
only against themselves.  It could well be that I misinterpreted the
standards in which case they will not be interoperable with the
respective standards.

The \function{find*()} routines might look a bit funny at first sight.
They are primarily meant to do echo cancellation.  A reasonably
fast way to do this is to pick the most energetic piece of the output
sample, locate that in the input sample and subtract the whole output
sample from the input sample:

\begin{verbatim}
def echocancel(outputdata, inputdata):
    pos = audioop.findmax(outputdata, 800)    # one tenth second
    out_test = outputdata[pos*2:]
    in_test = inputdata[pos*2:]
    ipos, factor = audioop.findfit(in_test, out_test)
    # Optional (for better cancellation):
    # factor = audioop.findfactor(in_test[ipos*2:ipos*2+len(out_test)], 
    #              out_test)
    prefill = '\0'*(pos+ipos)*2
    postfill = '\0'*(len(inputdata)-len(prefill)-len(outputdata))
    outputdata = prefill + audioop.mul(outputdata,2,-factor) + postfill
    return audioop.add(inputdata, outputdata, 2)
\end{verbatim}

\section{\module{imageop} ---
         Manipulate raw image data}

\declaremodule{builtin}{imageop}
\modulesynopsis{Manipulate raw image data.}


The \module{imageop} module contains some useful operations on images.
It operates on images consisting of 8 or 32 bit pixels stored in
Python strings.  This is the same format as used by
\function{gl.lrectwrite()} and the \refmodule{imgfile} module.

The module defines the following variables and functions:

\begin{excdesc}{error}
This exception is raised on all errors, such as unknown number of bits
per pixel, etc.
\end{excdesc}


\begin{funcdesc}{crop}{image, psize, width, height, x0, y0, x1, y1}
Return the selected part of \var{image}, which should by
\var{width} by \var{height} in size and consist of pixels of
\var{psize} bytes. \var{x0}, \var{y0}, \var{x1} and \var{y1} are like
the \function{gl.lrectread()} parameters, i.e.\ the boundary is
included in the new image.  The new boundaries need not be inside the
picture.  Pixels that fall outside the old image will have their value
set to zero.  If \var{x0} is bigger than \var{x1} the new image is
mirrored.  The same holds for the y coordinates.
\end{funcdesc}

\begin{funcdesc}{scale}{image, psize, width, height, newwidth, newheight}
Return \var{image} scaled to size \var{newwidth} by \var{newheight}.
No interpolation is done, scaling is done by simple-minded pixel
duplication or removal.  Therefore, computer-generated images or
dithered images will not look nice after scaling.
\end{funcdesc}

\begin{funcdesc}{tovideo}{image, psize, width, height}
Run a vertical low-pass filter over an image.  It does so by computing
each destination pixel as the average of two vertically-aligned source
pixels.  The main use of this routine is to forestall excessive
flicker if the image is displayed on a video device that uses
interlacing, hence the name.
\end{funcdesc}

\begin{funcdesc}{grey2mono}{image, width, height, threshold}
Convert a 8-bit deep greyscale image to a 1-bit deep image by
thresholding all the pixels.  The resulting image is tightly packed and
is probably only useful as an argument to \function{mono2grey()}.
\end{funcdesc}

\begin{funcdesc}{dither2mono}{image, width, height}
Convert an 8-bit greyscale image to a 1-bit monochrome image using a
(simple-minded) dithering algorithm.
\end{funcdesc}

\begin{funcdesc}{mono2grey}{image, width, height, p0, p1}
Convert a 1-bit monochrome image to an 8 bit greyscale or color image.
All pixels that are zero-valued on input get value \var{p0} on output
and all one-value input pixels get value \var{p1} on output.  To
convert a monochrome black-and-white image to greyscale pass the
values \code{0} and \code{255} respectively.
\end{funcdesc}

\begin{funcdesc}{grey2grey4}{image, width, height}
Convert an 8-bit greyscale image to a 4-bit greyscale image without
dithering.
\end{funcdesc}

\begin{funcdesc}{grey2grey2}{image, width, height}
Convert an 8-bit greyscale image to a 2-bit greyscale image without
dithering.
\end{funcdesc}

\begin{funcdesc}{dither2grey2}{image, width, height}
Convert an 8-bit greyscale image to a 2-bit greyscale image with
dithering.  As for \function{dither2mono()}, the dithering algorithm
is currently very simple.
\end{funcdesc}

\begin{funcdesc}{grey42grey}{image, width, height}
Convert a 4-bit greyscale image to an 8-bit greyscale image.
\end{funcdesc}

\begin{funcdesc}{grey22grey}{image, width, height}
Convert a 2-bit greyscale image to an 8-bit greyscale image.
\end{funcdesc}

\begin{datadesc}{backward_compatible}
If set to 0, the functions in this module use a non-backward
compatible way of representing multi-byte pixels on little-endian
systems.  The SGI for which this module was originally written is a
big-endian system, so setting this variable will have no effect.
However, the code wasn't originally intended to run on anything else,
so it made assumptions about byte order which are not universal.
Setting this variable to 0 will cause the byte order to be reversed on
little-endian systems, so that it then is the same as on big-endian
systems.
\end{datadesc}

\section{\module{aifc} ---
         Read and write AIFF and AIFC files}

\declaremodule{standard}{aifc}
\modulesynopsis{Read and write audio files in AIFF or AIFC format.}


This module provides support for reading and writing AIFF and AIFF-C
files.  AIFF is Audio Interchange File Format, a format for storing
digital audio samples in a file.  AIFF-C is a newer version of the
format that includes the ability to compress the audio data.
\index{Audio Interchange File Format}
\index{AIFF}
\index{AIFF-C}

\strong{Caveat:}  Some operations may only work under IRIX; these will
raise \exception{ImportError} when attempting to import the
\module{cl} module, which is only available on IRIX.

Audio files have a number of parameters that describe the audio data.
The sampling rate or frame rate is the number of times per second the
sound is sampled.  The number of channels indicate if the audio is
mono, stereo, or quadro.  Each frame consists of one sample per
channel.  The sample size is the size in bytes of each sample.  Thus a
frame consists of \var{nchannels}*\var{samplesize} bytes, and a
second's worth of audio consists of
\var{nchannels}*\var{samplesize}*\var{framerate} bytes.

For example, CD quality audio has a sample size of two bytes (16
bits), uses two channels (stereo) and has a frame rate of 44,100
frames/second.  This gives a frame size of 4 bytes (2*2), and a
second's worth occupies 2*2*44100 bytes (176,400 bytes).

Module \module{aifc} defines the following function:

\begin{funcdesc}{open}{file\optional{, mode}}
Open an AIFF or AIFF-C file and return an object instance with
methods that are described below.  The argument \var{file} is either a
string naming a file or a file object.  \var{mode} must be \code{'r'}
or \code{'rb'} when the file must be opened for reading, or \code{'w'} 
or \code{'wb'} when the file must be opened for writing.  If omitted,
\code{\var{file}.mode} is used if it exists, otherwise \code{'rb'} is
used.  When used for writing, the file object should be seekable,
unless you know ahead of time how many samples you are going to write
in total and use \method{writeframesraw()} and \method{setnframes()}.
\end{funcdesc}

Objects returned by \function{open()} when a file is opened for
reading have the following methods:

\begin{methoddesc}[aifc]{getnchannels}{}
Return the number of audio channels (1 for mono, 2 for stereo).
\end{methoddesc}

\begin{methoddesc}[aifc]{getsampwidth}{}
Return the size in bytes of individual samples.
\end{methoddesc}

\begin{methoddesc}[aifc]{getframerate}{}
Return the sampling rate (number of audio frames per second).
\end{methoddesc}

\begin{methoddesc}[aifc]{getnframes}{}
Return the number of audio frames in the file.
\end{methoddesc}

\begin{methoddesc}[aifc]{getcomptype}{}
Return a four-character string describing the type of compression used
in the audio file.  For AIFF files, the returned value is
\code{'NONE'}.
\end{methoddesc}

\begin{methoddesc}[aifc]{getcompname}{}
Return a human-readable description of the type of compression used in
the audio file.  For AIFF files, the returned value is \code{'not
compressed'}.
\end{methoddesc}

\begin{methoddesc}[aifc]{getparams}{}
Return a tuple consisting of all of the above values in the above
order.
\end{methoddesc}

\begin{methoddesc}[aifc]{getmarkers}{}
Return a list of markers in the audio file.  A marker consists of a
tuple of three elements.  The first is the mark ID (an integer), the
second is the mark position in frames from the beginning of the data
(an integer), the third is the name of the mark (a string).
\end{methoddesc}

\begin{methoddesc}[aifc]{getmark}{id}
Return the tuple as described in \method{getmarkers()} for the mark
with the given \var{id}.
\end{methoddesc}

\begin{methoddesc}[aifc]{readframes}{nframes}
Read and return the next \var{nframes} frames from the audio file.  The
returned data is a string containing for each frame the uncompressed
samples of all channels.
\end{methoddesc}

\begin{methoddesc}[aifc]{rewind}{}
Rewind the read pointer.  The next \method{readframes()} will start from
the beginning.
\end{methoddesc}

\begin{methoddesc}[aifc]{setpos}{pos}
Seek to the specified frame number.
\end{methoddesc}

\begin{methoddesc}[aifc]{tell}{}
Return the current frame number.
\end{methoddesc}

\begin{methoddesc}[aifc]{close}{}
Close the AIFF file.  After calling this method, the object can no
longer be used.
\end{methoddesc}

Objects returned by \function{open()} when a file is opened for
writing have all the above methods, except for \method{readframes()} and
\method{setpos()}.  In addition the following methods exist.  The
\method{get*()} methods can only be called after the corresponding
\method{set*()} methods have been called.  Before the first
\method{writeframes()} or \method{writeframesraw()}, all parameters
except for the number of frames must be filled in.

\begin{methoddesc}[aifc]{aiff}{}
Create an AIFF file.  The default is that an AIFF-C file is created,
unless the name of the file ends in \code{'.aiff'} in which case the
default is an AIFF file.
\end{methoddesc}

\begin{methoddesc}[aifc]{aifc}{}
Create an AIFF-C file.  The default is that an AIFF-C file is created,
unless the name of the file ends in \code{'.aiff'} in which case the
default is an AIFF file.
\end{methoddesc}

\begin{methoddesc}[aifc]{setnchannels}{nchannels}
Specify the number of channels in the audio file.
\end{methoddesc}

\begin{methoddesc}[aifc]{setsampwidth}{width}
Specify the size in bytes of audio samples.
\end{methoddesc}

\begin{methoddesc}[aifc]{setframerate}{rate}
Specify the sampling frequency in frames per second.
\end{methoddesc}

\begin{methoddesc}[aifc]{setnframes}{nframes}
Specify the number of frames that are to be written to the audio file.
If this parameter is not set, or not set correctly, the file needs to
support seeking.
\end{methoddesc}

\begin{methoddesc}[aifc]{setcomptype}{type, name}
Specify the compression type.  If not specified, the audio data will
not be compressed.  In AIFF files, compression is not possible.  The
name parameter should be a human-readable description of the
compression type, the type parameter should be a four-character
string.  Currently the following compression types are supported:
NONE, ULAW, ALAW, G722.
\index{u-LAW}
\index{A-LAW}
\index{G.722}
\end{methoddesc}

\begin{methoddesc}[aifc]{setparams}{nchannels, sampwidth, framerate, comptype, compname}
Set all the above parameters at once.  The argument is a tuple
consisting of the various parameters.  This means that it is possible
to use the result of a \method{getparams()} call as argument to
\method{setparams()}.
\end{methoddesc}

\begin{methoddesc}[aifc]{setmark}{id, pos, name}
Add a mark with the given id (larger than 0), and the given name at
the given position.  This method can be called at any time before
\method{close()}.
\end{methoddesc}

\begin{methoddesc}[aifc]{tell}{}
Return the current write position in the output file.  Useful in
combination with \method{setmark()}.
\end{methoddesc}

\begin{methoddesc}[aifc]{writeframes}{data}
Write data to the output file.  This method can only be called after
the audio file parameters have been set.
\end{methoddesc}

\begin{methoddesc}[aifc]{writeframesraw}{data}
Like \method{writeframes()}, except that the header of the audio file
is not updated.
\end{methoddesc}

\begin{methoddesc}[aifc]{close}{}
Close the AIFF file.  The header of the file is updated to reflect the
actual size of the audio data. After calling this method, the object
can no longer be used.
\end{methoddesc}

\section{\module{jpeg} ---
         Read and write JPEG files}

\declaremodule{builtin}{jpeg}
  \platform{IRIX}
\modulesynopsis{Read and write image files in compressed JPEG format.}


The module \module{jpeg} provides access to the jpeg compressor and
decompressor written by the Independent JPEG Group
\index{Independent JPEG Group}(IJG). JPEG is a standard for
compressing pictures; it is defined in ISO 10918.  For details on JPEG
or the Independent JPEG Group software refer to the JPEG standard or
the documentation provided with the software.

A portable interface to JPEG image files is available with the Python
Imaging Library (PIL) by Fredrik Lundh.  Information on PIL is
available at \url{http://www.pythonware.com/products/pil/}.
\index{Python Imaging Library}
\index{PIL (the Python Imaging Library)}
\index{Lundh, Fredrik}

The \module{jpeg} module defines an exception and some functions.

\begin{excdesc}{error}
Exception raised by \function{compress()} and \function{decompress()}
in case of errors.
\end{excdesc}

\begin{funcdesc}{compress}{data, w, h, b}
Treat data as a pixmap of width \var{w} and height \var{h}, with
\var{b} bytes per pixel.  The data is in SGI GL order, so the first
pixel is in the lower-left corner. This means that \function{gl.lrectread()}
return data can immediately be passed to \function{compress()}.
Currently only 1 byte and 4 byte pixels are allowed, the former being
treated as greyscale and the latter as RGB color.
\function{compress()} returns a string that contains the compressed
picture, in JFIF\index{JFIF} format.
\end{funcdesc}

\begin{funcdesc}{decompress}{data}
Data is a string containing a picture in JFIF\index{JFIF} format. It
returns a tuple \code{(\var{data}, \var{width}, \var{height},
\var{bytesperpixel})}.  Again, the data is suitable to pass to
\function{gl.lrectwrite()}.
\end{funcdesc}

\begin{funcdesc}{setoption}{name, value}
Set various options.  Subsequent \function{compress()} and
\function{decompress()} calls will use these options.  The following
options are available:

\begin{tableii}{l|p{3in}}{code}{Option}{Effect}
  \lineii{'forcegray'}{%
    Force output to be grayscale, even if input is RGB.}
  \lineii{'quality'}{%
    Set the quality of the compressed image to a value between
    \code{0} and \code{100} (default is \code{75}).  This only affects
    compression.}
  \lineii{'optimize'}{%
    Perform Huffman table optimization.  Takes longer, but results in
    smaller compressed image.  This only affects compression.}
  \lineii{'smooth'}{%
    Perform inter-block smoothing on uncompressed image.  Only useful
    for low-quality images.  This only affects decompression.}
\end{tableii}
\end{funcdesc}


\begin{seealso}
  \seetitle{JPEG Still Image Data Compression Standard}{The 
            canonical reference for the JPEG image format, by
            Pennebaker and Mitchell.}

  \seetitle[http://www.w3.org/Graphics/JPEG/itu-t81.pdf]{Information
            Technology - Digital Compression and Coding of
            Continuous-tone Still Images - Requirements and
            Guidelines}{The ISO standard for JPEG is also published as
            ITU T.81.  This is available online in PDF form.}
\end{seealso}

\section{\module{rgbimg} ---
         Read and write ``SGI RGB'' files}

\declaremodule{builtin}{rgbimg}
\modulesynopsis{Read and write image files in ``SGI RGB'' format (the module is
\emph{not} SGI specific though!).}


The \module{rgbimg} module allows Python programs to access SGI imglib image
files (also known as \file{.rgb} files).  The module is far from
complete, but is provided anyway since the functionality that there is
enough in some cases.  Currently, colormap files are not supported.

The module defines the following variables and functions:

\begin{excdesc}{error}
This exception is raised on all errors, such as unsupported file type, etc.
\end{excdesc}

\begin{funcdesc}{sizeofimage}{file}
This function returns a tuple \code{(\var{x}, \var{y})} where
\var{x} and \var{y} are the size of the image in pixels.
Only 4 byte RGBA pixels, 3 byte RGB pixels, and 1 byte greyscale pixels
are currently supported.
\end{funcdesc}

\begin{funcdesc}{longimagedata}{file}
This function reads and decodes the image on the specified file, and
returns it as a Python string. The string has 4 byte RGBA pixels.
The bottom left pixel is the first in
the string. This format is suitable to pass to \function{gl.lrectwrite()},
for instance.
\end{funcdesc}

\begin{funcdesc}{longstoimage}{data, x, y, z, file}
This function writes the RGBA data in \var{data} to image
file \var{file}. \var{x} and \var{y} give the size of the image.
\var{z} is 1 if the saved image should be 1 byte greyscale, 3 if the
saved image should be 3 byte RGB data, or 4 if the saved images should
be 4 byte RGBA data.  The input data always contains 4 bytes per pixel.
These are the formats returned by \function{gl.lrectread()}.
\end{funcdesc}

\begin{funcdesc}{ttob}{flag}
This function sets a global flag which defines whether the scan lines
of the image are read or written from bottom to top (flag is zero,
compatible with SGI GL) or from top to bottom(flag is one,
compatible with X).  The default is zero.
\end{funcdesc}


\chapter{Cryptographic Services}
\label{crypto}
\index{cryptography}

The modules described in this chapter implement various algorithms of
a cryptographic nature.  They are available at the discretion of the
installation.  Here's an overview:

\localmoduletable

Hardcore cypherpunks will probably find the cryptographic modules
written by A.M. Kuchling of further interest; the package adds
built-in modules for DES and IDEA encryption, provides a Python module
for reading and decrypting PGP files, and then some.  These modules
are not distributed with Python but available separately.  See the URL
\url{http://www.amk.ca/python/code/crypto.html} 
for more information.
\index{PGP}
\index{Pretty Good Privacy}
\indexii{DES}{cipher}
\indexii{IDEA}{cipher}
\index{cryptography}
\index{Kuchling, Andrew}
		% Cryptographic Services
\section{Built-in module \sectcode{md5}}
\bimodindex{md5}

This module implements the interface to RSA's MD5 message digest
algorithm (see also the file \file{md5.doc}). Its use is quite
straightforward:\ use the function \code{new} to create an
\dfn{md5}-object. You can now ``feed'' this object with arbitrary
strings.

At any time you can ask for the ``final'' digest of the object. Internally,
a temporary copy of the object is made and the digest is computed and
returned. Because of the copy, the digest operation is not destructive
for the object. Before a more exact description of the module's use, a small
example will be helpful: 
to obtain the digest of the string \code{'abc'}, use \ldots

\bcode\begin{verbatim}
>>> import md5
>>> m = md5.new()
>>> m.update('abc')
>>> m.digest()
'\220\001P\230<\322O\260\326\226?}(\341\177r'
\end{verbatim}\ecode

More condensed:

\bcode\begin{verbatim}
>>> md5.new('abc').digest()
'\220\001P\230<\322O\260\326\226?}(\341\177r'
\end{verbatim}\ecode

\renewcommand{\indexsubitem}{(in module md5)}

\begin{funcdesc}{new}{\optional{arg}}
  Create a new md5-object. If \var{arg} is present, an initial
  \code{update} method is called with \var{arg} as argument.
\end{funcdesc}

\begin{funcdesc}{md5}{\optional{arg}}
For backward compatibility reasons, this is an alternative name for the
\code{new} function.
\end{funcdesc}

An md5-object has the following methods:

\renewcommand{\indexsubitem}{(md5 method)}
\begin{funcdesc}{update}{arg}
  Update this md5-object with the string \var{arg}.
\end{funcdesc}

\begin{funcdesc}{digest}{}
% XXX The following is not quite clear; what does MD5Final do?
  Return the \dfn{digest} of this md5-object. Internally, a copy is made
  and the \C-function \code{MD5Final} is called. Finally the digest is
  returned.
\end{funcdesc}

\begin{funcdesc}{copy}{}
  Return a separate copy of this md5-object.  An \code{update} to this
  copy won't affect the original object.
\end{funcdesc}

\section{Built-in Module \sectcode{mpz}}
\label{module-mpz}
\bimodindex{mpz}

This is an optional module.  It is only available when Python is
configured to include it, which requires that the GNU MP software is
installed.

This module implements the interface to part of the GNU MP library,
which defines arbitrary precision integer and rational number
arithmetic routines.  Only the interfaces to the \emph{integer}
(\samp{mpz_{\rm \ldots}}) routines are provided. If not stated
otherwise, the description in the GNU MP documentation can be applied.

In general, \dfn{mpz}-numbers can be used just like other standard
Python numbers, e.g.\ you can use the built-in operators like \code{+},
\code{*}, etc., as well as the standard built-in functions like
\code{abs}, \code{int}, \ldots, \code{divmod}, \code{pow}.
\strong{Please note:} the {\it bitwise-xor} operation has been implemented as
a bunch of {\it and}s, {\it invert}s and {\it or}s, because the library
lacks an \code{mpz_xor} function, and I didn't need one.

You create an mpz-number by calling the function called \code{mpz} (see
below for an exact description). An mpz-number is printed like this:
\code{mpz(\var{value})}.

\renewcommand{\indexsubitem}{(in module mpz)}
\begin{funcdesc}{mpz}{value}
  Create a new mpz-number. \var{value} can be an integer, a long,
  another mpz-number, or even a string. If it is a string, it is
  interpreted as an array of radix-256 digits, least significant digit
  first, resulting in a positive number. See also the \code{binary}
  method, described below.
\end{funcdesc}

A number of \emph{extra} functions are defined in this module. Non
mpz-arguments are converted to mpz-values first, and the functions
return mpz-numbers.

\begin{funcdesc}{powm}{base\, exponent\, modulus}
  Return \code{pow(\var{base}, \var{exponent}) \%{} \var{modulus}}. If
  \code{\var{exponent} == 0}, return \code{mpz(1)}. In contrast to the
  \C-library function, this version can handle negative exponents.
\end{funcdesc}

\begin{funcdesc}{gcd}{op1\, op2}
  Return the greatest common divisor of \var{op1} and \var{op2}.
\end{funcdesc}

\begin{funcdesc}{gcdext}{a\, b}
  Return a tuple \code{(\var{g}, \var{s}, \var{t})}, such that
  \code{\var{a}*\var{s} + \var{b}*\var{t} == \var{g} == gcd(\var{a}, \var{b})}.
\end{funcdesc}

\begin{funcdesc}{sqrt}{op}
  Return the square root of \var{op}. The result is rounded towards zero.
\end{funcdesc}

\begin{funcdesc}{sqrtrem}{op}
  Return a tuple \code{(\var{root}, \var{remainder})}, such that
  \code{\var{root}*\var{root} + \var{remainder} == \var{op}}.
\end{funcdesc}

\begin{funcdesc}{divm}{numerator\, denominator\, modulus}
  Returns a number \var{q}. such that
  \code{\var{q} * \var{denominator} \%{} \var{modulus} == \var{numerator}}.
  One could also implement this function in Python, using \code{gcdext}.
\end{funcdesc}

An mpz-number has one method:

\renewcommand{\indexsubitem}{(mpz method)}
\begin{funcdesc}{binary}{}
  Convert this mpz-number to a binary string, where the number has been
  stored as an array of radix-256 digits, least significant digit first.

  The mpz-number must have a value greater than or equal to zero,
  otherwise a \code{ValueError}-exception will be raised.
\end{funcdesc}

\section{Built-in module \sectcode{rotor}}
\bimodindex{rotor}

This module implements a rotor-based encryption algorithm, contributed
by Lance Ellinghouse.  Currently no further documentation is available
--- you are kindly advised to read the source...


%\chapter{Amoeba Specific Services}

\section{\module{amoeba} ---
         Amoeba system support}

\declaremodule{builtin}{amoeba}
  \platform{Amoeba}
\modulesynopsis{Functions for the Amoeba operating system.}


This module provides some object types and operations useful for
Amoeba applications.  It is only available on systems that support
Amoeba operations.  RPC errors and other Amoeba errors are reported as
the exception \code{amoeba.error = 'amoeba.error'}.

The module \module{amoeba} defines the following items:

\begin{funcdesc}{name_append}{path, cap}
Stores a capability in the Amoeba directory tree.
Arguments are the pathname (a string) and the capability (a capability
object as returned by
\function{name_lookup()}).
\end{funcdesc}

\begin{funcdesc}{name_delete}{path}
Deletes a capability from the Amoeba directory tree.
Argument is the pathname.
\end{funcdesc}

\begin{funcdesc}{name_lookup}{path}
Looks up a capability.
Argument is the pathname.
Returns a
\dfn{capability}
object, to which various interesting operations apply, described below.
\end{funcdesc}

\begin{funcdesc}{name_replace}{path, cap}
Replaces a capability in the Amoeba directory tree.
Arguments are the pathname and the new capability.
(This differs from
\function{name_append()}
in the behavior when the pathname already exists:
\function{name_append()}
finds this an error while
\function{name_replace()}
allows it, as its name suggests.)
\end{funcdesc}

\begin{datadesc}{capv}
A table representing the capability environment at the time the
interpreter was started.
(Alas, modifying this table does not affect the capability environment
of the interpreter.)
For example,
\code{amoeba.capv['ROOT']}
is the capability of your root directory, similar to
\code{getcap("ROOT")}
in C.
\end{datadesc}

\begin{excdesc}{error}
The exception raised when an Amoeba function returns an error.
The value accompanying this exception is a pair containing the numeric
error code and the corresponding string, as returned by the C function
\cfunction{err_why()}.
\end{excdesc}

\begin{funcdesc}{timeout}{msecs}
Sets the transaction timeout, in milliseconds.
Returns the previous timeout.
Initially, the timeout is set to 2 seconds by the Python interpreter.
\end{funcdesc}

\subsection{Capability Operations}

Capabilities are written in a convenient \ASCII{} format, also used by the
Amoeba utilities
\emph{c2a}(U)
and
\emph{a2c}(U).
For example:

\begin{verbatim}
>>> amoeba.name_lookup('/profile/cap')
aa:1c:95:52:6a:fa/14(ff)/8e:ba:5b:8:11:1a
>>> 
\end{verbatim}
%
The following methods are defined for capability objects.

\setindexsubitem{(capability method)}
\begin{funcdesc}{dir_list}{}
Returns a list of the names of the entries in an Amoeba directory.
\end{funcdesc}

\begin{funcdesc}{b_read}{offset, maxsize}
Reads (at most)
\var{maxsize}
bytes from a bullet file at offset
\var{offset.}
The data is returned as a string.
EOF is reported as an empty string.
\end{funcdesc}

\begin{funcdesc}{b_size}{}
Returns the size of a bullet file.
\end{funcdesc}

\begin{funcdesc}{dir_append}{}
\funcline{dir_delete}{}
\funcline{dir_lookup}{}
\funcline{dir_replace}{}
Like the corresponding
\samp{name_}*
functions, but with a path relative to the capability.
(For paths beginning with a slash the capability is ignored, since this
is the defined semantics for Amoeba.)
\end{funcdesc}

\begin{funcdesc}{std_info}{}
Returns the standard info string of the object.
\end{funcdesc}

\begin{funcdesc}{tod_gettime}{}
Returns the time (in seconds since the Epoch, in UCT, as for \POSIX{}) from
a time server.
\end{funcdesc}

\begin{funcdesc}{tod_settime}{t}
Sets the time kept by a time server.
\end{funcdesc}
		% AMOEBA ONLY

\section{Introduction}
\label{intro}

The modules in this manual are available on the Apple Macintosh only.

Aside from the modules described here there are also interfaces to
various MacOS toolboxes, which are currently not extensively
described. The toolboxes for which modules exist are:
\module{AE} (Apple Events),
\module{Cm} (Component Manager),
\module{Ctl} (Control Manager),
\module{Dlg} (Dialog Manager),
\module{Evt} (Event Manager),
\module{Fm} (Font Manager),
\module{List} (List Manager),
\module{Menu} (Moenu Manager),
\module{Qd} (QuickDraw),
\module{Qt} (QuickTime),
\module{Res} (Resource Manager and Handles),
\module{Scrap} (Scrap Manager),
\module{Snd} (Sound Manager),
\module{TE} (TextEdit),
\module{Waste} (non-Apple \program{TextEdit} replacement) and
\module{Win} (Window Manager).

If applicable the module will define a number of Python objects for
the various structures declared by the toolbox, and operations will be
implemented as methods of the object. Other operations will be
implemented as functions in the module. Not all operations possible in
\C{} will also be possible in Python (callbacks are often a problem), and
parameters will occasionally be different in Python (input and output
buffers, especially). All methods and functions have a \code{__doc__}
string describing their arguments and return values, and for
additional description you are referred to \citetitle{Inside
Macintosh} or similar works.

The following modules are documented here:

\localmoduletable


\section{\module{mac} ---
         Implementations for the \module{os} module}

\declaremodule{builtin}{mac}
  \platform{Mac}
\modulesynopsis{Implementations for the \module{os} module.}


This module implements the operating system dependent functionality
provided by the standard module \module{os}\refstmodindex{os}.  It is
best accessed through the \module{os} module.

The following functions are available in this module:
\function{chdir()},
\function{close()},
\function{dup()},
\function{fdopen()},
\function{getcwd()},
\function{lseek()},
\function{listdir()},
\function{mkdir()},
\function{open()},
\function{read()},
\function{rename()},
\function{rmdir()},
\function{stat()},
\function{sync()},
\function{unlink()},
\function{write()},
as well as the exception \exception{error}. Note that the times
returned by \function{stat()} are floating-point values, like all time
values in MacPython.

One additional function is available:

\begin{funcdesc}{xstat}{path}
  This function returns the same information as \function{stat()}, but
  with three additional values appended: the size of the resource fork
  of the file and its 4-character creator and type.
\end{funcdesc}


\section{\module{macpath} ---
         MacOS path manipulation functions}

\declaremodule{standard}{macpath}
% Could be labeled \platform{Mac}, but the module should work anywhere and
% is distributed with the standard library.
\modulesynopsis{MacOS path manipulation functions.}


This module is the Macintosh implementation of the \module{os.path}
module.  It is most portably accessed as
\module{os.path}\refstmodindex{os.path}.  Refer to the
\citetitle[../lib/lib.html]{Python Library Reference} for
documentation of \module{os.path}.

The following functions are available in this module:
\function{normcase()},
\function{normpath()},
\function{isabs()},
\function{join()},
\function{split()},
\function{isdir()},
\function{isfile()},
\function{walk()},
\function{exists()}.
For other functions available in \module{os.path} dummy counterparts
are available.
			% MACINTOSH ONLY
\section{Built-in Module \sectcode{ctb}}
\bimodindex{ctb}
\renewcommand{\indexsubitem}{(in module ctb)}

This module provides a partial interface to the Macintosh
Communications Toolbox. Currently, only Connection Manager tools are
supported. 

\begin{datadesc}{error}
The exception raised on errors.
\end{datadesc}

\begin{datadesc}{cmData}
\dataline{cmCntl}
\dataline{cmAttn}
Flags for the \var{channel} argument of the \var{Read} and \var{Write}
methods.
\end{datadesc}

\begin{datadesc}{cmFlagsEOM}
End-of-message flag for \var{Read} and \var{Write}.
\end{datadesc}

\begin{datadesc}{choose*}
Values returned by \var{Choose}.
\end{datadesc}

\begin{datadesc}{cmStatus*}
Bits in the status as returned by \var{Status}.
\end{datadesc}

\begin{funcdesc}{available}{}
Return 1 if the communication toolbox is available, zero otherwise.
\end{funcdesc}

\begin{funcdesc}{CMNew}{name\, sizes}
Create a connection object using the connection tool named
\var{name}. \var{sizes} is a 6-tuple given buffer sizes for data in,
data out, control in, control out, attention in and attention out.
Alternatively, passing \code{None} will result in default buffer sizes.
\end{funcdesc}

\subsection{connection object}
For all connection methods that take a \var{timeout} argument, a value
of \code{-1} is indefinite, meaning that the command runs to completion.

\renewcommand{\indexsubitem}{(connection object attribute)}

\begin{datadesc}{callback}
If this member is set to a value other than \code{None} it should point
to a function accepting a single argument (the connection
object). This will make all connection object methods work
asynchronously, with the callback routine being called upon
completion.

{\em Note:} for reasons beyond my understanding the callback routine
is currently never called. You are advised against using asynchronous
calls for the time being.
\end{datadesc}


\renewcommand{\indexsubitem}{(connection object method)}

\begin{funcdesc}{Open}{timeout}
Open an outgoing connection, waiting at most \var{timeout} seconds for
the connection to be established.
\end{funcdesc}

\begin{funcdesc}{Listen}{timeout}
Wait for an incoming connection. Stop waiting after \var{timeout}
seconds. This call is only meaningful to some tools.
\end{funcdesc}

\begin{funcdesc}{accept}{yesno}
Accept (when \var{yesno} is non-zero) or reject an incoming call after
\var{Listen} returned.
\end{funcdesc}

\begin{funcdesc}{Close}{timeout\, now}
Close a connection. When \var{now} is zero, the close is orderly
(i.e.\ outstanding output is flushed, etc.)\ with a timeout of
\var{timeout} seconds. When \var{now} is non-zero the close is
immediate, discarding output.
\end{funcdesc}

\begin{funcdesc}{Read}{len\, chan\, timeout}
Read \var{len} bytes, or until \var{timeout} seconds have passed, from
the channel \var{chan} (which is one of \var{cmData}, \var{cmCntl} or
\var{cmAttn}). Return a 2-tuple:\ the data read and the end-of-message
flag.
\end{funcdesc}

\begin{funcdesc}{Write}{buf\, chan\, timeout\, eom}
Write \var{buf} to channel \var{chan}, aborting after \var{timeout}
seconds. When \var{eom} has the value \var{cmFlagsEOM} an
end-of-message indicator will be written after the data (if this
concept has a meaning for this communication tool). The method returns
the number of bytes written.
\end{funcdesc}

\begin{funcdesc}{Status}{}
Return connection status as the 2-tuple \code{(\var{sizes},
\var{flags})}. \var{sizes} is a 6-tuple giving the actual buffer sizes used
(see \var{CMNew}), \var{flags} is a set of bits describing the state
of the connection.
\end{funcdesc}

\begin{funcdesc}{GetConfig}{}
Return the configuration string of the communication tool. These
configuration strings are tool-dependent, but usually easily parsed
and modified.
\end{funcdesc}

\begin{funcdesc}{SetConfig}{str}
Set the configuration string for the tool. The strings are parsed
left-to-right, with later values taking precedence. This means
individual configuration parameters can be modified by simply appending
something like \code{'baud 4800'} to the end of the string returned by
\var{GetConfig} and passing that to this method. The method returns
the number of characters actually parsed by the tool before it
encountered an error (or completed successfully).
\end{funcdesc}

\begin{funcdesc}{Choose}{}
Present the user with a dialog to choose a communication tool and
configure it. If there is an outstanding connection some choices (like
selecting a different tool) may cause the connection to be
aborted. The return value (one of the \var{choose*} constants) will
indicate this.
\end{funcdesc}

\begin{funcdesc}{Idle}{}
Give the tool a chance to use the processor. You should call this
method regularly.
\end{funcdesc}

\begin{funcdesc}{Abort}{}
Abort an outstanding asynchronous \var{Open} or \var{Listen}.
\end{funcdesc}

\begin{funcdesc}{Reset}{}
Reset a connection. Exact meaning depends on the tool.
\end{funcdesc}

\begin{funcdesc}{Break}{length}
Send a break. Whether this means anything, what it means and
interpretation of the \var{length} parameter depend on the tool in
use.
\end{funcdesc}

\section{\module{macconsole} ---
         Think C's console package}

\declaremodule{builtin}{macconsole}
  \platform{Mac}
\modulesynopsis{Think C's console package.}


This module is available on the Macintosh, provided Python has been
built using the Think C compiler. It provides an interface to the
Think console package, with which basic text windows can be created.

\begin{datadesc}{options}
An object allowing you to set various options when creating windows,
see below.
\end{datadesc}

\begin{datadesc}{C_ECHO}
\dataline{C_NOECHO}
\dataline{C_CBREAK}
\dataline{C_RAW}
Options for the \code{setmode} method. \constant{C_ECHO} and
\constant{C_CBREAK} enable character echo, the other two disable it,
\constant{C_ECHO} and \constant{C_NOECHO} enable line-oriented input
(erase/kill processing, etc).
\end{datadesc}

\begin{funcdesc}{copen}{}
Open a new console window. Return a console window object.
\end{funcdesc}

\begin{funcdesc}{fopen}{fp}
Return the console window object corresponding with the given file
object. \var{fp} should be one of \code{sys.stdin}, \code{sys.stdout} or
\code{sys.stderr}.
\end{funcdesc}

\subsection{macconsole options object}
These options are examined when a window is created:

\setindexsubitem{(macconsole option)}
\begin{datadesc}{top}
\dataline{left}
The origin of the window.
\end{datadesc}

\begin{datadesc}{nrows}
\dataline{ncols}
The size of the window.
\end{datadesc}

\begin{datadesc}{txFont}
\dataline{txSize}
\dataline{txStyle}
The font, fontsize and fontstyle to be used in the window.
\end{datadesc}

\begin{datadesc}{title}
The title of the window.
\end{datadesc}

\begin{datadesc}{pause_atexit}
If set non-zero, the window will wait for user action before closing.
\end{datadesc}

\subsection{console window object}

\setindexsubitem{(console window attribute)}

\begin{datadesc}{file}
The file object corresponding to this console window. If the file is
buffered, you should call \code{\var{file}.flush()} between
\code{write()} and \code{read()} calls.
\end{datadesc}

\setindexsubitem{(console window method)}

\begin{funcdesc}{setmode}{mode}
Set the input mode of the console to \constant{C_ECHO}, etc.
\end{funcdesc}

\begin{funcdesc}{settabs}{n}
Set the tabsize to \var{n} spaces.
\end{funcdesc}

\begin{funcdesc}{cleos}{}
Clear to end-of-screen.
\end{funcdesc}

\begin{funcdesc}{cleol}{}
Clear to end-of-line.
\end{funcdesc}

\begin{funcdesc}{inverse}{onoff}
Enable inverse-video mode:\ characters with the high bit set are
displayed in inverse video (this disables the upper half of a
non-\ASCII{} character set).
\end{funcdesc}

\begin{funcdesc}{gotoxy}{x, y}
Set the cursor to position \code{(\var{x}, \var{y})}.
\end{funcdesc}

\begin{funcdesc}{hide}{}
Hide the window, remembering the contents.
\end{funcdesc}

\begin{funcdesc}{show}{}
Show the window again.
\end{funcdesc}

\begin{funcdesc}{echo2printer}{}
Copy everything written to the window to the printer as well.
\end{funcdesc}


\section{Built-in Module \sectcode{macdnr}}
\bimodindex{macdnr}

This module provides an interface to the Macintosh Domain Name
Resolver. It is usually used in conjunction with the \var{mactcp} module, to
map hostnames to IP-addresses.

The \code{macdnr} module defines the following functions:

\renewcommand{\indexsubitem}{(in module macdnr)}

\begin{funcdesc}{Open}{\optional{filename}}
Open the domain name resolver extension. If \var{filename} is given it
should be the pathname of the extension, otherwise a default is
used. Normally, this call is not needed since the other calls will
open the extension automatically.
\end{funcdesc}

\begin{funcdesc}{Close}{}
Close the resolver extension. Again, not needed for normal use.
\end{funcdesc}

\begin{funcdesc}{StrToAddr}{hostname}
Look up the IP address for \var{hostname}. This call returns a dnr
result object of the ``address'' variation.
\end{funcdesc}

\begin{funcdesc}{AddrToName}{addr}
Do a reverse lookup on the 32-bit integer IP-address
\var{addr}. Returns a dnr result object of the ``address'' variation.
\end{funcdesc}

\begin{funcdesc}{AddrToStr}{addr}
Convert the 32-bit integer IP-address \var{addr} to a dotted-decimal
string. Returns the string.
\end{funcdesc}

\begin{funcdesc}{HInfo}{hostname}
Query the nameservers for a \code{HInfo} record for host
\var{hostname}. These records contain hardware and software
information about the machine in question (if they are available in
the first place). Returns a dnr result object of the ``hinfo''
variety.
\end{funcdesc}

\begin{funcdesc}{MXInfo}{domain}
Query the nameservers for a mail exchanger for \var{domain}. This is
the hostname of a host willing to accept SMTP mail for the given
domain. Returns a dnr result object of the ``mx'' variety.
\end{funcdesc}

\subsection{dnr result object}

Since the DNR calls all execute asynchronously you do not get the
results back immedeately. In stead, you get a dnr result object. You
can check this object to see whether the query is complete, and access
its attributes to obtain the information when it is.

Alternatively, you can also reference the result attributes directly,
this will result in an implicit wait for the query to complete.

The \var{rtnCode} and \var{cname} attributes are always available, the
others depend on the type of query (address, hinfo or mx).

\renewcommand{\indexsubitem}{(dnr result object method)}

% Add args, as in {arg1\, arg2 \optional{\, arg3}}
\begin{funcdesc}{wait}{}
Wait for the query to complete.
\end{funcdesc}

% Add args, as in {arg1\, arg2 \optional{\, arg3}}
\begin{funcdesc}{isdone}{}
Return 1 if the query is complete.
\end{funcdesc}

\renewcommand{\indexsubitem}{(dnr result object attribute)}

\begin{datadesc}{rtnCode}
The error code returned by the query.
\end{datadesc}

\begin{datadesc}{cname}
The canonical name of the host that was queried.
\end{datadesc}

\begin{datadesc}{ip0}
\dataline{ip1}
\dataline{ip2}
\dataline{ip3}
At most four integer IP addresses for this host. Unused entries are
zero. Valid only for address queries.
\end{datadesc}

\begin{datadesc}{cpuType}
\dataline{osType}
Textual strings giving the machine type an OS name. Valid for hinfo
queries.
\end{datadesc}

\begin{datadesc}{exchange}
The name of a mail-exchanger host. Valid for mx queries.
\end{datadesc}

\begin{datadesc}{preference}
The preference of this mx record. Not too useful, since the Macintosh
will only return a single mx record. Mx queries only.
\end{datadesc}

The simplest way to use the module to convert names to dotted-decimal
strings, without worrying about idle time, etc:
\begin{verbatim}
>>> def gethostname(name):
...     import macdnr
...     dnrr = macdnr.StrToAddr(name)
...     return macdnr.AddrToStr(dnrr.ip0)
\end{verbatim}

\section{Built-in Module \sectcode{macfs}}
\bimodindex{macfs}

\renewcommand{\indexsubitem}{(in module macfs)}

This module provides access to macintosh FSSpec handling, the Alias
Manager, finder aliases and the Standard File package.

Whenever a function or method expects a \var{file} argument, this
argument can be one of three things:\ (1) a full or partial Macintosh
pathname, (2) an FSSpec object or (3) a 3-tuple \code{(wdRefNum,
parID, name)} as described in Inside Mac VI\@. A description of aliases
and the standard file package can also be found there.

\begin{funcdesc}{FSSpec}{file}
Create an FSSpec object for the specified file.
\end{funcdesc}

\begin{funcdesc}{RawFSSpec}{data}
Create an FSSpec object given the raw data for the C structure for the
FSSpec as a string.  This is mainly useful if you have obtained an
FSSpec structure over a network.
\end{funcdesc}

\begin{funcdesc}{RawAlias}{data}
Create an Alias object given the raw data for the C structure for the
alias as a string.  This is mainly useful if you have obtained an
FSSpec structure over a network.
\end{funcdesc}

\begin{funcdesc}{FInfo}{}
Create a zero-filled FInfo object.
\end{funcdesc}

\begin{funcdesc}{ResolveAliasFile}{file}
Resolve an alias file. Returns a 3-tuple \code{(\var{fsspec}, \var{isfolder},
\var{aliased})} where \var{fsspec} is the resulting FSSpec object,
\var{isfolder} is true if \var{fsspec} points to a folder and
\var{aliased} is true if the file was an alias in the first place
(otherwise the FSSpec object for the file itself is returned).
\end{funcdesc}

\begin{funcdesc}{StandardGetFile}{\optional{type\, ...}}
Present the user with a standard ``open input file''
dialog. Optionally, you can pass up to four 4-char file types to limit
the files the user can choose from. The function returns an FSSpec
object and a flag indicating that the user completed the dialog
without cancelling.
\end{funcdesc}

\begin{funcdesc}{StandardPutFile}{prompt\, \optional{default}}
Present the user with a standard ``open output file''
dialog. \var{prompt} is the prompt string, and the optional
\var{default} argument initializes the output file name. The function
returns an FSSpec object and a flag indicating that the user completed
the dialog without cancelling.
\end{funcdesc}

\begin{funcdesc}{GetDirectory}{}
Present the user with a non-standard ``select a directory''
dialog. Return an FSSpec object and a success-indicator.
\end{funcdesc}

\begin{funcdesc}{FindFolder}{where\, which\, create}
Locates one of the ``special'' folders that MacOS knows about, such as
the trash or the Preferences folder. \var{Where} is the disk to search
(\code{0x8000} for the boot disk), \var{which} is the 4-char string
specifying which folder to locate. Setting \var{create} causes the
folder to be created if it does not exist. Returns a \code{(vrefnum,
dirid)} tuple. See Inside Mac VI for a complete description, including
4-char names.
\end{funcdesc}

\subsection{FSSpec objects}

\renewcommand{\indexsubitem}{(FSSpec object attribute)}
\begin{datadesc}{data}
The raw data from the FSSpec object, suitable for passing
to other applications, for instance.
\end{datadesc}

\renewcommand{\indexsubitem}{(FSSpec object method)}
\begin{funcdesc}{as_pathname}{}
Return the full pathname of the file described by the FSSpec object.
\end{funcdesc}

\begin{funcdesc}{as_tuple}{}
Return the \code{(\var{wdRefNum}, \var{parID}, \var{name})} tuple of the file described
by the FSSpec object.
\end{funcdesc}

\begin{funcdesc}{NewAlias}{\optional{file}}
Create an Alias object pointing to the file described by this
FSSpec. If the optional \var{file} parameter is present the alias
will be relative to that file, otherwise it will be absolute.
\end{funcdesc}

\begin{funcdesc}{NewAliasMinimal}{}
Create a minimal alias pointing to this file.
\end{funcdesc}

\begin{funcdesc}{GetCreatorType}{}
Return the 4-char creator and type of the file.
\end{funcdesc}

\begin{funcdesc}{SetCreatorType}{creator\, type}
Set the 4-char creator and type of the file.
\end{funcdesc}

\begin{funcdesc}{GetFInfo}{}
Return a FInfo object describing the finder info for the file.
\end{funcdesc}

\begin{funcdesc}{SetFInfo}{finfo}
Set the finder info for the file to the values specified in the
\var{finfo} object.
\end{funcdesc}

\subsection{alias objects}

\renewcommand{\indexsubitem}{(alias object attribute)}
\begin{datadesc}{data}
The raw data for the Alias record, suitable for storing in a resource
or transmitting to other programs.
\end{datadesc}

\renewcommand{\indexsubitem}{(alias object method)}
\begin{funcdesc}{Resolve}{\optional{file}}
Resolve the alias. If the alias was created as a relative alias you
should pass the file relative to which it is. Return the FSSpec for
the file pointed to and a flag indicating whether the alias object
itself was modified during the search process. 
\end{funcdesc}

\begin{funcdesc}{GetInfo}{num}
An interface to the C routine \code{GetAliasInfo()}.
\end{funcdesc}

\begin{funcdesc}{Update}{file\, \optional{file2}}
Update the alias to point to the \var{file} given. If \var{file2} is
present a relative alias will be created.
\end{funcdesc}

Note that it is currently not possible to directly manipulate a resource
as an alias object. Hence, after calling \var{Update} or after
\var{Resolve} indicates that the alias has changed the Python program
is responsible for getting the \var{data} from the alias object and
modifying the resource.


\subsection{FInfo objects}

See Inside Mac for a complete description of what the various fields
mean.

\renewcommand{\indexsubitem}{(FInfo object attribute)}
\begin{datadesc}{Creator}
The 4-char creator code of the file.
\end{datadesc}

\begin{datadesc}{Type}
The 4-char type code of the file.
\end{datadesc}

\begin{datadesc}{Flags}
The finder flags for the file as 16-bit integer.
\end{datadesc}

\begin{datadesc}{Location}
A Point giving the position of the file's icon in its folder.
\end{datadesc}

\begin{datadesc}{Fldr}
The folder the file is in (as an integer).
\end{datadesc}

\section{Built-in Module \sectcode{mactcp}}
\label{module-mactcp}
\bimodindex{mactcp}

\setindexsubitem{(in module mactcp)}

This module provides an interface to the Macintosh TCP/IP driver
MacTCP\@. There is an accompanying module \code{macdnr} which provides an
interface to the name-server (allowing you to translate hostnames to
ip-addresses), a module \code{MACTCPconst} which has symbolic names for
constants constants used by MacTCP. Since the builtin module
\code{socket} is also available on the mac it is usually easier to use
sockets in stead of the mac-specific MacTCP API.

A complete description of the MacTCP interface can be found in the
Apple MacTCP API documentation.

\begin{funcdesc}{MTU}{}
Return the Maximum Transmit Unit (the packet size) of the network
interface.
\end{funcdesc}

\begin{funcdesc}{IPAddr}{}
Return the 32-bit integer IP address of the network interface.
\end{funcdesc}

\begin{funcdesc}{NetMask}{}
Return the 32-bit integer network mask of the interface.
\end{funcdesc}

\begin{funcdesc}{TCPCreate}{size}
Create a TCP Stream object. \var{size} is the size of the receive
buffer, \code{4096} is suggested by various sources.
\end{funcdesc}

\begin{funcdesc}{UDPCreate}{size, port}
Create a UDP stream object. \var{size} is the size of the receive
buffer (and, hence, the size of the biggest datagram you can receive
on this port). \var{port} is the UDP port number you want to receive
datagrams on, a value of zero will make MacTCP select a free port.
\end{funcdesc}

\subsection{TCP Stream Objects}

\setindexsubitem{(TCP stream attribute)}

\begin{datadesc}{asr}
When set to a value different than \code{None} this should point to a
function with two integer parameters:\ an event code and a detail. This
function will be called upon network-generated events such as urgent
data arrival. In addition, it is called with eventcode
\code{MACTCP.PassiveOpenDone} when a \code{PassiveOpen} completes. This
is a Python addition to the MacTCP semantics.
It is safe to do further calls from the \code{asr}.
\end{datadesc}

\setindexsubitem{(TCP stream method)}

\begin{funcdesc}{PassiveOpen}{port}
Wait for an incoming connection on TCP port \var{port} (zero makes the
system pick a free port). The call returns immediately, and you should
use \var{wait} to wait for completion. You should not issue any method
calls other than
\code{wait}, \code{isdone} or \code{GetSockName} before the call
completes.
\end{funcdesc}

\begin{funcdesc}{wait}{}
Wait for \code{PassiveOpen} to complete.
\end{funcdesc}

\begin{funcdesc}{isdone}{}
Return 1 if a \code{PassiveOpen} has completed.
\end{funcdesc}

\begin{funcdesc}{GetSockName}{}
Return the TCP address of this side of a connection as a 2-tuple
\code{(host, port)}, both integers.
\end{funcdesc}

\begin{funcdesc}{ActiveOpen}{lport, host, rport}
Open an outgoing connection to TCP address \code{(\var{host}, \var{rport})}. Use
local port \var{lport} (zero makes the system pick a free port). This
call blocks until the connection has been established.
\end{funcdesc}

\begin{funcdesc}{Send}{buf, push, urgent}
Send data \var{buf} over the connection. \var{Push} and \var{urgent}
are flags as specified by the TCP standard.
\end{funcdesc}

\begin{funcdesc}{Rcv}{timeout}
Receive data. The call returns when \var{timeout} seconds have passed
or when (according to the MacTCP documentation) ``a reasonable amount
of data has been received''. The return value is a 3-tuple
\code{(\var{data}, \var{urgent}, \var{mark})}. If urgent data is outstanding \code{Rcv}
will always return that before looking at any normal data. The first
call returning urgent data will have the \var{urgent} flag set, the
last will have the \var{mark} flag set.
\end{funcdesc}

\begin{funcdesc}{Close}{}
Tell MacTCP that no more data will be transmitted on this
connection. The call returns when all data has been acknowledged by
the receiving side.
\end{funcdesc}

\begin{funcdesc}{Abort}{}
Forcibly close both sides of a connection, ignoring outstanding data.
\end{funcdesc}

\begin{funcdesc}{Status}{}
Return a TCP status object for this stream giving the current status
(see below).
\end{funcdesc}

\subsection{TCP Status Objects}
This object has no methods, only some members holding information on
the connection. A complete description of all fields in this objects
can be found in the Apple documentation. The most interesting ones are:

\setindexsubitem{(TCP status attribute)}

\begin{datadesc}{localHost}
\dataline{localPort}
\dataline{remoteHost}
\dataline{remotePort}
The integer IP-addresses and port numbers of both endpoints of the
connection. 
\end{datadesc}

\begin{datadesc}{sendWindow}
The current window size.
\end{datadesc}

\begin{datadesc}{amtUnackedData}
The number of bytes sent but not yet acknowledged. \code{sendWindow -
amtUnackedData} is what you can pass to \code{Send} without blocking.
\end{datadesc}

\begin{datadesc}{amtUnreadData}
The number of bytes received but not yet read (what you can \code{Recv}
without blocking).
\end{datadesc}



\subsection{UDP Stream Objects}
Note that, unlike the name suggests, there is nothing stream-like
about UDP.

\setindexsubitem{(UDP stream attribute)}

\begin{datadesc}{asr}
The asynchronous service routine to be called on events such as
datagram arrival without outstanding \code{Read} call. The \code{asr} has a
single argument, the event code.
\end{datadesc}

\begin{datadesc}{port}
A read-only member giving the port number of this UDP stream.
\end{datadesc}

\setindexsubitem{(UDP stream method)}

\begin{funcdesc}{Read}{timeout}
Read a datagram, waiting at most \var{timeout} seconds (-1 is
infinite).  Return the data.
\end{funcdesc}

\begin{funcdesc}{Write}{host, port, buf}
Send \var{buf} as a datagram to IP-address \var{host}, port
\var{port}.
\end{funcdesc}

\section{Built-in Module \sectcode{macspeech}}
\label{module-macspeech}
\bimodindex{macspeech}

\renewcommand{\indexsubitem}{(in module macspeech)}

This module provides an interface to the Macintosh Speech Manager,
allowing you to let the Macintosh utter phrases. You need a version of
the speech manager extension (version 1 and 2 have been tested) in
your \code{Extensions} folder for this to work. The module does not
provide full access to all features of the Speech Manager yet.  It may
not be available in all Mac Python versions.

\begin{funcdesc}{Available}{}
Test availability of the Speech Manager extension (and, on the
PowerPC, the Speech Manager shared library). Return 0 or 1. 
\end{funcdesc}

\begin{funcdesc}{Version}{}
Return the (integer) version number of the Speech Manager.
\end{funcdesc}

\begin{funcdesc}{SpeakString}{str}
Utter the string \var{str} using the default voice,
asynchronously. This aborts any speech that may still be active from
prior \code{SpeakString} invocations.
\end{funcdesc}

\begin{funcdesc}{Busy}{}
Return the number of speech channels busy, system-wide.
\end{funcdesc}

\begin{funcdesc}{CountVoices}{}
Return the number of different voices available.
\end{funcdesc}

\begin{funcdesc}{GetIndVoice}{num}
Return a voice object for voice number \var{num}.
\end{funcdesc}

\subsection{voice objects}
Voice objects contain the description of a voice. It is currently not
yet possible to access the parameters of a voice.

\renewcommand{\indexsubitem}{(voice object method)}

\begin{funcdesc}{GetGender}{}
Return the gender of the voice: 0 for male, 1 for female and -1 for neuter.
\end{funcdesc}

\begin{funcdesc}{NewChannel}{}
Return a new speech channel object using this voice.
\end{funcdesc}

\subsection{speech channel objects}
A speech channel object allows you to speak strings with slightly more
control than \code{SpeakString()}, and allows you to use multiple
speakers at the same time. Please note that channel pitch and rate are
interrelated in some way, so that to make your Macintosh sing you will
have to adjust both.

\renewcommand{\indexsubitem}{(speech channel object method)}
\begin{funcdesc}{SpeakText}{str}
Start uttering the given string.
\end{funcdesc}

\begin{funcdesc}{Stop}{}
Stop babbling.
\end{funcdesc}

\begin{funcdesc}{GetPitch}{}
Return the current pitch of the channel, as a floating-point number.
\end{funcdesc}

\begin{funcdesc}{SetPitch}{pitch}
Set the pitch of the channel.
\end{funcdesc}

\begin{funcdesc}{GetRate}{}
Get the speech rate (utterances per minute) of the channel as a
floating point number.
\end{funcdesc}

\begin{funcdesc}{SetRate}{rate}
Set the speech rate of the channel.
\end{funcdesc}



\chapter{Standard Windowing Interface}

The modules in this chapter are available only on those systems where
the STDWIN library is available.  STDWIN runs on \UNIX{} under X11 and
on the Macintosh.  See CWI report CS-R8817.

\strong{Warning:} Using STDWIN is not recommended for new
applications.  It has never been ported to Microsoft Windows or
Windows NT, and for X11 or the Macintosh it lacks important
functionality --- in particular, it has no tools for the construction
of dialogs.  For most platforms, alternative, native solutions exist
(though none are currently documented in this manual): Tkinter for
\UNIX{} under X11, native Xt with Motif or Athena widgets for \UNIX{}
under X11, Win32 for Windows and Windows NT, and a collection of
native toolkit interfaces for the Macintosh.

\section{Built-in Module \sectcode{stdwin}}
\bimodindex{stdwin}

This module defines several new object types and functions that
provide access to the functionality of STDWIN.

On Unix running X11, it can only be used if the \code{DISPLAY}
environment variable is set or an explicit \samp{-display
\var{displayname}} argument is passed to the Python interpreter.

Functions have names that usually resemble their C STDWIN counterparts
with the initial `w' dropped.
Points are represented by pairs of integers; rectangles
by pairs of points.
For a complete description of STDWIN please refer to the documentation
of STDWIN for C programmers (aforementioned CWI report).

\subsection{Functions Defined in Module \sectcode{stdwin}}
\nodename{STDWIN Functions}

The following functions are defined in the \code{stdwin} module:

\renewcommand{\indexsubitem}{(in module stdwin)}
\begin{funcdesc}{open}{title}
Open a new window whose initial title is given by the string argument.
Return a window object; window object methods are described below.%
\footnote{The Python version of STDWIN does not support draw procedures; all
	drawing requests are reported as draw events.}
\end{funcdesc}

\begin{funcdesc}{getevent}{}
Wait for and return the next event.
An event is returned as a triple: the first element is the event
type, a small integer; the second element is the window object to which
the event applies, or
\code{None}
if it applies to no window in particular;
the third element is type-dependent.
Names for event types and command codes are defined in the standard
module
\code{stdwinevent}.
\end{funcdesc}

\begin{funcdesc}{pollevent}{}
Return the next event, if one is immediately available.
If no event is available, return \code{()}.
\end{funcdesc}

\begin{funcdesc}{getactive}{}
Return the window that is currently active, or \code{None} if no
window is currently active.  (This can be emulated by monitoring
WE_ACTIVATE and WE_DEACTIVATE events.)
\end{funcdesc}

\begin{funcdesc}{listfontnames}{pattern}
Return the list of font names in the system that match the pattern (a
string).  The pattern should normally be \code{'*'}; returns all
available fonts.  If the underlying window system is X11, other
patterns follow the standard X11 font selection syntax (as used e.g.
in resource definitions), i.e. the wildcard character \code{'*'}
matches any sequence of characters (including none) and \code{'?'}
matches any single character.
On the Macintosh this function currently returns an empty list.
\end{funcdesc}

\begin{funcdesc}{setdefscrollbars}{hflag\, vflag}
Set the flags controlling whether subsequently opened windows will
have horizontal and/or vertical scroll bars.
\end{funcdesc}

\begin{funcdesc}{setdefwinpos}{h\, v}
Set the default window position for windows opened subsequently.
\end{funcdesc}

\begin{funcdesc}{setdefwinsize}{width\, height}
Set the default window size for windows opened subsequently.
\end{funcdesc}

\begin{funcdesc}{getdefscrollbars}{}
Return the flags controlling whether subsequently opened windows will
have horizontal and/or vertical scroll bars.
\end{funcdesc}

\begin{funcdesc}{getdefwinpos}{}
Return the default window position for windows opened subsequently.
\end{funcdesc}

\begin{funcdesc}{getdefwinsize}{}
Return the default window size for windows opened subsequently.
\end{funcdesc}

\begin{funcdesc}{getscrsize}{}
Return the screen size in pixels.
\end{funcdesc}

\begin{funcdesc}{getscrmm}{}
Return the screen size in millimeters.
\end{funcdesc}

\begin{funcdesc}{fetchcolor}{colorname}
Return the pixel value corresponding to the given color name.
Return the default foreground color for unknown color names.
Hint: the following code tests whether you are on a machine that
supports more than two colors:
\bcode\begin{verbatim}
if stdwin.fetchcolor('black') <> \
          stdwin.fetchcolor('red') <> \
          stdwin.fetchcolor('white'):
    print 'color machine'
else:
    print 'monochrome machine'
\end{verbatim}\ecode
\end{funcdesc}

\begin{funcdesc}{setfgcolor}{pixel}
Set the default foreground color.
This will become the default foreground color of windows opened
subsequently, including dialogs.
\end{funcdesc}

\begin{funcdesc}{setbgcolor}{pixel}
Set the default background color.
This will become the default background color of windows opened
subsequently, including dialogs.
\end{funcdesc}

\begin{funcdesc}{getfgcolor}{}
Return the pixel value of the current default foreground color.
\end{funcdesc}

\begin{funcdesc}{getbgcolor}{}
Return the pixel value of the current default background color.
\end{funcdesc}

\begin{funcdesc}{setfont}{fontname}
Set the current default font.
This will become the default font for windows opened subsequently,
and is also used by the text measuring functions \code{textwidth},
\code{textbreak}, \code{lineheight} and \code{baseline} below.
This accepts two more optional parameters, size and style:
Size is the font size (in `points').
Style is a single character specifying the style, as follows:
\code{'b'} = bold,
\code{'i'} = italic,
\code{'o'} = bold + italic,
\code{'u'} = underline;
default style is roman.
Size and style are ignored under X11 but used on the Macintosh.
(Sorry for all this complexity --- a more uniform interface is being designed.)
\end{funcdesc}

\begin{funcdesc}{menucreate}{title}
Create a menu object referring to a global menu (a menu that appears in
all windows).
Methods of menu objects are described below.
Note: normally, menus are created locally; see the window method
\code{menucreate} below.
\strong{Warning:} the menu only appears in a window as long as the object
returned by this call exists.
\end{funcdesc}

\begin{funcdesc}{newbitmap}{width\, height}
Create a new bitmap object of the given dimensions.
Methods of bitmap objects are described below.
Not available on the Macintosh.
\end{funcdesc}

\begin{funcdesc}{fleep}{}
Cause a beep or bell (or perhaps a `visual bell' or flash, hence the
name).
\end{funcdesc}

\begin{funcdesc}{message}{string}
Display a dialog box containing the string.
The user must click OK before the function returns.
\end{funcdesc}

\begin{funcdesc}{askync}{prompt\, default}
Display a dialog that prompts the user to answer a question with yes or
no.
Return 0 for no, 1 for yes.
If the user hits the Return key, the default (which must be 0 or 1) is
returned.
If the user cancels the dialog, the
\code{KeyboardInterrupt}
exception is raised.
\end{funcdesc}

\begin{funcdesc}{askstr}{prompt\, default}
Display a dialog that prompts the user for a string.
If the user hits the Return key, the default string is returned.
If the user cancels the dialog, the
\code{KeyboardInterrupt}
exception is raised.
\end{funcdesc}

\begin{funcdesc}{askfile}{prompt\, default\, new}
Ask the user to specify a filename.
If
\var{new}
is zero it must be an existing file; otherwise, it must be a new file.
If the user cancels the dialog, the
\code{KeyboardInterrupt}
exception is raised.
\end{funcdesc}

\begin{funcdesc}{setcutbuffer}{i\, string}
Store the string in the system's cut buffer number
\var{i},
where it can be found (for pasting) by other applications.
On X11, there are 8 cut buffers (numbered 0..7).
Cut buffer number 0 is the `clipboard' on the Macintosh.
\end{funcdesc}

\begin{funcdesc}{getcutbuffer}{i}
Return the contents of the system's cut buffer number
\var{i}.
\end{funcdesc}

\begin{funcdesc}{rotatecutbuffers}{n}
On X11, rotate the 8 cut buffers by
\var{n}.
Ignored on the Macintosh.
\end{funcdesc}

\begin{funcdesc}{getselection}{i}
Return X11 selection number
\var{i.}
Selections are not cut buffers.
Selection numbers are defined in module
\code{stdwinevents}.
Selection \code{WS_PRIMARY} is the
\dfn{primary}
selection (used by
xterm,
for instance);
selection \code{WS_SECONDARY} is the
\dfn{secondary}
selection; selection \code{WS_CLIPBOARD} is the
\dfn{clipboard}
selection (used by
xclipboard).
On the Macintosh, this always returns an empty string.
\end{funcdesc}

\begin{funcdesc}{resetselection}{i}
Reset selection number
\var{i},
if this process owns it.
(See window method
\code{setselection()}).
\end{funcdesc}

\begin{funcdesc}{baseline}{}
Return the baseline of the current font (defined by STDWIN as the
vertical distance between the baseline and the top of the
characters).
\end{funcdesc}

\begin{funcdesc}{lineheight}{}
Return the total line height of the current font.
\end{funcdesc}

\begin{funcdesc}{textbreak}{str\, width}
Return the number of characters of the string that fit into a space of
\var{width}
bits wide when drawn in the curent font.
\end{funcdesc}

\begin{funcdesc}{textwidth}{str}
Return the width in bits of the string when drawn in the current font.
\end{funcdesc}

\begin{funcdesc}{connectionnumber}{}
\funcline{fileno}{}
(X11 under \UNIX{} only) Return the ``connection number'' used by the
underlying X11 implementation.  (This is normally the file number of
the socket.)  Both functions return the same value;
\code{connectionnumber()} is named after the corresponding function in
X11 and STDWIN, while \code{fileno()} makes it possible to use the
\code{stdwin} module as a ``file'' object parameter to
\code{select.select()}.  Note that if \code{select()} implies that
input is possible on \code{stdwin}, this does not guarantee that an
event is ready --- it may be some internal communication going on
between the X server and the client library.  Thus, you should call
\code{stdwin.pollevent()} until it returns \code{None} to check for
events if you don't want your program to block.  Because of internal
buffering in X11, it is also possible that \code{stdwin.pollevent()}
returns an event while \code{select()} does not find \code{stdwin} to
be ready, so you should read any pending events with
\code{stdwin.pollevent()} until it returns \code{None} before entering
a blocking \code{select()} call.
\ttindex{select}
\end{funcdesc}

\subsection{Window Objects}

Window objects are created by \code{stdwin.open()}.  They are closed
by their \code{close()} method or when they are garbage-collected.
Window objects have the following methods:

\renewcommand{\indexsubitem}{(window method)}

\begin{funcdesc}{begindrawing}{}
Return a drawing object, whose methods (described below) allow drawing
in the window.
\end{funcdesc}

\begin{funcdesc}{change}{rect}
Invalidate the given rectangle; this may cause a draw event.
\end{funcdesc}

\begin{funcdesc}{gettitle}{}
Returns the window's title string.
\end{funcdesc}

\begin{funcdesc}{getdocsize}{}
\begin{sloppypar}
Return a pair of integers giving the size of the document as set by
\code{setdocsize()}.
\end{sloppypar}
\end{funcdesc}

\begin{funcdesc}{getorigin}{}
Return a pair of integers giving the origin of the window with respect
to the document.
\end{funcdesc}

\begin{funcdesc}{gettitle}{}
Return the window's title string.
\end{funcdesc}

\begin{funcdesc}{getwinsize}{}
Return a pair of integers giving the size of the window.
\end{funcdesc}

\begin{funcdesc}{getwinpos}{}
Return a pair of integers giving the position of the window's upper
left corner (relative to the upper left corner of the screen).
\end{funcdesc}

\begin{funcdesc}{menucreate}{title}
Create a menu object referring to a local menu (a menu that appears
only in this window).
Methods of menu objects are described below.
{\bf Warning:} the menu only appears as long as the object
returned by this call exists.
\end{funcdesc}

\begin{funcdesc}{scroll}{rect\, point}
Scroll the given rectangle by the vector given by the point.
\end{funcdesc}

\begin{funcdesc}{setdocsize}{point}
Set the size of the drawing document.
\end{funcdesc}

\begin{funcdesc}{setorigin}{point}
Move the origin of the window (its upper left corner)
to the given point in the document.
\end{funcdesc}

\begin{funcdesc}{setselection}{i\, str}
Attempt to set X11 selection number
\var{i}
to the string
\var{str}.
(See stdwin method
\code{getselection()}
for the meaning of
\var{i}.)
Return true if it succeeds.
If  succeeds, the window ``owns'' the selection until
(a) another application takes ownership of the selection; or
(b) the window is deleted; or
(c) the application clears ownership by calling
\code{stdwin.resetselection(\var{i})}.
When another application takes ownership of the selection, a
\code{WE_LOST_SEL}
event is received for no particular window and with the selection number
as detail.
Ignored on the Macintosh.
\end{funcdesc}

\begin{funcdesc}{settimer}{dsecs}
Schedule a timer event for the window in
\code{\var{dsecs}/10}
seconds.
\end{funcdesc}

\begin{funcdesc}{settitle}{title}
Set the window's title string.
\end{funcdesc}

\begin{funcdesc}{setwincursor}{name}
\begin{sloppypar}
Set the window cursor to a cursor of the given name.
It raises the
\code{RuntimeError}
exception if no cursor of the given name exists.
Suitable names include
\code{'ibeam'},
\code{'arrow'},
\code{'cross'},
\code{'watch'}
and
\code{'plus'}.
On X11, there are many more (see
\file{<X11/cursorfont.h>}).
\end{sloppypar}
\end{funcdesc}

\begin{funcdesc}{setwinpos}{h\, v}
Set the the position of the window's upper left corner (relative to
the upper left corner of the screen).
\end{funcdesc}

\begin{funcdesc}{setwinsize}{width\, height}
Set the window's size.
\end{funcdesc}

\begin{funcdesc}{show}{rect}
Try to ensure that the given rectangle of the document is visible in
the window.
\end{funcdesc}

\begin{funcdesc}{textcreate}{rect}
Create a text-edit object in the document at the given rectangle.
Methods of text-edit objects are described below.
\end{funcdesc}

\begin{funcdesc}{setactive}{}
Attempt to make this window the active window.  If successful, this
will generate a WE_ACTIVATE event (and a WE_DEACTIVATE event in case
another window in this application became inactive).
\end{funcdesc}

\begin{funcdesc}{close}{}
Discard the window object.  It should not be used again.
\end{funcdesc}

\subsection{Drawing Objects}

Drawing objects are created exclusively by the window method
\code{begindrawing()}.
Only one drawing object can exist at any given time; the drawing object
must be deleted to finish drawing.
No drawing object may exist when
\code{stdwin.getevent()}
is called.
Drawing objects have the following methods:

\renewcommand{\indexsubitem}{(drawing method)}

\begin{funcdesc}{box}{rect}
Draw a box just inside a rectangle.
\end{funcdesc}

\begin{funcdesc}{circle}{center\, radius}
Draw a circle with given center point and radius.
\end{funcdesc}

\begin{funcdesc}{elarc}{center\, \(rh\, rv\)\, \(a1\, a2\)}
Draw an elliptical arc with given center point.
\code{(\var{rh}, \var{rv})}
gives the half sizes of the horizontal and vertical radii.
\code{(\var{a1}, \var{a2})}
gives the angles (in degrees) of the begin and end points.
0 degrees is at 3 o'clock, 90 degrees is at 12 o'clock.
\end{funcdesc}

\begin{funcdesc}{erase}{rect}
Erase a rectangle.
\end{funcdesc}

\begin{funcdesc}{fillcircle}{center\, radius}
Draw a filled circle with given center point and radius.
\end{funcdesc}

\begin{funcdesc}{fillelarc}{center\, \(rh\, rv\)\, \(a1\, a2\)}
Draw a filled elliptical arc; arguments as for \code{elarc}.
\end{funcdesc}

\begin{funcdesc}{fillpoly}{points}
Draw a filled polygon given by a list (or tuple) of points.
\end{funcdesc}

\begin{funcdesc}{invert}{rect}
Invert a rectangle.
\end{funcdesc}

\begin{funcdesc}{line}{p1\, p2}
Draw a line from point
\var{p1}
to
\var{p2}.
\end{funcdesc}

\begin{funcdesc}{paint}{rect}
Fill a rectangle.
\end{funcdesc}

\begin{funcdesc}{poly}{points}
Draw the lines connecting the given list (or tuple) of points.
\end{funcdesc}

\begin{funcdesc}{shade}{rect\, percent}
Fill a rectangle with a shading pattern that is about
\var{percent}
percent filled.
\end{funcdesc}

\begin{funcdesc}{text}{p\, str}
Draw a string starting at point p (the point specifies the
top left coordinate of the string).
\end{funcdesc}

\begin{funcdesc}{xorcircle}{center\, radius}
\funcline{xorelarc}{center\, \(rh\, rv\)\, \(a1\, a2\)}
\funcline{xorline}{p1\, p2}
\funcline{xorpoly}{points}
Draw a circle, an elliptical arc, a line or a polygon, respectively,
in XOR mode.
\end{funcdesc}

\begin{funcdesc}{setfgcolor}{}
\funcline{setbgcolor}{}
\funcline{getfgcolor}{}
\funcline{getbgcolor}{}
These functions are similar to the corresponding functions described
above for the
\code{stdwin}
module, but affect or return the colors currently used for drawing
instead of the global default colors.
When a drawing object is created, its colors are set to the window's
default colors, which are in turn initialized from the global default
colors when the window is created.
\end{funcdesc}

\begin{funcdesc}{setfont}{}
\funcline{baseline}{}
\funcline{lineheight}{}
\funcline{textbreak}{}
\funcline{textwidth}{}
These functions are similar to the corresponding functions described
above for the
\code{stdwin}
module, but affect or use the current drawing font instead of
the global default font.
When a drawing object is created, its font is set to the window's
default font, which is in turn initialized from the global default
font when the window is created.
\end{funcdesc}

\begin{funcdesc}{bitmap}{point\, bitmap\, mask}
Draw the \var{bitmap} with its top left corner at \var{point}.
If the optional \var{mask} argument is present, it should be either
the same object as \var{bitmap}, to draw only those bits that are set
in the bitmap, in the foreground color, or \code{None}, to draw all
bits (ones are drawn in the foreground color, zeros in the background
color).
Not available on the Macintosh.
\end{funcdesc}

\begin{funcdesc}{cliprect}{rect}
Set the ``clipping region'' to a rectangle.
The clipping region limits the effect of all drawing operations, until
it is changed again or until the drawing object is closed.  When a
drawing object is created the clipping region is set to the entire
window.  When an object to be drawn falls partly outside the clipping
region, the set of pixels drawn is the intersection of the clipping
region and the set of pixels that would be drawn by the same operation
in the absence of a clipping region.
\end{funcdesc}

\begin{funcdesc}{noclip}{}
Reset the clipping region to the entire window.
\end{funcdesc}

\begin{funcdesc}{close}{}
\funcline{enddrawing}{}
Discard the drawing object.  It should not be used again.
\end{funcdesc}

\subsection{Menu Objects}

A menu object represents a menu.
The menu is destroyed when the menu object is deleted.
The following methods are defined:

\renewcommand{\indexsubitem}{(menu method)}

\begin{funcdesc}{additem}{text\, shortcut}
Add a menu item with given text.
The shortcut must be a string of length 1, or omitted (to specify no
shortcut).
\end{funcdesc}

\begin{funcdesc}{setitem}{i\, text}
Set the text of item number
\var{i}.
\end{funcdesc}

\begin{funcdesc}{enable}{i\, flag}
Enable or disables item
\var{i}.
\end{funcdesc}

\begin{funcdesc}{check}{i\, flag}
Set or clear the
\dfn{check mark}
for item
\var{i}.
\end{funcdesc}

\begin{funcdesc}{close}{}
Discard the menu object.  It should not be used again.
\end{funcdesc}

\subsection{Bitmap Objects}

A bitmap represents a rectangular array of bits.
The top left bit has coordinate (0, 0).
A bitmap can be drawn with the \code{bitmap} method of a drawing object.
Bitmaps are currently not available on the Macintosh.

The following methods are defined:

\renewcommand{\indexsubitem}{(bitmap method)}

\begin{funcdesc}{getsize}{}
Return a tuple representing the width and height of the bitmap.
(This returns the values that have been passed to the \code{newbitmap}
function.)
\end{funcdesc}

\begin{funcdesc}{setbit}{point\, bit}
Set the value of the bit indicated by \var{point} to \var{bit}.
\end{funcdesc}

\begin{funcdesc}{getbit}{point}
Return the value of the bit indicated by \var{point}.
\end{funcdesc}

\begin{funcdesc}{close}{}
Discard the bitmap object.  It should not be used again.
\end{funcdesc}

\subsection{Text-edit Objects}

A text-edit object represents a text-edit block.
For semantics, see the STDWIN documentation for C programmers.
The following methods exist:

\renewcommand{\indexsubitem}{(text-edit method)}

\begin{funcdesc}{arrow}{code}
Pass an arrow event to the text-edit block.
The
\var{code}
must be one of
\code{WC_LEFT},
\code{WC_RIGHT},
\code{WC_UP}
or
\code{WC_DOWN}
(see module
\code{stdwinevents}).
\end{funcdesc}

\begin{funcdesc}{draw}{rect}
Pass a draw event to the text-edit block.
The rectangle specifies the redraw area.
\end{funcdesc}

\begin{funcdesc}{event}{type\, window\, detail}
Pass an event gotten from
\code{stdwin.getevent()}
to the text-edit block.
Return true if the event was handled.
\end{funcdesc}

\begin{funcdesc}{getfocus}{}
Return 2 integers representing the start and end positions of the
focus, usable as slice indices on the string returned by
\code{gettext()}.
\end{funcdesc}

\begin{funcdesc}{getfocustext}{}
Return the text in the focus.
\end{funcdesc}

\begin{funcdesc}{getrect}{}
Return a rectangle giving the actual position of the text-edit block.
(The bottom coordinate may differ from the initial position because
the block automatically shrinks or grows to fit.)
\end{funcdesc}

\begin{funcdesc}{gettext}{}
Return the entire text buffer.
\end{funcdesc}

\begin{funcdesc}{move}{rect}
Specify a new position for the text-edit block in the document.
\end{funcdesc}

\begin{funcdesc}{replace}{str}
Replace the text in the focus by the given string.
The new focus is an insert point at the end of the string.
\end{funcdesc}

\begin{funcdesc}{setfocus}{i\, j}
Specify the new focus.
Out-of-bounds values are silently clipped.
\end{funcdesc}

\begin{funcdesc}{settext}{str}
Replace the entire text buffer by the given string and set the focus
to \code{(0, 0)}.
\end{funcdesc}

\begin{funcdesc}{setview}{rect}
Set the view rectangle to \var{rect}.  If \var{rect} is \code{None},
viewing mode is reset.  In viewing mode, all output from the text-edit
object is clipped to the viewing rectangle.  This may be useful to
implement your own scrolling text subwindow.
\end{funcdesc}

\begin{funcdesc}{close}{}
Discard the text-edit object.  It should not be used again.
\end{funcdesc}

\subsection{Example}
\nodename{STDWIN Example}

Here is a minimal example of using STDWIN in Python.
It creates a window and draws the string ``Hello world'' in the top
left corner of the window.
The window will be correctly redrawn when covered and re-exposed.
The program quits when the close icon or menu item is requested.

\bcode\begin{verbatim}
import stdwin
from stdwinevents import *

def main():
    mywin = stdwin.open('Hello')
    #
    while 1:
        (type, win, detail) = stdwin.getevent()
        if type == WE_DRAW:
            draw = win.begindrawing()
            draw.text((0, 0), 'Hello, world')
            del draw
        elif type == WE_CLOSE:
            break

main()
\end{verbatim}\ecode

\section{Standard Module \sectcode{stdwinevents}}
\stmodindex{stdwinevents}

This module defines constants used by STDWIN for event types
(\code{WE_ACTIVATE} etc.), command codes (\code{WC_LEFT} etc.)
and selection types (\code{WS_PRIMARY} etc.).
Read the file for details.
Suggested usage is

\bcode\begin{verbatim}
>>> from stdwinevents import *
>>> 
\end{verbatim}\ecode

\section{Standard Module \sectcode{rect}}
\stmodindex{rect}

This module contains useful operations on rectangles.
A rectangle is defined as in module
\code{stdwin}:
a pair of points, where a point is a pair of integers.
For example, the rectangle

\bcode\begin{verbatim}
(10, 20), (90, 80)
\end{verbatim}\ecode

is a rectangle whose left, top, right and bottom edges are 10, 20, 90
and 80, respectively.
Note that the positive vertical axis points down (as in
\code{stdwin}).

The module defines the following objects:

\renewcommand{\indexsubitem}{(in module rect)}
\begin{excdesc}{error}
The exception raised by functions in this module when they detect an
error.
The exception argument is a string describing the problem in more
detail.
\end{excdesc}

\begin{datadesc}{empty}
The rectangle returned when some operations return an empty result.
This makes it possible to quickly check whether a result is empty:

\bcode\begin{verbatim}
>>> import rect
>>> r1 = (10, 20), (90, 80)
>>> r2 = (0, 0), (10, 20)
>>> r3 = rect.intersect([r1, r2])
>>> if r3 is rect.empty: print 'Empty intersection'
Empty intersection
>>> 
\end{verbatim}\ecode
\end{datadesc}

\begin{funcdesc}{is_empty}{r}
Returns true if the given rectangle is empty.
A rectangle
\code{(\var{left}, \var{top}), (\var{right}, \var{bottom})}
is empty if
\iftexi
\code{\var{left} >= \var{right}} or \code{\var{top} => \var{bottom}}.
\else
$\var{left} \geq \var{right}$ or $\var{top} \geq \var{bottom}$.
%%JHXXX{\em left~$\geq$~right} or {\em top~$\leq$~bottom}.
\fi
\end{funcdesc}

\begin{funcdesc}{intersect}{list}
Returns the intersection of all rectangles in the list argument.
It may also be called with a tuple argument.
Raises
\code{rect.error}
if the list is empty.
Returns
\code{rect.empty}
if the intersection of the rectangles is empty.
\end{funcdesc}

\begin{funcdesc}{union}{list}
Returns the smallest rectangle that contains all non-empty rectangles in
the list argument.
It may also be called with a tuple argument or with two or more
rectangles as arguments.
Returns
\code{rect.empty}
if the list is empty or all its rectangles are empty.
\end{funcdesc}

\begin{funcdesc}{pointinrect}{point\, rect}
Returns true if the point is inside the rectangle.
By definition, a point
\code{(\var{h}, \var{v})}
is inside a rectangle
\code{(\var{left}, \var{top}), (\var{right}, \var{bottom})} if
\iftexi
\code{\var{left} <= \var{h} < \var{right}} and
\code{\var{top} <= \var{v} < \var{bottom}}.
\else
$\var{left} \leq \var{h} < \var{right}$ and
$\var{top} \leq \var{v} < \var{bottom}$.
\fi
\end{funcdesc}

\begin{funcdesc}{inset}{rect\, \(dh\, dv\)}
Returns a rectangle that lies inside the
\code{rect}
argument by
\var{dh}
pixels horizontally
and
\var{dv}
pixels
vertically.
If
\var{dh}
or
\var{dv}
is negative, the result lies outside
\var{rect}.
\end{funcdesc}

\begin{funcdesc}{rect2geom}{rect}
Converts a rectangle to geometry representation:
\code{(\var{left}, \var{top}), (\var{width}, \var{height})}.
\end{funcdesc}

\begin{funcdesc}{geom2rect}{geom}
Converts a rectangle given in geometry representation back to the
standard rectangle representation
\code{(\var{left}, \var{top}), (\var{right}, \var{bottom})}.
\end{funcdesc}
		% STDWIN ONLY

\chapter{SGI IRIX Specific Services}
\label{sgi}

The modules described in this chapter provide interfaces to features
that are unique to SGI's IRIX operating system (versions 4 and 5).

\localmoduletable
			% SGI IRIX ONLY
\section{\module{al} ---
         Audio functions on the SGI}

\declaremodule{builtin}{al}
  \platform{IRIX}
\modulesynopsis{Audio functions on the SGI.}


This module provides access to the audio facilities of the SGI Indy
and Indigo workstations.  See section 3A of the IRIX man pages for
details.  You'll need to read those man pages to understand what these
functions do!  Some of the functions are not available in IRIX
releases before 4.0.5.  Again, see the manual to check whether a
specific function is available on your platform.

All functions and methods defined in this module are equivalent to
the C functions with \samp{AL} prefixed to their name.

Symbolic constants from the C header file \code{<audio.h>} are
defined in the standard module
\refmodule[al-constants]{AL}\refstmodindex{AL}, see below.

\strong{Warning:} the current version of the audio library may dump core
when bad argument values are passed rather than returning an error
status.  Unfortunately, since the precise circumstances under which
this may happen are undocumented and hard to check, the Python
interface can provide no protection against this kind of problems.
(One example is specifying an excessive queue size --- there is no
documented upper limit.)

The module defines the following functions:


\begin{funcdesc}{openport}{name, direction\optional{, config}}
The name and direction arguments are strings.  The optional
\var{config} argument is a configuration object as returned by
\function{newconfig()}.  The return value is an \dfn{audio port
object}; methods of audio port objects are described below.
\end{funcdesc}

\begin{funcdesc}{newconfig}{}
The return value is a new \dfn{audio configuration object}; methods of
audio configuration objects are described below.
\end{funcdesc}

\begin{funcdesc}{queryparams}{device}
The device argument is an integer.  The return value is a list of
integers containing the data returned by \cfunction{ALqueryparams()}.
\end{funcdesc}

\begin{funcdesc}{getparams}{device, list}
The \var{device} argument is an integer.  The list argument is a list
such as returned by \function{queryparams()}; it is modified in place
(!).
\end{funcdesc}

\begin{funcdesc}{setparams}{device, list}
The \var{device} argument is an integer.  The \var{list} argument is a
list such as returned by \function{queryparams()}.
\end{funcdesc}


\subsection{Configuration Objects \label{al-config-objects}}

Configuration objects (returned by \function{newconfig()} have the
following methods:

\begin{methoddesc}[audio configuration]{getqueuesize}{}
Return the queue size.
\end{methoddesc}

\begin{methoddesc}[audio configuration]{setqueuesize}{size}
Set the queue size.
\end{methoddesc}

\begin{methoddesc}[audio configuration]{getwidth}{}
Get the sample width.
\end{methoddesc}

\begin{methoddesc}[audio configuration]{setwidth}{width}
Set the sample width.
\end{methoddesc}

\begin{methoddesc}[audio configuration]{getchannels}{}
Get the channel count.
\end{methoddesc}

\begin{methoddesc}[audio configuration]{setchannels}{nchannels}
Set the channel count.
\end{methoddesc}

\begin{methoddesc}[audio configuration]{getsampfmt}{}
Get the sample format.
\end{methoddesc}

\begin{methoddesc}[audio configuration]{setsampfmt}{sampfmt}
Set the sample format.
\end{methoddesc}

\begin{methoddesc}[audio configuration]{getfloatmax}{}
Get the maximum value for floating sample formats.
\end{methoddesc}

\begin{methoddesc}[audio configuration]{setfloatmax}{floatmax}
Set the maximum value for floating sample formats.
\end{methoddesc}


\subsection{Port Objects \label{al-port-objects}}

Port objects, as returned by \function{openport()}, have the following
methods:

\begin{methoddesc}[audio port]{closeport}{}
Close the port.
\end{methoddesc}

\begin{methoddesc}[audio port]{getfd}{}
Return the file descriptor as an int.
\end{methoddesc}

\begin{methoddesc}[audio port]{getfilled}{}
Return the number of filled samples.
\end{methoddesc}

\begin{methoddesc}[audio port]{getfillable}{}
Return the number of fillable samples.
\end{methoddesc}

\begin{methoddesc}[audio port]{readsamps}{nsamples}
Read a number of samples from the queue, blocking if necessary.
Return the data as a string containing the raw data, (e.g., 2 bytes per
sample in big-endian byte order (high byte, low byte) if you have set
the sample width to 2 bytes).
\end{methoddesc}

\begin{methoddesc}[audio port]{writesamps}{samples}
Write samples into the queue, blocking if necessary.  The samples are
encoded as described for the \method{readsamps()} return value.
\end{methoddesc}

\begin{methoddesc}[audio port]{getfillpoint}{}
Return the `fill point'.
\end{methoddesc}

\begin{methoddesc}[audio port]{setfillpoint}{fillpoint}
Set the `fill point'.
\end{methoddesc}

\begin{methoddesc}[audio port]{getconfig}{}
Return a configuration object containing the current configuration of
the port.
\end{methoddesc}

\begin{methoddesc}[audio port]{setconfig}{config}
Set the configuration from the argument, a configuration object.
\end{methoddesc}

\begin{methoddesc}[audio port]{getstatus}{list}
Get status information on last error.
\end{methoddesc}


\section{\module{AL} ---
         Constants used with the \module{al} module}

\declaremodule[al-constants]{standard}{AL}
  \platform{IRIX}
\modulesynopsis{Constants used with the \module{al} module.}


This module defines symbolic constants needed to use the built-in
module \refmodule{al} (see above); they are equivalent to those defined
in the C header file \code{<audio.h>} except that the name prefix
\samp{AL_} is omitted.  Read the module source for a complete list of
the defined names.  Suggested use:

\begin{verbatim}
import al
from AL import *
\end{verbatim}

%\section{Built-in Module \sectcode{audio}}
\bimodindex{audio}

\strong{Note:} This module is obsolete, since the hardware to which it
interfaces is obsolete.  For audio on the Indigo or 4D/35, see
built-in module \code{al} above.

This module provides rudimentary access to the audio I/O device
\file{/dev/audio} on the Silicon Graphics Personal IRIS 4D/25;
see {\it audio}(7). It supports the following operations:

\renewcommand{\indexsubitem}{(in module audio)}
\begin{funcdesc}{setoutgain}{n}
Sets the output gain.
\iftexi
\code{0 <= \var{n} < 256}.
\else
$0 \leq \var{n} < 256$.
%%JHXXX Sets the output gain (0-255).
\fi
\end{funcdesc}

\begin{funcdesc}{getoutgain}{}
Returns the output gain.
\end{funcdesc}

\begin{funcdesc}{setrate}{n}
Sets the sampling rate: \code{1} = 32K/sec, \code{2} = 16K/sec,
\code{3} = 8K/sec.
\end{funcdesc}

\begin{funcdesc}{setduration}{n}
Sets the `sound duration' in units of 1/100 seconds.
\end{funcdesc}

\begin{funcdesc}{read}{n}
Reads a chunk of
\var{n}
sampled bytes from the audio input (line in or microphone).
The chunk is returned as a string of length n.
Each byte encodes one sample as a signed 8-bit quantity using linear
encoding.
This string can be converted to numbers using \code{chr2num()} described
below.
\end{funcdesc}

\begin{funcdesc}{write}{buf}
Writes a chunk of samples to the audio output (speaker).
\end{funcdesc}

These operations support asynchronous audio I/O:

\renewcommand{\indexsubitem}{(in module audio)}
\begin{funcdesc}{start_recording}{n}
Starts a second thread (a process with shared memory) that begins reading
\var{n}
bytes from the audio device.
The main thread immediately continues.
\end{funcdesc}

\begin{funcdesc}{wait_recording}{}
Waits for the second thread to finish and returns the data read.
\end{funcdesc}

\begin{funcdesc}{stop_recording}{}
Makes the second thread stop reading as soon as possible.
Returns the data read so far.
\end{funcdesc}

\begin{funcdesc}{poll_recording}{}
Returns true if the second thread has finished reading (so
\code{wait_recording()} would return the data without delay).
\end{funcdesc}

\begin{funcdesc}{start_playing}{}
\funcline{wait_playing}{}
\funcline{stop_playing}{}
\funcline{poll_playing}{}
\begin{sloppypar}
Similar but for output.
\code{stop_playing()}
returns a lower bound for the number of bytes actually played (not very
accurate).
\end{sloppypar}
\end{funcdesc}

The following operations do not affect the audio device but are
implemented in C for efficiency:

\renewcommand{\indexsubitem}{(in module audio)}
\begin{funcdesc}{amplify}{buf\, f1\, f2}
Amplifies a chunk of samples by a variable factor changing from
\code{\var{f1}/256} to \code{\var{f2}/256.}
Negative factors are allowed.
Resulting values that are to large to fit in a byte are clipped.         
\end{funcdesc}

\begin{funcdesc}{reverse}{buf}
Returns a chunk of samples backwards.
\end{funcdesc}

\begin{funcdesc}{add}{buf1\, buf2}
Bytewise adds two chunks of samples.
Bytes that exceed the range are clipped.
If one buffer is shorter, it is assumed to be padded with zeros.
\end{funcdesc}

\begin{funcdesc}{chr2num}{buf}
Converts a string of sampled bytes as returned by \code{read()} into
a list containing the numeric values of the samples.
\end{funcdesc}

\begin{funcdesc}{num2chr}{list}
\begin{sloppypar}
Converts a list as returned by
\code{chr2num()}
back to a buffer acceptable by
\code{write()}.
\end{sloppypar}
\end{funcdesc}

\section{\module{cd} ---
         CD-ROM access on SGI systems}

\declaremodule{builtin}{cd}
  \platform{IRIX}
\modulesynopsis{Interface to the CD-ROM on Silicon Graphics systems.}


This module provides an interface to the Silicon Graphics CD library.
It is available only on Silicon Graphics systems.

The way the library works is as follows.  A program opens the CD-ROM
device with \function{open()} and creates a parser to parse the data
from the CD with \function{createparser()}.  The object returned by
\function{open()} can be used to read data from the CD, but also to get
status information for the CD-ROM device, and to get information about
the CD, such as the table of contents.  Data from the CD is passed to
the parser, which parses the frames, and calls any callback
functions that have previously been added.

An audio CD is divided into \dfn{tracks} or \dfn{programs} (the terms
are used interchangeably).  Tracks can be subdivided into
\dfn{indices}.  An audio CD contains a \dfn{table of contents} which
gives the starts of the tracks on the CD.  Index 0 is usually the
pause before the start of a track.  The start of the track as given by
the table of contents is normally the start of index 1.

Positions on a CD can be represented in two ways.  Either a frame
number or a tuple of three values, minutes, seconds and frames.  Most
functions use the latter representation.  Positions can be both
relative to the beginning of the CD, and to the beginning of the
track.

Module \module{cd} defines the following functions and constants:


\begin{funcdesc}{createparser}{}
Create and return an opaque parser object.  The methods of the parser
object are described below.
\end{funcdesc}

\begin{funcdesc}{msftoframe}{minutes, seconds, frames}
Converts a \code{(\var{minutes}, \var{seconds}, \var{frames})} triple
representing time in absolute time code into the corresponding CD
frame number.
\end{funcdesc}

\begin{funcdesc}{open}{\optional{device\optional{, mode}}}
Open the CD-ROM device.  The return value is an opaque player object;
methods of the player object are described below.  The device is the
name of the SCSI device file, e.g. \code{'/dev/scsi/sc0d4l0'}, or
\code{None}.  If omitted or \code{None}, the hardware inventory is
consulted to locate a CD-ROM drive.  The \var{mode}, if not omited,
should be the string \code{'r'}.
\end{funcdesc}

The module defines the following variables:

\begin{excdesc}{error}
Exception raised on various errors.
\end{excdesc}

\begin{datadesc}{DATASIZE}
The size of one frame's worth of audio data.  This is the size of the
audio data as passed to the callback of type \code{audio}.
\end{datadesc}

\begin{datadesc}{BLOCKSIZE}
The size of one uninterpreted frame of audio data.
\end{datadesc}

The following variables are states as returned by
\function{getstatus()}:

\begin{datadesc}{READY}
The drive is ready for operation loaded with an audio CD.
\end{datadesc}

\begin{datadesc}{NODISC}
The drive does not have a CD loaded.
\end{datadesc}

\begin{datadesc}{CDROM}
The drive is loaded with a CD-ROM.  Subsequent play or read operations
will return I/O errors.
\end{datadesc}

\begin{datadesc}{ERROR}
An error occurred while trying to read the disc or its table of
contents.
\end{datadesc}

\begin{datadesc}{PLAYING}
The drive is in CD player mode playing an audio CD through its audio
jacks.
\end{datadesc}

\begin{datadesc}{PAUSED}
The drive is in CD layer mode with play paused.
\end{datadesc}

\begin{datadesc}{STILL}
The equivalent of \constant{PAUSED} on older (non 3301) model Toshiba
CD-ROM drives.  Such drives have never been shipped by SGI.
\end{datadesc}

\begin{datadesc}{audio}
\dataline{pnum}
\dataline{index}
\dataline{ptime}
\dataline{atime}
\dataline{catalog}
\dataline{ident}
\dataline{control}
Integer constants describing the various types of parser callbacks
that can be set by the \method{addcallback()} method of CD parser
objects (see below).
\end{datadesc}


\subsection{Player Objects}
\label{player-objects}

Player objects (returned by \function{open()}) have the following
methods:

\begin{methoddesc}[CD player]{allowremoval}{}
Unlocks the eject button on the CD-ROM drive permitting the user to
eject the caddy if desired.
\end{methoddesc}

\begin{methoddesc}[CD player]{bestreadsize}{}
Returns the best value to use for the \var{num_frames} parameter of
the \method{readda()} method.  Best is defined as the value that
permits a continuous flow of data from the CD-ROM drive.
\end{methoddesc}

\begin{methoddesc}[CD player]{close}{}
Frees the resources associated with the player object.  After calling
\method{close()}, the methods of the object should no longer be used.
\end{methoddesc}

\begin{methoddesc}[CD player]{eject}{}
Ejects the caddy from the CD-ROM drive.
\end{methoddesc}

\begin{methoddesc}[CD player]{getstatus}{}
Returns information pertaining to the current state of the CD-ROM
drive.  The returned information is a tuple with the following values:
\var{state}, \var{track}, \var{rtime}, \var{atime}, \var{ttime},
\var{first}, \var{last}, \var{scsi_audio}, \var{cur_block}.
\var{rtime} is the time relative to the start of the current track;
\var{atime} is the time relative to the beginning of the disc;
\var{ttime} is the total time on the disc.  For more information on
the meaning of the values, see the man page \manpage{CDgetstatus}{3dm}.
The value of \var{state} is one of the following: \constant{ERROR},
\constant{NODISC}, \constant{READY}, \constant{PLAYING},
\constant{PAUSED}, \constant{STILL}, or \constant{CDROM}.
\end{methoddesc}

\begin{methoddesc}[CD player]{gettrackinfo}{track}
Returns information about the specified track.  The returned
information is a tuple consisting of two elements, the start time of
the track and the duration of the track.
\end{methoddesc}

\begin{methoddesc}[CD player]{msftoblock}{min, sec, frame}
Converts a minutes, seconds, frames triple representing a time in
absolute time code into the corresponding logical block number for the
given CD-ROM drive.  You should use \function{msftoframe()} rather than
\method{msftoblock()} for comparing times.  The logical block number
differs from the frame number by an offset required by certain CD-ROM
drives.
\end{methoddesc}

\begin{methoddesc}[CD player]{play}{start, play}
Starts playback of an audio CD in the CD-ROM drive at the specified
track.  The audio output appears on the CD-ROM drive's headphone and
audio jacks (if fitted).  Play stops at the end of the disc.
\var{start} is the number of the track at which to start playing the
CD; if \var{play} is 0, the CD will be set to an initial paused
state.  The method \method{togglepause()} can then be used to commence
play.
\end{methoddesc}

\begin{methoddesc}[CD player]{playabs}{minutes, seconds, frames, play}
Like \method{play()}, except that the start is given in minutes,
seconds, and frames instead of a track number.
\end{methoddesc}

\begin{methoddesc}[CD player]{playtrack}{start, play}
Like \method{play()}, except that playing stops at the end of the
track.
\end{methoddesc}

\begin{methoddesc}[CD player]{playtrackabs}{track, minutes, seconds, frames, play}
Like \method{play()}, except that playing begins at the specified
absolute time and ends at the end of the specified track.
\end{methoddesc}

\begin{methoddesc}[CD player]{preventremoval}{}
Locks the eject button on the CD-ROM drive thus preventing the user
from arbitrarily ejecting the caddy.
\end{methoddesc}

\begin{methoddesc}[CD player]{readda}{num_frames}
Reads the specified number of frames from an audio CD mounted in the
CD-ROM drive.  The return value is a string representing the audio
frames.  This string can be passed unaltered to the
\method{parseframe()} method of the parser object.
\end{methoddesc}

\begin{methoddesc}[CD player]{seek}{minutes, seconds, frames}
Sets the pointer that indicates the starting point of the next read of
digital audio data from a CD-ROM.  The pointer is set to an absolute
time code location specified in \var{minutes}, \var{seconds}, and
\var{frames}.  The return value is the logical block number to which
the pointer has been set.
\end{methoddesc}

\begin{methoddesc}[CD player]{seekblock}{block}
Sets the pointer that indicates the starting point of the next read of
digital audio data from a CD-ROM.  The pointer is set to the specified
logical block number.  The return value is the logical block number to
which the pointer has been set.
\end{methoddesc}

\begin{methoddesc}[CD player]{seektrack}{track}
Sets the pointer that indicates the starting point of the next read of
digital audio data from a CD-ROM.  The pointer is set to the specified
track.  The return value is the logical block number to which the
pointer has been set.
\end{methoddesc}

\begin{methoddesc}[CD player]{stop}{}
Stops the current playing operation.
\end{methoddesc}

\begin{methoddesc}[CD player]{togglepause}{}
Pauses the CD if it is playing, and makes it play if it is paused.
\end{methoddesc}


\subsection{Parser Objects}
\label{cd-parser-objects}

Parser objects (returned by \function{createparser()}) have the
following methods:

\begin{methoddesc}[CD parser]{addcallback}{type, func, arg}
Adds a callback for the parser.  The parser has callbacks for eight
different types of data in the digital audio data stream.  Constants
for these types are defined at the \module{cd} module level (see above).
The callback is called as follows: \code{\var{func}(\var{arg}, type,
data)}, where \var{arg} is the user supplied argument, \var{type} is
the particular type of callback, and \var{data} is the data returned
for this \var{type} of callback.  The type of the data depends on the
\var{type} of callback as follows:

\begin{tableii}{l|p{4in}}{code}{Type}{Value}
  \lineii{audio}{String which can be passed unmodified to
\function{al.writesamps()}.}
  \lineii{pnum}{Integer giving the program (track) number.}
  \lineii{index}{Integer giving the index number.}
  \lineii{ptime}{Tuple consisting of the program time in minutes,
seconds, and frames.}
  \lineii{atime}{Tuple consisting of the absolute time in minutes,
seconds, and frames.}
  \lineii{catalog}{String of 13 characters, giving the catalog number
of the CD.}
  \lineii{ident}{String of 12 characters, giving the ISRC
identification number of the recording.  The string consists of two
characters country code, three characters owner code, two characters
giving the year, and five characters giving a serial number.}
  \lineii{control}{Integer giving the control bits from the CD
subcode data}
\end{tableii}
\end{methoddesc}

\begin{methoddesc}[CD parser]{deleteparser}{}
Deletes the parser and frees the memory it was using.  The object
should not be used after this call.  This call is done automatically
when the last reference to the object is removed.
\end{methoddesc}

\begin{methoddesc}[CD parser]{parseframe}{frame}
Parses one or more frames of digital audio data from a CD such as
returned by \method{readda()}.  It determines which subcodes are
present in the data.  If these subcodes have changed since the last
frame, then \method{parseframe()} executes a callback of the
appropriate type passing to it the subcode data found in the frame.
Unlike the \C{} function, more than one frame of digital audio data
can be passed to this method.
\end{methoddesc}

\begin{methoddesc}[CD parser]{removecallback}{type}
Removes the callback for the given \var{type}.
\end{methoddesc}

\begin{methoddesc}[CD parser]{resetparser}{}
Resets the fields of the parser used for tracking subcodes to an
initial state.  \method{resetparser()} should be called after the disc
has been changed.
\end{methoddesc}

\section{\module{fl} ---
         FORMS library interface for GUI applications}

\declaremodule{builtin}{fl}
  \platform{IRIX}
\modulesynopsis{FORMS library interface for GUI applications.}


This module provides an interface to the FORMS Library\index{FORMS
Library} by Mark Overmars\index{Overmars, Mark}.  The source for the
library can be retrieved by anonymous ftp from host
\samp{ftp.cs.ruu.nl}, directory \file{SGI/FORMS}.  It was last tested
with version 2.0b.

Most functions are literal translations of their C equivalents,
dropping the initial \samp{fl_} from their name.  Constants used by
the library are defined in module \refmodule[fl-constants]{FL}
described below.

The creation of objects is a little different in Python than in C:
instead of the `current form' maintained by the library to which new
FORMS objects are added, all functions that add a FORMS object to a
form are methods of the Python object representing the form.
Consequently, there are no Python equivalents for the C functions
\cfunction{fl_addto_form()} and \cfunction{fl_end_form()}, and the
equivalent of \cfunction{fl_bgn_form()} is called
\function{fl.make_form()}.

Watch out for the somewhat confusing terminology: FORMS uses the word
\dfn{object} for the buttons, sliders etc. that you can place in a form.
In Python, `object' means any value.  The Python interface to FORMS
introduces two new Python object types: form objects (representing an
entire form) and FORMS objects (representing one button, slider etc.).
Hopefully this isn't too confusing.

There are no `free objects' in the Python interface to FORMS, nor is
there an easy way to add object classes written in Python.  The FORMS
interface to GL event handling is available, though, so you can mix
FORMS with pure GL windows.

\strong{Please note:} importing \module{fl} implies a call to the GL
function \cfunction{foreground()} and to the FORMS routine
\cfunction{fl_init()}.

\subsection{Functions Defined in Module \module{fl}}
\nodename{FL Functions}

Module \module{fl} defines the following functions.  For more
information about what they do, see the description of the equivalent
C function in the FORMS documentation:

\begin{funcdesc}{make_form}{type, width, height}
Create a form with given type, width and height.  This returns a
\dfn{form} object, whose methods are described below.
\end{funcdesc}

\begin{funcdesc}{do_forms}{}
The standard FORMS main loop.  Returns a Python object representing
the FORMS object needing interaction, or the special value
\constant{FL.EVENT}.
\end{funcdesc}

\begin{funcdesc}{check_forms}{}
Check for FORMS events.  Returns what \function{do_forms()} above
returns, or \code{None} if there is no event that immediately needs
interaction.
\end{funcdesc}

\begin{funcdesc}{set_event_call_back}{function}
Set the event callback function.
\end{funcdesc}

\begin{funcdesc}{set_graphics_mode}{rgbmode, doublebuffering}
Set the graphics modes.
\end{funcdesc}

\begin{funcdesc}{get_rgbmode}{}
Return the current rgb mode.  This is the value of the C global
variable \cdata{fl_rgbmode}.
\end{funcdesc}

\begin{funcdesc}{show_message}{str1, str2, str3}
Show a dialog box with a three-line message and an OK button.
\end{funcdesc}

\begin{funcdesc}{show_question}{str1, str2, str3}
Show a dialog box with a three-line message and YES and NO buttons.
It returns \code{1} if the user pressed YES, \code{0} if NO.
\end{funcdesc}

\begin{funcdesc}{show_choice}{str1, str2, str3, but1\optional{,
                              but2\optional{, but3}}}
Show a dialog box with a three-line message and up to three buttons.
It returns the number of the button clicked by the user
(\code{1}, \code{2} or \code{3}).
\end{funcdesc}

\begin{funcdesc}{show_input}{prompt, default}
Show a dialog box with a one-line prompt message and text field in
which the user can enter a string.  The second argument is the default
input string.  It returns the string value as edited by the user.
\end{funcdesc}

\begin{funcdesc}{show_file_selector}{message, directory, pattern, default}
Show a dialog box in which the user can select a file.  It returns
the absolute filename selected by the user, or \code{None} if the user
presses Cancel.
\end{funcdesc}

\begin{funcdesc}{get_directory}{}
\funcline{get_pattern}{}
\funcline{get_filename}{}
These functions return the directory, pattern and filename (the tail
part only) selected by the user in the last
\function{show_file_selector()} call.
\end{funcdesc}

\begin{funcdesc}{qdevice}{dev}
\funcline{unqdevice}{dev}
\funcline{isqueued}{dev}
\funcline{qtest}{}
\funcline{qread}{}
%\funcline{blkqread}{?}
\funcline{qreset}{}
\funcline{qenter}{dev, val}
\funcline{get_mouse}{}
\funcline{tie}{button, valuator1, valuator2}
These functions are the FORMS interfaces to the corresponding GL
functions.  Use these if you want to handle some GL events yourself
when using \function{fl.do_events()}.  When a GL event is detected that
FORMS cannot handle, \function{fl.do_forms()} returns the special value
\constant{FL.EVENT} and you should call \function{fl.qread()} to read
the event from the queue.  Don't use the equivalent GL functions!
\end{funcdesc}

\begin{funcdesc}{color}{}
\funcline{mapcolor}{}
\funcline{getmcolor}{}
See the description in the FORMS documentation of
\cfunction{fl_color()}, \cfunction{fl_mapcolor()} and
\cfunction{fl_getmcolor()}.
\end{funcdesc}

\subsection{Form Objects}
\label{form-objects}

Form objects (returned by \function{make_form()} above) have the
following methods.  Each method corresponds to a C function whose
name is prefixed with \samp{fl_}; and whose first argument is a form
pointer; please refer to the official FORMS documentation for
descriptions.

All the \method{add_*()} methods return a Python object representing
the FORMS object.  Methods of FORMS objects are described below.  Most
kinds of FORMS object also have some methods specific to that kind;
these methods are listed here.

\begin{flushleft}

\begin{methoddesc}[form]{show_form}{placement, bordertype, name}
  Show the form.
\end{methoddesc}

\begin{methoddesc}[form]{hide_form}{}
  Hide the form.
\end{methoddesc}

\begin{methoddesc}[form]{redraw_form}{}
  Redraw the form.
\end{methoddesc}

\begin{methoddesc}[form]{set_form_position}{x, y}
Set the form's position.
\end{methoddesc}

\begin{methoddesc}[form]{freeze_form}{}
Freeze the form.
\end{methoddesc}

\begin{methoddesc}[form]{unfreeze_form}{}
  Unfreeze the form.
\end{methoddesc}

\begin{methoddesc}[form]{activate_form}{}
  Activate the form.
\end{methoddesc}

\begin{methoddesc}[form]{deactivate_form}{}
  Deactivate the form.
\end{methoddesc}

\begin{methoddesc}[form]{bgn_group}{}
  Begin a new group of objects; return a group object.
\end{methoddesc}

\begin{methoddesc}[form]{end_group}{}
  End the current group of objects.
\end{methoddesc}

\begin{methoddesc}[form]{find_first}{}
  Find the first object in the form.
\end{methoddesc}

\begin{methoddesc}[form]{find_last}{}
  Find the last object in the form.
\end{methoddesc}

%---

\begin{methoddesc}[form]{add_box}{type, x, y, w, h, name}
Add a box object to the form.
No extra methods.
\end{methoddesc}

\begin{methoddesc}[form]{add_text}{type, x, y, w, h, name}
Add a text object to the form.
No extra methods.
\end{methoddesc}

%\begin{methoddesc}[form]{add_bitmap}{type, x, y, w, h, name}
%Add a bitmap object to the form.
%\end{methoddesc}

\begin{methoddesc}[form]{add_clock}{type, x, y, w, h, name}
Add a clock object to the form. \\
Method:
\method{get_clock()}.
\end{methoddesc}

%---

\begin{methoddesc}[form]{add_button}{type, x, y, w, h,  name}
Add a button object to the form. \\
Methods:
\method{get_button()},
\method{set_button()}.
\end{methoddesc}

\begin{methoddesc}[form]{add_lightbutton}{type, x, y, w, h, name}
Add a lightbutton object to the form. \\
Methods:
\method{get_button()},
\method{set_button()}.
\end{methoddesc}

\begin{methoddesc}[form]{add_roundbutton}{type, x, y, w, h, name}
Add a roundbutton object to the form. \\
Methods:
\method{get_button()},
\method{set_button()}.
\end{methoddesc}

%---

\begin{methoddesc}[form]{add_slider}{type, x, y, w, h, name}
Add a slider object to the form. \\
Methods:
\method{set_slider_value()},
\method{get_slider_value()},
\method{set_slider_bounds()},
\method{get_slider_bounds()},
\method{set_slider_return()},
\method{set_slider_size()},
\method{set_slider_precision()},
\method{set_slider_step()}.
\end{methoddesc}

\begin{methoddesc}[form]{add_valslider}{type, x, y, w, h, name}
Add a valslider object to the form. \\
Methods:
\method{set_slider_value()},
\method{get_slider_value()},
\method{set_slider_bounds()},
\method{get_slider_bounds()},
\method{set_slider_return()},
\method{set_slider_size()},
\method{set_slider_precision()},
\method{set_slider_step()}.
\end{methoddesc}

\begin{methoddesc}[form]{add_dial}{type, x, y, w, h, name}
Add a dial object to the form. \\
Methods:
\method{set_dial_value()},
\method{get_dial_value()},
\method{set_dial_bounds()},
\method{get_dial_bounds()}.
\end{methoddesc}

\begin{methoddesc}[form]{add_positioner}{type, x, y, w, h, name}
Add a positioner object to the form. \\
Methods:
\method{set_positioner_xvalue()},
\method{set_positioner_yvalue()},
\method{set_positioner_xbounds()},
\method{set_positioner_ybounds()},
\method{get_positioner_xvalue()},
\method{get_positioner_yvalue()},
\method{get_positioner_xbounds()},
\method{get_positioner_ybounds()}.
\end{methoddesc}

\begin{methoddesc}[form]{add_counter}{type, x, y, w, h, name}
Add a counter object to the form. \\
Methods:
\method{set_counter_value()},
\method{get_counter_value()},
\method{set_counter_bounds()},
\method{set_counter_step()},
\method{set_counter_precision()},
\method{set_counter_return()}.
\end{methoddesc}

%---

\begin{methoddesc}[form]{add_input}{type, x, y, w, h, name}
Add a input object to the form. \\
Methods:
\method{set_input()},
\method{get_input()},
\method{set_input_color()},
\method{set_input_return()}.
\end{methoddesc}

%---

\begin{methoddesc}[form]{add_menu}{type, x, y, w, h, name}
Add a menu object to the form. \\
Methods:
\method{set_menu()},
\method{get_menu()},
\method{addto_menu()}.
\end{methoddesc}

\begin{methoddesc}[form]{add_choice}{type, x, y, w, h, name}
Add a choice object to the form. \\
Methods:
\method{set_choice()},
\method{get_choice()},
\method{clear_choice()},
\method{addto_choice()},
\method{replace_choice()},
\method{delete_choice()},
\method{get_choice_text()},
\method{set_choice_fontsize()},
\method{set_choice_fontstyle()}.
\end{methoddesc}

\begin{methoddesc}[form]{add_browser}{type, x, y, w, h, name}
Add a browser object to the form. \\
Methods:
\method{set_browser_topline()},
\method{clear_browser()},
\method{add_browser_line()},
\method{addto_browser()},
\method{insert_browser_line()},
\method{delete_browser_line()},
\method{replace_browser_line()},
\method{get_browser_line()},
\method{load_browser()},
\method{get_browser_maxline()},
\method{select_browser_line()},
\method{deselect_browser_line()},
\method{deselect_browser()},
\method{isselected_browser_line()},
\method{get_browser()},
\method{set_browser_fontsize()},
\method{set_browser_fontstyle()},
\method{set_browser_specialkey()}.
\end{methoddesc}

%---

\begin{methoddesc}[form]{add_timer}{type, x, y, w, h, name}
Add a timer object to the form. \\
Methods:
\method{set_timer()},
\method{get_timer()}.
\end{methoddesc}
\end{flushleft}

Form objects have the following data attributes; see the FORMS
documentation:

\begin{tableiii}{l|l|l}{member}{Name}{C Type}{Meaning}
  \lineiii{window}{int (read-only)}{GL window id}
  \lineiii{w}{float}{form width}
  \lineiii{h}{float}{form height}
  \lineiii{x}{float}{form x origin}
  \lineiii{y}{float}{form y origin}
  \lineiii{deactivated}{int}{nonzero if form is deactivated}
  \lineiii{visible}{int}{nonzero if form is visible}
  \lineiii{frozen}{int}{nonzero if form is frozen}
  \lineiii{doublebuf}{int}{nonzero if double buffering on}
\end{tableiii}

\subsection{FORMS Objects}
\label{forms-objects}

Besides methods specific to particular kinds of FORMS objects, all
FORMS objects also have the following methods:

\begin{methoddesc}[FORMS object]{set_call_back}{function, argument}
Set the object's callback function and argument.  When the object
needs interaction, the callback function will be called with two
arguments: the object, and the callback argument.  (FORMS objects
without a callback function are returned by \function{fl.do_forms()}
or \function{fl.check_forms()} when they need interaction.)  Call this
method without arguments to remove the callback function.
\end{methoddesc}

\begin{methoddesc}[FORMS object]{delete_object}{}
  Delete the object.
\end{methoddesc}

\begin{methoddesc}[FORMS object]{show_object}{}
  Show the object.
\end{methoddesc}

\begin{methoddesc}[FORMS object]{hide_object}{}
  Hide the object.
\end{methoddesc}

\begin{methoddesc}[FORMS object]{redraw_object}{}
  Redraw the object.
\end{methoddesc}

\begin{methoddesc}[FORMS object]{freeze_object}{}
  Freeze the object.
\end{methoddesc}

\begin{methoddesc}[FORMS object]{unfreeze_object}{}
  Unfreeze the object.
\end{methoddesc}

%\begin{methoddesc}[FORMS object]{handle_object}{} XXX
%\end{methoddesc}

%\begin{methoddesc}[FORMS object]{handle_object_direct}{} XXX
%\end{methoddesc}

FORMS objects have these data attributes; see the FORMS documentation:

\begin{tableiii}{l|l|l}{member}{Name}{C Type}{Meaning}
  \lineiii{objclass}{int (read-only)}{object class}
  \lineiii{type}{int (read-only)}{object type}
  \lineiii{boxtype}{int}{box type}
  \lineiii{x}{float}{x origin}
  \lineiii{y}{float}{y origin}
  \lineiii{w}{float}{width}
  \lineiii{h}{float}{height}
  \lineiii{col1}{int}{primary color}
  \lineiii{col2}{int}{secondary color}
  \lineiii{align}{int}{alignment}
  \lineiii{lcol}{int}{label color}
  \lineiii{lsize}{float}{label font size}
  \lineiii{label}{string}{label string}
  \lineiii{lstyle}{int}{label style}
  \lineiii{pushed}{int (read-only)}{(see FORMS docs)}
  \lineiii{focus}{int (read-only)}{(see FORMS docs)}
  \lineiii{belowmouse}{int (read-only)}{(see FORMS docs)}
  \lineiii{frozen}{int (read-only)}{(see FORMS docs)}
  \lineiii{active}{int (read-only)}{(see FORMS docs)}
  \lineiii{input}{int (read-only)}{(see FORMS docs)}
  \lineiii{visible}{int (read-only)}{(see FORMS docs)}
  \lineiii{radio}{int (read-only)}{(see FORMS docs)}
  \lineiii{automatic}{int (read-only)}{(see FORMS docs)}
\end{tableiii}


\section{\module{FL} ---
         Constants used with the \module{fl} module}

\declaremodule[fl-constants]{standard}{FL}
  \platform{IRIX}
\modulesynopsis{Constants used with the \module{fl} module.}


This module defines symbolic constants needed to use the built-in
module \refmodule{fl} (see above); they are equivalent to those defined in
the C header file \code{<forms.h>} except that the name prefix
\samp{FL_} is omitted.  Read the module source for a complete list of
the defined names.  Suggested use:

\begin{verbatim}
import fl
from FL import *
\end{verbatim}


\section{\module{flp} ---
         Functions for loading stored FORMS designs}

\declaremodule{standard}{flp}
  \platform{IRIX}
\modulesynopsis{Functions for loading stored FORMS designs.}


This module defines functions that can read form definitions created
by the `form designer' (\program{fdesign}) program that comes with the
FORMS library (see module \refmodule{fl} above).

For now, see the file \file{flp.doc} in the Python library source
directory for a description.

XXX A complete description should be inserted here!

\section{\module{fm} ---
         \emph{Font Manager} interface}

\declaremodule{builtin}{fm}
  \platform{IRIX}
\modulesynopsis{\emph{Font Manager} interface for SGI workstations.}


This module provides access to the IRIS \emph{Font Manager} library.
\index{Font Manager, IRIS}
\index{IRIS Font Manager}
It is available only on Silicon Graphics machines.
See also: \emph{4Sight User's Guide}, section 1, chapter 5: ``Using
the IRIS Font Manager.''

This is not yet a full interface to the IRIS Font Manager.
Among the unsupported features are: matrix operations; cache
operations; character operations (use string operations instead); some
details of font info; individual glyph metrics; and printer matching.

It supports the following operations:

\begin{funcdesc}{init}{}
Initialization function.
Calls \cfunction{fminit()}.
It is normally not necessary to call this function, since it is called
automatically the first time the \module{fm} module is imported.
\end{funcdesc}

\begin{funcdesc}{findfont}{fontname}
Return a font handle object.
Calls \code{fmfindfont(\var{fontname})}.
\end{funcdesc}

\begin{funcdesc}{enumerate}{}
Returns a list of available font names.
This is an interface to \cfunction{fmenumerate()}.
\end{funcdesc}

\begin{funcdesc}{prstr}{string}
Render a string using the current font (see the \function{setfont()} font
handle method below).
Calls \code{fmprstr(\var{string})}.
\end{funcdesc}

\begin{funcdesc}{setpath}{string}
Sets the font search path.
Calls \code{fmsetpath(\var{string})}.
(XXX Does not work!?!)
\end{funcdesc}

\begin{funcdesc}{fontpath}{}
Returns the current font search path.
\end{funcdesc}

Font handle objects support the following operations:

\setindexsubitem{(font handle method)}
\begin{funcdesc}{scalefont}{factor}
Returns a handle for a scaled version of this font.
Calls \code{fmscalefont(\var{fh}, \var{factor})}.
\end{funcdesc}

\begin{funcdesc}{setfont}{}
Makes this font the current font.
Note: the effect is undone silently when the font handle object is
deleted.
Calls \code{fmsetfont(\var{fh})}.
\end{funcdesc}

\begin{funcdesc}{getfontname}{}
Returns this font's name.
Calls \code{fmgetfontname(\var{fh})}.
\end{funcdesc}

\begin{funcdesc}{getcomment}{}
Returns the comment string associated with this font.
Raises an exception if there is none.
Calls \code{fmgetcomment(\var{fh})}.
\end{funcdesc}

\begin{funcdesc}{getfontinfo}{}
Returns a tuple giving some pertinent data about this font.
This is an interface to \code{fmgetfontinfo()}.
The returned tuple contains the following numbers:
\code{(}\var{printermatched}, \var{fixed_width}, \var{xorig},
\var{yorig}, \var{xsize}, \var{ysize}, \var{height},
\var{nglyphs}\code{)}.
\end{funcdesc}

\begin{funcdesc}{getstrwidth}{string}
Returns the width, in pixels, of \var{string} when drawn in this font.
Calls \code{fmgetstrwidth(\var{fh}, \var{string})}.
\end{funcdesc}

\section{\module{gl} ---
         \emph{Graphics Library} interface}

\declaremodule{builtin}{gl}
  \platform{IRIX}
\modulesynopsis{Functions from the Silicon Graphics \emph{Graphics Library}.}


This module provides access to the Silicon Graphics
\emph{Graphics Library}.
It is available only on Silicon Graphics machines.

\warning{Some illegal calls to the GL library cause the Python
interpreter to dump core.
In particular, the use of most GL calls is unsafe before the first
window is opened.}

The module is too large to document here in its entirety, but the
following should help you to get started.
The parameter conventions for the C functions are translated to Python as
follows:

\begin{itemize}
\item
All (short, long, unsigned) int values are represented by Python
integers.
\item
All float and double values are represented by Python floating point
numbers.
In most cases, Python integers are also allowed.
\item
All arrays are represented by one-dimensional Python lists.
In most cases, tuples are also allowed.
\item
\begin{sloppypar}
All string and character arguments are represented by Python strings,
for instance,
\code{winopen('Hi There!')}
and
\code{rotate(900, 'z')}.
\end{sloppypar}
\item
All (short, long, unsigned) integer arguments or return values that are
only used to specify the length of an array argument are omitted.
For example, the C call

\begin{verbatim}
lmdef(deftype, index, np, props)
\end{verbatim}

is translated to Python as

\begin{verbatim}
lmdef(deftype, index, props)
\end{verbatim}

\item
Output arguments are omitted from the argument list; they are
transmitted as function return values instead.
If more than one value must be returned, the return value is a tuple.
If the C function has both a regular return value (that is not omitted
because of the previous rule) and an output argument, the return value
comes first in the tuple.
Examples: the C call

\begin{verbatim}
getmcolor(i, &red, &green, &blue)
\end{verbatim}

is translated to Python as

\begin{verbatim}
red, green, blue = getmcolor(i)
\end{verbatim}

\end{itemize}

The following functions are non-standard or have special argument
conventions:

\begin{funcdesc}{varray}{argument}
%JHXXX the argument-argument added
Equivalent to but faster than a number of
\code{v3d()}
calls.
The \var{argument} is a list (or tuple) of points.
Each point must be a tuple of coordinates
\code{(\var{x}, \var{y}, \var{z})} or \code{(\var{x}, \var{y})}.
The points may be 2- or 3-dimensional but must all have the
same dimension.
Float and int values may be mixed however.
The points are always converted to 3D double precision points
by assuming \code{\var{z} = 0.0} if necessary (as indicated in the man page),
and for each point
\code{v3d()}
is called.
\end{funcdesc}

\begin{funcdesc}{nvarray}{}
Equivalent to but faster than a number of
\code{n3f}
and
\code{v3f}
calls.
The argument is an array (list or tuple) of pairs of normals and points.
Each pair is a tuple of a point and a normal for that point.
Each point or normal must be a tuple of coordinates
\code{(\var{x}, \var{y}, \var{z})}.
Three coordinates must be given.
Float and int values may be mixed.
For each pair,
\code{n3f()}
is called for the normal, and then
\code{v3f()}
is called for the point.
\end{funcdesc}

\begin{funcdesc}{vnarray}{}
Similar to 
\code{nvarray()}
but the pairs have the point first and the normal second.
\end{funcdesc}

\begin{funcdesc}{nurbssurface}{s_k, t_k, ctl, s_ord, t_ord, type}
% XXX s_k[], t_k[], ctl[][]
Defines a nurbs surface.
The dimensions of
\code{\var{ctl}[][]}
are computed as follows:
\code{[len(\var{s_k}) - \var{s_ord}]},
\code{[len(\var{t_k}) - \var{t_ord}]}.
\end{funcdesc}

\begin{funcdesc}{nurbscurve}{knots, ctlpoints, order, type}
Defines a nurbs curve.
The length of ctlpoints is
\code{len(\var{knots}) - \var{order}}.
\end{funcdesc}

\begin{funcdesc}{pwlcurve}{points, type}
Defines a piecewise-linear curve.
\var{points}
is a list of points.
\var{type}
must be
\code{N_ST}.
\end{funcdesc}

\begin{funcdesc}{pick}{n}
\funcline{select}{n}
The only argument to these functions specifies the desired size of the
pick or select buffer.
\end{funcdesc}

\begin{funcdesc}{endpick}{}
\funcline{endselect}{}
These functions have no arguments.
They return a list of integers representing the used part of the
pick/select buffer.
No method is provided to detect buffer overrun.
\end{funcdesc}

Here is a tiny but complete example GL program in Python:

\begin{verbatim}
import gl, GL, time

def main():
    gl.foreground()
    gl.prefposition(500, 900, 500, 900)
    w = gl.winopen('CrissCross')
    gl.ortho2(0.0, 400.0, 0.0, 400.0)
    gl.color(GL.WHITE)
    gl.clear()
    gl.color(GL.RED)
    gl.bgnline()
    gl.v2f(0.0, 0.0)
    gl.v2f(400.0, 400.0)
    gl.endline()
    gl.bgnline()
    gl.v2f(400.0, 0.0)
    gl.v2f(0.0, 400.0)
    gl.endline()
    time.sleep(5)

main()
\end{verbatim}


\begin{seealso}
  \seetitle[http://pyopengl.sourceforge.net/]
           {PyOpenGL: The Python OpenGL Binding}
           {An interface to OpenGL\index{OpenGL} is also available;
            see information about the
            \strong{PyOpenGL}\index{PyOpenGL} project online at
            \url{http://pyopengl.sourceforge.net/}.  This may be a
            better option if support for SGI hardware from before
            about 1996 is not required.}
\end{seealso}


\section{\module{DEVICE} ---
         Constants used with the \module{gl} module}

\declaremodule{standard}{DEVICE}
  \platform{IRIX}
\modulesynopsis{Constants used with the \module{gl} module.}

This modules defines the constants used by the Silicon Graphics
\emph{Graphics Library} that C programmers find in the header file
\code{<gl/device.h>}.
Read the module source file for details.


\section{\module{GL} ---
         Constants used with the \module{gl} module}

\declaremodule[gl-constants]{standard}{GL}
  \platform{IRIX}
\modulesynopsis{Constants used with the \module{gl} module.}

This module contains constants used by the Silicon Graphics
\emph{Graphics Library} from the C header file \code{<gl/gl.h>}.
Read the module source file for details.

\section{\module{imgfile} ---
         Support for SGI imglib files}

\declaremodule{builtin}{imgfile}
  \platform{IRIX}
\modulesynopsis{Support for SGI imglib files.}


The \module{imgfile} module allows Python programs to access SGI imglib image
files (also known as \file{.rgb} files).  The module is far from
complete, but is provided anyway since the functionality that there is
is enough in some cases.  Currently, colormap files are not supported.

The module defines the following variables and functions:

\begin{excdesc}{error}
This exception is raised on all errors, such as unsupported file type, etc.
\end{excdesc}

\begin{funcdesc}{getsizes}{file}
This function returns a tuple \code{(\var{x}, \var{y}, \var{z})} where
\var{x} and \var{y} are the size of the image in pixels and
\var{z} is the number of
bytes per pixel. Only 3 byte RGB pixels and 1 byte greyscale pixels
are currently supported.
\end{funcdesc}

\begin{funcdesc}{read}{file}
This function reads and decodes the image on the specified file, and
returns it as a Python string. The string has either 1 byte greyscale
pixels or 4 byte RGBA pixels. The bottom left pixel is the first in
the string. This format is suitable to pass to \function{gl.lrectwrite()},
for instance.
\end{funcdesc}

\begin{funcdesc}{readscaled}{file, x, y, filter\optional{, blur}}
This function is identical to read but it returns an image that is
scaled to the given \var{x} and \var{y} sizes. If the \var{filter} and
\var{blur} parameters are omitted scaling is done by
simply dropping or duplicating pixels, so the result will be less than
perfect, especially for computer-generated images.

Alternatively, you can specify a filter to use to smoothen the image
after scaling. The filter forms supported are \code{'impulse'},
\code{'box'}, \code{'triangle'}, \code{'quadratic'} and
\code{'gaussian'}. If a filter is specified \var{blur} is an optional
parameter specifying the blurriness of the filter. It defaults to \code{1.0}.

\function{readscaled()} makes no attempt to keep the aspect ratio
correct, so that is the users' responsibility.
\end{funcdesc}

\begin{funcdesc}{ttob}{flag}
This function sets a global flag which defines whether the scan lines
of the image are read or written from bottom to top (flag is zero,
compatible with SGI GL) or from top to bottom(flag is one,
compatible with X).  The default is zero.
\end{funcdesc}

\begin{funcdesc}{write}{file, data, x, y, z}
This function writes the RGB or greyscale data in \var{data} to image
file \var{file}. \var{x} and \var{y} give the size of the image,
\var{z} is 1 for 1 byte greyscale images or 3 for RGB images (which are
stored as 4 byte values of which only the lower three bytes are used).
These are the formats returned by \function{gl.lrectread()}.
\end{funcdesc}

%\section{\module{panel} ---
         None}
\declaremodule{standard}{panel}

\modulesynopsis{None}


\strong{Please note:} The FORMS library, to which the
\code{fl}\refbimodindex{fl} module described above interfaces, is a
simpler and more accessible user interface library for use with GL
than the \code{panel} module (besides also being by a Dutch author).

This module should be used instead of the built-in module
\code{pnl}\refbimodindex{pnl}
to interface with the
\emph{Panel Library}.

The module is too large to document here in its entirety.
One interesting function:

\begin{funcdesc}{defpanellist}{filename}
Parses a panel description file containing S-expressions written by the
\emph{Panel Editor}
that accompanies the Panel Library and creates the described panels.
It returns a list of panel objects.
\end{funcdesc}

\strong{Warning:}
the Python interpreter will dump core if you don't create a GL window
before calling
\code{panel.mkpanel()}
or
\code{panel.defpanellist()}.

\section{\module{panelparser} ---
         None}
\declaremodule{standard}{panelparser}

\modulesynopsis{None}


This module defines a self-contained parser for S-expressions as output
by the Panel Editor (which is written in Scheme so it can't help writing
S-expressions).
The relevant function is
\code{panelparser.parse_file(\var{file})}
which has a file object (not a filename!) as argument and returns a list
of parsed S-expressions.
Each S-expression is converted into a Python list, with atoms converted
to Python strings and sub-expressions (recursively) to Python lists.
For more details, read the module file.
% XXXXJH should be funcdesc, I think

\section{\module{pnl} ---
         None}
\declaremodule{builtin}{pnl}

\modulesynopsis{None}


This module provides access to the
\emph{Panel Library}
built by NASA Ames\index{NASA} (to get it, send e-mail to
\code{panel-request@nas.nasa.gov}).
All access to it should be done through the standard module
\code{panel}\refstmodindex{panel},
which transparently exports most functions from
\code{pnl}
but redefines
\code{pnl.dopanel()}.

\strong{Warning:}
the Python interpreter will dump core if you don't create a GL window
before calling
\code{pnl.mkpanel()}.

The module is too large to document here in its entirety.


\chapter{SunOS Specific Services}
\label{sunos}

The modules described in this chapter provide interfaces to features
that are unique to SunOS 5 (also known as Solaris version 2).
			% SUNOS ONLY

\documentstyle[twoside,11pt,myformat]{report}

% NOTE: this file controls which chapters/sections of the library
% manual are actually printed.  It is easy to customize your manual
% by commenting out sections that you're not interested in.

\title{Python Library Reference}

\input{boilerplate}

\makeindex			% tell \index to actually write the .idx file


\begin{document}

\pagenumbering{roman}

\maketitle

\input{copyright}

\begin{abstract}

\noindent
This document describes the built-in and standard types, exceptions,
functions and modules that come with the Python system.  It assumes
basic knowledge about the Python language.  For an informal
introduction to the language, see the {\em Python Tutorial}.  The {\em
Python Reference Manual} gives a more formal definition of the
language.

\end{abstract}

\pagebreak

{
\parskip = 0mm
\tableofcontents
}

\pagebreak

\pagenumbering{arabic}

				% Chapter title:

\input{libintro}		% Introduction

\input{libobjs}			% Built-in Types, Exceptions and Functions
\input{libtypes}
\input{libexcs}
\input{libfuncs}

\input{libpython}		% Python Services
\input{libsys}
\input{libtypes2}		% types is already taken :-(
\input{libtraceback}
\input{libpickle}
\input{libshelve}
\input{libcopy}
\input{libmarshal}
\input{libimp}
\input{libbltin}		% really __builtin__
\input{libmain}			% really __main__

\input{libstrings}		% String Services
\input{libstring}
\input{libregex}
\input{libregsub}
\input{libstruct}

\input{libmisc}			% Miscellaneous Services
\input{libmath}
\input{librand}
\input{libwhrandom}
\input{libarray}

\input{liballos}		% Generic Operating System Services
\input{libos}
\input{libtime}
\input{libgetopt}
\input{libtempfile}

\input{libsomeos}		% Optional Operating System Services
\input{libsignal}
\input{libsocket}
\input{libselect}
\input{libthread}

\input{libunix}			% UNIX Specific Services
\input{libposix}
\input{libppath}		% == posixpath
\input{libpwd}
\input{libgrp}
\input{libdbm}
\input{libgdbm}
\input{libtermios}
\input{libfcntl}
\input{libposixfile}

\input{libpdb}			% The Python Debugger

\input{libprofile}		% The Python Profiler

\input{libwww}			% Internet and WWW Services
\input{libcgi}
\input{liburllib}
\input{libhttplib}
\input{libftplib}
\input{libgopherlib}
\input{libnntplib}
\input{liburlparse}
\input{libhtmllib}
\input{libsgmllib}
\input{librfc822}
\input{libmimetools}

\input{libmm}			% Multimedia Services
\input{libaudioop}
\input{libimageop}
\input{libaifc}
\input{libjpeg}
\input{librgbimg}

\input{libcrypto}		% Cryptographic Services
\input{libmd5}
\input{libmpz}
\input{librotor}

%\input{libamoeba}		% AMOEBA ONLY

\input{libmac}			% MACINTOSH ONLY
\input{libctb}
\input{libmacconsole}
\input{libmacdnr}
\input{libmacfs}
\input{libmactcp}
\input{libmacspeech}

\input{libstdwin}		% STDWIN ONLY

\input{libsgi}			% SGI IRIX ONLY
\input{libal}
%\input{libaudio}
\input{libcd}
\input{libfl}
\input{libfm}
\input{libgl}
\input{libimgfile}
%\input{libpanel}

\input{libsun}			% SUNOS ONLY

\input{lib.ind}			% Index

\end{document}
			% Index

\end{document}
			% Index

\end{document}
		% The index

\end{document}
