\documentstyle[twoside,11pt,myformat]{report}

\title{Python-C API Reference}

\author{Guido van Rossum\\
	Fred L. Drake, Jr., editor}
\authoraddress{
	BeOpen PythonLabs\\
	E-mail: \email{python-docs@python.org}
}

\date{September 5, 2000}			% XXX update before release!
\release{2.0b1}


\makeindex			% tell \index to actually write the .idx file


\begin{document}

\pagenumbering{roman}

\maketitle

\begin{small}
Copyright \copyright{} 2001 Python Software Foundation.
All rights reserved.

Copyright \copyright{} 2000 BeOpen.com.
All rights reserved.

Copyright \copyright{} 1995-2000 Corporation for National Research Initiatives.
All rights reserved.

Copyright \copyright{} 1991-1995 Stichting Mathematisch Centrum.
All rights reserved.

%%begin{latexonly}
\vskip 4mm
%%end{latexonly}

\centerline{\strong{BEOPEN.COM TERMS AND CONDITIONS FOR PYTHON 2.0}}

\centerline{\strong{BEOPEN PYTHON OPEN SOURCE LICENSE AGREEMENT VERSION 1}}

\begin{enumerate}

\item
This LICENSE AGREEMENT is between BeOpen.com (``BeOpen''), having an
office at 160 Saratoga Avenue, Santa Clara, CA 95051, and the
Individual or Organization (``Licensee'') accessing and otherwise
using this software in source or binary form and its associated
documentation (``the Software'').

\item
Subject to the terms and conditions of this BeOpen Python License
Agreement, BeOpen hereby grants Licensee a non-exclusive,
royalty-free, world-wide license to reproduce, analyze, test, perform
and/or display publicly, prepare derivative works, distribute, and
otherwise use the Software alone or in any derivative version,
provided, however, that the BeOpen Python License is retained in the
Software, alone or in any derivative version prepared by Licensee.

\item
BeOpen is making the Software available to Licensee on an ``AS IS''
basis.  BEOPEN MAKES NO REPRESENTATIONS OR WARRANTIES, EXPRESS OR
IMPLIED.  BY WAY OF EXAMPLE, BUT NOT LIMITATION, BEOPEN MAKES NO AND
DISCLAIMS ANY REPRESENTATION OR WARRANTY OF MERCHANTABILITY OR FITNESS
FOR ANY PARTICULAR PURPOSE OR THAT THE USE OF THE SOFTWARE WILL NOT
INFRINGE ANY THIRD PARTY RIGHTS.

\item
BEOPEN SHALL NOT BE LIABLE TO LICENSEE OR ANY OTHER USERS OF THE
SOFTWARE FOR ANY INCIDENTAL, SPECIAL, OR CONSEQUENTIAL DAMAGES OR LOSS
AS A RESULT OF USING, MODIFYING OR DISTRIBUTING THE SOFTWARE, OR ANY
DERIVATIVE THEREOF, EVEN IF ADVISED OF THE POSSIBILITY THEREOF.

\item
This License Agreement will automatically terminate upon a material
breach of its terms and conditions.

\item
This License Agreement shall be governed by and interpreted in all
respects by the law of the State of California, excluding conflict of
law provisions.  Nothing in this License Agreement shall be deemed to
create any relationship of agency, partnership, or joint venture
between BeOpen and Licensee.  This License Agreement does not grant
permission to use BeOpen trademarks or trade names in a trademark
sense to endorse or promote products or services of Licensee, or any
third party.  As an exception, the ``BeOpen Python'' logos available
at http://www.pythonlabs.com/logos.html may be used according to the
permissions granted on that web page.

\item
By copying, installing or otherwise using the software, Licensee
agrees to be bound by the terms and conditions of this License
Agreement.
\end{enumerate}


\centerline{\strong{CNRI OPEN SOURCE GPL-COMPATIBLE LICENSE AGREEMENT}}

Python 1.6.1 is made available subject to the terms and conditions in
CNRI's License Agreement.  This Agreement together with Python 1.6.1 may
be located on the Internet using the following unique, persistent
identifier (known as a handle): 1895.22/1013.  This Agreement may also
be obtained from a proxy server on the Internet using the following
URL: \url{http://hdl.handle.net/1895.22/1013}.


\centerline{\strong{CWI PERMISSIONS STATEMENT AND DISCLAIMER}}

Copyright \copyright{} 1991 - 1995, Stichting Mathematisch Centrum
Amsterdam, The Netherlands.  All rights reserved.

Permission to use, copy, modify, and distribute this software and its
documentation for any purpose and without fee is hereby granted,
provided that the above copyright notice appear in all copies and that
both that copyright notice and this permission notice appear in
supporting documentation, and that the name of Stichting Mathematisch
Centrum or CWI not be used in advertising or publicity pertaining to
distribution of the software without specific, written prior
permission.

STICHTING MATHEMATISCH CENTRUM DISCLAIMS ALL WARRANTIES WITH REGARD TO
THIS SOFTWARE, INCLUDING ALL IMPLIED WARRANTIES OF MERCHANTABILITY AND
FITNESS, IN NO EVENT SHALL STICHTING MATHEMATISCH CENTRUM BE LIABLE
FOR ANY SPECIAL, INDIRECT OR CONSEQUENTIAL DAMAGES OR ANY DAMAGES
WHATSOEVER RESULTING FROM LOSS OF USE, DATA OR PROFITS, WHETHER IN AN
ACTION OF CONTRACT, NEGLIGENCE OR OTHER TORTIOUS ACTION, ARISING OUT
OF OR IN CONNECTION WITH THE USE OR PERFORMANCE OF THIS SOFTWARE.
\end{small}


\begin{abstract}

\noindent
This manual documents the API used by C (or C++) programmers who want
to write extension modules or embed Python.  It is a companion to
``Extending and Embedding the Python Interpreter'', which describes
the general principles of extension writing but does not document the
API functions in detail.

\end{abstract}

\pagebreak

{
\parskip = 0mm
\tableofcontents
}

\pagebreak

\pagenumbering{arabic}


\chapter{Introduction}

The Application Programmer's Interface to Python gives C and C++
programmers access to the Python interpreter at a variety of levels.
There are two fundamentally different reasons for using the Python/C 
API.  (The API is equally usable from C++, but for brevity it is 
generally referred to as the Python/C API.)  The first reason is to 
write ``extension modules'' for specific purposes; these are C modules 
that extend the Python interpreter.  This is probably the most common 
use.  The second reason is to use Python as a component in a larger 
application; this technique is generally referred to as ``embedding'' 
Python in an application.

Writing an extension module is a relatively well-understood process, 
where a ``cookbook'' approach works well.  There are several tools 
that automate the process to some extent.  While people have embedded 
Python in other applications since its early existence, the process of 
embedding Python is less straightforward that writing an extension.  
Python 1.5 introduces a number of new API functions as well as some 
changes to the build process that make embedding much simpler.  
This manual describes the 1.5 state of affair (as of Python 1.5a3).
% XXX Eventually, take the historical notes out

Many API functions are useful independent of whether you're embedding 
or extending Python; moreover, most applications that embed Python 
will need to provide a custom extension as well, so it's probably a 
good idea to become familiar with writing an extension before 
attempting to embed Python in a real application.

\section{Objects, Types and Reference Counts}

Most Python/C API functions have one or more arguments as well as a 
return value of type \code{PyObject *}.  This type is a pointer 
(obviously!)  to an opaque data type representing an arbitrary Python 
object.  Since all Python object types are treated the same way by the 
Python language in most situations (e.g., assignments, scope rules, 
and argument passing), it is only fitting that they should be 
represented by a single C type.  All Python objects live on the heap:
you never declare an automatic or static variable of type 
\code{PyObject}, only pointer variables of type \code{PyObject *} can 
be declared.

All Python objects (even Python integers) have a ``type'' and a 
``reference count''.  An object's type determines what kind of object 
it is (e.g., an integer, a list, or a user-defined function; there are 
many more as explained in the Python Language Reference Manual).  For 
each of the well-known types there is a macro to check whether an 
object is of that type; for instance, \code{PyList_Check(a)} is true 
iff the object pointed to by \code{a} is a Python list.

The reference count is important only because today's computers have a 
finite (and often severly limited) memory size; it counts how many 
different places there are that have a reference to an object.  Such a 
place could be another object, or a global (or static) C variable, or 
a local variable in some C function.  When an object's reference count 
becomes zero, the object is deallocated.  If it contains references to 
other objects, their reference count is decremented.  Those other 
objects may be deallocated in turn, if this decrement makes their 
reference count become zero, and so on.  (There's an obvious problem 
with objects that reference each other here; for now, the solution is 
``don't do that''.)

Reference counts are always manipulated explicitly.  The normal way is 
to use the macro \code{Py_INCREF(a)} to increment an object's 
reference count by one, and \code{Py_DECREF(a)} to decrement it by 
one.  The latter macro is considerably more complex than the former, 
since it must check whether the reference count becomes zero and then 
cause the object's deallocator, which is a function pointer contained 
in the object's type structure.  The type-specific deallocator takes 
care of decrementing the reference counts for other objects contained 
in the object, and so on, if this is a compound object type such as a 
list.  There's no chance that the reference count can overflow; at 
least as many bits are used to hold the reference count as there are 
distinct memory locations in virtual memory (assuming 
\code{sizeof(long) >= sizeof(char *)}).  Thus, the reference count 
increment is a simple operation.

It is not necessary to increment an object's reference count for every 
local variable that contains a pointer to an object.  In theory, the 
oject's reference count goes up by one when the variable is made to 
point to it and it goes down by one when the variable goes out of 
scope.  However, these two cancel each other out, so at the end the 
reference count hasn't changed.  The only real reason to use the 
reference count is to prevent the object from being deallocated as 
long as our variable is pointing to it.  If we know that there is at 
least one other reference to the object that lives at least as long as 
our variable, there is no need to increment the reference count 
temporarily.  An important situation where this arises is in objects 
that are passed as arguments to C functions in an extension module 
that are called from Python; the call mechanism guarantees to hold a 
reference to every argument for the duration of the call.

However, a common pitfall is to extract an object from a list and 
holding on to it for a while without incrementing its reference count.  
Some other operation might conceivably remove the object from the 
list, decrementing its reference count and possible deallocating it.  
The real danger is that innocent-looking operations may invoke 
arbitrary Python code which could do this; there is a code path which 
allows control to flow back to the user from a \code{Py_DECREF()}, so 
almost any operation is potentially dangerous.

A safe approach is to always use the generic operations (functions 
whose name begins with \code{PyObject_}, \code{PyNumber_}, 
\code{PySequence_} or \code{PyMapping_}).  These operations always 
increment the reference count of the object they return.  This leaves 
the caller with the responsibility to call \code{Py_DECREF()} when 
they are done with the result; this soon becomes second nature.

There are very few other data types that play a significant role in 
the Python/C API; most are all simple C types such as \code{int}, 
\code{long}, \code{double} and \code{char *}.  A few structure types 
are used to describe static tables used to list the functions exported 
by a module or the data attributes of a new object type.  These will 
be discussed together with the functions that use them.

\section{Exceptions}

The Python programmer only needs to deal with exceptions if specific 
error handling is required; unhandled exceptions are automatically 
propagated to the caller, then to the caller's caller, and so on, till 
they reach the top-level interpreter, where they are reported to the 
user accompanied by a stack trace.

For C programmers, however, error checking always has to be explicit.
% XXX add more stuff here

\section{Embedding Python}

The one important task that only embedders of the Python interpreter 
have to worry about is the initialization (and possibly the 
finalization) of the Python interpreter.  Most functionality of the 
interpreter can only be used after the interpreter has been 
initialized.

The basic initialization function is \code{Py_Initialize()}.  This 
initializes the table of loaded modules, and creates the fundamental 
modules \code{__builtin__}, \code{__main__} and \code{sys}.  It also 
initializes the module search path (\code{sys.path}).

\code{Py_Initialize()} does not set the ``script argument list'' 
(\code{sys.argv}).  If this variable is needed by Python code that 
will be executed later, it must be set explicitly with a call to 
\code{PySys_SetArgv(\var{argc}, \var{argv})} subsequent to the call 
to \code{Py_Initialize()}.

On Unix, \code{Py_Initialize()} calculates the module search path 
based upon its best guess for the location of the standard Python 
interpreter executable, assuming that the Python library is found in a 
fixed location relative to the Python interpreter executable.  In 
particular, it looks for a directory named \code{lib/python1.5} 
(replacing \code{1.5} with the current interpreter version) relative 
to the parent directory where the executable named \code{python} is 
found on the shell command search path (the environment variable 
\code{\$PATH}).  For instance, if the Python executable is found in 
\code{/usr/local/bin/python}, it will assume that the libraries are in 
\code{/usr/local/lib/python1.5}.  In fact, this also the ``fallback'' 
location, used when no executable file named \code{python} is found 
along \code{\$PATH}.  The user can change this behavior by setting the 
environment variable \code{\$PYTHONHOME}, and can insert additional 
directories in front of the standard path by setting 
\code{\$PYTHONPATH}.

The embedding application can steer the search by calling 
\code{Py_SetProgramName(\var{file})} \emph{before} calling 
\code{Py_Initialize()}.  Note that \code{\$PYTHONHOME} still overrides 
this and \code{\$PYTHONPATH} is still inserted in front of the 
standard path.

Sometimes, it is desirable to ``uninitialize'' Python.  For instance, 
the application may want to start over (make another call to 
\code{Py_Initialize()}) or the application is simply done with its 
use of Python and wants to free all memory allocated by Python.  This 
can be accomplished by calling \code{Py_Finalize()}.
% XXX More...

\section{Embedding Python in Threaded Applications}










\chapter{Old Introduction}

(XXX This is the old introduction, mostly by Jim Fulton -- should be
rewritten.)

From the viewpoint of of C access to Python services, we have:

\begin{enumerate}

\item "Very high level layer": two or three functions that let you
exec or eval arbitrary Python code given as a string in a module whose
name is given, passing C values in and getting C values out using
mkvalue/getargs style format strings.  This does not require the user
to declare any variables of type \code{PyObject *}.  This should be
enough to write a simple application that gets Python code from the
user, execs it, and returns the output or errors.

\item "Abstract objects layer": which is the subject of this chapter.
It has many functions operating on objects, and lets you do many
things from C that you can also write in Python, without going through
the Python parser.

\item "Concrete objects layer": This is the public type-dependent
interface provided by the standard built-in types, such as floats,
strings, and lists.  This interface exists and is currently documented
by the collection of include files provides with the Python
distributions.

\end{enumerate}

From the point of view of Python accessing services provided by C
modules:

\begin{enumerate}

\item[4.] "Python module interface": this interface consist of the basic
routines used to define modules and their members.  Most of the
current extensions-writing guide deals with this interface.

\item[5.] "Built-in object interface": this is the interface that a new
built-in type must provide and the mechanisms and rules that a
developer of a new built-in type must use and follow.

\end{enumerate}

The Python C API provides four groups of operations on objects,
corresponding to the same operations in the Python language: object,
numeric, sequence, and mapping.  Each protocol consists of a
collection of related operations.  If an operation that is not
provided by a particular type is invoked, then the standard exception
\code{TypeError} is raised with a operation name as an argument.

In addition, for convenience this interface defines a set of
constructors for building objects of built-in types.  This is needed
so new objects can be returned from C functions that otherwise treat
objects generically.

\section{Reference Counting}

For most of the functions in the Python-C API, if a function retains a
reference to a Python object passed as an argument, then the function
will increase the reference count of the object.  It is unnecessary
for the caller to increase the reference count of an argument in
anticipation of the object's retention.

Usually, Python objects returned from functions should be treated as
new objects.  Functions that return objects assume that the caller
will retain a reference and the reference count of the object has
already been incremented to account for this fact.  A caller that does
not retain a reference to an object that is returned from a function
must decrement the reference count of the object (using
\code{Py_DECREF()}) to prevent memory leaks.

Exceptions to these rules will be noted with the individual functions.

\section{Include Files}

All function, type and macro definitions needed to use the Python-C
API are included in your code by the following line:

\code{\#include "Python.h"}

This implies inclusion of the following standard header files:
stdio.h, string.h, errno.h, and stdlib.h (if available).

All user visible names defined by Python.h (except those defined by
the included standard headers) have one of the prefixes \code{Py} or
\code{_Py}.  Names beginning with \code{_Py} are for internal use
only.


\chapter{Initialization and Shutdown of an Embedded Python Interpreter}

When embedding the Python interpreter in a C or C++ program, the
interpreter must be initialized.

\begin{cfuncdesc}{void}{PyInitialize}{}
This function initializes the interpreter.  It must be called before
any interaction with the interpreter takes place.  If it is called
more than once, the second and further calls have no effect.

The function performs the following tasks: create an environment in
which modules can be imported and Python code can be executed;
initialize the \code{__builtin__} module; initialize the \code{sys}
module; initialize \code{sys.path}; initialize signal handling; and
create the empty \code{__main__} module.

In the current system, there is no way to undo all these
initializations or to create additional interpreter environments.
\end{cfuncdesc}

\begin{cfuncdesc}{int}{Py_AtExit}{void (*func) ()}
Register a cleanup function to be called when Python exits.  The
cleanup function will be called with no arguments and should return no
value.  At most 32 cleanup functions can be registered.  When the
registration is successful, \code{Py_AtExit} returns 0; on failure, it
returns -1.  Each cleanup function will be called t most once.  The
cleanup function registered last is called first.
\end{cfuncdesc}

\begin{cfuncdesc}{void}{Py_Exit}{int status}
Exit the current process.  This calls \code{Py_Cleanup()} (see next
item) and performs additional cleanup (under some circumstances it
will attempt to delete all modules), and then calls the standard C
library function \code{exit(status)}.
\end{cfuncdesc}

\begin{cfuncdesc}{void}{Py_Cleanup}{}
Perform some of the cleanup that \code{Py_Exit} performs, but don't
exit the process.  In particular, this invokes the user's
\code{sys.exitfunc} function (if defined at all), and it invokes the
cleanup functions registered with \code{Py_AtExit()}, in reverse order
of their registration.
\end{cfuncdesc}

\begin{cfuncdesc}{void}{Py_FatalError}{char *message}
Print a fatal error message and die.  No cleanup is performed.  This
function should only be invoked when a condition is detected that
would make it dangerous to continue using the Python interpreter;
e.g., when the object administration appears to be corrupted.
\end{cfuncdesc}

\begin{cfuncdesc}{void}{PyImport_Init}{}
Initialize the module table.  For internal use only.
\end{cfuncdesc}

\begin{cfuncdesc}{void}{PyImport_Cleanup}{}
Empty the module table.  For internal use only.
\end{cfuncdesc}

\begin{cfuncdesc}{void}{PyBuiltin_Init}{}
Initialize the \code{__builtin__} module.  For internal use only.
\end{cfuncdesc}

XXX Other init functions: PyEval_InitThreads, PyOS_InitInterrupts,
PyMarshal_Init, PySys_Init.

\chapter{Reference Counting}

The functions in this chapter are used for managing reference counts
of Python objects.

\begin{cfuncdesc}{void}{Py_INCREF}{PyObject *o}
Increment the reference count for object \code{o}.  The object must
not be \NULL{}; if you aren't sure that it isn't \NULL{}, use
\code{Py_XINCREF()}.
\end{cfuncdesc}

\begin{cfuncdesc}{void}{Py_XINCREF}{PyObject *o}
Increment the reference count for object \code{o}.  The object may be
\NULL{}, in which case the function has no effect.
\end{cfuncdesc}

\begin{cfuncdesc}{void}{Py_DECREF}{PyObject *o}
Decrement the reference count for object \code{o}.  The object must
not be \NULL{}; if you aren't sure that it isn't \NULL{}, use
\code{Py_XDECREF()}.  If the reference count reaches zero, the object's
type's deallocation function (which must not be \NULL{}) is invoked.

\strong{Warning:} The deallocation function can cause arbitrary Python
code to be invoked (e.g. when a class instance with a \code{__del__()}
method is deallocated).  While exceptions in such code are not
propagated, the executed code has free access to all Python global
variables.  This means that any object that is reachable from a global
variable should be in a consistent state before \code{Py_DECREF()} is
invoked.  For example, code to delete an object from a list should
copy a reference to the deleted object in a temporary variable, update
the list data structure, and then call \code{Py_DECREF()} for the
temporary variable.
\end{cfuncdesc}

\begin{cfuncdesc}{void}{Py_XDECREF}{PyObject *o}
Decrement the reference count for object \code{o}.The object may be
\NULL{}, in which case the function has no effect; otherwise the
effect is the same as for \code{Py_DECREF()}, and the same warning
applies.
\end{cfuncdesc}

The following functions are only for internal use:
\code{_Py_Dealloc}, \code{_Py_ForgetReference}, \code{_Py_NewReference},
as well as the global variable \code{_Py_RefTotal}.


\chapter{Exception Handling}

The functions in this chapter will let you handle and raise Python
exceptions.  It is important to understand some of the basics of
Python exception handling.  It works somewhat like the Unix
\code{errno} variable: there is a global indicator (per thread) of the
last error that occurred.  Most functions don't clear this on success,
but will set it to indicate the cause of the error on failure.  Most
functions also return an error indicator, usually \NULL{} if they are
supposed to return a pointer, or -1 if they return an integer
(exception: the \code{PyArg_Parse*()} functions return 1 for success and
0 for failure).  When a function must fail because of some function it
called failed, it generally doesn't set the error indicator; the
function it called already set it.

The error indicator consists of three Python objects corresponding to
the Python variables \code{sys.exc_type}, \code{sys.exc_value} and
\code{sys.exc_traceback}.  API functions exist to interact with the
error indicator in various ways.  There is a separate error indicator
for each thread.

% XXX Order of these should be more thoughtful.
% Either alphabetical or some kind of structure.

\begin{cfuncdesc}{void}{PyErr_Print}{}
Print a standard traceback to \code{sys.stderr} and clear the error
indicator.  Call this function only when the error indicator is set.
(Otherwise it will cause a fatal error!)
\end{cfuncdesc}

\begin{cfuncdesc}{PyObject *}{PyErr_Occurred}{}
Test whether the error indicator is set.  If set, return the exception
\code{type} (the first argument to the last call to one of the
\code{PyErr_Set*()} functions or to \code{PyErr_Restore()}).  If not
set, return \NULL{}.  You do not own a reference to the return value,
so you do not need to \code{Py_DECREF()} it.
\end{cfuncdesc}

\begin{cfuncdesc}{void}{PyErr_Clear}{}
Clear the error indicator.  If the error indicator is not set, there
is no effect.
\end{cfuncdesc}

\begin{cfuncdesc}{void}{PyErr_Fetch}{PyObject **ptype, PyObject **pvalue, PyObject **ptraceback}
Retrieve the error indicator into three variables whose addresses are
passed.  If the error indicator is not set, set all three variables to
\NULL{}.  If it is set, it will be cleared and you own a reference to
each object retrieved.  The value and traceback object may be \NULL{}
even when the type object is not.  \strong{Note:} this function is
normally only used by code that needs to handle exceptions or by code
that needs to save and restore the error indicator temporarily.
\end{cfuncdesc}

\begin{cfuncdesc}{void}{PyErr_Restore}{PyObject *type, PyObject *value, PyObject *traceback}
Set  the error indicator from the three objects.  If the error
indicator is already set, it is cleared first.  If the objects are
\NULL{}, the error indicator is cleared.  Do not pass a \NULL{} type
and non-\NULL{} value or traceback.  The exception type should be a
string or class; if it is a class, the value should be an instance of
that class.  Do not pass an invalid exception type or value.
(Violating these rules will cause subtle problems later.)  This call
takes away a reference to each object, i.e. you must own a reference
to each object before the call and after the call you no longer own
these references.  (If you don't understand this, don't use this
function.  I warned you.)  \strong{Note:} this function is normally
only used by code that needs to save and restore the error indicator
temporarily.
\end{cfuncdesc}

\begin{cfuncdesc}{void}{PyErr_SetString}{PyObject *type, char *message}
This is the most common way to set the error indicator.  The first
argument specifies the exception type; it is normally one of the
standard exceptions, e.g. \code{PyExc_RuntimeError}.  You need not
increment its reference count.  The second argument is an error
message; it is converted to a string object.
\end{cfuncdesc}

\begin{cfuncdesc}{void}{PyErr_SetObject}{PyObject *type, PyObject *value}
This function is similar to \code{PyErr_SetString()} but lets you
specify an arbitrary Python object for the ``value'' of the exception.
You need not increment its reference count.
\end{cfuncdesc}

\begin{cfuncdesc}{void}{PyErr_SetNone}{PyObject *type}
This is a shorthand for \code{PyErr_SetString(\var{type}, Py_None}.
\end{cfuncdesc}

\begin{cfuncdesc}{int}{PyErr_BadArgument}{}
This is a shorthand for \code{PyErr_SetString(PyExc_TypeError,
\var{message})}, where \var{message} indicates that a built-in operation
was invoked with an illegal argument.  It is mostly for internal use.
\end{cfuncdesc}

\begin{cfuncdesc}{PyObject *}{PyErr_NoMemory}{}
This is a shorthand for \code{PyErr_SetNone(PyExc_MemoryError)}; it
returns \NULL{} so an object allocation function can write
\code{return PyErr_NoMemory();} when  it runs out of memory.
\end{cfuncdesc}

\begin{cfuncdesc}{PyObject *}{PyErr_SetFromErrno}{PyObject *type}
This is a convenience function to raise an exception when a C library
function has returned an error and set the C variable \code{errno}.
It constructs a tuple object whose first item is the integer
\code{errno} value and whose second item is the corresponding error
message (gotten from \code{strerror()}), and then calls
\code{PyErr_SetObject(\var{type}, \var{object})}.  On \UNIX{}, when
the \code{errno} value is \code{EINTR}, indicating an interrupted
system call, this calls \code{PyErr_CheckSignals()}, and if that set
the error indicator, leaves it set to that.  The function always
returns \NULL{}, so a wrapper function around a system call can write 
\code{return PyErr_NoMemory();} when  the system call returns an error.
\end{cfuncdesc}

\begin{cfuncdesc}{void}{PyErr_BadInternalCall}{}
This is a shorthand for \code{PyErr_SetString(PyExc_TypeError,
\var{message})}, where \var{message} indicates that an internal
operation (e.g. a Python-C API function) was invoked with an illegal
argument.  It is mostly for internal use.
\end{cfuncdesc}

\begin{cfuncdesc}{int}{PyErr_CheckSignals}{}
This function interacts with Python's signal handling.  It checks
whether a signal has been sent to the processes and if so, invokes the
corresponding signal handler.  If the \code{signal} module is
supported, this can invoke a signal handler written in Python.  In all
cases, the default effect for \code{SIGINT} is to raise the
\code{KeyboadInterrupt} exception.  If an exception is raised the
error indicator is set and the function returns 1; otherwise the
function returns 0.  The error indicator may or may not be cleared if
it was previously set.
\end{cfuncdesc}

\begin{cfuncdesc}{void}{PyErr_SetInterrupt}{}
This function is obsolete (XXX or platform dependent?).  It simulates
the effect of a \code{SIGINT} signal arriving -- the next time
\code{PyErr_CheckSignals()} is called, \code{KeyboadInterrupt} will be
raised.
\end{cfuncdesc}

\section{Standard Exceptions}

All standard Python exceptions are available as global variables whose
names are \code{PyExc_} followed by the Python exception name.
These have the type \code{PyObject *}; they are all string objects.
For completion, here are all the variables:
\code{PyExc_AccessError},
\code{PyExc_AssertionError},
\code{PyExc_AttributeError},
\code{PyExc_EOFError},
\code{PyExc_FloatingPointError},
\code{PyExc_IOError},
\code{PyExc_ImportError},
\code{PyExc_IndexError},
\code{PyExc_KeyError},
\code{PyExc_KeyboardInterrupt},
\code{PyExc_MemoryError},
\code{PyExc_NameError},
\code{PyExc_OverflowError},
\code{PyExc_RuntimeError},
\code{PyExc_SyntaxError},
\code{PyExc_SystemError},
\code{PyExc_SystemExit},
\code{PyExc_TypeError},
\code{PyExc_ValueError},
\code{PyExc_ZeroDivisionError}.


\chapter{Utilities}

The functions in this chapter perform various utility tasks, such as
parsing function arguments and constructing Python values from C
values.

\begin{cfuncdesc}{int}{Py_FdIsInteractive}{FILE *fp, char *filename}
Return true (nonzero) if the standard I/O file \code{fp} with name
\code{filename} is deemed interactive.  This is the case for files for
which \code{isatty(fileno(fp))} is true.  If the global flag
\code{Py_InteractiveFlag} is true, this function also returns true if
the \code{name} pointer is \NULL{} or if the name is equal to one of
the strings \code{"<stdin>"} or \code{"???"}.
\end{cfuncdesc}

\begin{cfuncdesc}{long}{PyOS_GetLastModificationTime}{char *filename}
Return the time of last modification of the file \code{filename}.
The result is encoded in the same way as the timestamp returned by
the standard C library function \code{time()}.
\end{cfuncdesc}


\chapter{Debugging}

XXX Explain Py_DEBUG, Py_TRACE_REFS, Py_REF_DEBUG.


\chapter{The Very High Level Layer}

The functions in this chapter will let you execute Python source code
given in a file or a buffer, but they will not let you interact in a
more detailed way with the interpreter.

\begin{cfuncdesc}{int}{PyRun_AnyFile}{FILE *, char *}
\end{cfuncdesc}

\begin{cfuncdesc}{int}{PyRun_SimpleString}{char *}
\end{cfuncdesc}

\begin{cfuncdesc}{int}{PyRun_SimpleFile}{FILE *, char *}
\end{cfuncdesc}

\begin{cfuncdesc}{int}{PyRun_InteractiveOne}{FILE *, char *}
\end{cfuncdesc}

\begin{cfuncdesc}{int}{PyRun_InteractiveLoop}{FILE *, char *}
\end{cfuncdesc}

\begin{cfuncdesc}{struct _node *}{PyParser_SimpleParseString}{char *, int}
\end{cfuncdesc}

\begin{cfuncdesc}{struct _node *}{PyParser_SimpleParseFile}{FILE *, char *, int}
\end{cfuncdesc}

\begin{cfuncdesc}{}{PyObject *PyRun}{ROTO((char *, int, PyObject *, PyObject *}
\end{cfuncdesc}

\begin{cfuncdesc}{}{PyObject *PyRun}{ROTO((FILE *, char *, int, PyObject *, PyObject *}
\end{cfuncdesc}

\begin{cfuncdesc}{}{PyObject *Py}{ROTO((char *, char *, int}
\end{cfuncdesc}


\chapter{Abstract Objects Layer}

The functions in this chapter interact with Python objects regardless
of their type, or with wide classes of object types (e.g. all
numerical types, or all sequence types).  When used on object types
for which they do not apply, they will flag a Python exception.

\section{Object Protocol}

\begin{cfuncdesc}{int}{PyObject_Print}{PyObject *o, FILE *fp, int flags}
Print an object \code{o}, on file \code{fp}.  Returns -1 on error
The flags argument is used to enable certain printing
options. The only option currently supported is \code{Py_Print_RAW}. 
\end{cfuncdesc}

\begin{cfuncdesc}{int}{PyObject_HasAttrString}{PyObject *o, char *attr_name}
Returns 1 if o has the attribute attr_name, and 0 otherwise.
This is equivalent to the Python expression:
\code{hasattr(o,attr_name)}.
This function always succeeds.
\end{cfuncdesc}

\begin{cfuncdesc}{PyObject*}{PyObject_GetAttrString}{PyObject *o, char *attr_name}
Retrieve an attributed named attr_name from object o.
Returns the attribute value on success, or \NULL{} on failure.
This is the equivalent of the Python expression: \code{o.attr_name}.
\end{cfuncdesc}


\begin{cfuncdesc}{int}{PyObject_HasAttr}{PyObject *o, PyObject *attr_name}
Returns 1 if o has the attribute attr_name, and 0 otherwise.
This is equivalent to the Python expression:
\code{hasattr(o,attr_name)}. 
This function always succeeds.
\end{cfuncdesc}


\begin{cfuncdesc}{PyObject*}{PyObject_GetAttr}{PyObject *o, PyObject *attr_name}
Retrieve an attributed named attr_name form object o.
Returns the attribute value on success, or \NULL{} on failure.
This is the equivalent of the Python expression: o.attr_name.
\end{cfuncdesc}


\begin{cfuncdesc}{int}{PyObject_SetAttrString}{PyObject *o, char *attr_name, PyObject *v}
Set the value of the attribute named \code{attr_name}, for object \code{o},
to the value \code{v}. Returns -1 on failure.  This is
the equivalent of the Python statement: \code{o.attr_name=v}.
\end{cfuncdesc}


\begin{cfuncdesc}{int}{PyObject_SetAttr}{PyObject *o, PyObject *attr_name, PyObject *v}
Set the value of the attribute named \code{attr_name}, for
object \code{o},
to the value \code{v}. Returns -1 on failure.  This is
the equivalent of the Python statement: \code{o.attr_name=v}.
\end{cfuncdesc}


\begin{cfuncdesc}{int}{PyObject_DelAttrString}{PyObject *o, char *attr_name}
Delete attribute named \code{attr_name}, for object \code{o}. Returns -1 on
failure.  This is the equivalent of the Python
statement: \code{del o.attr_name}.
\end{cfuncdesc}


\begin{cfuncdesc}{int}{PyObject_DelAttr}{PyObject *o, PyObject *attr_name}
Delete attribute named \code{attr_name}, for object \code{o}. Returns -1 on
failure.  This is the equivalent of the Python
statement: \code{del o.attr_name}.
\end{cfuncdesc}


\begin{cfuncdesc}{int}{PyObject_Cmp}{PyObject *o1, PyObject *o2, int *result}
Compare the values of \code{o1} and \code{o2} using a routine provided by
\code{o1}, if one exists, otherwise with a routine provided by \code{o2}.
The result of the comparison is returned in \code{result}.  Returns
-1 on failure.  This is the equivalent of the Python
statement: \code{result=cmp(o1,o2)}.
\end{cfuncdesc}


\begin{cfuncdesc}{int}{PyObject_Compare}{PyObject *o1, PyObject *o2}
Compare the values of \code{o1} and \code{o2} using a routine provided by
\code{o1}, if one exists, otherwise with a routine provided by \code{o2}.
Returns the result of the comparison on success.  On error,
the value returned is undefined. This is equivalent to the
Python expression: \code{cmp(o1,o2)}.
\end{cfuncdesc}


\begin{cfuncdesc}{PyObject*}{PyObject_Repr}{PyObject *o}
Compute the string representation of object, \code{o}.  Returns the
string representation on success, \NULL{} on failure.  This is
the equivalent of the Python expression: \code{repr(o)}.
Called by the \code{repr()} built-in function and by reverse quotes.
\end{cfuncdesc}


\begin{cfuncdesc}{PyObject*}{PyObject_Str}{PyObject *o}
Compute the string representation of object, \code{o}.  Returns the
string representation on success, \NULL{} on failure.  This is
the equivalent of the Python expression: \code{str(o)}.
Called by the \code{str()} built-in function and by the \code{print}
statement.
\end{cfuncdesc}


\begin{cfuncdesc}{int}{PyCallable_Check}{PyObject *o}
Determine if the object \code{o}, is callable.  Return 1 if the
object is callable and 0 otherwise.
This function always succeeds.
\end{cfuncdesc}


\begin{cfuncdesc}{PyObject*}{PyObject_CallObject}{PyObject *callable_object, PyObject *args}
Call a callable Python object \code{callable_object}, with
arguments given by the tuple \code{args}.  If no arguments are
needed, then args may be \NULL{}.  Returns the result of the
call on success, or \NULL{} on failure.  This is the equivalent
of the Python expression: \code{apply(o, args)}.
\end{cfuncdesc}

\begin{cfuncdesc}{PyObject*}{PyObject_CallFunction}{PyObject *callable_object, char *format, ...}
Call a callable Python object \code{callable_object}, with a
variable number of C arguments. The C arguments are described
using a mkvalue-style format string. The format may be \NULL{},
indicating that no arguments are provided.  Returns the
result of the call on success, or \NULL{} on failure.  This is
the equivalent of the Python expression: \code{apply(o,args)}.
\end{cfuncdesc}


\begin{cfuncdesc}{PyObject*}{PyObject_CallMethod}{PyObject *o, char *m, char *format, ...}
Call the method named \code{m} of object \code{o} with a variable number of
C arguments.  The C arguments are described by a mkvalue
format string.  The format may be \NULL{}, indicating that no
arguments are provided. Returns the result of the call on
success, or \NULL{} on failure.  This is the equivalent of the
Python expression: \code{o.method(args)}.
Note that Special method names, such as "\code{__add__}",
"\code{__getitem__}", and so on are not supported. The specific
abstract-object routines for these must be used.
\end{cfuncdesc}


\begin{cfuncdesc}{int}{PyObject_Hash}{PyObject *o}
Compute and return the hash value of an object \code{o}.  On
failure, return -1.  This is the equivalent of the Python
expression: \code{hash(o)}.
\end{cfuncdesc}


\begin{cfuncdesc}{int}{PyObject_IsTrue}{PyObject *o}
Returns 1 if the object \code{o} is considered to be true, and
0 otherwise. This is equivalent to the Python expression:
\code{not not o}.
This function always succeeds.
\end{cfuncdesc}


\begin{cfuncdesc}{PyObject*}{PyObject_Type}{PyObject *o}
On success, returns a type object corresponding to the object
type of object \code{o}. On failure, returns \NULL{}.  This is
equivalent to the Python expression: \code{type(o)}.
\end{cfuncdesc}

\begin{cfuncdesc}{int}{PyObject_Length}{PyObject *o}
Return the length of object \code{o}.  If the object \code{o} provides
both sequence and mapping protocols, the sequence length is
returned. On error, -1 is returned.  This is the equivalent
to the Python expression: \code{len(o)}.
\end{cfuncdesc}


\begin{cfuncdesc}{PyObject*}{PyObject_GetItem}{PyObject *o, PyObject *key}
Return element of \code{o} corresponding to the object \code{key} or \NULL{}
on failure. This is the equivalent of the Python expression:
\code{o[key]}.
\end{cfuncdesc}


\begin{cfuncdesc}{int}{PyObject_SetItem}{PyObject *o, PyObject *key, PyObject *v}
Map the object \code{key} to the value \code{v}.
Returns -1 on failure.  This is the equivalent
of the Python statement: \code{o[key]=v}.
\end{cfuncdesc}


\begin{cfuncdesc}{int}{PyObject_DelItem}{PyObject *o, PyObject *key, PyObject *v}
Delete the mapping for \code{key} from \code{*o}.  Returns -1
on failure.
This is the equivalent of the Python statement: \code{del o[key]}.
\end{cfuncdesc}


\section{Number Protocol}

\begin{cfuncdesc}{int}{PyNumber_Check}{PyObject *o}
Returns 1 if the object \code{o} provides numeric protocols, and
false otherwise. 
This function always succeeds.
\end{cfuncdesc}


\begin{cfuncdesc}{PyObject*}{PyNumber_Add}{PyObject *o1, PyObject *o2}
Returns the result of adding \code{o1} and \code{o2}, or null on failure.
This is the equivalent of the Python expression: \code{o1+o2}.
\end{cfuncdesc}


\begin{cfuncdesc}{PyObject*}{PyNumber_Subtract}{PyObject *o1, PyObject *o2}
Returns the result of subtracting \code{o2} from \code{o1}, or null on
failure.  This is the equivalent of the Python expression:
\code{o1-o2}.
\end{cfuncdesc}


\begin{cfuncdesc}{PyObject*}{PyNumber_Multiply}{PyObject *o1, PyObject *o2}
Returns the result of multiplying \code{o1} and \code{o2}, or null on
failure.  This is the equivalent of the Python expression:
\code{o1*o2}.
\end{cfuncdesc}


\begin{cfuncdesc}{PyObject*}{PyNumber_Divide}{PyObject *o1, PyObject *o2}
Returns the result of dividing \code{o1} by \code{o2}, or null on failure.
This is the equivalent of the Python expression: \code{o1/o2}.
\end{cfuncdesc}


\begin{cfuncdesc}{PyObject*}{PyNumber_Remainder}{PyObject *o1, PyObject *o2}
Returns the remainder of dividing \code{o1} by \code{o2}, or null on
failure.  This is the equivalent of the Python expression:
\code{o1\%o2}.
\end{cfuncdesc}


\begin{cfuncdesc}{PyObject*}{PyNumber_Divmod}{PyObject *o1, PyObject *o2}
See the built-in function divmod.  Returns \NULL{} on failure.
This is the equivalent of the Python expression:
\code{divmod(o1,o2)}.
\end{cfuncdesc}


\begin{cfuncdesc}{PyObject*}{PyNumber_Power}{PyObject *o1, PyObject *o2, PyObject *o3}
See the built-in function pow.  Returns \NULL{} on failure.
This is the equivalent of the Python expression:
\code{pow(o1,o2,o3)}, where \code{o3} is optional.
\end{cfuncdesc}


\begin{cfuncdesc}{PyObject*}{PyNumber_Negative}{PyObject *o}
Returns the negation of \code{o} on success, or null on failure.
This is the equivalent of the Python expression: \code{-o}.
\end{cfuncdesc}


\begin{cfuncdesc}{PyObject*}{PyNumber_Positive}{PyObject *o}
Returns \code{o} on success, or \NULL{} on failure.
This is the equivalent of the Python expression: \code{+o}.
\end{cfuncdesc}


\begin{cfuncdesc}{PyObject*}{PyNumber_Absolute}{PyObject *o}
Returns the absolute value of \code{o}, or null on failure.  This is
the equivalent of the Python expression: \code{abs(o)}.
\end{cfuncdesc}


\begin{cfuncdesc}{PyObject*}{PyNumber_Invert}{PyObject *o}
Returns the bitwise negation of \code{o} on success, or \NULL{} on
failure.  This is the equivalent of the Python expression:
\code{\~o}.
\end{cfuncdesc}


\begin{cfuncdesc}{PyObject*}{PyNumber_Lshift}{PyObject *o1, PyObject *o2}
Returns the result of left shifting \code{o1} by \code{o2} on success, or
\NULL{} on failure.  This is the equivalent of the Python
expression: \code{o1 << o2}.
\end{cfuncdesc}


\begin{cfuncdesc}{PyObject*}{PyNumber_Rshift}{PyObject *o1, PyObject *o2}
Returns the result of right shifting \code{o1} by \code{o2} on success, or
\NULL{} on failure.  This is the equivalent of the Python
expression: \code{o1 >> o2}.
\end{cfuncdesc}


\begin{cfuncdesc}{PyObject*}{PyNumber_And}{PyObject *o1, PyObject *o2}
Returns the result of "anding" \code{o2} and \code{o2} on success and \NULL{}
on failure. This is the equivalent of the Python
expression: \code{o1 and o2}.
\end{cfuncdesc}


\begin{cfuncdesc}{PyObject*}{PyNumber_Xor}{PyObject *o1, PyObject *o2}
Returns the bitwise exclusive or of \code{o1} by \code{o2} on success, or
\NULL{} on failure.  This is the equivalent of the Python
expression: \code{o1\^{ }o2}.
\end{cfuncdesc}

\begin{cfuncdesc}{PyObject*}{PyNumber_Or}{PyObject *o1, PyObject *o2}
Returns the result of \code{o1} and \code{o2} on success, or \NULL{} on
failure.  This is the equivalent of the Python expression: 
\code{o1 or o2}.
\end{cfuncdesc}


\begin{cfuncdesc}{PyObject*}{PyNumber_Coerce}{PyObject *o1, PyObject *o2}
This function takes the addresses of two variables of type
\code{PyObject*}.

If the objects pointed to by \code{*p1} and \code{*p2} have the same type,
increment their reference count and return 0 (success).
If the objects can be converted to a common numeric type,
replace \code{*p1} and \code{*p2} by their converted value (with 'new'
reference counts), and return 0.
If no conversion is possible, or if some other error occurs,
return -1 (failure) and don't increment the reference counts.
The call \code{PyNumber_Coerce(\&o1, \&o2)} is equivalent to the Python
statement \code{o1, o2 = coerce(o1, o2)}.
\end{cfuncdesc}


\begin{cfuncdesc}{PyObject*}{PyNumber_Int}{PyObject *o}
Returns the \code{o} converted to an integer object on success, or
\NULL{} on failure.  This is the equivalent of the Python
expression: \code{int(o)}.
\end{cfuncdesc}


\begin{cfuncdesc}{PyObject*}{PyNumber_Long}{PyObject *o}
Returns the \code{o} converted to a long integer object on success,
or \NULL{} on failure.  This is the equivalent of the Python
expression: \code{long(o)}.
\end{cfuncdesc}


\begin{cfuncdesc}{PyObject*}{PyNumber_Float}{PyObject *o}
Returns the \code{o} converted to a float object on success, or \NULL{}
on failure.  This is the equivalent of the Python expression:
\code{float(o)}.
\end{cfuncdesc}


\section{Sequence protocol}

\begin{cfuncdesc}{int}{PySequence_Check}{PyObject *o}
Return 1 if the object provides sequence protocol, and 0
otherwise.  
This function always succeeds.
\end{cfuncdesc}


\begin{cfuncdesc}{PyObject*}{PySequence_Concat}{PyObject *o1, PyObject *o2}
Return the concatination of \code{o1} and \code{o2} on success, and \NULL{} on
failure.   This is the equivalent of the Python
expression: \code{o1+o2}.
\end{cfuncdesc}


\begin{cfuncdesc}{PyObject*}{PySequence_Repeat}{PyObject *o, int count}
Return the result of repeating sequence object \code{o} \code{count} times,
or \NULL{} on failure.  This is the equivalent of the Python
expression: \code{o*count}.
\end{cfuncdesc}


\begin{cfuncdesc}{PyObject*}{PySequence_GetItem}{PyObject *o, int i}
Return the ith element of \code{o}, or \NULL{} on failure. This is the
equivalent of the Python expression: \code{o[i]}.
\end{cfuncdesc}


\begin{cfuncdesc}{PyObject*}{PySequence_GetSlice}{PyObject *o, int i1, int i2}
Return the slice of sequence object \code{o} between \code{i1} and \code{i2}, or
\NULL{} on failure. This is the equivalent of the Python
expression, \code{o[i1:i2]}.
\end{cfuncdesc}


\begin{cfuncdesc}{int}{PySequence_SetItem}{PyObject *o, int i, PyObject *v}
Assign object \code{v} to the \code{i}th element of \code{o}.
Returns -1 on failure.  This is the equivalent of the Python
statement, \code{o[i]=v}.
\end{cfuncdesc}

\begin{cfuncdesc}{int}{PySequence_DelItem}{PyObject *o, int i}
Delete the \code{i}th element of object \code{v}.  Returns
-1 on failure.  This is the equivalent of the Python
statement: \code{del o[i]}.
\end{cfuncdesc}

\begin{cfuncdesc}{int}{PySequence_SetSlice}{PyObject *o, int i1, int i2, PyObject *v}
Assign the sequence object \code{v} to the slice in sequence
object \code{o} from \code{i1} to \code{i2}.  This is the equivalent of the Python
statement, \code{o[i1:i2]=v}.
\end{cfuncdesc}

\begin{cfuncdesc}{int}{PySequence_DelSlice}{PyObject *o, int i1, int i2}
Delete the slice in sequence object, \code{o}, from \code{i1} to \code{i2}.
Returns -1 on failure. This is the equivalent of the Python
statement: \code{del o[i1:i2]}.
\end{cfuncdesc}

\begin{cfuncdesc}{PyObject*}{PySequence_Tuple}{PyObject *o}
Returns the \code{o} as a tuple on success, and \NULL{} on failure.
This is equivalent to the Python expression: \code{tuple(o)}.
\end{cfuncdesc}

\begin{cfuncdesc}{int}{PySequence_Count}{PyObject *o, PyObject *value}
Return the number of occurrences of \code{value} on \code{o}, that is,
return the number of keys for which \code{o[key]==value}.  On
failure, return -1.  This is equivalent to the Python
expression: \code{o.count(value)}.
\end{cfuncdesc}

\begin{cfuncdesc}{int}{PySequence_In}{PyObject *o, PyObject *value}
Determine if \code{o} contains \code{value}.  If an item in \code{o} is equal to
\code{value}, return 1, otherwise return 0.  On error, return -1.  This
is equivalent to the Python expression: \code{value in o}.
\end{cfuncdesc}

\begin{cfuncdesc}{int}{PySequence_Index}{PyObject *o, PyObject *value}
Return the first index for which \code{o[i]==value}.  On error,
return -1.    This is equivalent to the Python
expression: \code{o.index(value)}.
\end{cfuncdesc}

\section{Mapping protocol}

\begin{cfuncdesc}{int}{PyMapping_Check}{PyObject *o}
Return 1 if the object provides mapping protocol, and 0
otherwise.  
This function always succeeds.
\end{cfuncdesc}


\begin{cfuncdesc}{int}{PyMapping_Length}{PyObject *o}
Returns the number of keys in object \code{o} on success, and -1 on
failure.  For objects that do not provide sequence protocol,
this is equivalent to the Python expression: \code{len(o)}.
\end{cfuncdesc}


\begin{cfuncdesc}{int}{PyMapping_DelItemString}{PyObject *o, char *key}
Remove the mapping for object \code{key} from the object \code{o}.
Return -1 on failure.  This is equivalent to
the Python statement: \code{del o[key]}.
\end{cfuncdesc}


\begin{cfuncdesc}{int}{PyMapping_DelItem}{PyObject *o, PyObject *key}
Remove the mapping for object \code{key} from the object \code{o}.
Return -1 on failure.  This is equivalent to
the Python statement: \code{del o[key]}.
\end{cfuncdesc}


\begin{cfuncdesc}{int}{PyMapping_HasKeyString}{PyObject *o, char *key}
On success, return 1 if the mapping object has the key \code{key}
and 0 otherwise.  This is equivalent to the Python expression:
\code{o.has_key(key)}. 
This function always succeeds.
\end{cfuncdesc}


\begin{cfuncdesc}{int}{PyMapping_HasKey}{PyObject *o, PyObject *key}
Return 1 if the mapping object has the key \code{key}
and 0 otherwise.  This is equivalent to the Python expression:
\code{o.has_key(key)}. 
This function always succeeds.
\end{cfuncdesc}


\begin{cfuncdesc}{PyObject*}{PyMapping_Keys}{PyObject *o}
On success, return a list of the keys in object \code{o}.  On
failure, return \NULL{}. This is equivalent to the Python
expression: \code{o.keys()}.
\end{cfuncdesc}


\begin{cfuncdesc}{PyObject*}{PyMapping_Values}{PyObject *o}
On success, return a list of the values in object \code{o}.  On
failure, return \NULL{}. This is equivalent to the Python
expression: \code{o.values()}.
\end{cfuncdesc}


\begin{cfuncdesc}{PyObject*}{PyMapping_Items}{PyObject *o}
On success, return a list of the items in object \code{o}, where
each item is a tuple containing a key-value pair.  On
failure, return \NULL{}. This is equivalent to the Python
expression: \code{o.items()}.
\end{cfuncdesc}

\begin{cfuncdesc}{int}{PyMapping_Clear}{PyObject *o}
Make object \code{o} empty.  Returns 1 on success and 0 on failure.
This is equivalent to the Python statement:
\code{for key in o.keys(): del o[key]}
\end{cfuncdesc}


\begin{cfuncdesc}{PyObject*}{PyMapping_GetItemString}{PyObject *o, char *key}
Return element of \code{o} corresponding to the object \code{key} or \NULL{}
on failure. This is the equivalent of the Python expression:
\code{o[key]}.
\end{cfuncdesc}

\begin{cfuncdesc}{PyObject*}{PyMapping_SetItemString}{PyObject *o, char *key, PyObject *v}
Map the object \code{key} to the value \code{v} in object \code{o}.  Returns 
-1 on failure.  This is the equivalent of the Python
statement: \code{o[key]=v}.
\end{cfuncdesc}


\section{Constructors}

\begin{cfuncdesc}{PyObject*}{PyFile_FromString}{char *file_name, char *mode}
On success, returns a new file object that is opened on the
file given by \code{file_name}, with a file mode given by \code{mode},
where \code{mode} has the same semantics as the standard C routine,
fopen.  On failure, return -1.
\end{cfuncdesc}

\begin{cfuncdesc}{PyObject*}{PyFile_FromFile}{FILE *fp, char *file_name, char *mode, int close_on_del}
Return a new file object for an already opened standard C
file pointer, \code{fp}.  A file name, \code{file_name}, and open mode,
\code{mode}, must be provided as well as a flag, \code{close_on_del}, that
indicates whether the file is to be closed when the file
object is destroyed.  On failure, return -1.
\end{cfuncdesc}

\begin{cfuncdesc}{PyObject*}{PyFloat_FromDouble}{double v}
Returns a new float object with the value \code{v} on success, and
\NULL{} on failure.
\end{cfuncdesc}

\begin{cfuncdesc}{PyObject*}{PyInt_FromLong}{long v}
Returns a new int object with the value \code{v} on success, and
\NULL{} on failure.
\end{cfuncdesc}

\begin{cfuncdesc}{PyObject*}{PyList_New}{int l}
Returns a new list of length \code{l} on success, and \NULL{} on
failure.
\end{cfuncdesc}

\begin{cfuncdesc}{PyObject*}{PyLong_FromLong}{long v}
Returns a new long object with the value \code{v} on success, and
\NULL{} on failure.
\end{cfuncdesc}

\begin{cfuncdesc}{PyObject*}{PyLong_FromDouble}{double v}
Returns a new long object with the value \code{v} on success, and
\NULL{} on failure.
\end{cfuncdesc}

\begin{cfuncdesc}{PyObject*}{PyDict_New}{}
Returns a new empty dictionary on success, and \NULL{} on
failure.
\end{cfuncdesc}

\begin{cfuncdesc}{PyObject*}{PyString_FromString}{char *v}
Returns a new string object with the value \code{v} on success, and
\NULL{} on failure.
\end{cfuncdesc}

\begin{cfuncdesc}{PyObject*}{PyString_FromStringAndSize}{char *v, int l}
Returns a new string object with the value \code{v} and length \code{l}
on success, and \NULL{} on failure.
\end{cfuncdesc}

\begin{cfuncdesc}{PyObject*}{PyTuple_New}{int l}
Returns a new tuple of length \code{l} on success, and \NULL{} on
failure.
\end{cfuncdesc}


\chapter{Concrete Objects Layer}

The functions in this chapter are specific to certain Python object
types.  Passing them an object of the wrong type is not a good idea;
if you receive an object from a Python program and you are not sure
that it has the right type, you must perform a type check first;
e.g. to check that an object is a dictionary, use
\code{PyDict_Check()}.


\chapter{Defining New Object Types}

\begin{cfuncdesc}{PyObject *}{_PyObject_New}{PyTypeObject *type}
\end{cfuncdesc}

\begin{cfuncdesc}{PyObject *}{_PyObject_NewVar}{PyTypeObject *type, int size}
\end{cfuncdesc}

\begin{cfuncdesc}{TYPE}{_PyObject_NEW}{TYPE, PyTypeObject *}
\end{cfuncdesc}

\begin{cfuncdesc}{TYPE}{_PyObject_NEW_VAR}{TYPE, PyTypeObject *, int size}
\end{cfuncdesc}

\chapter{Initialization, Finalization, and Threads}

% XXX Check argument/return type of all these

\begin{cfuncdesc}{void}{Py_Initialize}{}
Initialize the Python interpreter.  In an application embedding 
Python, this should be called before using any other Python/C API 
functions; with the exception of \code{Py_SetProgramName()}, 
\code{PyEval_InitThreads()}, \code{PyEval_ReleaseLock()}, and 
\code{PyEval_AcquireLock()}.  This initializes the table of loaded 
modules (\code{sys.modules}), and creates the fundamental modules 
\code{__builtin__}, \code{__main__} and \code{sys}.  It also 
initializes the module search path (\code{sys.path}).  It does not set 
\code{sys.argv}; use \code{PySys_SetArgv()} for that.  It is a fatal 
error to call it for a second time without calling 
\code{Py_Finalize()} first.  There is no return value; it is a fatal 
error if the initialization fails.
\end{cfuncdesc}

\begin{cfuncdesc}{void}{Py_Finalize}{}
Undo all initializations made by \code{Py_Initialize()} and subsequent 
use of Python/C API functions, and destroy all sub-interpreters (see 
\code{Py_NewInterpreter()} below) that were created and not yet 
destroyed since the last call to \code{Py_Initialize()}.  Ideally, 
this frees all memory allocated by the Python interpreter.  It is a 
fatal error to call it for a second time without calling 
\code{Py_Initialize()} again first.  There is no return value; errors 
during finalization are ignored.

This function is provided for a number of reasons.  An embedding 
application might want to restart Python without having to restart the 
application itself.  An application that has loaded the Python 
interpreter from a dynamically loadable library (or DLL) might want to 
free all memory allocated by Python before unloading the DLL. During a 
hunt for memory leaks in an application a developer might want to free 
all memory allocated by Python before exiting from the application.

\emph{Bugs and caveats:} The destruction of modules and objects in 
modules is done in random order; this may cause destructors 
(\code{__del__} methods) to fail when they depend on other objects 
(even functions) or modules.  Dynamically loaded extension modules 
loaded by Python are not unloaded.  Small amounts of memory allocated 
by the Python interpreter may not be freed (if you find a leak, please 
report it).  Memory tied up in circular references between objects is 
not freed.  Some memory allocated by extension modules may not be 
freed.  Some extension may not work properly if their initialization 
routine is called more than once; this can happen if an applcation 
calls \code{Py_Initialize()} and \code{Py_Finalize()} more than once.
\end{cfuncdesc}

\begin{cfuncdesc}{PyThreadState *}{Py_NewInterpreter}{}
Create a new sub-interpreter.  This is an (almost) totally separate 
environment for the execution of Python code.  In particular, the new 
interpreter has separate, independent versions of all imported 
modules, including the fundamental modules \code{__builtin__}, 
\code{__main__} and \code{sys}.  The table of loaded modules 
(\code{sys.modules}) and the module search path (\code{sys.path}) are 
also separate.  The new environment has no \code{sys.argv} variable.  
It has new standard I/O stream file objects \code{sys.stdin}, 
\code{sys.stdout} and \code{sys.stderr} (however these refer to the 
same underlying \code{FILE} structures in the C library).

The return value points to the first thread state created in the new 
sub-interpreter.  This thread state is made the current thread state.  
Note that no actual thread is created; see the discussion of thread 
states below.  If creation of the new interpreter is unsuccessful, 
\code{NULL} is returned; no exception is set since the exception state 
is stored in the current thread state and there may not be a current 
thread state.  (Like all other Python/C API functions, the global 
interpreter lock must be held before calling this function and is 
still held when it returns; however, unlike most other Python/C API 
functions, there needn't be a current thread state on entry.)

Extension modules are shared between (sub-)interpreters as follows: 
the first time a particular extension is imported, it is initialized 
normally, and a (shallow) copy of its module's dictionary is 
squirreled away.  When the same extension is imported by another 
(sub-)interpreter, a new module is initialized and filled with the 
contents of this copy; the extension's \code{init} function is not 
called.  Note that this is different from what happens when as 
extension is imported after the interpreter has been completely 
re-initialized by calling \code{Py_Finalize()} and 
\code{Py_Initialize()}; in that case, the extension's \code{init} 
function \emph{is} called again.

\emph{Bugs and caveats:} Because sub-interpreters (and the main 
interpreter) are part of the same process, the insulation between them 
isn't perfect -- for example, using low-level file operations like 
\code{os.close()} they can (accidentally or maliciously) affect each 
other's open files.  Because of the way extensions are shared between 
(sub-)interpreters, some extensions may not work properly; this is 
especially likely when the extension makes use of (static) global 
variables, or when the extension manipulates its module's dictionary 
after its initialization.  It is possible to insert objects created in 
one sub-interpreter into a namespace of another sub-interpreter; this 
should be done with great care to avoid sharing user-defined 
functions, methods, instances or classes between sub-interpreters, 
since import operations executed by such objects may affect the 
wrong (sub-)interpreter's dictionary of loaded modules.  (XXX This is 
a hard-to-fix bug that will be addressed in a future release.)
\end{cfuncdesc}

\begin{cfuncdesc}{void}{Py_EndInterpreter}{PyThreadState *tstate}
Destroy the (sub-)interpreter represented by the given thread state.  
The given thread state must be the current thread state.  See the 
discussion of thread states below.  When the call returns, the current 
thread state is \code{NULL}.  All thread states associated with this 
interpreted are destroyed.  (The global interpreter lock must be held 
before calling this function and is still held when it returns.)  
\code{Py_Finalize()} will destroy all sub-interpreters that haven't 
been explicitly destroyed at that point.
\end{cfuncdesc}

\begin{cfuncdesc}{void}{Py_SetProgramName}{char *name}
This function should be called before \code{Py_Initialize()} is called 
for the first time, if it is called at all.  It tells the interpreter 
the value of the \code{argv[0]} argument to the \code{main()} function 
of the program.  This is used by \code{Py_GetPath()} and some other 
functions below to find the Python run-time libraries relative to the 
interpreter executable.  The default value is \code{"python"}.  The 
argument should point to a zero-terminated character string in static 
storage whose contents will not change for the duration of the 
program's execution.  No code in the Python interpreter will change 
the contents of this storage.
\end{cfuncdesc}

\begin{cfuncdesc}{char *}{Py_GetProgramName}{}
Return the program name set with \code{Py_SetProgramName()}, or the 
default.  The returned string points into static storage; the caller 
should not modify its value.
\end{cfuncdesc}

\begin{cfuncdesc}{char *}{Py_GetPrefix}{}
Return the ``prefix'' for installed platform-independent files.  This 
is derived through a number of complicated rules from the program name 
set with \code{Py_SetProgramName()} and some environment variables; 
for example, if the program name is \code{"/usr/local/bin/python"}, 
the prefix is \code{"/usr/local"}.  The returned string points into 
static storage; the caller should not modify its value.  This 
corresponds to the \code{prefix} variable in the top-level 
\code{Makefile} and the \code{--prefix} argument to the 
\code{configure} script at build time.  The value is available to 
Python code as \code{sys.prefix}.  It is only useful on Unix.  See 
also the next function.
\end{cfuncdesc}

\begin{cfuncdesc}{char *}{Py_GetExecPrefix}{}
Return the ``exec-prefix'' for installed platform-\emph{de}pendent 
files.  This is derived through a number of complicated rules from the 
program name set with \code{Py_SetProgramName()} and some environment 
variables; for example, if the program name is 
\code{"/usr/local/bin/python"}, the exec-prefix is 
\code{"/usr/local"}.  The returned string points into static storage; 
the caller should not modify its value.  This corresponds to the 
\code{exec_prefix} variable in the top-level \code{Makefile} and the 
\code{--exec_prefix} argument to the \code{configure} script at build 
time.  The value is available to Python code as 
\code{sys.exec_prefix}.  It is only useful on Unix.

Background: The exec-prefix differs from the prefix when platform 
dependent files (such as executables and shared libraries) are 
installed in a different directory tree.  In a typical installation, 
platform dependent files may be installed in the 
\code{"/usr/local/plat"} subtree while platform independent may be 
installed in \code{"/usr/local"}.

Generally speaking, a platform is a combination of hardware and 
software families, e.g.  Sparc machines running the Solaris 2.x 
operating system are considered the same platform, but Intel machines 
running Solaris 2.x are another platform, and Intel machines running 
Linux are yet another platform.  Different major revisions of the same 
operating system generally also form different platforms.  Non-Unix 
operating systems are a different story; the installation strategies 
on those systems are so different that the prefix and exec-prefix are 
meaningless, and set to the empty string.  Note that compiled Python 
bytecode files are platform independent (but not independent from the 
Python version by which they were compiled!).

System administrators will know how to configure the \code{mount} or 
\code{automount} programs to share \code{"/usr/local"} between platforms 
while having \code{"/usr/local/plat"} be a different filesystem for each 
platform.
\end{cfuncdesc}

\begin{cfuncdesc}{char *}{Py_GetProgramFullPath}{}
Return the full program name of the Python executable; this is 
computed as a side-effect of deriving the default module search path 
from the program name (set by \code{Py_SetProgramName()} above).  The 
returned string points into static storage; the caller should not 
modify its value.  The value is available to Python code as 
\code{sys.executable}.  % XXX is that the right sys.name?
\end{cfuncdesc}

\begin{cfuncdesc}{char *}{Py_GetPath}{}
Return the default module search path; this is computed from the 
program name (set by \code{Py_SetProgramName()} above) and some 
environment variables.  The returned string consists of a series of 
directory names separated by a platform dependent delimiter character.  
The delimiter character is \code{':'} on Unix, \code{';'} on 
DOS/Windows, and \code{'\\n'} (the ASCII newline character) on 
Macintosh.  The returned string points into static storage; the caller 
should not modify its value.  The value is available to Python code 
as the list \code{sys.path}, which may be modified to change the 
future search path for loaded modules.

% XXX should give the exact rules
\end{cfuncdesc}

\begin{cfuncdesc}{const char *}{Py_GetVersion}{}
Return the version of this Python interpreter.  This is a string that 
looks something like

\begin{verbatim}
"1.5a3 (#67, Aug 1 1997, 22:34:28) [GCC 2.7.2.2]"
\end{verbatim}

The first word (up to the first space character) is the current Python 
version; the first three characters are the major and minor version 
separated by a period.  The returned string points into static storage; 
the caller should not modify its value.  The value is available to 
Python code as the list \code{sys.version}.
\end{cfuncdesc}

\begin{cfuncdesc}{const char *}{Py_GetPlatform}{}
Return the platform identifier for the current platform.  On Unix, 
this is formed from the ``official'' name of the operating system, 
converted to lower case, followed by the major revision number; e.g., 
for Solaris 2.x, which is also known as SunOS 5.x, the value is 
\code{"sunos5"}.  On Macintosh, it is \code{"mac"}.  On Windows, it 
is \code{"win"}.  The returned string points into static storage; 
the caller should not modify its value.  The value is available to 
Python code as \code{sys.platform}.
\end{cfuncdesc}

\begin{cfuncdesc}{const char *}{Py_GetCopyright}{}
Return the official copyright string for the current Python version, 
for example

\code{"Copyright 1991-1995 Stichting Mathematisch Centrum, Amsterdam"}

The returned string points into static storage; the caller should not 
modify its value.  The value is available to Python code as the list 
\code{sys.copyright}.
\end{cfuncdesc}

\begin{cfuncdesc}{const char *}{Py_GetCompiler}{}
Return an indication of the compiler used to build the current Python 
version, in square brackets, for example

\code{"[GCC 2.7.2.2]"}

The returned string points into static storage; the caller should not 
modify its value.  The value is available to Python code as part of 
the variable \code{sys.version}.
\end{cfuncdesc}

\begin{cfuncdesc}{const char *}{Py_GetBuildInfo}{}
Return information about the sequence number and build date and time 
of the current Python interpreter instance, for example

\begin{verbatim}
"#67, Aug  1 1997, 22:34:28"
\end{verbatim}

The returned string points into static storage; the caller should not 
modify its value.  The value is available to Python code as part of 
the variable \code{sys.version}.
\end{cfuncdesc}

\begin{cfuncdesc}{int}{PySys_SetArgv}{int argc, char **argv}
% XXX
\end{cfuncdesc}

% XXX Other PySys thingies (doesn't really belong in this chapter)

\section{Thread State and the Global Interpreter Lock}

\begin{cfuncdesc}{void}{PyEval_AcquireLock}{}
\end{cfuncdesc}

\begin{cfuncdesc}{void}{PyEval_ReleaseLock}{}
\end{cfuncdesc}

\begin{cfuncdesc}{void}{PyEval_AcquireThread}{PyThreadState *tstate}
\end{cfuncdesc}

\begin{cfuncdesc}{void}{PyEval_ReleaseThread}{PyThreadState *tstate}
\end{cfuncdesc}

\begin{cfuncdesc}{void}{PyEval_RestoreThread}{PyThreadState *tstate}
\end{cfuncdesc}

\begin{cfuncdesc}{PyThreadState *}{PyEval_SaveThread}{}
\end{cfuncdesc}

% XXX These aren't really C functions!
\begin{cfuncdesc}{}{Py_BEGIN_ALLOW_THREADS}{}
\end{cfuncdesc}

\begin{cfuncdesc}{}{Py_BEGIN_END_THREADS}{}
\end{cfuncdesc}

\begin{cfuncdesc}{}{Py_BEGIN_XXX_THREADS}{}
\end{cfuncdesc}


XXX To be done:

PyObject, PyVarObject

PyObject_HEAD, PyObject_HEAD_INIT, PyObject_VAR_HEAD

Typedefs:
unaryfunc, binaryfunc, ternaryfunc, inquiry, coercion, intargfunc,
intintargfunc, intobjargproc, intintobjargproc, objobjargproc,
getreadbufferproc, getwritebufferproc, getsegcountproc,
destructor, printfunc, getattrfunc, getattrofunc, setattrfunc,
setattrofunc, cmpfunc, reprfunc, hashfunc

PyNumberMethods

PySequenceMethods

PyMappingMethods

PyBufferProcs

PyTypeObject

DL_IMPORT

PyType_Type

Py*_Check

Py_None, _Py_NoneStruct

_PyObject_New, _PyObject_NewVar

PyObject_NEW, PyObject_NEW_VAR


\chapter{Specific Data Types}

This chapter describes the functions that deal with specific types of 
Python objects.  It is structured like the ``family tree'' of Python 
object types.


\section{Fundamental Objects}

This section describes Python type objects and the singleton object 
\code{None}.


\subsection{Type Objects}

\begin{ctypedesc}{PyTypeObject}

\end{ctypedesc}

\begin{cvardesc}{PyObject *}{PyType_Type}

\end{cvardesc}


\subsection{The None Object}

\begin{cvardesc}{PyObject *}{Py_None}
macro
\end{cvardesc}


\section{Sequence Objects}

Generic operations on sequence objects were discussed in the previous 
chapter; this section deals with the specific kinds of sequence 
objects that are intrinsuc to the Python language.


\subsection{String Objects}

\begin{ctypedesc}{PyStringObject}
This subtype of \code{PyObject} represents a Python string object.
\end{ctypedesc}

\begin{cvardesc}{PyTypeObject}{PyString_Type}
This instance of \code{PyTypeObject} represents the Python string type.
\end{cvardesc}

\begin{cfuncdesc}{int}{PyString_Check}{PyObject *o}

\end{cfuncdesc}

\begin{cfuncdesc}{PyObject *}{PyString_FromStringAndSize}{const char *, int}

\end{cfuncdesc}

\begin{cfuncdesc}{PyObject *}{PyString_FromString}{const char *}

\end{cfuncdesc}

\begin{cfuncdesc}{int}{PyString_Size}{PyObject *}

\end{cfuncdesc}

\begin{cfuncdesc}{char *}{PyString_AsString}{PyObject *}

\end{cfuncdesc}

\begin{cfuncdesc}{void}{PyString_Concat}{PyObject **, PyObject *}

\end{cfuncdesc}

\begin{cfuncdesc}{void}{PyString_ConcatAndDel}{PyObject **, PyObject *}

\end{cfuncdesc}

\begin{cfuncdesc}{int}{_PyString_Resize}{PyObject **, int}

\end{cfuncdesc}

\begin{cfuncdesc}{PyObject *}{PyString_Format}{PyObject *, PyObject *}

\end{cfuncdesc}

\begin{cfuncdesc}{void}{PyString_InternInPlace}{PyObject **}

\end{cfuncdesc}

\begin{cfuncdesc}{PyObject *}{PyString_InternFromString}{const char *}

\end{cfuncdesc}

\begin{cfuncdesc}{char *}{PyString_AS_STRING}{PyStringObject *}

\end{cfuncdesc}

\begin{cfuncdesc}{int}{PyString_GET_SIZE}{PyStringObject *}

\end{cfuncdesc}


\subsection{Tuple Objects}

\begin{ctypedesc}{PyTupleObject}
This subtype of \code{PyObject} represents a Python tuple object.
\end{ctypedesc}

\begin{cvardesc}{PyTypeObject}{PyTuple_Type}
This instance of \code{PyTypeObject} represents the Python tuple type.
\end{cvardesc}

\begin{cfuncdesc}{int}{PyTuple_Check}{PyObject *p}
Return true if the argument is a tuple object.
\end{cfuncdesc}

\begin{cfuncdesc}{PyTupleObject *}{PyTuple_New}{int s}
Return a new tuple object of size \code{s}
\end{cfuncdesc}

\begin{cfuncdesc}{int}{PyTuple_Size}{PyTupleObject *p}
akes a pointer to a tuple object, and returns the size
of that tuple.
\end{cfuncdesc}

\begin{cfuncdesc}{PyObject *}{PyTuple_GetItem}{PyTupleObject *p, int pos}
returns the object at position \code{pos} in the tuple pointed
to by \code{p}.
\end{cfuncdesc}

\begin{cfuncdesc}{PyObject *}{PyTuple_GET_ITEM}{PyTupleObject *p, int pos}
does the same, but does no checking of it's
arguments.
\end{cfuncdesc}

\begin{cfuncdesc}{PyTupleObject *}{PyTuple_GetSlice}{PyTupleObject *p,
            int low,
            int high}
takes a slice of the tuple pointed to by \code{p} from
\code{low} to \code{high} and returns it as a new tuple.
\end{cfuncdesc}

\begin{cfuncdesc}{int}{PyTuple_SetItem}{PyTupleObject *p,
            int pos,
            PyObject *o}
inserts a reference to object \code{o} at position \code{pos} of
the tuple pointed to by \code{p}. It returns 0 on success.
\end{cfuncdesc}

\begin{cfuncdesc}{void}{PyTuple_SET_ITEM}{PyTupleObject *p,
            int pos,
            PyObject *o}

does the same, but does no error checking, and
should \emph{only} be used to fill in brand new tuples.
\end{cfuncdesc}

\begin{cfuncdesc}{PyTupleObject *}{_PyTuple_Resize}{PyTupleObject *p,
            int new,
            int last_is_sticky}
can be used to resize a tuple. Because tuples are
\emph{supposed} to be immutable, this should only be used if there is only
one module referencing the object. Do \emph{not} use this if the tuple may
already be known to some other part of the code. \code{last_is_sticky} is
a flag - if set, the tuple will grow or shrink at the front, otherwise
it will grow or shrink at the end. Think of this as destroying the old
tuple and creating a new one, only more efficiently.
\end{cfuncdesc}


\subsection{List Objects}

\begin{ctypedesc}{PyListObject}
This subtype of \code{PyObject} represents a Python list object.
\end{ctypedesc}

\begin{cvardesc}{PyTypeObject}{PyList_Type}
This instance of \code{PyTypeObject} represents the Python list type.
\end{cvardesc}

\begin{cfuncdesc}{int}{PyList_Check}{PyObject *p}
returns true if it's argument is a \code{PyListObject}
\end{cfuncdesc}

\begin{cfuncdesc}{PyObject *}{PyList_New}{int size}

\end{cfuncdesc}

\begin{cfuncdesc}{int}{PyList_Size}{PyObject *}

\end{cfuncdesc}

\begin{cfuncdesc}{PyObject *}{PyList_GetItem}{PyObject *, int}

\end{cfuncdesc}

\begin{cfuncdesc}{int}{PyList_SetItem}{PyObject *, int, PyObject *}

\end{cfuncdesc}

\begin{cfuncdesc}{int}{PyList_Insert}{PyObject *, int, PyObject *}

\end{cfuncdesc}

\begin{cfuncdesc}{int}{PyList_Append}{PyObject *, PyObject *}

\end{cfuncdesc}

\begin{cfuncdesc}{PyObject *}{PyList_GetSlice}{PyObject *, int, int}

\end{cfuncdesc}

\begin{cfuncdesc}{int}{PyList_SetSlice}{PyObject *, int, int, PyObject *}

\end{cfuncdesc}

\begin{cfuncdesc}{int}{PyList_Sort}{PyObject *}

\end{cfuncdesc}

\begin{cfuncdesc}{int}{PyList_Reverse}{PyObject *}

\end{cfuncdesc}

\begin{cfuncdesc}{PyObject *}{PyList_AsTuple}{PyObject *}

\end{cfuncdesc}

\begin{cfuncdesc}{PyObject *}{PyList_GET_ITEM}{PyObject *list, int i}

\end{cfuncdesc}

\begin{cfuncdesc}{int}{PyList_GET_SIZE}{PyObject *list}

\end{cfuncdesc}


\section{Mapping Objects}

\subsection{Dictionary Objects}

\begin{ctypedesc}{PyDictObject}
This subtype of \code{PyObject} represents a Python dictionary object.
\end{ctypedesc}

\begin{cvardesc}{PyTypeObject}{PyDict_Type}
This instance of \code{PyTypeObject} represents the Python dictionary type.
\end{cvardesc}

\begin{cfuncdesc}{int}{PyDict_Check}{PyObject *p}
returns true if it's argument is a PyDictObject
\end{cfuncdesc}

\begin{cfuncdesc}{PyDictObject *}{PyDict_New}{}
returns a new empty dictionary.
\end{cfuncdesc}

\begin{cfuncdesc}{void}{PyDict_Clear}{PyDictObject *p}
empties an existing dictionary and deletes it.
\end{cfuncdesc}

\begin{cfuncdesc}{int}{PyDict_SetItem}{PyDictObject *p,
            PyObject *key,
            PyObject *val}
inserts \code{value} into the dictionary with a key of
\code{key}. Both \code{key} and \code{value} should be PyObjects, and \code{key} should
be hashable.
\end{cfuncdesc}

\begin{cfuncdesc}{int}{PyDict_SetItemString}{PyDictObject *p,
            char *key,
            PyObject *val}
inserts \code{value} into the dictionary using \code{key}
as a key. \code{key} should be a char *
\end{cfuncdesc}

\begin{cfuncdesc}{int}{PyDict_DelItem}{PyDictObject *p, PyObject *key}
removes the entry in dictionary \code{p} with key \code{key}.
\code{key} is a PyObject.
\end{cfuncdesc}

\begin{cfuncdesc}{int}{PyDict_DelItemString}{PyDictObject *p, char *key}
removes the entry in dictionary \code{p} which has a key
specified by the \code{char *}\code{key}.
\end{cfuncdesc}

\begin{cfuncdesc}{PyObject *}{PyDict_GetItem}{PyDictObject *p, PyObject *key}
returns the object from dictionary \code{p} which has a key
\code{key}.
\end{cfuncdesc}

\begin{cfuncdesc}{PyObject *}{PyDict_GetItemString}{PyDictObject *p, char *key}
does the same, but \code{key} is specified as a
\code{char *}, rather than a \code{PyObject *}.
\end{cfuncdesc}

\begin{cfuncdesc}{PyListObject *}{PyDict_Items}{PyDictObject *p}
returns a PyListObject containing all the items 
from the dictionary, as in the mapping method \code{items()} (see the Reference
Guide)
\end{cfuncdesc}

\begin{cfuncdesc}{PyListObject *}{PyDict_Keys}{PyDictObject *p}
returns a PyListObject containing all the keys 
from the dictionary, as in the mapping method \code{keys()} (see the Reference Guide)
\end{cfuncdesc}

\begin{cfuncdesc}{PyListObject *}{PyDict_Values}{PyDictObject *p}
returns a PyListObject containing all the values 
from the dictionary, as in the mapping method \code{values()} (see the Reference Guide)
\end{cfuncdesc}

\begin{cfuncdesc}{int}{PyDict_Size}{PyDictObject *p}
returns the number of items in the dictionary.
\end{cfuncdesc}

\begin{cfuncdesc}{int}{PyDict_Next}{PyDictObject *p,
            int ppos,
            PyObject **pkey,
            PyObject **pvalue}

\end{cfuncdesc}


\section{Numeric Objects}

\subsection{Plain Integer Objects}

\begin{ctypedesc}{PyIntObject}
This subtype of \code{PyObject} represents a Python integer object.
\end{ctypedesc}

\begin{cvardesc}{PyTypeObject}{PyInt_Type}
This instance of \code{PyTypeObject} represents the Python plain 
integer type.
\end{cvardesc}

\begin{cfuncdesc}{int}{PyInt_Check}{PyObject *}

\end{cfuncdesc}

\begin{cfuncdesc}{PyIntObject *}{PyInt_FromLong}{long ival}
creates a new integer object with a value of \code{ival}.

The current implementation keeps an array of integer objects for all
integers between -1 and 100, when you create an int in that range you
actually just get back a reference to the existing object. So it should
be possible to change the value of 1. I suspect the behaviour of python
in this case is undefined. :-)
\end{cfuncdesc}

\begin{cfuncdesc}{long}{PyInt_AS_LONG}{PyIntObject *io}
returns the value of the object \code{io}.
\end{cfuncdesc}

\begin{cfuncdesc}{long}{PyInt_AsLong}{PyObject *io}
will first attempt to cast the object to a PyIntObject, if
it is not already one, and the return it's value.
\end{cfuncdesc}

\begin{cfuncdesc}{long}{PyInt_GetMax}{}
returns the systems idea of the largest int it can handle
(LONG_MAX, as defined in the system header files)
\end{cfuncdesc}


\subsection{Long Integer Objects}

\begin{ctypedesc}{PyLongObject}
This subtype of \code{PyObject} represents a Python long integer object.
\end{ctypedesc}

\begin{cvardesc}{PyTypeObject}{PyLong_Type}
This instance of \code{PyTypeObject} represents the Python long integer type.
\end{cvardesc}

\begin{cfuncdesc}{int}{PyLong_Check}{PyObject *p}
returns true if it's argument is a \code{PyLongObject}
\end{cfuncdesc}

\begin{cfuncdesc}{PyObject *}{PyLong_FromLong}{long}

\end{cfuncdesc}

\begin{cfuncdesc}{PyObject *}{PyLong_FromUnsignedLong}{unsigned long}

\end{cfuncdesc}

\begin{cfuncdesc}{PyObject *}{PyLong_FromDouble}{double}

\end{cfuncdesc}

\begin{cfuncdesc}{long}{PyLong_AsLong}{PyObject *}

\end{cfuncdesc}

\begin{cfuncdesc}{unsigned long}{PyLong_AsUnsignedLong}{PyObject }

\end{cfuncdesc}

\begin{cfuncdesc}{double}{PyLong_AsDouble}{PyObject *}

\end{cfuncdesc}

\begin{cfuncdesc}{PyObject *}{*PyLong_FromString}{char *, char **, int}

\end{cfuncdesc}


\subsection{Floating Point Objects}

\begin{ctypedesc}{PyFloatObject}
This subtype of \code{PyObject} represents a Python floating point object.
\end{ctypedesc}

\begin{cvardesc}{PyTypeObject}{PyFloat_Type}
This instance of \code{PyTypeObject} represents the Python floating 
point type.
\end{cvardesc}

\begin{cfuncdesc}{int}{PyFloat_Check}{PyObject *p}
returns true if it's argument is a \code{PyFloatObject}
\end{cfuncdesc}

\begin{cfuncdesc}{PyObject *}{PyFloat_FromDouble}{double}

\end{cfuncdesc}

\begin{cfuncdesc}{double}{PyFloat_AsDouble}{PyObject *}

\end{cfuncdesc}

\begin{cfuncdesc}{double}{PyFloat_AS_DOUBLE}{PyFloatObject *}

\end{cfuncdesc}


\subsection{Complex Number Objects}

\begin{ctypedesc}{Py_complex}
typedef struct {
   double real;
   double imag;
} 
\end{ctypedesc}

\begin{ctypedesc}{PyComplexObject}
This subtype of \code{PyObject} represents a Python complex number object.
\end{ctypedesc}

\begin{cvardesc}{PyTypeObject}{PyComplex_Type}
This instance of \code{PyTypeObject} represents the Python complex 
number type.
\end{cvardesc}

\begin{cfuncdesc}{int}{PyComplex_Check}{PyObject *p}
returns true if it's argument is a \code{PyComplexObject}
\end{cfuncdesc}

\begin{cfuncdesc}{Py_complex}{_Py_c_sum}{Py_complex, Py_complex}

\end{cfuncdesc}

\begin{cfuncdesc}{Py_complex}{_Py_c_diff}{Py_complex, Py_complex}

\end{cfuncdesc}

\begin{cfuncdesc}{Py_complex}{_Py_c_neg}{Py_complex}

\end{cfuncdesc}

\begin{cfuncdesc}{Py_complex}{_Py_c_prod}{Py_complex, Py_complex}

\end{cfuncdesc}

\begin{cfuncdesc}{Py_complex}{_Py_c_quot}{Py_complex, Py_complex}

\end{cfuncdesc}

\begin{cfuncdesc}{Py_complex}{_Py_c_pow}{Py_complex, Py_complex}

\end{cfuncdesc}

\begin{cfuncdesc}{PyObject *}{PyComplex_FromCComplex}{Py_complex}

\end{cfuncdesc}

\begin{cfuncdesc}{PyObject *}{PyComplex_FromDoubles}{double real, double imag}

\end{cfuncdesc}

\begin{cfuncdesc}{double}{PyComplex_RealAsDouble}{PyObject *op}

\end{cfuncdesc}

\begin{cfuncdesc}{double}{PyComplex_ImagAsDouble}{PyObject *op}

\end{cfuncdesc}

\begin{cfuncdesc}{Py_complex}{PyComplex_AsCComplex}{PyObject *op}

\end{cfuncdesc}



\section{Other Objects}

\subsection{File Objects}

\begin{ctypedesc}{PyFileObject}
This subtype of \code{PyObject} represents a Python file object.
\end{ctypedesc}

\begin{cvardesc}{PyTypeObject}{PyFile_Type}
This instance of \code{PyTypeObject} represents the Python file type.
\end{cvardesc}

\begin{cfuncdesc}{int}{PyFile_Check}{PyObject *p}
returns true if it's argument is a \code{PyFileObject}
\end{cfuncdesc}

\begin{cfuncdesc}{PyObject *}{PyFile_FromString}{char *name, char *mode}
creates a new PyFileObject pointing to the file
specified in \code{name} with the mode specified in \code{mode}
\end{cfuncdesc}

\begin{cfuncdesc}{PyObject *}{PyFile_FromFile}{FILE *fp,
              char *name, char *mode, int (*close})
creates a new PyFileObject from the already-open \code{fp}.
The function \code{close} will be called when the file should be closed.
\end{cfuncdesc}

\begin{cfuncdesc}{FILE *}{PyFile_AsFile}{PyFileObject *p}
returns the file object associated with \code{p} as a \code{FILE *}
\end{cfuncdesc}

\begin{cfuncdesc}{PyStringObject *}{PyFile_GetLine}{PyObject *p, int n}
undocumented as yet
\end{cfuncdesc}

\begin{cfuncdesc}{PyStringObject *}{PyFile_Name}{PyObject *p}
returns the name of the file specified by \code{p} as a 
PyStringObject
\end{cfuncdesc}

\begin{cfuncdesc}{void}{PyFile_SetBufSize}{PyFileObject *p, int n}
on systems with \code{setvbuf} only
\end{cfuncdesc}

\begin{cfuncdesc}{int}{PyFile_SoftSpace}{PyFileObject *p, int newflag}
same as the file object method \code{softspace}
\end{cfuncdesc}

\begin{cfuncdesc}{int}{PyFile_WriteObject}{PyObject *obj, PyFileObject *p}
writes object \code{obj} to file object \code{p}
\end{cfuncdesc}

\begin{cfuncdesc}{int}{PyFile_WriteString}{char *s, PyFileObject *p}
writes string \code{s} to file object \code{p}
\end{cfuncdesc}


\documentclass{manual}

\title{Python/C API Reference Manual}

\author{Guido van Rossum\\
	Fred L. Drake, Jr., editor}
\authoraddress{
	BeOpen PythonLabs\\
	E-mail: \email{python-docs@python.org}
}

\date{September 5, 2000}			% XXX update before release!
\release{2.0b1}


\makeindex			% tell \index to actually write the .idx file


\begin{document}

\maketitle

\ifhtml
\chapter*{Front Matter\label{front}}
\fi

\begin{small}
Copyright \copyright{} 2001 Python Software Foundation.
All rights reserved.

Copyright \copyright{} 2000 BeOpen.com.
All rights reserved.

Copyright \copyright{} 1995-2000 Corporation for National Research Initiatives.
All rights reserved.

Copyright \copyright{} 1991-1995 Stichting Mathematisch Centrum.
All rights reserved.

%%begin{latexonly}
\vskip 4mm
%%end{latexonly}

\centerline{\strong{BEOPEN.COM TERMS AND CONDITIONS FOR PYTHON 2.0}}

\centerline{\strong{BEOPEN PYTHON OPEN SOURCE LICENSE AGREEMENT VERSION 1}}

\begin{enumerate}

\item
This LICENSE AGREEMENT is between BeOpen.com (``BeOpen''), having an
office at 160 Saratoga Avenue, Santa Clara, CA 95051, and the
Individual or Organization (``Licensee'') accessing and otherwise
using this software in source or binary form and its associated
documentation (``the Software'').

\item
Subject to the terms and conditions of this BeOpen Python License
Agreement, BeOpen hereby grants Licensee a non-exclusive,
royalty-free, world-wide license to reproduce, analyze, test, perform
and/or display publicly, prepare derivative works, distribute, and
otherwise use the Software alone or in any derivative version,
provided, however, that the BeOpen Python License is retained in the
Software, alone or in any derivative version prepared by Licensee.

\item
BeOpen is making the Software available to Licensee on an ``AS IS''
basis.  BEOPEN MAKES NO REPRESENTATIONS OR WARRANTIES, EXPRESS OR
IMPLIED.  BY WAY OF EXAMPLE, BUT NOT LIMITATION, BEOPEN MAKES NO AND
DISCLAIMS ANY REPRESENTATION OR WARRANTY OF MERCHANTABILITY OR FITNESS
FOR ANY PARTICULAR PURPOSE OR THAT THE USE OF THE SOFTWARE WILL NOT
INFRINGE ANY THIRD PARTY RIGHTS.

\item
BEOPEN SHALL NOT BE LIABLE TO LICENSEE OR ANY OTHER USERS OF THE
SOFTWARE FOR ANY INCIDENTAL, SPECIAL, OR CONSEQUENTIAL DAMAGES OR LOSS
AS A RESULT OF USING, MODIFYING OR DISTRIBUTING THE SOFTWARE, OR ANY
DERIVATIVE THEREOF, EVEN IF ADVISED OF THE POSSIBILITY THEREOF.

\item
This License Agreement will automatically terminate upon a material
breach of its terms and conditions.

\item
This License Agreement shall be governed by and interpreted in all
respects by the law of the State of California, excluding conflict of
law provisions.  Nothing in this License Agreement shall be deemed to
create any relationship of agency, partnership, or joint venture
between BeOpen and Licensee.  This License Agreement does not grant
permission to use BeOpen trademarks or trade names in a trademark
sense to endorse or promote products or services of Licensee, or any
third party.  As an exception, the ``BeOpen Python'' logos available
at http://www.pythonlabs.com/logos.html may be used according to the
permissions granted on that web page.

\item
By copying, installing or otherwise using the software, Licensee
agrees to be bound by the terms and conditions of this License
Agreement.
\end{enumerate}


\centerline{\strong{CNRI OPEN SOURCE GPL-COMPATIBLE LICENSE AGREEMENT}}

Python 1.6.1 is made available subject to the terms and conditions in
CNRI's License Agreement.  This Agreement together with Python 1.6.1 may
be located on the Internet using the following unique, persistent
identifier (known as a handle): 1895.22/1013.  This Agreement may also
be obtained from a proxy server on the Internet using the following
URL: \url{http://hdl.handle.net/1895.22/1013}.


\centerline{\strong{CWI PERMISSIONS STATEMENT AND DISCLAIMER}}

Copyright \copyright{} 1991 - 1995, Stichting Mathematisch Centrum
Amsterdam, The Netherlands.  All rights reserved.

Permission to use, copy, modify, and distribute this software and its
documentation for any purpose and without fee is hereby granted,
provided that the above copyright notice appear in all copies and that
both that copyright notice and this permission notice appear in
supporting documentation, and that the name of Stichting Mathematisch
Centrum or CWI not be used in advertising or publicity pertaining to
distribution of the software without specific, written prior
permission.

STICHTING MATHEMATISCH CENTRUM DISCLAIMS ALL WARRANTIES WITH REGARD TO
THIS SOFTWARE, INCLUDING ALL IMPLIED WARRANTIES OF MERCHANTABILITY AND
FITNESS, IN NO EVENT SHALL STICHTING MATHEMATISCH CENTRUM BE LIABLE
FOR ANY SPECIAL, INDIRECT OR CONSEQUENTIAL DAMAGES OR ANY DAMAGES
WHATSOEVER RESULTING FROM LOSS OF USE, DATA OR PROFITS, WHETHER IN AN
ACTION OF CONTRACT, NEGLIGENCE OR OTHER TORTIOUS ACTION, ARISING OUT
OF OR IN CONNECTION WITH THE USE OR PERFORMANCE OF THIS SOFTWARE.
\end{small}


\begin{abstract}

\noindent
This manual documents the API used by C and \Cpp{} programmers who
want to write extension modules or embed Python.  It is a companion to
\citetitle[../ext/ext.html]{Extending and Embedding the Python
Interpreter}, which describes the general principles of extension
writing but does not document the API functions in detail.

\warning{The current version of this document is incomplete.  I hope
that it is nevertheless useful.  I will continue to work on it, and
release new versions from time to time, independent from Python source
code releases.}

\end{abstract}

\tableofcontents


\chapter{Introduction \label{intro}}


The Application Programmer's Interface to Python gives C and
\Cpp{} programmers access to the Python interpreter at a variety of
levels.  The API is equally usable from \Cpp, but for brevity it is
generally referred to as the Python/C API.  There are two
fundamentally different reasons for using the Python/C API.  The first
reason is to write \emph{extension modules} for specific purposes;
these are C modules that extend the Python interpreter.  This is
probably the most common use.  The second reason is to use Python as a
component in a larger application; this technique is generally
referred to as \dfn{embedding} Python in an application.

Writing an extension module is a relatively well-understood process, 
where a ``cookbook'' approach works well.  There are several tools 
that automate the process to some extent.  While people have embedded 
Python in other applications since its early existence, the process of 
embedding Python is less straightforward than writing an extension.  

Many API functions are useful independent of whether you're embedding 
or extending Python; moreover, most applications that embed Python 
will need to provide a custom extension as well, so it's probably a 
good idea to become familiar with writing an extension before 
attempting to embed Python in a real application.


\section{Include Files \label{includes}}

All function, type and macro definitions needed to use the Python/C
API are included in your code by the following line:

\begin{verbatim}
#include "Python.h"
\end{verbatim}

This implies inclusion of the following standard headers:
\code{<stdio.h>}, \code{<string.h>}, \code{<errno.h>},
\code{<limits.h>}, and \code{<stdlib.h>} (if available).

\begin{notice}[warning]
  Since Python may define some pre-processor definitions which affect
  the standard headers on some systems, you \emph{must} include
  \file{Python.h} before any standard headers are included.
\end{notice}

All user visible names defined by Python.h (except those defined by
the included standard headers) have one of the prefixes \samp{Py} or
\samp{_Py}.  Names beginning with \samp{_Py} are for internal use by
the Python implementation and should not be used by extension writers.
Structure member names do not have a reserved prefix.

\strong{Important:} user code should never define names that begin
with \samp{Py} or \samp{_Py}.  This confuses the reader, and
jeopardizes the portability of the user code to future Python
versions, which may define additional names beginning with one of
these prefixes.

The header files are typically installed with Python.  On \UNIX, these 
are located in the directories
\file{\envvar{prefix}/include/python\var{version}/} and
\file{\envvar{exec_prefix}/include/python\var{version}/}, where
\envvar{prefix} and \envvar{exec_prefix} are defined by the
corresponding parameters to Python's \program{configure} script and
\var{version} is \code{sys.version[:3]}.  On Windows, the headers are
installed in \file{\envvar{prefix}/include}, where \envvar{prefix} is
the installation directory specified to the installer.

To include the headers, place both directories (if different) on your
compiler's search path for includes.  Do \emph{not} place the parent
directories on the search path and then use
\samp{\#include <python\shortversion/Python.h>}; this will break on
multi-platform builds since the platform independent headers under
\envvar{prefix} include the platform specific headers from
\envvar{exec_prefix}.

\Cpp{} users should note that though the API is defined entirely using
C, the header files do properly declare the entry points to be
\code{extern "C"}, so there is no need to do anything special to use
the API from \Cpp.


\section{Objects, Types and Reference Counts \label{objects}}

Most Python/C API functions have one or more arguments as well as a
return value of type \ctype{PyObject*}.  This type is a pointer
to an opaque data type representing an arbitrary Python
object.  Since all Python object types are treated the same way by the
Python language in most situations (e.g., assignments, scope rules,
and argument passing), it is only fitting that they should be
represented by a single C type.  Almost all Python objects live on the
heap: you never declare an automatic or static variable of type
\ctype{PyObject}, only pointer variables of type \ctype{PyObject*} can 
be declared.  The sole exception are the type objects\obindex{type};
since these must never be deallocated, they are typically static
\ctype{PyTypeObject} objects.

All Python objects (even Python integers) have a \dfn{type} and a
\dfn{reference count}.  An object's type determines what kind of object 
it is (e.g., an integer, a list, or a user-defined function; there are 
many more as explained in the \citetitle[../ref/ref.html]{Python
Reference Manual}).  For each of the well-known types there is a macro
to check whether an object is of that type; for instance,
\samp{PyList_Check(\var{a})} is true if (and only if) the object
pointed to by \var{a} is a Python list.


\subsection{Reference Counts \label{refcounts}}

The reference count is important because today's computers have a 
finite (and often severely limited) memory size; it counts how many 
different places there are that have a reference to an object.  Such a 
place could be another object, or a global (or static) C variable, or 
a local variable in some C function.  When an object's reference count 
becomes zero, the object is deallocated.  If it contains references to 
other objects, their reference count is decremented.  Those other 
objects may be deallocated in turn, if this decrement makes their 
reference count become zero, and so on.  (There's an obvious problem 
with objects that reference each other here; for now, the solution is 
``don't do that.'')

Reference counts are always manipulated explicitly.  The normal way is 
to use the macro \cfunction{Py_INCREF()}\ttindex{Py_INCREF()} to
increment an object's reference count by one, and
\cfunction{Py_DECREF()}\ttindex{Py_DECREF()} to decrement it by  
one.  The \cfunction{Py_DECREF()} macro is considerably more complex
than the incref one, since it must check whether the reference count
becomes zero and then cause the object's deallocator to be called.
The deallocator is a function pointer contained in the object's type
structure.  The type-specific deallocator takes care of decrementing
the reference counts for other objects contained in the object if this
is a compound object type, such as a list, as well as performing any
additional finalization that's needed.  There's no chance that the
reference count can overflow; at least as many bits are used to hold
the reference count as there are distinct memory locations in virtual
memory (assuming \code{sizeof(long) >= sizeof(char*)}).  Thus, the
reference count increment is a simple operation.

It is not necessary to increment an object's reference count for every 
local variable that contains a pointer to an object.  In theory, the 
object's reference count goes up by one when the variable is made to 
point to it and it goes down by one when the variable goes out of 
scope.  However, these two cancel each other out, so at the end the 
reference count hasn't changed.  The only real reason to use the 
reference count is to prevent the object from being deallocated as 
long as our variable is pointing to it.  If we know that there is at 
least one other reference to the object that lives at least as long as 
our variable, there is no need to increment the reference count 
temporarily.  An important situation where this arises is in objects 
that are passed as arguments to C functions in an extension module 
that are called from Python; the call mechanism guarantees to hold a 
reference to every argument for the duration of the call.

However, a common pitfall is to extract an object from a list and
hold on to it for a while without incrementing its reference count.
Some other operation might conceivably remove the object from the
list, decrementing its reference count and possible deallocating it.
The real danger is that innocent-looking operations may invoke
arbitrary Python code which could do this; there is a code path which
allows control to flow back to the user from a \cfunction{Py_DECREF()},
so almost any operation is potentially dangerous.

A safe approach is to always use the generic operations (functions 
whose name begins with \samp{PyObject_}, \samp{PyNumber_},
\samp{PySequence_} or \samp{PyMapping_}).  These operations always
increment the reference count of the object they return.  This leaves
the caller with the responsibility to call
\cfunction{Py_DECREF()} when they are done with the result; this soon
becomes second nature.


\subsubsection{Reference Count Details \label{refcountDetails}}

The reference count behavior of functions in the Python/C API is best 
explained in terms of \emph{ownership of references}.  Ownership
pertains to references, never to objects (objects are not owned: they
are always shared).  "Owning a reference" means being responsible for
calling Py_DECREF on it when the reference is no longer needed. 
Ownership can also be transferred, meaning that the code that receives
ownership of the reference then becomes responsible for eventually
decref'ing it by calling \cfunction{Py_DECREF()} or
\cfunction{Py_XDECREF()} when it's no longer needed --or passing on
this responsibility (usually to its caller).
When a function passes ownership of a reference on to its caller, the
caller is said to receive a \emph{new} reference.  When no ownership
is transferred, the caller is said to \emph{borrow} the reference.
Nothing needs to be done for a borrowed reference.

Conversely, when a calling function passes it a reference to an 
object, there are two possibilities: the function \emph{steals} a 
reference to the object, or it does not.  Few functions steal 
references; the two notable exceptions are
\cfunction{PyList_SetItem()}\ttindex{PyList_SetItem()} and
\cfunction{PyTuple_SetItem()}\ttindex{PyTuple_SetItem()}, which 
steal a reference to the item (but not to the tuple or list into which
the item is put!).  These functions were designed to steal a reference
because of a common idiom for populating a tuple or list with newly
created objects; for example, the code to create the tuple \code{(1,
2, "three")} could look like this (forgetting about error handling for
the moment; a better way to code this is shown below):

\begin{verbatim}
PyObject *t;

t = PyTuple_New(3);
PyTuple_SetItem(t, 0, PyInt_FromLong(1L));
PyTuple_SetItem(t, 1, PyInt_FromLong(2L));
PyTuple_SetItem(t, 2, PyString_FromString("three"));
\end{verbatim}

Incidentally, \cfunction{PyTuple_SetItem()} is the \emph{only} way to
set tuple items; \cfunction{PySequence_SetItem()} and
\cfunction{PyObject_SetItem()} refuse to do this since tuples are an
immutable data type.  You should only use
\cfunction{PyTuple_SetItem()} for tuples that you are creating
yourself.

Equivalent code for populating a list can be written using 
\cfunction{PyList_New()} and \cfunction{PyList_SetItem()}.  Such code
can also use \cfunction{PySequence_SetItem()}; this illustrates the
difference between the two (the extra \cfunction{Py_DECREF()} calls):

\begin{verbatim}
PyObject *l, *x;

l = PyList_New(3);
x = PyInt_FromLong(1L);
PySequence_SetItem(l, 0, x); Py_DECREF(x);
x = PyInt_FromLong(2L);
PySequence_SetItem(l, 1, x); Py_DECREF(x);
x = PyString_FromString("three");
PySequence_SetItem(l, 2, x); Py_DECREF(x);
\end{verbatim}

You might find it strange that the ``recommended'' approach takes more
code.  However, in practice, you will rarely use these ways of
creating and populating a tuple or list.  There's a generic function,
\cfunction{Py_BuildValue()}, that can create most common objects from
C values, directed by a \dfn{format string}.  For example, the
above two blocks of code could be replaced by the following (which
also takes care of the error checking):

\begin{verbatim}
PyObject *t, *l;

t = Py_BuildValue("(iis)", 1, 2, "three");
l = Py_BuildValue("[iis]", 1, 2, "three");
\end{verbatim}

It is much more common to use \cfunction{PyObject_SetItem()} and
friends with items whose references you are only borrowing, like
arguments that were passed in to the function you are writing.  In
that case, their behaviour regarding reference counts is much saner,
since you don't have to increment a reference count so you can give a
reference away (``have it be stolen'').  For example, this function
sets all items of a list (actually, any mutable sequence) to a given
item:

\begin{verbatim}
int
set_all(PyObject *target, PyObject *item)
{
    int i, n;

    n = PyObject_Length(target);
    if (n < 0)
        return -1;
    for (i = 0; i < n; i++) {
        if (PyObject_SetItem(target, i, item) < 0)
            return -1;
    }
    return 0;
}
\end{verbatim}
\ttindex{set_all()}

The situation is slightly different for function return values.  
While passing a reference to most functions does not change your 
ownership responsibilities for that reference, many functions that 
return a reference to an object give you ownership of the reference.
The reason is simple: in many cases, the returned object is created 
on the fly, and the reference you get is the only reference to the 
object.  Therefore, the generic functions that return object 
references, like \cfunction{PyObject_GetItem()} and 
\cfunction{PySequence_GetItem()}, always return a new reference (the
caller becomes the owner of the reference).

It is important to realize that whether you own a reference returned 
by a function depends on which function you call only --- \emph{the
plumage} (the type of the object passed as an
argument to the function) \emph{doesn't enter into it!}  Thus, if you 
extract an item from a list using \cfunction{PyList_GetItem()}, you
don't own the reference --- but if you obtain the same item from the
same list using \cfunction{PySequence_GetItem()} (which happens to
take exactly the same arguments), you do own a reference to the
returned object.

Here is an example of how you could write a function that computes the
sum of the items in a list of integers; once using 
\cfunction{PyList_GetItem()}\ttindex{PyList_GetItem()}, and once using
\cfunction{PySequence_GetItem()}\ttindex{PySequence_GetItem()}.

\begin{verbatim}
long
sum_list(PyObject *list)
{
    int i, n;
    long total = 0;
    PyObject *item;

    n = PyList_Size(list);
    if (n < 0)
        return -1; /* Not a list */
    for (i = 0; i < n; i++) {
        item = PyList_GetItem(list, i); /* Can't fail */
        if (!PyInt_Check(item)) continue; /* Skip non-integers */
        total += PyInt_AsLong(item);
    }
    return total;
}
\end{verbatim}
\ttindex{sum_list()}

\begin{verbatim}
long
sum_sequence(PyObject *sequence)
{
    int i, n;
    long total = 0;
    PyObject *item;
    n = PySequence_Length(sequence);
    if (n < 0)
        return -1; /* Has no length */
    for (i = 0; i < n; i++) {
        item = PySequence_GetItem(sequence, i);
        if (item == NULL)
            return -1; /* Not a sequence, or other failure */
        if (PyInt_Check(item))
            total += PyInt_AsLong(item);
        Py_DECREF(item); /* Discard reference ownership */
    }
    return total;
}
\end{verbatim}
\ttindex{sum_sequence()}


\subsection{Types \label{types}}

There are few other data types that play a significant role in 
the Python/C API; most are simple C types such as \ctype{int}, 
\ctype{long}, \ctype{double} and \ctype{char*}.  A few structure types 
are used to describe static tables used to list the functions exported 
by a module or the data attributes of a new object type, and another
is used to describe the value of a complex number.  These will 
be discussed together with the functions that use them.


\section{Exceptions \label{exceptions}}

The Python programmer only needs to deal with exceptions if specific 
error handling is required; unhandled exceptions are automatically 
propagated to the caller, then to the caller's caller, and so on, until
they reach the top-level interpreter, where they are reported to the 
user accompanied by a stack traceback.

For C programmers, however, error checking always has to be explicit.  
All functions in the Python/C API can raise exceptions, unless an 
explicit claim is made otherwise in a function's documentation.  In 
general, when a function encounters an error, it sets an exception, 
discards any object references that it owns, and returns an 
error indicator --- usually \NULL{} or \code{-1}.  A few functions 
return a Boolean true/false result, with false indicating an error.
Very few functions return no explicit error indicator or have an 
ambiguous return value, and require explicit testing for errors with 
\cfunction{PyErr_Occurred()}\ttindex{PyErr_Occurred()}.

Exception state is maintained in per-thread storage (this is 
equivalent to using global storage in an unthreaded application).  A 
thread can be in one of two states: an exception has occurred, or not.
The function \cfunction{PyErr_Occurred()} can be used to check for
this: it returns a borrowed reference to the exception type object
when an exception has occurred, and \NULL{} otherwise.  There are a
number of functions to set the exception state:
\cfunction{PyErr_SetString()}\ttindex{PyErr_SetString()} is the most
common (though not the most general) function to set the exception
state, and \cfunction{PyErr_Clear()}\ttindex{PyErr_Clear()} clears the
exception state.

The full exception state consists of three objects (all of which can 
be \NULL): the exception type, the corresponding exception 
value, and the traceback.  These have the same meanings as the Python
\withsubitem{(in module sys)}{
  \ttindex{exc_type}\ttindex{exc_value}\ttindex{exc_traceback}}
objects \code{sys.exc_type}, \code{sys.exc_value}, and
\code{sys.exc_traceback}; however, they are not the same: the Python
objects represent the last exception being handled by a Python 
\keyword{try} \ldots\ \keyword{except} statement, while the C level
exception state only exists while an exception is being passed on
between C functions until it reaches the Python bytecode interpreter's 
main loop, which takes care of transferring it to \code{sys.exc_type}
and friends.

Note that starting with Python 1.5, the preferred, thread-safe way to 
access the exception state from Python code is to call the function
\withsubitem{(in module sys)}{\ttindex{exc_info()}}
\function{sys.exc_info()}, which returns the per-thread exception state 
for Python code.  Also, the semantics of both ways to access the 
exception state have changed so that a function which catches an 
exception will save and restore its thread's exception state so as to 
preserve the exception state of its caller.  This prevents common bugs 
in exception handling code caused by an innocent-looking function 
overwriting the exception being handled; it also reduces the often 
unwanted lifetime extension for objects that are referenced by the 
stack frames in the traceback.

As a general principle, a function that calls another function to 
perform some task should check whether the called function raised an 
exception, and if so, pass the exception state on to its caller.  It 
should discard any object references that it owns, and return an 
error indicator, but it should \emph{not} set another exception ---
that would overwrite the exception that was just raised, and lose
important information about the exact cause of the error.

A simple example of detecting exceptions and passing them on is shown
in the \cfunction{sum_sequence()}\ttindex{sum_sequence()} example
above.  It so happens that that example doesn't need to clean up any
owned references when it detects an error.  The following example
function shows some error cleanup.  First, to remind you why you like
Python, we show the equivalent Python code:

\begin{verbatim}
def incr_item(dict, key):
    try:
        item = dict[key]
    except KeyError:
        item = 0
    dict[key] = item + 1
\end{verbatim}
\ttindex{incr_item()}

Here is the corresponding C code, in all its glory:

\begin{verbatim}
int
incr_item(PyObject *dict, PyObject *key)
{
    /* Objects all initialized to NULL for Py_XDECREF */
    PyObject *item = NULL, *const_one = NULL, *incremented_item = NULL;
    int rv = -1; /* Return value initialized to -1 (failure) */

    item = PyObject_GetItem(dict, key);
    if (item == NULL) {
        /* Handle KeyError only: */
        if (!PyErr_ExceptionMatches(PyExc_KeyError))
            goto error;

        /* Clear the error and use zero: */
        PyErr_Clear();
        item = PyInt_FromLong(0L);
        if (item == NULL)
            goto error;
    }
    const_one = PyInt_FromLong(1L);
    if (const_one == NULL)
        goto error;

    incremented_item = PyNumber_Add(item, const_one);
    if (incremented_item == NULL)
        goto error;

    if (PyObject_SetItem(dict, key, incremented_item) < 0)
        goto error;
    rv = 0; /* Success */
    /* Continue with cleanup code */

 error:
    /* Cleanup code, shared by success and failure path */

    /* Use Py_XDECREF() to ignore NULL references */
    Py_XDECREF(item);
    Py_XDECREF(const_one);
    Py_XDECREF(incremented_item);

    return rv; /* -1 for error, 0 for success */
}
\end{verbatim}
\ttindex{incr_item()}

This example represents an endorsed use of the \keyword{goto} statement 
in C!  It illustrates the use of
\cfunction{PyErr_ExceptionMatches()}\ttindex{PyErr_ExceptionMatches()} and
\cfunction{PyErr_Clear()}\ttindex{PyErr_Clear()} to
handle specific exceptions, and the use of
\cfunction{Py_XDECREF()}\ttindex{Py_XDECREF()} to
dispose of owned references that may be \NULL{} (note the
\character{X} in the name; \cfunction{Py_DECREF()} would crash when
confronted with a \NULL{} reference).  It is important that the
variables used to hold owned references are initialized to \NULL{} for
this to work; likewise, the proposed return value is initialized to
\code{-1} (failure) and only set to success after the final call made
is successful.


\section{Embedding Python \label{embedding}}

The one important task that only embedders (as opposed to extension
writers) of the Python interpreter have to worry about is the
initialization, and possibly the finalization, of the Python
interpreter.  Most functionality of the interpreter can only be used
after the interpreter has been initialized.

The basic initialization function is
\cfunction{Py_Initialize()}\ttindex{Py_Initialize()}.
This initializes the table of loaded modules, and creates the
fundamental modules \module{__builtin__}\refbimodindex{__builtin__},
\module{__main__}\refbimodindex{__main__}, \module{sys}\refbimodindex{sys},
and \module{exceptions}.\refbimodindex{exceptions}  It also initializes
the module search path (\code{sys.path}).%
\indexiii{module}{search}{path}
\withsubitem{(in module sys)}{\ttindex{path}}

\cfunction{Py_Initialize()} does not set the ``script argument list'' 
(\code{sys.argv}).  If this variable is needed by Python code that 
will be executed later, it must be set explicitly with a call to 
\code{PySys_SetArgv(\var{argc},
\var{argv})}\ttindex{PySys_SetArgv()} subsequent to the call to
\cfunction{Py_Initialize()}.

On most systems (in particular, on \UNIX{} and Windows, although the
details are slightly different),
\cfunction{Py_Initialize()} calculates the module search path based
upon its best guess for the location of the standard Python
interpreter executable, assuming that the Python library is found in a
fixed location relative to the Python interpreter executable.  In
particular, it looks for a directory named
\file{lib/python\shortversion} relative to the parent directory where
the executable named \file{python} is found on the shell command
search path (the environment variable \envvar{PATH}).

For instance, if the Python executable is found in
\file{/usr/local/bin/python}, it will assume that the libraries are in
\file{/usr/local/lib/python\shortversion}.  (In fact, this particular path
is also the ``fallback'' location, used when no executable file named
\file{python} is found along \envvar{PATH}.)  The user can override
this behavior by setting the environment variable \envvar{PYTHONHOME},
or insert additional directories in front of the standard path by
setting \envvar{PYTHONPATH}.

The embedding application can steer the search by calling 
\code{Py_SetProgramName(\var{file})}\ttindex{Py_SetProgramName()} \emph{before} calling 
\cfunction{Py_Initialize()}.  Note that \envvar{PYTHONHOME} still
overrides this and \envvar{PYTHONPATH} is still inserted in front of
the standard path.  An application that requires total control has to
provide its own implementation of
\cfunction{Py_GetPath()}\ttindex{Py_GetPath()},
\cfunction{Py_GetPrefix()}\ttindex{Py_GetPrefix()},
\cfunction{Py_GetExecPrefix()}\ttindex{Py_GetExecPrefix()}, and
\cfunction{Py_GetProgramFullPath()}\ttindex{Py_GetProgramFullPath()} (all
defined in \file{Modules/getpath.c}).

Sometimes, it is desirable to ``uninitialize'' Python.  For instance, 
the application may want to start over (make another call to 
\cfunction{Py_Initialize()}) or the application is simply done with its 
use of Python and wants to free all memory allocated by Python.  This
can be accomplished by calling \cfunction{Py_Finalize()}.  The function
\cfunction{Py_IsInitialized()}\ttindex{Py_IsInitialized()} returns
true if Python is currently in the initialized state.  More
information about these functions is given in a later chapter.

\chapter{The Very High Level Layer \label{veryhigh}}


The functions in this chapter will let you execute Python source code
given in a file or a buffer, but they will not let you interact in a
more detailed way with the interpreter.

Several of these functions accept a start symbol from the grammar as a 
parameter.  The available start symbols are \constant{Py_eval_input},
\constant{Py_file_input}, and \constant{Py_single_input}.  These are
described following the functions which accept them as parameters.

Note also that several of these functions take \ctype{FILE*}
parameters.  On particular issue which needs to be handled carefully
is that the \ctype{FILE} structure for different C libraries can be
different and incompatible.  Under Windows (at least), it is possible
for dynamically linked extensions to actually use different libraries,
so care should be taken that \ctype{FILE*} parameters are only passed
to these functions if it is certain that they were created by the same
library that the Python runtime is using.


\begin{cfuncdesc}{int}{Py_Main}{int argc, char **argv}
  The main program for the standard interpreter.  This is made
  available for programs which embed Python.  The \var{argc} and
  \var{argv} parameters should be prepared exactly as those which are
  passed to a C program's \cfunction{main()} function.  It is
  important to note that the argument list may be modified (but the
  contents of the strings pointed to by the argument list are not).
  The return value will be the integer passed to the
  \function{sys.exit()} function, \code{1} if the interpreter exits
  due to an exception, or \code{2} if the parameter list does not
  represent a valid Python command line.
\end{cfuncdesc}

\begin{cfuncdesc}{int}{PyRun_AnyFile}{FILE *fp, char *filename}
  If \var{fp} refers to a file associated with an interactive device
  (console or terminal input or \UNIX{} pseudo-terminal), return the
  value of \cfunction{PyRun_InteractiveLoop()}, otherwise return the
  result of \cfunction{PyRun_SimpleFile()}.  If \var{filename} is
  \NULL, this function uses \code{"???"} as the filename.
\end{cfuncdesc}

\begin{cfuncdesc}{int}{PyRun_SimpleString}{char *command}
  Executes the Python source code from \var{command} in the
  \module{__main__} module.  If \module{__main__} does not already
  exist, it is created.  Returns \code{0} on success or \code{-1} if
  an exception was raised.  If there was an error, there is no way to
  get the exception information.
\end{cfuncdesc}

\begin{cfuncdesc}{int}{PyRun_SimpleFile}{FILE *fp, char *filename}
  Similar to \cfunction{PyRun_SimpleString()}, but the Python source
  code is read from \var{fp} instead of an in-memory string.
  \var{filename} should be the name of the file.
\end{cfuncdesc}

\begin{cfuncdesc}{int}{PyRun_InteractiveOne}{FILE *fp, char *filename}
  Read and execute a single statement from a file associated with an
  interactive device.  If \var{filename} is \NULL, \code{"???"} is
  used instead.  The user will be prompted using \code{sys.ps1} and
  \code{sys.ps2}.  Returns \code{0} when the input was executed
  successfully, \code{-1} if there was an exception, or an error code
  from the \file{errcode.h} include file distributed as part of Python
  if there was a parse error.  (Note that \file{errcode.h} is not
  included by \file{Python.h}, so must be included specifically if
  needed.)
\end{cfuncdesc}

\begin{cfuncdesc}{int}{PyRun_InteractiveLoop}{FILE *fp, char *filename}
  Read and execute statements from a file associated with an
  interactive device until \EOF{} is reached.  If \var{filename} is
  \NULL, \code{"???"} is used instead.  The user will be prompted
  using \code{sys.ps1} and \code{sys.ps2}.  Returns \code{0} at \EOF.
\end{cfuncdesc}

\begin{cfuncdesc}{struct _node*}{PyParser_SimpleParseString}{char *str,
                                                             int start}
  Parse Python source code from \var{str} using the start token
  \var{start}.  The result can be used to create a code object which
  can be evaluated efficiently.  This is useful if a code fragment
  must be evaluated many times.
\end{cfuncdesc}

\begin{cfuncdesc}{struct _node*}{PyParser_SimpleParseFile}{FILE *fp,
                                 char *filename, int start}
  Similar to \cfunction{PyParser_SimpleParseString()}, but the Python
  source code is read from \var{fp} instead of an in-memory string.
  \var{filename} should be the name of the file.
\end{cfuncdesc}

\begin{cfuncdesc}{PyObject*}{PyRun_String}{char *str, int start,
                                           PyObject *globals,
                                           PyObject *locals}
  Execute Python source code from \var{str} in the context specified
  by the dictionaries \var{globals} and \var{locals}.  The parameter
  \var{start} specifies the start token that should be used to parse
  the source code.

  Returns the result of executing the code as a Python object, or
  \NULL{} if an exception was raised.
\end{cfuncdesc}

\begin{cfuncdesc}{PyObject*}{PyRun_File}{FILE *fp, char *filename,
                                         int start, PyObject *globals,
                                         PyObject *locals}
  Similar to \cfunction{PyRun_String()}, but the Python source code is
  read from \var{fp} instead of an in-memory string.
  \var{filename} should be the name of the file.
\end{cfuncdesc}

\begin{cfuncdesc}{PyObject*}{Py_CompileString}{char *str, char *filename,
                                               int start}
  Parse and compile the Python source code in \var{str}, returning the
  resulting code object.  The start token is given by \var{start};
  this can be used to constrain the code which can be compiled and should
  be \constant{Py_eval_input}, \constant{Py_file_input}, or
  \constant{Py_single_input}.  The filename specified by
  \var{filename} is used to construct the code object and may appear
  in tracebacks or \exception{SyntaxError} exception messages.  This
  returns \NULL{} if the code cannot be parsed or compiled.
\end{cfuncdesc}

\begin{cvardesc}{int}{Py_eval_input}
  The start symbol from the Python grammar for isolated expressions;
  for use with
  \cfunction{Py_CompileString()}\ttindex{Py_CompileString()}.
\end{cvardesc}

\begin{cvardesc}{int}{Py_file_input}
  The start symbol from the Python grammar for sequences of statements
  as read from a file or other source; for use with
  \cfunction{Py_CompileString()}\ttindex{Py_CompileString()}.  This is
  the symbol to use when compiling arbitrarily long Python source code.
\end{cvardesc}

\begin{cvardesc}{int}{Py_single_input}
  The start symbol from the Python grammar for a single statement; for
  use with \cfunction{Py_CompileString()}\ttindex{Py_CompileString()}.
  This is the symbol used for the interactive interpreter loop.
\end{cvardesc}

\chapter{Reference Counting \label{countingRefs}}


The macros in this section are used for managing reference counts
of Python objects.


\begin{cfuncdesc}{void}{Py_INCREF}{PyObject *o}
  Increment the reference count for object \var{o}.  The object must
  not be \NULL; if you aren't sure that it isn't \NULL, use
  \cfunction{Py_XINCREF()}.
\end{cfuncdesc}

\begin{cfuncdesc}{void}{Py_XINCREF}{PyObject *o}
  Increment the reference count for object \var{o}.  The object may be
  \NULL, in which case the macro has no effect.
\end{cfuncdesc}

\begin{cfuncdesc}{void}{Py_DECREF}{PyObject *o}
  Decrement the reference count for object \var{o}.  The object must
  not be \NULL; if you aren't sure that it isn't \NULL, use
  \cfunction{Py_XDECREF()}.  If the reference count reaches zero, the
  object's type's deallocation function (which must not be \NULL) is
  invoked.

  \warning{The deallocation function can cause arbitrary Python code
  to be invoked (e.g. when a class instance with a \method{__del__()}
  method is deallocated).  While exceptions in such code are not
  propagated, the executed code has free access to all Python global
  variables.  This means that any object that is reachable from a
  global variable should be in a consistent state before
  \cfunction{Py_DECREF()} is invoked.  For example, code to delete an
  object from a list should copy a reference to the deleted object in
  a temporary variable, update the list data structure, and then call
  \cfunction{Py_DECREF()} for the temporary variable.}
\end{cfuncdesc}

\begin{cfuncdesc}{void}{Py_XDECREF}{PyObject *o}
  Decrement the reference count for object \var{o}.  The object may be
  \NULL, in which case the macro has no effect; otherwise the effect
  is the same as for \cfunction{Py_DECREF()}, and the same warning
  applies.
\end{cfuncdesc}

\begin{cfuncdesc}{void}{Py_CLEAR}{PyObject *o}
  Decrement the reference count for object \var{o}.  The object may be
  \NULL, in which case the macro has no effect; otherwise the effect
  is the same as for \cfunction{Py_DECREF()}, except that the argument
  is also set to \NULL.  The warning for \cfunction{Py_DECREF()} does
  not apply with respect to the object passed because the macro
  carefully uses a temporary variable and sets the argument to \NULL
  before decrementing its reference count.

  It is a good idea to use this macro whenever decrementing the value
  of a variable that might be traversed during garbage collection.

\versionadded{2.4}
\end{cfuncdesc}


The following functions are for runtime dynamic embedding of Python:
\cfunction{Py_IncRef(PyObject *o)}, \cfunction{Py_DecRef(PyObject *o)}.
They are simply exported function versions of \cfunction{Py_XINCREF()} and 
\cfunction{Py_XDECREF()}, respectively.

The following functions or macros are only for use within the
interpreter core: \cfunction{_Py_Dealloc()},
\cfunction{_Py_ForgetReference()}, \cfunction{_Py_NewReference()}, as
well as the global variable \cdata{_Py_RefTotal}.

\chapter{Exception Handling \label{exceptionHandling}}

The functions described in this chapter will let you handle and raise Python
exceptions.  It is important to understand some of the basics of
Python exception handling.  It works somewhat like the
\UNIX{} \cdata{errno} variable: there is a global indicator (per
thread) of the last error that occurred.  Most functions don't clear
this on success, but will set it to indicate the cause of the error on
failure.  Most functions also return an error indicator, usually
\NULL{} if they are supposed to return a pointer, or \code{-1} if they
return an integer (exception: the \cfunction{PyArg_*()} functions
return \code{1} for success and \code{0} for failure).

When a function must fail because some function it called failed, it
generally doesn't set the error indicator; the function it called
already set it.  It is responsible for either handling the error and
clearing the exception or returning after cleaning up any resources it
holds (such as object references or memory allocations); it should
\emph{not} continue normally if it is not prepared to handle the
error.  If returning due to an error, it is important to indicate to
the caller that an error has been set.  If the error is not handled or
carefully propagated, additional calls into the Python/C API may not
behave as intended and may fail in mysterious ways.

The error indicator consists of three Python objects corresponding to
\withsubitem{(in module sys)}{
  \ttindex{exc_type}\ttindex{exc_value}\ttindex{exc_traceback}}
the Python variables \code{sys.exc_type}, \code{sys.exc_value} and
\code{sys.exc_traceback}.  API functions exist to interact with the
error indicator in various ways.  There is a separate error indicator
for each thread.

% XXX Order of these should be more thoughtful.
% Either alphabetical or some kind of structure.

\begin{cfuncdesc}{void}{PyErr_Print}{}
  Print a standard traceback to \code{sys.stderr} and clear the error
  indicator.  Call this function only when the error indicator is
  set.  (Otherwise it will cause a fatal error!)
\end{cfuncdesc}

\begin{cfuncdesc}{PyObject*}{PyErr_Occurred}{}
  Test whether the error indicator is set.  If set, return the
  exception \emph{type} (the first argument to the last call to one of
  the \cfunction{PyErr_Set*()} functions or to
  \cfunction{PyErr_Restore()}).  If not set, return \NULL.  You do
  not own a reference to the return value, so you do not need to
  \cfunction{Py_DECREF()} it.  \note{Do not compare the return value
    to a specific exception; use \cfunction{PyErr_ExceptionMatches()}
    instead, shown below.  (The comparison could easily fail since the
    exception may be an instance instead of a class, in the case of a
    class exception, or it may the a subclass of the expected
    exception.)}
\end{cfuncdesc}

\begin{cfuncdesc}{int}{PyErr_ExceptionMatches}{PyObject *exc}
  Equivalent to \samp{PyErr_GivenExceptionMatches(PyErr_Occurred(),
  \var{exc})}.  This should only be called when an exception is
  actually set; a memory access violation will occur if no exception
  has been raised.
\end{cfuncdesc}

\begin{cfuncdesc}{int}{PyErr_GivenExceptionMatches}{PyObject *given, PyObject *exc}
  Return true if the \var{given} exception matches the exception in
  \var{exc}.  If \var{exc} is a class object, this also returns true
  when \var{given} is an instance of a subclass.  If \var{exc} is a
  tuple, all exceptions in the tuple (and recursively in subtuples)
  are searched for a match.  If \var{given} is \NULL, a memory access
  violation will occur.
\end{cfuncdesc}

\begin{cfuncdesc}{void}{PyErr_NormalizeException}{PyObject**exc, PyObject**val, PyObject**tb}
  Under certain circumstances, the values returned by
  \cfunction{PyErr_Fetch()} below can be ``unnormalized'', meaning
  that \code{*\var{exc}} is a class object but \code{*\var{val}} is
  not an instance of the  same class.  This function can be used to
  instantiate the class in that case.  If the values are already
  normalized, nothing happens.  The delayed normalization is
  implemented to improve performance.
\end{cfuncdesc}

\begin{cfuncdesc}{void}{PyErr_Clear}{}
  Clear the error indicator.  If the error indicator is not set, there
  is no effect.
\end{cfuncdesc}

\begin{cfuncdesc}{void}{PyErr_Fetch}{PyObject **ptype, PyObject **pvalue,
                                     PyObject **ptraceback}
  Retrieve the error indicator into three variables whose addresses
  are passed.  If the error indicator is not set, set all three
  variables to \NULL.  If it is set, it will be cleared and you own a
  reference to each object retrieved.  The value and traceback object
  may be \NULL{} even when the type object is not.  \note{This
  function is normally only used by code that needs to handle
  exceptions or by code that needs to save and restore the error
  indicator temporarily.}
\end{cfuncdesc}

\begin{cfuncdesc}{void}{PyErr_Restore}{PyObject *type, PyObject *value,
                                       PyObject *traceback}
  Set  the error indicator from the three objects.  If the error
  indicator is already set, it is cleared first.  If the objects are
  \NULL, the error indicator is cleared.  Do not pass a \NULL{} type
  and non-\NULL{} value or traceback.  The exception type should be a
  class.  Do not pass an invalid exception type or value.
  (Violating these rules will cause subtle problems later.)  This call
  takes away a reference to each object: you must own a reference to
  each object before the call and after the call you no longer own
  these references.  (If you don't understand this, don't use this
  function.  I warned you.)  \note{This function is normally only used
  by code that needs to save and restore the error indicator
  temporarily; use \cfunction{PyErr_Fetch()} to save the current
  exception state.}
\end{cfuncdesc}

\begin{cfuncdesc}{void}{PyErr_SetString}{PyObject *type, char *message}
  This is the most common way to set the error indicator.  The first
  argument specifies the exception type; it is normally one of the
  standard exceptions, e.g. \cdata{PyExc_RuntimeError}.  You need not
  increment its reference count.  The second argument is an error
  message; it is converted to a string object.
\end{cfuncdesc}

\begin{cfuncdesc}{void}{PyErr_SetObject}{PyObject *type, PyObject *value}
  This function is similar to \cfunction{PyErr_SetString()} but lets
  you specify an arbitrary Python object for the ``value'' of the
  exception.
\end{cfuncdesc}

\begin{cfuncdesc}{PyObject*}{PyErr_Format}{PyObject *exception,
                                           const char *format, \moreargs}
  This function sets the error indicator and returns \NULL.
  \var{exception} should be a Python exception (class, not
  an instance).  \var{format} should be a string, containing format
  codes, similar to \cfunction{printf()}. The \code{width.precision}
  before a format code is parsed, but the width part is ignored.

  \begin{tableii}{c|l}{character}{Character}{Meaning}
    \lineii{c}{Character, as an \ctype{int} parameter}
    \lineii{d}{Number in decimal, as an \ctype{int} parameter}
    \lineii{x}{Number in hexadecimal, as an \ctype{int} parameter}
    \lineii{s}{A string, as a \ctype{char *} parameter}
    \lineii{p}{A hex pointer, as a \ctype{void *} parameter}
  \end{tableii}

  An unrecognized format character causes all the rest of the format
  string to be copied as-is to the result string, and any extra
  arguments discarded.
\end{cfuncdesc}

\begin{cfuncdesc}{void}{PyErr_SetNone}{PyObject *type}
  This is a shorthand for \samp{PyErr_SetObject(\var{type},
  Py_None)}.
\end{cfuncdesc}

\begin{cfuncdesc}{int}{PyErr_BadArgument}{}
  This is a shorthand for \samp{PyErr_SetString(PyExc_TypeError,
  \var{message})}, where \var{message} indicates that a built-in
  operation was invoked with an illegal argument.  It is mostly for
  internal use.
\end{cfuncdesc}

\begin{cfuncdesc}{PyObject*}{PyErr_NoMemory}{}
  This is a shorthand for \samp{PyErr_SetNone(PyExc_MemoryError)}; it
  returns \NULL{} so an object allocation function can write
  \samp{return PyErr_NoMemory();} when it runs out of memory.
\end{cfuncdesc}

\begin{cfuncdesc}{PyObject*}{PyErr_SetFromErrno}{PyObject *type}
  This is a convenience function to raise an exception when a C
  library function has returned an error and set the C variable
  \cdata{errno}.  It constructs a tuple object whose first item is the
  integer \cdata{errno} value and whose second item is the
  corresponding error message (gotten from
  \cfunction{strerror()}\ttindex{strerror()}), and then calls
  \samp{PyErr_SetObject(\var{type}, \var{object})}.  On \UNIX, when
  the \cdata{errno} value is \constant{EINTR}, indicating an
  interrupted system call, this calls
  \cfunction{PyErr_CheckSignals()}, and if that set the error
  indicator, leaves it set to that.  The function always returns
  \NULL, so a wrapper function around a system call can write
  \samp{return PyErr_SetFromErrno(\var{type});} when the system call
  returns an error.
\end{cfuncdesc}

\begin{cfuncdesc}{PyObject*}{PyErr_SetFromErrnoWithFilename}{PyObject *type,
                                                             char *filename}
  Similar to \cfunction{PyErr_SetFromErrno()}, with the additional
  behavior that if \var{filename} is not \NULL, it is passed to the
  constructor of \var{type} as a third parameter.  In the case of
  exceptions such as \exception{IOError} and \exception{OSError}, this
  is used to define the \member{filename} attribute of the exception
  instance.
\end{cfuncdesc}

\begin{cfuncdesc}{PyObject*}{PyErr_SetFromWindowsErr}{int ierr}
  This is a convenience function to raise \exception{WindowsError}.
  If called with \var{ierr} of \cdata{0}, the error code returned by a
  call to \cfunction{GetLastError()} is used instead.  It calls the
  Win32 function \cfunction{FormatMessage()} to retrieve the Windows
  description of error code given by \var{ierr} or
  \cfunction{GetLastError()}, then it constructs a tuple object whose
  first item is the \var{ierr} value and whose second item is the
  corresponding error message (gotten from
  \cfunction{FormatMessage()}), and then calls
  \samp{PyErr_SetObject(\var{PyExc_WindowsError}, \var{object})}.
  This function always returns \NULL.
  Availability: Windows.
\end{cfuncdesc}

\begin{cfuncdesc}{PyObject*}{PyErr_SetExcFromWindowsErr}{PyObject *type,
	                                                 int ierr}
  Similar to \cfunction{PyErr_SetFromWindowsErr()}, with an additional
  parameter specifying the exception type to be raised.
  Availability: Windows.
  \versionadded{2.3}
\end{cfuncdesc}

\begin{cfuncdesc}{PyObject*}{PyErr_SetFromWindowsErrWithFilename}{int ierr,
                                                                char *filename}
  Similar to \cfunction{PyErr_SetFromWindowsErr()}, with the
  additional behavior that if \var{filename} is not \NULL, it is
  passed to the constructor of \exception{WindowsError} as a third
  parameter.
  Availability: Windows.
\end{cfuncdesc}

\begin{cfuncdesc}{PyObject*}{PyErr_SetExcFromWindowsErrWithFilename}
	{PyObject *type, int ierr, char *filename}
  Similar to \cfunction{PyErr_SetFromWindowsErrWithFilename()}, with
  an additional parameter specifying the exception type to be raised.
  Availability: Windows.
  \versionadded{2.3}
\end{cfuncdesc}

\begin{cfuncdesc}{void}{PyErr_BadInternalCall}{}
  This is a shorthand for \samp{PyErr_SetString(PyExc_TypeError,
  \var{message})}, where \var{message} indicates that an internal
  operation (e.g. a Python/C API function) was invoked with an illegal
  argument.  It is mostly for internal use.
\end{cfuncdesc}

\begin{cfuncdesc}{int}{PyErr_Warn}{PyObject *category, char *message}
  Issue a warning message.  The \var{category} argument is a warning
  category (see below) or \NULL; the \var{message} argument is a
  message string.

  This function normally prints a warning message to \var{sys.stderr};
  however, it is also possible that the user has specified that
  warnings are to be turned into errors, and in that case this will
  raise an exception.  It is also possible that the function raises an
  exception because of a problem with the warning machinery (the
  implementation imports the \module{warnings} module to do the heavy
  lifting).  The return value is \code{0} if no exception is raised,
  or \code{-1} if an exception is raised.  (It is not possible to
  determine whether a warning message is actually printed, nor what
  the reason is for the exception; this is intentional.)  If an
  exception is raised, the caller should do its normal exception
  handling (for example, \cfunction{Py_DECREF()} owned references and
  return an error value).

  Warning categories must be subclasses of \cdata{Warning}; the
  default warning category is \cdata{RuntimeWarning}.  The standard
  Python warning categories are available as global variables whose
  names are \samp{PyExc_} followed by the Python exception name.
  These have the type \ctype{PyObject*}; they are all class objects.
  Their names are \cdata{PyExc_Warning}, \cdata{PyExc_UserWarning},
  \cdata{PyExc_DeprecationWarning}, \cdata{PyExc_SyntaxWarning},
  \cdata{PyExc_RuntimeWarning}, and \cdata{PyExc_FutureWarning}.
  \cdata{PyExc_Warning} is a subclass of \cdata{PyExc_Exception}; the
  other warning categories are subclasses of \cdata{PyExc_Warning}.

  For information about warning control, see the documentation for the
  \module{warnings} module and the \programopt{-W} option in the
  command line documentation.  There is no C API for warning control.
\end{cfuncdesc}

\begin{cfuncdesc}{int}{PyErr_WarnExplicit}{PyObject *category, char *message,
                char *filename, int lineno, char *module, PyObject *registry}
  Issue a warning message with explicit control over all warning
  attributes.  This is a straightforward wrapper around the Python
  function \function{warnings.warn_explicit()}, see there for more
  information.  The \var{module} and \var{registry} arguments may be
  set to \NULL{} to get the default effect described there.
\end{cfuncdesc}

\begin{cfuncdesc}{int}{PyErr_CheckSignals}{}
  This function interacts with Python's signal handling.  It checks
  whether a signal has been sent to the processes and if so, invokes
  the corresponding signal handler.  If the
  \module{signal}\refbimodindex{signal} module is supported, this can
  invoke a signal handler written in Python.  In all cases, the
  default effect for \constant{SIGINT}\ttindex{SIGINT} is to raise the
  \withsubitem{(built-in exception)}{\ttindex{KeyboardInterrupt}}
  \exception{KeyboardInterrupt} exception.  If an exception is raised
  the error indicator is set and the function returns \code{1};
  otherwise the function returns \code{0}.  The error indicator may or
  may not be cleared if it was previously set.
\end{cfuncdesc}

\begin{cfuncdesc}{void}{PyErr_SetInterrupt}{}
  This function simulates the effect of a
  \constant{SIGINT}\ttindex{SIGINT} signal arriving --- the next time
  \cfunction{PyErr_CheckSignals()} is called,
  \withsubitem{(built-in exception)}{\ttindex{KeyboardInterrupt}}
  \exception{KeyboardInterrupt} will be raised.  It may be called
  without holding the interpreter lock.
  % XXX This was described as obsolete, but is used in
  % thread.interrupt_main() (used from IDLE), so it's still needed.
\end{cfuncdesc}

\begin{cfuncdesc}{PyObject*}{PyErr_NewException}{char *name,
                                                 PyObject *base,
                                                 PyObject *dict}
  This utility function creates and returns a new exception object.
  The \var{name} argument must be the name of the new exception, a C
  string of the form \code{module.class}.  The \var{base} and
  \var{dict} arguments are normally \NULL.  This creates a class
  object derived from the root for all exceptions, the built-in name
  \exception{Exception} (accessible in C as \cdata{PyExc_Exception}).
  The \member{__module__} attribute of the new class is set to the
  first part (up to the last dot) of the \var{name} argument, and the
  class name is set to the last part (after the last dot).  The
  \var{base} argument can be used to specify an alternate base class.
  The \var{dict} argument can be used to specify a dictionary of class
  variables and methods.
\end{cfuncdesc}

\begin{cfuncdesc}{void}{PyErr_WriteUnraisable}{PyObject *obj}
  This utility function prints a warning message to \code{sys.stderr}
  when an exception has been set but it is impossible for the
  interpreter to actually raise the exception.  It is used, for
  example, when an exception occurs in an \method{__del__()} method.

  The function is called with a single argument \var{obj} that
  identifies the context in which the unraisable exception occurred.
  The repr of \var{obj} will be printed in the warning message.
\end{cfuncdesc}

\section{Standard Exceptions \label{standardExceptions}}

All standard Python exceptions are available as global variables whose
names are \samp{PyExc_} followed by the Python exception name.  These
have the type \ctype{PyObject*}; they are all class objects.  For
completeness, here are all the variables:

\begin{tableiii}{l|l|c}{cdata}{C Name}{Python Name}{Notes}
  \lineiii{PyExc_Exception\ttindex{PyExc_Exception}}{\exception{Exception}}{(1)}
  \lineiii{PyExc_StandardError\ttindex{PyExc_StandardError}}{\exception{StandardError}}{(1)}
  \lineiii{PyExc_ArithmeticError\ttindex{PyExc_ArithmeticError}}{\exception{ArithmeticError}}{(1)}
  \lineiii{PyExc_LookupError\ttindex{PyExc_LookupError}}{\exception{LookupError}}{(1)}
  \lineiii{PyExc_AssertionError\ttindex{PyExc_AssertionError}}{\exception{AssertionError}}{}
  \lineiii{PyExc_AttributeError\ttindex{PyExc_AttributeError}}{\exception{AttributeError}}{}
  \lineiii{PyExc_EOFError\ttindex{PyExc_EOFError}}{\exception{EOFError}}{}
  \lineiii{PyExc_EnvironmentError\ttindex{PyExc_EnvironmentError}}{\exception{EnvironmentError}}{(1)}
  \lineiii{PyExc_FloatingPointError\ttindex{PyExc_FloatingPointError}}{\exception{FloatingPointError}}{}
  \lineiii{PyExc_IOError\ttindex{PyExc_IOError}}{\exception{IOError}}{}
  \lineiii{PyExc_ImportError\ttindex{PyExc_ImportError}}{\exception{ImportError}}{}
  \lineiii{PyExc_IndexError\ttindex{PyExc_IndexError}}{\exception{IndexError}}{}
  \lineiii{PyExc_KeyError\ttindex{PyExc_KeyError}}{\exception{KeyError}}{}
  \lineiii{PyExc_KeyboardInterrupt\ttindex{PyExc_KeyboardInterrupt}}{\exception{KeyboardInterrupt}}{}
  \lineiii{PyExc_MemoryError\ttindex{PyExc_MemoryError}}{\exception{MemoryError}}{}
  \lineiii{PyExc_NameError\ttindex{PyExc_NameError}}{\exception{NameError}}{}
  \lineiii{PyExc_NotImplementedError\ttindex{PyExc_NotImplementedError}}{\exception{NotImplementedError}}{}
  \lineiii{PyExc_OSError\ttindex{PyExc_OSError}}{\exception{OSError}}{}
  \lineiii{PyExc_OverflowError\ttindex{PyExc_OverflowError}}{\exception{OverflowError}}{}
  \lineiii{PyExc_ReferenceError\ttindex{PyExc_ReferenceError}}{\exception{ReferenceError}}{(2)}
  \lineiii{PyExc_RuntimeError\ttindex{PyExc_RuntimeError}}{\exception{RuntimeError}}{}
  \lineiii{PyExc_SyntaxError\ttindex{PyExc_SyntaxError}}{\exception{SyntaxError}}{}
  \lineiii{PyExc_SystemError\ttindex{PyExc_SystemError}}{\exception{SystemError}}{}
  \lineiii{PyExc_SystemExit\ttindex{PyExc_SystemExit}}{\exception{SystemExit}}{}
  \lineiii{PyExc_TypeError\ttindex{PyExc_TypeError}}{\exception{TypeError}}{}
  \lineiii{PyExc_ValueError\ttindex{PyExc_ValueError}}{\exception{ValueError}}{}
  \lineiii{PyExc_WindowsError\ttindex{PyExc_WindowsError}}{\exception{WindowsError}}{(3)}
  \lineiii{PyExc_ZeroDivisionError\ttindex{PyExc_ZeroDivisionError}}{\exception{ZeroDivisionError}}{}
\end{tableiii}

\noindent
Notes:
\begin{description}
\item[(1)]
  This is a base class for other standard exceptions.

\item[(2)]
  This is the same as \exception{weakref.ReferenceError}.

\item[(3)]
  Only defined on Windows; protect code that uses this by testing that
  the preprocessor macro \code{MS_WINDOWS} is defined.
\end{description}


\section{Deprecation of String Exceptions}

All exceptions built into Python or provided in the standard library
are derived from \exception{Exception}.
\withsubitem{(built-in exception)}{\ttindex{Exception}}

String exceptions are still supported in the interpreter to allow
existing code to run unmodified, but this will also change in a future 
release.

\chapter{Utilities \label{utilities}}

The functions in this chapter perform various utility tasks, ranging
from helping C code be more portable across platforms, using Python
modules from C, and parsing function arguments and constructing Python
values from C values.


\section{Operating System Utilities \label{os}}

\begin{cfuncdesc}{int}{Py_FdIsInteractive}{FILE *fp, char *filename}
  Return true (nonzero) if the standard I/O file \var{fp} with name
  \var{filename} is deemed interactive.  This is the case for files
  for which \samp{isatty(fileno(\var{fp}))} is true.  If the global
  flag \cdata{Py_InteractiveFlag} is true, this function also returns
  true if the \var{filename} pointer is \NULL{} or if the name is
  equal to one of the strings \code{'<stdin>'} or \code{'???'}.
\end{cfuncdesc}

\begin{cfuncdesc}{long}{PyOS_GetLastModificationTime}{char *filename}
  Return the time of last modification of the file \var{filename}.
  The result is encoded in the same way as the timestamp returned by
  the standard C library function \cfunction{time()}.
\end{cfuncdesc}

\begin{cfuncdesc}{void}{PyOS_AfterFork}{}
  Function to update some internal state after a process fork; this
  should be called in the new process if the Python interpreter will
  continue to be used.  If a new executable is loaded into the new
  process, this function does not need to be called.
\end{cfuncdesc}

\begin{cfuncdesc}{int}{PyOS_CheckStack}{}
  Return true when the interpreter runs out of stack space.  This is a
  reliable check, but is only available when \constant{USE_STACKCHECK}
  is defined (currently on Windows using the Microsoft Visual \Cpp{}
  compiler).  \constant{USE_CHECKSTACK} will be
  defined automatically; you should never change the definition in
  your own code.
\end{cfuncdesc}

\begin{cfuncdesc}{PyOS_sighandler_t}{PyOS_getsig}{int i}
  Return the current signal handler for signal \var{i}.  This is a
  thin wrapper around either \cfunction{sigaction()} or
  \cfunction{signal()}.  Do not call those functions directly!
  \ctype{PyOS_sighandler_t} is a typedef alias for \ctype{void
  (*)(int)}.
\end{cfuncdesc}

\begin{cfuncdesc}{PyOS_sighandler_t}{PyOS_setsig}{int i, PyOS_sighandler_t h}
  Set the signal handler for signal \var{i} to be \var{h}; return the
  old signal handler.  This is a thin wrapper around either
  \cfunction{sigaction()} or \cfunction{signal()}.  Do not call those
  functions directly!  \ctype{PyOS_sighandler_t} is a typedef alias
  for \ctype{void (*)(int)}.
\end{cfuncdesc}


\section{Process Control \label{processControl}}

\begin{cfuncdesc}{void}{Py_FatalError}{const char *message}
  Print a fatal error message and kill the process.  No cleanup is
  performed.  This function should only be invoked when a condition is
  detected that would make it dangerous to continue using the Python
  interpreter; e.g., when the object administration appears to be
  corrupted.  On \UNIX, the standard C library function
  \cfunction{abort()}\ttindex{abort()} is called which will attempt to
  produce a \file{core} file.
\end{cfuncdesc}

\begin{cfuncdesc}{void}{Py_Exit}{int status}
  Exit the current process.  This calls
  \cfunction{Py_Finalize()}\ttindex{Py_Finalize()} and then calls the
  standard C library function
  \code{exit(\var{status})}\ttindex{exit()}.
\end{cfuncdesc}

\begin{cfuncdesc}{int}{Py_AtExit}{void (*func) ()}
  Register a cleanup function to be called by
  \cfunction{Py_Finalize()}\ttindex{Py_Finalize()}.  The cleanup
  function will be called with no arguments and should return no
  value.  At most 32 \index{cleanup functions}cleanup functions can be
  registered.  When the registration is successful,
  \cfunction{Py_AtExit()} returns \code{0}; on failure, it returns
  \code{-1}.  The cleanup function registered last is called first.
  Each cleanup function will be called at most once.  Since Python's
  internal finalization will have completed before the cleanup
  function, no Python APIs should be called by \var{func}.
\end{cfuncdesc}


\section{Importing Modules \label{importing}}

\begin{cfuncdesc}{PyObject*}{PyImport_ImportModule}{char *name}
  This is a simplified interface to
  \cfunction{PyImport_ImportModuleEx()} below, leaving the
  \var{globals} and \var{locals} arguments set to \NULL.  When the
  \var{name} argument contains a dot (when it specifies a submodule of
  a package), the \var{fromlist} argument is set to the list
  \code{['*']} so that the return value is the named module rather
  than the top-level package containing it as would otherwise be the
  case.  (Unfortunately, this has an additional side effect when
  \var{name} in fact specifies a subpackage instead of a submodule:
  the submodules specified in the package's \code{__all__} variable
  are \index{package variable!\code{__all__}}
  \withsubitem{(package variable)}{\ttindex{__all__}}loaded.)  Return
  a new reference to the imported module, or \NULL{} with an exception
  set on failure.  Before Python 2.4, the module may still be created in
  the failure case --- examine \code{sys.modules} to find out.  Starting
  with Python 2.4, a failing import of a module no longer leaves the
  module in \code{sys.modules}.
  \versionchanged[failing imports remove incomplete module objects]{2.4}
  \withsubitem{(in module sys)}{\ttindex{modules}}
\end{cfuncdesc}

\begin{cfuncdesc}{PyObject*}{PyImport_ImportModuleEx}{char *name,
                       PyObject *globals, PyObject *locals, PyObject *fromlist}
  Import a module.  This is best described by referring to the
  built-in Python function
  \function{__import__()}\bifuncindex{__import__}, as the standard
  \function{__import__()} function calls this function directly.

  The return value is a new reference to the imported module or
  top-level package, or \NULL{} with an exception set on failure (before
  Python 2.4, the
  module may still be created in this case).  Like for
  \function{__import__()}, the return value when a submodule of a
  package was requested is normally the top-level package, unless a
  non-empty \var{fromlist} was given.
  \versionchanged[failing imports remove incomplete module objects]{2.4}
\end{cfuncdesc}

\begin{cfuncdesc}{PyObject*}{PyImport_Import}{PyObject *name}
  This is a higher-level interface that calls the current ``import
  hook function''.  It invokes the \function{__import__()} function
  from the \code{__builtins__} of the current globals.  This means
  that the import is done using whatever import hooks are installed in
  the current environment, e.g. by \module{rexec}\refstmodindex{rexec}
  or \module{ihooks}\refstmodindex{ihooks}.
\end{cfuncdesc}

\begin{cfuncdesc}{PyObject*}{PyImport_ReloadModule}{PyObject *m}
  Reload a module.  This is best described by referring to the
  built-in Python function \function{reload()}\bifuncindex{reload}, as
  the standard \function{reload()} function calls this function
  directly.  Return a new reference to the reloaded module, or \NULL{}
  with an exception set on failure (the module still exists in this
  case).
\end{cfuncdesc}

\begin{cfuncdesc}{PyObject*}{PyImport_AddModule}{char *name}
  Return the module object corresponding to a module name.  The
  \var{name} argument may be of the form \code{package.module}.
  First check the modules dictionary if there's one there, and if not,
  create a new one and insert it in the modules dictionary.
  Return \NULL{} with an exception set on failure.
  \note{This function does not load or import the module; if the
  module wasn't already loaded, you will get an empty module object.
  Use \cfunction{PyImport_ImportModule()} or one of its variants to
  import a module.  Package structures implied by a dotted name for
  \var{name} are not created if not already present.}
\end{cfuncdesc}

\begin{cfuncdesc}{PyObject*}{PyImport_ExecCodeModule}{char *name, PyObject *co}
  Given a module name (possibly of the form \code{package.module}) and
  a code object read from a Python bytecode file or obtained from the
  built-in function \function{compile()}\bifuncindex{compile}, load
  the module.  Return a new reference to the module object, or \NULL{}
  with an exception set if an error occurred.  Before Python 2.4, the module
  could still be created in error cases.  Starting with Python 2.4,
  \var{name} is removed from \code{sys.modules} in error cases, and even
  if \var{name} was already in \code{sys.modules} on entry to
  \cfunction{PyImport_ExecCodeModule()}.  Leaving incompletely initialized
  modules in \code{sys.modules} is dangerous, as imports of such modules
  have no way to know that the module object is an unknown (and probably
  damaged with respect to the module author's intents) state.

  This function will reload the module if it was already imported.  See
  \cfunction{PyImport_ReloadModule()} for the intended way to reload a
  module.

  If \var{name} points to a dotted name of the
  form \code{package.module}, any package structures not already
  created will still not be created.

  \versionchanged[\var{name} is removed from \code{sys.modules} in error cases]{2.4}

\end{cfuncdesc}

\begin{cfuncdesc}{long}{PyImport_GetMagicNumber}{}
  Return the magic number for Python bytecode files
  (a.k.a. \file{.pyc} and \file{.pyo} files).  The magic number should
  be present in the first four bytes of the bytecode file, in
  little-endian byte order.
\end{cfuncdesc}

\begin{cfuncdesc}{PyObject*}{PyImport_GetModuleDict}{}
  Return the dictionary used for the module administration
  (a.k.a.\ \code{sys.modules}).  Note that this is a per-interpreter
  variable.
\end{cfuncdesc}

\begin{cfuncdesc}{void}{_PyImport_Init}{}
  Initialize the import mechanism.  For internal use only.
\end{cfuncdesc}

\begin{cfuncdesc}{void}{PyImport_Cleanup}{}
  Empty the module table.  For internal use only.
\end{cfuncdesc}

\begin{cfuncdesc}{void}{_PyImport_Fini}{}
  Finalize the import mechanism.  For internal use only.
\end{cfuncdesc}

\begin{cfuncdesc}{PyObject*}{_PyImport_FindExtension}{char *, char *}
  For internal use only.
\end{cfuncdesc}

\begin{cfuncdesc}{PyObject*}{_PyImport_FixupExtension}{char *, char *}
  For internal use only.
\end{cfuncdesc}

\begin{cfuncdesc}{int}{PyImport_ImportFrozenModule}{char *name}
  Load a frozen module named \var{name}.  Return \code{1} for success,
  \code{0} if the module is not found, and \code{-1} with an exception
  set if the initialization failed.  To access the imported module on
  a successful load, use \cfunction{PyImport_ImportModule()}.  (Note
  the misnomer --- this function would reload the module if it was
  already imported.)
\end{cfuncdesc}

\begin{ctypedesc}[_frozen]{struct _frozen}
  This is the structure type definition for frozen module descriptors,
  as generated by the \program{freeze}\index{freeze utility} utility
  (see \file{Tools/freeze/} in the Python source distribution).  Its
  definition, found in \file{Include/import.h}, is:

\begin{verbatim}
struct _frozen {
    char *name;
    unsigned char *code;
    int size;
};
\end{verbatim}
\end{ctypedesc}

\begin{cvardesc}{struct _frozen*}{PyImport_FrozenModules}
  This pointer is initialized to point to an array of \ctype{struct
  _frozen} records, terminated by one whose members are all \NULL{} or
  zero.  When a frozen module is imported, it is searched in this
  table.  Third-party code could play tricks with this to provide a
  dynamically created collection of frozen modules.
\end{cvardesc}

\begin{cfuncdesc}{int}{PyImport_AppendInittab}{char *name,
                                               void (*initfunc)(void)}
  Add a single module to the existing table of built-in modules.  This
  is a convenience wrapper around
  \cfunction{PyImport_ExtendInittab()}, returning \code{-1} if the
  table could not be extended.  The new module can be imported by the
  name \var{name}, and uses the function \var{initfunc} as the
  initialization function called on the first attempted import.  This
  should be called before \cfunction{Py_Initialize()}.
\end{cfuncdesc}

\begin{ctypedesc}[_inittab]{struct _inittab}
  Structure describing a single entry in the list of built-in
  modules.  Each of these structures gives the name and initialization
  function for a module built into the interpreter.  Programs which
  embed Python may use an array of these structures in conjunction
  with \cfunction{PyImport_ExtendInittab()} to provide additional
  built-in modules.  The structure is defined in
  \file{Include/import.h} as:

\begin{verbatim}
struct _inittab {
    char *name;
    void (*initfunc)(void);
};
\end{verbatim}
\end{ctypedesc}

\begin{cfuncdesc}{int}{PyImport_ExtendInittab}{struct _inittab *newtab}
  Add a collection of modules to the table of built-in modules.  The
  \var{newtab} array must end with a sentinel entry which contains
  \NULL{} for the \member{name} field; failure to provide the sentinel
  value can result in a memory fault.  Returns \code{0} on success or
  \code{-1} if insufficient memory could be allocated to extend the
  internal table.  In the event of failure, no modules are added to
  the internal table.  This should be called before
  \cfunction{Py_Initialize()}.
\end{cfuncdesc}


\section{Data marshalling support \label{marshalling-utils}}

These routines allow C code to work with serialized objects using the
same data format as the \module{marshal} module.  There are functions
to write data into the serialization format, and additional functions
that can be used to read the data back.  Files used to store marshalled
data must be opened in binary mode.

Numeric values are stored with the least significant byte first.

The module supports two versions of the data format: version 0 is the
historical version, version 1 (new in Python 2.4) shares interned
strings in the file, and upon unmarshalling. \var{Py_MARSHAL_VERSION}
indicates the current file format (currently 1).

\begin{cfuncdesc}{void}{PyMarshal_WriteLongToFile}{long value, FILE *file, int version}
  Marshal a \ctype{long} integer, \var{value}, to \var{file}.  This
  will only write the least-significant 32 bits of \var{value};
  regardless of the size of the native \ctype{long} type.

  \versionchanged[\var{version} indicates the file format]{2.4}
\end{cfuncdesc}

\begin{cfuncdesc}{void}{PyMarshal_WriteObjectToFile}{PyObject *value,
                                                     FILE *file, int version}
  Marshal a Python object, \var{value}, to \var{file}.

  \versionchanged[\var{version} indicates the file format]{2.4}
\end{cfuncdesc}

\begin{cfuncdesc}{PyObject*}{PyMarshal_WriteObjectToString}{PyObject *value, int version}
  Return a string object containing the marshalled representation of
  \var{value}.

  \versionchanged[\var{version} indicates the file format]{2.4}
\end{cfuncdesc}

The following functions allow marshalled values to be read back in.

XXX What about error detection?  It appears that reading past the end
of the file will always result in a negative numeric value (where
that's relevant), but it's not clear that negative values won't be
handled properly when there's no error.  What's the right way to tell?
Should only non-negative values be written using these routines?

\begin{cfuncdesc}{long}{PyMarshal_ReadLongFromFile}{FILE *file}
  Return a C \ctype{long} from the data stream in a \ctype{FILE*}
  opened for reading.  Only a 32-bit value can be read in using
  this function, regardless of the native size of \ctype{long}.
\end{cfuncdesc}

\begin{cfuncdesc}{int}{PyMarshal_ReadShortFromFile}{FILE *file}
  Return a C \ctype{short} from the data stream in a \ctype{FILE*}
  opened for reading.  Only a 16-bit value can be read in using
  this function, regardless of the native size of \ctype{short}.
\end{cfuncdesc}

\begin{cfuncdesc}{PyObject*}{PyMarshal_ReadObjectFromFile}{FILE *file}
  Return a Python object from the data stream in a \ctype{FILE*}
  opened for reading.  On error, sets the appropriate exception
  (\exception{EOFError} or \exception{TypeError}) and returns \NULL.
\end{cfuncdesc}

\begin{cfuncdesc}{PyObject*}{PyMarshal_ReadLastObjectFromFile}{FILE *file}
  Return a Python object from the data stream in a \ctype{FILE*}
  opened for reading.  Unlike
  \cfunction{PyMarshal_ReadObjectFromFile()}, this function assumes
  that no further objects will be read from the file, allowing it to
  aggressively load file data into memory so that the de-serialization
  can operate from data in memory rather than reading a byte at a time
  from the file.  Only use these variant if you are certain that you
  won't be reading anything else from the file.  On error, sets the
  appropriate exception (\exception{EOFError} or
  \exception{TypeError}) and returns \NULL.
\end{cfuncdesc}

\begin{cfuncdesc}{PyObject*}{PyMarshal_ReadObjectFromString}{char *string,
                                                             int len}
  Return a Python object from the data stream in a character buffer
  containing \var{len} bytes pointed to by \var{string}.  On error,
  sets the appropriate exception (\exception{EOFError} or
  \exception{TypeError}) and returns \NULL.
\end{cfuncdesc}


\section{Parsing arguments and building values
         \label{arg-parsing}}

These functions are useful when creating your own extensions functions
and methods.  Additional information and examples are available in
\citetitle[../ext/ext.html]{Extending and Embedding the Python
Interpreter}.

The first three of these functions described,
\cfunction{PyArg_ParseTuple()},
\cfunction{PyArg_ParseTupleAndKeywords()}, and
\cfunction{PyArg_Parse()}, all use \emph{format strings} which are
used to tell the function about the expected arguments.  The format
strings use the same syntax for each of these functions.

A format string consists of zero or more ``format units.''  A format
unit describes one Python object; it is usually a single character or
a parenthesized sequence of format units.  With a few exceptions, a
format unit that is not a parenthesized sequence normally corresponds
to a single address argument to these functions.  In the following
description, the quoted form is the format unit; the entry in (round)
parentheses is the Python object type that matches the format unit;
and the entry in [square] brackets is the type of the C variable(s)
whose address should be passed.

\begin{description}
  \item[\samp{s} (string or Unicode object) {[const char *]}]
  Convert a Python string or Unicode object to a C pointer to a
  character string.  You must not provide storage for the string
  itself; a pointer to an existing string is stored into the character
  pointer variable whose address you pass.  The C string is
  NUL-terminated.  The Python string must not contain embedded NUL
  bytes; if it does, a \exception{TypeError} exception is raised.
  Unicode objects are converted to C strings using the default
  encoding.  If this conversion fails, a \exception{UnicodeError} is
  raised.

  \item[\samp{s\#} (string, Unicode or any read buffer compatible object)
  {[const char *, int]}]
  This variant on \samp{s} stores into two C variables, the first one
  a pointer to a character string, the second one its length.  In this
  case the Python string may contain embedded null bytes.  Unicode
  objects pass back a pointer to the default encoded string version of
  the object if such a conversion is possible.  All other read-buffer
  compatible objects pass back a reference to the raw internal data
  representation.

  \item[\samp{z} (string or \code{None}) {[const char *]}]
  Like \samp{s}, but the Python object may also be \code{None}, in
  which case the C pointer is set to \NULL.

  \item[\samp{z\#} (string or \code{None} or any read buffer
  compatible object) {[const char *, int]}]
  This is to \samp{s\#} as \samp{z} is to \samp{s}.

  \item[\samp{u} (Unicode object) {[Py_UNICODE *]}]
  Convert a Python Unicode object to a C pointer to a NUL-terminated
  buffer of 16-bit Unicode (UTF-16) data.  As with \samp{s}, there is
  no need to provide storage for the Unicode data buffer; a pointer to
  the existing Unicode data is stored into the \ctype{Py_UNICODE}
  pointer variable whose address you pass.

  \item[\samp{u\#} (Unicode object) {[Py_UNICODE *, int]}]
  This variant on \samp{u} stores into two C variables, the first one
  a pointer to a Unicode data buffer, the second one its length.
  Non-Unicode objects are handled by interpreting their read-buffer
  pointer as pointer to a \ctype{Py_UNICODE} array.

  \item[\samp{es} (string, Unicode object or character buffer
  compatible object) {[const char *encoding, char **buffer]}]
  This variant on \samp{s} is used for encoding Unicode and objects
  convertible to Unicode into a character buffer. It only works for
  encoded data without embedded NUL bytes.

  This format requires two arguments.  The first is only used as
  input, and must be a \ctype{const char*} which points to the name of an
  encoding as a NUL-terminated string, or \NULL, in which case the
  default encoding is used.  An exception is raised if the named
  encoding is not known to Python.  The second argument must be a
  \ctype{char**}; the value of the pointer it references will be set
  to a buffer with the contents of the argument text.  The text will
  be encoded in the encoding specified by the first argument.

  \cfunction{PyArg_ParseTuple()} will allocate a buffer of the needed
  size, copy the encoded data into this buffer and adjust
  \var{*buffer} to reference the newly allocated storage.  The caller
  is responsible for calling \cfunction{PyMem_Free()} to free the
  allocated buffer after use.

  \item[\samp{et} (string, Unicode object or character buffer
  compatible object) {[const char *encoding, char **buffer]}]
  Same as \samp{es} except that 8-bit string objects are passed
  through without recoding them.  Instead, the implementation assumes
  that the string object uses the encoding passed in as parameter.

  \item[\samp{es\#} (string, Unicode object or character buffer compatible
  object) {[const char *encoding, char **buffer, int *buffer_length]}]
  This variant on \samp{s\#} is used for encoding Unicode and objects
  convertible to Unicode into a character buffer.  Unlike the
  \samp{es} format, this variant allows input data which contains NUL
  characters.

  It requires three arguments.  The first is only used as input, and
  must be a \ctype{const char*} which points to the name of an encoding as a
  NUL-terminated string, or \NULL, in which case the default encoding
  is used.  An exception is raised if the named encoding is not known
  to Python.  The second argument must be a \ctype{char**}; the value
  of the pointer it references will be set to a buffer with the
  contents of the argument text.  The text will be encoded in the
  encoding specified by the first argument.  The third argument must
  be a pointer to an integer; the referenced integer will be set to
  the number of bytes in the output buffer.

  There are two modes of operation:

  If \var{*buffer} points a \NULL{} pointer, the function will
  allocate a buffer of the needed size, copy the encoded data into
  this buffer and set \var{*buffer} to reference the newly allocated
  storage.  The caller is responsible for calling
  \cfunction{PyMem_Free()} to free the allocated buffer after usage.

  If \var{*buffer} points to a non-\NULL{} pointer (an already
  allocated buffer), \cfunction{PyArg_ParseTuple()} will use this
  location as the buffer and interpret the initial value of
  \var{*buffer_length} as the buffer size.  It will then copy the
  encoded data into the buffer and NUL-terminate it.  If the buffer
  is not large enough, a \exception{ValueError} will be set.

  In both cases, \var{*buffer_length} is set to the length of the
  encoded data without the trailing NUL byte.

  \item[\samp{et\#} (string, Unicode object or character buffer compatible
  object) {[const char *encoding, char **buffer]}]
  Same as \samp{es\#} except that string objects are passed through
  without recoding them. Instead, the implementation assumes that the
  string object uses the encoding passed in as parameter.

  \item[\samp{b} (integer) {[char]}]
  Convert a Python integer to a tiny int, stored in a C \ctype{char}.

  \item[\samp{B} (integer) {[unsigned char]}]
  Convert a Python integer to a tiny int without overflow checking,
  stored in a C \ctype{unsigned char}. \versionadded{2.3}

  \item[\samp{h} (integer) {[short int]}]
  Convert a Python integer to a C \ctype{short int}.

  \item[\samp{H} (integer) {[unsigned short int]}]
  Convert a Python integer to a C \ctype{unsigned short int}, without
  overflow checking.  \versionadded{2.3}

  \item[\samp{i} (integer) {[int]}]
  Convert a Python integer to a plain C \ctype{int}.

  \item[\samp{I} (integer) {[unsigned int]}]
  Convert a Python integer to a C \ctype{unsigned int}, without
  overflow checking.  \versionadded{2.3}

  \item[\samp{l} (integer) {[long int]}]
  Convert a Python integer to a C \ctype{long int}.

  \item[\samp{k} (integer) {[unsigned long]}]
  Convert a Python integer to a C \ctype{unsigned long} without
  overflow checking.  \versionadded{2.3}

  \item[\samp{L} (integer) {[PY_LONG_LONG]}]
  Convert a Python integer to a C \ctype{long long}.  This format is
  only available on platforms that support \ctype{long long} (or
  \ctype{_int64} on Windows).

  \item[\samp{K} (integer) {[unsigned PY_LONG_LONG]}]
  Convert a Python integer to a C \ctype{unsigned long long}
  without overflow checking.  This format is only available on
  platforms that support \ctype{unsigned long long} (or
  \ctype{unsigned _int64} on Windows).  \versionadded{2.3}

  \item[\samp{c} (string of length 1) {[char]}]
  Convert a Python character, represented as a string of length 1, to
  a C \ctype{char}.

  \item[\samp{f} (float) {[float]}]
  Convert a Python floating point number to a C \ctype{float}.

  \item[\samp{d} (float) {[double]}]
  Convert a Python floating point number to a C \ctype{double}.

  \item[\samp{D} (complex) {[Py_complex]}]
  Convert a Python complex number to a C \ctype{Py_complex} structure.

  \item[\samp{O} (object) {[PyObject *]}]
  Store a Python object (without any conversion) in a C object
  pointer.  The C program thus receives the actual object that was
  passed.  The object's reference count is not increased.  The pointer
  stored is not \NULL.

  \item[\samp{O!} (object) {[\var{typeobject}, PyObject *]}]
  Store a Python object in a C object pointer.  This is similar to
  \samp{O}, but takes two C arguments: the first is the address of a
  Python type object, the second is the address of the C variable (of
  type \ctype{PyObject*}) into which the object pointer is stored.  If
  the Python object does not have the required type,
  \exception{TypeError} is raised.

  \item[\samp{O\&} (object) {[\var{converter}, \var{anything}]}]
  Convert a Python object to a C variable through a \var{converter}
  function.  This takes two arguments: the first is a function, the
  second is the address of a C variable (of arbitrary type), converted
  to \ctype{void *}.  The \var{converter} function in turn is called
  as follows:

  \var{status}\code{ = }\var{converter}\code{(}\var{object},
  \var{address}\code{);}

  where \var{object} is the Python object to be converted and
  \var{address} is the \ctype{void*} argument that was passed to the
  \cfunction{PyArg_Parse*()} function.  The returned \var{status}
  should be \code{1} for a successful conversion and \code{0} if the
  conversion has failed.  When the conversion fails, the
  \var{converter} function should raise an exception.

  \item[\samp{S} (string) {[PyStringObject *]}]
  Like \samp{O} but requires that the Python object is a string
  object.  Raises \exception{TypeError} if the object is not a string
  object.  The C variable may also be declared as \ctype{PyObject*}.

  \item[\samp{U} (Unicode string) {[PyUnicodeObject *]}]
  Like \samp{O} but requires that the Python object is a Unicode
  object.  Raises \exception{TypeError} if the object is not a Unicode
  object.  The C variable may also be declared as \ctype{PyObject*}.

  \item[\samp{t\#} (read-only character buffer) {[char *, int]}]
  Like \samp{s\#}, but accepts any object which implements the
  read-only buffer interface.  The \ctype{char*} variable is set to
  point to the first byte of the buffer, and the \ctype{int} is set to
  the length of the buffer.  Only single-segment buffer objects are
  accepted; \exception{TypeError} is raised for all others.

  \item[\samp{w} (read-write character buffer) {[char *]}]
  Similar to \samp{s}, but accepts any object which implements the
  read-write buffer interface.  The caller must determine the length
  of the buffer by other means, or use \samp{w\#} instead.  Only
  single-segment buffer objects are accepted; \exception{TypeError} is
  raised for all others.

  \item[\samp{w\#} (read-write character buffer) {[char *, int]}]
  Like \samp{s\#}, but accepts any object which implements the
  read-write buffer interface.  The \ctype{char *} variable is set to
  point to the first byte of the buffer, and the \ctype{int} is set to
  the length of the buffer.  Only single-segment buffer objects are
  accepted; \exception{TypeError} is raised for all others.

  \item[\samp{(\var{items})} (tuple) {[\var{matching-items}]}]
  The object must be a Python sequence whose length is the number of
  format units in \var{items}.  The C arguments must correspond to the
  individual format units in \var{items}.  Format units for sequences
  may be nested.

  \note{Prior to Python version 1.5.2, this format specifier only
  accepted a tuple containing the individual parameters, not an
  arbitrary sequence.  Code which previously caused
  \exception{TypeError} to be raised here may now proceed without an
  exception.  This is not expected to be a problem for existing code.}
\end{description}

It is possible to pass Python long integers where integers are
requested; however no proper range checking is done --- the most
significant bits are silently truncated when the receiving field is
too small to receive the value (actually, the semantics are inherited
from downcasts in C --- your mileage may vary).

A few other characters have a meaning in a format string.  These may
not occur inside nested parentheses.  They are:

\begin{description}
  \item[\samp{|}]
  Indicates that the remaining arguments in the Python argument list
  are optional.  The C variables corresponding to optional arguments
  should be initialized to their default value --- when an optional
  argument is not specified, \cfunction{PyArg_ParseTuple()} does not
  touch the contents of the corresponding C variable(s).

  \item[\samp{:}]
  The list of format units ends here; the string after the colon is
  used as the function name in error messages (the ``associated
  value'' of the exception that \cfunction{PyArg_ParseTuple()}
  raises).

  \item[\samp{;}]
  The list of format units ends here; the string after the semicolon
  is used as the error message \emph{instead} of the default error
  message.  Clearly, \samp{:} and \samp{;} mutually exclude each
  other.
\end{description}

Note that any Python object references which are provided to the
caller are \emph{borrowed} references; do not decrement their
reference count!

Additional arguments passed to these functions must be addresses of
variables whose type is determined by the format string; these are
used to store values from the input tuple.  There are a few cases, as
described in the list of format units above, where these parameters
are used as input values; they should match what is specified for the
corresponding format unit in that case.

For the conversion to succeed, the \var{arg} object must match the
format and the format must be exhausted.  On success, the
\cfunction{PyArg_Parse*()} functions return true, otherwise they
return false and raise an appropriate exception.

\begin{cfuncdesc}{int}{PyArg_ParseTuple}{PyObject *args, char *format,
                                         \moreargs}
  Parse the parameters of a function that takes only positional
  parameters into local variables.  Returns true on success; on
  failure, it returns false and raises the appropriate exception.
\end{cfuncdesc}

\begin{cfuncdesc}{int}{PyArg_VaParse}{PyObject *args, char *format,
                                         va_list vargs}
  Identical to \cfunction{PyArg_ParseTuple()}, except that it accepts a
  va_list rather than a variable number of arguments.
\end{cfuncdesc}

\begin{cfuncdesc}{int}{PyArg_ParseTupleAndKeywords}{PyObject *args,
                       PyObject *kw, char *format, char *keywords[],
                       \moreargs}
  Parse the parameters of a function that takes both positional and
  keyword parameters into local variables.  Returns true on success;
  on failure, it returns false and raises the appropriate exception.
\end{cfuncdesc}

\begin{cfuncdesc}{int}{PyArg_VaParseTupleAndKeywords}{PyObject *args,
                       PyObject *kw, char *format, char *keywords[],
                       va_list vargs}
  Identical to \cfunction{PyArg_ParseTupleAndKeywords()}, except that it
  accepts a va_list rather than a variable number of arguments.
\end{cfuncdesc}

\begin{cfuncdesc}{int}{PyArg_Parse}{PyObject *args, char *format,
                                    \moreargs}
  Function used to deconstruct the argument lists of ``old-style''
  functions --- these are functions which use the
  \constant{METH_OLDARGS} parameter parsing method.  This is not
  recommended for use in parameter parsing in new code, and most code
  in the standard interpreter has been modified to no longer use this
  for that purpose.  It does remain a convenient way to decompose
  other tuples, however, and may continue to be used for that
  purpose.
\end{cfuncdesc}

\begin{cfuncdesc}{int}{PyArg_UnpackTuple}{PyObject *args, char *name,
                                          int min, int max, \moreargs}
  A simpler form of parameter retrieval which does not use a format
  string to specify the types of the arguments.  Functions which use
  this method to retrieve their parameters should be declared as
  \constant{METH_VARARGS} in function or method tables.  The tuple
  containing the actual parameters should be passed as \var{args}; it
  must actually be a tuple.  The length of the tuple must be at least
  \var{min} and no more than \var{max}; \var{min} and \var{max} may be
  equal.  Additional arguments must be passed to the function, each of
  which should be a pointer to a \ctype{PyObject*} variable; these
  will be filled in with the values from \var{args}; they will contain
  borrowed references.  The variables which correspond to optional
  parameters not given by \var{args} will not be filled in; these
  should be initialized by the caller.
  This function returns true on success and false if \var{args} is not
  a tuple or contains the wrong number of elements; an exception will
  be set if there was a failure.

  This is an example of the use of this function, taken from the
  sources for the \module{_weakref} helper module for weak references:

\begin{verbatim}
static PyObject *
weakref_ref(PyObject *self, PyObject *args)
{
    PyObject *object;
    PyObject *callback = NULL;
    PyObject *result = NULL;

    if (PyArg_UnpackTuple(args, "ref", 1, 2, &object, &callback)) {
        result = PyWeakref_NewRef(object, callback);
    }
    return result;
}
\end{verbatim}

  The call to \cfunction{PyArg_UnpackTuple()} in this example is
  entirely equivalent to this call to \cfunction{PyArg_ParseTuple()}:

\begin{verbatim}
PyArg_ParseTuple(args, "O|O:ref", &object, &callback)
\end{verbatim}

  \versionadded{2.2}
\end{cfuncdesc}

\begin{cfuncdesc}{PyObject*}{Py_BuildValue}{char *format,
                                            \moreargs}
  Create a new value based on a format string similar to those
  accepted by the \cfunction{PyArg_Parse*()} family of functions and a
  sequence of values.  Returns the value or \NULL{} in the case of an
  error; an exception will be raised if \NULL{} is returned.

  \cfunction{Py_BuildValue()} does not always build a tuple.  It
  builds a tuple only if its format string contains two or more format
  units.  If the format string is empty, it returns \code{None}; if it
  contains exactly one format unit, it returns whatever object is
  described by that format unit.  To force it to return a tuple of
  size 0 or one, parenthesize the format string.

  When memory buffers are passed as parameters to supply data to build
  objects, as for the \samp{s} and \samp{s\#} formats, the required
  data is copied.  Buffers provided by the caller are never referenced
  by the objects created by \cfunction{Py_BuildValue()}.  In other
  words, if your code invokes \cfunction{malloc()} and passes the
  allocated memory to \cfunction{Py_BuildValue()}, your code is
  responsible for calling \cfunction{free()} for that memory once
  \cfunction{Py_BuildValue()} returns.

  In the following description, the quoted form is the format unit;
  the entry in (round) parentheses is the Python object type that the
  format unit will return; and the entry in [square] brackets is the
  type of the C value(s) to be passed.

  The characters space, tab, colon and comma are ignored in format
  strings (but not within format units such as \samp{s\#}).  This can
  be used to make long format strings a tad more readable.

  \begin{description}
    \item[\samp{s} (string) {[char *]}]
    Convert a null-terminated C string to a Python object.  If the C
    string pointer is \NULL, \code{None} is used.

    \item[\samp{s\#} (string) {[char *, int]}]
    Convert a C string and its length to a Python object.  If the C
    string pointer is \NULL, the length is ignored and \code{None} is
    returned.

    \item[\samp{z} (string or \code{None}) {[char *]}]
    Same as \samp{s}.

    \item[\samp{z\#} (string or \code{None}) {[char *, int]}]
    Same as \samp{s\#}.

    \item[\samp{u} (Unicode string) {[Py_UNICODE *]}]
    Convert a null-terminated buffer of Unicode (UCS-2 or UCS-4)
    data to a Python Unicode object.  If the Unicode buffer pointer
    is \NULL, \code{None} is returned.

    \item[\samp{u\#} (Unicode string) {[Py_UNICODE *, int]}]
    Convert a Unicode (UCS-2 or UCS-4) data buffer and its length
    to a Python Unicode object.   If the Unicode buffer pointer
    is \NULL, the length is ignored and \code{None} is returned.

    \item[\samp{i} (integer) {[int]}]
    Convert a plain C \ctype{int} to a Python integer object.

    \item[\samp{b} (integer) {[char]}]
    Same as \samp{i}.

    \item[\samp{h} (integer) {[short int]}]
    Same as \samp{i}.

    \item[\samp{l} (integer) {[long int]}]
    Convert a C \ctype{long int} to a Python integer object.

    \item[\samp{c} (string of length 1) {[char]}]
    Convert a C \ctype{int} representing a character to a Python
    string of length 1.

    \item[\samp{d} (float) {[double]}]
    Convert a C \ctype{double} to a Python floating point number.

    \item[\samp{f} (float) {[float]}]
    Same as \samp{d}.

    \item[\samp{D} (complex) {[Py_complex *]}]
    Convert a C \ctype{Py_complex} structure to a Python complex
    number.

    \item[\samp{O} (object) {[PyObject *]}]
    Pass a Python object untouched (except for its reference count,
    which is incremented by one).  If the object passed in is a
    \NULL{} pointer, it is assumed that this was caused because the
    call producing the argument found an error and set an exception.
    Therefore, \cfunction{Py_BuildValue()} will return \NULL{} but
    won't raise an exception.  If no exception has been raised yet,
    \exception{SystemError} is set.

    \item[\samp{S} (object) {[PyObject *]}]
    Same as \samp{O}.

    \item[\samp{N} (object) {[PyObject *]}]
    Same as \samp{O}, except it doesn't increment the reference count
    on the object.  Useful when the object is created by a call to an
    object constructor in the argument list.

    \item[\samp{O\&} (object) {[\var{converter}, \var{anything}]}]
    Convert \var{anything} to a Python object through a
    \var{converter} function.  The function is called with
    \var{anything} (which should be compatible with \ctype{void *}) as
    its argument and should return a ``new'' Python object, or \NULL{}
    if an error occurred.

    \item[\samp{(\var{items})} (tuple) {[\var{matching-items}]}]
    Convert a sequence of C values to a Python tuple with the same
    number of items.

    \item[\samp{[\var{items}]} (list) {[\var{matching-items}]}]
    Convert a sequence of C values to a Python list with the same
    number of items.

    \item[\samp{\{\var{items}\}} (dictionary) {[\var{matching-items}]}]
    Convert a sequence of C values to a Python dictionary.  Each pair
    of consecutive C values adds one item to the dictionary, serving
    as key and value, respectively.

  \end{description}

  If there is an error in the format string, the
  \exception{SystemError} exception is set and \NULL{} returned.
\end{cfuncdesc}

\chapter{Abstract Objects Layer \label{abstract}}

The functions in this chapter interact with Python objects regardless
of their type, or with wide classes of object types (e.g. all
numerical types, or all sequence types).  When used on object types
for which they do not apply, they will raise a Python exception.


\section{Object Protocol \label{object}}

\begin{cfuncdesc}{int}{PyObject_Print}{PyObject *o, FILE *fp, int flags}
  Print an object \var{o}, on file \var{fp}.  Returns \code{-1} on
  error.  The flags argument is used to enable certain printing
  options.  The only option currently supported is
  \constant{Py_PRINT_RAW}; if given, the \function{str()} of the
  object is written instead of the \function{repr()}.
\end{cfuncdesc}

\begin{cfuncdesc}{int}{PyObject_HasAttrString}{PyObject *o, char *attr_name}
  Returns \code{1} if \var{o} has the attribute \var{attr_name}, and
  \code{0} otherwise.  This is equivalent to the Python expression
  \samp{hasattr(\var{o}, \var{attr_name})}.  This function always
  succeeds.
\end{cfuncdesc}

\begin{cfuncdesc}{PyObject*}{PyObject_GetAttrString}{PyObject *o,
                                                     char *attr_name}
  Retrieve an attribute named \var{attr_name} from object \var{o}.
  Returns the attribute value on success, or \NULL{} on failure.
  This is the equivalent of the Python expression
  \samp{\var{o}.\var{attr_name}}.
\end{cfuncdesc}


\begin{cfuncdesc}{int}{PyObject_HasAttr}{PyObject *o, PyObject *attr_name}
  Returns \code{1} if \var{o} has the attribute \var{attr_name}, and
  \code{0} otherwise.  This is equivalent to the Python expression
  \samp{hasattr(\var{o}, \var{attr_name})}.  This function always
  succeeds.
\end{cfuncdesc}


\begin{cfuncdesc}{PyObject*}{PyObject_GetAttr}{PyObject *o,
                                               PyObject *attr_name}
  Retrieve an attribute named \var{attr_name} from object \var{o}.
  Returns the attribute value on success, or \NULL{} on failure.  This
  is the equivalent of the Python expression
  \samp{\var{o}.\var{attr_name}}.
\end{cfuncdesc}


\begin{cfuncdesc}{int}{PyObject_SetAttrString}{PyObject *o,
                                               char *attr_name, PyObject *v}
  Set the value of the attribute named \var{attr_name}, for object
  \var{o}, to the value \var{v}. Returns \code{-1} on failure.  This
  is the equivalent of the Python statement
  \samp{\var{o}.\var{attr_name} = \var{v}}.
\end{cfuncdesc}


\begin{cfuncdesc}{int}{PyObject_SetAttr}{PyObject *o,
                                         PyObject *attr_name, PyObject *v}
  Set the value of the attribute named \var{attr_name}, for object
  \var{o}, to the value \var{v}. Returns \code{-1} on failure.  This
  is the equivalent of the Python statement
  \samp{\var{o}.\var{attr_name} = \var{v}}.
\end{cfuncdesc}


\begin{cfuncdesc}{int}{PyObject_DelAttrString}{PyObject *o, char *attr_name}
  Delete attribute named \var{attr_name}, for object \var{o}. Returns
  \code{-1} on failure.  This is the equivalent of the Python
  statement: \samp{del \var{o}.\var{attr_name}}.
\end{cfuncdesc}


\begin{cfuncdesc}{int}{PyObject_DelAttr}{PyObject *o, PyObject *attr_name}
  Delete attribute named \var{attr_name}, for object \var{o}. Returns
  \code{-1} on failure.  This is the equivalent of the Python
  statement \samp{del \var{o}.\var{attr_name}}.
\end{cfuncdesc}


\begin{cfuncdesc}{PyObject*}{PyObject_RichCompare}{PyObject *o1,
                                                   PyObject *o2, int opid}
  Compare the values of \var{o1} and \var{o2} using the operation
  specified by \var{opid}, which must be one of
  \constant{Py_LT},
  \constant{Py_LE},
  \constant{Py_EQ},
  \constant{Py_NE},
  \constant{Py_GT}, or
  \constant{Py_GE}, corresponding to
  \code{<},
  \code{<=},
  \code{==},
  \code{!=},
  \code{>}, or
  \code{>=} respectively. This is the equivalent of the Python expression
  \samp{\var{o1} op \var{o2}}, where \code{op} is the operator
  corresponding to \var{opid}. Returns the value of the comparison on
  success, or \NULL{} on failure.
\end{cfuncdesc}

\begin{cfuncdesc}{int}{PyObject_RichCompareBool}{PyObject *o1,
                                                 PyObject *o2, int opid}
  Compare the values of \var{o1} and \var{o2} using the operation
  specified by \var{opid}, which must be one of
  \constant{Py_LT},
  \constant{Py_LE},
  \constant{Py_EQ},
  \constant{Py_NE},
  \constant{Py_GT}, or
  \constant{Py_GE}, corresponding to
  \code{<},
  \code{<=},
  \code{==},
  \code{!=},
  \code{>}, or
  \code{>=} respectively. Returns \code{-1} on error, \code{0} if the
  result is false, \code{1} otherwise. This is the equivalent of the
  Python expression \samp{\var{o1} op \var{o2}}, where
  \code{op} is the operator corresponding to \var{opid}.
\end{cfuncdesc}

\begin{cfuncdesc}{int}{PyObject_Cmp}{PyObject *o1, PyObject *o2, int *result}
  Compare the values of \var{o1} and \var{o2} using a routine provided
  by \var{o1}, if one exists, otherwise with a routine provided by
  \var{o2}.  The result of the comparison is returned in
  \var{result}.  Returns \code{-1} on failure.  This is the equivalent
  of the Python statement\bifuncindex{cmp} \samp{\var{result} =
  cmp(\var{o1}, \var{o2})}.
\end{cfuncdesc}


\begin{cfuncdesc}{int}{PyObject_Compare}{PyObject *o1, PyObject *o2}
  Compare the values of \var{o1} and \var{o2} using a routine provided
  by \var{o1}, if one exists, otherwise with a routine provided by
  \var{o2}.  Returns the result of the comparison on success.  On
  error, the value returned is undefined; use
  \cfunction{PyErr_Occurred()} to detect an error.  This is equivalent
  to the Python expression\bifuncindex{cmp} \samp{cmp(\var{o1},
  \var{o2})}.
\end{cfuncdesc}


\begin{cfuncdesc}{PyObject*}{PyObject_Repr}{PyObject *o}
  Compute a string representation of object \var{o}.  Returns the
  string representation on success, \NULL{} on failure.  This is the
  equivalent of the Python expression \samp{repr(\var{o})}.  Called by
  the \function{repr()}\bifuncindex{repr} built-in function and by
  reverse quotes.
\end{cfuncdesc}


\begin{cfuncdesc}{PyObject*}{PyObject_Str}{PyObject *o}
  Compute a string representation of object \var{o}.  Returns the
  string representation on success, \NULL{} on failure.  This is the
  equivalent of the Python expression \samp{str(\var{o})}.  Called by
  the \function{str()}\bifuncindex{str} built-in function and by the
  \keyword{print} statement.
\end{cfuncdesc}


\begin{cfuncdesc}{PyObject*}{PyObject_Unicode}{PyObject *o}
  Compute a Unicode string representation of object \var{o}.  Returns
  the Unicode string representation on success, \NULL{} on failure.
  This is the equivalent of the Python expression
  \samp{unicode(\var{o})}.  Called by the
  \function{unicode()}\bifuncindex{unicode} built-in function.
\end{cfuncdesc}

\begin{cfuncdesc}{int}{PyObject_IsInstance}{PyObject *inst, PyObject *cls}
  Returns \code{1} if \var{inst} is an instance of the class \var{cls}
  or a subclass of \var{cls}, or \code{0} if not.  On error, returns
  \code{-1} and sets an exception.  If \var{cls} is a type object
  rather than a class object, \cfunction{PyObject_IsInstance()}
  returns \code{1} if \var{inst} is of type \var{cls}.  If \var{cls}
  is a tuple, the check will be done against every entry in \var{cls}.
  The result will be \code{1} when at least one of the checks returns
  \code{1}, otherwise it will be \code{0}. If \var{inst} is not a class
  instance and \var{cls} is neither a type object, nor a class object,
  nor a tuple, \var{inst} must have a \member{__class__} attribute
  --- the class relationship of the value of that attribute with
  \var{cls} will be used to determine the result of this function.
  \versionadded{2.1}
  \versionchanged[Support for a tuple as the second argument added]{2.2}
\end{cfuncdesc}

Subclass determination is done in a fairly straightforward way, but
includes a wrinkle that implementors of extensions to the class system
may want to be aware of.  If \class{A} and \class{B} are class
objects, \class{B} is a subclass of \class{A} if it inherits from
\class{A} either directly or indirectly.  If either is not a class
object, a more general mechanism is used to determine the class
relationship of the two objects.  When testing if \var{B} is a
subclass of \var{A}, if \var{A} is \var{B},
\cfunction{PyObject_IsSubclass()} returns true.  If \var{A} and
\var{B} are different objects, \var{B}'s \member{__bases__} attribute
is searched in a depth-first fashion for \var{A} --- the presence of
the \member{__bases__} attribute is considered sufficient for this
determination.

\begin{cfuncdesc}{int}{PyObject_IsSubclass}{PyObject *derived,
                                            PyObject *cls}
  Returns \code{1} if the class \var{derived} is identical to or
  derived from the class \var{cls}, otherwise returns \code{0}.  In
  case of an error, returns \code{-1}. If \var{cls}
  is a tuple, the check will be done against every entry in \var{cls}.
  The result will be \code{1} when at least one of the checks returns
  \code{1}, otherwise it will be \code{0}. If either \var{derived} or
  \var{cls} is not an actual class object (or tuple), this function
  uses the generic algorithm described above.
  \versionadded{2.1}
  \versionchanged[Older versions of Python did not support a tuple
                  as the second argument]{2.3}
\end{cfuncdesc}


\begin{cfuncdesc}{int}{PyCallable_Check}{PyObject *o}
  Determine if the object \var{o} is callable.  Return \code{1} if the
  object is callable and \code{0} otherwise.  This function always
  succeeds.
\end{cfuncdesc}


\begin{cfuncdesc}{PyObject*}{PyObject_Call}{PyObject *callable_object,
                                            PyObject *args,
                                            PyObject *kw}
  Call a callable Python object \var{callable_object}, with arguments
  given by the tuple \var{args}, and named arguments given by the
  dictionary \var{kw}. If no named arguments are needed, \var{kw} may
  be \NULL{}. \var{args} must not be \NULL{}, use an empty tuple if
  no arguments are needed. Returns the result of the call on success,
  or \NULL{} on failure.  This is the equivalent of the Python
  expression \samp{apply(\var{callable_object}, \var{args}, \var{kw})}
  or \samp{\var{callable_object}(*\var{args}, **\var{kw})}.
  \bifuncindex{apply}
  \versionadded{2.2}
\end{cfuncdesc}


\begin{cfuncdesc}{PyObject*}{PyObject_CallObject}{PyObject *callable_object,
                                                  PyObject *args}
  Call a callable Python object \var{callable_object}, with arguments
  given by the tuple \var{args}.  If no arguments are needed, then
  \var{args} may be \NULL.  Returns the result of the call on
  success, or \NULL{} on failure.  This is the equivalent of the
  Python expression \samp{apply(\var{callable_object}, \var{args})} or
  \samp{\var{callable_object}(*\var{args})}.
  \bifuncindex{apply}
\end{cfuncdesc}

\begin{cfuncdesc}{PyObject*}{PyObject_CallFunction}{PyObject *callable,
                                                    char *format, \moreargs}
  Call a callable Python object \var{callable}, with a variable
  number of C arguments.  The C arguments are described using a
  \cfunction{Py_BuildValue()} style format string.  The format may be
  \NULL, indicating that no arguments are provided.  Returns the
  result of the call on success, or \NULL{} on failure.  This is the
  equivalent of the Python expression \samp{apply(\var{callable},
  \var{args})} or \samp{\var{callable}(*\var{args})}.
  \bifuncindex{apply}
\end{cfuncdesc}


\begin{cfuncdesc}{PyObject*}{PyObject_CallMethod}{PyObject *o,
                                                  char *method, char *format,
                                                  \moreargs}
  Call the method named \var{method} of object \var{o} with a variable
  number of C arguments.  The C arguments are described by a
  \cfunction{Py_BuildValue()} format string that should 
  produce a tuple.  The format may be \NULL,
  indicating that no arguments are provided. Returns the result of the
  call on success, or \NULL{} on failure.  This is the equivalent of
  the Python expression \samp{\var{o}.\var{method}(\var{args})}.
\end{cfuncdesc}


\begin{cfuncdesc}{PyObject*}{PyObject_CallFunctionObjArgs}{PyObject *callable,
                                                           \moreargs,
                                                           \code{NULL}}
  Call a callable Python object \var{callable}, with a variable
  number of \ctype{PyObject*} arguments.  The arguments are provided
  as a variable number of parameters followed by \NULL.
  Returns the result of the call on success, or \NULL{} on failure.
  \versionadded{2.2}
\end{cfuncdesc}


\begin{cfuncdesc}{PyObject*}{PyObject_CallMethodObjArgs}{PyObject *o,
                                                         PyObject *name,
                                                         \moreargs,
                                                         \code{NULL}}
  Calls a method of the object \var{o}, where the name of the method
  is given as a Python string object in \var{name}.  It is called with
  a variable number of \ctype{PyObject*} arguments.  The arguments are
  provided as a variable number of parameters followed by \NULL.
  Returns the result of the call on success, or \NULL{} on failure.
  \versionadded{2.2}
\end{cfuncdesc}


\begin{cfuncdesc}{int}{PyObject_Hash}{PyObject *o}
  Compute and return the hash value of an object \var{o}.  On failure,
  return \code{-1}.  This is the equivalent of the Python expression
  \samp{hash(\var{o})}.\bifuncindex{hash}
\end{cfuncdesc}


\begin{cfuncdesc}{int}{PyObject_IsTrue}{PyObject *o}
  Returns \code{1} if the object \var{o} is considered to be true, and
  \code{0} otherwise.  This is equivalent to the Python expression
  \samp{not not \var{o}}.  On failure, return \code{-1}. 
\end{cfuncdesc}


\begin{cfuncdesc}{int}{PyObject_Not}{PyObject *o}
  Returns \code{0} if the object \var{o} is considered to be true, and
  \code{1} otherwise.  This is equivalent to the Python expression
  \samp{not \var{o}}.  On failure, return \code{-1}. 
\end{cfuncdesc}


\begin{cfuncdesc}{PyObject*}{PyObject_Type}{PyObject *o}
  When \var{o} is non-\NULL, returns a type object corresponding to
  the object type of object \var{o}. On failure, raises
  \exception{SystemError} and returns \NULL.  This is equivalent to
  the Python expression \code{type(\var{o})}.\bifuncindex{type}
  This function increments the reference count of the return value.
  There's really no reason to use this function instead of the
  common expression \code{\var{o}->ob_type}, which returns a pointer
  of type \ctype{PyTypeObject*}, except when the incremented reference
  count is needed.
\end{cfuncdesc}

\begin{cfuncdesc}{int}{PyObject_TypeCheck}{PyObject *o, PyTypeObject *type}
  Return true if the object \var{o} is of type \var{type} or a subtype
  of \var{type}.  Both parameters must be non-\NULL.
  \versionadded{2.2}
\end{cfuncdesc}

\begin{cfuncdesc}{int}{PyObject_Length}{PyObject *o}
\cfuncline{int}{PyObject_Size}{PyObject *o}
  Return the length of object \var{o}.  If the object \var{o} provides
  either the sequence and mapping protocols, the sequence length is
  returned.  On error, \code{-1} is returned.  This is the equivalent
  to the Python expression \samp{len(\var{o})}.\bifuncindex{len}
\end{cfuncdesc}


\begin{cfuncdesc}{PyObject*}{PyObject_GetItem}{PyObject *o, PyObject *key}
  Return element of \var{o} corresponding to the object \var{key} or
  \NULL{} on failure.  This is the equivalent of the Python expression
  \samp{\var{o}[\var{key}]}.
\end{cfuncdesc}


\begin{cfuncdesc}{int}{PyObject_SetItem}{PyObject *o,
                                         PyObject *key, PyObject *v}
  Map the object \var{key} to the value \var{v}.  Returns \code{-1} on
  failure.  This is the equivalent of the Python statement
  \samp{\var{o}[\var{key}] = \var{v}}.
\end{cfuncdesc}


\begin{cfuncdesc}{int}{PyObject_DelItem}{PyObject *o, PyObject *key}
  Delete the mapping for \var{key} from \var{o}.  Returns \code{-1} on
  failure. This is the equivalent of the Python statement \samp{del
  \var{o}[\var{key}]}.
\end{cfuncdesc}

\begin{cfuncdesc}{int}{PyObject_AsFileDescriptor}{PyObject *o}
  Derives a file-descriptor from a Python object.  If the object is an
  integer or long integer, its value is returned.  If not, the
  object's \method{fileno()} method is called if it exists; the method
  must return an integer or long integer, which is returned as the
  file descriptor value.  Returns \code{-1} on failure.
\end{cfuncdesc}

\begin{cfuncdesc}{PyObject*}{PyObject_Dir}{PyObject *o}
  This is equivalent to the Python expression \samp{dir(\var{o})},
  returning a (possibly empty) list of strings appropriate for the
  object argument, or \NULL{} if there was an error.  If the argument
  is \NULL, this is like the Python \samp{dir()}, returning the names
  of the current locals; in this case, if no execution frame is active
  then \NULL{} is returned but \cfunction{PyErr_Occurred()} will
  return false.
\end{cfuncdesc}

\begin{cfuncdesc}{PyObject*}{PyObject_GetIter}{PyObject *o}
  This is equivalent to the Python expression \samp{iter(\var{o})}.
  It returns a new iterator for the object argument, or the object 
  itself if the object is already an iterator.  Raises
  \exception{TypeError} and returns \NULL{} if the object cannot be
  iterated.
\end{cfuncdesc}


\section{Number Protocol \label{number}}

\begin{cfuncdesc}{int}{PyNumber_Check}{PyObject *o}
  Returns \code{1} if the object \var{o} provides numeric protocols,
  and false otherwise.  This function always succeeds.
\end{cfuncdesc}


\begin{cfuncdesc}{PyObject*}{PyNumber_Add}{PyObject *o1, PyObject *o2}
  Returns the result of adding \var{o1} and \var{o2}, or \NULL{} on
  failure.  This is the equivalent of the Python expression
  \samp{\var{o1} + \var{o2}}.
\end{cfuncdesc}


\begin{cfuncdesc}{PyObject*}{PyNumber_Subtract}{PyObject *o1, PyObject *o2}
  Returns the result of subtracting \var{o2} from \var{o1}, or \NULL{}
  on failure.  This is the equivalent of the Python expression
  \samp{\var{o1} - \var{o2}}.
\end{cfuncdesc}


\begin{cfuncdesc}{PyObject*}{PyNumber_Multiply}{PyObject *o1, PyObject *o2}
  Returns the result of multiplying \var{o1} and \var{o2}, or \NULL{}
  on failure.  This is the equivalent of the Python expression
  \samp{\var{o1} * \var{o2}}.
\end{cfuncdesc}


\begin{cfuncdesc}{PyObject*}{PyNumber_Divide}{PyObject *o1, PyObject *o2}
  Returns the result of dividing \var{o1} by \var{o2}, or \NULL{} on
  failure.  This is the equivalent of the Python expression
  \samp{\var{o1} / \var{o2}}.
\end{cfuncdesc}


\begin{cfuncdesc}{PyObject*}{PyNumber_FloorDivide}{PyObject *o1, PyObject *o2}
  Return the floor of \var{o1} divided by \var{o2}, or \NULL{} on
  failure.  This is equivalent to the ``classic'' division of
  integers.
  \versionadded{2.2}
\end{cfuncdesc}


\begin{cfuncdesc}{PyObject*}{PyNumber_TrueDivide}{PyObject *o1, PyObject *o2}
  Return a reasonable approximation for the mathematical value of
  \var{o1} divided by \var{o2}, or \NULL{} on failure.  The return
  value is ``approximate'' because binary floating point numbers are
  approximate; it is not possible to represent all real numbers in
  base two.  This function can return a floating point value when
  passed two integers.
  \versionadded{2.2}
\end{cfuncdesc}


\begin{cfuncdesc}{PyObject*}{PyNumber_Remainder}{PyObject *o1, PyObject *o2}
  Returns the remainder of dividing \var{o1} by \var{o2}, or \NULL{}
  on failure.  This is the equivalent of the Python expression
  \samp{\var{o1} \%\ \var{o2}}.
\end{cfuncdesc}


\begin{cfuncdesc}{PyObject*}{PyNumber_Divmod}{PyObject *o1, PyObject *o2}
  See the built-in function \function{divmod()}\bifuncindex{divmod}.
  Returns \NULL{} on failure.  This is the equivalent of the Python
  expression \samp{divmod(\var{o1}, \var{o2})}.
\end{cfuncdesc}


\begin{cfuncdesc}{PyObject*}{PyNumber_Power}{PyObject *o1,
                                             PyObject *o2, PyObject *o3}
  See the built-in function \function{pow()}\bifuncindex{pow}.
  Returns \NULL{} on failure.  This is the equivalent of the Python
  expression \samp{pow(\var{o1}, \var{o2}, \var{o3})}, where \var{o3}
  is optional.  If \var{o3} is to be ignored, pass \cdata{Py_None} in
  its place (passing \NULL{} for \var{o3} would cause an illegal
  memory access).
\end{cfuncdesc}


\begin{cfuncdesc}{PyObject*}{PyNumber_Negative}{PyObject *o}
  Returns the negation of \var{o} on success, or \NULL{} on failure.
  This is the equivalent of the Python expression \samp{-\var{o}}.
\end{cfuncdesc}


\begin{cfuncdesc}{PyObject*}{PyNumber_Positive}{PyObject *o}
  Returns \var{o} on success, or \NULL{} on failure.  This is the
  equivalent of the Python expression \samp{+\var{o}}.
\end{cfuncdesc}


\begin{cfuncdesc}{PyObject*}{PyNumber_Absolute}{PyObject *o}
  Returns the absolute value of \var{o}, or \NULL{} on failure.  This
  is the equivalent of the Python expression \samp{abs(\var{o})}.
  \bifuncindex{abs}
\end{cfuncdesc}


\begin{cfuncdesc}{PyObject*}{PyNumber_Invert}{PyObject *o}
  Returns the bitwise negation of \var{o} on success, or \NULL{} on
  failure.  This is the equivalent of the Python expression
  \samp{\~\var{o}}.
\end{cfuncdesc}


\begin{cfuncdesc}{PyObject*}{PyNumber_Lshift}{PyObject *o1, PyObject *o2}
  Returns the result of left shifting \var{o1} by \var{o2} on success,
  or \NULL{} on failure.  This is the equivalent of the Python
  expression \samp{\var{o1} <\code{<} \var{o2}}.
\end{cfuncdesc}


\begin{cfuncdesc}{PyObject*}{PyNumber_Rshift}{PyObject *o1, PyObject *o2}
  Returns the result of right shifting \var{o1} by \var{o2} on
  success, or \NULL{} on failure.  This is the equivalent of the
  Python expression \samp{\var{o1} >\code{>} \var{o2}}.
\end{cfuncdesc}


\begin{cfuncdesc}{PyObject*}{PyNumber_And}{PyObject *o1, PyObject *o2}
  Returns the ``bitwise and'' of \var{o1} and \var{o2} on success and
  \NULL{} on failure. This is the equivalent of the Python expression
  \samp{\var{o1} \&\ \var{o2}}.
\end{cfuncdesc}


\begin{cfuncdesc}{PyObject*}{PyNumber_Xor}{PyObject *o1, PyObject *o2}
  Returns the ``bitwise exclusive or'' of \var{o1} by \var{o2} on
  success, or \NULL{} on failure.  This is the equivalent of the
  Python expression \samp{\var{o1} \textasciicircum{} \var{o2}}.
\end{cfuncdesc}

\begin{cfuncdesc}{PyObject*}{PyNumber_Or}{PyObject *o1, PyObject *o2}
  Returns the ``bitwise or'' of \var{o1} and \var{o2} on success, or
  \NULL{} on failure.  This is the equivalent of the Python expression
  \samp{\var{o1} | \var{o2}}.
\end{cfuncdesc}


\begin{cfuncdesc}{PyObject*}{PyNumber_InPlaceAdd}{PyObject *o1, PyObject *o2}
  Returns the result of adding \var{o1} and \var{o2}, or \NULL{} on
  failure.  The operation is done \emph{in-place} when \var{o1}
  supports it.  This is the equivalent of the Python statement
  \samp{\var{o1} += \var{o2}}.
\end{cfuncdesc}


\begin{cfuncdesc}{PyObject*}{PyNumber_InPlaceSubtract}{PyObject *o1,
                                                       PyObject *o2}
  Returns the result of subtracting \var{o2} from \var{o1}, or \NULL{}
  on failure.  The operation is done \emph{in-place} when \var{o1}
  supports it.  This is the equivalent of the Python statement
  \samp{\var{o1} -= \var{o2}}.
\end{cfuncdesc}


\begin{cfuncdesc}{PyObject*}{PyNumber_InPlaceMultiply}{PyObject *o1,
                                                       PyObject *o2}
  Returns the result of multiplying \var{o1} and \var{o2}, or \NULL{}
  on failure.  The operation is done \emph{in-place} when \var{o1}
  supports it.  This is the equivalent of the Python statement
  \samp{\var{o1} *= \var{o2}}.
\end{cfuncdesc}


\begin{cfuncdesc}{PyObject*}{PyNumber_InPlaceDivide}{PyObject *o1,
                                                     PyObject *o2}
  Returns the result of dividing \var{o1} by \var{o2}, or \NULL{} on
  failure.  The operation is done \emph{in-place} when \var{o1}
  supports it. This is the equivalent of the Python statement
  \samp{\var{o1} /= \var{o2}}.
\end{cfuncdesc}


\begin{cfuncdesc}{PyObject*}{PyNumber_InPlaceFloorDivide}{PyObject *o1,
                                                          PyObject *o2}
  Returns the mathematical floor of dividing \var{o1} by \var{o2}, or
  \NULL{} on failure.  The operation is done \emph{in-place} when
  \var{o1} supports it.  This is the equivalent of the Python
  statement \samp{\var{o1} //= \var{o2}}.
  \versionadded{2.2}
\end{cfuncdesc}


\begin{cfuncdesc}{PyObject*}{PyNumber_InPlaceTrueDivide}{PyObject *o1,
                                                         PyObject *o2}
  Return a reasonable approximation for the mathematical value of
  \var{o1} divided by \var{o2}, or \NULL{} on failure.  The return
  value is ``approximate'' because binary floating point numbers are
  approximate; it is not possible to represent all real numbers in
  base two.  This function can return a floating point value when
  passed two integers.  The operation is done \emph{in-place} when
  \var{o1} supports it.
  \versionadded{2.2}
\end{cfuncdesc}


\begin{cfuncdesc}{PyObject*}{PyNumber_InPlaceRemainder}{PyObject *o1,
                                                        PyObject *o2}
  Returns the remainder of dividing \var{o1} by \var{o2}, or \NULL{}
  on failure.  The operation is done \emph{in-place} when \var{o1}
  supports it.  This is the equivalent of the Python statement
  \samp{\var{o1} \%= \var{o2}}.
\end{cfuncdesc}


\begin{cfuncdesc}{PyObject*}{PyNumber_InPlacePower}{PyObject *o1,
                                                    PyObject *o2, PyObject *o3}
  See the built-in function \function{pow()}.\bifuncindex{pow}
  Returns \NULL{} on failure.  The operation is done \emph{in-place}
  when \var{o1} supports it.  This is the equivalent of the Python
  statement \samp{\var{o1} **= \var{o2}} when o3 is \cdata{Py_None},
  or an in-place variant of \samp{pow(\var{o1}, \var{o2}, \var{o3})}
  otherwise. If \var{o3} is to be ignored, pass \cdata{Py_None} in its
  place (passing \NULL{} for \var{o3} would cause an illegal memory
  access).
\end{cfuncdesc}

\begin{cfuncdesc}{PyObject*}{PyNumber_InPlaceLshift}{PyObject *o1,
                                                     PyObject *o2}
  Returns the result of left shifting \var{o1} by \var{o2} on success,
  or \NULL{} on failure.  The operation is done \emph{in-place} when
  \var{o1} supports it.  This is the equivalent of the Python
  statement \samp{\var{o1} <\code{<=} \var{o2}}.
\end{cfuncdesc}


\begin{cfuncdesc}{PyObject*}{PyNumber_InPlaceRshift}{PyObject *o1,
                                                     PyObject *o2}
  Returns the result of right shifting \var{o1} by \var{o2} on
  success, or \NULL{} on failure.  The operation is done
  \emph{in-place} when \var{o1} supports it.  This is the equivalent
  of the Python statement \samp{\var{o1} >\code{>=} \var{o2}}.
\end{cfuncdesc}


\begin{cfuncdesc}{PyObject*}{PyNumber_InPlaceAnd}{PyObject *o1, PyObject *o2}
  Returns the ``bitwise and'' of \var{o1} and \var{o2} on success and
  \NULL{} on failure. The operation is done \emph{in-place} when
  \var{o1} supports it.  This is the equivalent of the Python
  statement \samp{\var{o1} \&= \var{o2}}.
\end{cfuncdesc}


\begin{cfuncdesc}{PyObject*}{PyNumber_InPlaceXor}{PyObject *o1, PyObject *o2}
  Returns the ``bitwise exclusive or'' of \var{o1} by \var{o2} on
  success, or \NULL{} on failure.  The operation is done
  \emph{in-place} when \var{o1} supports it.  This is the equivalent
  of the Python statement \samp{\var{o1} \textasciicircum= \var{o2}}.
\end{cfuncdesc}

\begin{cfuncdesc}{PyObject*}{PyNumber_InPlaceOr}{PyObject *o1, PyObject *o2}
  Returns the ``bitwise or'' of \var{o1} and \var{o2} on success, or
  \NULL{} on failure.  The operation is done \emph{in-place} when
  \var{o1} supports it.  This is the equivalent of the Python
  statement \samp{\var{o1} |= \var{o2}}.
\end{cfuncdesc}

\begin{cfuncdesc}{int}{PyNumber_Coerce}{PyObject **p1, PyObject **p2}
  This function takes the addresses of two variables of type
  \ctype{PyObject*}.  If the objects pointed to by \code{*\var{p1}}
  and \code{*\var{p2}} have the same type, increment their reference
  count and return \code{0} (success). If the objects can be converted
  to a common numeric type, replace \code{*p1} and \code{*p2} by their
  converted value (with 'new' reference counts), and return \code{0}.
  If no conversion is possible, or if some other error occurs, return
  \code{-1} (failure) and don't increment the reference counts.  The
  call \code{PyNumber_Coerce(\&o1, \&o2)} is equivalent to the Python
  statement \samp{\var{o1}, \var{o2} = coerce(\var{o1}, \var{o2})}.
  \bifuncindex{coerce}
\end{cfuncdesc}

\begin{cfuncdesc}{PyObject*}{PyNumber_Int}{PyObject *o}
  Returns the \var{o} converted to an integer object on success, or
  \NULL{} on failure.  If the argument is outside the integer range
  a long object will be returned instead. This is the equivalent
  of the Python expression \samp{int(\var{o})}.\bifuncindex{int}
\end{cfuncdesc}

\begin{cfuncdesc}{PyObject*}{PyNumber_Long}{PyObject *o}
  Returns the \var{o} converted to a long integer object on success,
  or \NULL{} on failure.  This is the equivalent of the Python
  expression \samp{long(\var{o})}.\bifuncindex{long}
\end{cfuncdesc}

\begin{cfuncdesc}{PyObject*}{PyNumber_Float}{PyObject *o}
  Returns the \var{o} converted to a float object on success, or
  \NULL{} on failure.  This is the equivalent of the Python expression
  \samp{float(\var{o})}.\bifuncindex{float}
\end{cfuncdesc}


\section{Sequence Protocol \label{sequence}}

\begin{cfuncdesc}{int}{PySequence_Check}{PyObject *o}
  Return \code{1} if the object provides sequence protocol, and
  \code{0} otherwise.  This function always succeeds.
\end{cfuncdesc}

\begin{cfuncdesc}{int}{PySequence_Size}{PyObject *o}
  Returns the number of objects in sequence \var{o} on success, and
  \code{-1} on failure.  For objects that do not provide sequence
  protocol, this is equivalent to the Python expression
  \samp{len(\var{o})}.\bifuncindex{len}
\end{cfuncdesc}

\begin{cfuncdesc}{int}{PySequence_Length}{PyObject *o}
  Alternate name for \cfunction{PySequence_Size()}.
\end{cfuncdesc}

\begin{cfuncdesc}{PyObject*}{PySequence_Concat}{PyObject *o1, PyObject *o2}
  Return the concatenation of \var{o1} and \var{o2} on success, and
  \NULL{} on failure.   This is the equivalent of the Python
  expression \samp{\var{o1} + \var{o2}}.
\end{cfuncdesc}


\begin{cfuncdesc}{PyObject*}{PySequence_Repeat}{PyObject *o, int count}
  Return the result of repeating sequence object \var{o} \var{count}
  times, or \NULL{} on failure.  This is the equivalent of the Python
  expression \samp{\var{o} * \var{count}}.
\end{cfuncdesc}

\begin{cfuncdesc}{PyObject*}{PySequence_InPlaceConcat}{PyObject *o1,
                                                       PyObject *o2}
  Return the concatenation of \var{o1} and \var{o2} on success, and
  \NULL{} on failure.  The operation is done \emph{in-place} when
  \var{o1} supports it.  This is the equivalent of the Python
  expression \samp{\var{o1} += \var{o2}}.
\end{cfuncdesc}


\begin{cfuncdesc}{PyObject*}{PySequence_InPlaceRepeat}{PyObject *o, int count}
  Return the result of repeating sequence object \var{o} \var{count}
  times, or \NULL{} on failure.  The operation is done \emph{in-place}
  when \var{o} supports it.  This is the equivalent of the Python
  expression \samp{\var{o} *= \var{count}}.
\end{cfuncdesc}


\begin{cfuncdesc}{PyObject*}{PySequence_GetItem}{PyObject *o, int i}
  Return the \var{i}th element of \var{o}, or \NULL{} on failure.
  This is the equivalent of the Python expression
  \samp{\var{o}[\var{i}]}.
\end{cfuncdesc}


\begin{cfuncdesc}{PyObject*}{PySequence_GetSlice}{PyObject *o, int i1, int i2}
  Return the slice of sequence object \var{o} between \var{i1} and
  \var{i2}, or \NULL{} on failure. This is the equivalent of the
  Python expression \samp{\var{o}[\var{i1}:\var{i2}]}.
\end{cfuncdesc}


\begin{cfuncdesc}{int}{PySequence_SetItem}{PyObject *o, int i, PyObject *v}
  Assign object \var{v} to the \var{i}th element of \var{o}.  Returns
  \code{-1} on failure.  This is the equivalent of the Python
  statement \samp{\var{o}[\var{i}] = \var{v}}.  This function \emph{does not}
  steal a reference to \var{v}.
\end{cfuncdesc}

\begin{cfuncdesc}{int}{PySequence_DelItem}{PyObject *o, int i}
  Delete the \var{i}th element of object \var{o}.  Returns \code{-1}
  on failure.  This is the equivalent of the Python statement
  \samp{del \var{o}[\var{i}]}.
\end{cfuncdesc}

\begin{cfuncdesc}{int}{PySequence_SetSlice}{PyObject *o, int i1,
                                            int i2, PyObject *v}
  Assign the sequence object \var{v} to the slice in sequence object
  \var{o} from \var{i1} to \var{i2}.  This is the equivalent of the
  Python statement \samp{\var{o}[\var{i1}:\var{i2}] = \var{v}}.
\end{cfuncdesc}

\begin{cfuncdesc}{int}{PySequence_DelSlice}{PyObject *o, int i1, int i2}
  Delete the slice in sequence object \var{o} from \var{i1} to
  \var{i2}.  Returns \code{-1} on failure.  This is the equivalent of
  the Python statement \samp{del \var{o}[\var{i1}:\var{i2}]}.
\end{cfuncdesc}

\begin{cfuncdesc}{PyObject*}{PySequence_Tuple}{PyObject *o}
  Returns the \var{o} as a tuple on success, and \NULL{} on failure.
  This is equivalent to the Python expression \samp{tuple(\var{o})}.
  \bifuncindex{tuple}
\end{cfuncdesc}

\begin{cfuncdesc}{int}{PySequence_Count}{PyObject *o, PyObject *value}
  Return the number of occurrences of \var{value} in \var{o}, that is,
  return the number of keys for which \code{\var{o}[\var{key}] ==
  \var{value}}.  On failure, return \code{-1}.  This is equivalent to
  the Python expression \samp{\var{o}.count(\var{value})}.
\end{cfuncdesc}

\begin{cfuncdesc}{int}{PySequence_Contains}{PyObject *o, PyObject *value}
  Determine if \var{o} contains \var{value}.  If an item in \var{o} is
  equal to \var{value}, return \code{1}, otherwise return \code{0}.
  On error, return \code{-1}.  This is equivalent to the Python
  expression \samp{\var{value} in \var{o}}.
\end{cfuncdesc}

\begin{cfuncdesc}{int}{PySequence_Index}{PyObject *o, PyObject *value}
  Return the first index \var{i} for which \code{\var{o}[\var{i}] ==
  \var{value}}.  On error, return \code{-1}.    This is equivalent to
  the Python expression \samp{\var{o}.index(\var{value})}.
\end{cfuncdesc}

\begin{cfuncdesc}{PyObject*}{PySequence_List}{PyObject *o}
  Return a list object with the same contents as the arbitrary
  sequence \var{o}.  The returned list is guaranteed to be new.
\end{cfuncdesc}

\begin{cfuncdesc}{PyObject*}{PySequence_Tuple}{PyObject *o}
  Return a tuple object with the same contents as the arbitrary
  sequence \var{o}.  If \var{o} is a tuple, a new reference will be
  returned, otherwise a tuple will be constructed with the appropriate
  contents.
\end{cfuncdesc}

\begin{cfuncdesc}{PyObject*}{PySequence_Fast}{PyObject *o, const char *m}
  Returns the sequence \var{o} as a tuple, unless it is already a
  tuple or list, in which case \var{o} is returned.  Use
  \cfunction{PySequence_Fast_GET_ITEM()} to access the members of the
  result.  Returns \NULL{} on failure.  If the object is not a
  sequence, raises \exception{TypeError} with \var{m} as the message
  text.
\end{cfuncdesc}

\begin{cfuncdesc}{PyObject*}{PySequence_Fast_GET_ITEM}{PyObject *o, int i}
  Return the \var{i}th element of \var{o}, assuming that \var{o} was
  returned by \cfunction{PySequence_Fast()}, \var{o} is not \NULL,
  and that \var{i} is within bounds.
\end{cfuncdesc}

\begin{cfuncdesc}{PyObject**}{PySequence_Fast_ITEMS}{PyObject *o}
  Return the underlying array of PyObject pointers.  Assumes that
  \var{o} was returned by \cfunction{PySequence_Fast()} and
  \var{o} is not \NULL.
  \versionadded{2.4}  
\end{cfuncdesc}

\begin{cfuncdesc}{PyObject*}{PySequence_ITEM}{PyObject *o, int i}
  Return the \var{i}th element of \var{o} or \NULL{} on failure.
  Macro form of \cfunction{PySequence_GetItem()} but without checking
  that \cfunction{PySequence_Check(\var{o})} is true and without
  adjustment for negative indices.
  \versionadded{2.3}
\end{cfuncdesc}

\begin{cfuncdesc}{int}{PySequence_Fast_GET_SIZE}{PyObject *o}
  Returns the length of \var{o}, assuming that \var{o} was
  returned by \cfunction{PySequence_Fast()} and that \var{o} is
  not \NULL.  The size can also be gotten by calling
  \cfunction{PySequence_Size()} on \var{o}, but
  \cfunction{PySequence_Fast_GET_SIZE()} is faster because it can
  assume \var{o} is a list or tuple.
\end{cfuncdesc}


\section{Mapping Protocol \label{mapping}}

\begin{cfuncdesc}{int}{PyMapping_Check}{PyObject *o}
  Return \code{1} if the object provides mapping protocol, and
  \code{0} otherwise.  This function always succeeds.
\end{cfuncdesc}


\begin{cfuncdesc}{int}{PyMapping_Length}{PyObject *o}
  Returns the number of keys in object \var{o} on success, and
  \code{-1} on failure.  For objects that do not provide mapping
  protocol, this is equivalent to the Python expression
  \samp{len(\var{o})}.\bifuncindex{len}
\end{cfuncdesc}


\begin{cfuncdesc}{int}{PyMapping_DelItemString}{PyObject *o, char *key}
  Remove the mapping for object \var{key} from the object \var{o}.
  Return \code{-1} on failure.  This is equivalent to the Python
  statement \samp{del \var{o}[\var{key}]}.
\end{cfuncdesc}


\begin{cfuncdesc}{int}{PyMapping_DelItem}{PyObject *o, PyObject *key}
  Remove the mapping for object \var{key} from the object \var{o}.
  Return \code{-1} on failure.  This is equivalent to the Python
  statement \samp{del \var{o}[\var{key}]}.
\end{cfuncdesc}


\begin{cfuncdesc}{int}{PyMapping_HasKeyString}{PyObject *o, char *key}
  On success, return \code{1} if the mapping object has the key
  \var{key} and \code{0} otherwise.  This is equivalent to the Python
  expression \samp{\var{o}.has_key(\var{key})}.  This function always
  succeeds.
\end{cfuncdesc}


\begin{cfuncdesc}{int}{PyMapping_HasKey}{PyObject *o, PyObject *key}
  Return \code{1} if the mapping object has the key \var{key} and
  \code{0} otherwise.  This is equivalent to the Python expression
  \samp{\var{o}.has_key(\var{key})}.  This function always succeeds.
\end{cfuncdesc}


\begin{cfuncdesc}{PyObject*}{PyMapping_Keys}{PyObject *o}
  On success, return a list of the keys in object \var{o}.  On
  failure, return \NULL. This is equivalent to the Python expression
  \samp{\var{o}.keys()}.
\end{cfuncdesc}


\begin{cfuncdesc}{PyObject*}{PyMapping_Values}{PyObject *o}
  On success, return a list of the values in object \var{o}.  On
  failure, return \NULL. This is equivalent to the Python expression
  \samp{\var{o}.values()}.
\end{cfuncdesc}


\begin{cfuncdesc}{PyObject*}{PyMapping_Items}{PyObject *o}
  On success, return a list of the items in object \var{o}, where each
  item is a tuple containing a key-value pair.  On failure, return
  \NULL. This is equivalent to the Python expression
  \samp{\var{o}.items()}.
\end{cfuncdesc}


\begin{cfuncdesc}{PyObject*}{PyMapping_GetItemString}{PyObject *o, char *key}
  Return element of \var{o} corresponding to the object \var{key} or
  \NULL{} on failure. This is the equivalent of the Python expression
  \samp{\var{o}[\var{key}]}.
\end{cfuncdesc}

\begin{cfuncdesc}{int}{PyMapping_SetItemString}{PyObject *o, char *key,
                                                PyObject *v}
  Map the object \var{key} to the value \var{v} in object \var{o}.
  Returns \code{-1} on failure.  This is the equivalent of the Python
  statement \samp{\var{o}[\var{key}] = \var{v}}.
\end{cfuncdesc}


\section{Iterator Protocol \label{iterator}}

\versionadded{2.2}

There are only a couple of functions specifically for working with
iterators.

\begin{cfuncdesc}{int}{PyIter_Check}{PyObject *o}
  Return true if the object \var{o} supports the iterator protocol.
\end{cfuncdesc}

\begin{cfuncdesc}{PyObject*}{PyIter_Next}{PyObject *o}
  Return the next value from the iteration \var{o}.  If the object is
  an iterator, this retrieves the next value from the iteration, and
  returns \NULL{} with no exception set if there are no remaining
  items.  If the object is not an iterator, \exception{TypeError} is
  raised, or if there is an error in retrieving the item, returns
  \NULL{} and passes along the exception.
\end{cfuncdesc}

To write a loop which iterates over an iterator, the C code should
look something like this:

\begin{verbatim}
PyObject *iterator = PyObject_GetIter(obj);
PyObject *item;

if (iterator == NULL) {
    /* propagate error */
}

while (item = PyIter_Next(iterator)) {
    /* do something with item */
    ...
    /* release reference when done */
    Py_DECREF(item);
}

Py_DECREF(iterator);

if (PyErr_Occurred()) {
    /* propagate error */
}
else {
    /* continue doing useful work */
}
\end{verbatim}


\section{Buffer Protocol \label{abstract-buffer}}

\begin{cfuncdesc}{int}{PyObject_AsCharBuffer}{PyObject *obj,
                                              const char **buffer,
                                              int *buffer_len}
  Returns a pointer to a read-only memory location useable as character-
  based input.  The \var{obj} argument must support the single-segment
  character buffer interface.  On success, returns \code{0}, sets
  \var{buffer} to the memory location and \var{buffer_len} to the buffer
  length.  Returns \code{-1} and sets a \exception{TypeError} on error.
  \versionadded{1.6}
\end{cfuncdesc}

\begin{cfuncdesc}{int}{PyObject_AsReadBuffer}{PyObject *obj,
                                              const void **buffer,
                                              int *buffer_len}
  Returns a pointer to a read-only memory location containing
  arbitrary data.  The \var{obj} argument must support the
  single-segment readable buffer interface.  On success, returns
  \code{0}, sets \var{buffer} to the memory location and \var{buffer_len}
  to the buffer length.  Returns \code{-1} and sets a
  \exception{TypeError} on error.
  \versionadded{1.6}
\end{cfuncdesc}

\begin{cfuncdesc}{int}{PyObject_CheckReadBuffer}{PyObject *o}
  Returns \code{1} if \var{o} supports the single-segment readable
  buffer interface.  Otherwise returns \code{0}.
  \versionadded{2.2}
\end{cfuncdesc}

\begin{cfuncdesc}{int}{PyObject_AsWriteBuffer}{PyObject *obj,
                                               void **buffer,
                                               int *buffer_len}
  Returns a pointer to a writeable memory location.  The \var{obj}
  argument must support the single-segment, character buffer
  interface.  On success, returns \code{0}, sets \var{buffer} to the
  memory location and \var{buffer_len} to the buffer length.  Returns
  \code{-1} and sets a \exception{TypeError} on error.
  \versionadded{1.6}
\end{cfuncdesc}

\chapter{Concrete Objects Layer \label{concrete}}


The functions in this chapter are specific to certain Python object
types.  Passing them an object of the wrong type is not a good idea;
if you receive an object from a Python program and you are not sure
that it has the right type, you must perform a type check first;
for example, to check that an object is a dictionary, use
\cfunction{PyDict_Check()}.  The chapter is structured like the
``family tree'' of Python object types.

\warning{While the functions described in this chapter carefully check
the type of the objects which are passed in, many of them do not check
for \NULL{} being passed instead of a valid object.  Allowing \NULL{}
to be passed in can cause memory access violations and immediate
termination of the interpreter.}


\section{Fundamental Objects \label{fundamental}}

This section describes Python type objects and the singleton object
\code{None}.


\subsection{Type Objects \label{typeObjects}}

\obindex{type}
\begin{ctypedesc}{PyTypeObject}
  The C structure of the objects used to describe built-in types.
\end{ctypedesc}

\begin{cvardesc}{PyObject*}{PyType_Type}
  This is the type object for type objects; it is the same object as
  \code{type} and \code{types.TypeType} in the Python layer.
  \withsubitem{(in module types)}{\ttindex{TypeType}}
\end{cvardesc}

\begin{cfuncdesc}{int}{PyType_Check}{PyObject *o}
  Return true if the object \var{o} is a type object, including
  instances of types derived from the standard type object.  Return
  false in all other cases.
\end{cfuncdesc}

\begin{cfuncdesc}{int}{PyType_CheckExact}{PyObject *o}
  Return true if the object \var{o} is a type object, but not a
  subtype of the standard type object.  Return false in all other
  cases.
  \versionadded{2.2}
\end{cfuncdesc}

\begin{cfuncdesc}{int}{PyType_HasFeature}{PyObject *o, int feature}
  Return true if the type object \var{o} sets the feature
  \var{feature}.  Type features are denoted by single bit flags.
\end{cfuncdesc}

\begin{cfuncdesc}{int}{PyType_IS_GC}{PyObject *o}
  Return true if the type object includes support for the cycle
  detector; this tests the type flag \constant{Py_TPFLAGS_HAVE_GC}.
  \versionadded{2.0}
\end{cfuncdesc}

\begin{cfuncdesc}{int}{PyType_IsSubtype}{PyTypeObject *a, PyTypeObject *b}
  Return true if \var{a} is a subtype of \var{b}.
  \versionadded{2.2}
\end{cfuncdesc}

\begin{cfuncdesc}{PyObject*}{PyType_GenericAlloc}{PyTypeObject *type,
                                                  Py_ssize_t nitems}
  \versionadded{2.2}
\end{cfuncdesc}

\begin{cfuncdesc}{PyObject*}{PyType_GenericNew}{PyTypeObject *type,
                                            PyObject *args, PyObject *kwds}
  \versionadded{2.2}
\end{cfuncdesc}

\begin{cfuncdesc}{int}{PyType_Ready}{PyTypeObject *type}
  Finalize a type object.  This should be called on all type objects
  to finish their initialization.  This function is responsible for
  adding inherited slots from a type's base class.  Return \code{0}
  on success, or return \code{-1} and sets an exception on error.
  \versionadded{2.2}
\end{cfuncdesc}


\subsection{The None Object \label{noneObject}}

\obindex{None}
Note that the \ctype{PyTypeObject} for \code{None} is not directly
exposed in the Python/C API.  Since \code{None} is a singleton,
testing for object identity (using \samp{==} in C) is sufficient.
There is no \cfunction{PyNone_Check()} function for the same reason.

\begin{cvardesc}{PyObject*}{Py_None}
  The Python \code{None} object, denoting lack of value.  This object
  has no methods.  It needs to be treated just like any other object
  with respect to reference counts.
\end{cvardesc}

\begin{csimplemacrodesc}{Py_RETURN_NONE}
  Properly handle returning \cdata{Py_None} from within a C function.
  \versionadded{2.4}
\end{csimplemacrodesc}


\section{Numeric Objects \label{numericObjects}}

\obindex{numeric}


\subsection{Plain Integer Objects \label{intObjects}}

\obindex{integer}
\begin{ctypedesc}{PyIntObject}
  This subtype of \ctype{PyObject} represents a Python integer
  object.
\end{ctypedesc}

\begin{cvardesc}{PyTypeObject}{PyInt_Type}
  This instance of \ctype{PyTypeObject} represents the Python plain
  integer type.  This is the same object as \code{int} and
  \code{types.IntType}.
  \withsubitem{(in modules types)}{\ttindex{IntType}}
\end{cvardesc}

\begin{cfuncdesc}{int}{PyInt_Check}{PyObject *o}
  Return true if \var{o} is of type \cdata{PyInt_Type} or a subtype
  of \cdata{PyInt_Type}.
  \versionchanged[Allowed subtypes to be accepted]{2.2}
\end{cfuncdesc}

\begin{cfuncdesc}{int}{PyInt_CheckExact}{PyObject *o}
  Return true if \var{o} is of type \cdata{PyInt_Type}, but not a
  subtype of \cdata{PyInt_Type}.
  \versionadded{2.2}
\end{cfuncdesc}

\begin{cfuncdesc}{PyObject*}{PyInt_FromString}{char *str, char **pend,
                                               int base}
  Return a new \ctype{PyIntObject} or \ctype{PyLongObject} based on the
  string value in \var{str}, which is interpreted according to the radix in
  \var{base}.  If \var{pend} is non-\NULL{}, \code{*\var{pend}} will point to
  the first character in \var{str} which follows the representation of the
  number.  If \var{base} is \code{0}, the radix will be determined based on
  the leading characters of \var{str}: if \var{str} starts with \code{'0x'}
  or \code{'0X'}, radix 16 will be used; if \var{str} starts with
  \code{'0'}, radix 8 will be used; otherwise radix 10 will be used.  If
  \var{base} is not \code{0}, it must be between \code{2} and \code{36},
  inclusive.  Leading spaces are ignored.  If there are no digits,
  \exception{ValueError} will be raised.  If the string represents a number
  too large to be contained within the machine's \ctype{long int} type and
  overflow warnings are being suppressed, a \ctype{PyLongObject} will be
  returned.  If overflow warnings are not being suppressed, \NULL{} will be
  returned in this case.
\end{cfuncdesc}

\begin{cfuncdesc}{PyObject*}{PyInt_FromLong}{long ival}
  Create a new integer object with a value of \var{ival}.

  The current implementation keeps an array of integer objects for all
  integers between \code{-5} and \code{256}, when you create an int in
  that range you actually just get back a reference to the existing
  object. So it should be possible to change the value of \code{1}.  I
  suspect the behaviour of Python in this case is undefined. :-)
\end{cfuncdesc}

\begin{cfuncdesc}{PyObject*}{PyInt_FromSsize_t}{Py_ssize_t ival}
  Create a new integer object with a value of \var{ival}.
  If the value exceeds \code{LONG_MAX}, a long integer object is
  returned.

 \versionadded{2.5}
\end{cfuncdesc}

\begin{cfuncdesc}{long}{PyInt_AsLong}{PyObject *io}
  Will first attempt to cast the object to a \ctype{PyIntObject}, if
  it is not already one, and then return its value. If there is an
  error, \code{-1} is returned, and the caller should check
  \code{PyErr_Occurred()} to find out whether there was an error, or
  whether the value just happened to be -1.
\end{cfuncdesc}

\begin{cfuncdesc}{long}{PyInt_AS_LONG}{PyObject *io}
  Return the value of the object \var{io}.  No error checking is
  performed.
\end{cfuncdesc}

\begin{cfuncdesc}{unsigned long}{PyInt_AsUnsignedLongMask}{PyObject *io}
  Will first attempt to cast the object to a \ctype{PyIntObject} or
  \ctype{PyLongObject}, if it is not already one, and then return its
  value as unsigned long.  This function does not check for overflow.
  \versionadded{2.3}
\end{cfuncdesc}

\begin{cfuncdesc}{unsigned PY_LONG_LONG}{PyInt_AsUnsignedLongLongMask}{PyObject *io}
  Will first attempt to cast the object to a \ctype{PyIntObject} or
  \ctype{PyLongObject}, if it is not already one, and then return its
  value as unsigned long long, without checking for overflow.
  \versionadded{2.3}
\end{cfuncdesc}

\begin{cfuncdesc}{Py_ssize_t}{PyInt_AsSsize_t}{PyObject *io}
  Will first attempt to cast the object to a \ctype{PyIntObject} or
  \ctype{PyLongObject}, if it is not already one, and then return its
  value as \ctype{Py_ssize_t}.
  \versionadded{2.5}
\end{cfuncdesc}

\begin{cfuncdesc}{long}{PyInt_GetMax}{}
  Return the system's idea of the largest integer it can handle
  (\constant{LONG_MAX}\ttindex{LONG_MAX}, as defined in the system
  header files).
\end{cfuncdesc}

\subsection{Boolean Objects \label{boolObjects}}

Booleans in Python are implemented as a subclass of integers.  There
are only two booleans, \constant{Py_False} and \constant{Py_True}.  As
such, the normal creation and deletion functions don't apply to
booleans.  The following macros are available, however.

\begin{cfuncdesc}{int}{PyBool_Check}{PyObject *o}
  Return true if \var{o} is of type \cdata{PyBool_Type}.
  \versionadded{2.3}
\end{cfuncdesc}

\begin{cvardesc}{PyObject*}{Py_False}
  The Python \code{False} object.  This object has no methods.  It needs to
  be treated just like any other object with respect to reference counts.
\end{cvardesc}

\begin{cvardesc}{PyObject*}{Py_True}
  The Python \code{True} object.  This object has no methods.  It needs to
  be treated just like any other object with respect to reference counts.
\end{cvardesc}

\begin{csimplemacrodesc}{Py_RETURN_FALSE}
  Return \constant{Py_False} from a function, properly incrementing its
  reference count.
\versionadded{2.4}
\end{csimplemacrodesc}

\begin{csimplemacrodesc}{Py_RETURN_TRUE}
  Return \constant{Py_True} from a function, properly incrementing its
  reference count.
\versionadded{2.4}
\end{csimplemacrodesc}

\begin{cfuncdesc}{PyObject*}{PyBool_FromLong}{long v}
  Return a new reference to \constant{Py_True} or \constant{Py_False}
  depending on the truth value of \var{v}.
\versionadded{2.3}
\end{cfuncdesc}

\subsection{Long Integer Objects \label{longObjects}}

\obindex{long integer}
\begin{ctypedesc}{PyLongObject}
  This subtype of \ctype{PyObject} represents a Python long integer
  object.
\end{ctypedesc}

\begin{cvardesc}{PyTypeObject}{PyLong_Type}
  This instance of \ctype{PyTypeObject} represents the Python long
  integer type.  This is the same object as \code{long} and
  \code{types.LongType}.
  \withsubitem{(in modules types)}{\ttindex{LongType}}
\end{cvardesc}

\begin{cfuncdesc}{int}{PyLong_Check}{PyObject *p}
  Return true if its argument is a \ctype{PyLongObject} or a subtype
  of \ctype{PyLongObject}.
  \versionchanged[Allowed subtypes to be accepted]{2.2}
\end{cfuncdesc}

\begin{cfuncdesc}{int}{PyLong_CheckExact}{PyObject *p}
  Return true if its argument is a \ctype{PyLongObject}, but not a
  subtype of \ctype{PyLongObject}.
  \versionadded{2.2}
\end{cfuncdesc}

\begin{cfuncdesc}{PyObject*}{PyLong_FromLong}{long v}
  Return a new \ctype{PyLongObject} object from \var{v}, or \NULL{}
  on failure.
\end{cfuncdesc}

\begin{cfuncdesc}{PyObject*}{PyLong_FromUnsignedLong}{unsigned long v}
  Return a new \ctype{PyLongObject} object from a C \ctype{unsigned
  long}, or \NULL{} on failure.
\end{cfuncdesc}

\begin{cfuncdesc}{PyObject*}{PyLong_FromLongLong}{PY_LONG_LONG v}
  Return a new \ctype{PyLongObject} object from a C \ctype{long long},
  or \NULL{} on failure.
\end{cfuncdesc}

\begin{cfuncdesc}{PyObject*}{PyLong_FromUnsignedLongLong}{unsigned PY_LONG_LONG v}
  Return a new \ctype{PyLongObject} object from a C \ctype{unsigned
  long long}, or \NULL{} on failure.
\end{cfuncdesc}

\begin{cfuncdesc}{PyObject*}{PyLong_FromDouble}{double v}
  Return a new \ctype{PyLongObject} object from the integer part of
  \var{v}, or \NULL{} on failure.
\end{cfuncdesc}

\begin{cfuncdesc}{PyObject*}{PyLong_FromString}{char *str, char **pend,
                                                int base}
  Return a new \ctype{PyLongObject} based on the string value in
  \var{str}, which is interpreted according to the radix in
  \var{base}.  If \var{pend} is non-\NULL{}, \code{*\var{pend}} will
  point to the first character in \var{str} which follows the
  representation of the number.  If \var{base} is \code{0}, the radix
  will be determined based on the leading characters of \var{str}: if
  \var{str} starts with \code{'0x'} or \code{'0X'}, radix 16 will be
  used; if \var{str} starts with \code{'0'}, radix 8 will be used;
  otherwise radix 10 will be used.  If \var{base} is not \code{0}, it
  must be between \code{2} and \code{36}, inclusive.  Leading spaces
  are ignored.  If there are no digits, \exception{ValueError} will be
  raised.
\end{cfuncdesc}

\begin{cfuncdesc}{PyObject*}{PyLong_FromUnicode}{Py_UNICODE *u,
                                                 Py_ssize_t length, int base}
  Convert a sequence of Unicode digits to a Python long integer
  value.  The first parameter, \var{u}, points to the first character
  of the Unicode string, \var{length} gives the number of characters,
  and \var{base} is the radix for the conversion.  The radix must be
  in the range [2, 36]; if it is out of range, \exception{ValueError}
  will be raised.
  \versionadded{1.6}
\end{cfuncdesc}

\begin{cfuncdesc}{PyObject*}{PyLong_FromVoidPtr}{void *p}
  Create a Python integer or long integer from the pointer \var{p}.
  The pointer value can be retrieved from the resulting value using
  \cfunction{PyLong_AsVoidPtr()}.
  \versionadded{1.5.2}
  \versionchanged[If the integer is larger than LONG_MAX,
  a positive long integer is returned]{2.5}
 \end{cfuncdesc}

\begin{cfuncdesc}{long}{PyLong_AsLong}{PyObject *pylong}
  Return a C \ctype{long} representation of the contents of
  \var{pylong}.  If \var{pylong} is greater than
  \constant{LONG_MAX}\ttindex{LONG_MAX}, an \exception{OverflowError}
  is raised.
  \withsubitem{(built-in exception)}{\ttindex{OverflowError}}
\end{cfuncdesc}

\begin{cfuncdesc}{unsigned long}{PyLong_AsUnsignedLong}{PyObject *pylong}
  Return a C \ctype{unsigned long} representation of the contents of
  \var{pylong}.  If \var{pylong} is greater than
  \constant{ULONG_MAX}\ttindex{ULONG_MAX}, an
  \exception{OverflowError} is raised.
  \withsubitem{(built-in exception)}{\ttindex{OverflowError}}
\end{cfuncdesc}

\begin{cfuncdesc}{PY_LONG_LONG}{PyLong_AsLongLong}{PyObject *pylong}
  Return a C \ctype{long long} from a Python long integer.  If
  \var{pylong} cannot be represented as a \ctype{long long}, an
  \exception{OverflowError} will be raised.
  \versionadded{2.2}
\end{cfuncdesc}

\begin{cfuncdesc}{unsigned PY_LONG_LONG}{PyLong_AsUnsignedLongLong}{PyObject
                                                                 *pylong}
  Return a C \ctype{unsigned long long} from a Python long integer.
  If \var{pylong} cannot be represented as an \ctype{unsigned long
  long}, an \exception{OverflowError} will be raised if the value is
  positive, or a \exception{TypeError} will be raised if the value is
  negative.
  \versionadded{2.2}
\end{cfuncdesc}

\begin{cfuncdesc}{unsigned long}{PyLong_AsUnsignedLongMask}{PyObject *io}
  Return a C \ctype{unsigned long} from a Python long integer, without
  checking for overflow.
  \versionadded{2.3}
\end{cfuncdesc}

\begin{cfuncdesc}{unsigned PY_LONG_LONG}{PyLong_AsUnsignedLongLongMask}{PyObject *io}
  Return a C \ctype{unsigned long long} from a Python long integer, without
  checking for overflow.
  \versionadded{2.3}
\end{cfuncdesc}

\begin{cfuncdesc}{double}{PyLong_AsDouble}{PyObject *pylong}
  Return a C \ctype{double} representation of the contents of
  \var{pylong}.  If \var{pylong} cannot be approximately represented
  as a \ctype{double}, an \exception{OverflowError} exception is
  raised and \code{-1.0} will be returned.
\end{cfuncdesc}

\begin{cfuncdesc}{void*}{PyLong_AsVoidPtr}{PyObject *pylong}
  Convert a Python integer or long integer \var{pylong} to a C
  \ctype{void} pointer.  If \var{pylong} cannot be converted, an
  \exception{OverflowError} will be raised.  This is only assured to
  produce a usable \ctype{void} pointer for values created with
  \cfunction{PyLong_FromVoidPtr()}.
  \versionadded{1.5.2}
  \versionchanged[For values outside 0..LONG_MAX, both signed and
  unsigned integers are acccepted]{2.5}
\end{cfuncdesc}


\subsection{Floating Point Objects \label{floatObjects}}

\obindex{floating point}
\begin{ctypedesc}{PyFloatObject}
  This subtype of \ctype{PyObject} represents a Python floating point
  object.
\end{ctypedesc}

\begin{cvardesc}{PyTypeObject}{PyFloat_Type}
  This instance of \ctype{PyTypeObject} represents the Python floating
  point type.  This is the same object as \code{float} and
  \code{types.FloatType}.
  \withsubitem{(in modules types)}{\ttindex{FloatType}}
\end{cvardesc}

\begin{cfuncdesc}{int}{PyFloat_Check}{PyObject *p}
  Return true if its argument is a \ctype{PyFloatObject} or a subtype
  of \ctype{PyFloatObject}.
  \versionchanged[Allowed subtypes to be accepted]{2.2}
\end{cfuncdesc}

\begin{cfuncdesc}{int}{PyFloat_CheckExact}{PyObject *p}
  Return true if its argument is a \ctype{PyFloatObject}, but not a
  subtype of \ctype{PyFloatObject}.
  \versionadded{2.2}
\end{cfuncdesc}

\begin{cfuncdesc}{PyObject*}{PyFloat_FromString}{PyObject *str}
  Create a \ctype{PyFloatObject} object based on the string value in
  \var{str}, or \NULL{} on failure.
\end{cfuncdesc}

\begin{cfuncdesc}{PyObject*}{PyFloat_FromDouble}{double v}
  Create a \ctype{PyFloatObject} object from \var{v}, or \NULL{} on
  failure.
\end{cfuncdesc}

\begin{cfuncdesc}{double}{PyFloat_AsDouble}{PyObject *pyfloat}
  Return a C \ctype{double} representation of the contents of
  \var{pyfloat}.  If \var{pyfloat} is not a Python floating point
  object but has a \method{__float__} method, this method will first
  be called to convert \var{pyfloat} into a float.
\end{cfuncdesc}

\begin{cfuncdesc}{double}{PyFloat_AS_DOUBLE}{PyObject *pyfloat}
  Return a C \ctype{double} representation of the contents of
  \var{pyfloat}, but without error checking.
\end{cfuncdesc}


\subsection{Complex Number Objects \label{complexObjects}}

\obindex{complex number}
Python's complex number objects are implemented as two distinct types
when viewed from the C API:  one is the Python object exposed to
Python programs, and the other is a C structure which represents the
actual complex number value.  The API provides functions for working
with both.

\subsubsection{Complex Numbers as C Structures}

Note that the functions which accept these structures as parameters
and return them as results do so \emph{by value} rather than
dereferencing them through pointers.  This is consistent throughout
the API.

\begin{ctypedesc}{Py_complex}
  The C structure which corresponds to the value portion of a Python
  complex number object.  Most of the functions for dealing with
  complex number objects use structures of this type as input or
  output values, as appropriate.  It is defined as:

\begin{verbatim}
typedef struct {
   double real;
   double imag;
} Py_complex;
\end{verbatim}
\end{ctypedesc}

\begin{cfuncdesc}{Py_complex}{_Py_c_sum}{Py_complex left, Py_complex right}
  Return the sum of two complex numbers, using the C
  \ctype{Py_complex} representation.
\end{cfuncdesc}

\begin{cfuncdesc}{Py_complex}{_Py_c_diff}{Py_complex left, Py_complex right}
  Return the difference between two complex numbers, using the C
  \ctype{Py_complex} representation.
\end{cfuncdesc}

\begin{cfuncdesc}{Py_complex}{_Py_c_neg}{Py_complex complex}
  Return the negation of the complex number \var{complex}, using the C
  \ctype{Py_complex} representation.
\end{cfuncdesc}

\begin{cfuncdesc}{Py_complex}{_Py_c_prod}{Py_complex left, Py_complex right}
  Return the product of two complex numbers, using the C
  \ctype{Py_complex} representation.
\end{cfuncdesc}

\begin{cfuncdesc}{Py_complex}{_Py_c_quot}{Py_complex dividend,
                                          Py_complex divisor}
  Return the quotient of two complex numbers, using the C
  \ctype{Py_complex} representation.
\end{cfuncdesc}

\begin{cfuncdesc}{Py_complex}{_Py_c_pow}{Py_complex num, Py_complex exp}
  Return the exponentiation of \var{num} by \var{exp}, using the C
  \ctype{Py_complex} representation.
\end{cfuncdesc}


\subsubsection{Complex Numbers as Python Objects}

\begin{ctypedesc}{PyComplexObject}
  This subtype of \ctype{PyObject} represents a Python complex number
  object.
\end{ctypedesc}

\begin{cvardesc}{PyTypeObject}{PyComplex_Type}
  This instance of \ctype{PyTypeObject} represents the Python complex
  number type. It is the same object as \code{complex} and
  \code{types.ComplexType}.
\end{cvardesc}

\begin{cfuncdesc}{int}{PyComplex_Check}{PyObject *p}
  Return true if its argument is a \ctype{PyComplexObject} or a
  subtype of \ctype{PyComplexObject}.
  \versionchanged[Allowed subtypes to be accepted]{2.2}
\end{cfuncdesc}

\begin{cfuncdesc}{int}{PyComplex_CheckExact}{PyObject *p}
  Return true if its argument is a \ctype{PyComplexObject}, but not a
  subtype of \ctype{PyComplexObject}.
  \versionadded{2.2}
\end{cfuncdesc}

\begin{cfuncdesc}{PyObject*}{PyComplex_FromCComplex}{Py_complex v}
  Create a new Python complex number object from a C
  \ctype{Py_complex} value.
\end{cfuncdesc}

\begin{cfuncdesc}{PyObject*}{PyComplex_FromDoubles}{double real, double imag}
  Return a new \ctype{PyComplexObject} object from \var{real} and
  \var{imag}.
\end{cfuncdesc}

\begin{cfuncdesc}{double}{PyComplex_RealAsDouble}{PyObject *op}
  Return the real part of \var{op} as a C \ctype{double}.
\end{cfuncdesc}

\begin{cfuncdesc}{double}{PyComplex_ImagAsDouble}{PyObject *op}
  Return the imaginary part of \var{op} as a C \ctype{double}.
\end{cfuncdesc}

\begin{cfuncdesc}{Py_complex}{PyComplex_AsCComplex}{PyObject *op}
  Return the \ctype{Py_complex} value of the complex number \var{op}.
  \versionchanged[If \var{op} is not a Python complex number object
                  but has a \method{__complex__} method, this method
		  will first be called to convert \var{op} to a Python
		  complex number object]{2.6}
\end{cfuncdesc}



\section{Sequence Objects \label{sequenceObjects}}

\obindex{sequence}
Generic operations on sequence objects were discussed in the previous
chapter; this section deals with the specific kinds of sequence
objects that are intrinsic to the Python language.


\subsection{String Objects \label{stringObjects}}

These functions raise \exception{TypeError} when expecting a string
parameter and are called with a non-string parameter.

\obindex{string}
\begin{ctypedesc}{PyStringObject}
  This subtype of \ctype{PyObject} represents a Python string object.
\end{ctypedesc}

\begin{cvardesc}{PyTypeObject}{PyString_Type}
  This instance of \ctype{PyTypeObject} represents the Python string
  type; it is the same object as \code{str} and \code{types.StringType}
  in the Python layer.
  \withsubitem{(in module types)}{\ttindex{StringType}}.
\end{cvardesc}

\begin{cfuncdesc}{int}{PyString_Check}{PyObject *o}
  Return true if the object \var{o} is a string object or an instance
  of a subtype of the string type.
  \versionchanged[Allowed subtypes to be accepted]{2.2}
\end{cfuncdesc}

\begin{cfuncdesc}{int}{PyString_CheckExact}{PyObject *o}
  Return true if the object \var{o} is a string object, but not an
  instance of a subtype of the string type.
  \versionadded{2.2}
\end{cfuncdesc}

\begin{cfuncdesc}{PyObject*}{PyString_FromString}{const char *v}
  Return a new string object with a copy of the string \var{v} as value
  on success, and \NULL{} on failure.  The parameter \var{v} must not be
  \NULL{}; it will not be checked.
\end{cfuncdesc}

\begin{cfuncdesc}{PyObject*}{PyString_FromStringAndSize}{const char *v,
                                                         Py_ssize_t len}
  Return a new string object with a copy of the string \var{v} as value
  and length \var{len} on success, and \NULL{} on failure.  If \var{v} is
  \NULL{}, the contents of the string are uninitialized.
\end{cfuncdesc}

\begin{cfuncdesc}{PyObject*}{PyString_FromFormat}{const char *format, ...}
  Take a C \cfunction{printf()}-style \var{format} string and a
  variable number of arguments, calculate the size of the resulting
  Python string and return a string with the values formatted into
  it.  The variable arguments must be C types and must correspond
  exactly to the format characters in the \var{format} string.  The
  following format characters are allowed:

  % This should be exactly the same as the table in PyErr_Format.
  % One should just refer to the other.

  % The descriptions for %zd and %zu are wrong, but the truth is complicated
  % because not all compilers support the %z width modifier -- we fake it
  % when necessary via interpolating PY_FORMAT_SIZE_T.

  % %u, %lu, %zu should have "new in Python 2.5" blurbs.

  \begin{tableiii}{l|l|l}{member}{Format Characters}{Type}{Comment}
    \lineiii{\%\%}{\emph{n/a}}{The literal \% character.}
    \lineiii{\%c}{int}{A single character, represented as an C int.}
    \lineiii{\%d}{int}{Exactly equivalent to \code{printf("\%d")}.}
    \lineiii{\%u}{unsigned int}{Exactly equivalent to \code{printf("\%u")}.}
    \lineiii{\%ld}{long}{Exactly equivalent to \code{printf("\%ld")}.}
    \lineiii{\%lu}{unsigned long}{Exactly equivalent to \code{printf("\%lu")}.}
    \lineiii{\%zd}{Py_ssize_t}{Exactly equivalent to \code{printf("\%zd")}.}
    \lineiii{\%zu}{size_t}{Exactly equivalent to \code{printf("\%zu")}.}
    \lineiii{\%i}{int}{Exactly equivalent to \code{printf("\%i")}.}
    \lineiii{\%x}{int}{Exactly equivalent to \code{printf("\%x")}.}
    \lineiii{\%s}{char*}{A null-terminated C character array.}
    \lineiii{\%p}{void*}{The hex representation of a C pointer.
	Mostly equivalent to \code{printf("\%p")} except that it is
	guaranteed to start with the literal \code{0x} regardless of
	what the platform's \code{printf} yields.}
  \end{tableiii}

  An unrecognized format character causes all the rest of the format
  string to be copied as-is to the result string, and any extra
  arguments discarded.
\end{cfuncdesc}

\begin{cfuncdesc}{PyObject*}{PyString_FromFormatV}{const char *format,
                                                   va_list vargs}
  Identical to \function{PyString_FromFormat()} except that it takes
  exactly two arguments.
\end{cfuncdesc}

\begin{cfuncdesc}{Py_ssize_t}{PyString_Size}{PyObject *string}
  Return the length of the string in string object \var{string}.
\end{cfuncdesc}

\begin{cfuncdesc}{Py_ssize_t}{PyString_GET_SIZE}{PyObject *string}
  Macro form of \cfunction{PyString_Size()} but without error
  checking.
\end{cfuncdesc}

\begin{cfuncdesc}{char*}{PyString_AsString}{PyObject *string}
  Return a NUL-terminated representation of the contents of
  \var{string}.  The pointer refers to the internal buffer of
  \var{string}, not a copy.  The data must not be modified in any way,
  unless the string was just created using
  \code{PyString_FromStringAndSize(NULL, \var{size})}.
  It must not be deallocated.  If \var{string} is a Unicode object,
  this function computes the default encoding of \var{string} and
  operates on that.  If \var{string} is not a string object at all,
  \cfunction{PyString_AsString()} returns \NULL{} and raises
  \exception{TypeError}.
\end{cfuncdesc}

\begin{cfuncdesc}{char*}{PyString_AS_STRING}{PyObject *string}
  Macro form of \cfunction{PyString_AsString()} but without error
  checking.  Only string objects are supported; no Unicode objects
  should be passed.
\end{cfuncdesc}

\begin{cfuncdesc}{int}{PyString_AsStringAndSize}{PyObject *obj,
                                                 char **buffer,
                                                 Py_ssize_t *length}
  Return a NUL-terminated representation of the contents of the
  object \var{obj} through the output variables \var{buffer} and
  \var{length}.

  The function accepts both string and Unicode objects as input. For
  Unicode objects it returns the default encoded version of the
  object.  If \var{length} is \NULL{}, the resulting buffer may not
  contain NUL characters; if it does, the function returns \code{-1}
  and a \exception{TypeError} is raised.

  The buffer refers to an internal string buffer of \var{obj}, not a
  copy. The data must not be modified in any way, unless the string
  was just created using \code{PyString_FromStringAndSize(NULL,
  \var{size})}.  It must not be deallocated.  If \var{string} is a
  Unicode object, this function computes the default encoding of
  \var{string} and operates on that.  If \var{string} is not a string
  object at all, \cfunction{PyString_AsStringAndSize()} returns
  \code{-1} and raises \exception{TypeError}.
\end{cfuncdesc}

\begin{cfuncdesc}{void}{PyString_Concat}{PyObject **string,
                                         PyObject *newpart}
  Create a new string object in \var{*string} containing the contents
  of \var{newpart} appended to \var{string}; the caller will own the
  new reference.  The reference to the old value of \var{string} will
  be stolen.  If the new string cannot be created, the old reference
  to \var{string} will still be discarded and the value of
  \var{*string} will be set to \NULL{}; the appropriate exception will
  be set.
\end{cfuncdesc}

\begin{cfuncdesc}{void}{PyString_ConcatAndDel}{PyObject **string,
                                               PyObject *newpart}
  Create a new string object in \var{*string} containing the contents
  of \var{newpart} appended to \var{string}.  This version decrements
  the reference count of \var{newpart}.
\end{cfuncdesc}

\begin{cfuncdesc}{int}{_PyString_Resize}{PyObject **string, Py_ssize_t newsize}
  A way to resize a string object even though it is ``immutable''.
  Only use this to build up a brand new string object; don't use this
  if the string may already be known in other parts of the code.  It
  is an error to call this function if the refcount on the input string
  object is not one.
  Pass the address of an existing string object as an lvalue (it may
  be written into), and the new size desired.  On success, \var{*string}
  holds the resized string object and \code{0} is returned; the address in
  \var{*string} may differ from its input value.  If the
  reallocation fails, the original string object at \var{*string} is
  deallocated, \var{*string} is set to \NULL{}, a memory exception is set,
  and \code{-1} is returned.
\end{cfuncdesc}

\begin{cfuncdesc}{PyObject*}{PyString_Format}{PyObject *format,
                                              PyObject *args}
  Return a new string object from \var{format} and \var{args}.
  Analogous to \code{\var{format} \%\ \var{args}}.  The \var{args}
  argument must be a tuple.
\end{cfuncdesc}

\begin{cfuncdesc}{void}{PyString_InternInPlace}{PyObject **string}
  Intern the argument \var{*string} in place.  The argument must be
  the address of a pointer variable pointing to a Python string
  object.  If there is an existing interned string that is the same as
  \var{*string}, it sets \var{*string} to it (decrementing the
  reference count of the old string object and incrementing the
  reference count of the interned string object), otherwise it leaves
  \var{*string} alone and interns it (incrementing its reference
  count).  (Clarification: even though there is a lot of talk about
  reference counts, think of this function as reference-count-neutral;
  you own the object after the call if and only if you owned it before
  the call.)
\end{cfuncdesc}

\begin{cfuncdesc}{PyObject*}{PyString_InternFromString}{const char *v}
  A combination of \cfunction{PyString_FromString()} and
  \cfunction{PyString_InternInPlace()}, returning either a new string
  object that has been interned, or a new (``owned'') reference to an
  earlier interned string object with the same value.
\end{cfuncdesc}

\begin{cfuncdesc}{PyObject*}{PyString_Decode}{const char *s,
                                               Py_ssize_t size,
                                               const char *encoding,
                                               const char *errors}
  Create an object by decoding \var{size} bytes of the encoded
  buffer \var{s} using the codec registered for
  \var{encoding}.  \var{encoding} and \var{errors} have the same
  meaning as the parameters of the same name in the
  \function{unicode()} built-in function.  The codec to be used is
  looked up using the Python codec registry.  Return \NULL{} if
  an exception was raised by the codec.
\end{cfuncdesc}

\begin{cfuncdesc}{PyObject*}{PyString_AsDecodedObject}{PyObject *str,
                                               const char *encoding,
                                               const char *errors}
  Decode a string object by passing it to the codec registered for
  \var{encoding} and return the result as Python
  object. \var{encoding} and \var{errors} have the same meaning as the
  parameters of the same name in the string \method{encode()} method.
  The codec to be used is looked up using the Python codec registry.
  Return \NULL{} if an exception was raised by the codec.
\end{cfuncdesc}

\begin{cfuncdesc}{PyObject*}{PyString_Encode}{const char *s,
                                               Py_ssize_t size,
                                               const char *encoding,
                                               const char *errors}
  Encode the \ctype{char} buffer of the given size by passing it to
  the codec registered for \var{encoding} and return a Python object.
  \var{encoding} and \var{errors} have the same meaning as the
  parameters of the same name in the string \method{encode()} method.
  The codec to be used is looked up using the Python codec
  registry.  Return \NULL{} if an exception was raised by the
  codec.
\end{cfuncdesc}

\begin{cfuncdesc}{PyObject*}{PyString_AsEncodedObject}{PyObject *str,
                                               const char *encoding,
                                               const char *errors}
  Encode a string object using the codec registered for
  \var{encoding} and return the result as Python object.
  \var{encoding} and \var{errors} have the same meaning as the
  parameters of the same name in the string \method{encode()} method.
  The codec to be used is looked up using the Python codec registry.
  Return \NULL{} if an exception was raised by the codec.
\end{cfuncdesc}


\subsection{Unicode Objects \label{unicodeObjects}}
\sectionauthor{Marc-Andre Lemburg}{mal@lemburg.com}

%--- Unicode Type -------------------------------------------------------

These are the basic Unicode object types used for the Unicode
implementation in Python:

\begin{ctypedesc}{Py_UNICODE}
  This type represents the storage type which is used by Python
  internally as basis for holding Unicode ordinals.  Python's default
  builds use a 16-bit type for \ctype{Py_UNICODE} and store Unicode
  values internally as UCS2. It is also possible to build a UCS4
  version of Python (most recent Linux distributions come with UCS4
  builds of Python). These builds then use a 32-bit type for
  \ctype{Py_UNICODE} and store Unicode data internally as UCS4. On
  platforms where \ctype{wchar_t} is available and compatible with the
  chosen Python Unicode build variant, \ctype{Py_UNICODE} is a typedef
  alias for \ctype{wchar_t} to enhance native platform compatibility.
  On all other platforms, \ctype{Py_UNICODE} is a typedef alias for
  either \ctype{unsigned short} (UCS2) or \ctype{unsigned long}
  (UCS4).
\end{ctypedesc}

Note that UCS2 and UCS4 Python builds are not binary compatible.
Please keep this in mind when writing extensions or interfaces.

\begin{ctypedesc}{PyUnicodeObject}
  This subtype of \ctype{PyObject} represents a Python Unicode object.
\end{ctypedesc}

\begin{cvardesc}{PyTypeObject}{PyUnicode_Type}
  This instance of \ctype{PyTypeObject} represents the Python Unicode
  type.  It is exposed to Python code as \code{unicode} and
  \code{types.UnicodeType}.
\end{cvardesc}

The following APIs are really C macros and can be used to do fast
checks and to access internal read-only data of Unicode objects:

\begin{cfuncdesc}{int}{PyUnicode_Check}{PyObject *o}
  Return true if the object \var{o} is a Unicode object or an
  instance of a Unicode subtype.
  \versionchanged[Allowed subtypes to be accepted]{2.2}
\end{cfuncdesc}

\begin{cfuncdesc}{int}{PyUnicode_CheckExact}{PyObject *o}
  Return true if the object \var{o} is a Unicode object, but not an
  instance of a subtype.
  \versionadded{2.2}
\end{cfuncdesc}

\begin{cfuncdesc}{Py_ssize_t}{PyUnicode_GET_SIZE}{PyObject *o}
  Return the size of the object.  \var{o} has to be a
  \ctype{PyUnicodeObject} (not checked).
\end{cfuncdesc}

\begin{cfuncdesc}{Py_ssize_t}{PyUnicode_GET_DATA_SIZE}{PyObject *o}
  Return the size of the object's internal buffer in bytes.  \var{o}
  has to be a \ctype{PyUnicodeObject} (not checked).
\end{cfuncdesc}

\begin{cfuncdesc}{Py_UNICODE*}{PyUnicode_AS_UNICODE}{PyObject *o}
  Return a pointer to the internal \ctype{Py_UNICODE} buffer of the
  object.  \var{o} has to be a \ctype{PyUnicodeObject} (not checked).
\end{cfuncdesc}

\begin{cfuncdesc}{const char*}{PyUnicode_AS_DATA}{PyObject *o}
  Return a pointer to the internal buffer of the object.
  \var{o} has to be a \ctype{PyUnicodeObject} (not checked).
\end{cfuncdesc}

% --- Unicode character properties ---------------------------------------

Unicode provides many different character properties. The most often
needed ones are available through these macros which are mapped to C
functions depending on the Python configuration.

\begin{cfuncdesc}{int}{Py_UNICODE_ISSPACE}{Py_UNICODE ch}
  Return 1 or 0 depending on whether \var{ch} is a whitespace
  character.
\end{cfuncdesc}

\begin{cfuncdesc}{int}{Py_UNICODE_ISLOWER}{Py_UNICODE ch}
  Return 1 or 0 depending on whether \var{ch} is a lowercase character.
\end{cfuncdesc}

\begin{cfuncdesc}{int}{Py_UNICODE_ISUPPER}{Py_UNICODE ch}
  Return 1 or 0 depending on whether \var{ch} is an uppercase
  character.
\end{cfuncdesc}

\begin{cfuncdesc}{int}{Py_UNICODE_ISTITLE}{Py_UNICODE ch}
  Return 1 or 0 depending on whether \var{ch} is a titlecase character.
\end{cfuncdesc}

\begin{cfuncdesc}{int}{Py_UNICODE_ISLINEBREAK}{Py_UNICODE ch}
  Return 1 or 0 depending on whether \var{ch} is a linebreak character.
\end{cfuncdesc}

\begin{cfuncdesc}{int}{Py_UNICODE_ISDECIMAL}{Py_UNICODE ch}
  Return 1 or 0 depending on whether \var{ch} is a decimal character.
\end{cfuncdesc}

\begin{cfuncdesc}{int}{Py_UNICODE_ISDIGIT}{Py_UNICODE ch}
  Return 1 or 0 depending on whether \var{ch} is a digit character.
\end{cfuncdesc}

\begin{cfuncdesc}{int}{Py_UNICODE_ISNUMERIC}{Py_UNICODE ch}
  Return 1 or 0 depending on whether \var{ch} is a numeric character.
\end{cfuncdesc}

\begin{cfuncdesc}{int}{Py_UNICODE_ISALPHA}{Py_UNICODE ch}
  Return 1 or 0 depending on whether \var{ch} is an alphabetic
  character.
\end{cfuncdesc}

\begin{cfuncdesc}{int}{Py_UNICODE_ISALNUM}{Py_UNICODE ch}
  Return 1 or 0 depending on whether \var{ch} is an alphanumeric
  character.
\end{cfuncdesc}

These APIs can be used for fast direct character conversions:

\begin{cfuncdesc}{Py_UNICODE}{Py_UNICODE_TOLOWER}{Py_UNICODE ch}
  Return the character \var{ch} converted to lower case.
\end{cfuncdesc}

\begin{cfuncdesc}{Py_UNICODE}{Py_UNICODE_TOUPPER}{Py_UNICODE ch}
  Return the character \var{ch} converted to upper case.
\end{cfuncdesc}

\begin{cfuncdesc}{Py_UNICODE}{Py_UNICODE_TOTITLE}{Py_UNICODE ch}
  Return the character \var{ch} converted to title case.
\end{cfuncdesc}

\begin{cfuncdesc}{int}{Py_UNICODE_TODECIMAL}{Py_UNICODE ch}
  Return the character \var{ch} converted to a decimal positive
  integer.  Return \code{-1} if this is not possible.  This macro
  does not raise exceptions.
\end{cfuncdesc}

\begin{cfuncdesc}{int}{Py_UNICODE_TODIGIT}{Py_UNICODE ch}
  Return the character \var{ch} converted to a single digit integer.
  Return \code{-1} if this is not possible.  This macro does not raise
  exceptions.
\end{cfuncdesc}

\begin{cfuncdesc}{double}{Py_UNICODE_TONUMERIC}{Py_UNICODE ch}
  Return the character \var{ch} converted to a double.
  Return \code{-1.0} if this is not possible.  This macro does not raise
  exceptions.
\end{cfuncdesc}

% --- Plain Py_UNICODE ---------------------------------------------------

To create Unicode objects and access their basic sequence properties,
use these APIs:

\begin{cfuncdesc}{PyObject*}{PyUnicode_FromUnicode}{const Py_UNICODE *u,
                                                    Py_ssize_t size}
  Create a Unicode Object from the Py_UNICODE buffer \var{u} of the
  given size. \var{u} may be \NULL{} which causes the contents to be
  undefined. It is the user's responsibility to fill in the needed
  data.  The buffer is copied into the new object. If the buffer is
  not \NULL{}, the return value might be a shared object. Therefore,
  modification of the resulting Unicode object is only allowed when
  \var{u} is \NULL{}.
\end{cfuncdesc}

\begin{cfuncdesc}{PyObject*}{PyUnicode_FromStringAndSize}{const char *u,
                                                       Py_ssize_t size}
  Create a Unicode Object from the char buffer \var{u}.
  The bytes will be interpreted as being UTF-8 encoded. 
  \var{u} may also be \NULL{} which causes the
  contents to be undefined. It is the user's responsibility to fill
  in the needed data.  The buffer is copied into the new object.
  If the buffer is not \NULL{}, the return value might be a shared object.
  Therefore, modification of the resulting Unicode object is only allowed
  when \var{u} is \NULL{}.
  \versionadded{3.0}
\end{cfuncdesc}

\begin{cfuncdesc}{PyObject*}{PyUnicode_FromString}{const char*u}
   Create a Unicode object from an UTF-8 encoded null-terminated
   char buffer \var{u}.
   \versionadded{3.0}
\end{funcdesc}

\begin{cfuncdesc}{PyObject*}{PyUnicode_FromFormat}{const char *format, ...}
  Take a C \cfunction{printf()}-style \var{format} string and a
  variable number of arguments, calculate the size of the resulting
  Python unicode string and return a string with the values formatted into
  it.  The variable arguments must be C types and must correspond
  exactly to the format characters in the \var{format} string.  The
  following format characters are allowed:

  % The descriptions for %zd and %zu are wrong, but the truth is complicated
  % because not all compilers support the %z width modifier -- we fake it
  % when necessary via interpolating PY_FORMAT_SIZE_T.

  \begin{tableiii}{l|l|l}{member}{Format Characters}{Type}{Comment}
    \lineiii{\%\%}{\emph{n/a}}{The literal \% character.}
    \lineiii{\%c}{int}{A single character, represented as an C int.}
    \lineiii{\%d}{int}{Exactly equivalent to \code{printf("\%d")}.}
    \lineiii{\%u}{unsigned int}{Exactly equivalent to \code{printf("\%u")}.}
    \lineiii{\%ld}{long}{Exactly equivalent to \code{printf("\%ld")}.}
    \lineiii{\%lu}{unsigned long}{Exactly equivalent to \code{printf("\%lu")}.}
    \lineiii{\%zd}{Py_ssize_t}{Exactly equivalent to \code{printf("\%zd")}.}
    \lineiii{\%zu}{size_t}{Exactly equivalent to \code{printf("\%zu")}.}
    \lineiii{\%i}{int}{Exactly equivalent to \code{printf("\%i")}.}
    \lineiii{\%x}{int}{Exactly equivalent to \code{printf("\%x")}.}
    \lineiii{\%s}{char*}{A null-terminated C character array.}
    \lineiii{\%p}{void*}{The hex representation of a C pointer.
	Mostly equivalent to \code{printf("\%p")} except that it is
	guaranteed to start with the literal \code{0x} regardless of
	what the platform's \code{printf} yields.}
    \lineiii{\%U}{PyObject*}{A unicode object.}
    \lineiii{\%V}{PyObject*, char *}{A unicode object (which may be \NULL{})
	and a null-terminated C character array as a second parameter (which
	will be used, if the first parameter is \NULL{}).}
    \lineiii{\%S}{PyObject*}{The result of calling \function{PyObject_Unicode()}.}
    \lineiii{\%R}{PyObject*}{The result of calling \function{PyObject_Repr()}.}
  \end{tableiii}

  An unrecognized format character causes all the rest of the format
  string to be copied as-is to the result string, and any extra
  arguments discarded.
  \versionadded{3.0}
\end{cfuncdesc}

\begin{cfuncdesc}{PyObject*}{PyUnicode_FromFormatV}{const char *format,
                                                   va_list vargs}
  Identical to \function{PyUnicode_FromFormat()} except that it takes
  exactly two arguments.
  \versionadded{3.0}
\end{cfuncdesc}

\begin{cfuncdesc}{Py_UNICODE*}{PyUnicode_AsUnicode}{PyObject *unicode}
  Return a read-only pointer to the Unicode object's internal
  \ctype{Py_UNICODE} buffer, \NULL{} if \var{unicode} is not a Unicode
  object.
\end{cfuncdesc}

\begin{cfuncdesc}{Py_ssize_t}{PyUnicode_GetSize}{PyObject *unicode}
  Return the length of the Unicode object.
\end{cfuncdesc}

\begin{cfuncdesc}{PyObject*}{PyUnicode_FromEncodedObject}{PyObject *obj,
                                                      const char *encoding,
                                                      const char *errors}
  Coerce an encoded object \var{obj} to an Unicode object and return a
  reference with incremented refcount.
  
  String and other char buffer compatible objects are decoded
  according to the given encoding and using the error handling
  defined by errors.  Both can be \NULL{} to have the interface
  use the default values (see the next section for details).

  All other objects, including Unicode objects, cause a
  \exception{TypeError} to be set.

  The API returns \NULL{} if there was an error.  The caller is
  responsible for decref'ing the returned objects.
\end{cfuncdesc}

\begin{cfuncdesc}{PyObject*}{PyUnicode_FromObject}{PyObject *obj}
  Shortcut for \code{PyUnicode_FromEncodedObject(obj, NULL, "strict")}
  which is used throughout the interpreter whenever coercion to
  Unicode is needed.
\end{cfuncdesc}

% --- wchar_t support for platforms which support it ---------------------

If the platform supports \ctype{wchar_t} and provides a header file
wchar.h, Python can interface directly to this type using the
following functions. Support is optimized if Python's own
\ctype{Py_UNICODE} type is identical to the system's \ctype{wchar_t}.

\begin{cfuncdesc}{PyObject*}{PyUnicode_FromWideChar}{const wchar_t *w,
                                                     Py_ssize_t size}
  Create a Unicode object from the \ctype{wchar_t} buffer \var{w} of
  the given size.  Return \NULL{} on failure.
\end{cfuncdesc}

\begin{cfuncdesc}{Py_ssize_t}{PyUnicode_AsWideChar}{PyUnicodeObject *unicode,
                                             wchar_t *w,
                                             Py_ssize_t size}
  Copy the Unicode object contents into the \ctype{wchar_t} buffer
  \var{w}.  At most \var{size} \ctype{wchar_t} characters are copied
  (excluding a possibly trailing 0-termination character).  Return
  the number of \ctype{wchar_t} characters copied or -1 in case of an
  error.  Note that the resulting \ctype{wchar_t} string may or may
  not be 0-terminated.  It is the responsibility of the caller to make
  sure that the \ctype{wchar_t} string is 0-terminated in case this is
  required by the application.
\end{cfuncdesc}


\subsubsection{Built-in Codecs \label{builtinCodecs}}

Python provides a set of builtin codecs which are written in C
for speed. All of these codecs are directly usable via the
following functions.

Many of the following APIs take two arguments encoding and
errors. These parameters encoding and errors have the same semantics
as the ones of the builtin unicode() Unicode object constructor.

Setting encoding to \NULL{} causes the default encoding to be used
which is \ASCII.  The file system calls should use
\cdata{Py_FileSystemDefaultEncoding} as the encoding for file
names. This variable should be treated as read-only: On some systems,
it will be a pointer to a static string, on others, it will change at
run-time (such as when the application invokes setlocale).

Error handling is set by errors which may also be set to \NULL{}
meaning to use the default handling defined for the codec.  Default
error handling for all builtin codecs is ``strict''
(\exception{ValueError} is raised).

The codecs all use a similar interface.  Only deviation from the
following generic ones are documented for simplicity.

% --- Generic Codecs -----------------------------------------------------

These are the generic codec APIs:

\begin{cfuncdesc}{PyObject*}{PyUnicode_Decode}{const char *s,
                                               Py_ssize_t size,
                                               const char *encoding,
                                               const char *errors}
  Create a Unicode object by decoding \var{size} bytes of the encoded
  string \var{s}.  \var{encoding} and \var{errors} have the same
  meaning as the parameters of the same name in the
  \function{unicode()} builtin function.  The codec to be used is
  looked up using the Python codec registry.  Return \NULL{} if an
  exception was raised by the codec.
\end{cfuncdesc}

\begin{cfuncdesc}{PyObject*}{PyUnicode_Encode}{const Py_UNICODE *s,
                                               Py_ssize_t size,
                                               const char *encoding,
                                               const char *errors}
  Encode the \ctype{Py_UNICODE} buffer of the given size and return
  a Python string object.  \var{encoding} and \var{errors} have the
  same meaning as the parameters of the same name in the Unicode
  \method{encode()} method.  The codec to be used is looked up using
  the Python codec registry.  Return \NULL{} if an exception was
  raised by the codec.
\end{cfuncdesc}

\begin{cfuncdesc}{PyObject*}{PyUnicode_AsEncodedString}{PyObject *unicode,
                                               const char *encoding,
                                               const char *errors}
  Encode a Unicode object and return the result as Python string
  object. \var{encoding} and \var{errors} have the same meaning as the
  parameters of the same name in the Unicode \method{encode()} method.
  The codec to be used is looked up using the Python codec registry.
  Return \NULL{} if an exception was raised by the codec.
\end{cfuncdesc}

% --- UTF-8 Codecs -------------------------------------------------------

These are the UTF-8 codec APIs:

\begin{cfuncdesc}{PyObject*}{PyUnicode_DecodeUTF8}{const char *s,
                                               Py_ssize_t size,
                                               const char *errors}
  Create a Unicode object by decoding \var{size} bytes of the UTF-8
  encoded string \var{s}. Return \NULL{} if an exception was raised
  by the codec.
\end{cfuncdesc}

\begin{cfuncdesc}{PyObject*}{PyUnicode_DecodeUTF8Stateful}{const char *s,
                                               Py_ssize_t size,
                                               const char *errors,
                                               Py_ssize_t *consumed}
  If \var{consumed} is \NULL{}, behave like \cfunction{PyUnicode_DecodeUTF8()}.
  If \var{consumed} is not \NULL{}, trailing incomplete UTF-8 byte sequences
  will not be treated as an error. Those bytes will not be decoded and the
  number of bytes that have been decoded will be stored in \var{consumed}.
  \versionadded{2.4}
\end{cfuncdesc}

\begin{cfuncdesc}{PyObject*}{PyUnicode_EncodeUTF8}{const Py_UNICODE *s,
                                               Py_ssize_t size,
                                               const char *errors}
  Encode the \ctype{Py_UNICODE} buffer of the given size using UTF-8
  and return a Python string object.  Return \NULL{} if an exception
  was raised by the codec.
\end{cfuncdesc}

\begin{cfuncdesc}{PyObject*}{PyUnicode_AsUTF8String}{PyObject *unicode}
  Encode a Unicode objects using UTF-8 and return the result as
  Python string object.  Error handling is ``strict''.  Return
  \NULL{} if an exception was raised by the codec.
\end{cfuncdesc}

% --- UTF-16 Codecs ------------------------------------------------------ */

These are the UTF-16 codec APIs:

\begin{cfuncdesc}{PyObject*}{PyUnicode_DecodeUTF16}{const char *s,
                                               Py_ssize_t size,
                                               const char *errors,
                                               int *byteorder}
  Decode \var{length} bytes from a UTF-16 encoded buffer string and
  return the corresponding Unicode object.  \var{errors} (if
  non-\NULL{}) defines the error handling. It defaults to ``strict''.

  If \var{byteorder} is non-\NULL{}, the decoder starts decoding using
  the given byte order:

\begin{verbatim}
   *byteorder == -1: little endian
   *byteorder == 0:  native order
   *byteorder == 1:  big endian
\end{verbatim}

  and then switches if the first two bytes of the input data are a byte order
  mark (BOM) and the specified byte order is native order.  This BOM is not
  copied into the resulting Unicode string.  After completion, \var{*byteorder}
  is set to the current byte order at the.

  If \var{byteorder} is \NULL{}, the codec starts in native order mode.

  Return \NULL{} if an exception was raised by the codec.
\end{cfuncdesc}

\begin{cfuncdesc}{PyObject*}{PyUnicode_DecodeUTF16Stateful}{const char *s,
                                               Py_ssize_t size,
                                               const char *errors,
                                               int *byteorder,
                                               Py_ssize_t *consumed}
  If \var{consumed} is \NULL{}, behave like
  \cfunction{PyUnicode_DecodeUTF16()}. If \var{consumed} is not \NULL{},
  \cfunction{PyUnicode_DecodeUTF16Stateful()} will not treat trailing incomplete
  UTF-16 byte sequences (such as an odd number of bytes or a split surrogate pair)
  as an error. Those bytes will not be decoded and the number of bytes that
  have been decoded will be stored in \var{consumed}.
  \versionadded{2.4}
\end{cfuncdesc}

\begin{cfuncdesc}{PyObject*}{PyUnicode_EncodeUTF16}{const Py_UNICODE *s,
                                               Py_ssize_t size,
                                               const char *errors,
                                               int byteorder}
  Return a Python string object holding the UTF-16 encoded value of
  the Unicode data in \var{s}.  If \var{byteorder} is not \code{0},
  output is written according to the following byte order:

\begin{verbatim}
   byteorder == -1: little endian
   byteorder == 0:  native byte order (writes a BOM mark)
   byteorder == 1:  big endian
\end{verbatim}

  If byteorder is \code{0}, the output string will always start with
  the Unicode BOM mark (U+FEFF). In the other two modes, no BOM mark
  is prepended.

  If \var{Py_UNICODE_WIDE} is defined, a single \ctype{Py_UNICODE}
  value may get represented as a surrogate pair. If it is not
  defined, each \ctype{Py_UNICODE} values is interpreted as an
  UCS-2 character.

  Return \NULL{} if an exception was raised by the codec.
\end{cfuncdesc}

\begin{cfuncdesc}{PyObject*}{PyUnicode_AsUTF16String}{PyObject *unicode}
  Return a Python string using the UTF-16 encoding in native byte
  order. The string always starts with a BOM mark.  Error handling is
  ``strict''.  Return \NULL{} if an exception was raised by the
  codec.
\end{cfuncdesc}

% --- Unicode-Escape Codecs ----------------------------------------------

These are the ``Unicode Escape'' codec APIs:

\begin{cfuncdesc}{PyObject*}{PyUnicode_DecodeUnicodeEscape}{const char *s,
                                               Py_ssize_t size,
                                               const char *errors}
  Create a Unicode object by decoding \var{size} bytes of the
  Unicode-Escape encoded string \var{s}.  Return \NULL{} if an
  exception was raised by the codec.
\end{cfuncdesc}

\begin{cfuncdesc}{PyObject*}{PyUnicode_EncodeUnicodeEscape}{const Py_UNICODE *s,
                                               Py_ssize_t size}
  Encode the \ctype{Py_UNICODE} buffer of the given size using
  Unicode-Escape and return a Python string object.  Return \NULL{}
  if an exception was raised by the codec.
\end{cfuncdesc}

\begin{cfuncdesc}{PyObject*}{PyUnicode_AsUnicodeEscapeString}{PyObject *unicode}
  Encode a Unicode objects using Unicode-Escape and return the
  result as Python string object.  Error handling is ``strict''.
  Return \NULL{} if an exception was raised by the codec.
\end{cfuncdesc}

% --- Raw-Unicode-Escape Codecs ------------------------------------------

These are the ``Raw Unicode Escape'' codec APIs:

\begin{cfuncdesc}{PyObject*}{PyUnicode_DecodeRawUnicodeEscape}{const char *s,
                                               Py_ssize_t size,
                                               const char *errors}
  Create a Unicode object by decoding \var{size} bytes of the
  Raw-Unicode-Escape encoded string \var{s}.  Return \NULL{} if an
  exception was raised by the codec.
\end{cfuncdesc}

\begin{cfuncdesc}{PyObject*}{PyUnicode_EncodeRawUnicodeEscape}{const Py_UNICODE *s,
                                               Py_ssize_t size,
                                               const char *errors}
  Encode the \ctype{Py_UNICODE} buffer of the given size using
  Raw-Unicode-Escape and return a Python string object.  Return
  \NULL{} if an exception was raised by the codec.
\end{cfuncdesc}

\begin{cfuncdesc}{PyObject*}{PyUnicode_AsRawUnicodeEscapeString}{PyObject *unicode}
  Encode a Unicode objects using Raw-Unicode-Escape and return the
  result as Python string object. Error handling is ``strict''.
  Return \NULL{} if an exception was raised by the codec.
\end{cfuncdesc}

% --- Latin-1 Codecs -----------------------------------------------------

These are the Latin-1 codec APIs:
Latin-1 corresponds to the first 256 Unicode ordinals and only these
are accepted by the codecs during encoding.

\begin{cfuncdesc}{PyObject*}{PyUnicode_DecodeLatin1}{const char *s,
                                                     Py_ssize_t size,
                                                     const char *errors}
  Create a Unicode object by decoding \var{size} bytes of the Latin-1
  encoded string \var{s}.  Return \NULL{} if an exception was raised
  by the codec.
\end{cfuncdesc}

\begin{cfuncdesc}{PyObject*}{PyUnicode_EncodeLatin1}{const Py_UNICODE *s,
                                                     Py_ssize_t size,
                                                     const char *errors}
  Encode the \ctype{Py_UNICODE} buffer of the given size using
  Latin-1 and return a Python string object.  Return \NULL{} if an
  exception was raised by the codec.
\end{cfuncdesc}

\begin{cfuncdesc}{PyObject*}{PyUnicode_AsLatin1String}{PyObject *unicode}
  Encode a Unicode objects using Latin-1 and return the result as
  Python string object.  Error handling is ``strict''.  Return
  \NULL{} if an exception was raised by the codec.
\end{cfuncdesc}

% --- ASCII Codecs -------------------------------------------------------

These are the \ASCII{} codec APIs.  Only 7-bit \ASCII{} data is
accepted. All other codes generate errors.

\begin{cfuncdesc}{PyObject*}{PyUnicode_DecodeASCII}{const char *s,
                                                    Py_ssize_t size,
                                                    const char *errors}
  Create a Unicode object by decoding \var{size} bytes of the
  \ASCII{} encoded string \var{s}.  Return \NULL{} if an exception
  was raised by the codec.
\end{cfuncdesc}

\begin{cfuncdesc}{PyObject*}{PyUnicode_EncodeASCII}{const Py_UNICODE *s,
                                                    Py_ssize_t size,
                                                    const char *errors}
  Encode the \ctype{Py_UNICODE} buffer of the given size using
  \ASCII{} and return a Python string object.  Return \NULL{} if an
  exception was raised by the codec.
\end{cfuncdesc}

\begin{cfuncdesc}{PyObject*}{PyUnicode_AsASCIIString}{PyObject *unicode}
  Encode a Unicode objects using \ASCII{} and return the result as
  Python string object.  Error handling is ``strict''.  Return
  \NULL{} if an exception was raised by the codec.
\end{cfuncdesc}

% --- Character Map Codecs -----------------------------------------------

These are the mapping codec APIs:

This codec is special in that it can be used to implement many
different codecs (and this is in fact what was done to obtain most of
the standard codecs included in the \module{encodings} package). The
codec uses mapping to encode and decode characters.

Decoding mappings must map single string characters to single Unicode
characters, integers (which are then interpreted as Unicode ordinals)
or None (meaning "undefined mapping" and causing an error).

Encoding mappings must map single Unicode characters to single string
characters, integers (which are then interpreted as Latin-1 ordinals)
or None (meaning "undefined mapping" and causing an error).

The mapping objects provided must only support the __getitem__ mapping
interface.

If a character lookup fails with a LookupError, the character is
copied as-is meaning that its ordinal value will be interpreted as
Unicode or Latin-1 ordinal resp. Because of this, mappings only need
to contain those mappings which map characters to different code
points.

\begin{cfuncdesc}{PyObject*}{PyUnicode_DecodeCharmap}{const char *s,
                                               Py_ssize_t size,
                                               PyObject *mapping,
                                               const char *errors}
  Create a Unicode object by decoding \var{size} bytes of the encoded
  string \var{s} using the given \var{mapping} object.  Return
  \NULL{} if an exception was raised by the codec. If \var{mapping} is \NULL{}
  latin-1 decoding will be done. Else it can be a dictionary mapping byte or a
  unicode string, which is treated as a lookup table. Byte values greater
  that the length of the string and U+FFFE "characters" are treated as
  "undefined mapping".
  \versionchanged[Allowed unicode string as mapping argument]{2.4}
\end{cfuncdesc}

\begin{cfuncdesc}{PyObject*}{PyUnicode_EncodeCharmap}{const Py_UNICODE *s,
                                               Py_ssize_t size,
                                               PyObject *mapping,
                                               const char *errors}
  Encode the \ctype{Py_UNICODE} buffer of the given size using the
  given \var{mapping} object and return a Python string object.
  Return \NULL{} if an exception was raised by the codec.
\end{cfuncdesc}

\begin{cfuncdesc}{PyObject*}{PyUnicode_AsCharmapString}{PyObject *unicode,
                                                        PyObject *mapping}
  Encode a Unicode objects using the given \var{mapping} object and
  return the result as Python string object.  Error handling is
  ``strict''.  Return \NULL{} if an exception was raised by the
  codec.
\end{cfuncdesc}

The following codec API is special in that maps Unicode to Unicode.

\begin{cfuncdesc}{PyObject*}{PyUnicode_TranslateCharmap}{const Py_UNICODE *s,
                                               Py_ssize_t size,
                                               PyObject *table,
                                               const char *errors}
  Translate a \ctype{Py_UNICODE} buffer of the given length by
  applying a character mapping \var{table} to it and return the
  resulting Unicode object.  Return \NULL{} when an exception was
  raised by the codec.

  The \var{mapping} table must map Unicode ordinal integers to Unicode
  ordinal integers or None (causing deletion of the character).

  Mapping tables need only provide the \method{__getitem__()}
  interface; dictionaries and sequences work well.  Unmapped character
  ordinals (ones which cause a \exception{LookupError}) are left
  untouched and are copied as-is.
\end{cfuncdesc}

% --- MBCS codecs for Windows --------------------------------------------

These are the MBCS codec APIs. They are currently only available on
Windows and use the Win32 MBCS converters to implement the
conversions.  Note that MBCS (or DBCS) is a class of encodings, not
just one.  The target encoding is defined by the user settings on the
machine running the codec.

\begin{cfuncdesc}{PyObject*}{PyUnicode_DecodeMBCS}{const char *s,
                                               Py_ssize_t size,
                                               const char *errors}
  Create a Unicode object by decoding \var{size} bytes of the MBCS
  encoded string \var{s}.  Return \NULL{} if an exception was
  raised by the codec.
\end{cfuncdesc}

\begin{cfuncdesc}{PyObject*}{PyUnicode_DecodeMBCSStateful}{const char *s,
                                               int size,
                                               const char *errors,
                                               int *consumed}
  If \var{consumed} is \NULL{}, behave like
  \cfunction{PyUnicode_DecodeMBCS()}. If \var{consumed} is not \NULL{},
  \cfunction{PyUnicode_DecodeMBCSStateful()} will not decode trailing lead
  byte and the number of bytes that have been decoded will be stored in
  \var{consumed}.
  \versionadded{2.5}
\end{cfuncdesc}

\begin{cfuncdesc}{PyObject*}{PyUnicode_EncodeMBCS}{const Py_UNICODE *s,
                                               Py_ssize_t size,
                                               const char *errors}
  Encode the \ctype{Py_UNICODE} buffer of the given size using MBCS
  and return a Python string object.  Return \NULL{} if an exception
  was raised by the codec.
\end{cfuncdesc}

\begin{cfuncdesc}{PyObject*}{PyUnicode_AsMBCSString}{PyObject *unicode}
  Encode a Unicode objects using MBCS and return the result as
  Python string object.  Error handling is ``strict''.  Return
  \NULL{} if an exception was raised by the codec.
\end{cfuncdesc}

% --- Methods & Slots ----------------------------------------------------

\subsubsection{Methods and Slot Functions \label{unicodeMethodsAndSlots}}

The following APIs are capable of handling Unicode objects and strings
on input (we refer to them as strings in the descriptions) and return
Unicode objects or integers as appropriate.

They all return \NULL{} or \code{-1} if an exception occurs.

\begin{cfuncdesc}{PyObject*}{PyUnicode_Concat}{PyObject *left,
                                               PyObject *right}
  Concat two strings giving a new Unicode string.
\end{cfuncdesc}

\begin{cfuncdesc}{PyObject*}{PyUnicode_Split}{PyObject *s,
                                              PyObject *sep,
                                              Py_ssize_t maxsplit}
  Split a string giving a list of Unicode strings.  If sep is \NULL{},
  splitting will be done at all whitespace substrings.  Otherwise,
  splits occur at the given separator.  At most \var{maxsplit} splits
  will be done.  If negative, no limit is set.  Separators are not
  included in the resulting list.
\end{cfuncdesc}

\begin{cfuncdesc}{PyObject*}{PyUnicode_Splitlines}{PyObject *s,
                                                   int keepend}
  Split a Unicode string at line breaks, returning a list of Unicode
  strings.  CRLF is considered to be one line break.  If \var{keepend}
  is 0, the Line break characters are not included in the resulting
  strings.
\end{cfuncdesc}

\begin{cfuncdesc}{PyObject*}{PyUnicode_Translate}{PyObject *str,
                                                  PyObject *table,
                                                  const char *errors}
  Translate a string by applying a character mapping table to it and
  return the resulting Unicode object.

  The mapping table must map Unicode ordinal integers to Unicode
  ordinal integers or None (causing deletion of the character).

  Mapping tables need only provide the \method{__getitem__()}
  interface; dictionaries and sequences work well.  Unmapped character
  ordinals (ones which cause a \exception{LookupError}) are left
  untouched and are copied as-is.

  \var{errors} has the usual meaning for codecs. It may be \NULL{}
  which indicates to use the default error handling.
\end{cfuncdesc}

\begin{cfuncdesc}{PyObject*}{PyUnicode_Join}{PyObject *separator,
                                             PyObject *seq}
  Join a sequence of strings using the given separator and return the
  resulting Unicode string.
\end{cfuncdesc}

\begin{cfuncdesc}{int}{PyUnicode_Tailmatch}{PyObject *str,
                                                  PyObject *substr,
                                                  Py_ssize_t start,
                                                  Py_ssize_t end,
                                                  int direction}
  Return 1 if \var{substr} matches \var{str}[\var{start}:\var{end}] at
  the given tail end (\var{direction} == -1 means to do a prefix
  match, \var{direction} == 1 a suffix match), 0 otherwise.
  Return \code{-1} if an error occurred.
\end{cfuncdesc}

\begin{cfuncdesc}{Py_ssize_t}{PyUnicode_Find}{PyObject *str,
                                       PyObject *substr,
                                       Py_ssize_t start,
                                       Py_ssize_t end,
                                       int direction}
  Return the first position of \var{substr} in
  \var{str}[\var{start}:\var{end}] using the given \var{direction}
  (\var{direction} == 1 means to do a forward search,
  \var{direction} == -1 a backward search).  The return value is the
  index of the first match; a value of \code{-1} indicates that no
  match was found, and \code{-2} indicates that an error occurred and
  an exception has been set.
\end{cfuncdesc}

\begin{cfuncdesc}{Py_ssize_t}{PyUnicode_Count}{PyObject *str,
                                        PyObject *substr,
                                        Py_ssize_t start,
                                        Py_ssize_t end}
  Return the number of non-overlapping occurrences of \var{substr} in
  \code{\var{str}[\var{start}:\var{end}]}.  Return \code{-1} if an
  error occurred.
\end{cfuncdesc}

\begin{cfuncdesc}{PyObject*}{PyUnicode_Replace}{PyObject *str,
                                                PyObject *substr,
                                                PyObject *replstr,
                                                Py_ssize_t maxcount}
  Replace at most \var{maxcount} occurrences of \var{substr} in
  \var{str} with \var{replstr} and return the resulting Unicode object.
  \var{maxcount} == -1 means replace all occurrences.
\end{cfuncdesc}

\begin{cfuncdesc}{int}{PyUnicode_Compare}{PyObject *left, PyObject *right}
  Compare two strings and return -1, 0, 1 for less than, equal, and
  greater than, respectively.
\end{cfuncdesc}

\begin{cfuncdesc}{int}{PyUnicode_RichCompare}{PyObject *left, 
                                              PyObject *right, 
                                              int op}

  Rich compare two unicode strings and return one of the following:
  \begin{itemize}
    \item \code{NULL} in case an exception was raised
    \item \constant{Py_True} or \constant{Py_False} for successful comparisons
    \item \constant{Py_NotImplemented} in case the type combination is unknown
  \end{itemize}

   Note that \constant{Py_EQ} and \constant{Py_NE} comparisons can cause a
   \exception{UnicodeWarning} in case the conversion of the arguments to
   Unicode fails with a \exception{UnicodeDecodeError}.

   Possible values for \var{op} are
   \constant{Py_GT}, \constant{Py_GE}, \constant{Py_EQ},
   \constant{Py_NE}, \constant{Py_LT}, and \constant{Py_LE}.
\end{cfuncdesc}

\begin{cfuncdesc}{PyObject*}{PyUnicode_Format}{PyObject *format,
                                              PyObject *args}
  Return a new string object from \var{format} and \var{args}; this
  is analogous to \code{\var{format} \%\ \var{args}}.  The
  \var{args} argument must be a tuple.
\end{cfuncdesc}

\begin{cfuncdesc}{int}{PyUnicode_Contains}{PyObject *container,
                                           PyObject *element}
  Check whether \var{element} is contained in \var{container} and
  return true or false accordingly.

  \var{element} has to coerce to a one element Unicode
  string. \code{-1} is returned if there was an error.
\end{cfuncdesc}

\begin{cfuncdesc}{void}{PyUnicode_InternInPlace}{PyObject **string}
  Intern the argument \var{*string} in place.  The argument must be
  the address of a pointer variable pointing to a Python unicode string
  object.  If there is an existing interned string that is the same as
  \var{*string}, it sets \var{*string} to it (decrementing the
  reference count of the old string object and incrementing the
  reference count of the interned string object), otherwise it leaves
  \var{*string} alone and interns it (incrementing its reference
  count).  (Clarification: even though there is a lot of talk about
  reference counts, think of this function as reference-count-neutral;
  you own the object after the call if and only if you owned it before
  the call.)
\end{cfuncdesc}

\begin{cfuncdesc}{PyObject*}{PyUnicode_InternFromString}{const char *v}
  A combination of \cfunction{PyUnicode_FromString()} and
  \cfunction{PyUnicode_InternInPlace()}, returning either a new unicode
  string object that has been interned, or a new (``owned'') reference to
  an earlier interned string object with the same value.
\end{cfuncdesc}


\subsection{Buffer Objects \label{bufferObjects}}
\sectionauthor{Greg Stein}{gstein@lyra.org}

\obindex{buffer}
Python objects implemented in C can export a group of functions called
the ``buffer\index{buffer interface} interface.''  These functions can
be used by an object to expose its data in a raw, byte-oriented
format. Clients of the object can use the buffer interface to access
the object data directly, without needing to copy it first.

Two examples of objects that support
the buffer interface are strings and arrays. The string object exposes
the character contents in the buffer interface's byte-oriented
form. An array can also expose its contents, but it should be noted
that array elements may be multi-byte values.

An example user of the buffer interface is the file object's
\method{write()} method. Any object that can export a series of bytes
through the buffer interface can be written to a file. There are a
number of format codes to \cfunction{PyArg_ParseTuple()} that operate
against an object's buffer interface, returning data from the target
object.

More information on the buffer interface is provided in the section
``Buffer Object Structures'' (section~\ref{buffer-structs}), under
the description for \ctype{PyBufferProcs}\ttindex{PyBufferProcs}.

A ``buffer object'' is defined in the \file{bufferobject.h} header
(included by \file{Python.h}). These objects look very similar to
string objects at the Python programming level: they support slicing,
indexing, concatenation, and some other standard string
operations. However, their data can come from one of two sources: from
a block of memory, or from another object which exports the buffer
interface.

Buffer objects are useful as a way to expose the data from another
object's buffer interface to the Python programmer. They can also be
used as a zero-copy slicing mechanism. Using their ability to
reference a block of memory, it is possible to expose any data to the
Python programmer quite easily. The memory could be a large, constant
array in a C extension, it could be a raw block of memory for
manipulation before passing to an operating system library, or it
could be used to pass around structured data in its native, in-memory
format.

\begin{ctypedesc}{PyBufferObject}
  This subtype of \ctype{PyObject} represents a buffer object.
\end{ctypedesc}

\begin{cvardesc}{PyTypeObject}{PyBuffer_Type}
  The instance of \ctype{PyTypeObject} which represents the Python
  buffer type; it is the same object as \code{buffer} and 
  \code{types.BufferType} in the Python layer.
  \withsubitem{(in module types)}{\ttindex{BufferType}}.
\end{cvardesc}

\begin{cvardesc}{int}{Py_END_OF_BUFFER}
  This constant may be passed as the \var{size} parameter to
  \cfunction{PyBuffer_FromObject()} or
  \cfunction{PyBuffer_FromReadWriteObject()}.  It indicates that the
  new \ctype{PyBufferObject} should refer to \var{base} object from
  the specified \var{offset} to the end of its exported buffer.  Using
  this enables the caller to avoid querying the \var{base} object for
  its length.
\end{cvardesc}

\begin{cfuncdesc}{int}{PyBuffer_Check}{PyObject *p}
  Return true if the argument has type \cdata{PyBuffer_Type}.
\end{cfuncdesc}

\begin{cfuncdesc}{PyObject*}{PyBuffer_FromObject}{PyObject *base,
                                                  Py_ssize_t offset, Py_ssize_t size}
  Return a new read-only buffer object.  This raises
  \exception{TypeError} if \var{base} doesn't support the read-only
  buffer protocol or doesn't provide exactly one buffer segment, or it
  raises \exception{ValueError} if \var{offset} is less than zero. The
  buffer will hold a reference to the \var{base} object, and the
  buffer's contents will refer to the \var{base} object's buffer
  interface, starting as position \var{offset} and extending for
  \var{size} bytes. If \var{size} is \constant{Py_END_OF_BUFFER}, then
  the new buffer's contents extend to the length of the \var{base}
  object's exported buffer data.
\end{cfuncdesc}

\begin{cfuncdesc}{PyObject*}{PyBuffer_FromReadWriteObject}{PyObject *base,
                                                           Py_ssize_t offset,
                                                           Py_ssize_t size}
  Return a new writable buffer object.  Parameters and exceptions are
  similar to those for \cfunction{PyBuffer_FromObject()}.  If the
  \var{base} object does not export the writeable buffer protocol,
  then \exception{TypeError} is raised.
\end{cfuncdesc}

\begin{cfuncdesc}{PyObject*}{PyBuffer_FromMemory}{void *ptr, Py_ssize_t size}
  Return a new read-only buffer object that reads from a specified
  location in memory, with a specified size.  The caller is
  responsible for ensuring that the memory buffer, passed in as
  \var{ptr}, is not deallocated while the returned buffer object
  exists.  Raises \exception{ValueError} if \var{size} is less than
  zero.  Note that \constant{Py_END_OF_BUFFER} may \emph{not} be
  passed for the \var{size} parameter; \exception{ValueError} will be
  raised in that case.
\end{cfuncdesc}

\begin{cfuncdesc}{PyObject*}{PyBuffer_FromReadWriteMemory}{void *ptr, Py_ssize_t size}
  Similar to \cfunction{PyBuffer_FromMemory()}, but the returned
  buffer is writable.
\end{cfuncdesc}

\begin{cfuncdesc}{PyObject*}{PyBuffer_New}{Py_ssize_t size}
  Return a new writable buffer object that maintains its own memory
  buffer of \var{size} bytes.  \exception{ValueError} is returned if
  \var{size} is not zero or positive.  Note that the memory buffer (as
  returned by \cfunction{PyObject_AsWriteBuffer()}) is not specifically
  aligned.
\end{cfuncdesc}


\subsection{Tuple Objects \label{tupleObjects}}

\obindex{tuple}
\begin{ctypedesc}{PyTupleObject}
  This subtype of \ctype{PyObject} represents a Python tuple object.
\end{ctypedesc}

\begin{cvardesc}{PyTypeObject}{PyTuple_Type}
  This instance of \ctype{PyTypeObject} represents the Python tuple
  type; it is the same object as \code{tuple} and \code{types.TupleType}
  in the Python layer.\withsubitem{(in module types)}{\ttindex{TupleType}}.
\end{cvardesc}

\begin{cfuncdesc}{int}{PyTuple_Check}{PyObject *p}
  Return true if \var{p} is a tuple object or an instance of a subtype
  of the tuple type.
  \versionchanged[Allowed subtypes to be accepted]{2.2}
\end{cfuncdesc}

\begin{cfuncdesc}{int}{PyTuple_CheckExact}{PyObject *p}
  Return true if \var{p} is a tuple object, but not an instance of a
  subtype of the tuple type.
  \versionadded{2.2}
\end{cfuncdesc}

\begin{cfuncdesc}{PyObject*}{PyTuple_New}{Py_ssize_t len}
  Return a new tuple object of size \var{len}, or \NULL{} on failure.
\end{cfuncdesc}

\begin{cfuncdesc}{PyObject*}{PyTuple_Pack}{Py_ssize_t n, \moreargs}
  Return a new tuple object of size \var{n}, or \NULL{} on failure.
  The tuple values are initialized to the subsequent \var{n} C arguments
  pointing to Python objects.  \samp{PyTuple_Pack(2, \var{a}, \var{b})}
  is equivalent to \samp{Py_BuildValue("(OO)", \var{a}, \var{b})}.
  \versionadded{2.4}
\end{cfuncdesc}

\begin{cfuncdesc}{int}{PyTuple_Size}{PyObject *p}
  Take a pointer to a tuple object, and return the size of that
  tuple.
\end{cfuncdesc}

\begin{cfuncdesc}{int}{PyTuple_GET_SIZE}{PyObject *p}
  Return the size of the tuple \var{p}, which must be non-\NULL{} and
  point to a tuple; no error checking is performed.
\end{cfuncdesc}

\begin{cfuncdesc}{PyObject*}{PyTuple_GetItem}{PyObject *p, Py_ssize_t pos}
  Return the object at position \var{pos} in the tuple pointed to by
  \var{p}.  If \var{pos} is out of bounds, return \NULL{} and sets an
  \exception{IndexError} exception.
\end{cfuncdesc}

\begin{cfuncdesc}{PyObject*}{PyTuple_GET_ITEM}{PyObject *p, Py_ssize_t pos}
  Like \cfunction{PyTuple_GetItem()}, but does no checking of its
  arguments.
\end{cfuncdesc}

\begin{cfuncdesc}{PyObject*}{PyTuple_GetSlice}{PyObject *p,
                                               Py_ssize_t low, Py_ssize_t high}
  Take a slice of the tuple pointed to by \var{p} from \var{low} to
  \var{high} and return it as a new tuple.
\end{cfuncdesc}

\begin{cfuncdesc}{int}{PyTuple_SetItem}{PyObject *p,
                                        Py_ssize_t pos, PyObject *o}
  Insert a reference to object \var{o} at position \var{pos} of the
  tuple pointed to by \var{p}. Return \code{0} on success.
  \note{This function ``steals'' a reference to \var{o}.}
\end{cfuncdesc}

\begin{cfuncdesc}{void}{PyTuple_SET_ITEM}{PyObject *p,
                                          Py_ssize_t pos, PyObject *o}
  Like \cfunction{PyTuple_SetItem()}, but does no error checking, and
  should \emph{only} be used to fill in brand new tuples.  \note{This
  function ``steals'' a reference to \var{o}.}
\end{cfuncdesc}

\begin{cfuncdesc}{int}{_PyTuple_Resize}{PyObject **p, Py_ssize_t newsize}
  Can be used to resize a tuple.  \var{newsize} will be the new length
  of the tuple.  Because tuples are \emph{supposed} to be immutable,
  this should only be used if there is only one reference to the
  object.  Do \emph{not} use this if the tuple may already be known to
  some other part of the code.  The tuple will always grow or shrink
  at the end.  Think of this as destroying the old tuple and creating
  a new one, only more efficiently.  Returns \code{0} on success.
  Client code should never assume that the resulting value of
  \code{*\var{p}} will be the same as before calling this function.
  If the object referenced by \code{*\var{p}} is replaced, the
  original \code{*\var{p}} is destroyed.  On failure, returns
  \code{-1} and sets \code{*\var{p}} to \NULL{}, and raises
  \exception{MemoryError} or
  \exception{SystemError}.
  \versionchanged[Removed unused third parameter, \var{last_is_sticky}]{2.2}
\end{cfuncdesc}


\subsection{List Objects \label{listObjects}}

\obindex{list}
\begin{ctypedesc}{PyListObject}
  This subtype of \ctype{PyObject} represents a Python list object.
\end{ctypedesc}

\begin{cvardesc}{PyTypeObject}{PyList_Type}
  This instance of \ctype{PyTypeObject} represents the Python list
  type.  This is the same object as \code{list} and \code{types.ListType}
  in the Python layer.\withsubitem{(in module types)}{\ttindex{ListType}}
\end{cvardesc}

\begin{cfuncdesc}{int}{PyList_Check}{PyObject *p}
  Return true if \var{p} is a list object or an instance of a
  subtype of the list type.
  \versionchanged[Allowed subtypes to be accepted]{2.2}
\end{cfuncdesc}

\begin{cfuncdesc}{int}{PyList_CheckExact}{PyObject *p}
  Return true if \var{p} is a list object, but not an instance of a
  subtype of the list type.
  \versionadded{2.2}
\end{cfuncdesc}

\begin{cfuncdesc}{PyObject*}{PyList_New}{Py_ssize_t len}
  Return a new list of length \var{len} on success, or \NULL{} on
  failure.
  \note{If \var{length} is greater than zero, the returned list object's
        items are set to \code{NULL}.  Thus you cannot use abstract
        API functions such as \cfunction{PySequence_SetItem()} 
        or expose the object to Python code before setting all items to a
        real object with \cfunction{PyList_SetItem()}.}
\end{cfuncdesc}

\begin{cfuncdesc}{Py_ssize_t}{PyList_Size}{PyObject *list}
  Return the length of the list object in \var{list}; this is
  equivalent to \samp{len(\var{list})} on a list object.
  \bifuncindex{len}
\end{cfuncdesc}

\begin{cfuncdesc}{Py_ssize_t}{PyList_GET_SIZE}{PyObject *list}
  Macro form of \cfunction{PyList_Size()} without error checking.
\end{cfuncdesc}

\begin{cfuncdesc}{PyObject*}{PyList_GetItem}{PyObject *list, Py_ssize_t index}
  Return the object at position \var{pos} in the list pointed to by
  \var{p}.  The position must be positive, indexing from the end of the
  list is not supported.  If \var{pos} is out of bounds, return \NULL{}
  and set an \exception{IndexError} exception.
\end{cfuncdesc}

\begin{cfuncdesc}{PyObject*}{PyList_GET_ITEM}{PyObject *list, Py_ssize_t i}
  Macro form of \cfunction{PyList_GetItem()} without error checking.
\end{cfuncdesc}

\begin{cfuncdesc}{int}{PyList_SetItem}{PyObject *list, Py_ssize_t index,
                                       PyObject *item}
  Set the item at index \var{index} in list to \var{item}.  Return
  \code{0} on success or \code{-1} on failure.  \note{This function
  ``steals'' a reference to \var{item} and discards a reference to an
  item already in the list at the affected position.}
\end{cfuncdesc}

\begin{cfuncdesc}{void}{PyList_SET_ITEM}{PyObject *list, Py_ssize_t i,
                                              PyObject *o}
  Macro form of \cfunction{PyList_SetItem()} without error checking.
  This is normally only used to fill in new lists where there is no
  previous content.
  \note{This function ``steals'' a reference to \var{item}, and,
  unlike \cfunction{PyList_SetItem()}, does \emph{not} discard a
  reference to any item that it being replaced; any reference in
  \var{list} at position \var{i} will be leaked.}
\end{cfuncdesc}

\begin{cfuncdesc}{int}{PyList_Insert}{PyObject *list, Py_ssize_t index,
                                      PyObject *item}
  Insert the item \var{item} into list \var{list} in front of index
  \var{index}.  Return \code{0} if successful; return \code{-1} and
  set an exception if unsuccessful.  Analogous to
  \code{\var{list}.insert(\var{index}, \var{item})}.
\end{cfuncdesc}

\begin{cfuncdesc}{int}{PyList_Append}{PyObject *list, PyObject *item}
  Append the object \var{item} at the end of list \var{list}.
  Return \code{0} if successful; return \code{-1} and set an
  exception if unsuccessful.  Analogous to
  \code{\var{list}.append(\var{item})}.
\end{cfuncdesc}

\begin{cfuncdesc}{PyObject*}{PyList_GetSlice}{PyObject *list,
                                              Py_ssize_t low, Py_ssize_t high}
  Return a list of the objects in \var{list} containing the objects
  \emph{between} \var{low} and \var{high}.  Return \NULL{} and set
  an exception if unsuccessful.
  Analogous to \code{\var{list}[\var{low}:\var{high}]}.
\end{cfuncdesc}

\begin{cfuncdesc}{int}{PyList_SetSlice}{PyObject *list,
                                        Py_ssize_t low, Py_ssize_t high,
                                        PyObject *itemlist}
  Set the slice of \var{list} between \var{low} and \var{high} to the
  contents of \var{itemlist}.  Analogous to
  \code{\var{list}[\var{low}:\var{high}] = \var{itemlist}}.
  The \var{itemlist} may be \NULL{}, indicating the assignment
  of an empty list (slice deletion).
  Return \code{0} on success, \code{-1} on failure.
\end{cfuncdesc}

\begin{cfuncdesc}{int}{PyList_Sort}{PyObject *list}
  Sort the items of \var{list} in place.  Return \code{0} on
  success, \code{-1} on failure.  This is equivalent to
  \samp{\var{list}.sort()}.
\end{cfuncdesc}

\begin{cfuncdesc}{int}{PyList_Reverse}{PyObject *list}
  Reverse the items of \var{list} in place.  Return \code{0} on
  success, \code{-1} on failure.  This is the equivalent of
  \samp{\var{list}.reverse()}.
\end{cfuncdesc}

\begin{cfuncdesc}{PyObject*}{PyList_AsTuple}{PyObject *list}
  Return a new tuple object containing the contents of \var{list};
  equivalent to \samp{tuple(\var{list})}.\bifuncindex{tuple}
\end{cfuncdesc}


\section{Mapping Objects \label{mapObjects}}

\obindex{mapping}


\subsection{Dictionary Objects \label{dictObjects}}

\obindex{dictionary}
\begin{ctypedesc}{PyDictObject}
  This subtype of \ctype{PyObject} represents a Python dictionary
  object.
\end{ctypedesc}

\begin{cvardesc}{PyTypeObject}{PyDict_Type}
  This instance of \ctype{PyTypeObject} represents the Python
  dictionary type.  This is exposed to Python programs as
  \code{dict} and \code{types.DictType}.
  \withsubitem{(in module types)}{\ttindex{DictType}\ttindex{DictionaryType}}
\end{cvardesc}

\begin{cfuncdesc}{int}{PyDict_Check}{PyObject *p}
  Return true if \var{p} is a dict object or an instance of a
  subtype of the dict type.
  \versionchanged[Allowed subtypes to be accepted]{2.2}
\end{cfuncdesc}

\begin{cfuncdesc}{int}{PyDict_CheckExact}{PyObject *p}
  Return true if \var{p} is a dict object, but not an instance of a
  subtype of the dict type.
  \versionadded{2.4}
\end{cfuncdesc}

\begin{cfuncdesc}{PyObject*}{PyDict_New}{}
  Return a new empty dictionary, or \NULL{} on failure.
\end{cfuncdesc}

\begin{cfuncdesc}{PyObject*}{PyDictProxy_New}{PyObject *dict}
  Return a proxy object for a mapping which enforces read-only
  behavior.  This is normally used to create a proxy to prevent
  modification of the dictionary for non-dynamic class types.
  \versionadded{2.2}
\end{cfuncdesc}

\begin{cfuncdesc}{void}{PyDict_Clear}{PyObject *p}
  Empty an existing dictionary of all key-value pairs.
\end{cfuncdesc}

\begin{cfuncdesc}{int}{PyDict_Contains}{PyObject *p, PyObject *key}
  Determine if dictionary \var{p} contains \var{key}.  If an item
  in \var{p} is matches \var{key}, return \code{1}, otherwise return
  \code{0}.  On error, return \code{-1}.  This is equivalent to the
  Python expression \samp{\var{key} in \var{p}}.
  \versionadded{2.4}
\end{cfuncdesc}

\begin{cfuncdesc}{PyObject*}{PyDict_Copy}{PyObject *p}
  Return a new dictionary that contains the same key-value pairs as
  \var{p}.
  \versionadded{1.6}
\end{cfuncdesc}

\begin{cfuncdesc}{int}{PyDict_SetItem}{PyObject *p, PyObject *key,
                                       PyObject *val}
  Insert \var{value} into the dictionary \var{p} with a key of
  \var{key}.  \var{key} must be hashable; if it isn't,
  \exception{TypeError} will be raised.
  Return \code{0} on success or \code{-1} on failure.
\end{cfuncdesc}

\begin{cfuncdesc}{int}{PyDict_SetItemString}{PyObject *p,
            const char *key,
            PyObject *val}
  Insert \var{value} into the dictionary \var{p} using \var{key} as a
  key. \var{key} should be a \ctype{char*}.  The key object is created
  using \code{PyString_FromString(\var{key})}. Return \code{0} on
  success or \code{-1} on failure.
  \ttindex{PyString_FromString()}
\end{cfuncdesc}

\begin{cfuncdesc}{int}{PyDict_DelItem}{PyObject *p, PyObject *key}
  Remove the entry in dictionary \var{p} with key \var{key}.
  \var{key} must be hashable; if it isn't, \exception{TypeError} is
  raised.  Return \code{0} on success or \code{-1} on failure.
\end{cfuncdesc}

\begin{cfuncdesc}{int}{PyDict_DelItemString}{PyObject *p, char *key}
  Remove the entry in dictionary \var{p} which has a key specified by
  the string \var{key}.  Return \code{0} on success or \code{-1} on
  failure.
\end{cfuncdesc}

\begin{cfuncdesc}{PyObject*}{PyDict_GetItem}{PyObject *p, PyObject *key}
  Return the object from dictionary \var{p} which has a key
  \var{key}.  Return \NULL{} if the key \var{key} is not present, but
  \emph{without} setting an exception.
\end{cfuncdesc}

\begin{cfuncdesc}{PyObject*}{PyDict_GetItemString}{PyObject *p, const char *key}
  This is the same as \cfunction{PyDict_GetItem()}, but \var{key} is
  specified as a \ctype{char*}, rather than a \ctype{PyObject*}.
\end{cfuncdesc}

\begin{cfuncdesc}{PyObject*}{PyDict_Items}{PyObject *p}
  Return a \ctype{PyListObject} containing all the items from the
  dictionary, as in the dictionary method \method{items()} (see the
  \citetitle[../lib/lib.html]{Python Library Reference}).
\end{cfuncdesc}

\begin{cfuncdesc}{PyObject*}{PyDict_Keys}{PyObject *p}
  Return a \ctype{PyListObject} containing all the keys from the
  dictionary, as in the dictionary method \method{keys()} (see the
  \citetitle[../lib/lib.html]{Python Library Reference}).
\end{cfuncdesc}

\begin{cfuncdesc}{PyObject*}{PyDict_Values}{PyObject *p}
  Return a \ctype{PyListObject} containing all the values from the
  dictionary \var{p}, as in the dictionary method \method{values()}
  (see the \citetitle[../lib/lib.html]{Python Library Reference}).
\end{cfuncdesc}

\begin{cfuncdesc}{Py_ssize_t}{PyDict_Size}{PyObject *p}
  Return the number of items in the dictionary.  This is equivalent
  to \samp{len(\var{p})} on a dictionary.\bifuncindex{len}
\end{cfuncdesc}

\begin{cfuncdesc}{int}{PyDict_Next}{PyObject *p, Py_ssize_t *ppos,
                                    PyObject **pkey, PyObject **pvalue}
  Iterate over all key-value pairs in the dictionary \var{p}.  The
  \ctype{int} referred to by \var{ppos} must be initialized to
  \code{0} prior to the first call to this function to start the
  iteration; the function returns true for each pair in the
  dictionary, and false once all pairs have been reported.  The
  parameters \var{pkey} and \var{pvalue} should either point to
  \ctype{PyObject*} variables that will be filled in with each key and
  value, respectively, or may be \NULL{}.  Any references returned through
  them are borrowed.  \var{ppos} should not be altered during iteration.
  Its value represents offsets within the internal dictionary structure,
  and since the structure is sparse, the offsets are not consecutive.

  For example:

\begin{verbatim}
PyObject *key, *value;
Py_ssize_t pos = 0;

while (PyDict_Next(self->dict, &pos, &key, &value)) {
    /* do something interesting with the values... */
    ...
}
\end{verbatim}

  The dictionary \var{p} should not be mutated during iteration.  It
  is safe (since Python 2.1) to modify the values of the keys as you
  iterate over the dictionary, but only so long as the set of keys
  does not change.  For example:

\begin{verbatim}
PyObject *key, *value;
Py_ssize_t pos = 0;

while (PyDict_Next(self->dict, &pos, &key, &value)) {
    int i = PyInt_AS_LONG(value) + 1;
    PyObject *o = PyInt_FromLong(i);
    if (o == NULL)
        return -1;
    if (PyDict_SetItem(self->dict, key, o) < 0) {
        Py_DECREF(o);
        return -1;
    }
    Py_DECREF(o);
}
\end{verbatim}
\end{cfuncdesc}

\begin{cfuncdesc}{int}{PyDict_Merge}{PyObject *a, PyObject *b, int override}
  Iterate over mapping object \var{b} adding key-value pairs to dictionary
  \var{a}.
  \var{b} may be a dictionary, or any object supporting
  \function{PyMapping_Keys()} and \function{PyObject_GetItem()}.
  If \var{override} is true, existing pairs in \var{a} will
  be replaced if a matching key is found in \var{b}, otherwise pairs
  will only be added if there is not a matching key in \var{a}.
  Return \code{0} on success or \code{-1} if an exception was
  raised.
\versionadded{2.2}
\end{cfuncdesc}

\begin{cfuncdesc}{int}{PyDict_Update}{PyObject *a, PyObject *b}
  This is the same as \code{PyDict_Merge(\var{a}, \var{b}, 1)} in C,
  or \code{\var{a}.update(\var{b})} in Python.  Return \code{0} on
  success or \code{-1} if an exception was raised.
  \versionadded{2.2}
\end{cfuncdesc}

\begin{cfuncdesc}{int}{PyDict_MergeFromSeq2}{PyObject *a, PyObject *seq2,
                                             int override}
  Update or merge into dictionary \var{a}, from the key-value pairs in
  \var{seq2}.  \var{seq2} must be an iterable object producing
  iterable objects of length 2, viewed as key-value pairs.  In case of
  duplicate keys, the last wins if \var{override} is true, else the
  first wins.
  Return \code{0} on success or \code{-1} if an exception
  was raised.
  Equivalent Python (except for the return value):

\begin{verbatim}
def PyDict_MergeFromSeq2(a, seq2, override):
    for key, value in seq2:
        if override or key not in a:
            a[key] = value
\end{verbatim}

  \versionadded{2.2}
\end{cfuncdesc}


\section{Other Objects \label{otherObjects}}

\subsection{Class Objects \label{classObjects}}

\obindex{class}
Note that the class objects described here represent old-style classes,
which will go away in Python 3. When creating new types for extension
modules, you will want to work with type objects (section
\ref{typeObjects}).

\begin{ctypedesc}{PyClassObject}
  The C structure of the objects used to describe built-in classes.
\end{ctypedesc}

\begin{cvardesc}{PyObject*}{PyClass_Type}
  This is the type object for class objects; it is the same object as
  \code{types.ClassType} in the Python layer.
  \withsubitem{(in module types)}{\ttindex{ClassType}}
\end{cvardesc}

\begin{cfuncdesc}{int}{PyClass_Check}{PyObject *o}
  Return true if the object \var{o} is a class object, including
  instances of types derived from the standard class object.  Return
  false in all other cases.
\end{cfuncdesc}

\begin{cfuncdesc}{int}{PyClass_IsSubclass}{PyObject *klass, PyObject *base}
  Return true if \var{klass} is a subclass of \var{base}. Return false in
  all other cases.
\end{cfuncdesc}

\subsection{File Objects \label{fileObjects}}

\obindex{file}
Python's built-in file objects are implemented entirely on the
\ctype{FILE*} support from the C standard library.  This is an
implementation detail and may change in future releases of Python.

\begin{ctypedesc}{PyFileObject}
  This subtype of \ctype{PyObject} represents a Python file object.
\end{ctypedesc}

\begin{cvardesc}{PyTypeObject}{PyFile_Type}
  This instance of \ctype{PyTypeObject} represents the Python file
  type.  This is exposed to Python programs as \code{file} and
  \code{types.FileType}.
  \withsubitem{(in module types)}{\ttindex{FileType}}
\end{cvardesc}

\begin{cfuncdesc}{int}{PyFile_Check}{PyObject *p}
  Return true if its argument is a \ctype{PyFileObject} or a subtype
  of \ctype{PyFileObject}.
  \versionchanged[Allowed subtypes to be accepted]{2.2}
\end{cfuncdesc}

\begin{cfuncdesc}{int}{PyFile_CheckExact}{PyObject *p}
  Return true if its argument is a \ctype{PyFileObject}, but not a
  subtype of \ctype{PyFileObject}.
  \versionadded{2.2}
\end{cfuncdesc}

\begin{cfuncdesc}{PyObject*}{PyFile_FromString}{char *filename, char *mode}
  On success, return a new file object that is opened on the file
  given by \var{filename}, with a file mode given by \var{mode}, where
  \var{mode} has the same semantics as the standard C routine
  \cfunction{fopen()}\ttindex{fopen()}.  On failure, return \NULL{}.
\end{cfuncdesc}

\begin{cfuncdesc}{PyObject*}{PyFile_FromFile}{FILE *fp,
                                              char *name, char *mode,
                                              int (*close)(FILE*)}
  Create a new \ctype{PyFileObject} from the already-open standard C
  file pointer, \var{fp}.  The function \var{close} will be called
  when the file should be closed.  Return \NULL{} on failure.
\end{cfuncdesc}

\begin{cfuncdesc}{FILE*}{PyFile_AsFile}{PyObject *p}
  Return the file object associated with \var{p} as a \ctype{FILE*}.
\end{cfuncdesc}

\begin{cfuncdesc}{PyObject*}{PyFile_GetLine}{PyObject *p, int n}
  Equivalent to \code{\var{p}.readline(\optional{\var{n}})}, this
  function reads one line from the object \var{p}.  \var{p} may be a
  file object or any object with a \method{readline()} method.  If
  \var{n} is \code{0}, exactly one line is read, regardless of the
  length of the line.  If \var{n} is greater than \code{0}, no more
  than \var{n} bytes will be read from the file; a partial line can be
  returned.  In both cases, an empty string is returned if the end of
  the file is reached immediately.  If \var{n} is less than \code{0},
  however, one line is read regardless of length, but
  \exception{EOFError} is raised if the end of the file is reached
  immediately.
  \withsubitem{(built-in exception)}{\ttindex{EOFError}}
\end{cfuncdesc}

\begin{cfuncdesc}{PyObject*}{PyFile_Name}{PyObject *p}
  Return the name of the file specified by \var{p} as a string
  object.
\end{cfuncdesc}

\begin{cfuncdesc}{void}{PyFile_SetBufSize}{PyFileObject *p, int n}
  Available on systems with \cfunction{setvbuf()}\ttindex{setvbuf()}
  only.  This should only be called immediately after file object
  creation.
\end{cfuncdesc}

\begin{cfuncdesc}{int}{PyFile_Encoding}{PyFileObject *p, char *enc}
  Set the file's encoding for Unicode output to \var{enc}. Return
  1 on success and 0 on failure.
  \versionadded{2.3}
\end{cfuncdesc}

\begin{cfuncdesc}{int}{PyFile_SoftSpace}{PyObject *p, int newflag}
  This function exists for internal use by the interpreter.  Set the
  \member{softspace} attribute of \var{p} to \var{newflag} and
  \withsubitem{(file attribute)}{\ttindex{softspace}}return the
  previous value.  \var{p} does not have to be a file object for this
  function to work properly; any object is supported (thought its only
  interesting if the \member{softspace} attribute can be set).  This
  function clears any errors, and will return \code{0} as the previous
  value if the attribute either does not exist or if there were errors
  in retrieving it.  There is no way to detect errors from this
  function, but doing so should not be needed.
\end{cfuncdesc}

\begin{cfuncdesc}{int}{PyFile_WriteObject}{PyObject *obj, PyObject *p,
                                           int flags}
  Write object \var{obj} to file object \var{p}.  The only supported
  flag for \var{flags} is
  \constant{Py_PRINT_RAW}\ttindex{Py_PRINT_RAW}; if given, the
  \function{str()} of the object is written instead of the
  \function{repr()}.  Return \code{0} on success or \code{-1} on
  failure; the appropriate exception will be set.
\end{cfuncdesc}

\begin{cfuncdesc}{int}{PyFile_WriteString}{const char *s, PyObject *p}
  Write string \var{s} to file object \var{p}.  Return \code{0} on
  success or \code{-1} on failure; the appropriate exception will be
  set.
\end{cfuncdesc}


\subsection{Instance Objects \label{instanceObjects}}

\obindex{instance}
There are very few functions specific to instance objects.

\begin{cvardesc}{PyTypeObject}{PyInstance_Type}
  Type object for class instances.
\end{cvardesc}

\begin{cfuncdesc}{int}{PyInstance_Check}{PyObject *obj}
  Return true if \var{obj} is an instance.
\end{cfuncdesc}

\begin{cfuncdesc}{PyObject*}{PyInstance_New}{PyObject *class,
                                             PyObject *arg,
                                             PyObject *kw}
  Create a new instance of a specific class.  The parameters \var{arg}
  and \var{kw} are used as the positional and keyword parameters to
  the object's constructor.
\end{cfuncdesc}

\begin{cfuncdesc}{PyObject*}{PyInstance_NewRaw}{PyObject *class,
                                                PyObject *dict}
  Create a new instance of a specific class without calling its
  constructor.  \var{class} is the class of new object.  The
  \var{dict} parameter will be used as the object's \member{__dict__};
  if \NULL{}, a new dictionary will be created for the instance.
\end{cfuncdesc}


\subsection{Function Objects \label{function-objects}}

\obindex{function}
There are a few functions specific to Python functions.

\begin{ctypedesc}{PyFunctionObject}
  The C structure used for functions.
\end{ctypedesc}

\begin{cvardesc}{PyTypeObject}{PyFunction_Type}
  This is an instance of \ctype{PyTypeObject} and represents the
  Python function type.  It is exposed to Python programmers as
  \code{types.FunctionType}.
  \withsubitem{(in module types)}{\ttindex{MethodType}}
\end{cvardesc}

\begin{cfuncdesc}{int}{PyFunction_Check}{PyObject *o}
  Return true if \var{o} is a function object (has type
  \cdata{PyFunction_Type}).  The parameter must not be \NULL{}.
\end{cfuncdesc}

\begin{cfuncdesc}{PyObject*}{PyFunction_New}{PyObject *code,
                                             PyObject *globals}
  Return a new function object associated with the code object
  \var{code}. \var{globals} must be a dictionary with the global
  variables accessible to the function.

  The function's docstring, name and \var{__module__} are retrieved
  from the code object, the argument defaults and closure are set to
  \NULL{}.
\end{cfuncdesc}

\begin{cfuncdesc}{PyObject*}{PyFunction_GetCode}{PyObject *op}
  Return the code object associated with the function object \var{op}.
\end{cfuncdesc}

\begin{cfuncdesc}{PyObject*}{PyFunction_GetGlobals}{PyObject *op}
  Return the globals dictionary associated with the function object
  \var{op}.
\end{cfuncdesc}

\begin{cfuncdesc}{PyObject*}{PyFunction_GetModule}{PyObject *op}
  Return the \var{__module__} attribute of the function object \var{op}.
  This is normally a string containing the module name, but can be set
  to any other object by Python code.
\end{cfuncdesc}

\begin{cfuncdesc}{PyObject*}{PyFunction_GetDefaults}{PyObject *op}
  Return the argument default values of the function object \var{op}.
  This can be a tuple of arguments or \NULL{}.
\end{cfuncdesc}

\begin{cfuncdesc}{int}{PyFunction_SetDefaults}{PyObject *op,
                                               PyObject *defaults}
  Set the argument default values for the function object \var{op}.
  \var{defaults} must be \var{Py_None} or a tuple.

  Raises \exception{SystemError} and returns \code{-1} on failure.
\end{cfuncdesc}

\begin{cfuncdesc}{PyObject*}{PyFunction_GetClosure}{PyObject *op}
  Return the closure associated with the function object \var{op}.
  This can be \NULL{} or a tuple of cell objects.
\end{cfuncdesc}

\begin{cfuncdesc}{int}{PyFunction_SetClosure}{PyObject *op,
                                              PyObject *closure}
  Set the closure associated with the function object \var{op}.
  \var{closure} must be \var{Py_None} or a tuple of cell objects.

  Raises \exception{SystemError} and returns \code{-1} on failure.
\end{cfuncdesc}


\subsection{Method Objects \label{method-objects}}

\obindex{method}
There are some useful functions that are useful for working with
method objects.

\begin{cvardesc}{PyTypeObject}{PyMethod_Type}
  This instance of \ctype{PyTypeObject} represents the Python method
  type.  This is exposed to Python programs as \code{types.MethodType}.
  \withsubitem{(in module types)}{\ttindex{MethodType}}
\end{cvardesc}

\begin{cfuncdesc}{int}{PyMethod_Check}{PyObject *o}
  Return true if \var{o} is a method object (has type
  \cdata{PyMethod_Type}).  The parameter must not be \NULL{}.
\end{cfuncdesc}

\begin{cfuncdesc}{PyObject*}{PyMethod_New}{PyObject *func,
                                           PyObject *self, PyObject *class}
  Return a new method object, with \var{func} being any callable
  object; this is the function that will be called when the method is
  called.  If this method should be bound to an instance, \var{self}
  should be the instance and \var{class} should be the class of
  \var{self}, otherwise \var{self} should be \NULL{} and \var{class}
  should be the class which provides the unbound method..
\end{cfuncdesc}

\begin{cfuncdesc}{PyObject*}{PyMethod_Class}{PyObject *meth}
  Return the class object from which the method \var{meth} was
  created; if this was created from an instance, it will be the class
  of the instance.
\end{cfuncdesc}

\begin{cfuncdesc}{PyObject*}{PyMethod_GET_CLASS}{PyObject *meth}
  Macro version of \cfunction{PyMethod_Class()} which avoids error
  checking.
\end{cfuncdesc}

\begin{cfuncdesc}{PyObject*}{PyMethod_Function}{PyObject *meth}
  Return the function object associated with the method \var{meth}.
\end{cfuncdesc}

\begin{cfuncdesc}{PyObject*}{PyMethod_GET_FUNCTION}{PyObject *meth}
  Macro version of \cfunction{PyMethod_Function()} which avoids error
  checking.
\end{cfuncdesc}

\begin{cfuncdesc}{PyObject*}{PyMethod_Self}{PyObject *meth}
  Return the instance associated with the method \var{meth} if it is
  bound, otherwise return \NULL{}.
\end{cfuncdesc}

\begin{cfuncdesc}{PyObject*}{PyMethod_GET_SELF}{PyObject *meth}
  Macro version of \cfunction{PyMethod_Self()} which avoids error
  checking.
\end{cfuncdesc}


\subsection{Module Objects \label{moduleObjects}}

\obindex{module}
There are only a few functions special to module objects.

\begin{cvardesc}{PyTypeObject}{PyModule_Type}
  This instance of \ctype{PyTypeObject} represents the Python module
  type.  This is exposed to Python programs as
  \code{types.ModuleType}.
  \withsubitem{(in module types)}{\ttindex{ModuleType}}
\end{cvardesc}

\begin{cfuncdesc}{int}{PyModule_Check}{PyObject *p}
  Return true if \var{p} is a module object, or a subtype of a module
  object.
  \versionchanged[Allowed subtypes to be accepted]{2.2}
\end{cfuncdesc}

\begin{cfuncdesc}{int}{PyModule_CheckExact}{PyObject *p}
  Return true if \var{p} is a module object, but not a subtype of
  \cdata{PyModule_Type}.
  \versionadded{2.2}
\end{cfuncdesc}

\begin{cfuncdesc}{PyObject*}{PyModule_New}{const char *name}
  Return a new module object with the \member{__name__} attribute set
  to \var{name}.  Only the module's \member{__doc__} and
  \member{__name__} attributes are filled in; the caller is
  responsible for providing a \member{__file__} attribute.
  \withsubitem{(module attribute)}{
    \ttindex{__name__}\ttindex{__doc__}\ttindex{__file__}}
\end{cfuncdesc}

\begin{cfuncdesc}{PyObject*}{PyModule_GetDict}{PyObject *module}
  Return the dictionary object that implements \var{module}'s
  namespace; this object is the same as the \member{__dict__}
  attribute of the module object.  This function never fails.
  \withsubitem{(module attribute)}{\ttindex{__dict__}}
  It is recommended extensions use other \cfunction{PyModule_*()}
  and \cfunction{PyObject_*()} functions rather than directly
  manipulate a module's \member{__dict__}.
\end{cfuncdesc}

\begin{cfuncdesc}{char*}{PyModule_GetName}{PyObject *module}
  Return \var{module}'s \member{__name__} value.  If the module does
  not provide one, or if it is not a string, \exception{SystemError}
  is raised and \NULL{} is returned.
  \withsubitem{(module attribute)}{\ttindex{__name__}}
  \withsubitem{(built-in exception)}{\ttindex{SystemError}}
\end{cfuncdesc}

\begin{cfuncdesc}{char*}{PyModule_GetFilename}{PyObject *module}
  Return the name of the file from which \var{module} was loaded using
  \var{module}'s \member{__file__} attribute.  If this is not defined,
  or if it is not a string, raise \exception{SystemError} and return
  \NULL{}.
  \withsubitem{(module attribute)}{\ttindex{__file__}}
  \withsubitem{(built-in exception)}{\ttindex{SystemError}}
\end{cfuncdesc}

\begin{cfuncdesc}{int}{PyModule_AddObject}{PyObject *module,
                                           const char *name, PyObject *value}
  Add an object to \var{module} as \var{name}.  This is a convenience
  function which can be used from the module's initialization
  function.  This steals a reference to \var{value}.  Return
  \code{-1} on error, \code{0} on success.
  \versionadded{2.0}
\end{cfuncdesc}

\begin{cfuncdesc}{int}{PyModule_AddIntConstant}{PyObject *module,
                                                const char *name, long value}
  Add an integer constant to \var{module} as \var{name}.  This
  convenience function can be used from the module's initialization
  function. Return \code{-1} on error, \code{0} on success.
  \versionadded{2.0}
\end{cfuncdesc}

\begin{cfuncdesc}{int}{PyModule_AddStringConstant}{PyObject *module,
                                                   const char *name, const char *value}
  Add a string constant to \var{module} as \var{name}.  This
  convenience function can be used from the module's initialization
  function.  The string \var{value} must be null-terminated.  Return
  \code{-1} on error, \code{0} on success.
  \versionadded{2.0}
\end{cfuncdesc}


\subsection{Iterator Objects \label{iterator-objects}}

Python provides two general-purpose iterator objects.  The first, a
sequence iterator, works with an arbitrary sequence supporting the
\method{__getitem__()} method.  The second works with a callable
object and a sentinel value, calling the callable for each item in the
sequence, and ending the iteration when the sentinel value is
returned.

\begin{cvardesc}{PyTypeObject}{PySeqIter_Type}
  Type object for iterator objects returned by
  \cfunction{PySeqIter_New()} and the one-argument form of the
  \function{iter()} built-in function for built-in sequence types.
  \versionadded{2.2}
\end{cvardesc}

\begin{cfuncdesc}{int}{PySeqIter_Check}{op}
  Return true if the type of \var{op} is \cdata{PySeqIter_Type}.
  \versionadded{2.2}
\end{cfuncdesc}

\begin{cfuncdesc}{PyObject*}{PySeqIter_New}{PyObject *seq}
  Return an iterator that works with a general sequence object,
  \var{seq}.  The iteration ends when the sequence raises
  \exception{IndexError} for the subscripting operation.
  \versionadded{2.2}
\end{cfuncdesc}

\begin{cvardesc}{PyTypeObject}{PyCallIter_Type}
  Type object for iterator objects returned by
  \cfunction{PyCallIter_New()} and the two-argument form of the
  \function{iter()} built-in function.
  \versionadded{2.2}
\end{cvardesc}

\begin{cfuncdesc}{int}{PyCallIter_Check}{op}
  Return true if the type of \var{op} is \cdata{PyCallIter_Type}.
  \versionadded{2.2}
\end{cfuncdesc}

\begin{cfuncdesc}{PyObject*}{PyCallIter_New}{PyObject *callable,
                                             PyObject *sentinel}
  Return a new iterator.  The first parameter, \var{callable}, can be
  any Python callable object that can be called with no parameters;
  each call to it should return the next item in the iteration.  When
  \var{callable} returns a value equal to \var{sentinel}, the
  iteration will be terminated.
  \versionadded{2.2}
\end{cfuncdesc}


\subsection{Descriptor Objects \label{descriptor-objects}}

``Descriptors'' are objects that describe some attribute of an object.
They are found in the dictionary of type objects.

\begin{cvardesc}{PyTypeObject}{PyProperty_Type}
  The type object for the built-in descriptor types.
  \versionadded{2.2}
\end{cvardesc}

\begin{cfuncdesc}{PyObject*}{PyDescr_NewGetSet}{PyTypeObject *type,
					        struct PyGetSetDef *getset}
  \versionadded{2.2}
\end{cfuncdesc}

\begin{cfuncdesc}{PyObject*}{PyDescr_NewMember}{PyTypeObject *type,
					        struct PyMemberDef *meth}
  \versionadded{2.2}
\end{cfuncdesc}

\begin{cfuncdesc}{PyObject*}{PyDescr_NewMethod}{PyTypeObject *type,
                                                struct PyMethodDef *meth}
  \versionadded{2.2}
\end{cfuncdesc}

\begin{cfuncdesc}{PyObject*}{PyDescr_NewWrapper}{PyTypeObject *type,
						 struct wrapperbase *wrapper,
                                                 void *wrapped}
  \versionadded{2.2}
\end{cfuncdesc}

\begin{cfuncdesc}{PyObject*}{PyDescr_NewClassMethod}{PyTypeObject *type,
						     PyMethodDef *method}
  \versionadded{2.3}
\end{cfuncdesc}

\begin{cfuncdesc}{int}{PyDescr_IsData}{PyObject *descr}
  Return true if the descriptor objects \var{descr} describes a data
  attribute, or false if it describes a method.  \var{descr} must be a
  descriptor object; there is no error checking.
  \versionadded{2.2}
\end{cfuncdesc}

\begin{cfuncdesc}{PyObject*}{PyWrapper_New}{PyObject *, PyObject *}
  \versionadded{2.2}
\end{cfuncdesc}


\subsection{Slice Objects \label{slice-objects}}

\begin{cvardesc}{PyTypeObject}{PySlice_Type}
  The type object for slice objects.  This is the same as
  \code{slice} and \code{types.SliceType}.
  \withsubitem{(in module types)}{\ttindex{SliceType}}
\end{cvardesc}

\begin{cfuncdesc}{int}{PySlice_Check}{PyObject *ob}
  Return true if \var{ob} is a slice object; \var{ob} must not be
  \NULL{}.
\end{cfuncdesc}

\begin{cfuncdesc}{PyObject*}{PySlice_New}{PyObject *start, PyObject *stop,
                                          PyObject *step}
  Return a new slice object with the given values.  The \var{start},
  \var{stop}, and \var{step} parameters are used as the values of the
  slice object attributes of the same names.  Any of the values may be
  \NULL{}, in which case the \code{None} will be used for the
  corresponding attribute.  Return \NULL{} if the new object could
  not be allocated.
\end{cfuncdesc}

\begin{cfuncdesc}{int}{PySlice_GetIndices}{PySliceObject *slice, Py_ssize_t length,
                                           Py_ssize_t *start, Py_ssize_t *stop, Py_ssize_t *step}
Retrieve the start, stop and step indices from the slice object
\var{slice}, assuming a sequence of length \var{length}. Treats
indices greater than \var{length} as errors.

Returns 0 on success and -1 on error with no exception set (unless one
of the indices was not \constant{None} and failed to be converted to
an integer, in which case -1 is returned with an exception set).

You probably do not want to use this function.  If you want to use
slice objects in versions of Python prior to 2.3, you would probably
do well to incorporate the source of \cfunction{PySlice_GetIndicesEx},
suitably renamed, in the source of your extension.
\end{cfuncdesc}

\begin{cfuncdesc}{int}{PySlice_GetIndicesEx}{PySliceObject *slice, Py_ssize_t length,
                                             Py_ssize_t *start, Py_ssize_t *stop, Py_ssize_t *step,
                                             Py_ssize_t *slicelength}
Usable replacement for \cfunction{PySlice_GetIndices}.  Retrieve the
start, stop, and step indices from the slice object \var{slice}
assuming a sequence of length \var{length}, and store the length of
the slice in \var{slicelength}.  Out of bounds indices are clipped in
a manner consistent with the handling of normal slices.

Returns 0 on success and -1 on error with exception set.

\versionadded{2.3}
\end{cfuncdesc}


\subsection{Weak Reference Objects \label{weakref-objects}}

Python supports \emph{weak references} as first-class objects.  There
are two specific object types which directly implement weak
references.  The first is a simple reference object, and the second
acts as a proxy for the original object as much as it can.

\begin{cfuncdesc}{int}{PyWeakref_Check}{ob}
  Return true if \var{ob} is either a reference or proxy object.
  \versionadded{2.2}
\end{cfuncdesc}

\begin{cfuncdesc}{int}{PyWeakref_CheckRef}{ob}
  Return true if \var{ob} is a reference object.
  \versionadded{2.2}
\end{cfuncdesc}

\begin{cfuncdesc}{int}{PyWeakref_CheckProxy}{ob}
  Return true if \var{ob} is a proxy object.
  \versionadded{2.2}
\end{cfuncdesc}

\begin{cfuncdesc}{PyObject*}{PyWeakref_NewRef}{PyObject *ob,
                                               PyObject *callback}
  Return a weak reference object for the object \var{ob}.  This will
  always return a new reference, but is not guaranteed to create a new
  object; an existing reference object may be returned.  The second
  parameter, \var{callback}, can be a callable object that receives
  notification when \var{ob} is garbage collected; it should accept a
  single parameter, which will be the weak reference object itself.
  \var{callback} may also be \code{None} or \NULL{}.  If \var{ob}
  is not a weakly-referencable object, or if \var{callback} is not
  callable, \code{None}, or \NULL{}, this will return \NULL{} and
  raise \exception{TypeError}.
  \versionadded{2.2}
\end{cfuncdesc}

\begin{cfuncdesc}{PyObject*}{PyWeakref_NewProxy}{PyObject *ob,
                                                 PyObject *callback}
  Return a weak reference proxy object for the object \var{ob}.  This
  will always return a new reference, but is not guaranteed to create
  a new object; an existing proxy object may be returned.  The second
  parameter, \var{callback}, can be a callable object that receives
  notification when \var{ob} is garbage collected; it should accept a
  single parameter, which will be the weak reference object itself.
  \var{callback} may also be \code{None} or \NULL{}.  If \var{ob} is not
  a weakly-referencable object, or if \var{callback} is not callable,
  \code{None}, or \NULL{}, this will return \NULL{} and raise
  \exception{TypeError}.
  \versionadded{2.2}
\end{cfuncdesc}

\begin{cfuncdesc}{PyObject*}{PyWeakref_GetObject}{PyObject *ref}
  Return the referenced object from a weak reference, \var{ref}.  If
  the referent is no longer live, returns \code{None}.
  \versionadded{2.2}
\end{cfuncdesc}

\begin{cfuncdesc}{PyObject*}{PyWeakref_GET_OBJECT}{PyObject *ref}
  Similar to \cfunction{PyWeakref_GetObject()}, but implemented as a
  macro that does no error checking.
  \versionadded{2.2}
\end{cfuncdesc}


\subsection{CObjects \label{cObjects}}

\obindex{CObject}
Refer to \emph{Extending and Embedding the Python Interpreter},
section~1.12, ``Providing a C API for an Extension Module,'' for more
information on using these objects.


\begin{ctypedesc}{PyCObject}
  This subtype of \ctype{PyObject} represents an opaque value, useful
  for C extension modules who need to pass an opaque value (as a
  \ctype{void*} pointer) through Python code to other C code.  It is
  often used to make a C function pointer defined in one module
  available to other modules, so the regular import mechanism can be
  used to access C APIs defined in dynamically loaded modules.
\end{ctypedesc}

\begin{cfuncdesc}{int}{PyCObject_Check}{PyObject *p}
  Return true if its argument is a \ctype{PyCObject}.
\end{cfuncdesc}

\begin{cfuncdesc}{PyObject*}{PyCObject_FromVoidPtr}{void* cobj,
                                                    void (*destr)(void *)}
  Create a \ctype{PyCObject} from the \code{void *}\var{cobj}.  The
  \var{destr} function will be called when the object is reclaimed,
  unless it is \NULL{}.
\end{cfuncdesc}

\begin{cfuncdesc}{PyObject*}{PyCObject_FromVoidPtrAndDesc}{void* cobj,
	                          void* desc, void (*destr)(void *, void *)}
  Create a \ctype{PyCObject} from the \ctype{void *}\var{cobj}.  The
  \var{destr} function will be called when the object is reclaimed.
  The \var{desc} argument can be used to pass extra callback data for
  the destructor function.
\end{cfuncdesc}

\begin{cfuncdesc}{void*}{PyCObject_AsVoidPtr}{PyObject* self}
  Return the object \ctype{void *} that the \ctype{PyCObject}
  \var{self} was created with.
\end{cfuncdesc}

\begin{cfuncdesc}{void*}{PyCObject_GetDesc}{PyObject* self}
  Return the description \ctype{void *} that the \ctype{PyCObject}
  \var{self} was created with.
\end{cfuncdesc}

\begin{cfuncdesc}{int}{PyCObject_SetVoidPtr}{PyObject* self, void* cobj}
  Set the void pointer inside \var{self} to \var{cobj}.
  The \ctype{PyCObject} must not have an associated destructor.
  Return true on success, false on failure.
\end{cfuncdesc}


\subsection{Cell Objects \label{cell-objects}}

``Cell'' objects are used to implement variables referenced by
multiple scopes.  For each such variable, a cell object is created to
store the value; the local variables of each stack frame that
references the value contains a reference to the cells from outer
scopes which also use that variable.  When the value is accessed, the
value contained in the cell is used instead of the cell object
itself.  This de-referencing of the cell object requires support from
the generated byte-code; these are not automatically de-referenced
when accessed.  Cell objects are not likely to be useful elsewhere.

\begin{ctypedesc}{PyCellObject}
  The C structure used for cell objects.
\end{ctypedesc}

\begin{cvardesc}{PyTypeObject}{PyCell_Type}
  The type object corresponding to cell objects.
\end{cvardesc}

\begin{cfuncdesc}{int}{PyCell_Check}{ob}
  Return true if \var{ob} is a cell object; \var{ob} must not be
  \NULL{}.
\end{cfuncdesc}

\begin{cfuncdesc}{PyObject*}{PyCell_New}{PyObject *ob}
  Create and return a new cell object containing the value \var{ob}.
  The parameter may be \NULL{}.
\end{cfuncdesc}

\begin{cfuncdesc}{PyObject*}{PyCell_Get}{PyObject *cell}
  Return the contents of the cell \var{cell}.
\end{cfuncdesc}

\begin{cfuncdesc}{PyObject*}{PyCell_GET}{PyObject *cell}
  Return the contents of the cell \var{cell}, but without checking
  that \var{cell} is non-\NULL{} and a cell object.
\end{cfuncdesc}

\begin{cfuncdesc}{int}{PyCell_Set}{PyObject *cell, PyObject *value}
  Set the contents of the cell object \var{cell} to \var{value}.  This
  releases the reference to any current content of the cell.
  \var{value} may be \NULL{}.  \var{cell} must be non-\NULL{}; if it is
  not a cell object, \code{-1} will be returned.  On success, \code{0}
  will be returned.
\end{cfuncdesc}

\begin{cfuncdesc}{void}{PyCell_SET}{PyObject *cell, PyObject *value}
  Sets the value of the cell object \var{cell} to \var{value}.  No
  reference counts are adjusted, and no checks are made for safety;
  \var{cell} must be non-\NULL{} and must be a cell object.
\end{cfuncdesc}


\subsection{Generator Objects \label{gen-objects}}

Generator objects are what Python uses to implement generator iterators.
They are normally created by iterating over a function that yields values,
rather than explicitly calling \cfunction{PyGen_New}.

\begin{ctypedesc}{PyGenObject}
  The C structure used for generator objects.
\end{ctypedesc}

\begin{cvardesc}{PyTypeObject}{PyGen_Type}
  The type object corresponding to generator objects
\end{cvardesc}

\begin{cfuncdesc}{int}{PyGen_Check}{ob}
  Return true if \var{ob} is a generator object; \var{ob} must not be
  \NULL{}.
\end{cfuncdesc}

\begin{cfuncdesc}{int}{PyGen_CheckExact}{ob}
  Return true if \var{ob}'s type is \var{PyGen_Type}
  is a generator object; \var{ob} must not be
  \NULL{}.
\end{cfuncdesc}

\begin{cfuncdesc}{PyObject*}{PyGen_New}{PyFrameObject *frame}
  Create and return a new generator object based on the \var{frame} object.
  A reference to \var{frame} is stolen by this function.
  The parameter must not be \NULL{}.
\end{cfuncdesc}


\subsection{DateTime Objects \label{datetime-objects}}

Various date and time objects are supplied by the \module{datetime}
module.  Before using any of these functions, the header file
\file{datetime.h} must be included in your source (note that this is
not included by \file{Python.h}), and the macro
\cfunction{PyDateTime_IMPORT} must be invoked.  The macro puts a
pointer to a C structure into a static variable, 
\code{PyDateTimeAPI}, that is used by the following macros.

Type-check macros:

\begin{cfuncdesc}{int}{PyDate_Check}{PyObject *ob}
  Return true if \var{ob} is of type \cdata{PyDateTime_DateType} or
  a subtype of \cdata{PyDateTime_DateType}.  \var{ob} must not be
  \NULL{}.
  \versionadded{2.4}
\end{cfuncdesc}

\begin{cfuncdesc}{int}{PyDate_CheckExact}{PyObject *ob}
  Return true if \var{ob} is of type \cdata{PyDateTime_DateType}.
  \var{ob} must not be \NULL{}.
  \versionadded{2.4}
\end{cfuncdesc}

\begin{cfuncdesc}{int}{PyDateTime_Check}{PyObject *ob}
  Return true if \var{ob} is of type \cdata{PyDateTime_DateTimeType} or
  a subtype of \cdata{PyDateTime_DateTimeType}.  \var{ob} must not be
  \NULL{}.
  \versionadded{2.4}
\end{cfuncdesc}

\begin{cfuncdesc}{int}{PyDateTime_CheckExact}{PyObject *ob}
  Return true if \var{ob} is of type \cdata{PyDateTime_DateTimeType}.
  \var{ob} must not be \NULL{}.
  \versionadded{2.4}
\end{cfuncdesc}

\begin{cfuncdesc}{int}{PyTime_Check}{PyObject *ob}
  Return true if \var{ob} is of type \cdata{PyDateTime_TimeType} or
  a subtype of \cdata{PyDateTime_TimeType}.  \var{ob} must not be
  \NULL{}.
  \versionadded{2.4}
\end{cfuncdesc}

\begin{cfuncdesc}{int}{PyTime_CheckExact}{PyObject *ob}
  Return true if \var{ob} is of type \cdata{PyDateTime_TimeType}.
  \var{ob} must not be \NULL{}.
  \versionadded{2.4}
\end{cfuncdesc}

\begin{cfuncdesc}{int}{PyDelta_Check}{PyObject *ob}
  Return true if \var{ob} is of type \cdata{PyDateTime_DeltaType} or
  a subtype of \cdata{PyDateTime_DeltaType}.  \var{ob} must not be
  \NULL{}.
  \versionadded{2.4}
\end{cfuncdesc}

\begin{cfuncdesc}{int}{PyDelta_CheckExact}{PyObject *ob}
  Return true if \var{ob} is of type \cdata{PyDateTime_DeltaType}.
  \var{ob} must not be \NULL{}.
  \versionadded{2.4}
\end{cfuncdesc}

\begin{cfuncdesc}{int}{PyTZInfo_Check}{PyObject *ob}
  Return true if \var{ob} is of type \cdata{PyDateTime_TZInfoType} or
  a subtype of \cdata{PyDateTime_TZInfoType}.  \var{ob} must not be
  \NULL{}.
  \versionadded{2.4}
\end{cfuncdesc}

\begin{cfuncdesc}{int}{PyTZInfo_CheckExact}{PyObject *ob}
  Return true if \var{ob} is of type \cdata{PyDateTime_TZInfoType}.
  \var{ob} must not be \NULL{}.
  \versionadded{2.4}
\end{cfuncdesc}

Macros to create objects:

\begin{cfuncdesc}{PyObject*}{PyDate_FromDate}{int year, int month, int day}
  Return a \code{datetime.date} object with the specified year, month
  and day.
  \versionadded{2.4}
\end{cfuncdesc}

\begin{cfuncdesc}{PyObject*}{PyDateTime_FromDateAndTime}{int year, int month,
        int day, int hour, int minute, int second, int usecond}
  Return a \code{datetime.datetime} object with the specified year, month,
  day, hour, minute, second and microsecond.
  \versionadded{2.4}
\end{cfuncdesc}

\begin{cfuncdesc}{PyObject*}{PyTime_FromTime}{int hour, int minute,
        int second, int usecond}
  Return a \code{datetime.time} object with the specified hour, minute,
  second and microsecond.
  \versionadded{2.4}
\end{cfuncdesc}

\begin{cfuncdesc}{PyObject*}{PyDelta_FromDSU}{int days, int seconds,
        int useconds}
  Return a \code{datetime.timedelta} object representing the given number
  of days, seconds and microseconds.  Normalization is performed so that
  the resulting number of microseconds and seconds lie in the ranges
  documented for \code{datetime.timedelta} objects.
  \versionadded{2.4}
\end{cfuncdesc}

Macros to extract fields from date objects.  The argument must be an
instance of \cdata{PyDateTime_Date}, including subclasses (such as
\cdata{PyDateTime_DateTime}).  The argument must not be \NULL{}, and
the type is not checked:

\begin{cfuncdesc}{int}{PyDateTime_GET_YEAR}{PyDateTime_Date *o}
  Return the year, as a positive int.
  \versionadded{2.4}
\end{cfuncdesc}

\begin{cfuncdesc}{int}{PyDateTime_GET_MONTH}{PyDateTime_Date *o}
  Return the month, as an int from 1 through 12.
  \versionadded{2.4}
\end{cfuncdesc}

\begin{cfuncdesc}{int}{PyDateTime_GET_DAY}{PyDateTime_Date *o}
  Return the day, as an int from 1 through 31.
  \versionadded{2.4}
\end{cfuncdesc}

Macros to extract fields from datetime objects.  The argument must be an
instance of \cdata{PyDateTime_DateTime}, including subclasses.
The argument must not be \NULL{}, and the type is not checked:

\begin{cfuncdesc}{int}{PyDateTime_DATE_GET_HOUR}{PyDateTime_DateTime *o}
  Return the hour, as an int from 0 through 23.
  \versionadded{2.4}
\end{cfuncdesc}

\begin{cfuncdesc}{int}{PyDateTime_DATE_GET_MINUTE}{PyDateTime_DateTime *o}
  Return the minute, as an int from 0 through 59.
  \versionadded{2.4}
\end{cfuncdesc}

\begin{cfuncdesc}{int}{PyDateTime_DATE_GET_SECOND}{PyDateTime_DateTime *o}
  Return the second, as an int from 0 through 59.
  \versionadded{2.4}
\end{cfuncdesc}

\begin{cfuncdesc}{int}{PyDateTime_DATE_GET_MICROSECOND}{PyDateTime_DateTime *o}
  Return the microsecond, as an int from 0 through 999999.
  \versionadded{2.4}
\end{cfuncdesc}

Macros to extract fields from time objects.  The argument must be an
instance of \cdata{PyDateTime_Time}, including subclasses.
The argument must not be \NULL{}, and the type is not checked:

\begin{cfuncdesc}{int}{PyDateTime_TIME_GET_HOUR}{PyDateTime_Time *o}
  Return the hour, as an int from 0 through 23.
  \versionadded{2.4}
\end{cfuncdesc}

\begin{cfuncdesc}{int}{PyDateTime_TIME_GET_MINUTE}{PyDateTime_Time *o}
  Return the minute, as an int from 0 through 59.
  \versionadded{2.4}
\end{cfuncdesc}

\begin{cfuncdesc}{int}{PyDateTime_TIME_GET_SECOND}{PyDateTime_Time *o}
  Return the second, as an int from 0 through 59.
  \versionadded{2.4}
\end{cfuncdesc}

\begin{cfuncdesc}{int}{PyDateTime_TIME_GET_MICROSECOND}{PyDateTime_Time *o}
  Return the microsecond, as an int from 0 through 999999.
  \versionadded{2.4}
\end{cfuncdesc}

Macros for the convenience of modules implementing the DB API:

\begin{cfuncdesc}{PyObject*}{PyDateTime_FromTimestamp}{PyObject *args}
  Create and return a new \code{datetime.datetime} object given an argument
  tuple suitable for passing to \code{datetime.datetime.fromtimestamp()}.
  \versionadded{2.4}
\end{cfuncdesc}

\begin{cfuncdesc}{PyObject*}{PyDate_FromTimestamp}{PyObject *args}
  Create and return a new \code{datetime.date} object given an argument
  tuple suitable for passing to \code{datetime.date.fromtimestamp()}.
  \versionadded{2.4}
\end{cfuncdesc}


\subsection{Set Objects \label{setObjects}}
\sectionauthor{Raymond D. Hettinger}{python@rcn.com}

\obindex{set}
\obindex{frozenset}
\versionadded{2.5}

This section details the public API for \class{set} and \class{frozenset}
objects.  Any functionality not listed below is best accessed using the
either the abstract object protocol (including
\cfunction{PyObject_CallMethod()}, \cfunction{PyObject_RichCompareBool()},
\cfunction{PyObject_Hash()}, \cfunction{PyObject_Repr()},
\cfunction{PyObject_IsTrue()}, \cfunction{PyObject_Print()}, and
\cfunction{PyObject_GetIter()})
or the abstract number protocol (including
\cfunction{PyNumber_And()}, \cfunction{PyNumber_Subtract()},
\cfunction{PyNumber_Or()}, \cfunction{PyNumber_Xor()},
\cfunction{PyNumber_InPlaceAnd()}, \cfunction{PyNumber_InPlaceSubtract()},
\cfunction{PyNumber_InPlaceOr()}, and \cfunction{PyNumber_InPlaceXor()}).

\begin{ctypedesc}{PySetObject}
  This subtype of \ctype{PyObject} is used to hold the internal data for
  both \class{set} and \class{frozenset} objects.  It is like a
  \ctype{PyDictObject} in that it is a fixed size for small sets
  (much like tuple storage) and will point to a separate, variable sized
  block of memory for medium and large sized sets (much like list storage).
  None of the fields of this structure should be considered public and
  are subject to change.  All access should be done through the
  documented API rather than by manipulating the values in the structure.

\end{ctypedesc}

\begin{cvardesc}{PyTypeObject}{PySet_Type}
  This is an instance of \ctype{PyTypeObject} representing the Python
  \class{set} type.
\end{cvardesc}

\begin{cvardesc}{PyTypeObject}{PyFrozenSet_Type}
  This is an instance of \ctype{PyTypeObject} representing the Python
  \class{frozenset} type.
\end{cvardesc}


The following type check macros work on pointers to any Python object.
Likewise, the constructor functions work with any iterable Python object.

\begin{cfuncdesc}{int}{PyAnySet_Check}{PyObject *p}
  Return true if \var{p} is a \class{set} object, a \class{frozenset}
  object, or an instance of a subtype.
\end{cfuncdesc}

\begin{cfuncdesc}{int}{PyAnySet_CheckExact}{PyObject *p}
  Return true if \var{p} is a \class{set} object or a \class{frozenset}
  object but not an instance of a subtype.
\end{cfuncdesc}

\begin{cfuncdesc}{int}{PyFrozenSet_CheckExact}{PyObject *p}
  Return true if \var{p} is a \class{frozenset} object
  but not an instance of a subtype.
\end{cfuncdesc}

\begin{cfuncdesc}{PyObject*}{PySet_New}{PyObject *iterable}
  Return a new \class{set} containing objects returned by the
  \var{iterable}.  The \var{iterable} may be \NULL{} to create a
  new empty set.  Return the new set on success or \NULL{} on
  failure.  Raise \exception{TypeError} if \var{iterable} is
  not actually iterable.  The constructor is also useful for
  copying a set (\code{c=set(s)}).
\end{cfuncdesc}

\begin{cfuncdesc}{PyObject*}{PyFrozenSet_New}{PyObject *iterable}
  Return a new \class{frozenset} containing objects returned by the
  \var{iterable}.  The \var{iterable} may be \NULL{} to create a
  new empty frozenset.  Return the new set on success or \NULL{} on
  failure.  Raise \exception{TypeError} if \var{iterable} is
  not actually iterable.
\end{cfuncdesc}


The following functions and macros are available for instances of
\class{set} or \class{frozenset} or instances of their subtypes.

\begin{cfuncdesc}{int}{PySet_Size}{PyObject *anyset}
  Return the length of a \class{set} or \class{frozenset} object.
  Equivalent to \samp{len(\var{anyset})}.  Raises a
  \exception{PyExc_SystemError} if \var{anyset} is not a \class{set},
  \class{frozenset}, or an instance of a subtype.
  \bifuncindex{len}
\end{cfuncdesc}

\begin{cfuncdesc}{int}{PySet_GET_SIZE}{PyObject *anyset}
  Macro form of \cfunction{PySet_Size()} without error checking.
\end{cfuncdesc}

\begin{cfuncdesc}{int}{PySet_Contains}{PyObject *anyset, PyObject *key}
  Return 1 if found, 0 if not found, and -1 if an error is
  encountered.  Unlike the Python \method{__contains__()} method, this
  function does not automatically convert unhashable sets into temporary
  frozensets.  Raise a \exception{TypeError} if the \var{key} is unhashable.
  Raise \exception{PyExc_SystemError} if \var{anyset} is not a \class{set},
  \class{frozenset}, or an instance of a subtype.
\end{cfuncdesc}

The following functions are available for instances of \class{set} or
its subtypes but not for instances of \class{frozenset} or its subtypes.

\begin{cfuncdesc}{int}{PySet_Add}{PyObject *set, PyObject *key}
  Add \var{key} to a \class{set} instance.  Does not apply to
  \class{frozenset} instances.  Return 0 on success or -1 on failure.
  Raise a \exception{TypeError} if the \var{key} is unhashable.
  Raise a \exception{MemoryError} if there is no room to grow.
  Raise a \exception{SystemError} if \var{set} is an not an instance
  of \class{set} or its subtype.
\end{cfuncdesc}

\begin{cfuncdesc}{int}{PySet_Discard}{PyObject *set, PyObject *key}
  Return 1 if found and removed, 0 if not found (no action taken),
  and -1 if an error is encountered.  Does not raise \exception{KeyError}
  for missing keys.  Raise a \exception{TypeError} if the \var{key} is
  unhashable.  Unlike the Python \method{discard()} method, this function
  does not automatically convert unhashable sets into temporary frozensets.
  Raise \exception{PyExc_SystemError} if \var{set} is an not an instance
  of \class{set} or its subtype.
\end{cfuncdesc}

\begin{cfuncdesc}{PyObject*}{PySet_Pop}{PyObject *set}
  Return a new reference to an arbitrary object in the \var{set},
  and removes the object from the \var{set}.  Return \NULL{} on
  failure.  Raise \exception{KeyError} if the set is empty.
  Raise a \exception{SystemError} if \var{set} is an not an instance
  of \class{set} or its subtype.
\end{cfuncdesc}

\begin{cfuncdesc}{int}{PySet_Clear}{PyObject *set}
  Empty an existing set of all elements.
\end{cfuncdesc}

\chapter{Initialization, Finalization, and Threads
         \label{initialization}}

\begin{cfuncdesc}{void}{Py_Initialize}{}
  Initialize the Python interpreter.  In an application embedding 
  Python, this should be called before using any other Python/C API
  functions; with the exception of
  \cfunction{Py_SetProgramName()}\ttindex{Py_SetProgramName()},
  \cfunction{PyEval_InitThreads()}\ttindex{PyEval_InitThreads()},
  \cfunction{PyEval_ReleaseLock()}\ttindex{PyEval_ReleaseLock()},
  and \cfunction{PyEval_AcquireLock()}\ttindex{PyEval_AcquireLock()}.
  This initializes the table of loaded modules (\code{sys.modules}),
  and\withsubitem{(in module sys)}{\ttindex{modules}\ttindex{path}}
  creates the fundamental modules
  \module{__builtin__}\refbimodindex{__builtin__},
  \module{__main__}\refbimodindex{__main__} and
  \module{sys}\refbimodindex{sys}.  It also initializes the module
  search\indexiii{module}{search}{path} path (\code{sys.path}).
  It does not set \code{sys.argv}; use
  \cfunction{PySys_SetArgv()}\ttindex{PySys_SetArgv()} for that.  This
  is a no-op when called for a second time (without calling
  \cfunction{Py_Finalize()}\ttindex{Py_Finalize()} first).  There is
  no return value; it is a fatal error if the initialization fails.
\end{cfuncdesc}

\begin{cfuncdesc}{void}{Py_InitializeEx}{int initsigs}
  This function works like \cfunction{Py_Initialize()} if
  \var{initsigs} is 1. If \var{initsigs} is 0, it skips
  initialization registration of signal handlers, which
  might be useful when Python is embedded. \versionadded{2.4}
\end{cfuncdesc}

\begin{cfuncdesc}{int}{Py_IsInitialized}{}
  Return true (nonzero) when the Python interpreter has been
  initialized, false (zero) if not.  After \cfunction{Py_Finalize()}
  is called, this returns false until \cfunction{Py_Initialize()} is
  called again.
\end{cfuncdesc}

\begin{cfuncdesc}{void}{Py_Finalize}{}
  Undo all initializations made by \cfunction{Py_Initialize()} and
  subsequent use of Python/C API functions, and destroy all
  sub-interpreters (see \cfunction{Py_NewInterpreter()} below) that
  were created and not yet destroyed since the last call to
  \cfunction{Py_Initialize()}.  Ideally, this frees all memory
  allocated by the Python interpreter.  This is a no-op when called
  for a second time (without calling \cfunction{Py_Initialize()} again
  first).  There is no return value; errors during finalization are
  ignored.

  This function is provided for a number of reasons.  An embedding
  application might want to restart Python without having to restart
  the application itself.  An application that has loaded the Python
  interpreter from a dynamically loadable library (or DLL) might want
  to free all memory allocated by Python before unloading the
  DLL. During a hunt for memory leaks in an application a developer
  might want to free all memory allocated by Python before exiting
  from the application.

  \strong{Bugs and caveats:} The destruction of modules and objects in
  modules is done in random order; this may cause destructors
  (\method{__del__()} methods) to fail when they depend on other
  objects (even functions) or modules.  Dynamically loaded extension
  modules loaded by Python are not unloaded.  Small amounts of memory
  allocated by the Python interpreter may not be freed (if you find a
  leak, please report it).  Memory tied up in circular references
  between objects is not freed.  Some memory allocated by extension
  modules may not be freed.  Some extensions may not work properly if
  their initialization routine is called more than once; this can
  happen if an application calls \cfunction{Py_Initialize()} and
  \cfunction{Py_Finalize()} more than once.
\end{cfuncdesc}

\begin{cfuncdesc}{PyThreadState*}{Py_NewInterpreter}{}
  Create a new sub-interpreter.  This is an (almost) totally separate
  environment for the execution of Python code.  In particular, the
  new interpreter has separate, independent versions of all imported
  modules, including the fundamental modules
  \module{__builtin__}\refbimodindex{__builtin__},
  \module{__main__}\refbimodindex{__main__} and
  \module{sys}\refbimodindex{sys}.  The table of loaded modules
  (\code{sys.modules}) and the module search path (\code{sys.path})
  are also separate.  The new environment has no \code{sys.argv}
  variable.  It has new standard I/O stream file objects
  \code{sys.stdin}, \code{sys.stdout} and \code{sys.stderr} (however
  these refer to the same underlying \ctype{FILE} structures in the C
  library).
  \withsubitem{(in module sys)}{
    \ttindex{stdout}\ttindex{stderr}\ttindex{stdin}}

  The return value points to the first thread state created in the new
  sub-interpreter.  This thread state is made in the current thread
  state.  Note that no actual thread is created; see the discussion of
  thread states below.  If creation of the new interpreter is
  unsuccessful, \NULL{} is returned; no exception is set since the
  exception state is stored in the current thread state and there may
  not be a current thread state.  (Like all other Python/C API
  functions, the global interpreter lock must be held before calling
  this function and is still held when it returns; however, unlike
  most other Python/C API functions, there needn't be a current thread
  state on entry.)

  Extension modules are shared between (sub-)interpreters as follows:
  the first time a particular extension is imported, it is initialized
  normally, and a (shallow) copy of its module's dictionary is
  squirreled away.  When the same extension is imported by another
  (sub-)interpreter, a new module is initialized and filled with the
  contents of this copy; the extension's \code{init} function is not
  called.  Note that this is different from what happens when an
  extension is imported after the interpreter has been completely
  re-initialized by calling
  \cfunction{Py_Finalize()}\ttindex{Py_Finalize()} and
  \cfunction{Py_Initialize()}\ttindex{Py_Initialize()}; in that case,
  the extension's \code{init\var{module}} function \emph{is} called
  again.

  \strong{Bugs and caveats:} Because sub-interpreters (and the main
  interpreter) are part of the same process, the insulation between
  them isn't perfect --- for example, using low-level file operations
  like \withsubitem{(in module os)}{\ttindex{close()}}
  \function{os.close()} they can (accidentally or maliciously) affect
  each other's open files.  Because of the way extensions are shared
  between (sub-)interpreters, some extensions may not work properly;
  this is especially likely when the extension makes use of (static)
  global variables, or when the extension manipulates its module's
  dictionary after its initialization.  It is possible to insert
  objects created in one sub-interpreter into a namespace of another
  sub-interpreter; this should be done with great care to avoid
  sharing user-defined functions, methods, instances or classes
  between sub-interpreters, since import operations executed by such
  objects may affect the wrong (sub-)interpreter's dictionary of
  loaded modules.  (XXX This is a hard-to-fix bug that will be
  addressed in a future release.)
\end{cfuncdesc}

\begin{cfuncdesc}{void}{Py_EndInterpreter}{PyThreadState *tstate}
  Destroy the (sub-)interpreter represented by the given thread state.
  The given thread state must be the current thread state.  See the
  discussion of thread states below.  When the call returns, the
  current thread state is \NULL.  All thread states associated with
  this interpreter are destroyed.  (The global interpreter lock must
  be held before calling this function and is still held when it
  returns.)  \cfunction{Py_Finalize()}\ttindex{Py_Finalize()} will
  destroy all sub-interpreters that haven't been explicitly destroyed
  at that point.
\end{cfuncdesc}

\begin{cfuncdesc}{void}{Py_SetProgramName}{char *name}
  This function should be called before
  \cfunction{Py_Initialize()}\ttindex{Py_Initialize()} is called
  for the first time, if it is called at all.  It tells the
  interpreter the value of the \code{argv[0]} argument to the
  \cfunction{main()}\ttindex{main()} function of the program.  This is
  used by \cfunction{Py_GetPath()}\ttindex{Py_GetPath()} and some
  other functions below to find the Python run-time libraries relative
  to the interpreter executable.  The default value is
  \code{'python'}.  The argument should point to a zero-terminated
  character string in static storage whose contents will not change
  for the duration of the program's execution.  No code in the Python
  interpreter will change the contents of this storage.
\end{cfuncdesc}

\begin{cfuncdesc}{char*}{Py_GetProgramName}{}
  Return the program name set with
  \cfunction{Py_SetProgramName()}\ttindex{Py_SetProgramName()}, or the
  default.  The returned string points into static storage; the caller
  should not modify its value.
\end{cfuncdesc}

\begin{cfuncdesc}{char*}{Py_GetPrefix}{}
  Return the \emph{prefix} for installed platform-independent files.
  This is derived through a number of complicated rules from the
  program name set with \cfunction{Py_SetProgramName()} and some
  environment variables; for example, if the program name is
  \code{'/usr/local/bin/python'}, the prefix is \code{'/usr/local'}.
  The returned string points into static storage; the caller should
  not modify its value.  This corresponds to the \makevar{prefix}
  variable in the top-level \file{Makefile} and the
  \longprogramopt{prefix} argument to the \program{configure} script
  at build time.  The value is available to Python code as
  \code{sys.prefix}.  It is only useful on \UNIX.  See also the next
  function.
\end{cfuncdesc}

\begin{cfuncdesc}{char*}{Py_GetExecPrefix}{}
  Return the \emph{exec-prefix} for installed
  platform-\emph{de}pendent files.  This is derived through a number
  of complicated rules from the program name set with
  \cfunction{Py_SetProgramName()} and some environment variables; for
  example, if the program name is \code{'/usr/local/bin/python'}, the
  exec-prefix is \code{'/usr/local'}.  The returned string points into
  static storage; the caller should not modify its value.  This
  corresponds to the \makevar{exec_prefix} variable in the top-level
  \file{Makefile} and the \longprogramopt{exec-prefix} argument to the
  \program{configure} script at build  time.  The value is available
  to Python code as \code{sys.exec_prefix}.  It is only useful on
  \UNIX.

  Background: The exec-prefix differs from the prefix when platform
  dependent files (such as executables and shared libraries) are
  installed in a different directory tree.  In a typical installation,
  platform dependent files may be installed in the
  \file{/usr/local/plat} subtree while platform independent may be
  installed in \file{/usr/local}.

  Generally speaking, a platform is a combination of hardware and
  software families, e.g.  Sparc machines running the Solaris 2.x
  operating system are considered the same platform, but Intel
  machines running Solaris 2.x are another platform, and Intel
  machines running Linux are yet another platform.  Different major
  revisions of the same operating system generally also form different
  platforms.  Non-\UNIX{} operating systems are a different story; the
  installation strategies on those systems are so different that the
  prefix and exec-prefix are meaningless, and set to the empty string.
  Note that compiled Python bytecode files are platform independent
  (but not independent from the Python version by which they were
  compiled!).

  System administrators will know how to configure the \program{mount}
  or \program{automount} programs to share \file{/usr/local} between
  platforms while having \file{/usr/local/plat} be a different
  filesystem for each platform.
\end{cfuncdesc}

\begin{cfuncdesc}{char*}{Py_GetProgramFullPath}{}
  Return the full program name of the Python executable; this is 
  computed as a side-effect of deriving the default module search path 
  from the program name (set by
  \cfunction{Py_SetProgramName()}\ttindex{Py_SetProgramName()} above).
  The returned string points into static storage; the caller should
  not modify its value.  The value is available to Python code as
  \code{sys.executable}.
  \withsubitem{(in module sys)}{\ttindex{executable}}
\end{cfuncdesc}

\begin{cfuncdesc}{char*}{Py_GetPath}{}
  \indexiii{module}{search}{path}
  Return the default module search path; this is computed from the 
  program name (set by \cfunction{Py_SetProgramName()} above) and some
  environment variables.  The returned string consists of a series of
  directory names separated by a platform dependent delimiter
  character.  The delimiter character is \character{:} on \UNIX and Mac OS X,
  \character{;} on Windows.  The returned string points into
  static storage; the caller should not modify its value.  The value
  is available to Python code as the list
  \code{sys.path}\withsubitem{(in module sys)}{\ttindex{path}}, which
  may be modified to change the future search path for loaded
  modules.

  % XXX should give the exact rules
\end{cfuncdesc}

\begin{cfuncdesc}{const char*}{Py_GetVersion}{}
  Return the version of this Python interpreter.  This is a string
  that looks something like

\begin{verbatim}
"1.5 (#67, Dec 31 1997, 22:34:28) [GCC 2.7.2.2]"
\end{verbatim}

  The first word (up to the first space character) is the current
  Python version; the first three characters are the major and minor
  version separated by a period.  The returned string points into
  static storage; the caller should not modify its value.  The value
  is available to Python code as \code{sys.version}.
  \withsubitem{(in module sys)}{\ttindex{version}}
\end{cfuncdesc}

\begin{cfuncdesc}{const char*}{Py_GetPlatform}{}
  Return the platform identifier for the current platform.  On \UNIX,
  this is formed from the ``official'' name of the operating system,
  converted to lower case, followed by the major revision number;
  e.g., for Solaris 2.x, which is also known as SunOS 5.x, the value
  is \code{'sunos5'}.  On Mac OS X, it is \code{'darwin'}.  On Windows,
  it is \code{'win'}.  The returned string points into static storage;
  the caller should not modify its value.  The value is available to
  Python code as \code{sys.platform}.
  \withsubitem{(in module sys)}{\ttindex{platform}}
\end{cfuncdesc}

\begin{cfuncdesc}{const char*}{Py_GetCopyright}{}
  Return the official copyright string for the current Python version,
  for example

  \code{'Copyright 1991-1995 Stichting Mathematisch Centrum, Amsterdam'}

  The returned string points into static storage; the caller should
  not modify its value.  The value is available to Python code as
  \code{sys.copyright}.
  \withsubitem{(in module sys)}{\ttindex{copyright}}
\end{cfuncdesc}

\begin{cfuncdesc}{const char*}{Py_GetCompiler}{}
  Return an indication of the compiler used to build the current
  Python version, in square brackets, for example:

\begin{verbatim}
"[GCC 2.7.2.2]"
\end{verbatim}

  The returned string points into static storage; the caller should
  not modify its value.  The value is available to Python code as part
  of the variable \code{sys.version}.
  \withsubitem{(in module sys)}{\ttindex{version}}
\end{cfuncdesc}

\begin{cfuncdesc}{const char*}{Py_GetBuildInfo}{}
  Return information about the sequence number and build date and time 
  of the current Python interpreter instance, for example

\begin{verbatim}
"#67, Aug  1 1997, 22:34:28"
\end{verbatim}

  The returned string points into static storage; the caller should
  not modify its value.  The value is available to Python code as part
  of the variable \code{sys.version}.
  \withsubitem{(in module sys)}{\ttindex{version}}
\end{cfuncdesc}

\begin{cfuncdesc}{int}{PySys_SetArgv}{int argc, char **argv}
  Set \code{sys.argv} based on \var{argc} and \var{argv}.  These
  parameters are similar to those passed to the program's
  \cfunction{main()}\ttindex{main()} function with the difference that
  the first entry should refer to the script file to be executed
  rather than the executable hosting the Python interpreter.  If there
  isn't a script that will be run, the first entry in \var{argv} can
  be an empty string.  If this function fails to initialize
  \code{sys.argv}, a fatal condition is signalled using
  \cfunction{Py_FatalError()}\ttindex{Py_FatalError()}.
  \withsubitem{(in module sys)}{\ttindex{argv}}
  % XXX impl. doesn't seem consistent in allowing 0/NULL for the params; 
  % check w/ Guido.
\end{cfuncdesc}

% XXX Other PySys thingies (doesn't really belong in this chapter)

\section{Thread State and the Global Interpreter Lock
         \label{threads}}

\index{global interpreter lock}
\index{interpreter lock}
\index{lock, interpreter}

The Python interpreter is not fully thread safe.  In order to support
multi-threaded Python programs, there's a global lock that must be
held by the current thread before it can safely access Python objects.
Without the lock, even the simplest operations could cause problems in
a multi-threaded program: for example, when two threads simultaneously
increment the reference count of the same object, the reference count
could end up being incremented only once instead of twice.

Therefore, the rule exists that only the thread that has acquired the
global interpreter lock may operate on Python objects or call Python/C
API functions.  In order to support multi-threaded Python programs,
the interpreter regularly releases and reacquires the lock --- by
default, every 100 bytecode instructions (this can be changed with
\withsubitem{(in module sys)}{\ttindex{setcheckinterval()}}
\function{sys.setcheckinterval()}).  The lock is also released and
reacquired around potentially blocking I/O operations like reading or
writing a file, so that other threads can run while the thread that
requests the I/O is waiting for the I/O operation to complete.

The Python interpreter needs to keep some bookkeeping information
separate per thread --- for this it uses a data structure called
\ctype{PyThreadState}\ttindex{PyThreadState}.  There's one global
variable, however: the pointer to the current
\ctype{PyThreadState}\ttindex{PyThreadState} structure.  While most
thread packages have a way to store ``per-thread global data,''
Python's internal platform independent thread abstraction doesn't
support this yet.  Therefore, the current thread state must be
manipulated explicitly.

This is easy enough in most cases.  Most code manipulating the global
interpreter lock has the following simple structure:

\begin{verbatim}
Save the thread state in a local variable.
Release the interpreter lock.
...Do some blocking I/O operation...
Reacquire the interpreter lock.
Restore the thread state from the local variable.
\end{verbatim}

This is so common that a pair of macros exists to simplify it:

\begin{verbatim}
Py_BEGIN_ALLOW_THREADS
...Do some blocking I/O operation...
Py_END_ALLOW_THREADS
\end{verbatim}

The
\csimplemacro{Py_BEGIN_ALLOW_THREADS}\ttindex{Py_BEGIN_ALLOW_THREADS}
macro opens a new block and declares a hidden local variable; the
\csimplemacro{Py_END_ALLOW_THREADS}\ttindex{Py_END_ALLOW_THREADS}
macro closes the block.  Another advantage of using these two macros
is that when Python is compiled without thread support, they are
defined empty, thus saving the thread state and lock manipulations.

When thread support is enabled, the block above expands to the
following code:

\begin{verbatim}
    PyThreadState *_save;

    _save = PyEval_SaveThread();
    ...Do some blocking I/O operation...
    PyEval_RestoreThread(_save);
\end{verbatim}

Using even lower level primitives, we can get roughly the same effect
as follows:

\begin{verbatim}
    PyThreadState *_save;

    _save = PyThreadState_Swap(NULL);
    PyEval_ReleaseLock();
    ...Do some blocking I/O operation...
    PyEval_AcquireLock();
    PyThreadState_Swap(_save);
\end{verbatim}

There are some subtle differences; in particular,
\cfunction{PyEval_RestoreThread()}\ttindex{PyEval_RestoreThread()} saves
and restores the value of the  global variable
\cdata{errno}\ttindex{errno}, since the lock manipulation does not
guarantee that \cdata{errno} is left alone.  Also, when thread support
is disabled,
\cfunction{PyEval_SaveThread()}\ttindex{PyEval_SaveThread()} and
\cfunction{PyEval_RestoreThread()} don't manipulate the lock; in this
case, \cfunction{PyEval_ReleaseLock()}\ttindex{PyEval_ReleaseLock()} and
\cfunction{PyEval_AcquireLock()}\ttindex{PyEval_AcquireLock()} are not
available.  This is done so that dynamically loaded extensions
compiled with thread support enabled can be loaded by an interpreter
that was compiled with disabled thread support.

The global interpreter lock is used to protect the pointer to the
current thread state.  When releasing the lock and saving the thread
state, the current thread state pointer must be retrieved before the
lock is released (since another thread could immediately acquire the
lock and store its own thread state in the global variable).
Conversely, when acquiring the lock and restoring the thread state,
the lock must be acquired before storing the thread state pointer.

Why am I going on with so much detail about this?  Because when
threads are created from C, they don't have the global interpreter
lock, nor is there a thread state data structure for them.  Such
threads must bootstrap themselves into existence, by first creating a
thread state data structure, then acquiring the lock, and finally
storing their thread state pointer, before they can start using the
Python/C API.  When they are done, they should reset the thread state
pointer, release the lock, and finally free their thread state data
structure.

Beginning with version 2.3, threads can now take advantage of the 
\cfunction{PyGILState_*()} functions to do all of the above
automatically.  The typical idiom for calling into Python from a C
thread is now:

\begin{verbatim}
    PyGILState_STATE gstate;
    gstate = PyGILState_Ensure();

    /* Perform Python actions here.  */
    result = CallSomeFunction();
    /* evaluate result */

    /* Release the thread. No Python API allowed beyond this point. */
    PyGILState_Release(gstate);
\end{verbatim}

Note that the \cfunction{PyGILState_*()} functions assume there is only
one global  interpreter (created automatically by
\cfunction{Py_Initialize()}).  Python still supports the creation of
additional interpreters  (using \cfunction{Py_NewInterpreter()}), but
mixing multiple interpreters and the \cfunction{PyGILState_*()} API is
unsupported.

\begin{ctypedesc}{PyInterpreterState}
  This data structure represents the state shared by a number of
  cooperating threads.  Threads belonging to the same interpreter
  share their module administration and a few other internal items.
  There are no public members in this structure.

  Threads belonging to different interpreters initially share nothing,
  except process state like available memory, open file descriptors
  and such.  The global interpreter lock is also shared by all
  threads, regardless of to which interpreter they belong.
\end{ctypedesc}

\begin{ctypedesc}{PyThreadState}
  This data structure represents the state of a single thread.  The
  only public data member is \ctype{PyInterpreterState
  *}\member{interp}, which points to this thread's interpreter state.
\end{ctypedesc}

\begin{cfuncdesc}{void}{PyEval_InitThreads}{}
  Initialize and acquire the global interpreter lock.  It should be
  called in the main thread before creating a second thread or
  engaging in any other thread operations such as
  \cfunction{PyEval_ReleaseLock()}\ttindex{PyEval_ReleaseLock()} or
  \code{PyEval_ReleaseThread(\var{tstate})}\ttindex{PyEval_ReleaseThread()}.
  It is not needed before calling
  \cfunction{PyEval_SaveThread()}\ttindex{PyEval_SaveThread()} or
  \cfunction{PyEval_RestoreThread()}\ttindex{PyEval_RestoreThread()}.

  This is a no-op when called for a second time.  It is safe to call
  this function before calling
  \cfunction{Py_Initialize()}\ttindex{Py_Initialize()}.

  When only the main thread exists, no lock operations are needed.
  This is a common situation (most Python programs do not use
  threads), and the lock operations slow the interpreter down a bit.
  Therefore, the lock is not created initially.  This situation is
  equivalent to having acquired the lock:  when there is only a single
  thread, all object accesses are safe.  Therefore, when this function
  initializes the lock, it also acquires it.  Before the Python
  \module{thread}\refbimodindex{thread} module creates a new thread,
  knowing that either it has the lock or the lock hasn't been created
  yet, it calls \cfunction{PyEval_InitThreads()}.  When this call
  returns, it is guaranteed that the lock has been created and that the
  calling thread has acquired it.

  It is \strong{not} safe to call this function when it is unknown
  which thread (if any) currently has the global interpreter lock.

  This function is not available when thread support is disabled at
  compile time.
\end{cfuncdesc}

\begin{cfuncdesc}{int}{PyEval_ThreadsInitialized}{}
  Returns a non-zero value if \cfunction{PyEval_InitThreads()} has been
  called.  This function can be called without holding the lock, and
  therefore can be used to avoid calls to the locking API when running
  single-threaded.  This function is not available when thread support
  is disabled at compile time. \versionadded{2.4}
\end{cfuncdesc}

\begin{cfuncdesc}{void}{PyEval_AcquireLock}{}
  Acquire the global interpreter lock.  The lock must have been
  created earlier.  If this thread already has the lock, a deadlock
  ensues.  This function is not available when thread support is
  disabled at compile time.
\end{cfuncdesc}

\begin{cfuncdesc}{void}{PyEval_ReleaseLock}{}
  Release the global interpreter lock.  The lock must have been
  created earlier.  This function is not available when thread support
  is disabled at compile time.
\end{cfuncdesc}

\begin{cfuncdesc}{void}{PyEval_AcquireThread}{PyThreadState *tstate}
  Acquire the global interpreter lock and set the current thread
  state to \var{tstate}, which should not be \NULL.  The lock must
  have been created earlier.  If this thread already has the lock,
  deadlock ensues.  This function is not available when thread support
  is disabled at compile time.
\end{cfuncdesc}

\begin{cfuncdesc}{void}{PyEval_ReleaseThread}{PyThreadState *tstate}
  Reset the current thread state to \NULL{} and release the global
  interpreter lock.  The lock must have been created earlier and must
  be held by the current thread.  The \var{tstate} argument, which
  must not be \NULL, is only used to check that it represents the
  current thread state --- if it isn't, a fatal error is reported.
  This function is not available when thread support is disabled at
  compile time.
\end{cfuncdesc}

\begin{cfuncdesc}{PyThreadState*}{PyEval_SaveThread}{}
  Release the interpreter lock (if it has been created and thread
  support is enabled) and reset the thread state to \NULL, returning
  the previous thread state (which is not \NULL).  If the lock has
  been created, the current thread must have acquired it.  (This
  function is available even when thread support is disabled at
  compile time.)
\end{cfuncdesc}

\begin{cfuncdesc}{void}{PyEval_RestoreThread}{PyThreadState *tstate}
  Acquire the interpreter lock (if it has been created and thread
  support is enabled) and set the thread state to \var{tstate}, which
  must not be \NULL.  If the lock has been created, the current thread
  must not have acquired it, otherwise deadlock ensues.  (This
  function is available even when thread support is disabled at
  compile time.)
\end{cfuncdesc}

The following macros are normally used without a trailing semicolon;
look for example usage in the Python source distribution.

\begin{csimplemacrodesc}{Py_BEGIN_ALLOW_THREADS}
  This macro expands to
  \samp{\{ PyThreadState *_save; _save = PyEval_SaveThread();}.
  Note that it contains an opening brace; it must be matched with a
  following \csimplemacro{Py_END_ALLOW_THREADS} macro.  See above for
  further discussion of this macro.  It is a no-op when thread support
  is disabled at compile time.
\end{csimplemacrodesc}

\begin{csimplemacrodesc}{Py_END_ALLOW_THREADS}
  This macro expands to \samp{PyEval_RestoreThread(_save); \}}.
  Note that it contains a closing brace; it must be matched with an
  earlier \csimplemacro{Py_BEGIN_ALLOW_THREADS} macro.  See above for
  further discussion of this macro.  It is a no-op when thread support
  is disabled at compile time.
\end{csimplemacrodesc}

\begin{csimplemacrodesc}{Py_BLOCK_THREADS}
  This macro expands to \samp{PyEval_RestoreThread(_save);}: it is
  equivalent to \csimplemacro{Py_END_ALLOW_THREADS} without the
  closing brace.  It is a no-op when thread support is disabled at
  compile time.
\end{csimplemacrodesc}

\begin{csimplemacrodesc}{Py_UNBLOCK_THREADS}
  This macro expands to \samp{_save = PyEval_SaveThread();}: it is
  equivalent to \csimplemacro{Py_BEGIN_ALLOW_THREADS} without the
  opening brace and variable declaration.  It is a no-op when thread
  support is disabled at compile time.
\end{csimplemacrodesc}

All of the following functions are only available when thread support
is enabled at compile time, and must be called only when the
interpreter lock has been created.

\begin{cfuncdesc}{PyInterpreterState*}{PyInterpreterState_New}{}
  Create a new interpreter state object.  The interpreter lock need
  not be held, but may be held if it is necessary to serialize calls
  to this function.
\end{cfuncdesc}

\begin{cfuncdesc}{void}{PyInterpreterState_Clear}{PyInterpreterState *interp}
  Reset all information in an interpreter state object.  The
  interpreter lock must be held.
\end{cfuncdesc}

\begin{cfuncdesc}{void}{PyInterpreterState_Delete}{PyInterpreterState *interp}
  Destroy an interpreter state object.  The interpreter lock need not
  be held.  The interpreter state must have been reset with a previous
  call to \cfunction{PyInterpreterState_Clear()}.
\end{cfuncdesc}

\begin{cfuncdesc}{PyThreadState*}{PyThreadState_New}{PyInterpreterState *interp}
  Create a new thread state object belonging to the given interpreter
  object.  The interpreter lock need not be held, but may be held if
  it is necessary to serialize calls to this function.
\end{cfuncdesc}

\begin{cfuncdesc}{void}{PyThreadState_Clear}{PyThreadState *tstate}
  Reset all information in a thread state object.  The interpreter lock
  must be held.
\end{cfuncdesc}

\begin{cfuncdesc}{void}{PyThreadState_Delete}{PyThreadState *tstate}
  Destroy a thread state object.  The interpreter lock need not be
  held.  The thread state must have been reset with a previous call to
  \cfunction{PyThreadState_Clear()}.
\end{cfuncdesc}

\begin{cfuncdesc}{PyThreadState*}{PyThreadState_Get}{}
  Return the current thread state.  The interpreter lock must be
  held.  When the current thread state is \NULL, this issues a fatal
  error (so that the caller needn't check for \NULL).
\end{cfuncdesc}

\begin{cfuncdesc}{PyThreadState*}{PyThreadState_Swap}{PyThreadState *tstate}
  Swap the current thread state with the thread state given by the
  argument \var{tstate}, which may be \NULL.  The interpreter lock
  must be held.
\end{cfuncdesc}

\begin{cfuncdesc}{PyObject*}{PyThreadState_GetDict}{}
  Return a dictionary in which extensions can store thread-specific
  state information.  Each extension should use a unique key to use to
  store state in the dictionary.  It is okay to call this function
  when no current thread state is available.
  If this function returns \NULL, no exception has been raised and the
  caller should assume no current thread state is available.
  \versionchanged[Previously this could only be called when a current
  thread is active, and \NULL{} meant that an exception was raised]{2.3}
\end{cfuncdesc}

\begin{cfuncdesc}{int}{PyThreadState_SetAsyncExc}{long id, PyObject *exc}
  Asynchronously raise an exception in a thread. 
  The \var{id} argument is the thread id of the target thread;
  \var{exc} is the exception object to be raised.
  This function does not steal any references to \var{exc}.
  To prevent naive misuse, you must write your own C extension 
  to call this.  Must be called with the GIL held. 
  Returns the number of thread states modified; if it returns a number 
  greater than one, you're in trouble, and you should call it again 
  with \var{exc} set to \constant{NULL} to revert the effect.
  This raises no exceptions.
  \versionadded{2.3}
\end{cfuncdesc}

\begin{cfuncdesc}{PyGILState_STATE}{PyGILState_Ensure}{}
Ensure that the current thread is ready to call the Python
C API regardless of the current state of Python, or of its
thread lock.  This may be called as many times as desired
by a thread as long as each call is matched with a call to 
\cfunction{PyGILState_Release()}.  
In general, other thread-related APIs may 
be used between \cfunction{PyGILState_Ensure()} and \cfunction{PyGILState_Release()} calls as long as the 
thread state is restored to its previous state before the Release().
For example, normal usage of the \csimplemacro{Py_BEGIN_ALLOW_THREADS}
and \csimplemacro{Py_END_ALLOW_THREADS} macros is acceptable.
    
The return value is an opaque "handle" to the thread state when
\cfunction{PyGILState_Acquire()} was called, and must be passed to
\cfunction{PyGILState_Release()} to ensure Python is left in the same
state. Even though recursive calls are allowed, these handles
\emph{cannot} be shared - each unique call to
\cfunction{PyGILState_Ensure} must save the handle for its call to
\cfunction{PyGILState_Release}.
    
When the function returns, the current thread will hold the GIL.
Failure is a fatal error.
  \versionadded{2.3}
\end{cfuncdesc}

\begin{cfuncdesc}{void}{PyGILState_Release}{PyGILState_STATE}
Release any resources previously acquired.  After this call, Python's
state will be the same as it was prior to the corresponding
\cfunction{PyGILState_Ensure} call (but generally this state will be
unknown to the caller, hence the use of the GILState API.)
    
Every call to \cfunction{PyGILState_Ensure()} must be matched by a call to 
\cfunction{PyGILState_Release()} on the same thread.
  \versionadded{2.3}
\end{cfuncdesc}


\section{Profiling and Tracing \label{profiling}}

\sectionauthor{Fred L. Drake, Jr.}{fdrake@acm.org}

The Python interpreter provides some low-level support for attaching
profiling and execution tracing facilities.  These are used for
profiling, debugging, and coverage analysis tools.

Starting with Python 2.2, the implementation of this facility was
substantially revised, and an interface from C was added.  This C
interface allows the profiling or tracing code to avoid the overhead
of calling through Python-level callable objects, making a direct C
function call instead.  The essential attributes of the facility have
not changed; the interface allows trace functions to be installed
per-thread, and the basic events reported to the trace function are
the same as had been reported to the Python-level trace functions in
previous versions.

\begin{ctypedesc}[Py_tracefunc]{int (*Py_tracefunc)(PyObject *obj,
                                PyFrameObject *frame, int what,
                                PyObject *arg)}
  The type of the trace function registered using
  \cfunction{PyEval_SetProfile()} and \cfunction{PyEval_SetTrace()}.
  The first parameter is the object passed to the registration
  function as \var{obj}, \var{frame} is the frame object to which the
  event pertains, \var{what} is one of the constants
  \constant{PyTrace_CALL}, \constant{PyTrace_EXCEPTION},
  \constant{PyTrace_LINE}, \constant{PyTrace_RETURN},
  \constant{PyTrace_C_CALL}, \constant{PyTrace_C_EXCEPTION},
  or \constant{PyTrace_C_RETURN}, and \var{arg}
  depends on the value of \var{what}:

  \begin{tableii}{l|l}{constant}{Value of \var{what}}{Meaning of \var{arg}}
    \lineii{PyTrace_CALL}{Always \NULL.}
    \lineii{PyTrace_EXCEPTION}{Exception information as returned by
                            \function{sys.exc_info()}.}
    \lineii{PyTrace_LINE}{Always \NULL.}
    \lineii{PyTrace_RETURN}{Value being returned to the caller.}
    \lineii{PyTrace_C_CALL}{Name of function being called.}
    \lineii{PyTrace_C_EXCEPTION}{Always \NULL.}
    \lineii{PyTrace_C_RETURN}{Always \NULL.}
  \end{tableii}
\end{ctypedesc}

\begin{cvardesc}{int}{PyTrace_CALL}
  The value of the \var{what} parameter to a \ctype{Py_tracefunc}
  function when a new call to a function or method is being reported,
  or a new entry into a generator.  Note that the creation of the
  iterator for a generator function is not reported as there is no
  control transfer to the Python bytecode in the corresponding frame.
\end{cvardesc}

\begin{cvardesc}{int}{PyTrace_EXCEPTION}
  The value of the \var{what} parameter to a \ctype{Py_tracefunc}
  function when an exception has been raised.  The callback function
  is called with this value for \var{what} when after any bytecode is
  processed after which the exception becomes set within the frame
  being executed.  The effect of this is that as exception propagation
  causes the Python stack to unwind, the callback is called upon
  return to each frame as the exception propagates.  Only trace
  functions receives these events; they are not needed by the
  profiler.
\end{cvardesc}

\begin{cvardesc}{int}{PyTrace_LINE}
  The value passed as the \var{what} parameter to a trace function
  (but not a profiling function) when a line-number event is being
  reported.
\end{cvardesc}

\begin{cvardesc}{int}{PyTrace_RETURN}
  The value for the \var{what} parameter to \ctype{Py_tracefunc}
  functions when a call is returning without propagating an exception.
\end{cvardesc}

\begin{cvardesc}{int}{PyTrace_C_CALL}
  The value for the \var{what} parameter to \ctype{Py_tracefunc}
  functions when a C function is about to be called.
\end{cvardesc}

\begin{cvardesc}{int}{PyTrace_C_EXCEPTION}
  The value for the \var{what} parameter to \ctype{Py_tracefunc}
  functions when a C function has thrown an exception.
\end{cvardesc}

\begin{cvardesc}{int}{PyTrace_C_RETURN}
  The value for the \var{what} parameter to \ctype{Py_tracefunc}
  functions when a C function has returned.
\end{cvardesc}

\begin{cfuncdesc}{void}{PyEval_SetProfile}{Py_tracefunc func, PyObject *obj}
  Set the profiler function to \var{func}.  The \var{obj} parameter is
  passed to the function as its first parameter, and may be any Python
  object, or \NULL.  If the profile function needs to maintain state,
  using a different value for \var{obj} for each thread provides a
  convenient and thread-safe place to store it.  The profile function
  is called for all monitored events except the line-number events.
\end{cfuncdesc}

\begin{cfuncdesc}{void}{PyEval_SetTrace}{Py_tracefunc func, PyObject *obj}
  Set the tracing function to \var{func}.  This is similar to
  \cfunction{PyEval_SetProfile()}, except the tracing function does
  receive line-number events.
\end{cfuncdesc}


\section{Advanced Debugger Support \label{advanced-debugging}}
\sectionauthor{Fred L. Drake, Jr.}{fdrake@acm.org}

These functions are only intended to be used by advanced debugging
tools.

\begin{cfuncdesc}{PyInterpreterState*}{PyInterpreterState_Head}{}
  Return the interpreter state object at the head of the list of all
  such objects.
  \versionadded{2.2}
\end{cfuncdesc}

\begin{cfuncdesc}{PyInterpreterState*}{PyInterpreterState_Next}{PyInterpreterState *interp}
  Return the next interpreter state object after \var{interp} from the
  list of all such objects.
  \versionadded{2.2}
\end{cfuncdesc}

\begin{cfuncdesc}{PyThreadState *}{PyInterpreterState_ThreadHead}{PyInterpreterState *interp}
  Return the a pointer to the first \ctype{PyThreadState} object in
  the list of threads associated with the interpreter \var{interp}.
  \versionadded{2.2}
\end{cfuncdesc}

\begin{cfuncdesc}{PyThreadState*}{PyThreadState_Next}{PyThreadState *tstate}
  Return the next thread state object after \var{tstate} from the list
  of all such objects belonging to the same \ctype{PyInterpreterState}
  object.
  \versionadded{2.2}
\end{cfuncdesc}

\chapter{Memory Management \label{memory}}
\sectionauthor{Vladimir Marangozov}{Vladimir.Marangozov@inrialpes.fr}


\section{Overview \label{memoryOverview}}

Memory management in Python involves a private heap containing all
Python objects and data structures. The management of this private
heap is ensured internally by the \emph{Python memory manager}.  The
Python memory manager has different components which deal with various
dynamic storage management aspects, like sharing, segmentation,
preallocation or caching.

At the lowest level, a raw memory allocator ensures that there is
enough room in the private heap for storing all Python-related data
by interacting with the memory manager of the operating system. On top
of the raw memory allocator, several object-specific allocators
operate on the same heap and implement distinct memory management
policies adapted to the peculiarities of every object type. For
example, integer objects are managed differently within the heap than
strings, tuples or dictionaries because integers imply different
storage requirements and speed/space tradeoffs. The Python memory
manager thus delegates some of the work to the object-specific
allocators, but ensures that the latter operate within the bounds of
the private heap.

It is important to understand that the management of the Python heap
is performed by the interpreter itself and that the user has no
control over it, even if she regularly manipulates object pointers to
memory blocks inside that heap.  The allocation of heap space for
Python objects and other internal buffers is performed on demand by
the Python memory manager through the Python/C API functions listed in
this document.

To avoid memory corruption, extension writers should never try to
operate on Python objects with the functions exported by the C
library: \cfunction{malloc()}\ttindex{malloc()},
\cfunction{calloc()}\ttindex{calloc()},
\cfunction{realloc()}\ttindex{realloc()} and
\cfunction{free()}\ttindex{free()}.  This will result in 
mixed calls between the C allocator and the Python memory manager
with fatal consequences, because they implement different algorithms
and operate on different heaps.  However, one may safely allocate and
release memory blocks with the C library allocator for individual
purposes, as shown in the following example:

\begin{verbatim}
    PyObject *res;
    char *buf = (char *) malloc(BUFSIZ); /* for I/O */

    if (buf == NULL)
        return PyErr_NoMemory();
    ...Do some I/O operation involving buf...
    res = PyString_FromString(buf);
    free(buf); /* malloc'ed */
    return res;
\end{verbatim}

In this example, the memory request for the I/O buffer is handled by
the C library allocator. The Python memory manager is involved only
in the allocation of the string object returned as a result.

In most situations, however, it is recommended to allocate memory from
the Python heap specifically because the latter is under control of
the Python memory manager. For example, this is required when the
interpreter is extended with new object types written in C. Another
reason for using the Python heap is the desire to \emph{inform} the
Python memory manager about the memory needs of the extension module.
Even when the requested memory is used exclusively for internal,
highly-specific purposes, delegating all memory requests to the Python
memory manager causes the interpreter to have a more accurate image of
its memory footprint as a whole. Consequently, under certain
circumstances, the Python memory manager may or may not trigger
appropriate actions, like garbage collection, memory compaction or
other preventive procedures. Note that by using the C library
allocator as shown in the previous example, the allocated memory for
the I/O buffer escapes completely the Python memory manager.


\section{Memory Interface \label{memoryInterface}}

The following function sets, modeled after the ANSI C standard,
but specifying  behavior when requesting zero bytes,
are available for allocating and releasing memory from the Python heap:


\begin{cfuncdesc}{void*}{PyMem_Malloc}{size_t n}
  Allocates \var{n} bytes and returns a pointer of type \ctype{void*}
  to the allocated memory, or \NULL{} if the request fails.
  Requesting zero bytes returns a distinct non-\NULL{} pointer if
  possible, as if \cfunction{PyMem_Malloc(1)} had been called instead.
  The memory will not have been initialized in any way.
\end{cfuncdesc}

\begin{cfuncdesc}{void*}{PyMem_Realloc}{void *p, size_t n}
  Resizes the memory block pointed to by \var{p} to \var{n} bytes.
  The contents will be unchanged to the minimum of the old and the new
  sizes. If \var{p} is \NULL, the call is equivalent to
  \cfunction{PyMem_Malloc(\var{n})}; else if \var{n} is equal to zero, the
  memory block is resized but is not freed, and the returned pointer
  is non-\NULL.  Unless \var{p} is \NULL, it must have been
  returned by a previous call to \cfunction{PyMem_Malloc()} or
  \cfunction{PyMem_Realloc()}.
\end{cfuncdesc}

\begin{cfuncdesc}{void}{PyMem_Free}{void *p}
  Frees the memory block pointed to by \var{p}, which must have been
  returned by a previous call to \cfunction{PyMem_Malloc()} or
  \cfunction{PyMem_Realloc()}.  Otherwise, or if
  \cfunction{PyMem_Free(p)} has been called before, undefined
  behavior occurs. If \var{p} is \NULL, no operation is performed.
\end{cfuncdesc}

The following type-oriented macros are provided for convenience.  Note 
that \var{TYPE} refers to any C type.

\begin{cfuncdesc}{\var{TYPE}*}{PyMem_New}{TYPE, size_t n}
  Same as \cfunction{PyMem_Malloc()}, but allocates \code{(\var{n} *
  sizeof(\var{TYPE}))} bytes of memory.  Returns a pointer cast to
  \ctype{\var{TYPE}*}.  The memory will not have been initialized in
  any way.
\end{cfuncdesc}

\begin{cfuncdesc}{\var{TYPE}*}{PyMem_Resize}{void *p, TYPE, size_t n}
  Same as \cfunction{PyMem_Realloc()}, but the memory block is resized
  to \code{(\var{n} * sizeof(\var{TYPE}))} bytes.  Returns a pointer
  cast to \ctype{\var{TYPE}*}.
\end{cfuncdesc}

\begin{cfuncdesc}{void}{PyMem_Del}{void *p}
  Same as \cfunction{PyMem_Free()}.
\end{cfuncdesc}

In addition, the following macro sets are provided for calling the
Python memory allocator directly, without involving the C API functions
listed above. However, note that their use does not preserve binary
compatibility across Python versions and is therefore deprecated in
extension modules.

\cfunction{PyMem_MALLOC()}, \cfunction{PyMem_REALLOC()}, \cfunction{PyMem_FREE()}.

\cfunction{PyMem_NEW()}, \cfunction{PyMem_RESIZE()}, \cfunction{PyMem_DEL()}.


\section{Examples \label{memoryExamples}}

Here is the example from section \ref{memoryOverview}, rewritten so
that the I/O buffer is allocated from the Python heap by using the
first function set:

\begin{verbatim}
    PyObject *res;
    char *buf = (char *) PyMem_Malloc(BUFSIZ); /* for I/O */

    if (buf == NULL)
        return PyErr_NoMemory();
    /* ...Do some I/O operation involving buf... */
    res = PyString_FromString(buf);
    PyMem_Free(buf); /* allocated with PyMem_Malloc */
    return res;
\end{verbatim}

The same code using the type-oriented function set:

\begin{verbatim}
    PyObject *res;
    char *buf = PyMem_New(char, BUFSIZ); /* for I/O */

    if (buf == NULL)
        return PyErr_NoMemory();
    /* ...Do some I/O operation involving buf... */
    res = PyString_FromString(buf);
    PyMem_Del(buf); /* allocated with PyMem_New */
    return res;
\end{verbatim}

Note that in the two examples above, the buffer is always
manipulated via functions belonging to the same set. Indeed, it
is required to use the same memory API family for a given
memory block, so that the risk of mixing different allocators is
reduced to a minimum. The following code sequence contains two errors,
one of which is labeled as \emph{fatal} because it mixes two different
allocators operating on different heaps.

\begin{verbatim}
char *buf1 = PyMem_New(char, BUFSIZ);
char *buf2 = (char *) malloc(BUFSIZ);
char *buf3 = (char *) PyMem_Malloc(BUFSIZ);
...
PyMem_Del(buf3);  /* Wrong -- should be PyMem_Free() */
free(buf2);       /* Right -- allocated via malloc() */
free(buf1);       /* Fatal -- should be PyMem_Del()  */
\end{verbatim}

In addition to the functions aimed at handling raw memory blocks from
the Python heap, objects in Python are allocated and released with
\cfunction{PyObject_New()}, \cfunction{PyObject_NewVar()} and
\cfunction{PyObject_Del()}, or with their corresponding macros
\cfunction{PyObject_NEW()}, \cfunction{PyObject_NEW_VAR()} and
\cfunction{PyObject_DEL()}.

These will be explained in the next chapter on defining and
implementing new object types in C.

\chapter{Defining New Types
        \label{defining-new-types}}
\sectionauthor{Michael Hudson}{mwh@python.net}
\sectionauthor{Dave Kuhlman}{dkuhlman@rexx.com}
\sectionauthor{Jim Fulton}{jim@zope.com}

As mentioned in the last chapter, Python allows the writer of an
extension module to define new types that can be manipulated from
Python code, much like strings and lists in core Python.

This is not hard; the code for all extension types follows a pattern,
but there are some details that you need to understand before you can
get started.

\section{The Basics
    \label{dnt-basics}}

The Python runtime sees all Python objects as variables of type
\ctype{PyObject*}.  A \ctype{PyObject} is not a very magnificent
object - it just contains the refcount and a pointer to the object's
``type object''.  This is where the action is; the type object
determines which (C) functions get called when, for instance, an
attribute gets looked up on an object or it is multiplied by another
object.  These C functions are called ``type methods'' to distinguish
them from things like \code{[].append} (which we call ``object
methods'').

So, if you want to define a new object type, you need to create a new
type object.

This sort of thing can only be explained by example, so here's a
minimal, but complete, module that defines a new type:

\verbatiminput{noddy.c}

Now that's quite a bit to take in at once, but hopefully bits will
seem familiar from the last chapter.

The first bit that will be new is:

\begin{verbatim}
typedef struct {
    PyObject_HEAD
} noddy_NoddyObject;
\end{verbatim}

This is what a Noddy object will contain---in this case, nothing more
than every Python object contains, namely a refcount and a pointer to a type
object.  These are the fields the \code{PyObject_HEAD} macro brings
in.  The reason for the macro is to standardize the layout and to
enable special debugging fields in debug builds.  Note that there is
no semicolon after the \code{PyObject_HEAD} macro; one is included in
the macro definition.  Be wary of adding one by accident; it's easy to
do from habit, and your compiler might not complain, but someone
else's probably will!  (On Windows, MSVC is known to call this an
error and refuse to compile the code.)

For contrast, let's take a look at the corresponding definition for
standard Python integers:

\begin{verbatim}
typedef struct {
    PyObject_HEAD
    long ob_ival;
} PyIntObject;
\end{verbatim}

Moving on, we come to the crunch --- the type object.

\begin{verbatim}
static PyTypeObject noddy_NoddyType = {
    PyObject_HEAD_INIT(NULL)
    0,                         /*ob_size*/
    "noddy.Noddy",             /*tp_name*/
    sizeof(noddy_NoddyObject), /*tp_basicsize*/
    0,                         /*tp_itemsize*/
    0,                         /*tp_dealloc*/
    0,                         /*tp_print*/
    0,                         /*tp_getattr*/
    0,                         /*tp_setattr*/
    0,                         /*tp_compare*/
    0,                         /*tp_repr*/
    0,                         /*tp_as_number*/
    0,                         /*tp_as_sequence*/
    0,                         /*tp_as_mapping*/
    0,                         /*tp_hash */
    0,                         /*tp_call*/
    0,                         /*tp_str*/
    0,                         /*tp_getattro*/
    0,                         /*tp_setattro*/
    0,                         /*tp_as_buffer*/
    Py_TPFLAGS_DEFAULT,        /*tp_flags*/
    "Noddy objects",           /* tp_doc */
    0,		               /* tp_traverse */
    0,		               /* tp_clear */
    0,		               /* tp_richcompare */
    0,		               /* tp_weaklistoffset */
    0,		               /* tp_iter */
    0,		               /* tp_iternext */
    0,		               /* tp_methods */
    0,                         /* tp_members */
    0,                         /* tp_getset */
    0,                         /* tp_base */
    0,                         /* tp_dict */
    0,                         /* tp_descr_get */
    0,                         /* tp_descr_set */
    0,                         /* tp_dictoffset */
    0,                         /* tp_init */
    0,                         /* tp_alloc */
    PyType_GenericNew,         /* tp_new */
};
\end{verbatim}

Now if you go and look up the definition of \ctype{PyTypeObject} in
\file{object.h} you'll see that it has many more fields that the
definition above.  The remaining fields will be filled with zeros by
the C compiler, and it's common practice to not specify them
explicitly unless you need them.  

This is so important that we're going to pick the top of it apart still
further:

\begin{verbatim}
    PyObject_HEAD_INIT(NULL)
\end{verbatim}

This line is a bit of a wart; what we'd like to write is:

\begin{verbatim}
    PyObject_HEAD_INIT(&PyType_Type)
\end{verbatim}

as the type of a type object is ``type'', but this isn't strictly
conforming C and some compilers complain.  Fortunately, this member
will be filled in for us by \cfunction{PyType_Ready()}.

\begin{verbatim}
    0,                          /* ob_size */
\end{verbatim}

The \member{ob_size} field of the header is not used; its presence in
the type structure is a historical artifact that is maintained for
binary compatibility with extension modules compiled for older
versions of Python.  Always set this field to zero.

\begin{verbatim}
    "noddy.Noddy",              /* tp_name */
\end{verbatim}

The name of our type.  This will appear in the default textual
representation of our objects and in some error messages, for example:

\begin{verbatim}
>>> "" + noddy.new_noddy()
Traceback (most recent call last):
  File "<stdin>", line 1, in ?
TypeError: cannot add type "noddy.Noddy" to string
\end{verbatim}

Note that the name is a dotted name that includes both the module name
and the name of the type within the module. The module in this case is 
\module{noddy} and the type is \class{Noddy}, so we set the type name
to \class{noddy.Noddy}.

\begin{verbatim}
    sizeof(noddy_NoddyObject),  /* tp_basicsize */
\end{verbatim}

This is so that Python knows how much memory to allocate when you call
\cfunction{PyObject_New}.

\begin{verbatim}
    0,                          /* tp_itemsize */
\end{verbatim}

This has to do with variable length objects like lists and strings.
Ignore this for now.

Skipping a number of type methods that we don't provide, we set the
class flags to \constant{Py_TPFLAGS_DEFAULT}. 

\begin{verbatim}
    Py_TPFLAGS_DEFAULT,        /*tp_flags*/
\end{verbatim}

All types should include this constant in their flags.  It enables all
of the members defined by the current version of Python.

We provide a doc string for the type in \member{tp_doc}.

\begin{verbatim}
    "Noddy objects",           /* tp_doc */
\end{verbatim}

Now we get into the type methods, the things that make your objects
different from the others.  We aren't going to implement any of these
in this version of the module.  We'll expand this example later to 
have more interesting behavior.  

For now, all we want to be able to do is to create new \class{Noddy}
objects. To enable object creation, we have to provide a
\member{tp_new} implementation. In this case, we can just use the
default implementation provided by the API function
\cfunction{PyType_GenericNew}.

\begin{verbatim}
    PyType_GenericNew,         /* tp_new */
\end{verbatim}

All the other type methods are \NULL, so we'll go over them later
--- that's for a later section!

Everything else in the file should be familiar, except for some code
in \cfunction{initnoddy}:

\begin{verbatim}
    if (PyType_Ready(&noddy_NoddyType) < 0)
        return;
\end{verbatim}

This initializes the \class{Noddy} type, filing in a number of
members, including \member{ob_type} that we initially set to \NULL.

\begin{verbatim}
    PyModule_AddObject(m, "Noddy", (PyObject *)&noddy_NoddyType);
\end{verbatim}

This adds the type to the module dictionary.  This allows us to create
\class{Noddy} instances by calling the \class{Noddy} class:

\begin{verbatim}
import noddy
mynoddy = noddy.Noddy()
\end{verbatim}

That's it!  All that remains is to build it; put the above code in a
file called \file{noddy.c} and

\begin{verbatim}
from distutils.core import setup, Extension
setup(name="noddy", version="1.0",
      ext_modules=[Extension("noddy", ["noddy.c"])])
\end{verbatim}

in a file called \file{setup.py}; then typing

\begin{verbatim}
$ python setup.py build
\end{verbatim} %$ <-- bow to font-lock  ;-(

at a shell should produce a file \file{noddy.so} in a subdirectory;
move to that directory and fire up Python --- you should be able to
\code{import noddy} and play around with Noddy objects.

That wasn't so hard, was it?

Of course, the current Noddy type is pretty uninteresting. It has no
data and doesn't do anything. It can't even be subclasses.

\subsection{Adding data and methods to the Basic example}
    
Let's expend the basic example to add some data and methods.  Let's
also make the type usable as a base class. We'll create
a new module, \module{noddy2} that adds these capabilities:

\verbatiminput{noddy2.c}

This version of the module has a number of changes.

We've added an extra include:

\begin{verbatim}
#include "structmember.h"
\end{verbatim}

This include provides declarations that we use to handle attributes,
as described a bit later.

The name of the \class{Noddy} object structure has been shortened to
\class{Noddy}.  The type object name has been shortened to
\class{NoddyType}.

The  \class{Noddy} type now has three data attributes, \var{first},
\var{last}, and \var{number}.  The \var{first} and \var{last}
variables are Python strings containing first and last names. The
\var{number} attribute is an integer.

The object structure is updated accordingly:

\begin{verbatim}
typedef struct {
    PyObject_HEAD
    PyObject *first;
    PyObject *last;
    int number;
} Noddy;
\end{verbatim}

Because we now have data to manage, we have to be more careful about
object allocation and deallocation.  At a minimum, we need a
deallocation method:

\begin{verbatim}
static void
Noddy_dealloc(Noddy* self)
{
    Py_XDECREF(self->first);
    Py_XDECREF(self->last);
    self->ob_type->tp_free(self);
}
\end{verbatim}

which is assigned to the \member{tp_dealloc} member:

\begin{verbatim}
    (destructor)Noddy_dealloc, /*tp_dealloc*/
\end{verbatim}

This method decrements the reference counts of the two Python
attributes. We use \cfunction{Py_XDECREF} here because the
\member{first} and \member{last} members could be \NULL.  It then
calls the \member{tp_free} member of the object's type to free the
object's memory.  Note that the object's type might not be
\class{NoddyType}, because the object may be an instance of a
subclass.

We want to make sure that the first and last names are initialized to
empty strings, so we provide a new method:

\begin{verbatim}
static PyObject *
Noddy_new(PyTypeObject *type, PyObject *args, PyObject *kwds)
{
    Noddy *self;

    self = (Noddy *)type->tp_alloc(type, 0);
    if (self != NULL) {
        self->first = PyString_FromString("");
        if (self->first == NULL)
          {
            Py_DECREF(self);
            return NULL;
          }
        
        self->last = PyString_FromString("");
        if (self->last == NULL)
          {
            Py_DECREF(self);
            return NULL;
          }

        self->number = 0;
    }

    return (PyObject *)self;
}
\end{verbatim}

and install it in the \member{tp_new} member:

\begin{verbatim}
    Noddy_new,                 /* tp_new */
\end{verbatim}

The new member is responsible for creating (as opposed to
initializing) objects of the type.  It is exposed in Python as the
\method{__new__} method.  See the paper titled ``Unifying types and
classes in Python'' for a detailed discussion of the \method{__new__}
method.  One reason to implement a new method is to assure the initial
values of instance variables.  In this case, we use the new method to
make sure that the initial values of the members \member{first} and
\member{last} are not \NULL. If we didn't care whether the initial
values were \NULL, we could have used \cfunction{PyType_GenericNew} as
our new method, as we did before.  \cfunction{PyType_GenericNew}
initializes all of the instance variable members to NULLs.

The new method is a static method that is passed the type being
instantiated and any arguments passed when the type was called,
and that returns the new object created. New methods always accept
positional and keyword arguments, but they often ignore the arguments,
leaving the argument handling to initializer methods. Note that if the
type supports subclassing, the type passed may not be the type being
defined.  The new method calls the tp_alloc slot to allocate memory.
We don't fill the \member{tp_alloc} slot ourselves. Rather
\cfunction{PyType_Ready()} fills it for us by inheriting it from our
base class, which is \class{object} by default.  Most types use the
default allocation.

We provide an initialization function:

\begin{verbatim}
static PyObject *
Noddy_init(Noddy *self, PyObject *args, PyObject *kwds)
{
    PyObject *first=NULL, *last=NULL;

    static char *kwlist[] = {"first", "last", "number", NULL};

    if (! PyArg_ParseTupleAndKeywords(args, kwds, "|OOi", kwlist, 
                                      &first, &last, 
                                      &self->number))
        return NULL; 

    if (first) {
        Py_XDECREF(self->first);
        Py_INCREF(first);
        self->first = first;
    }

    if (last) {
        Py_XDECREF(self->last);
        Py_INCREF(last);
        self->last = last;
    }

    Py_INCREF(Py_None);
    return Py_None;
}
\end{verbatim}

by filling the \member{tp_init} slot.

\begin{verbatim}
    (initproc)Noddy_init,         /* tp_init */
\end{verbatim}

The \member{tp_init} slot is exposed in Python as the
\method{__init__} method. It is used to initialize an object after
it's created. Unlike the new method, we can't guarantee that the
initializer is called.  The initializer isn't called when unpickling
objects and it can be overridden.  Our initializer accepts arguments
to provide initial values for our instance. Initializers always accept
positional and keyword arguments.

We want to want to expose our instance variables as attributes. There
are a number of ways to do that. The simplest way is to define member
definitions:

\begin{verbatim}
static PyMemberDef Noddy_members[] = {
    {"first", T_OBJECT_EX, offsetof(Noddy, first), 0,
     "first name"},
    {"last", T_OBJECT_EX, offsetof(Noddy, last), 0,
     "last name"},
    {"number", T_INT, offsetof(Noddy, number), 0,
     "noddy number"},
    {NULL}  /* Sentinel */
};
\end{verbatim}

and put the definitions in the \member{tp_members} slot:

\begin{verbatim}
    Noddy_members,             /* tp_members */
\end{verbatim}

Each member definition has a member name, type, offset, access flags
and documentation string. See the ``Generic Attribute Management''
section below for details.

A disadvantage of this approach is that it doesn't provide a way to
restrict the types of objects that can be assigned to the Python
attributes.  We expect the first and last names to be strings, but any
Python objects can be assigned.  Further, the attributes can be
deleted, setting the C pointers to \NULL.  Even though we can make
sure the members are initialized to non-\NULL values, the members can
be set to \NULL if the attributes are deleted.

We define a single method, \method{name}, that outputs the objects
name as the concatenation of the first and last names.  

\begin{verbatim}
static PyObject *
Noddy_name(Noddy* self)
{
    static PyObject *format = NULL;
    PyObject *args, *result;

    if (format == NULL) {
        format = PyString_FromString("%s %s");
        if (format == NULL)
            return NULL;
    }

    if (self->first == NULL) {
        PyErr_SetString(PyExc_AttributeError, "first");
        return NULL;
    }

    if (self->last == NULL) {
        PyErr_SetString(PyExc_AttributeError, "last");
        return NULL;
    }

    args = Py_BuildValue("OO", self->first, self->last);
    if (args == NULL)
        return NULL;

    result = PyString_Format(format, args);
    Py_DECREF(args);
    
    return result;
}
\end{verbatim}

The method is implemented as a C function that takes a \class{Noddy} (or
\class{Noddy} subclass) instance as the first argument.  Methods
always take an instance as the first argument. Methods often take
positional and keyword arguments as well, but in this cased we don't
take any and don't need to accept a positional argument tuple or
keyword argument dictionary. This method is equivalent to the Python
method:

\begin{verbatim}
    def name(self):
       return "%s %s" % (self.first, self.last)
\end{verbatim}

Note that we have to check for the possibility that our \member{first}
and \member{last} members are \NULL.  This is because they can be
deleted, in which case they are set to \NULL.  It would be better to
prevent deletion of these attributes and to restrict the attribute
values to be strings.  We'll see how to do that in the next section.

Now that we've defined the method, we need to create an array of
method definitions:

\begin{verbatim}
static PyMethodDef Noddy_methods[] = {
    {"name", (PyCFunction)Noddy_name, METH_NOARGS,
     "Return the name, combining the first and last name"
    },
    {NULL}  /* Sentinel */
};
\end{verbatim}

and assign them to the \member{tp_methods} slot:

\begin{verbatim}
    Noddy_methods,             /* tp_methods */
\end{verbatim}

Note that used the \constant{METH_NOARGS} flag to indicate that the
method is passed no arguments.

Finally, we'll make our type usable as a base class.  We've written
our methods carefully so far so that they don't make any assumptions
about the type of the object being created or used, so all we need to
do is to add the \constant{Py_TPFLAGS_BASETYPE} to our class flag
definition:

\begin{verbatim}
    Py_TPFLAGS_DEFAULT | Py_TPFLAGS_BASETYPE, /*tp_flags*/
\end{verbatim}

We rename \cfunction{initnoddy} to \cfunction{initnoddy2}
and update the module name passed to \cfunction{Py_InitModule3}.

Finally, we update our \file{setup.py} file to build the new module:

\begin{verbatim}
from distutils.core import setup, Extension
setup(name="noddy", version="1.0",
      ext_modules=[
         Extension("noddy", ["noddy.c"]),
         Extension("noddy2", ["noddy2.c"]),
         ])
\end{verbatim}

\subsection{Providing finer control over data attributes}

In this section, we'll provide finer control over how the
\member{first} and \member{last} attributes are set in the
\class{Noddy} example. In the previous version of our module, the
instance variables \member{first} and \member{last} could be set to
non-string values or even deleted. We want to make sure that these
attributes always contain strings.

\verbatiminput{noddy3.c}

To provide greater control, over the \member{first} and \member{last}
attributes, we'll use custom getter and setter functions.  Here are
the functions for getting and setting the \member{first} attribute:

\begin{verbatim}
Noddy_getfirst(Noddy *self, void *closure)
{
    Py_INCREF(self->first);
    return self->first;
}

static int
Noddy_setfirst(Noddy *self, PyObject *value, void *closure)
{
  if (value == NULL) {
    PyErr_SetString(PyExc_TypeError, "Cannot delete the first attribute");
    return -1;
  }
  
  if (! PyString_Check(value)) {
    PyErr_SetString(PyExc_TypeError, 
                    "The first attribute value must be a string");
    return -1;
  }
      
  Py_DECREF(self->first);
  Py_INCREF(value);
  self->first = value;    

  return 0;
}
\end{verbatim}

The getter function is passed a \class{Noddy} object and a
``closure'', which is void pointer. In this case, the closure is
ignored. (The closure supports an advanced usage in which definition
data is passed to the getter and setter. This could, for example, be
used to allow a single set of getter and setter functions that decide
the attribute to get or set based on data in the closure.)

The setter function is passed the \class{Noddy} object, the new value,
and the closure. The new value may be \NULL, in which case the
attribute is being deleted.  In our setter, we raise an error if the
attribute is deleted or if the attribute value is not a string.

We create an array of \ctype{PyGetSetDef} structures:

\begin{verbatim}
static PyGetSetDef Noddy_getseters[] = {
    {"first", 
     (getter)Noddy_getfirst, (setter)Noddy_setfirst,
     "first name",
     NULL},
    {"last", 
     (getter)Noddy_getlast, (setter)Noddy_setlast,
     "last name",
     NULL},
    {NULL}  /* Sentinel */
};
\end{verbatim}

and register it in the \member{tp_getset} slot:

\begin{verbatim}
    Noddy_getseters,           /* tp_getset */
\end{verbatim}

to register out attribute getters and setters.  

The last item in a \ctype{PyGetSetDef} structure is the closure
mentioned above. In this case, we aren't using the closure, so we just
pass \NULL.

We also remove the member definitions for these attributes:

\begin{verbatim}
static PyMemberDef Noddy_members[] = {
    {"number", T_INT, offsetof(Noddy, number), 0,
     "noddy number"},
    {NULL}  /* Sentinel */
};
\end{verbatim}

With these changes, we can assure that the \member{first} and
\member{last} members are never NULL so we can remove checks for \NULL
values in almost all cases. This means that most of the
\cfunction{Py_XDECREF} calls can be converted to \cfunction{Py_DECREF}
calls. The only place we can't change these calls is in the
deallocator, where there is the possibility that the initialization of
these members failed in the constructor.

We also rename the module initialization function and module name in
the initialization function, as we did before, and we add an extra
definition to the \file{setup.py} file.

\section{Type Methods
         \label{dnt-type-methods}}

This section aims to give a quick fly-by on the various type methods
you can implement and what they do.

Here is the definition of \ctype{PyTypeObject}, with some fields only
used in debug builds omitted:

\verbatiminput{typestruct.h}

Now that's a \emph{lot} of methods.  Don't worry too much though - if
you have a type you want to define, the chances are very good that you
will only implement a handful of these.

As you probably expect by now, we're going to go over this and give
more information about the various handlers.  We won't go in the order
they are defined in the structure, because there is a lot of
historical baggage that impacts the ordering of the fields; be sure
your type initializaion keeps the fields in the right order!  It's
often easiest to find an example that includes all the fields you need
(even if they're initialized to \code{0}) and then change the values
to suit your new type.

\begin{verbatim}
    char *tp_name; /* For printing */
\end{verbatim}

The name of the type - as mentioned in the last section, this will
appear in various places, almost entirely for diagnostic purposes.
Try to choose something that will be helpful in such a situation!

\begin{verbatim}
    int tp_basicsize, tp_itemsize; /* For allocation */
\end{verbatim}

These fields tell the runtime how much memory to allocate when new
objects of this type are created.  Python has some builtin support
for variable length structures (think: strings, lists) which is where
the \member{tp_itemsize} field comes in.  This will be dealt with
later.

\begin{verbatim}
    char *tp_doc;
\end{verbatim}

Here you can put a string (or its address) that you want returned when
the Python script references \code{obj.__doc__} to retrieve the
docstring.
   
Now we come to the basic type methods---the ones most extension types
will implement.


\subsection{Finalization and De-allocation}

\index{object!deallocation}
\index{deallocation, object}
\index{object!finalization}
\index{finalization, of objects}

\begin{verbatim}
    destructor tp_dealloc;
\end{verbatim}

This function is called when the reference count of the instance of
your type is reduced to zero and the Python interpreter wants to
reclaim it.  If your type has memory to free or other clean-up to
perform, put it here.  The object itself needs to be freed here as
well.  Here is an example of this function:

\begin{verbatim}
static void
newdatatype_dealloc(newdatatypeobject * obj)
{
    free(obj->obj_UnderlyingDatatypePtr);
    obj->ob_type->tp_free(self);
}
\end{verbatim}

One important requirement of the deallocator function is that it
leaves any pending exceptions alone.  This is important since
deallocators are frequently called as the interpreter unwinds the
Python stack; when the stack is unwound due to an exception (rather
than normal returns), nothing is done to protect the deallocators from
seeing that an exception has already been set.  Any actions which a
deallocator performs which may cause additional Python code to be
executed may detect that an exception has been set.  This can lead to
misleading errors from the interpreter.  The proper way to protect
against this is to save a pending exception before performing the
unsafe action, and restoring it when done.  This can be done using the
\cfunction{PyErr_Fetch()}\ttindex{PyErr_Fetch()} and
\cfunction{PyErr_Restore()}\ttindex{PyErr_Restore()} functions:

\begin{verbatim}
static void
my_dealloc(PyObject *obj)
{
    MyObject *self = (MyObject *) obj;
    PyObject *cbresult;

    if (self->my_callback != NULL) {
        PyObject *err_type, *err_value, *err_traceback;
        int have_error = PyErr_Occurred() ? 1 : 0;

        if (have_error)
            PyErr_Fetch(&err_type, &err_value, &err_traceback);

        cbresult = PyObject_CallObject(self->my_callback, NULL);
        if (cbresult == NULL)
            PyErr_WriteUnraisable();
        else
            Py_DECREF(cbresult);

        if (have_error)
            PyErr_Restore(err_type, err_value, err_traceback);

        Py_DECREF(self->my_callback);
    }
    obj->ob_type->tp_free(self);
}
\end{verbatim}


\subsection{Object Presentation}

In Python, there are three ways to generate a textual representation
of an object: the \function{repr()}\bifuncindex{repr} function (or
equivalent backtick syntax), the \function{str()}\bifuncindex{str}
function, and the \keyword{print} statement.  For most objects, the
\keyword{print} statement is equivalent to the \function{str()}
function, but it is possible to special-case printing to a
\ctype{FILE*} if necessary; this should only be done if efficiency is
identified as a problem and profiling suggests that creating a
temporary string object to be written to a file is too expensive.

These handlers are all optional, and most types at most need to
implement the \member{tp_str} and \member{tp_repr} handlers.

\begin{verbatim}
    reprfunc tp_repr;
    reprfunc tp_str;
    printfunc tp_print;
\end{verbatim}

The \member{tp_repr} handler should return a string object containing
a representation of the instance for which it is called.  Here is a
simple example:

\begin{verbatim}
static PyObject *
newdatatype_repr(newdatatypeobject * obj)
{
    return PyString_FromFormat("Repr-ified_newdatatype{{size:\%d}}",
                               obj->obj_UnderlyingDatatypePtr->size);
}
\end{verbatim}

If no \member{tp_repr} handler is specified, the interpreter will
supply a representation that uses the type's \member{tp_name} and a
uniquely-identifying value for the object.

The \member{tp_str} handler is to \function{str()} what the
\member{tp_repr} handler described above is to \function{repr()}; that
is, it is called when Python code calls \function{str()} on an
instance of your object.  Its implementation is very similar to the
\member{tp_repr} function, but the resulting string is intended for
human consumption.  If \member{tp_str} is not specified, the
\member{tp_repr} handler is used instead.

Here is a simple example:

\begin{verbatim}
static PyObject *
newdatatype_str(newdatatypeobject * obj)
{
    return PyString_FromFormat("Stringified_newdatatype{{size:\%d}}",
                               obj->obj_UnderlyingDatatypePtr->size);
}
\end{verbatim}

The print function will be called whenever Python needs to "print" an
instance of the type.  For example, if 'node' is an instance of type
TreeNode, then the print function is called when Python code calls:

\begin{verbatim}
print node
\end{verbatim}

There is a flags argument and one flag, \constant{Py_PRINT_RAW}, and
it suggests that you print without string quotes and possibly without
interpreting escape sequences.

The print function receives a file object as an argument. You will
likely want to write to that file object.

Here is a sampe print function:

\begin{verbatim}
static int
newdatatype_print(newdatatypeobject *obj, FILE *fp, int flags)
{
    if (flags & Py_PRINT_RAW) {
        fprintf(fp, "<{newdatatype object--size: %d}>",
                obj->obj_UnderlyingDatatypePtr->size);
    }
    else {
        fprintf(fp, "\"<{newdatatype object--size: %d}>\"",
                obj->obj_UnderlyingDatatypePtr->size);
    }
    return 0;
}
\end{verbatim}


\subsection{Attribute Management}

For every object which can support attributes, the corresponding type
must provide the functions that control how the attributes are
resolved.  There needs to be a function which can retrieve attributes
(if any are defined), and another to set attributes (if setting
attributes is allowed).  Removing an attribute is a special case, for
which the new value passed to the handler is \NULL.

Python supports two pairs of attribute handlers; a type that supports
attributes only needs to implement the functions for one pair.  The
difference is that one pair takes the name of the attribute as a
\ctype{char*}, while the other accepts a \ctype{PyObject*}.  Each type
can use whichever pair makes more sense for the implementation's
convenience.

\begin{verbatim}
    getattrfunc  tp_getattr;        /* char * version */
    setattrfunc  tp_setattr;
    /* ... */
    getattrofunc tp_getattrofunc;   /* PyObject * version */
    setattrofunc tp_setattrofunc;
\end{verbatim}

If accessing attributes of an object is always a simple operation
(this will be explained shortly), there are generic implementations
which can be used to provide the \ctype{PyObject*} version of the
attribute management functions.  The actual need for type-specific
attribute handlers almost completely disappeared starting with Python
2.2, though there are many examples which have not been updated to use
some of the new generic mechanism that is available.


\subsubsection{Generic Attribute Management}

\versionadded{2.2}

Most extension types only use \emph{simple} attributes.  So, what
makes the attributes simple?  There are only a couple of conditions
that must be met:

\begin{enumerate}
  \item   The name of the attributes must be known when
          \cfunction{PyType_Ready()} is called.

  \item   No special processing is needed to record that an attribute
          was looked up or set, nor do actions need to be taken based
          on the value.
\end{enumerate}

Note that this list does not place any restrictions on the values of
the attributes, when the values are computed, or how relevant data is
stored.

When \cfunction{PyType_Ready()} is called, it uses three tables
referenced by the type object to create \emph{descriptors} which are
placed in the dictionary of the type object.  Each descriptor controls
access to one attribute of the instance object.  Each of the tables is
optional; if all three are \NULL, instances of the type will only have
attributes that are inherited from their base type, and should leave
the \member{tp_getattro} and \member{tp_setattro} fields \NULL{} as
well, allowing the base type to handle attributes.

The tables are declared as three fields of the type object:

\begin{verbatim}
    struct PyMethodDef *tp_methods;
    struct PyMemberDef *tp_members;
    struct PyGetSetDef *tp_getset;
\end{verbatim}

If \member{tp_methods} is not \NULL, it must refer to an array of
\ctype{PyMethodDef} structures.  Each entry in the table is an
instance of this structure:

\begin{verbatim}
typedef struct PyMethodDef {
    char        *ml_name;       /* method name */
    PyCFunction  ml_meth;       /* implementation function */
    int	         ml_flags;      /* flags */
    char        *ml_doc;        /* docstring */
} PyMethodDef;
\end{verbatim}

One entry should be defined for each method provided by the type; no
entries are needed for methods inherited from a base type.  One
additional entry is needed at the end; it is a sentinel that marks the
end of the array.  The \member{ml_name} field of the sentinel must be
\NULL.

XXX Need to refer to some unified discussion of the structure fields,
shared with the next section.

The second table is used to define attributes which map directly to
data stored in the instance.  A variety of primitive C types are
supported, and access may be read-only or read-write.  The structures
in the table are defined as:

\begin{verbatim}
typedef struct PyMemberDef {
    char *name;
    int   type;
    int   offset;
    int   flags;
    char *doc;
} PyMemberDef;
\end{verbatim}

For each entry in the table, a descriptor will be constructed and
added to the type which will be able to extract a value from the
instance structure.  The \member{type} field should contain one of the
type codes defined in the \file{structmember.h} header; the value will
be used to determine how to convert Python values to and from C
values.  The \member{flags} field is used to store flags which control
how the attribute can be accessed.

XXX Need to move some of this to a shared section!

The following flag constants are defined in \file{structmember.h};
they may be combined using bitwise-OR.

\begin{tableii}{l|l}{constant}{Constant}{Meaning}
  \lineii{READONLY \ttindex{READONLY}}
         {Never writable.}
  \lineii{RO \ttindex{RO}}
         {Shorthand for \constant{READONLY}.}
  \lineii{READ_RESTRICTED \ttindex{READ_RESTRICTED}}
         {Not readable in restricted mode.}
  \lineii{WRITE_RESTRICTED \ttindex{WRITE_RESTRICTED}}
         {Not writable in restricted mode.}
  \lineii{RESTRICTED \ttindex{RESTRICTED}}
         {Not readable or writable in restricted mode.}
\end{tableii}

An interesting advantage of using the \member{tp_members} table to
build descriptors that are used at runtime is that any attribute
defined this way can have an associated docstring simply by providing
the text in the table.  An application can use the introspection API
to retrieve the descriptor from the class object, and get the
docstring using its \member{__doc__} attribute.

As with the \member{tp_methods} table, a sentinel entry with a
\member{name} value of \NULL{} is required.  


% XXX Descriptors need to be explained in more detail somewhere, but
% not here.
%
% Descriptor objects have two handler functions which correspond to
% the \member{tp_getattro} and \member{tp_setattro} handlers.  The
% \method{__get__()} handler is a function which is passed the
% descriptor, instance, and type objects, and returns the value of the
% attribute, or it returns \NULL{} and sets an exception.  The
% \method{__set__()} handler is passed the descriptor, instance, type,
% and new value;


\subsubsection{Type-specific Attribute Management}

For simplicity, only the \ctype{char*} version will be demonstrated
here; the type of the name parameter is the only difference between
the \ctype{char*} and \ctype{PyObject*} flavors of the interface.
This example effectively does the same thing as the generic example
above, but does not use the generic support added in Python 2.2.  The
value in showing this is two-fold: it demonstrates how basic attribute
management can be done in a way that is portable to older versions of
Python, and explains how the handler functions are called, so that if
you do need to extend their functionality, you'll understand what
needs to be done.

The \member{tp_getattr} handler is called when the object requires an
attribute look-up.  It is called in the same situations where the
\method{__getattr__()} method of a class would be called.

A likely way to handle this is (1) to implement a set of functions
(such as \cfunction{newdatatype_getSize()} and
\cfunction{newdatatype_setSize()} in the example below), (2) provide a
method table listing these functions, and (3) provide a getattr
function that returns the result of a lookup in that table.  The
method table uses the same structure as the \member{tp_methods} field
of the type object.

Here is an example:

\begin{verbatim}
static PyMethodDef newdatatype_methods[] = {
    {"getSize", (PyCFunction)newdatatype_getSize, METH_VARARGS,
     "Return the current size."},
    {"setSize", (PyCFunction)newdatatype_setSize, METH_VARARGS,
     "Set the size."},
    {NULL, NULL, 0, NULL}           /* sentinel */
};

static PyObject *
newdatatype_getattr(newdatatypeobject *obj, char *name)
{
    return Py_FindMethod(newdatatype_methods, (PyObject *)obj, name);
}
\end{verbatim}

The \member{tp_setattr} handler is called when the
\method{__setattr__()} or \method{__delattr__()} method of a class
instance would be called.  When an attribute should be deleted, the
third parameter will be \NULL.  Here is an example that simply raises
an exception; if this were really all you wanted, the
\member{tp_setattr} handler should be set to \NULL.
   
\begin{verbatim}
static int
newdatatype_setattr(newdatatypeobject *obj, char *name, PyObject *v)
{
    (void)PyErr_Format(PyExc_RuntimeError, "Read-only attribute: \%s", name);
    return -1;
}
\end{verbatim}


\subsection{Object Comparison}

\begin{verbatim}
    cmpfunc tp_compare;
\end{verbatim}

The \member{tp_compare} handler is called when comparisons are needed
and the object does not implement the specific rich comparison method
which matches the requested comparison.  (It is always used if defined
and the \cfunction{PyObject_Compare()} or \cfunction{PyObject_Cmp()}
functions are used, or if \function{cmp()} is used from Python.)
It is analogous to the \method{__cmp__()} method.  This function
should return \code{-1} if \var{obj1} is less than
\var{obj2}, \code{0} if they are equal, and \code{1} if
\var{obj1} is greater than
\var{obj2}.
(It was previously allowed to return arbitrary negative or positive
integers for less than and greater than, respectively; as of Python
2.2, this is no longer allowed.  In the future, other return values
may be assigned a different meaning.)

A \member{tp_compare} handler may raise an exception.  In this case it
should return a negative value.  The caller has to test for the
exception using \cfunction{PyErr_Occurred()}.


Here is a sample implementation:

\begin{verbatim}
static int
newdatatype_compare(newdatatypeobject * obj1, newdatatypeobject * obj2)
{
    long result;
 
    if (obj1->obj_UnderlyingDatatypePtr->size <
        obj2->obj_UnderlyingDatatypePtr->size) {
        result = -1;
    }
    else if (obj1->obj_UnderlyingDatatypePtr->size >
             obj2->obj_UnderlyingDatatypePtr->size) {
        result = 1;
    }
    else {
        result = 0;
    }
    return result;
}
\end{verbatim}


\subsection{Abstract Protocol Support}

Python supports a variety of \emph{abstract} `protocols;' the specific
interfaces provided to use these interfaces are documented in the
\citetitle[../api/api.html]{Python/C API Reference Manual} in the
chapter ``\ulink{Abstract Objects Layer}{../api/abstract.html}.''

A number of these abstract interfaces were defined early in the
development of the Python implementation.  In particular, the number,
mapping, and sequence protocols have been part of Python since the
beginning.  Other protocols have been added over time.  For protocols
which depend on several handler routines from the type implementation,
the older protocols have been defined as optional blocks of handlers
referenced by the type object.  For newer protocols there are
additional slots in the main type object, with a flag bit being set to
indicate that the slots are present and should be checked by the
interpreter.  (The flag bit does not indicate that the slot values are
non-\NULL. The flag may be set to indicate the presense of a slot,
but a slot may still be unfilled.)

\begin{verbatim}
    PyNumberMethods   tp_as_number;
    PySequenceMethods tp_as_sequence;
    PyMappingMethods  tp_as_mapping;
\end{verbatim}

If you wish your object to be able to act like a number, a sequence,
or a mapping object, then you place the address of a structure that
implements the C type \ctype{PyNumberMethods},
\ctype{PySequenceMethods}, or \ctype{PyMappingMethods}, respectively.
It is up to you to fill in this structure with appropriate values. You
can find examples of the use of each of these in the \file{Objects}
directory of the Python source distribution.


\begin{verbatim}
    hashfunc tp_hash;
\end{verbatim}

This function, if you choose to provide it, should return a hash
number for an instance of your datatype. Here is a moderately
pointless example:

\begin{verbatim}
static long
newdatatype_hash(newdatatypeobject *obj)
{
    long result;
    result = obj->obj_UnderlyingDatatypePtr->size;
    result = result * 3;
    return result;
}
\end{verbatim}

\begin{verbatim}
    ternaryfunc tp_call;
\end{verbatim}

This function is called when an instance of your datatype is "called",
for example, if \code{obj1} is an instance of your datatype and the Python
script contains \code{obj1('hello')}, the \member{tp_call} handler is
invoked.

This function takes three arguments:

\begin{enumerate}
  \item
    \var{arg1} is the instance of the datatype which is the subject of
    the call. If the call is \code{obj1('hello')}, then \var{arg1} is
    \code{obj1}.

  \item
    \var{arg2} is a tuple containing the arguments to the call.  You
    can use \cfunction{PyArg_ParseTuple()} to extract the arguments.

  \item
    \var{arg3} is a dictionary of keyword arguments that were passed.
    If this is non-\NULL{} and you support keyword arguments, use
    \cfunction{PyArg_ParseTupleAndKeywords()} to extract the
    arguments.  If you do not want to support keyword arguments and
    this is non-\NULL, raise a \exception{TypeError} with a message
    saying that keyword arguments are not supported.
\end{enumerate}
       
Here is a desultory example of the implementation of the call function.

\begin{verbatim}
/* Implement the call function.
 *    obj1 is the instance receiving the call.
 *    obj2 is a tuple containing the arguments to the call, in this
 *         case 3 strings.
 */
static PyObject *
newdatatype_call(newdatatypeobject *obj, PyObject *args, PyObject *other)
{
    PyObject *result;
    char *arg1;
    char *arg2;
    char *arg3;

    if (!PyArg_ParseTuple(args, "sss:call", &arg1, &arg2, &arg3)) {
        return NULL;
    }
    result = PyString_FromFormat(
        "Returning -- value: [\%d] arg1: [\%s] arg2: [\%s] arg3: [\%s]\n",
        obj->obj_UnderlyingDatatypePtr->size,
        arg1, arg2, arg3);
    printf("\%s", PyString_AS_STRING(result));
    return result;
}
\end{verbatim}

XXX some fields need to be added here...


\begin{verbatim}
    /* Added in release 2.2 */
    /* Iterators */
    getiterfunc tp_iter;
    iternextfunc tp_iternext;
\end{verbatim}

These functions provide support for the iterator protocol.  Any object
which wishes to support iteration over its contents (which may be
generated during iteration) must implement the \code{tp_iter}
handler.  Objects which are returned by a \code{tp_iter} handler must
implement both the \code{tp_iter} and \code{tp_iternext} handlers.
Both handlers take exactly one parameter, the instance for which they
are being called, and return a new reference.  In the case of an
error, they should set an exception and return \NULL.

For an object which represents an iterable collection, the
\code{tp_iter} handler must return an iterator object.  The iterator
object is responsible for maintaining the state of the iteration.  For
collections which can support multiple iterators which do not
interfere with each other (as lists and tuples do), a new iterator
should be created and returned.  Objects which can only be iterated
over once (usually due to side effects of iteration) should implement
this handler by returning a new reference to themselves, and should
also implement the \code{tp_iternext} handler.  File objects are an
example of such an iterator.

Iterator objects should implement both handlers.  The \code{tp_iter}
handler should return a new reference to the iterator (this is the
same as the \code{tp_iter} handler for objects which can only be
iterated over destructively).  The \code{tp_iternext} handler should
return a new reference to the next object in the iteration if there is
one.  If the iteration has reached the end, it may return \NULL{}
without setting an exception or it may set \exception{StopIteration};
avoiding the exception can yield slightly better performance.  If an
actual error occurs, it should set an exception and return \NULL.


\subsection{Supporting the Cycle Collector
            \label{example-cycle-support}}

This example shows only enough of the implementation of an extension
type to show how the garbage collector support needs to be added.  It
shows the definition of the object structure, the
\member{tp_traverse}, \member{tp_clear} and \member{tp_dealloc}
implementations, the type structure, and a constructor --- the module
initialization needed to export the constructor to Python is not shown
as there are no special considerations there for the collector.  To
make this interesting, assume that the module exposes ways for the
\member{container} field of the object to be modified.  Note that
since no checks are made on the type of the object used to initialize
\member{container}, we have to assume that it may be a container.

\verbatiminput{cycle-gc.c}

Full details on the APIs related to the cycle detector are in
\ulink{Supporting Cyclic Garbarge
Collection}{../api/supporting-cycle-detection.html} in the
\citetitle[../api/api.html]{Python/C API Reference Manual}.


\subsection{More Suggestions}

Remember that you can omit most of these functions, in which case you
provide \code{0} as a value.

In the \file{Objects} directory of the Python source distribution,
there is a file \file{xxobject.c}, which is intended to be used as a
template for the implementation of new types.  One useful strategy
for implementing a new type is to copy and rename this file, then
read the instructions at the top of it.

There are type definitions for each of the functions you must
provide.  They are in \file{object.h} in the Python include
directory that comes with the source distribution of Python.

In order to learn how to implement any specific method for your new
datatype, do the following: Download and unpack the Python source
distribution.  Go the the \file{Objects} directory, then search the
C source files for \code{tp_} plus the function you want (for
example, \code{tp_print} or \code{tp_compare}).  You will find
examples of the function you want to implement.

When you need to verify that the type of an object is indeed the
object you are implementing and if you use xxobject.c as an starting
template for your implementation, then there is a macro defined for
this purpose. The macro definition will look something like this:

\begin{verbatim}
#define is_newdatatypeobject(v)  ((v)->ob_type == &Newdatatypetype)
\end{verbatim}

And, a sample of its use might be something like the following:

\begin{verbatim}
    if (!is_newdatatypeobject(objp1) {
        PyErr_SetString(PyExc_TypeError, "arg #1 not a newdatatype");
        return NULL;
    }
\end{verbatim}



% \chapter{Debugging \label{debugging}}
%
% XXX Explain Py_DEBUG, Py_TRACE_REFS, Py_REF_DEBUG.


\appendix
\chapter{Reporting Bugs}
\label{reporting-bugs}

Python is a mature programming language which has established a
reputation for stability.  In order to maintain this reputation, the
developers would like to know of any deficiencies you find in Python
or its documentation.

All bug reports should be submitted via the Python Bug Tracker on
SourceForge (\url{http://sourceforge.net/bugs/?group_id=5470}).  The
bug tracker offers a Web form which allows pertinent information to be
entered and submitted to the developers.

Before submitting a report, please log into SourceForge if you are a
member; this will make it possible for the developers to contact you
for additional information if needed.  If you are not a SourceForge
member but would not mind the developers contacting you, you may
include your email address in your bug description.  In this case,
please realize that the information is publically available and cannot
be protected.

The first step in filing a report is to determine whether the problem
has already been reported.  The advantage in doing so, aside from
saving the developers time, is that you learn what has been done to
fix it; it may be that the problem has already been fixed for the next
release, or additional information is needed (in which case you are
welcome to provide it if you can!).  To do this, search the bug
database using the search box near the bottom of the page.

If the problem you're reporting is not already in the bug tracker, go
back to the Python Bug Tracker
(\url{http://sourceforge.net/bugs/?group_id=5470}).  Select the
``Submit a Bug'' link at the top of the page to open the bug reporting
form.

The submission form has a number of fields.  The only fields that are
required are the ``Summary'' and ``Details'' fields.  For the summary,
enter a \emph{very} short description of the problem; less than ten
words is good.  In the Details field, describe the problem in detail,
including what you expected to happen and what did happen.  Be sure to
include the version of Python you used, whether any extension modules
were involved, and what hardware and software platform you were using
(including version information as appropriate).

The only other field that you may want to set is the ``Category''
field, which allows you to place the bug report into a broad category
(such as ``Documentation'' or ``Library'').

Each bug report will be assigned to a developer who will determine
what needs to be done to correct the problem.  If you have a
SourceForge account and logged in to report the problem, you will
receive an update each time action is taken on the bug.


\begin{seealso}
  \seetitle[http://www-mice.cs.ucl.ac.uk/multimedia/software/documentation/ReportingBugs.html]{How
        to Report Bugs Effectively}{Article which goes into some
        detail about how to create a useful bug report.  This
        describes what kind of information is useful and why it is
        useful.}

  \seetitle[http://www.mozilla.org/quality/bug-writing-guidelines.html]{Bug
        Writing Guidelines}{Information about writing a good bug
        report.  Some of this is specific to the Mozilla project, but
        describes general good practices.}
\end{seealso}


\chapter{History and License}
\input{license}

\documentclass{manual}

\title{Python/C API Reference Manual}

\author{Guido van Rossum\\
	Fred L. Drake, Jr., editor}
\authoraddress{
	BeOpen PythonLabs\\
	E-mail: \email{python-docs@python.org}
}

\date{September 5, 2000}			% XXX update before release!
\release{2.0b1}


\makeindex			% tell \index to actually write the .idx file


\begin{document}

\maketitle

\ifhtml
\chapter*{Front Matter\label{front}}
\fi

\begin{small}
Copyright \copyright{} 2001 Python Software Foundation.
All rights reserved.

Copyright \copyright{} 2000 BeOpen.com.
All rights reserved.

Copyright \copyright{} 1995-2000 Corporation for National Research Initiatives.
All rights reserved.

Copyright \copyright{} 1991-1995 Stichting Mathematisch Centrum.
All rights reserved.

%%begin{latexonly}
\vskip 4mm
%%end{latexonly}

\centerline{\strong{BEOPEN.COM TERMS AND CONDITIONS FOR PYTHON 2.0}}

\centerline{\strong{BEOPEN PYTHON OPEN SOURCE LICENSE AGREEMENT VERSION 1}}

\begin{enumerate}

\item
This LICENSE AGREEMENT is between BeOpen.com (``BeOpen''), having an
office at 160 Saratoga Avenue, Santa Clara, CA 95051, and the
Individual or Organization (``Licensee'') accessing and otherwise
using this software in source or binary form and its associated
documentation (``the Software'').

\item
Subject to the terms and conditions of this BeOpen Python License
Agreement, BeOpen hereby grants Licensee a non-exclusive,
royalty-free, world-wide license to reproduce, analyze, test, perform
and/or display publicly, prepare derivative works, distribute, and
otherwise use the Software alone or in any derivative version,
provided, however, that the BeOpen Python License is retained in the
Software, alone or in any derivative version prepared by Licensee.

\item
BeOpen is making the Software available to Licensee on an ``AS IS''
basis.  BEOPEN MAKES NO REPRESENTATIONS OR WARRANTIES, EXPRESS OR
IMPLIED.  BY WAY OF EXAMPLE, BUT NOT LIMITATION, BEOPEN MAKES NO AND
DISCLAIMS ANY REPRESENTATION OR WARRANTY OF MERCHANTABILITY OR FITNESS
FOR ANY PARTICULAR PURPOSE OR THAT THE USE OF THE SOFTWARE WILL NOT
INFRINGE ANY THIRD PARTY RIGHTS.

\item
BEOPEN SHALL NOT BE LIABLE TO LICENSEE OR ANY OTHER USERS OF THE
SOFTWARE FOR ANY INCIDENTAL, SPECIAL, OR CONSEQUENTIAL DAMAGES OR LOSS
AS A RESULT OF USING, MODIFYING OR DISTRIBUTING THE SOFTWARE, OR ANY
DERIVATIVE THEREOF, EVEN IF ADVISED OF THE POSSIBILITY THEREOF.

\item
This License Agreement will automatically terminate upon a material
breach of its terms and conditions.

\item
This License Agreement shall be governed by and interpreted in all
respects by the law of the State of California, excluding conflict of
law provisions.  Nothing in this License Agreement shall be deemed to
create any relationship of agency, partnership, or joint venture
between BeOpen and Licensee.  This License Agreement does not grant
permission to use BeOpen trademarks or trade names in a trademark
sense to endorse or promote products or services of Licensee, or any
third party.  As an exception, the ``BeOpen Python'' logos available
at http://www.pythonlabs.com/logos.html may be used according to the
permissions granted on that web page.

\item
By copying, installing or otherwise using the software, Licensee
agrees to be bound by the terms and conditions of this License
Agreement.
\end{enumerate}


\centerline{\strong{CNRI OPEN SOURCE GPL-COMPATIBLE LICENSE AGREEMENT}}

Python 1.6.1 is made available subject to the terms and conditions in
CNRI's License Agreement.  This Agreement together with Python 1.6.1 may
be located on the Internet using the following unique, persistent
identifier (known as a handle): 1895.22/1013.  This Agreement may also
be obtained from a proxy server on the Internet using the following
URL: \url{http://hdl.handle.net/1895.22/1013}.


\centerline{\strong{CWI PERMISSIONS STATEMENT AND DISCLAIMER}}

Copyright \copyright{} 1991 - 1995, Stichting Mathematisch Centrum
Amsterdam, The Netherlands.  All rights reserved.

Permission to use, copy, modify, and distribute this software and its
documentation for any purpose and without fee is hereby granted,
provided that the above copyright notice appear in all copies and that
both that copyright notice and this permission notice appear in
supporting documentation, and that the name of Stichting Mathematisch
Centrum or CWI not be used in advertising or publicity pertaining to
distribution of the software without specific, written prior
permission.

STICHTING MATHEMATISCH CENTRUM DISCLAIMS ALL WARRANTIES WITH REGARD TO
THIS SOFTWARE, INCLUDING ALL IMPLIED WARRANTIES OF MERCHANTABILITY AND
FITNESS, IN NO EVENT SHALL STICHTING MATHEMATISCH CENTRUM BE LIABLE
FOR ANY SPECIAL, INDIRECT OR CONSEQUENTIAL DAMAGES OR ANY DAMAGES
WHATSOEVER RESULTING FROM LOSS OF USE, DATA OR PROFITS, WHETHER IN AN
ACTION OF CONTRACT, NEGLIGENCE OR OTHER TORTIOUS ACTION, ARISING OUT
OF OR IN CONNECTION WITH THE USE OR PERFORMANCE OF THIS SOFTWARE.
\end{small}


\begin{abstract}

\noindent
This manual documents the API used by C and \Cpp{} programmers who
want to write extension modules or embed Python.  It is a companion to
\citetitle[../ext/ext.html]{Extending and Embedding the Python
Interpreter}, which describes the general principles of extension
writing but does not document the API functions in detail.

\warning{The current version of this document is incomplete.  I hope
that it is nevertheless useful.  I will continue to work on it, and
release new versions from time to time, independent from Python source
code releases.}

\end{abstract}

\tableofcontents


\chapter{Introduction \label{intro}}


The Application Programmer's Interface to Python gives C and
\Cpp{} programmers access to the Python interpreter at a variety of
levels.  The API is equally usable from \Cpp, but for brevity it is
generally referred to as the Python/C API.  There are two
fundamentally different reasons for using the Python/C API.  The first
reason is to write \emph{extension modules} for specific purposes;
these are C modules that extend the Python interpreter.  This is
probably the most common use.  The second reason is to use Python as a
component in a larger application; this technique is generally
referred to as \dfn{embedding} Python in an application.

Writing an extension module is a relatively well-understood process, 
where a ``cookbook'' approach works well.  There are several tools 
that automate the process to some extent.  While people have embedded 
Python in other applications since its early existence, the process of 
embedding Python is less straightforward than writing an extension.  

Many API functions are useful independent of whether you're embedding 
or extending Python; moreover, most applications that embed Python 
will need to provide a custom extension as well, so it's probably a 
good idea to become familiar with writing an extension before 
attempting to embed Python in a real application.


\section{Include Files \label{includes}}

All function, type and macro definitions needed to use the Python/C
API are included in your code by the following line:

\begin{verbatim}
#include "Python.h"
\end{verbatim}

This implies inclusion of the following standard headers:
\code{<stdio.h>}, \code{<string.h>}, \code{<errno.h>},
\code{<limits.h>}, and \code{<stdlib.h>} (if available).

\begin{notice}[warning]
  Since Python may define some pre-processor definitions which affect
  the standard headers on some systems, you \emph{must} include
  \file{Python.h} before any standard headers are included.
\end{notice}

All user visible names defined by Python.h (except those defined by
the included standard headers) have one of the prefixes \samp{Py} or
\samp{_Py}.  Names beginning with \samp{_Py} are for internal use by
the Python implementation and should not be used by extension writers.
Structure member names do not have a reserved prefix.

\strong{Important:} user code should never define names that begin
with \samp{Py} or \samp{_Py}.  This confuses the reader, and
jeopardizes the portability of the user code to future Python
versions, which may define additional names beginning with one of
these prefixes.

The header files are typically installed with Python.  On \UNIX, these 
are located in the directories
\file{\envvar{prefix}/include/python\var{version}/} and
\file{\envvar{exec_prefix}/include/python\var{version}/}, where
\envvar{prefix} and \envvar{exec_prefix} are defined by the
corresponding parameters to Python's \program{configure} script and
\var{version} is \code{sys.version[:3]}.  On Windows, the headers are
installed in \file{\envvar{prefix}/include}, where \envvar{prefix} is
the installation directory specified to the installer.

To include the headers, place both directories (if different) on your
compiler's search path for includes.  Do \emph{not} place the parent
directories on the search path and then use
\samp{\#include <python\shortversion/Python.h>}; this will break on
multi-platform builds since the platform independent headers under
\envvar{prefix} include the platform specific headers from
\envvar{exec_prefix}.

\Cpp{} users should note that though the API is defined entirely using
C, the header files do properly declare the entry points to be
\code{extern "C"}, so there is no need to do anything special to use
the API from \Cpp.


\section{Objects, Types and Reference Counts \label{objects}}

Most Python/C API functions have one or more arguments as well as a
return value of type \ctype{PyObject*}.  This type is a pointer
to an opaque data type representing an arbitrary Python
object.  Since all Python object types are treated the same way by the
Python language in most situations (e.g., assignments, scope rules,
and argument passing), it is only fitting that they should be
represented by a single C type.  Almost all Python objects live on the
heap: you never declare an automatic or static variable of type
\ctype{PyObject}, only pointer variables of type \ctype{PyObject*} can 
be declared.  The sole exception are the type objects\obindex{type};
since these must never be deallocated, they are typically static
\ctype{PyTypeObject} objects.

All Python objects (even Python integers) have a \dfn{type} and a
\dfn{reference count}.  An object's type determines what kind of object 
it is (e.g., an integer, a list, or a user-defined function; there are 
many more as explained in the \citetitle[../ref/ref.html]{Python
Reference Manual}).  For each of the well-known types there is a macro
to check whether an object is of that type; for instance,
\samp{PyList_Check(\var{a})} is true if (and only if) the object
pointed to by \var{a} is a Python list.


\subsection{Reference Counts \label{refcounts}}

The reference count is important because today's computers have a 
finite (and often severely limited) memory size; it counts how many 
different places there are that have a reference to an object.  Such a 
place could be another object, or a global (or static) C variable, or 
a local variable in some C function.  When an object's reference count 
becomes zero, the object is deallocated.  If it contains references to 
other objects, their reference count is decremented.  Those other 
objects may be deallocated in turn, if this decrement makes their 
reference count become zero, and so on.  (There's an obvious problem 
with objects that reference each other here; for now, the solution is 
``don't do that.'')

Reference counts are always manipulated explicitly.  The normal way is 
to use the macro \cfunction{Py_INCREF()}\ttindex{Py_INCREF()} to
increment an object's reference count by one, and
\cfunction{Py_DECREF()}\ttindex{Py_DECREF()} to decrement it by  
one.  The \cfunction{Py_DECREF()} macro is considerably more complex
than the incref one, since it must check whether the reference count
becomes zero and then cause the object's deallocator to be called.
The deallocator is a function pointer contained in the object's type
structure.  The type-specific deallocator takes care of decrementing
the reference counts for other objects contained in the object if this
is a compound object type, such as a list, as well as performing any
additional finalization that's needed.  There's no chance that the
reference count can overflow; at least as many bits are used to hold
the reference count as there are distinct memory locations in virtual
memory (assuming \code{sizeof(long) >= sizeof(char*)}).  Thus, the
reference count increment is a simple operation.

It is not necessary to increment an object's reference count for every 
local variable that contains a pointer to an object.  In theory, the 
object's reference count goes up by one when the variable is made to 
point to it and it goes down by one when the variable goes out of 
scope.  However, these two cancel each other out, so at the end the 
reference count hasn't changed.  The only real reason to use the 
reference count is to prevent the object from being deallocated as 
long as our variable is pointing to it.  If we know that there is at 
least one other reference to the object that lives at least as long as 
our variable, there is no need to increment the reference count 
temporarily.  An important situation where this arises is in objects 
that are passed as arguments to C functions in an extension module 
that are called from Python; the call mechanism guarantees to hold a 
reference to every argument for the duration of the call.

However, a common pitfall is to extract an object from a list and
hold on to it for a while without incrementing its reference count.
Some other operation might conceivably remove the object from the
list, decrementing its reference count and possible deallocating it.
The real danger is that innocent-looking operations may invoke
arbitrary Python code which could do this; there is a code path which
allows control to flow back to the user from a \cfunction{Py_DECREF()},
so almost any operation is potentially dangerous.

A safe approach is to always use the generic operations (functions 
whose name begins with \samp{PyObject_}, \samp{PyNumber_},
\samp{PySequence_} or \samp{PyMapping_}).  These operations always
increment the reference count of the object they return.  This leaves
the caller with the responsibility to call
\cfunction{Py_DECREF()} when they are done with the result; this soon
becomes second nature.


\subsubsection{Reference Count Details \label{refcountDetails}}

The reference count behavior of functions in the Python/C API is best 
explained in terms of \emph{ownership of references}.  Ownership
pertains to references, never to objects (objects are not owned: they
are always shared).  "Owning a reference" means being responsible for
calling Py_DECREF on it when the reference is no longer needed. 
Ownership can also be transferred, meaning that the code that receives
ownership of the reference then becomes responsible for eventually
decref'ing it by calling \cfunction{Py_DECREF()} or
\cfunction{Py_XDECREF()} when it's no longer needed --or passing on
this responsibility (usually to its caller).
When a function passes ownership of a reference on to its caller, the
caller is said to receive a \emph{new} reference.  When no ownership
is transferred, the caller is said to \emph{borrow} the reference.
Nothing needs to be done for a borrowed reference.

Conversely, when a calling function passes it a reference to an 
object, there are two possibilities: the function \emph{steals} a 
reference to the object, or it does not.  Few functions steal 
references; the two notable exceptions are
\cfunction{PyList_SetItem()}\ttindex{PyList_SetItem()} and
\cfunction{PyTuple_SetItem()}\ttindex{PyTuple_SetItem()}, which 
steal a reference to the item (but not to the tuple or list into which
the item is put!).  These functions were designed to steal a reference
because of a common idiom for populating a tuple or list with newly
created objects; for example, the code to create the tuple \code{(1,
2, "three")} could look like this (forgetting about error handling for
the moment; a better way to code this is shown below):

\begin{verbatim}
PyObject *t;

t = PyTuple_New(3);
PyTuple_SetItem(t, 0, PyInt_FromLong(1L));
PyTuple_SetItem(t, 1, PyInt_FromLong(2L));
PyTuple_SetItem(t, 2, PyString_FromString("three"));
\end{verbatim}

Incidentally, \cfunction{PyTuple_SetItem()} is the \emph{only} way to
set tuple items; \cfunction{PySequence_SetItem()} and
\cfunction{PyObject_SetItem()} refuse to do this since tuples are an
immutable data type.  You should only use
\cfunction{PyTuple_SetItem()} for tuples that you are creating
yourself.

Equivalent code for populating a list can be written using 
\cfunction{PyList_New()} and \cfunction{PyList_SetItem()}.  Such code
can also use \cfunction{PySequence_SetItem()}; this illustrates the
difference between the two (the extra \cfunction{Py_DECREF()} calls):

\begin{verbatim}
PyObject *l, *x;

l = PyList_New(3);
x = PyInt_FromLong(1L);
PySequence_SetItem(l, 0, x); Py_DECREF(x);
x = PyInt_FromLong(2L);
PySequence_SetItem(l, 1, x); Py_DECREF(x);
x = PyString_FromString("three");
PySequence_SetItem(l, 2, x); Py_DECREF(x);
\end{verbatim}

You might find it strange that the ``recommended'' approach takes more
code.  However, in practice, you will rarely use these ways of
creating and populating a tuple or list.  There's a generic function,
\cfunction{Py_BuildValue()}, that can create most common objects from
C values, directed by a \dfn{format string}.  For example, the
above two blocks of code could be replaced by the following (which
also takes care of the error checking):

\begin{verbatim}
PyObject *t, *l;

t = Py_BuildValue("(iis)", 1, 2, "three");
l = Py_BuildValue("[iis]", 1, 2, "three");
\end{verbatim}

It is much more common to use \cfunction{PyObject_SetItem()} and
friends with items whose references you are only borrowing, like
arguments that were passed in to the function you are writing.  In
that case, their behaviour regarding reference counts is much saner,
since you don't have to increment a reference count so you can give a
reference away (``have it be stolen'').  For example, this function
sets all items of a list (actually, any mutable sequence) to a given
item:

\begin{verbatim}
int
set_all(PyObject *target, PyObject *item)
{
    int i, n;

    n = PyObject_Length(target);
    if (n < 0)
        return -1;
    for (i = 0; i < n; i++) {
        if (PyObject_SetItem(target, i, item) < 0)
            return -1;
    }
    return 0;
}
\end{verbatim}
\ttindex{set_all()}

The situation is slightly different for function return values.  
While passing a reference to most functions does not change your 
ownership responsibilities for that reference, many functions that 
return a reference to an object give you ownership of the reference.
The reason is simple: in many cases, the returned object is created 
on the fly, and the reference you get is the only reference to the 
object.  Therefore, the generic functions that return object 
references, like \cfunction{PyObject_GetItem()} and 
\cfunction{PySequence_GetItem()}, always return a new reference (the
caller becomes the owner of the reference).

It is important to realize that whether you own a reference returned 
by a function depends on which function you call only --- \emph{the
plumage} (the type of the object passed as an
argument to the function) \emph{doesn't enter into it!}  Thus, if you 
extract an item from a list using \cfunction{PyList_GetItem()}, you
don't own the reference --- but if you obtain the same item from the
same list using \cfunction{PySequence_GetItem()} (which happens to
take exactly the same arguments), you do own a reference to the
returned object.

Here is an example of how you could write a function that computes the
sum of the items in a list of integers; once using 
\cfunction{PyList_GetItem()}\ttindex{PyList_GetItem()}, and once using
\cfunction{PySequence_GetItem()}\ttindex{PySequence_GetItem()}.

\begin{verbatim}
long
sum_list(PyObject *list)
{
    int i, n;
    long total = 0;
    PyObject *item;

    n = PyList_Size(list);
    if (n < 0)
        return -1; /* Not a list */
    for (i = 0; i < n; i++) {
        item = PyList_GetItem(list, i); /* Can't fail */
        if (!PyInt_Check(item)) continue; /* Skip non-integers */
        total += PyInt_AsLong(item);
    }
    return total;
}
\end{verbatim}
\ttindex{sum_list()}

\begin{verbatim}
long
sum_sequence(PyObject *sequence)
{
    int i, n;
    long total = 0;
    PyObject *item;
    n = PySequence_Length(sequence);
    if (n < 0)
        return -1; /* Has no length */
    for (i = 0; i < n; i++) {
        item = PySequence_GetItem(sequence, i);
        if (item == NULL)
            return -1; /* Not a sequence, or other failure */
        if (PyInt_Check(item))
            total += PyInt_AsLong(item);
        Py_DECREF(item); /* Discard reference ownership */
    }
    return total;
}
\end{verbatim}
\ttindex{sum_sequence()}


\subsection{Types \label{types}}

There are few other data types that play a significant role in 
the Python/C API; most are simple C types such as \ctype{int}, 
\ctype{long}, \ctype{double} and \ctype{char*}.  A few structure types 
are used to describe static tables used to list the functions exported 
by a module or the data attributes of a new object type, and another
is used to describe the value of a complex number.  These will 
be discussed together with the functions that use them.


\section{Exceptions \label{exceptions}}

The Python programmer only needs to deal with exceptions if specific 
error handling is required; unhandled exceptions are automatically 
propagated to the caller, then to the caller's caller, and so on, until
they reach the top-level interpreter, where they are reported to the 
user accompanied by a stack traceback.

For C programmers, however, error checking always has to be explicit.  
All functions in the Python/C API can raise exceptions, unless an 
explicit claim is made otherwise in a function's documentation.  In 
general, when a function encounters an error, it sets an exception, 
discards any object references that it owns, and returns an 
error indicator --- usually \NULL{} or \code{-1}.  A few functions 
return a Boolean true/false result, with false indicating an error.
Very few functions return no explicit error indicator or have an 
ambiguous return value, and require explicit testing for errors with 
\cfunction{PyErr_Occurred()}\ttindex{PyErr_Occurred()}.

Exception state is maintained in per-thread storage (this is 
equivalent to using global storage in an unthreaded application).  A 
thread can be in one of two states: an exception has occurred, or not.
The function \cfunction{PyErr_Occurred()} can be used to check for
this: it returns a borrowed reference to the exception type object
when an exception has occurred, and \NULL{} otherwise.  There are a
number of functions to set the exception state:
\cfunction{PyErr_SetString()}\ttindex{PyErr_SetString()} is the most
common (though not the most general) function to set the exception
state, and \cfunction{PyErr_Clear()}\ttindex{PyErr_Clear()} clears the
exception state.

The full exception state consists of three objects (all of which can 
be \NULL): the exception type, the corresponding exception 
value, and the traceback.  These have the same meanings as the Python
\withsubitem{(in module sys)}{
  \ttindex{exc_type}\ttindex{exc_value}\ttindex{exc_traceback}}
objects \code{sys.exc_type}, \code{sys.exc_value}, and
\code{sys.exc_traceback}; however, they are not the same: the Python
objects represent the last exception being handled by a Python 
\keyword{try} \ldots\ \keyword{except} statement, while the C level
exception state only exists while an exception is being passed on
between C functions until it reaches the Python bytecode interpreter's 
main loop, which takes care of transferring it to \code{sys.exc_type}
and friends.

Note that starting with Python 1.5, the preferred, thread-safe way to 
access the exception state from Python code is to call the function
\withsubitem{(in module sys)}{\ttindex{exc_info()}}
\function{sys.exc_info()}, which returns the per-thread exception state 
for Python code.  Also, the semantics of both ways to access the 
exception state have changed so that a function which catches an 
exception will save and restore its thread's exception state so as to 
preserve the exception state of its caller.  This prevents common bugs 
in exception handling code caused by an innocent-looking function 
overwriting the exception being handled; it also reduces the often 
unwanted lifetime extension for objects that are referenced by the 
stack frames in the traceback.

As a general principle, a function that calls another function to 
perform some task should check whether the called function raised an 
exception, and if so, pass the exception state on to its caller.  It 
should discard any object references that it owns, and return an 
error indicator, but it should \emph{not} set another exception ---
that would overwrite the exception that was just raised, and lose
important information about the exact cause of the error.

A simple example of detecting exceptions and passing them on is shown
in the \cfunction{sum_sequence()}\ttindex{sum_sequence()} example
above.  It so happens that that example doesn't need to clean up any
owned references when it detects an error.  The following example
function shows some error cleanup.  First, to remind you why you like
Python, we show the equivalent Python code:

\begin{verbatim}
def incr_item(dict, key):
    try:
        item = dict[key]
    except KeyError:
        item = 0
    dict[key] = item + 1
\end{verbatim}
\ttindex{incr_item()}

Here is the corresponding C code, in all its glory:

\begin{verbatim}
int
incr_item(PyObject *dict, PyObject *key)
{
    /* Objects all initialized to NULL for Py_XDECREF */
    PyObject *item = NULL, *const_one = NULL, *incremented_item = NULL;
    int rv = -1; /* Return value initialized to -1 (failure) */

    item = PyObject_GetItem(dict, key);
    if (item == NULL) {
        /* Handle KeyError only: */
        if (!PyErr_ExceptionMatches(PyExc_KeyError))
            goto error;

        /* Clear the error and use zero: */
        PyErr_Clear();
        item = PyInt_FromLong(0L);
        if (item == NULL)
            goto error;
    }
    const_one = PyInt_FromLong(1L);
    if (const_one == NULL)
        goto error;

    incremented_item = PyNumber_Add(item, const_one);
    if (incremented_item == NULL)
        goto error;

    if (PyObject_SetItem(dict, key, incremented_item) < 0)
        goto error;
    rv = 0; /* Success */
    /* Continue with cleanup code */

 error:
    /* Cleanup code, shared by success and failure path */

    /* Use Py_XDECREF() to ignore NULL references */
    Py_XDECREF(item);
    Py_XDECREF(const_one);
    Py_XDECREF(incremented_item);

    return rv; /* -1 for error, 0 for success */
}
\end{verbatim}
\ttindex{incr_item()}

This example represents an endorsed use of the \keyword{goto} statement 
in C!  It illustrates the use of
\cfunction{PyErr_ExceptionMatches()}\ttindex{PyErr_ExceptionMatches()} and
\cfunction{PyErr_Clear()}\ttindex{PyErr_Clear()} to
handle specific exceptions, and the use of
\cfunction{Py_XDECREF()}\ttindex{Py_XDECREF()} to
dispose of owned references that may be \NULL{} (note the
\character{X} in the name; \cfunction{Py_DECREF()} would crash when
confronted with a \NULL{} reference).  It is important that the
variables used to hold owned references are initialized to \NULL{} for
this to work; likewise, the proposed return value is initialized to
\code{-1} (failure) and only set to success after the final call made
is successful.


\section{Embedding Python \label{embedding}}

The one important task that only embedders (as opposed to extension
writers) of the Python interpreter have to worry about is the
initialization, and possibly the finalization, of the Python
interpreter.  Most functionality of the interpreter can only be used
after the interpreter has been initialized.

The basic initialization function is
\cfunction{Py_Initialize()}\ttindex{Py_Initialize()}.
This initializes the table of loaded modules, and creates the
fundamental modules \module{__builtin__}\refbimodindex{__builtin__},
\module{__main__}\refbimodindex{__main__}, \module{sys}\refbimodindex{sys},
and \module{exceptions}.\refbimodindex{exceptions}  It also initializes
the module search path (\code{sys.path}).%
\indexiii{module}{search}{path}
\withsubitem{(in module sys)}{\ttindex{path}}

\cfunction{Py_Initialize()} does not set the ``script argument list'' 
(\code{sys.argv}).  If this variable is needed by Python code that 
will be executed later, it must be set explicitly with a call to 
\code{PySys_SetArgv(\var{argc},
\var{argv})}\ttindex{PySys_SetArgv()} subsequent to the call to
\cfunction{Py_Initialize()}.

On most systems (in particular, on \UNIX{} and Windows, although the
details are slightly different),
\cfunction{Py_Initialize()} calculates the module search path based
upon its best guess for the location of the standard Python
interpreter executable, assuming that the Python library is found in a
fixed location relative to the Python interpreter executable.  In
particular, it looks for a directory named
\file{lib/python\shortversion} relative to the parent directory where
the executable named \file{python} is found on the shell command
search path (the environment variable \envvar{PATH}).

For instance, if the Python executable is found in
\file{/usr/local/bin/python}, it will assume that the libraries are in
\file{/usr/local/lib/python\shortversion}.  (In fact, this particular path
is also the ``fallback'' location, used when no executable file named
\file{python} is found along \envvar{PATH}.)  The user can override
this behavior by setting the environment variable \envvar{PYTHONHOME},
or insert additional directories in front of the standard path by
setting \envvar{PYTHONPATH}.

The embedding application can steer the search by calling 
\code{Py_SetProgramName(\var{file})}\ttindex{Py_SetProgramName()} \emph{before} calling 
\cfunction{Py_Initialize()}.  Note that \envvar{PYTHONHOME} still
overrides this and \envvar{PYTHONPATH} is still inserted in front of
the standard path.  An application that requires total control has to
provide its own implementation of
\cfunction{Py_GetPath()}\ttindex{Py_GetPath()},
\cfunction{Py_GetPrefix()}\ttindex{Py_GetPrefix()},
\cfunction{Py_GetExecPrefix()}\ttindex{Py_GetExecPrefix()}, and
\cfunction{Py_GetProgramFullPath()}\ttindex{Py_GetProgramFullPath()} (all
defined in \file{Modules/getpath.c}).

Sometimes, it is desirable to ``uninitialize'' Python.  For instance, 
the application may want to start over (make another call to 
\cfunction{Py_Initialize()}) or the application is simply done with its 
use of Python and wants to free all memory allocated by Python.  This
can be accomplished by calling \cfunction{Py_Finalize()}.  The function
\cfunction{Py_IsInitialized()}\ttindex{Py_IsInitialized()} returns
true if Python is currently in the initialized state.  More
information about these functions is given in a later chapter.

\chapter{The Very High Level Layer \label{veryhigh}}


The functions in this chapter will let you execute Python source code
given in a file or a buffer, but they will not let you interact in a
more detailed way with the interpreter.

Several of these functions accept a start symbol from the grammar as a 
parameter.  The available start symbols are \constant{Py_eval_input},
\constant{Py_file_input}, and \constant{Py_single_input}.  These are
described following the functions which accept them as parameters.

Note also that several of these functions take \ctype{FILE*}
parameters.  On particular issue which needs to be handled carefully
is that the \ctype{FILE} structure for different C libraries can be
different and incompatible.  Under Windows (at least), it is possible
for dynamically linked extensions to actually use different libraries,
so care should be taken that \ctype{FILE*} parameters are only passed
to these functions if it is certain that they were created by the same
library that the Python runtime is using.


\begin{cfuncdesc}{int}{Py_Main}{int argc, char **argv}
  The main program for the standard interpreter.  This is made
  available for programs which embed Python.  The \var{argc} and
  \var{argv} parameters should be prepared exactly as those which are
  passed to a C program's \cfunction{main()} function.  It is
  important to note that the argument list may be modified (but the
  contents of the strings pointed to by the argument list are not).
  The return value will be the integer passed to the
  \function{sys.exit()} function, \code{1} if the interpreter exits
  due to an exception, or \code{2} if the parameter list does not
  represent a valid Python command line.
\end{cfuncdesc}

\begin{cfuncdesc}{int}{PyRun_AnyFile}{FILE *fp, char *filename}
  If \var{fp} refers to a file associated with an interactive device
  (console or terminal input or \UNIX{} pseudo-terminal), return the
  value of \cfunction{PyRun_InteractiveLoop()}, otherwise return the
  result of \cfunction{PyRun_SimpleFile()}.  If \var{filename} is
  \NULL, this function uses \code{"???"} as the filename.
\end{cfuncdesc}

\begin{cfuncdesc}{int}{PyRun_SimpleString}{char *command}
  Executes the Python source code from \var{command} in the
  \module{__main__} module.  If \module{__main__} does not already
  exist, it is created.  Returns \code{0} on success or \code{-1} if
  an exception was raised.  If there was an error, there is no way to
  get the exception information.
\end{cfuncdesc}

\begin{cfuncdesc}{int}{PyRun_SimpleFile}{FILE *fp, char *filename}
  Similar to \cfunction{PyRun_SimpleString()}, but the Python source
  code is read from \var{fp} instead of an in-memory string.
  \var{filename} should be the name of the file.
\end{cfuncdesc}

\begin{cfuncdesc}{int}{PyRun_InteractiveOne}{FILE *fp, char *filename}
  Read and execute a single statement from a file associated with an
  interactive device.  If \var{filename} is \NULL, \code{"???"} is
  used instead.  The user will be prompted using \code{sys.ps1} and
  \code{sys.ps2}.  Returns \code{0} when the input was executed
  successfully, \code{-1} if there was an exception, or an error code
  from the \file{errcode.h} include file distributed as part of Python
  if there was a parse error.  (Note that \file{errcode.h} is not
  included by \file{Python.h}, so must be included specifically if
  needed.)
\end{cfuncdesc}

\begin{cfuncdesc}{int}{PyRun_InteractiveLoop}{FILE *fp, char *filename}
  Read and execute statements from a file associated with an
  interactive device until \EOF{} is reached.  If \var{filename} is
  \NULL, \code{"???"} is used instead.  The user will be prompted
  using \code{sys.ps1} and \code{sys.ps2}.  Returns \code{0} at \EOF.
\end{cfuncdesc}

\begin{cfuncdesc}{struct _node*}{PyParser_SimpleParseString}{char *str,
                                                             int start}
  Parse Python source code from \var{str} using the start token
  \var{start}.  The result can be used to create a code object which
  can be evaluated efficiently.  This is useful if a code fragment
  must be evaluated many times.
\end{cfuncdesc}

\begin{cfuncdesc}{struct _node*}{PyParser_SimpleParseFile}{FILE *fp,
                                 char *filename, int start}
  Similar to \cfunction{PyParser_SimpleParseString()}, but the Python
  source code is read from \var{fp} instead of an in-memory string.
  \var{filename} should be the name of the file.
\end{cfuncdesc}

\begin{cfuncdesc}{PyObject*}{PyRun_String}{char *str, int start,
                                           PyObject *globals,
                                           PyObject *locals}
  Execute Python source code from \var{str} in the context specified
  by the dictionaries \var{globals} and \var{locals}.  The parameter
  \var{start} specifies the start token that should be used to parse
  the source code.

  Returns the result of executing the code as a Python object, or
  \NULL{} if an exception was raised.
\end{cfuncdesc}

\begin{cfuncdesc}{PyObject*}{PyRun_File}{FILE *fp, char *filename,
                                         int start, PyObject *globals,
                                         PyObject *locals}
  Similar to \cfunction{PyRun_String()}, but the Python source code is
  read from \var{fp} instead of an in-memory string.
  \var{filename} should be the name of the file.
\end{cfuncdesc}

\begin{cfuncdesc}{PyObject*}{Py_CompileString}{char *str, char *filename,
                                               int start}
  Parse and compile the Python source code in \var{str}, returning the
  resulting code object.  The start token is given by \var{start};
  this can be used to constrain the code which can be compiled and should
  be \constant{Py_eval_input}, \constant{Py_file_input}, or
  \constant{Py_single_input}.  The filename specified by
  \var{filename} is used to construct the code object and may appear
  in tracebacks or \exception{SyntaxError} exception messages.  This
  returns \NULL{} if the code cannot be parsed or compiled.
\end{cfuncdesc}

\begin{cvardesc}{int}{Py_eval_input}
  The start symbol from the Python grammar for isolated expressions;
  for use with
  \cfunction{Py_CompileString()}\ttindex{Py_CompileString()}.
\end{cvardesc}

\begin{cvardesc}{int}{Py_file_input}
  The start symbol from the Python grammar for sequences of statements
  as read from a file or other source; for use with
  \cfunction{Py_CompileString()}\ttindex{Py_CompileString()}.  This is
  the symbol to use when compiling arbitrarily long Python source code.
\end{cvardesc}

\begin{cvardesc}{int}{Py_single_input}
  The start symbol from the Python grammar for a single statement; for
  use with \cfunction{Py_CompileString()}\ttindex{Py_CompileString()}.
  This is the symbol used for the interactive interpreter loop.
\end{cvardesc}

\chapter{Reference Counting \label{countingRefs}}


The macros in this section are used for managing reference counts
of Python objects.


\begin{cfuncdesc}{void}{Py_INCREF}{PyObject *o}
  Increment the reference count for object \var{o}.  The object must
  not be \NULL; if you aren't sure that it isn't \NULL, use
  \cfunction{Py_XINCREF()}.
\end{cfuncdesc}

\begin{cfuncdesc}{void}{Py_XINCREF}{PyObject *o}
  Increment the reference count for object \var{o}.  The object may be
  \NULL, in which case the macro has no effect.
\end{cfuncdesc}

\begin{cfuncdesc}{void}{Py_DECREF}{PyObject *o}
  Decrement the reference count for object \var{o}.  The object must
  not be \NULL; if you aren't sure that it isn't \NULL, use
  \cfunction{Py_XDECREF()}.  If the reference count reaches zero, the
  object's type's deallocation function (which must not be \NULL) is
  invoked.

  \warning{The deallocation function can cause arbitrary Python code
  to be invoked (e.g. when a class instance with a \method{__del__()}
  method is deallocated).  While exceptions in such code are not
  propagated, the executed code has free access to all Python global
  variables.  This means that any object that is reachable from a
  global variable should be in a consistent state before
  \cfunction{Py_DECREF()} is invoked.  For example, code to delete an
  object from a list should copy a reference to the deleted object in
  a temporary variable, update the list data structure, and then call
  \cfunction{Py_DECREF()} for the temporary variable.}
\end{cfuncdesc}

\begin{cfuncdesc}{void}{Py_XDECREF}{PyObject *o}
  Decrement the reference count for object \var{o}.  The object may be
  \NULL, in which case the macro has no effect; otherwise the effect
  is the same as for \cfunction{Py_DECREF()}, and the same warning
  applies.
\end{cfuncdesc}

\begin{cfuncdesc}{void}{Py_CLEAR}{PyObject *o}
  Decrement the reference count for object \var{o}.  The object may be
  \NULL, in which case the macro has no effect; otherwise the effect
  is the same as for \cfunction{Py_DECREF()}, except that the argument
  is also set to \NULL.  The warning for \cfunction{Py_DECREF()} does
  not apply with respect to the object passed because the macro
  carefully uses a temporary variable and sets the argument to \NULL
  before decrementing its reference count.

  It is a good idea to use this macro whenever decrementing the value
  of a variable that might be traversed during garbage collection.

\versionadded{2.4}
\end{cfuncdesc}


The following functions are for runtime dynamic embedding of Python:
\cfunction{Py_IncRef(PyObject *o)}, \cfunction{Py_DecRef(PyObject *o)}.
They are simply exported function versions of \cfunction{Py_XINCREF()} and 
\cfunction{Py_XDECREF()}, respectively.

The following functions or macros are only for use within the
interpreter core: \cfunction{_Py_Dealloc()},
\cfunction{_Py_ForgetReference()}, \cfunction{_Py_NewReference()}, as
well as the global variable \cdata{_Py_RefTotal}.

\chapter{Exception Handling \label{exceptionHandling}}

The functions described in this chapter will let you handle and raise Python
exceptions.  It is important to understand some of the basics of
Python exception handling.  It works somewhat like the
\UNIX{} \cdata{errno} variable: there is a global indicator (per
thread) of the last error that occurred.  Most functions don't clear
this on success, but will set it to indicate the cause of the error on
failure.  Most functions also return an error indicator, usually
\NULL{} if they are supposed to return a pointer, or \code{-1} if they
return an integer (exception: the \cfunction{PyArg_*()} functions
return \code{1} for success and \code{0} for failure).

When a function must fail because some function it called failed, it
generally doesn't set the error indicator; the function it called
already set it.  It is responsible for either handling the error and
clearing the exception or returning after cleaning up any resources it
holds (such as object references or memory allocations); it should
\emph{not} continue normally if it is not prepared to handle the
error.  If returning due to an error, it is important to indicate to
the caller that an error has been set.  If the error is not handled or
carefully propagated, additional calls into the Python/C API may not
behave as intended and may fail in mysterious ways.

The error indicator consists of three Python objects corresponding to
\withsubitem{(in module sys)}{
  \ttindex{exc_type}\ttindex{exc_value}\ttindex{exc_traceback}}
the Python variables \code{sys.exc_type}, \code{sys.exc_value} and
\code{sys.exc_traceback}.  API functions exist to interact with the
error indicator in various ways.  There is a separate error indicator
for each thread.

% XXX Order of these should be more thoughtful.
% Either alphabetical or some kind of structure.

\begin{cfuncdesc}{void}{PyErr_Print}{}
  Print a standard traceback to \code{sys.stderr} and clear the error
  indicator.  Call this function only when the error indicator is
  set.  (Otherwise it will cause a fatal error!)
\end{cfuncdesc}

\begin{cfuncdesc}{PyObject*}{PyErr_Occurred}{}
  Test whether the error indicator is set.  If set, return the
  exception \emph{type} (the first argument to the last call to one of
  the \cfunction{PyErr_Set*()} functions or to
  \cfunction{PyErr_Restore()}).  If not set, return \NULL.  You do
  not own a reference to the return value, so you do not need to
  \cfunction{Py_DECREF()} it.  \note{Do not compare the return value
    to a specific exception; use \cfunction{PyErr_ExceptionMatches()}
    instead, shown below.  (The comparison could easily fail since the
    exception may be an instance instead of a class, in the case of a
    class exception, or it may the a subclass of the expected
    exception.)}
\end{cfuncdesc}

\begin{cfuncdesc}{int}{PyErr_ExceptionMatches}{PyObject *exc}
  Equivalent to \samp{PyErr_GivenExceptionMatches(PyErr_Occurred(),
  \var{exc})}.  This should only be called when an exception is
  actually set; a memory access violation will occur if no exception
  has been raised.
\end{cfuncdesc}

\begin{cfuncdesc}{int}{PyErr_GivenExceptionMatches}{PyObject *given, PyObject *exc}
  Return true if the \var{given} exception matches the exception in
  \var{exc}.  If \var{exc} is a class object, this also returns true
  when \var{given} is an instance of a subclass.  If \var{exc} is a
  tuple, all exceptions in the tuple (and recursively in subtuples)
  are searched for a match.  If \var{given} is \NULL, a memory access
  violation will occur.
\end{cfuncdesc}

\begin{cfuncdesc}{void}{PyErr_NormalizeException}{PyObject**exc, PyObject**val, PyObject**tb}
  Under certain circumstances, the values returned by
  \cfunction{PyErr_Fetch()} below can be ``unnormalized'', meaning
  that \code{*\var{exc}} is a class object but \code{*\var{val}} is
  not an instance of the  same class.  This function can be used to
  instantiate the class in that case.  If the values are already
  normalized, nothing happens.  The delayed normalization is
  implemented to improve performance.
\end{cfuncdesc}

\begin{cfuncdesc}{void}{PyErr_Clear}{}
  Clear the error indicator.  If the error indicator is not set, there
  is no effect.
\end{cfuncdesc}

\begin{cfuncdesc}{void}{PyErr_Fetch}{PyObject **ptype, PyObject **pvalue,
                                     PyObject **ptraceback}
  Retrieve the error indicator into three variables whose addresses
  are passed.  If the error indicator is not set, set all three
  variables to \NULL.  If it is set, it will be cleared and you own a
  reference to each object retrieved.  The value and traceback object
  may be \NULL{} even when the type object is not.  \note{This
  function is normally only used by code that needs to handle
  exceptions or by code that needs to save and restore the error
  indicator temporarily.}
\end{cfuncdesc}

\begin{cfuncdesc}{void}{PyErr_Restore}{PyObject *type, PyObject *value,
                                       PyObject *traceback}
  Set  the error indicator from the three objects.  If the error
  indicator is already set, it is cleared first.  If the objects are
  \NULL, the error indicator is cleared.  Do not pass a \NULL{} type
  and non-\NULL{} value or traceback.  The exception type should be a
  class.  Do not pass an invalid exception type or value.
  (Violating these rules will cause subtle problems later.)  This call
  takes away a reference to each object: you must own a reference to
  each object before the call and after the call you no longer own
  these references.  (If you don't understand this, don't use this
  function.  I warned you.)  \note{This function is normally only used
  by code that needs to save and restore the error indicator
  temporarily; use \cfunction{PyErr_Fetch()} to save the current
  exception state.}
\end{cfuncdesc}

\begin{cfuncdesc}{void}{PyErr_SetString}{PyObject *type, char *message}
  This is the most common way to set the error indicator.  The first
  argument specifies the exception type; it is normally one of the
  standard exceptions, e.g. \cdata{PyExc_RuntimeError}.  You need not
  increment its reference count.  The second argument is an error
  message; it is converted to a string object.
\end{cfuncdesc}

\begin{cfuncdesc}{void}{PyErr_SetObject}{PyObject *type, PyObject *value}
  This function is similar to \cfunction{PyErr_SetString()} but lets
  you specify an arbitrary Python object for the ``value'' of the
  exception.
\end{cfuncdesc}

\begin{cfuncdesc}{PyObject*}{PyErr_Format}{PyObject *exception,
                                           const char *format, \moreargs}
  This function sets the error indicator and returns \NULL.
  \var{exception} should be a Python exception (class, not
  an instance).  \var{format} should be a string, containing format
  codes, similar to \cfunction{printf()}. The \code{width.precision}
  before a format code is parsed, but the width part is ignored.

  \begin{tableii}{c|l}{character}{Character}{Meaning}
    \lineii{c}{Character, as an \ctype{int} parameter}
    \lineii{d}{Number in decimal, as an \ctype{int} parameter}
    \lineii{x}{Number in hexadecimal, as an \ctype{int} parameter}
    \lineii{s}{A string, as a \ctype{char *} parameter}
    \lineii{p}{A hex pointer, as a \ctype{void *} parameter}
  \end{tableii}

  An unrecognized format character causes all the rest of the format
  string to be copied as-is to the result string, and any extra
  arguments discarded.
\end{cfuncdesc}

\begin{cfuncdesc}{void}{PyErr_SetNone}{PyObject *type}
  This is a shorthand for \samp{PyErr_SetObject(\var{type},
  Py_None)}.
\end{cfuncdesc}

\begin{cfuncdesc}{int}{PyErr_BadArgument}{}
  This is a shorthand for \samp{PyErr_SetString(PyExc_TypeError,
  \var{message})}, where \var{message} indicates that a built-in
  operation was invoked with an illegal argument.  It is mostly for
  internal use.
\end{cfuncdesc}

\begin{cfuncdesc}{PyObject*}{PyErr_NoMemory}{}
  This is a shorthand for \samp{PyErr_SetNone(PyExc_MemoryError)}; it
  returns \NULL{} so an object allocation function can write
  \samp{return PyErr_NoMemory();} when it runs out of memory.
\end{cfuncdesc}

\begin{cfuncdesc}{PyObject*}{PyErr_SetFromErrno}{PyObject *type}
  This is a convenience function to raise an exception when a C
  library function has returned an error and set the C variable
  \cdata{errno}.  It constructs a tuple object whose first item is the
  integer \cdata{errno} value and whose second item is the
  corresponding error message (gotten from
  \cfunction{strerror()}\ttindex{strerror()}), and then calls
  \samp{PyErr_SetObject(\var{type}, \var{object})}.  On \UNIX, when
  the \cdata{errno} value is \constant{EINTR}, indicating an
  interrupted system call, this calls
  \cfunction{PyErr_CheckSignals()}, and if that set the error
  indicator, leaves it set to that.  The function always returns
  \NULL, so a wrapper function around a system call can write
  \samp{return PyErr_SetFromErrno(\var{type});} when the system call
  returns an error.
\end{cfuncdesc}

\begin{cfuncdesc}{PyObject*}{PyErr_SetFromErrnoWithFilename}{PyObject *type,
                                                             char *filename}
  Similar to \cfunction{PyErr_SetFromErrno()}, with the additional
  behavior that if \var{filename} is not \NULL, it is passed to the
  constructor of \var{type} as a third parameter.  In the case of
  exceptions such as \exception{IOError} and \exception{OSError}, this
  is used to define the \member{filename} attribute of the exception
  instance.
\end{cfuncdesc}

\begin{cfuncdesc}{PyObject*}{PyErr_SetFromWindowsErr}{int ierr}
  This is a convenience function to raise \exception{WindowsError}.
  If called with \var{ierr} of \cdata{0}, the error code returned by a
  call to \cfunction{GetLastError()} is used instead.  It calls the
  Win32 function \cfunction{FormatMessage()} to retrieve the Windows
  description of error code given by \var{ierr} or
  \cfunction{GetLastError()}, then it constructs a tuple object whose
  first item is the \var{ierr} value and whose second item is the
  corresponding error message (gotten from
  \cfunction{FormatMessage()}), and then calls
  \samp{PyErr_SetObject(\var{PyExc_WindowsError}, \var{object})}.
  This function always returns \NULL.
  Availability: Windows.
\end{cfuncdesc}

\begin{cfuncdesc}{PyObject*}{PyErr_SetExcFromWindowsErr}{PyObject *type,
	                                                 int ierr}
  Similar to \cfunction{PyErr_SetFromWindowsErr()}, with an additional
  parameter specifying the exception type to be raised.
  Availability: Windows.
  \versionadded{2.3}
\end{cfuncdesc}

\begin{cfuncdesc}{PyObject*}{PyErr_SetFromWindowsErrWithFilename}{int ierr,
                                                                char *filename}
  Similar to \cfunction{PyErr_SetFromWindowsErr()}, with the
  additional behavior that if \var{filename} is not \NULL, it is
  passed to the constructor of \exception{WindowsError} as a third
  parameter.
  Availability: Windows.
\end{cfuncdesc}

\begin{cfuncdesc}{PyObject*}{PyErr_SetExcFromWindowsErrWithFilename}
	{PyObject *type, int ierr, char *filename}
  Similar to \cfunction{PyErr_SetFromWindowsErrWithFilename()}, with
  an additional parameter specifying the exception type to be raised.
  Availability: Windows.
  \versionadded{2.3}
\end{cfuncdesc}

\begin{cfuncdesc}{void}{PyErr_BadInternalCall}{}
  This is a shorthand for \samp{PyErr_SetString(PyExc_TypeError,
  \var{message})}, where \var{message} indicates that an internal
  operation (e.g. a Python/C API function) was invoked with an illegal
  argument.  It is mostly for internal use.
\end{cfuncdesc}

\begin{cfuncdesc}{int}{PyErr_Warn}{PyObject *category, char *message}
  Issue a warning message.  The \var{category} argument is a warning
  category (see below) or \NULL; the \var{message} argument is a
  message string.

  This function normally prints a warning message to \var{sys.stderr};
  however, it is also possible that the user has specified that
  warnings are to be turned into errors, and in that case this will
  raise an exception.  It is also possible that the function raises an
  exception because of a problem with the warning machinery (the
  implementation imports the \module{warnings} module to do the heavy
  lifting).  The return value is \code{0} if no exception is raised,
  or \code{-1} if an exception is raised.  (It is not possible to
  determine whether a warning message is actually printed, nor what
  the reason is for the exception; this is intentional.)  If an
  exception is raised, the caller should do its normal exception
  handling (for example, \cfunction{Py_DECREF()} owned references and
  return an error value).

  Warning categories must be subclasses of \cdata{Warning}; the
  default warning category is \cdata{RuntimeWarning}.  The standard
  Python warning categories are available as global variables whose
  names are \samp{PyExc_} followed by the Python exception name.
  These have the type \ctype{PyObject*}; they are all class objects.
  Their names are \cdata{PyExc_Warning}, \cdata{PyExc_UserWarning},
  \cdata{PyExc_DeprecationWarning}, \cdata{PyExc_SyntaxWarning},
  \cdata{PyExc_RuntimeWarning}, and \cdata{PyExc_FutureWarning}.
  \cdata{PyExc_Warning} is a subclass of \cdata{PyExc_Exception}; the
  other warning categories are subclasses of \cdata{PyExc_Warning}.

  For information about warning control, see the documentation for the
  \module{warnings} module and the \programopt{-W} option in the
  command line documentation.  There is no C API for warning control.
\end{cfuncdesc}

\begin{cfuncdesc}{int}{PyErr_WarnExplicit}{PyObject *category, char *message,
                char *filename, int lineno, char *module, PyObject *registry}
  Issue a warning message with explicit control over all warning
  attributes.  This is a straightforward wrapper around the Python
  function \function{warnings.warn_explicit()}, see there for more
  information.  The \var{module} and \var{registry} arguments may be
  set to \NULL{} to get the default effect described there.
\end{cfuncdesc}

\begin{cfuncdesc}{int}{PyErr_CheckSignals}{}
  This function interacts with Python's signal handling.  It checks
  whether a signal has been sent to the processes and if so, invokes
  the corresponding signal handler.  If the
  \module{signal}\refbimodindex{signal} module is supported, this can
  invoke a signal handler written in Python.  In all cases, the
  default effect for \constant{SIGINT}\ttindex{SIGINT} is to raise the
  \withsubitem{(built-in exception)}{\ttindex{KeyboardInterrupt}}
  \exception{KeyboardInterrupt} exception.  If an exception is raised
  the error indicator is set and the function returns \code{1};
  otherwise the function returns \code{0}.  The error indicator may or
  may not be cleared if it was previously set.
\end{cfuncdesc}

\begin{cfuncdesc}{void}{PyErr_SetInterrupt}{}
  This function simulates the effect of a
  \constant{SIGINT}\ttindex{SIGINT} signal arriving --- the next time
  \cfunction{PyErr_CheckSignals()} is called,
  \withsubitem{(built-in exception)}{\ttindex{KeyboardInterrupt}}
  \exception{KeyboardInterrupt} will be raised.  It may be called
  without holding the interpreter lock.
  % XXX This was described as obsolete, but is used in
  % thread.interrupt_main() (used from IDLE), so it's still needed.
\end{cfuncdesc}

\begin{cfuncdesc}{PyObject*}{PyErr_NewException}{char *name,
                                                 PyObject *base,
                                                 PyObject *dict}
  This utility function creates and returns a new exception object.
  The \var{name} argument must be the name of the new exception, a C
  string of the form \code{module.class}.  The \var{base} and
  \var{dict} arguments are normally \NULL.  This creates a class
  object derived from the root for all exceptions, the built-in name
  \exception{Exception} (accessible in C as \cdata{PyExc_Exception}).
  The \member{__module__} attribute of the new class is set to the
  first part (up to the last dot) of the \var{name} argument, and the
  class name is set to the last part (after the last dot).  The
  \var{base} argument can be used to specify an alternate base class.
  The \var{dict} argument can be used to specify a dictionary of class
  variables and methods.
\end{cfuncdesc}

\begin{cfuncdesc}{void}{PyErr_WriteUnraisable}{PyObject *obj}
  This utility function prints a warning message to \code{sys.stderr}
  when an exception has been set but it is impossible for the
  interpreter to actually raise the exception.  It is used, for
  example, when an exception occurs in an \method{__del__()} method.

  The function is called with a single argument \var{obj} that
  identifies the context in which the unraisable exception occurred.
  The repr of \var{obj} will be printed in the warning message.
\end{cfuncdesc}

\section{Standard Exceptions \label{standardExceptions}}

All standard Python exceptions are available as global variables whose
names are \samp{PyExc_} followed by the Python exception name.  These
have the type \ctype{PyObject*}; they are all class objects.  For
completeness, here are all the variables:

\begin{tableiii}{l|l|c}{cdata}{C Name}{Python Name}{Notes}
  \lineiii{PyExc_Exception\ttindex{PyExc_Exception}}{\exception{Exception}}{(1)}
  \lineiii{PyExc_StandardError\ttindex{PyExc_StandardError}}{\exception{StandardError}}{(1)}
  \lineiii{PyExc_ArithmeticError\ttindex{PyExc_ArithmeticError}}{\exception{ArithmeticError}}{(1)}
  \lineiii{PyExc_LookupError\ttindex{PyExc_LookupError}}{\exception{LookupError}}{(1)}
  \lineiii{PyExc_AssertionError\ttindex{PyExc_AssertionError}}{\exception{AssertionError}}{}
  \lineiii{PyExc_AttributeError\ttindex{PyExc_AttributeError}}{\exception{AttributeError}}{}
  \lineiii{PyExc_EOFError\ttindex{PyExc_EOFError}}{\exception{EOFError}}{}
  \lineiii{PyExc_EnvironmentError\ttindex{PyExc_EnvironmentError}}{\exception{EnvironmentError}}{(1)}
  \lineiii{PyExc_FloatingPointError\ttindex{PyExc_FloatingPointError}}{\exception{FloatingPointError}}{}
  \lineiii{PyExc_IOError\ttindex{PyExc_IOError}}{\exception{IOError}}{}
  \lineiii{PyExc_ImportError\ttindex{PyExc_ImportError}}{\exception{ImportError}}{}
  \lineiii{PyExc_IndexError\ttindex{PyExc_IndexError}}{\exception{IndexError}}{}
  \lineiii{PyExc_KeyError\ttindex{PyExc_KeyError}}{\exception{KeyError}}{}
  \lineiii{PyExc_KeyboardInterrupt\ttindex{PyExc_KeyboardInterrupt}}{\exception{KeyboardInterrupt}}{}
  \lineiii{PyExc_MemoryError\ttindex{PyExc_MemoryError}}{\exception{MemoryError}}{}
  \lineiii{PyExc_NameError\ttindex{PyExc_NameError}}{\exception{NameError}}{}
  \lineiii{PyExc_NotImplementedError\ttindex{PyExc_NotImplementedError}}{\exception{NotImplementedError}}{}
  \lineiii{PyExc_OSError\ttindex{PyExc_OSError}}{\exception{OSError}}{}
  \lineiii{PyExc_OverflowError\ttindex{PyExc_OverflowError}}{\exception{OverflowError}}{}
  \lineiii{PyExc_ReferenceError\ttindex{PyExc_ReferenceError}}{\exception{ReferenceError}}{(2)}
  \lineiii{PyExc_RuntimeError\ttindex{PyExc_RuntimeError}}{\exception{RuntimeError}}{}
  \lineiii{PyExc_SyntaxError\ttindex{PyExc_SyntaxError}}{\exception{SyntaxError}}{}
  \lineiii{PyExc_SystemError\ttindex{PyExc_SystemError}}{\exception{SystemError}}{}
  \lineiii{PyExc_SystemExit\ttindex{PyExc_SystemExit}}{\exception{SystemExit}}{}
  \lineiii{PyExc_TypeError\ttindex{PyExc_TypeError}}{\exception{TypeError}}{}
  \lineiii{PyExc_ValueError\ttindex{PyExc_ValueError}}{\exception{ValueError}}{}
  \lineiii{PyExc_WindowsError\ttindex{PyExc_WindowsError}}{\exception{WindowsError}}{(3)}
  \lineiii{PyExc_ZeroDivisionError\ttindex{PyExc_ZeroDivisionError}}{\exception{ZeroDivisionError}}{}
\end{tableiii}

\noindent
Notes:
\begin{description}
\item[(1)]
  This is a base class for other standard exceptions.

\item[(2)]
  This is the same as \exception{weakref.ReferenceError}.

\item[(3)]
  Only defined on Windows; protect code that uses this by testing that
  the preprocessor macro \code{MS_WINDOWS} is defined.
\end{description}


\section{Deprecation of String Exceptions}

All exceptions built into Python or provided in the standard library
are derived from \exception{Exception}.
\withsubitem{(built-in exception)}{\ttindex{Exception}}

String exceptions are still supported in the interpreter to allow
existing code to run unmodified, but this will also change in a future 
release.

\chapter{Utilities \label{utilities}}

The functions in this chapter perform various utility tasks, ranging
from helping C code be more portable across platforms, using Python
modules from C, and parsing function arguments and constructing Python
values from C values.


\section{Operating System Utilities \label{os}}

\begin{cfuncdesc}{int}{Py_FdIsInteractive}{FILE *fp, char *filename}
  Return true (nonzero) if the standard I/O file \var{fp} with name
  \var{filename} is deemed interactive.  This is the case for files
  for which \samp{isatty(fileno(\var{fp}))} is true.  If the global
  flag \cdata{Py_InteractiveFlag} is true, this function also returns
  true if the \var{filename} pointer is \NULL{} or if the name is
  equal to one of the strings \code{'<stdin>'} or \code{'???'}.
\end{cfuncdesc}

\begin{cfuncdesc}{long}{PyOS_GetLastModificationTime}{char *filename}
  Return the time of last modification of the file \var{filename}.
  The result is encoded in the same way as the timestamp returned by
  the standard C library function \cfunction{time()}.
\end{cfuncdesc}

\begin{cfuncdesc}{void}{PyOS_AfterFork}{}
  Function to update some internal state after a process fork; this
  should be called in the new process if the Python interpreter will
  continue to be used.  If a new executable is loaded into the new
  process, this function does not need to be called.
\end{cfuncdesc}

\begin{cfuncdesc}{int}{PyOS_CheckStack}{}
  Return true when the interpreter runs out of stack space.  This is a
  reliable check, but is only available when \constant{USE_STACKCHECK}
  is defined (currently on Windows using the Microsoft Visual \Cpp{}
  compiler).  \constant{USE_CHECKSTACK} will be
  defined automatically; you should never change the definition in
  your own code.
\end{cfuncdesc}

\begin{cfuncdesc}{PyOS_sighandler_t}{PyOS_getsig}{int i}
  Return the current signal handler for signal \var{i}.  This is a
  thin wrapper around either \cfunction{sigaction()} or
  \cfunction{signal()}.  Do not call those functions directly!
  \ctype{PyOS_sighandler_t} is a typedef alias for \ctype{void
  (*)(int)}.
\end{cfuncdesc}

\begin{cfuncdesc}{PyOS_sighandler_t}{PyOS_setsig}{int i, PyOS_sighandler_t h}
  Set the signal handler for signal \var{i} to be \var{h}; return the
  old signal handler.  This is a thin wrapper around either
  \cfunction{sigaction()} or \cfunction{signal()}.  Do not call those
  functions directly!  \ctype{PyOS_sighandler_t} is a typedef alias
  for \ctype{void (*)(int)}.
\end{cfuncdesc}


\section{Process Control \label{processControl}}

\begin{cfuncdesc}{void}{Py_FatalError}{const char *message}
  Print a fatal error message and kill the process.  No cleanup is
  performed.  This function should only be invoked when a condition is
  detected that would make it dangerous to continue using the Python
  interpreter; e.g., when the object administration appears to be
  corrupted.  On \UNIX, the standard C library function
  \cfunction{abort()}\ttindex{abort()} is called which will attempt to
  produce a \file{core} file.
\end{cfuncdesc}

\begin{cfuncdesc}{void}{Py_Exit}{int status}
  Exit the current process.  This calls
  \cfunction{Py_Finalize()}\ttindex{Py_Finalize()} and then calls the
  standard C library function
  \code{exit(\var{status})}\ttindex{exit()}.
\end{cfuncdesc}

\begin{cfuncdesc}{int}{Py_AtExit}{void (*func) ()}
  Register a cleanup function to be called by
  \cfunction{Py_Finalize()}\ttindex{Py_Finalize()}.  The cleanup
  function will be called with no arguments and should return no
  value.  At most 32 \index{cleanup functions}cleanup functions can be
  registered.  When the registration is successful,
  \cfunction{Py_AtExit()} returns \code{0}; on failure, it returns
  \code{-1}.  The cleanup function registered last is called first.
  Each cleanup function will be called at most once.  Since Python's
  internal finalization will have completed before the cleanup
  function, no Python APIs should be called by \var{func}.
\end{cfuncdesc}


\section{Importing Modules \label{importing}}

\begin{cfuncdesc}{PyObject*}{PyImport_ImportModule}{char *name}
  This is a simplified interface to
  \cfunction{PyImport_ImportModuleEx()} below, leaving the
  \var{globals} and \var{locals} arguments set to \NULL.  When the
  \var{name} argument contains a dot (when it specifies a submodule of
  a package), the \var{fromlist} argument is set to the list
  \code{['*']} so that the return value is the named module rather
  than the top-level package containing it as would otherwise be the
  case.  (Unfortunately, this has an additional side effect when
  \var{name} in fact specifies a subpackage instead of a submodule:
  the submodules specified in the package's \code{__all__} variable
  are \index{package variable!\code{__all__}}
  \withsubitem{(package variable)}{\ttindex{__all__}}loaded.)  Return
  a new reference to the imported module, or \NULL{} with an exception
  set on failure.  Before Python 2.4, the module may still be created in
  the failure case --- examine \code{sys.modules} to find out.  Starting
  with Python 2.4, a failing import of a module no longer leaves the
  module in \code{sys.modules}.
  \versionchanged[failing imports remove incomplete module objects]{2.4}
  \withsubitem{(in module sys)}{\ttindex{modules}}
\end{cfuncdesc}

\begin{cfuncdesc}{PyObject*}{PyImport_ImportModuleEx}{char *name,
                       PyObject *globals, PyObject *locals, PyObject *fromlist}
  Import a module.  This is best described by referring to the
  built-in Python function
  \function{__import__()}\bifuncindex{__import__}, as the standard
  \function{__import__()} function calls this function directly.

  The return value is a new reference to the imported module or
  top-level package, or \NULL{} with an exception set on failure (before
  Python 2.4, the
  module may still be created in this case).  Like for
  \function{__import__()}, the return value when a submodule of a
  package was requested is normally the top-level package, unless a
  non-empty \var{fromlist} was given.
  \versionchanged[failing imports remove incomplete module objects]{2.4}
\end{cfuncdesc}

\begin{cfuncdesc}{PyObject*}{PyImport_Import}{PyObject *name}
  This is a higher-level interface that calls the current ``import
  hook function''.  It invokes the \function{__import__()} function
  from the \code{__builtins__} of the current globals.  This means
  that the import is done using whatever import hooks are installed in
  the current environment, e.g. by \module{rexec}\refstmodindex{rexec}
  or \module{ihooks}\refstmodindex{ihooks}.
\end{cfuncdesc}

\begin{cfuncdesc}{PyObject*}{PyImport_ReloadModule}{PyObject *m}
  Reload a module.  This is best described by referring to the
  built-in Python function \function{reload()}\bifuncindex{reload}, as
  the standard \function{reload()} function calls this function
  directly.  Return a new reference to the reloaded module, or \NULL{}
  with an exception set on failure (the module still exists in this
  case).
\end{cfuncdesc}

\begin{cfuncdesc}{PyObject*}{PyImport_AddModule}{char *name}
  Return the module object corresponding to a module name.  The
  \var{name} argument may be of the form \code{package.module}.
  First check the modules dictionary if there's one there, and if not,
  create a new one and insert it in the modules dictionary.
  Return \NULL{} with an exception set on failure.
  \note{This function does not load or import the module; if the
  module wasn't already loaded, you will get an empty module object.
  Use \cfunction{PyImport_ImportModule()} or one of its variants to
  import a module.  Package structures implied by a dotted name for
  \var{name} are not created if not already present.}
\end{cfuncdesc}

\begin{cfuncdesc}{PyObject*}{PyImport_ExecCodeModule}{char *name, PyObject *co}
  Given a module name (possibly of the form \code{package.module}) and
  a code object read from a Python bytecode file or obtained from the
  built-in function \function{compile()}\bifuncindex{compile}, load
  the module.  Return a new reference to the module object, or \NULL{}
  with an exception set if an error occurred.  Before Python 2.4, the module
  could still be created in error cases.  Starting with Python 2.4,
  \var{name} is removed from \code{sys.modules} in error cases, and even
  if \var{name} was already in \code{sys.modules} on entry to
  \cfunction{PyImport_ExecCodeModule()}.  Leaving incompletely initialized
  modules in \code{sys.modules} is dangerous, as imports of such modules
  have no way to know that the module object is an unknown (and probably
  damaged with respect to the module author's intents) state.

  This function will reload the module if it was already imported.  See
  \cfunction{PyImport_ReloadModule()} for the intended way to reload a
  module.

  If \var{name} points to a dotted name of the
  form \code{package.module}, any package structures not already
  created will still not be created.

  \versionchanged[\var{name} is removed from \code{sys.modules} in error cases]{2.4}

\end{cfuncdesc}

\begin{cfuncdesc}{long}{PyImport_GetMagicNumber}{}
  Return the magic number for Python bytecode files
  (a.k.a. \file{.pyc} and \file{.pyo} files).  The magic number should
  be present in the first four bytes of the bytecode file, in
  little-endian byte order.
\end{cfuncdesc}

\begin{cfuncdesc}{PyObject*}{PyImport_GetModuleDict}{}
  Return the dictionary used for the module administration
  (a.k.a.\ \code{sys.modules}).  Note that this is a per-interpreter
  variable.
\end{cfuncdesc}

\begin{cfuncdesc}{void}{_PyImport_Init}{}
  Initialize the import mechanism.  For internal use only.
\end{cfuncdesc}

\begin{cfuncdesc}{void}{PyImport_Cleanup}{}
  Empty the module table.  For internal use only.
\end{cfuncdesc}

\begin{cfuncdesc}{void}{_PyImport_Fini}{}
  Finalize the import mechanism.  For internal use only.
\end{cfuncdesc}

\begin{cfuncdesc}{PyObject*}{_PyImport_FindExtension}{char *, char *}
  For internal use only.
\end{cfuncdesc}

\begin{cfuncdesc}{PyObject*}{_PyImport_FixupExtension}{char *, char *}
  For internal use only.
\end{cfuncdesc}

\begin{cfuncdesc}{int}{PyImport_ImportFrozenModule}{char *name}
  Load a frozen module named \var{name}.  Return \code{1} for success,
  \code{0} if the module is not found, and \code{-1} with an exception
  set if the initialization failed.  To access the imported module on
  a successful load, use \cfunction{PyImport_ImportModule()}.  (Note
  the misnomer --- this function would reload the module if it was
  already imported.)
\end{cfuncdesc}

\begin{ctypedesc}[_frozen]{struct _frozen}
  This is the structure type definition for frozen module descriptors,
  as generated by the \program{freeze}\index{freeze utility} utility
  (see \file{Tools/freeze/} in the Python source distribution).  Its
  definition, found in \file{Include/import.h}, is:

\begin{verbatim}
struct _frozen {
    char *name;
    unsigned char *code;
    int size;
};
\end{verbatim}
\end{ctypedesc}

\begin{cvardesc}{struct _frozen*}{PyImport_FrozenModules}
  This pointer is initialized to point to an array of \ctype{struct
  _frozen} records, terminated by one whose members are all \NULL{} or
  zero.  When a frozen module is imported, it is searched in this
  table.  Third-party code could play tricks with this to provide a
  dynamically created collection of frozen modules.
\end{cvardesc}

\begin{cfuncdesc}{int}{PyImport_AppendInittab}{char *name,
                                               void (*initfunc)(void)}
  Add a single module to the existing table of built-in modules.  This
  is a convenience wrapper around
  \cfunction{PyImport_ExtendInittab()}, returning \code{-1} if the
  table could not be extended.  The new module can be imported by the
  name \var{name}, and uses the function \var{initfunc} as the
  initialization function called on the first attempted import.  This
  should be called before \cfunction{Py_Initialize()}.
\end{cfuncdesc}

\begin{ctypedesc}[_inittab]{struct _inittab}
  Structure describing a single entry in the list of built-in
  modules.  Each of these structures gives the name and initialization
  function for a module built into the interpreter.  Programs which
  embed Python may use an array of these structures in conjunction
  with \cfunction{PyImport_ExtendInittab()} to provide additional
  built-in modules.  The structure is defined in
  \file{Include/import.h} as:

\begin{verbatim}
struct _inittab {
    char *name;
    void (*initfunc)(void);
};
\end{verbatim}
\end{ctypedesc}

\begin{cfuncdesc}{int}{PyImport_ExtendInittab}{struct _inittab *newtab}
  Add a collection of modules to the table of built-in modules.  The
  \var{newtab} array must end with a sentinel entry which contains
  \NULL{} for the \member{name} field; failure to provide the sentinel
  value can result in a memory fault.  Returns \code{0} on success or
  \code{-1} if insufficient memory could be allocated to extend the
  internal table.  In the event of failure, no modules are added to
  the internal table.  This should be called before
  \cfunction{Py_Initialize()}.
\end{cfuncdesc}


\section{Data marshalling support \label{marshalling-utils}}

These routines allow C code to work with serialized objects using the
same data format as the \module{marshal} module.  There are functions
to write data into the serialization format, and additional functions
that can be used to read the data back.  Files used to store marshalled
data must be opened in binary mode.

Numeric values are stored with the least significant byte first.

The module supports two versions of the data format: version 0 is the
historical version, version 1 (new in Python 2.4) shares interned
strings in the file, and upon unmarshalling. \var{Py_MARSHAL_VERSION}
indicates the current file format (currently 1).

\begin{cfuncdesc}{void}{PyMarshal_WriteLongToFile}{long value, FILE *file, int version}
  Marshal a \ctype{long} integer, \var{value}, to \var{file}.  This
  will only write the least-significant 32 bits of \var{value};
  regardless of the size of the native \ctype{long} type.

  \versionchanged[\var{version} indicates the file format]{2.4}
\end{cfuncdesc}

\begin{cfuncdesc}{void}{PyMarshal_WriteObjectToFile}{PyObject *value,
                                                     FILE *file, int version}
  Marshal a Python object, \var{value}, to \var{file}.

  \versionchanged[\var{version} indicates the file format]{2.4}
\end{cfuncdesc}

\begin{cfuncdesc}{PyObject*}{PyMarshal_WriteObjectToString}{PyObject *value, int version}
  Return a string object containing the marshalled representation of
  \var{value}.

  \versionchanged[\var{version} indicates the file format]{2.4}
\end{cfuncdesc}

The following functions allow marshalled values to be read back in.

XXX What about error detection?  It appears that reading past the end
of the file will always result in a negative numeric value (where
that's relevant), but it's not clear that negative values won't be
handled properly when there's no error.  What's the right way to tell?
Should only non-negative values be written using these routines?

\begin{cfuncdesc}{long}{PyMarshal_ReadLongFromFile}{FILE *file}
  Return a C \ctype{long} from the data stream in a \ctype{FILE*}
  opened for reading.  Only a 32-bit value can be read in using
  this function, regardless of the native size of \ctype{long}.
\end{cfuncdesc}

\begin{cfuncdesc}{int}{PyMarshal_ReadShortFromFile}{FILE *file}
  Return a C \ctype{short} from the data stream in a \ctype{FILE*}
  opened for reading.  Only a 16-bit value can be read in using
  this function, regardless of the native size of \ctype{short}.
\end{cfuncdesc}

\begin{cfuncdesc}{PyObject*}{PyMarshal_ReadObjectFromFile}{FILE *file}
  Return a Python object from the data stream in a \ctype{FILE*}
  opened for reading.  On error, sets the appropriate exception
  (\exception{EOFError} or \exception{TypeError}) and returns \NULL.
\end{cfuncdesc}

\begin{cfuncdesc}{PyObject*}{PyMarshal_ReadLastObjectFromFile}{FILE *file}
  Return a Python object from the data stream in a \ctype{FILE*}
  opened for reading.  Unlike
  \cfunction{PyMarshal_ReadObjectFromFile()}, this function assumes
  that no further objects will be read from the file, allowing it to
  aggressively load file data into memory so that the de-serialization
  can operate from data in memory rather than reading a byte at a time
  from the file.  Only use these variant if you are certain that you
  won't be reading anything else from the file.  On error, sets the
  appropriate exception (\exception{EOFError} or
  \exception{TypeError}) and returns \NULL.
\end{cfuncdesc}

\begin{cfuncdesc}{PyObject*}{PyMarshal_ReadObjectFromString}{char *string,
                                                             int len}
  Return a Python object from the data stream in a character buffer
  containing \var{len} bytes pointed to by \var{string}.  On error,
  sets the appropriate exception (\exception{EOFError} or
  \exception{TypeError}) and returns \NULL.
\end{cfuncdesc}


\section{Parsing arguments and building values
         \label{arg-parsing}}

These functions are useful when creating your own extensions functions
and methods.  Additional information and examples are available in
\citetitle[../ext/ext.html]{Extending and Embedding the Python
Interpreter}.

The first three of these functions described,
\cfunction{PyArg_ParseTuple()},
\cfunction{PyArg_ParseTupleAndKeywords()}, and
\cfunction{PyArg_Parse()}, all use \emph{format strings} which are
used to tell the function about the expected arguments.  The format
strings use the same syntax for each of these functions.

A format string consists of zero or more ``format units.''  A format
unit describes one Python object; it is usually a single character or
a parenthesized sequence of format units.  With a few exceptions, a
format unit that is not a parenthesized sequence normally corresponds
to a single address argument to these functions.  In the following
description, the quoted form is the format unit; the entry in (round)
parentheses is the Python object type that matches the format unit;
and the entry in [square] brackets is the type of the C variable(s)
whose address should be passed.

\begin{description}
  \item[\samp{s} (string or Unicode object) {[const char *]}]
  Convert a Python string or Unicode object to a C pointer to a
  character string.  You must not provide storage for the string
  itself; a pointer to an existing string is stored into the character
  pointer variable whose address you pass.  The C string is
  NUL-terminated.  The Python string must not contain embedded NUL
  bytes; if it does, a \exception{TypeError} exception is raised.
  Unicode objects are converted to C strings using the default
  encoding.  If this conversion fails, a \exception{UnicodeError} is
  raised.

  \item[\samp{s\#} (string, Unicode or any read buffer compatible object)
  {[const char *, int]}]
  This variant on \samp{s} stores into two C variables, the first one
  a pointer to a character string, the second one its length.  In this
  case the Python string may contain embedded null bytes.  Unicode
  objects pass back a pointer to the default encoded string version of
  the object if such a conversion is possible.  All other read-buffer
  compatible objects pass back a reference to the raw internal data
  representation.

  \item[\samp{z} (string or \code{None}) {[const char *]}]
  Like \samp{s}, but the Python object may also be \code{None}, in
  which case the C pointer is set to \NULL.

  \item[\samp{z\#} (string or \code{None} or any read buffer
  compatible object) {[const char *, int]}]
  This is to \samp{s\#} as \samp{z} is to \samp{s}.

  \item[\samp{u} (Unicode object) {[Py_UNICODE *]}]
  Convert a Python Unicode object to a C pointer to a NUL-terminated
  buffer of 16-bit Unicode (UTF-16) data.  As with \samp{s}, there is
  no need to provide storage for the Unicode data buffer; a pointer to
  the existing Unicode data is stored into the \ctype{Py_UNICODE}
  pointer variable whose address you pass.

  \item[\samp{u\#} (Unicode object) {[Py_UNICODE *, int]}]
  This variant on \samp{u} stores into two C variables, the first one
  a pointer to a Unicode data buffer, the second one its length.
  Non-Unicode objects are handled by interpreting their read-buffer
  pointer as pointer to a \ctype{Py_UNICODE} array.

  \item[\samp{es} (string, Unicode object or character buffer
  compatible object) {[const char *encoding, char **buffer]}]
  This variant on \samp{s} is used for encoding Unicode and objects
  convertible to Unicode into a character buffer. It only works for
  encoded data without embedded NUL bytes.

  This format requires two arguments.  The first is only used as
  input, and must be a \ctype{const char*} which points to the name of an
  encoding as a NUL-terminated string, or \NULL, in which case the
  default encoding is used.  An exception is raised if the named
  encoding is not known to Python.  The second argument must be a
  \ctype{char**}; the value of the pointer it references will be set
  to a buffer with the contents of the argument text.  The text will
  be encoded in the encoding specified by the first argument.

  \cfunction{PyArg_ParseTuple()} will allocate a buffer of the needed
  size, copy the encoded data into this buffer and adjust
  \var{*buffer} to reference the newly allocated storage.  The caller
  is responsible for calling \cfunction{PyMem_Free()} to free the
  allocated buffer after use.

  \item[\samp{et} (string, Unicode object or character buffer
  compatible object) {[const char *encoding, char **buffer]}]
  Same as \samp{es} except that 8-bit string objects are passed
  through without recoding them.  Instead, the implementation assumes
  that the string object uses the encoding passed in as parameter.

  \item[\samp{es\#} (string, Unicode object or character buffer compatible
  object) {[const char *encoding, char **buffer, int *buffer_length]}]
  This variant on \samp{s\#} is used for encoding Unicode and objects
  convertible to Unicode into a character buffer.  Unlike the
  \samp{es} format, this variant allows input data which contains NUL
  characters.

  It requires three arguments.  The first is only used as input, and
  must be a \ctype{const char*} which points to the name of an encoding as a
  NUL-terminated string, or \NULL, in which case the default encoding
  is used.  An exception is raised if the named encoding is not known
  to Python.  The second argument must be a \ctype{char**}; the value
  of the pointer it references will be set to a buffer with the
  contents of the argument text.  The text will be encoded in the
  encoding specified by the first argument.  The third argument must
  be a pointer to an integer; the referenced integer will be set to
  the number of bytes in the output buffer.

  There are two modes of operation:

  If \var{*buffer} points a \NULL{} pointer, the function will
  allocate a buffer of the needed size, copy the encoded data into
  this buffer and set \var{*buffer} to reference the newly allocated
  storage.  The caller is responsible for calling
  \cfunction{PyMem_Free()} to free the allocated buffer after usage.

  If \var{*buffer} points to a non-\NULL{} pointer (an already
  allocated buffer), \cfunction{PyArg_ParseTuple()} will use this
  location as the buffer and interpret the initial value of
  \var{*buffer_length} as the buffer size.  It will then copy the
  encoded data into the buffer and NUL-terminate it.  If the buffer
  is not large enough, a \exception{ValueError} will be set.

  In both cases, \var{*buffer_length} is set to the length of the
  encoded data without the trailing NUL byte.

  \item[\samp{et\#} (string, Unicode object or character buffer compatible
  object) {[const char *encoding, char **buffer]}]
  Same as \samp{es\#} except that string objects are passed through
  without recoding them. Instead, the implementation assumes that the
  string object uses the encoding passed in as parameter.

  \item[\samp{b} (integer) {[char]}]
  Convert a Python integer to a tiny int, stored in a C \ctype{char}.

  \item[\samp{B} (integer) {[unsigned char]}]
  Convert a Python integer to a tiny int without overflow checking,
  stored in a C \ctype{unsigned char}. \versionadded{2.3}

  \item[\samp{h} (integer) {[short int]}]
  Convert a Python integer to a C \ctype{short int}.

  \item[\samp{H} (integer) {[unsigned short int]}]
  Convert a Python integer to a C \ctype{unsigned short int}, without
  overflow checking.  \versionadded{2.3}

  \item[\samp{i} (integer) {[int]}]
  Convert a Python integer to a plain C \ctype{int}.

  \item[\samp{I} (integer) {[unsigned int]}]
  Convert a Python integer to a C \ctype{unsigned int}, without
  overflow checking.  \versionadded{2.3}

  \item[\samp{l} (integer) {[long int]}]
  Convert a Python integer to a C \ctype{long int}.

  \item[\samp{k} (integer) {[unsigned long]}]
  Convert a Python integer to a C \ctype{unsigned long} without
  overflow checking.  \versionadded{2.3}

  \item[\samp{L} (integer) {[PY_LONG_LONG]}]
  Convert a Python integer to a C \ctype{long long}.  This format is
  only available on platforms that support \ctype{long long} (or
  \ctype{_int64} on Windows).

  \item[\samp{K} (integer) {[unsigned PY_LONG_LONG]}]
  Convert a Python integer to a C \ctype{unsigned long long}
  without overflow checking.  This format is only available on
  platforms that support \ctype{unsigned long long} (or
  \ctype{unsigned _int64} on Windows).  \versionadded{2.3}

  \item[\samp{c} (string of length 1) {[char]}]
  Convert a Python character, represented as a string of length 1, to
  a C \ctype{char}.

  \item[\samp{f} (float) {[float]}]
  Convert a Python floating point number to a C \ctype{float}.

  \item[\samp{d} (float) {[double]}]
  Convert a Python floating point number to a C \ctype{double}.

  \item[\samp{D} (complex) {[Py_complex]}]
  Convert a Python complex number to a C \ctype{Py_complex} structure.

  \item[\samp{O} (object) {[PyObject *]}]
  Store a Python object (without any conversion) in a C object
  pointer.  The C program thus receives the actual object that was
  passed.  The object's reference count is not increased.  The pointer
  stored is not \NULL.

  \item[\samp{O!} (object) {[\var{typeobject}, PyObject *]}]
  Store a Python object in a C object pointer.  This is similar to
  \samp{O}, but takes two C arguments: the first is the address of a
  Python type object, the second is the address of the C variable (of
  type \ctype{PyObject*}) into which the object pointer is stored.  If
  the Python object does not have the required type,
  \exception{TypeError} is raised.

  \item[\samp{O\&} (object) {[\var{converter}, \var{anything}]}]
  Convert a Python object to a C variable through a \var{converter}
  function.  This takes two arguments: the first is a function, the
  second is the address of a C variable (of arbitrary type), converted
  to \ctype{void *}.  The \var{converter} function in turn is called
  as follows:

  \var{status}\code{ = }\var{converter}\code{(}\var{object},
  \var{address}\code{);}

  where \var{object} is the Python object to be converted and
  \var{address} is the \ctype{void*} argument that was passed to the
  \cfunction{PyArg_Parse*()} function.  The returned \var{status}
  should be \code{1} for a successful conversion and \code{0} if the
  conversion has failed.  When the conversion fails, the
  \var{converter} function should raise an exception.

  \item[\samp{S} (string) {[PyStringObject *]}]
  Like \samp{O} but requires that the Python object is a string
  object.  Raises \exception{TypeError} if the object is not a string
  object.  The C variable may also be declared as \ctype{PyObject*}.

  \item[\samp{U} (Unicode string) {[PyUnicodeObject *]}]
  Like \samp{O} but requires that the Python object is a Unicode
  object.  Raises \exception{TypeError} if the object is not a Unicode
  object.  The C variable may also be declared as \ctype{PyObject*}.

  \item[\samp{t\#} (read-only character buffer) {[char *, int]}]
  Like \samp{s\#}, but accepts any object which implements the
  read-only buffer interface.  The \ctype{char*} variable is set to
  point to the first byte of the buffer, and the \ctype{int} is set to
  the length of the buffer.  Only single-segment buffer objects are
  accepted; \exception{TypeError} is raised for all others.

  \item[\samp{w} (read-write character buffer) {[char *]}]
  Similar to \samp{s}, but accepts any object which implements the
  read-write buffer interface.  The caller must determine the length
  of the buffer by other means, or use \samp{w\#} instead.  Only
  single-segment buffer objects are accepted; \exception{TypeError} is
  raised for all others.

  \item[\samp{w\#} (read-write character buffer) {[char *, int]}]
  Like \samp{s\#}, but accepts any object which implements the
  read-write buffer interface.  The \ctype{char *} variable is set to
  point to the first byte of the buffer, and the \ctype{int} is set to
  the length of the buffer.  Only single-segment buffer objects are
  accepted; \exception{TypeError} is raised for all others.

  \item[\samp{(\var{items})} (tuple) {[\var{matching-items}]}]
  The object must be a Python sequence whose length is the number of
  format units in \var{items}.  The C arguments must correspond to the
  individual format units in \var{items}.  Format units for sequences
  may be nested.

  \note{Prior to Python version 1.5.2, this format specifier only
  accepted a tuple containing the individual parameters, not an
  arbitrary sequence.  Code which previously caused
  \exception{TypeError} to be raised here may now proceed without an
  exception.  This is not expected to be a problem for existing code.}
\end{description}

It is possible to pass Python long integers where integers are
requested; however no proper range checking is done --- the most
significant bits are silently truncated when the receiving field is
too small to receive the value (actually, the semantics are inherited
from downcasts in C --- your mileage may vary).

A few other characters have a meaning in a format string.  These may
not occur inside nested parentheses.  They are:

\begin{description}
  \item[\samp{|}]
  Indicates that the remaining arguments in the Python argument list
  are optional.  The C variables corresponding to optional arguments
  should be initialized to their default value --- when an optional
  argument is not specified, \cfunction{PyArg_ParseTuple()} does not
  touch the contents of the corresponding C variable(s).

  \item[\samp{:}]
  The list of format units ends here; the string after the colon is
  used as the function name in error messages (the ``associated
  value'' of the exception that \cfunction{PyArg_ParseTuple()}
  raises).

  \item[\samp{;}]
  The list of format units ends here; the string after the semicolon
  is used as the error message \emph{instead} of the default error
  message.  Clearly, \samp{:} and \samp{;} mutually exclude each
  other.
\end{description}

Note that any Python object references which are provided to the
caller are \emph{borrowed} references; do not decrement their
reference count!

Additional arguments passed to these functions must be addresses of
variables whose type is determined by the format string; these are
used to store values from the input tuple.  There are a few cases, as
described in the list of format units above, where these parameters
are used as input values; they should match what is specified for the
corresponding format unit in that case.

For the conversion to succeed, the \var{arg} object must match the
format and the format must be exhausted.  On success, the
\cfunction{PyArg_Parse*()} functions return true, otherwise they
return false and raise an appropriate exception.

\begin{cfuncdesc}{int}{PyArg_ParseTuple}{PyObject *args, char *format,
                                         \moreargs}
  Parse the parameters of a function that takes only positional
  parameters into local variables.  Returns true on success; on
  failure, it returns false and raises the appropriate exception.
\end{cfuncdesc}

\begin{cfuncdesc}{int}{PyArg_VaParse}{PyObject *args, char *format,
                                         va_list vargs}
  Identical to \cfunction{PyArg_ParseTuple()}, except that it accepts a
  va_list rather than a variable number of arguments.
\end{cfuncdesc}

\begin{cfuncdesc}{int}{PyArg_ParseTupleAndKeywords}{PyObject *args,
                       PyObject *kw, char *format, char *keywords[],
                       \moreargs}
  Parse the parameters of a function that takes both positional and
  keyword parameters into local variables.  Returns true on success;
  on failure, it returns false and raises the appropriate exception.
\end{cfuncdesc}

\begin{cfuncdesc}{int}{PyArg_VaParseTupleAndKeywords}{PyObject *args,
                       PyObject *kw, char *format, char *keywords[],
                       va_list vargs}
  Identical to \cfunction{PyArg_ParseTupleAndKeywords()}, except that it
  accepts a va_list rather than a variable number of arguments.
\end{cfuncdesc}

\begin{cfuncdesc}{int}{PyArg_Parse}{PyObject *args, char *format,
                                    \moreargs}
  Function used to deconstruct the argument lists of ``old-style''
  functions --- these are functions which use the
  \constant{METH_OLDARGS} parameter parsing method.  This is not
  recommended for use in parameter parsing in new code, and most code
  in the standard interpreter has been modified to no longer use this
  for that purpose.  It does remain a convenient way to decompose
  other tuples, however, and may continue to be used for that
  purpose.
\end{cfuncdesc}

\begin{cfuncdesc}{int}{PyArg_UnpackTuple}{PyObject *args, char *name,
                                          int min, int max, \moreargs}
  A simpler form of parameter retrieval which does not use a format
  string to specify the types of the arguments.  Functions which use
  this method to retrieve their parameters should be declared as
  \constant{METH_VARARGS} in function or method tables.  The tuple
  containing the actual parameters should be passed as \var{args}; it
  must actually be a tuple.  The length of the tuple must be at least
  \var{min} and no more than \var{max}; \var{min} and \var{max} may be
  equal.  Additional arguments must be passed to the function, each of
  which should be a pointer to a \ctype{PyObject*} variable; these
  will be filled in with the values from \var{args}; they will contain
  borrowed references.  The variables which correspond to optional
  parameters not given by \var{args} will not be filled in; these
  should be initialized by the caller.
  This function returns true on success and false if \var{args} is not
  a tuple or contains the wrong number of elements; an exception will
  be set if there was a failure.

  This is an example of the use of this function, taken from the
  sources for the \module{_weakref} helper module for weak references:

\begin{verbatim}
static PyObject *
weakref_ref(PyObject *self, PyObject *args)
{
    PyObject *object;
    PyObject *callback = NULL;
    PyObject *result = NULL;

    if (PyArg_UnpackTuple(args, "ref", 1, 2, &object, &callback)) {
        result = PyWeakref_NewRef(object, callback);
    }
    return result;
}
\end{verbatim}

  The call to \cfunction{PyArg_UnpackTuple()} in this example is
  entirely equivalent to this call to \cfunction{PyArg_ParseTuple()}:

\begin{verbatim}
PyArg_ParseTuple(args, "O|O:ref", &object, &callback)
\end{verbatim}

  \versionadded{2.2}
\end{cfuncdesc}

\begin{cfuncdesc}{PyObject*}{Py_BuildValue}{char *format,
                                            \moreargs}
  Create a new value based on a format string similar to those
  accepted by the \cfunction{PyArg_Parse*()} family of functions and a
  sequence of values.  Returns the value or \NULL{} in the case of an
  error; an exception will be raised if \NULL{} is returned.

  \cfunction{Py_BuildValue()} does not always build a tuple.  It
  builds a tuple only if its format string contains two or more format
  units.  If the format string is empty, it returns \code{None}; if it
  contains exactly one format unit, it returns whatever object is
  described by that format unit.  To force it to return a tuple of
  size 0 or one, parenthesize the format string.

  When memory buffers are passed as parameters to supply data to build
  objects, as for the \samp{s} and \samp{s\#} formats, the required
  data is copied.  Buffers provided by the caller are never referenced
  by the objects created by \cfunction{Py_BuildValue()}.  In other
  words, if your code invokes \cfunction{malloc()} and passes the
  allocated memory to \cfunction{Py_BuildValue()}, your code is
  responsible for calling \cfunction{free()} for that memory once
  \cfunction{Py_BuildValue()} returns.

  In the following description, the quoted form is the format unit;
  the entry in (round) parentheses is the Python object type that the
  format unit will return; and the entry in [square] brackets is the
  type of the C value(s) to be passed.

  The characters space, tab, colon and comma are ignored in format
  strings (but not within format units such as \samp{s\#}).  This can
  be used to make long format strings a tad more readable.

  \begin{description}
    \item[\samp{s} (string) {[char *]}]
    Convert a null-terminated C string to a Python object.  If the C
    string pointer is \NULL, \code{None} is used.

    \item[\samp{s\#} (string) {[char *, int]}]
    Convert a C string and its length to a Python object.  If the C
    string pointer is \NULL, the length is ignored and \code{None} is
    returned.

    \item[\samp{z} (string or \code{None}) {[char *]}]
    Same as \samp{s}.

    \item[\samp{z\#} (string or \code{None}) {[char *, int]}]
    Same as \samp{s\#}.

    \item[\samp{u} (Unicode string) {[Py_UNICODE *]}]
    Convert a null-terminated buffer of Unicode (UCS-2 or UCS-4)
    data to a Python Unicode object.  If the Unicode buffer pointer
    is \NULL, \code{None} is returned.

    \item[\samp{u\#} (Unicode string) {[Py_UNICODE *, int]}]
    Convert a Unicode (UCS-2 or UCS-4) data buffer and its length
    to a Python Unicode object.   If the Unicode buffer pointer
    is \NULL, the length is ignored and \code{None} is returned.

    \item[\samp{i} (integer) {[int]}]
    Convert a plain C \ctype{int} to a Python integer object.

    \item[\samp{b} (integer) {[char]}]
    Same as \samp{i}.

    \item[\samp{h} (integer) {[short int]}]
    Same as \samp{i}.

    \item[\samp{l} (integer) {[long int]}]
    Convert a C \ctype{long int} to a Python integer object.

    \item[\samp{c} (string of length 1) {[char]}]
    Convert a C \ctype{int} representing a character to a Python
    string of length 1.

    \item[\samp{d} (float) {[double]}]
    Convert a C \ctype{double} to a Python floating point number.

    \item[\samp{f} (float) {[float]}]
    Same as \samp{d}.

    \item[\samp{D} (complex) {[Py_complex *]}]
    Convert a C \ctype{Py_complex} structure to a Python complex
    number.

    \item[\samp{O} (object) {[PyObject *]}]
    Pass a Python object untouched (except for its reference count,
    which is incremented by one).  If the object passed in is a
    \NULL{} pointer, it is assumed that this was caused because the
    call producing the argument found an error and set an exception.
    Therefore, \cfunction{Py_BuildValue()} will return \NULL{} but
    won't raise an exception.  If no exception has been raised yet,
    \exception{SystemError} is set.

    \item[\samp{S} (object) {[PyObject *]}]
    Same as \samp{O}.

    \item[\samp{N} (object) {[PyObject *]}]
    Same as \samp{O}, except it doesn't increment the reference count
    on the object.  Useful when the object is created by a call to an
    object constructor in the argument list.

    \item[\samp{O\&} (object) {[\var{converter}, \var{anything}]}]
    Convert \var{anything} to a Python object through a
    \var{converter} function.  The function is called with
    \var{anything} (which should be compatible with \ctype{void *}) as
    its argument and should return a ``new'' Python object, or \NULL{}
    if an error occurred.

    \item[\samp{(\var{items})} (tuple) {[\var{matching-items}]}]
    Convert a sequence of C values to a Python tuple with the same
    number of items.

    \item[\samp{[\var{items}]} (list) {[\var{matching-items}]}]
    Convert a sequence of C values to a Python list with the same
    number of items.

    \item[\samp{\{\var{items}\}} (dictionary) {[\var{matching-items}]}]
    Convert a sequence of C values to a Python dictionary.  Each pair
    of consecutive C values adds one item to the dictionary, serving
    as key and value, respectively.

  \end{description}

  If there is an error in the format string, the
  \exception{SystemError} exception is set and \NULL{} returned.
\end{cfuncdesc}

\chapter{Abstract Objects Layer \label{abstract}}

The functions in this chapter interact with Python objects regardless
of their type, or with wide classes of object types (e.g. all
numerical types, or all sequence types).  When used on object types
for which they do not apply, they will raise a Python exception.


\section{Object Protocol \label{object}}

\begin{cfuncdesc}{int}{PyObject_Print}{PyObject *o, FILE *fp, int flags}
  Print an object \var{o}, on file \var{fp}.  Returns \code{-1} on
  error.  The flags argument is used to enable certain printing
  options.  The only option currently supported is
  \constant{Py_PRINT_RAW}; if given, the \function{str()} of the
  object is written instead of the \function{repr()}.
\end{cfuncdesc}

\begin{cfuncdesc}{int}{PyObject_HasAttrString}{PyObject *o, char *attr_name}
  Returns \code{1} if \var{o} has the attribute \var{attr_name}, and
  \code{0} otherwise.  This is equivalent to the Python expression
  \samp{hasattr(\var{o}, \var{attr_name})}.  This function always
  succeeds.
\end{cfuncdesc}

\begin{cfuncdesc}{PyObject*}{PyObject_GetAttrString}{PyObject *o,
                                                     char *attr_name}
  Retrieve an attribute named \var{attr_name} from object \var{o}.
  Returns the attribute value on success, or \NULL{} on failure.
  This is the equivalent of the Python expression
  \samp{\var{o}.\var{attr_name}}.
\end{cfuncdesc}


\begin{cfuncdesc}{int}{PyObject_HasAttr}{PyObject *o, PyObject *attr_name}
  Returns \code{1} if \var{o} has the attribute \var{attr_name}, and
  \code{0} otherwise.  This is equivalent to the Python expression
  \samp{hasattr(\var{o}, \var{attr_name})}.  This function always
  succeeds.
\end{cfuncdesc}


\begin{cfuncdesc}{PyObject*}{PyObject_GetAttr}{PyObject *o,
                                               PyObject *attr_name}
  Retrieve an attribute named \var{attr_name} from object \var{o}.
  Returns the attribute value on success, or \NULL{} on failure.  This
  is the equivalent of the Python expression
  \samp{\var{o}.\var{attr_name}}.
\end{cfuncdesc}


\begin{cfuncdesc}{int}{PyObject_SetAttrString}{PyObject *o,
                                               char *attr_name, PyObject *v}
  Set the value of the attribute named \var{attr_name}, for object
  \var{o}, to the value \var{v}. Returns \code{-1} on failure.  This
  is the equivalent of the Python statement
  \samp{\var{o}.\var{attr_name} = \var{v}}.
\end{cfuncdesc}


\begin{cfuncdesc}{int}{PyObject_SetAttr}{PyObject *o,
                                         PyObject *attr_name, PyObject *v}
  Set the value of the attribute named \var{attr_name}, for object
  \var{o}, to the value \var{v}. Returns \code{-1} on failure.  This
  is the equivalent of the Python statement
  \samp{\var{o}.\var{attr_name} = \var{v}}.
\end{cfuncdesc}


\begin{cfuncdesc}{int}{PyObject_DelAttrString}{PyObject *o, char *attr_name}
  Delete attribute named \var{attr_name}, for object \var{o}. Returns
  \code{-1} on failure.  This is the equivalent of the Python
  statement: \samp{del \var{o}.\var{attr_name}}.
\end{cfuncdesc}


\begin{cfuncdesc}{int}{PyObject_DelAttr}{PyObject *o, PyObject *attr_name}
  Delete attribute named \var{attr_name}, for object \var{o}. Returns
  \code{-1} on failure.  This is the equivalent of the Python
  statement \samp{del \var{o}.\var{attr_name}}.
\end{cfuncdesc}


\begin{cfuncdesc}{PyObject*}{PyObject_RichCompare}{PyObject *o1,
                                                   PyObject *o2, int opid}
  Compare the values of \var{o1} and \var{o2} using the operation
  specified by \var{opid}, which must be one of
  \constant{Py_LT},
  \constant{Py_LE},
  \constant{Py_EQ},
  \constant{Py_NE},
  \constant{Py_GT}, or
  \constant{Py_GE}, corresponding to
  \code{<},
  \code{<=},
  \code{==},
  \code{!=},
  \code{>}, or
  \code{>=} respectively. This is the equivalent of the Python expression
  \samp{\var{o1} op \var{o2}}, where \code{op} is the operator
  corresponding to \var{opid}. Returns the value of the comparison on
  success, or \NULL{} on failure.
\end{cfuncdesc}

\begin{cfuncdesc}{int}{PyObject_RichCompareBool}{PyObject *o1,
                                                 PyObject *o2, int opid}
  Compare the values of \var{o1} and \var{o2} using the operation
  specified by \var{opid}, which must be one of
  \constant{Py_LT},
  \constant{Py_LE},
  \constant{Py_EQ},
  \constant{Py_NE},
  \constant{Py_GT}, or
  \constant{Py_GE}, corresponding to
  \code{<},
  \code{<=},
  \code{==},
  \code{!=},
  \code{>}, or
  \code{>=} respectively. Returns \code{-1} on error, \code{0} if the
  result is false, \code{1} otherwise. This is the equivalent of the
  Python expression \samp{\var{o1} op \var{o2}}, where
  \code{op} is the operator corresponding to \var{opid}.
\end{cfuncdesc}

\begin{cfuncdesc}{int}{PyObject_Cmp}{PyObject *o1, PyObject *o2, int *result}
  Compare the values of \var{o1} and \var{o2} using a routine provided
  by \var{o1}, if one exists, otherwise with a routine provided by
  \var{o2}.  The result of the comparison is returned in
  \var{result}.  Returns \code{-1} on failure.  This is the equivalent
  of the Python statement\bifuncindex{cmp} \samp{\var{result} =
  cmp(\var{o1}, \var{o2})}.
\end{cfuncdesc}


\begin{cfuncdesc}{int}{PyObject_Compare}{PyObject *o1, PyObject *o2}
  Compare the values of \var{o1} and \var{o2} using a routine provided
  by \var{o1}, if one exists, otherwise with a routine provided by
  \var{o2}.  Returns the result of the comparison on success.  On
  error, the value returned is undefined; use
  \cfunction{PyErr_Occurred()} to detect an error.  This is equivalent
  to the Python expression\bifuncindex{cmp} \samp{cmp(\var{o1},
  \var{o2})}.
\end{cfuncdesc}


\begin{cfuncdesc}{PyObject*}{PyObject_Repr}{PyObject *o}
  Compute a string representation of object \var{o}.  Returns the
  string representation on success, \NULL{} on failure.  This is the
  equivalent of the Python expression \samp{repr(\var{o})}.  Called by
  the \function{repr()}\bifuncindex{repr} built-in function and by
  reverse quotes.
\end{cfuncdesc}


\begin{cfuncdesc}{PyObject*}{PyObject_Str}{PyObject *o}
  Compute a string representation of object \var{o}.  Returns the
  string representation on success, \NULL{} on failure.  This is the
  equivalent of the Python expression \samp{str(\var{o})}.  Called by
  the \function{str()}\bifuncindex{str} built-in function and by the
  \keyword{print} statement.
\end{cfuncdesc}


\begin{cfuncdesc}{PyObject*}{PyObject_Unicode}{PyObject *o}
  Compute a Unicode string representation of object \var{o}.  Returns
  the Unicode string representation on success, \NULL{} on failure.
  This is the equivalent of the Python expression
  \samp{unicode(\var{o})}.  Called by the
  \function{unicode()}\bifuncindex{unicode} built-in function.
\end{cfuncdesc}

\begin{cfuncdesc}{int}{PyObject_IsInstance}{PyObject *inst, PyObject *cls}
  Returns \code{1} if \var{inst} is an instance of the class \var{cls}
  or a subclass of \var{cls}, or \code{0} if not.  On error, returns
  \code{-1} and sets an exception.  If \var{cls} is a type object
  rather than a class object, \cfunction{PyObject_IsInstance()}
  returns \code{1} if \var{inst} is of type \var{cls}.  If \var{cls}
  is a tuple, the check will be done against every entry in \var{cls}.
  The result will be \code{1} when at least one of the checks returns
  \code{1}, otherwise it will be \code{0}. If \var{inst} is not a class
  instance and \var{cls} is neither a type object, nor a class object,
  nor a tuple, \var{inst} must have a \member{__class__} attribute
  --- the class relationship of the value of that attribute with
  \var{cls} will be used to determine the result of this function.
  \versionadded{2.1}
  \versionchanged[Support for a tuple as the second argument added]{2.2}
\end{cfuncdesc}

Subclass determination is done in a fairly straightforward way, but
includes a wrinkle that implementors of extensions to the class system
may want to be aware of.  If \class{A} and \class{B} are class
objects, \class{B} is a subclass of \class{A} if it inherits from
\class{A} either directly or indirectly.  If either is not a class
object, a more general mechanism is used to determine the class
relationship of the two objects.  When testing if \var{B} is a
subclass of \var{A}, if \var{A} is \var{B},
\cfunction{PyObject_IsSubclass()} returns true.  If \var{A} and
\var{B} are different objects, \var{B}'s \member{__bases__} attribute
is searched in a depth-first fashion for \var{A} --- the presence of
the \member{__bases__} attribute is considered sufficient for this
determination.

\begin{cfuncdesc}{int}{PyObject_IsSubclass}{PyObject *derived,
                                            PyObject *cls}
  Returns \code{1} if the class \var{derived} is identical to or
  derived from the class \var{cls}, otherwise returns \code{0}.  In
  case of an error, returns \code{-1}. If \var{cls}
  is a tuple, the check will be done against every entry in \var{cls}.
  The result will be \code{1} when at least one of the checks returns
  \code{1}, otherwise it will be \code{0}. If either \var{derived} or
  \var{cls} is not an actual class object (or tuple), this function
  uses the generic algorithm described above.
  \versionadded{2.1}
  \versionchanged[Older versions of Python did not support a tuple
                  as the second argument]{2.3}
\end{cfuncdesc}


\begin{cfuncdesc}{int}{PyCallable_Check}{PyObject *o}
  Determine if the object \var{o} is callable.  Return \code{1} if the
  object is callable and \code{0} otherwise.  This function always
  succeeds.
\end{cfuncdesc}


\begin{cfuncdesc}{PyObject*}{PyObject_Call}{PyObject *callable_object,
                                            PyObject *args,
                                            PyObject *kw}
  Call a callable Python object \var{callable_object}, with arguments
  given by the tuple \var{args}, and named arguments given by the
  dictionary \var{kw}. If no named arguments are needed, \var{kw} may
  be \NULL{}. \var{args} must not be \NULL{}, use an empty tuple if
  no arguments are needed. Returns the result of the call on success,
  or \NULL{} on failure.  This is the equivalent of the Python
  expression \samp{apply(\var{callable_object}, \var{args}, \var{kw})}
  or \samp{\var{callable_object}(*\var{args}, **\var{kw})}.
  \bifuncindex{apply}
  \versionadded{2.2}
\end{cfuncdesc}


\begin{cfuncdesc}{PyObject*}{PyObject_CallObject}{PyObject *callable_object,
                                                  PyObject *args}
  Call a callable Python object \var{callable_object}, with arguments
  given by the tuple \var{args}.  If no arguments are needed, then
  \var{args} may be \NULL.  Returns the result of the call on
  success, or \NULL{} on failure.  This is the equivalent of the
  Python expression \samp{apply(\var{callable_object}, \var{args})} or
  \samp{\var{callable_object}(*\var{args})}.
  \bifuncindex{apply}
\end{cfuncdesc}

\begin{cfuncdesc}{PyObject*}{PyObject_CallFunction}{PyObject *callable,
                                                    char *format, \moreargs}
  Call a callable Python object \var{callable}, with a variable
  number of C arguments.  The C arguments are described using a
  \cfunction{Py_BuildValue()} style format string.  The format may be
  \NULL, indicating that no arguments are provided.  Returns the
  result of the call on success, or \NULL{} on failure.  This is the
  equivalent of the Python expression \samp{apply(\var{callable},
  \var{args})} or \samp{\var{callable}(*\var{args})}.
  \bifuncindex{apply}
\end{cfuncdesc}


\begin{cfuncdesc}{PyObject*}{PyObject_CallMethod}{PyObject *o,
                                                  char *method, char *format,
                                                  \moreargs}
  Call the method named \var{method} of object \var{o} with a variable
  number of C arguments.  The C arguments are described by a
  \cfunction{Py_BuildValue()} format string that should 
  produce a tuple.  The format may be \NULL,
  indicating that no arguments are provided. Returns the result of the
  call on success, or \NULL{} on failure.  This is the equivalent of
  the Python expression \samp{\var{o}.\var{method}(\var{args})}.
\end{cfuncdesc}


\begin{cfuncdesc}{PyObject*}{PyObject_CallFunctionObjArgs}{PyObject *callable,
                                                           \moreargs,
                                                           \code{NULL}}
  Call a callable Python object \var{callable}, with a variable
  number of \ctype{PyObject*} arguments.  The arguments are provided
  as a variable number of parameters followed by \NULL.
  Returns the result of the call on success, or \NULL{} on failure.
  \versionadded{2.2}
\end{cfuncdesc}


\begin{cfuncdesc}{PyObject*}{PyObject_CallMethodObjArgs}{PyObject *o,
                                                         PyObject *name,
                                                         \moreargs,
                                                         \code{NULL}}
  Calls a method of the object \var{o}, where the name of the method
  is given as a Python string object in \var{name}.  It is called with
  a variable number of \ctype{PyObject*} arguments.  The arguments are
  provided as a variable number of parameters followed by \NULL.
  Returns the result of the call on success, or \NULL{} on failure.
  \versionadded{2.2}
\end{cfuncdesc}


\begin{cfuncdesc}{int}{PyObject_Hash}{PyObject *o}
  Compute and return the hash value of an object \var{o}.  On failure,
  return \code{-1}.  This is the equivalent of the Python expression
  \samp{hash(\var{o})}.\bifuncindex{hash}
\end{cfuncdesc}


\begin{cfuncdesc}{int}{PyObject_IsTrue}{PyObject *o}
  Returns \code{1} if the object \var{o} is considered to be true, and
  \code{0} otherwise.  This is equivalent to the Python expression
  \samp{not not \var{o}}.  On failure, return \code{-1}. 
\end{cfuncdesc}


\begin{cfuncdesc}{int}{PyObject_Not}{PyObject *o}
  Returns \code{0} if the object \var{o} is considered to be true, and
  \code{1} otherwise.  This is equivalent to the Python expression
  \samp{not \var{o}}.  On failure, return \code{-1}. 
\end{cfuncdesc}


\begin{cfuncdesc}{PyObject*}{PyObject_Type}{PyObject *o}
  When \var{o} is non-\NULL, returns a type object corresponding to
  the object type of object \var{o}. On failure, raises
  \exception{SystemError} and returns \NULL.  This is equivalent to
  the Python expression \code{type(\var{o})}.\bifuncindex{type}
  This function increments the reference count of the return value.
  There's really no reason to use this function instead of the
  common expression \code{\var{o}->ob_type}, which returns a pointer
  of type \ctype{PyTypeObject*}, except when the incremented reference
  count is needed.
\end{cfuncdesc}

\begin{cfuncdesc}{int}{PyObject_TypeCheck}{PyObject *o, PyTypeObject *type}
  Return true if the object \var{o} is of type \var{type} or a subtype
  of \var{type}.  Both parameters must be non-\NULL.
  \versionadded{2.2}
\end{cfuncdesc}

\begin{cfuncdesc}{int}{PyObject_Length}{PyObject *o}
\cfuncline{int}{PyObject_Size}{PyObject *o}
  Return the length of object \var{o}.  If the object \var{o} provides
  either the sequence and mapping protocols, the sequence length is
  returned.  On error, \code{-1} is returned.  This is the equivalent
  to the Python expression \samp{len(\var{o})}.\bifuncindex{len}
\end{cfuncdesc}


\begin{cfuncdesc}{PyObject*}{PyObject_GetItem}{PyObject *o, PyObject *key}
  Return element of \var{o} corresponding to the object \var{key} or
  \NULL{} on failure.  This is the equivalent of the Python expression
  \samp{\var{o}[\var{key}]}.
\end{cfuncdesc}


\begin{cfuncdesc}{int}{PyObject_SetItem}{PyObject *o,
                                         PyObject *key, PyObject *v}
  Map the object \var{key} to the value \var{v}.  Returns \code{-1} on
  failure.  This is the equivalent of the Python statement
  \samp{\var{o}[\var{key}] = \var{v}}.
\end{cfuncdesc}


\begin{cfuncdesc}{int}{PyObject_DelItem}{PyObject *o, PyObject *key}
  Delete the mapping for \var{key} from \var{o}.  Returns \code{-1} on
  failure. This is the equivalent of the Python statement \samp{del
  \var{o}[\var{key}]}.
\end{cfuncdesc}

\begin{cfuncdesc}{int}{PyObject_AsFileDescriptor}{PyObject *o}
  Derives a file-descriptor from a Python object.  If the object is an
  integer or long integer, its value is returned.  If not, the
  object's \method{fileno()} method is called if it exists; the method
  must return an integer or long integer, which is returned as the
  file descriptor value.  Returns \code{-1} on failure.
\end{cfuncdesc}

\begin{cfuncdesc}{PyObject*}{PyObject_Dir}{PyObject *o}
  This is equivalent to the Python expression \samp{dir(\var{o})},
  returning a (possibly empty) list of strings appropriate for the
  object argument, or \NULL{} if there was an error.  If the argument
  is \NULL, this is like the Python \samp{dir()}, returning the names
  of the current locals; in this case, if no execution frame is active
  then \NULL{} is returned but \cfunction{PyErr_Occurred()} will
  return false.
\end{cfuncdesc}

\begin{cfuncdesc}{PyObject*}{PyObject_GetIter}{PyObject *o}
  This is equivalent to the Python expression \samp{iter(\var{o})}.
  It returns a new iterator for the object argument, or the object 
  itself if the object is already an iterator.  Raises
  \exception{TypeError} and returns \NULL{} if the object cannot be
  iterated.
\end{cfuncdesc}


\section{Number Protocol \label{number}}

\begin{cfuncdesc}{int}{PyNumber_Check}{PyObject *o}
  Returns \code{1} if the object \var{o} provides numeric protocols,
  and false otherwise.  This function always succeeds.
\end{cfuncdesc}


\begin{cfuncdesc}{PyObject*}{PyNumber_Add}{PyObject *o1, PyObject *o2}
  Returns the result of adding \var{o1} and \var{o2}, or \NULL{} on
  failure.  This is the equivalent of the Python expression
  \samp{\var{o1} + \var{o2}}.
\end{cfuncdesc}


\begin{cfuncdesc}{PyObject*}{PyNumber_Subtract}{PyObject *o1, PyObject *o2}
  Returns the result of subtracting \var{o2} from \var{o1}, or \NULL{}
  on failure.  This is the equivalent of the Python expression
  \samp{\var{o1} - \var{o2}}.
\end{cfuncdesc}


\begin{cfuncdesc}{PyObject*}{PyNumber_Multiply}{PyObject *o1, PyObject *o2}
  Returns the result of multiplying \var{o1} and \var{o2}, or \NULL{}
  on failure.  This is the equivalent of the Python expression
  \samp{\var{o1} * \var{o2}}.
\end{cfuncdesc}


\begin{cfuncdesc}{PyObject*}{PyNumber_Divide}{PyObject *o1, PyObject *o2}
  Returns the result of dividing \var{o1} by \var{o2}, or \NULL{} on
  failure.  This is the equivalent of the Python expression
  \samp{\var{o1} / \var{o2}}.
\end{cfuncdesc}


\begin{cfuncdesc}{PyObject*}{PyNumber_FloorDivide}{PyObject *o1, PyObject *o2}
  Return the floor of \var{o1} divided by \var{o2}, or \NULL{} on
  failure.  This is equivalent to the ``classic'' division of
  integers.
  \versionadded{2.2}
\end{cfuncdesc}


\begin{cfuncdesc}{PyObject*}{PyNumber_TrueDivide}{PyObject *o1, PyObject *o2}
  Return a reasonable approximation for the mathematical value of
  \var{o1} divided by \var{o2}, or \NULL{} on failure.  The return
  value is ``approximate'' because binary floating point numbers are
  approximate; it is not possible to represent all real numbers in
  base two.  This function can return a floating point value when
  passed two integers.
  \versionadded{2.2}
\end{cfuncdesc}


\begin{cfuncdesc}{PyObject*}{PyNumber_Remainder}{PyObject *o1, PyObject *o2}
  Returns the remainder of dividing \var{o1} by \var{o2}, or \NULL{}
  on failure.  This is the equivalent of the Python expression
  \samp{\var{o1} \%\ \var{o2}}.
\end{cfuncdesc}


\begin{cfuncdesc}{PyObject*}{PyNumber_Divmod}{PyObject *o1, PyObject *o2}
  See the built-in function \function{divmod()}\bifuncindex{divmod}.
  Returns \NULL{} on failure.  This is the equivalent of the Python
  expression \samp{divmod(\var{o1}, \var{o2})}.
\end{cfuncdesc}


\begin{cfuncdesc}{PyObject*}{PyNumber_Power}{PyObject *o1,
                                             PyObject *o2, PyObject *o3}
  See the built-in function \function{pow()}\bifuncindex{pow}.
  Returns \NULL{} on failure.  This is the equivalent of the Python
  expression \samp{pow(\var{o1}, \var{o2}, \var{o3})}, where \var{o3}
  is optional.  If \var{o3} is to be ignored, pass \cdata{Py_None} in
  its place (passing \NULL{} for \var{o3} would cause an illegal
  memory access).
\end{cfuncdesc}


\begin{cfuncdesc}{PyObject*}{PyNumber_Negative}{PyObject *o}
  Returns the negation of \var{o} on success, or \NULL{} on failure.
  This is the equivalent of the Python expression \samp{-\var{o}}.
\end{cfuncdesc}


\begin{cfuncdesc}{PyObject*}{PyNumber_Positive}{PyObject *o}
  Returns \var{o} on success, or \NULL{} on failure.  This is the
  equivalent of the Python expression \samp{+\var{o}}.
\end{cfuncdesc}


\begin{cfuncdesc}{PyObject*}{PyNumber_Absolute}{PyObject *o}
  Returns the absolute value of \var{o}, or \NULL{} on failure.  This
  is the equivalent of the Python expression \samp{abs(\var{o})}.
  \bifuncindex{abs}
\end{cfuncdesc}


\begin{cfuncdesc}{PyObject*}{PyNumber_Invert}{PyObject *o}
  Returns the bitwise negation of \var{o} on success, or \NULL{} on
  failure.  This is the equivalent of the Python expression
  \samp{\~\var{o}}.
\end{cfuncdesc}


\begin{cfuncdesc}{PyObject*}{PyNumber_Lshift}{PyObject *o1, PyObject *o2}
  Returns the result of left shifting \var{o1} by \var{o2} on success,
  or \NULL{} on failure.  This is the equivalent of the Python
  expression \samp{\var{o1} <\code{<} \var{o2}}.
\end{cfuncdesc}


\begin{cfuncdesc}{PyObject*}{PyNumber_Rshift}{PyObject *o1, PyObject *o2}
  Returns the result of right shifting \var{o1} by \var{o2} on
  success, or \NULL{} on failure.  This is the equivalent of the
  Python expression \samp{\var{o1} >\code{>} \var{o2}}.
\end{cfuncdesc}


\begin{cfuncdesc}{PyObject*}{PyNumber_And}{PyObject *o1, PyObject *o2}
  Returns the ``bitwise and'' of \var{o1} and \var{o2} on success and
  \NULL{} on failure. This is the equivalent of the Python expression
  \samp{\var{o1} \&\ \var{o2}}.
\end{cfuncdesc}


\begin{cfuncdesc}{PyObject*}{PyNumber_Xor}{PyObject *o1, PyObject *o2}
  Returns the ``bitwise exclusive or'' of \var{o1} by \var{o2} on
  success, or \NULL{} on failure.  This is the equivalent of the
  Python expression \samp{\var{o1} \textasciicircum{} \var{o2}}.
\end{cfuncdesc}

\begin{cfuncdesc}{PyObject*}{PyNumber_Or}{PyObject *o1, PyObject *o2}
  Returns the ``bitwise or'' of \var{o1} and \var{o2} on success, or
  \NULL{} on failure.  This is the equivalent of the Python expression
  \samp{\var{o1} | \var{o2}}.
\end{cfuncdesc}


\begin{cfuncdesc}{PyObject*}{PyNumber_InPlaceAdd}{PyObject *o1, PyObject *o2}
  Returns the result of adding \var{o1} and \var{o2}, or \NULL{} on
  failure.  The operation is done \emph{in-place} when \var{o1}
  supports it.  This is the equivalent of the Python statement
  \samp{\var{o1} += \var{o2}}.
\end{cfuncdesc}


\begin{cfuncdesc}{PyObject*}{PyNumber_InPlaceSubtract}{PyObject *o1,
                                                       PyObject *o2}
  Returns the result of subtracting \var{o2} from \var{o1}, or \NULL{}
  on failure.  The operation is done \emph{in-place} when \var{o1}
  supports it.  This is the equivalent of the Python statement
  \samp{\var{o1} -= \var{o2}}.
\end{cfuncdesc}


\begin{cfuncdesc}{PyObject*}{PyNumber_InPlaceMultiply}{PyObject *o1,
                                                       PyObject *o2}
  Returns the result of multiplying \var{o1} and \var{o2}, or \NULL{}
  on failure.  The operation is done \emph{in-place} when \var{o1}
  supports it.  This is the equivalent of the Python statement
  \samp{\var{o1} *= \var{o2}}.
\end{cfuncdesc}


\begin{cfuncdesc}{PyObject*}{PyNumber_InPlaceDivide}{PyObject *o1,
                                                     PyObject *o2}
  Returns the result of dividing \var{o1} by \var{o2}, or \NULL{} on
  failure.  The operation is done \emph{in-place} when \var{o1}
  supports it. This is the equivalent of the Python statement
  \samp{\var{o1} /= \var{o2}}.
\end{cfuncdesc}


\begin{cfuncdesc}{PyObject*}{PyNumber_InPlaceFloorDivide}{PyObject *o1,
                                                          PyObject *o2}
  Returns the mathematical floor of dividing \var{o1} by \var{o2}, or
  \NULL{} on failure.  The operation is done \emph{in-place} when
  \var{o1} supports it.  This is the equivalent of the Python
  statement \samp{\var{o1} //= \var{o2}}.
  \versionadded{2.2}
\end{cfuncdesc}


\begin{cfuncdesc}{PyObject*}{PyNumber_InPlaceTrueDivide}{PyObject *o1,
                                                         PyObject *o2}
  Return a reasonable approximation for the mathematical value of
  \var{o1} divided by \var{o2}, or \NULL{} on failure.  The return
  value is ``approximate'' because binary floating point numbers are
  approximate; it is not possible to represent all real numbers in
  base two.  This function can return a floating point value when
  passed two integers.  The operation is done \emph{in-place} when
  \var{o1} supports it.
  \versionadded{2.2}
\end{cfuncdesc}


\begin{cfuncdesc}{PyObject*}{PyNumber_InPlaceRemainder}{PyObject *o1,
                                                        PyObject *o2}
  Returns the remainder of dividing \var{o1} by \var{o2}, or \NULL{}
  on failure.  The operation is done \emph{in-place} when \var{o1}
  supports it.  This is the equivalent of the Python statement
  \samp{\var{o1} \%= \var{o2}}.
\end{cfuncdesc}


\begin{cfuncdesc}{PyObject*}{PyNumber_InPlacePower}{PyObject *o1,
                                                    PyObject *o2, PyObject *o3}
  See the built-in function \function{pow()}.\bifuncindex{pow}
  Returns \NULL{} on failure.  The operation is done \emph{in-place}
  when \var{o1} supports it.  This is the equivalent of the Python
  statement \samp{\var{o1} **= \var{o2}} when o3 is \cdata{Py_None},
  or an in-place variant of \samp{pow(\var{o1}, \var{o2}, \var{o3})}
  otherwise. If \var{o3} is to be ignored, pass \cdata{Py_None} in its
  place (passing \NULL{} for \var{o3} would cause an illegal memory
  access).
\end{cfuncdesc}

\begin{cfuncdesc}{PyObject*}{PyNumber_InPlaceLshift}{PyObject *o1,
                                                     PyObject *o2}
  Returns the result of left shifting \var{o1} by \var{o2} on success,
  or \NULL{} on failure.  The operation is done \emph{in-place} when
  \var{o1} supports it.  This is the equivalent of the Python
  statement \samp{\var{o1} <\code{<=} \var{o2}}.
\end{cfuncdesc}


\begin{cfuncdesc}{PyObject*}{PyNumber_InPlaceRshift}{PyObject *o1,
                                                     PyObject *o2}
  Returns the result of right shifting \var{o1} by \var{o2} on
  success, or \NULL{} on failure.  The operation is done
  \emph{in-place} when \var{o1} supports it.  This is the equivalent
  of the Python statement \samp{\var{o1} >\code{>=} \var{o2}}.
\end{cfuncdesc}


\begin{cfuncdesc}{PyObject*}{PyNumber_InPlaceAnd}{PyObject *o1, PyObject *o2}
  Returns the ``bitwise and'' of \var{o1} and \var{o2} on success and
  \NULL{} on failure. The operation is done \emph{in-place} when
  \var{o1} supports it.  This is the equivalent of the Python
  statement \samp{\var{o1} \&= \var{o2}}.
\end{cfuncdesc}


\begin{cfuncdesc}{PyObject*}{PyNumber_InPlaceXor}{PyObject *o1, PyObject *o2}
  Returns the ``bitwise exclusive or'' of \var{o1} by \var{o2} on
  success, or \NULL{} on failure.  The operation is done
  \emph{in-place} when \var{o1} supports it.  This is the equivalent
  of the Python statement \samp{\var{o1} \textasciicircum= \var{o2}}.
\end{cfuncdesc}

\begin{cfuncdesc}{PyObject*}{PyNumber_InPlaceOr}{PyObject *o1, PyObject *o2}
  Returns the ``bitwise or'' of \var{o1} and \var{o2} on success, or
  \NULL{} on failure.  The operation is done \emph{in-place} when
  \var{o1} supports it.  This is the equivalent of the Python
  statement \samp{\var{o1} |= \var{o2}}.
\end{cfuncdesc}

\begin{cfuncdesc}{int}{PyNumber_Coerce}{PyObject **p1, PyObject **p2}
  This function takes the addresses of two variables of type
  \ctype{PyObject*}.  If the objects pointed to by \code{*\var{p1}}
  and \code{*\var{p2}} have the same type, increment their reference
  count and return \code{0} (success). If the objects can be converted
  to a common numeric type, replace \code{*p1} and \code{*p2} by their
  converted value (with 'new' reference counts), and return \code{0}.
  If no conversion is possible, or if some other error occurs, return
  \code{-1} (failure) and don't increment the reference counts.  The
  call \code{PyNumber_Coerce(\&o1, \&o2)} is equivalent to the Python
  statement \samp{\var{o1}, \var{o2} = coerce(\var{o1}, \var{o2})}.
  \bifuncindex{coerce}
\end{cfuncdesc}

\begin{cfuncdesc}{PyObject*}{PyNumber_Int}{PyObject *o}
  Returns the \var{o} converted to an integer object on success, or
  \NULL{} on failure.  If the argument is outside the integer range
  a long object will be returned instead. This is the equivalent
  of the Python expression \samp{int(\var{o})}.\bifuncindex{int}
\end{cfuncdesc}

\begin{cfuncdesc}{PyObject*}{PyNumber_Long}{PyObject *o}
  Returns the \var{o} converted to a long integer object on success,
  or \NULL{} on failure.  This is the equivalent of the Python
  expression \samp{long(\var{o})}.\bifuncindex{long}
\end{cfuncdesc}

\begin{cfuncdesc}{PyObject*}{PyNumber_Float}{PyObject *o}
  Returns the \var{o} converted to a float object on success, or
  \NULL{} on failure.  This is the equivalent of the Python expression
  \samp{float(\var{o})}.\bifuncindex{float}
\end{cfuncdesc}


\section{Sequence Protocol \label{sequence}}

\begin{cfuncdesc}{int}{PySequence_Check}{PyObject *o}
  Return \code{1} if the object provides sequence protocol, and
  \code{0} otherwise.  This function always succeeds.
\end{cfuncdesc}

\begin{cfuncdesc}{int}{PySequence_Size}{PyObject *o}
  Returns the number of objects in sequence \var{o} on success, and
  \code{-1} on failure.  For objects that do not provide sequence
  protocol, this is equivalent to the Python expression
  \samp{len(\var{o})}.\bifuncindex{len}
\end{cfuncdesc}

\begin{cfuncdesc}{int}{PySequence_Length}{PyObject *o}
  Alternate name for \cfunction{PySequence_Size()}.
\end{cfuncdesc}

\begin{cfuncdesc}{PyObject*}{PySequence_Concat}{PyObject *o1, PyObject *o2}
  Return the concatenation of \var{o1} and \var{o2} on success, and
  \NULL{} on failure.   This is the equivalent of the Python
  expression \samp{\var{o1} + \var{o2}}.
\end{cfuncdesc}


\begin{cfuncdesc}{PyObject*}{PySequence_Repeat}{PyObject *o, int count}
  Return the result of repeating sequence object \var{o} \var{count}
  times, or \NULL{} on failure.  This is the equivalent of the Python
  expression \samp{\var{o} * \var{count}}.
\end{cfuncdesc}

\begin{cfuncdesc}{PyObject*}{PySequence_InPlaceConcat}{PyObject *o1,
                                                       PyObject *o2}
  Return the concatenation of \var{o1} and \var{o2} on success, and
  \NULL{} on failure.  The operation is done \emph{in-place} when
  \var{o1} supports it.  This is the equivalent of the Python
  expression \samp{\var{o1} += \var{o2}}.
\end{cfuncdesc}


\begin{cfuncdesc}{PyObject*}{PySequence_InPlaceRepeat}{PyObject *o, int count}
  Return the result of repeating sequence object \var{o} \var{count}
  times, or \NULL{} on failure.  The operation is done \emph{in-place}
  when \var{o} supports it.  This is the equivalent of the Python
  expression \samp{\var{o} *= \var{count}}.
\end{cfuncdesc}


\begin{cfuncdesc}{PyObject*}{PySequence_GetItem}{PyObject *o, int i}
  Return the \var{i}th element of \var{o}, or \NULL{} on failure.
  This is the equivalent of the Python expression
  \samp{\var{o}[\var{i}]}.
\end{cfuncdesc}


\begin{cfuncdesc}{PyObject*}{PySequence_GetSlice}{PyObject *o, int i1, int i2}
  Return the slice of sequence object \var{o} between \var{i1} and
  \var{i2}, or \NULL{} on failure. This is the equivalent of the
  Python expression \samp{\var{o}[\var{i1}:\var{i2}]}.
\end{cfuncdesc}


\begin{cfuncdesc}{int}{PySequence_SetItem}{PyObject *o, int i, PyObject *v}
  Assign object \var{v} to the \var{i}th element of \var{o}.  Returns
  \code{-1} on failure.  This is the equivalent of the Python
  statement \samp{\var{o}[\var{i}] = \var{v}}.  This function \emph{does not}
  steal a reference to \var{v}.
\end{cfuncdesc}

\begin{cfuncdesc}{int}{PySequence_DelItem}{PyObject *o, int i}
  Delete the \var{i}th element of object \var{o}.  Returns \code{-1}
  on failure.  This is the equivalent of the Python statement
  \samp{del \var{o}[\var{i}]}.
\end{cfuncdesc}

\begin{cfuncdesc}{int}{PySequence_SetSlice}{PyObject *o, int i1,
                                            int i2, PyObject *v}
  Assign the sequence object \var{v} to the slice in sequence object
  \var{o} from \var{i1} to \var{i2}.  This is the equivalent of the
  Python statement \samp{\var{o}[\var{i1}:\var{i2}] = \var{v}}.
\end{cfuncdesc}

\begin{cfuncdesc}{int}{PySequence_DelSlice}{PyObject *o, int i1, int i2}
  Delete the slice in sequence object \var{o} from \var{i1} to
  \var{i2}.  Returns \code{-1} on failure.  This is the equivalent of
  the Python statement \samp{del \var{o}[\var{i1}:\var{i2}]}.
\end{cfuncdesc}

\begin{cfuncdesc}{PyObject*}{PySequence_Tuple}{PyObject *o}
  Returns the \var{o} as a tuple on success, and \NULL{} on failure.
  This is equivalent to the Python expression \samp{tuple(\var{o})}.
  \bifuncindex{tuple}
\end{cfuncdesc}

\begin{cfuncdesc}{int}{PySequence_Count}{PyObject *o, PyObject *value}
  Return the number of occurrences of \var{value} in \var{o}, that is,
  return the number of keys for which \code{\var{o}[\var{key}] ==
  \var{value}}.  On failure, return \code{-1}.  This is equivalent to
  the Python expression \samp{\var{o}.count(\var{value})}.
\end{cfuncdesc}

\begin{cfuncdesc}{int}{PySequence_Contains}{PyObject *o, PyObject *value}
  Determine if \var{o} contains \var{value}.  If an item in \var{o} is
  equal to \var{value}, return \code{1}, otherwise return \code{0}.
  On error, return \code{-1}.  This is equivalent to the Python
  expression \samp{\var{value} in \var{o}}.
\end{cfuncdesc}

\begin{cfuncdesc}{int}{PySequence_Index}{PyObject *o, PyObject *value}
  Return the first index \var{i} for which \code{\var{o}[\var{i}] ==
  \var{value}}.  On error, return \code{-1}.    This is equivalent to
  the Python expression \samp{\var{o}.index(\var{value})}.
\end{cfuncdesc}

\begin{cfuncdesc}{PyObject*}{PySequence_List}{PyObject *o}
  Return a list object with the same contents as the arbitrary
  sequence \var{o}.  The returned list is guaranteed to be new.
\end{cfuncdesc}

\begin{cfuncdesc}{PyObject*}{PySequence_Tuple}{PyObject *o}
  Return a tuple object with the same contents as the arbitrary
  sequence \var{o}.  If \var{o} is a tuple, a new reference will be
  returned, otherwise a tuple will be constructed with the appropriate
  contents.
\end{cfuncdesc}

\begin{cfuncdesc}{PyObject*}{PySequence_Fast}{PyObject *o, const char *m}
  Returns the sequence \var{o} as a tuple, unless it is already a
  tuple or list, in which case \var{o} is returned.  Use
  \cfunction{PySequence_Fast_GET_ITEM()} to access the members of the
  result.  Returns \NULL{} on failure.  If the object is not a
  sequence, raises \exception{TypeError} with \var{m} as the message
  text.
\end{cfuncdesc}

\begin{cfuncdesc}{PyObject*}{PySequence_Fast_GET_ITEM}{PyObject *o, int i}
  Return the \var{i}th element of \var{o}, assuming that \var{o} was
  returned by \cfunction{PySequence_Fast()}, \var{o} is not \NULL,
  and that \var{i} is within bounds.
\end{cfuncdesc}

\begin{cfuncdesc}{PyObject**}{PySequence_Fast_ITEMS}{PyObject *o}
  Return the underlying array of PyObject pointers.  Assumes that
  \var{o} was returned by \cfunction{PySequence_Fast()} and
  \var{o} is not \NULL.
  \versionadded{2.4}  
\end{cfuncdesc}

\begin{cfuncdesc}{PyObject*}{PySequence_ITEM}{PyObject *o, int i}
  Return the \var{i}th element of \var{o} or \NULL{} on failure.
  Macro form of \cfunction{PySequence_GetItem()} but without checking
  that \cfunction{PySequence_Check(\var{o})} is true and without
  adjustment for negative indices.
  \versionadded{2.3}
\end{cfuncdesc}

\begin{cfuncdesc}{int}{PySequence_Fast_GET_SIZE}{PyObject *o}
  Returns the length of \var{o}, assuming that \var{o} was
  returned by \cfunction{PySequence_Fast()} and that \var{o} is
  not \NULL.  The size can also be gotten by calling
  \cfunction{PySequence_Size()} on \var{o}, but
  \cfunction{PySequence_Fast_GET_SIZE()} is faster because it can
  assume \var{o} is a list or tuple.
\end{cfuncdesc}


\section{Mapping Protocol \label{mapping}}

\begin{cfuncdesc}{int}{PyMapping_Check}{PyObject *o}
  Return \code{1} if the object provides mapping protocol, and
  \code{0} otherwise.  This function always succeeds.
\end{cfuncdesc}


\begin{cfuncdesc}{int}{PyMapping_Length}{PyObject *o}
  Returns the number of keys in object \var{o} on success, and
  \code{-1} on failure.  For objects that do not provide mapping
  protocol, this is equivalent to the Python expression
  \samp{len(\var{o})}.\bifuncindex{len}
\end{cfuncdesc}


\begin{cfuncdesc}{int}{PyMapping_DelItemString}{PyObject *o, char *key}
  Remove the mapping for object \var{key} from the object \var{o}.
  Return \code{-1} on failure.  This is equivalent to the Python
  statement \samp{del \var{o}[\var{key}]}.
\end{cfuncdesc}


\begin{cfuncdesc}{int}{PyMapping_DelItem}{PyObject *o, PyObject *key}
  Remove the mapping for object \var{key} from the object \var{o}.
  Return \code{-1} on failure.  This is equivalent to the Python
  statement \samp{del \var{o}[\var{key}]}.
\end{cfuncdesc}


\begin{cfuncdesc}{int}{PyMapping_HasKeyString}{PyObject *o, char *key}
  On success, return \code{1} if the mapping object has the key
  \var{key} and \code{0} otherwise.  This is equivalent to the Python
  expression \samp{\var{o}.has_key(\var{key})}.  This function always
  succeeds.
\end{cfuncdesc}


\begin{cfuncdesc}{int}{PyMapping_HasKey}{PyObject *o, PyObject *key}
  Return \code{1} if the mapping object has the key \var{key} and
  \code{0} otherwise.  This is equivalent to the Python expression
  \samp{\var{o}.has_key(\var{key})}.  This function always succeeds.
\end{cfuncdesc}


\begin{cfuncdesc}{PyObject*}{PyMapping_Keys}{PyObject *o}
  On success, return a list of the keys in object \var{o}.  On
  failure, return \NULL. This is equivalent to the Python expression
  \samp{\var{o}.keys()}.
\end{cfuncdesc}


\begin{cfuncdesc}{PyObject*}{PyMapping_Values}{PyObject *o}
  On success, return a list of the values in object \var{o}.  On
  failure, return \NULL. This is equivalent to the Python expression
  \samp{\var{o}.values()}.
\end{cfuncdesc}


\begin{cfuncdesc}{PyObject*}{PyMapping_Items}{PyObject *o}
  On success, return a list of the items in object \var{o}, where each
  item is a tuple containing a key-value pair.  On failure, return
  \NULL. This is equivalent to the Python expression
  \samp{\var{o}.items()}.
\end{cfuncdesc}


\begin{cfuncdesc}{PyObject*}{PyMapping_GetItemString}{PyObject *o, char *key}
  Return element of \var{o} corresponding to the object \var{key} or
  \NULL{} on failure. This is the equivalent of the Python expression
  \samp{\var{o}[\var{key}]}.
\end{cfuncdesc}

\begin{cfuncdesc}{int}{PyMapping_SetItemString}{PyObject *o, char *key,
                                                PyObject *v}
  Map the object \var{key} to the value \var{v} in object \var{o}.
  Returns \code{-1} on failure.  This is the equivalent of the Python
  statement \samp{\var{o}[\var{key}] = \var{v}}.
\end{cfuncdesc}


\section{Iterator Protocol \label{iterator}}

\versionadded{2.2}

There are only a couple of functions specifically for working with
iterators.

\begin{cfuncdesc}{int}{PyIter_Check}{PyObject *o}
  Return true if the object \var{o} supports the iterator protocol.
\end{cfuncdesc}

\begin{cfuncdesc}{PyObject*}{PyIter_Next}{PyObject *o}
  Return the next value from the iteration \var{o}.  If the object is
  an iterator, this retrieves the next value from the iteration, and
  returns \NULL{} with no exception set if there are no remaining
  items.  If the object is not an iterator, \exception{TypeError} is
  raised, or if there is an error in retrieving the item, returns
  \NULL{} and passes along the exception.
\end{cfuncdesc}

To write a loop which iterates over an iterator, the C code should
look something like this:

\begin{verbatim}
PyObject *iterator = PyObject_GetIter(obj);
PyObject *item;

if (iterator == NULL) {
    /* propagate error */
}

while (item = PyIter_Next(iterator)) {
    /* do something with item */
    ...
    /* release reference when done */
    Py_DECREF(item);
}

Py_DECREF(iterator);

if (PyErr_Occurred()) {
    /* propagate error */
}
else {
    /* continue doing useful work */
}
\end{verbatim}


\section{Buffer Protocol \label{abstract-buffer}}

\begin{cfuncdesc}{int}{PyObject_AsCharBuffer}{PyObject *obj,
                                              const char **buffer,
                                              int *buffer_len}
  Returns a pointer to a read-only memory location useable as character-
  based input.  The \var{obj} argument must support the single-segment
  character buffer interface.  On success, returns \code{0}, sets
  \var{buffer} to the memory location and \var{buffer_len} to the buffer
  length.  Returns \code{-1} and sets a \exception{TypeError} on error.
  \versionadded{1.6}
\end{cfuncdesc}

\begin{cfuncdesc}{int}{PyObject_AsReadBuffer}{PyObject *obj,
                                              const void **buffer,
                                              int *buffer_len}
  Returns a pointer to a read-only memory location containing
  arbitrary data.  The \var{obj} argument must support the
  single-segment readable buffer interface.  On success, returns
  \code{0}, sets \var{buffer} to the memory location and \var{buffer_len}
  to the buffer length.  Returns \code{-1} and sets a
  \exception{TypeError} on error.
  \versionadded{1.6}
\end{cfuncdesc}

\begin{cfuncdesc}{int}{PyObject_CheckReadBuffer}{PyObject *o}
  Returns \code{1} if \var{o} supports the single-segment readable
  buffer interface.  Otherwise returns \code{0}.
  \versionadded{2.2}
\end{cfuncdesc}

\begin{cfuncdesc}{int}{PyObject_AsWriteBuffer}{PyObject *obj,
                                               void **buffer,
                                               int *buffer_len}
  Returns a pointer to a writeable memory location.  The \var{obj}
  argument must support the single-segment, character buffer
  interface.  On success, returns \code{0}, sets \var{buffer} to the
  memory location and \var{buffer_len} to the buffer length.  Returns
  \code{-1} and sets a \exception{TypeError} on error.
  \versionadded{1.6}
\end{cfuncdesc}

\chapter{Concrete Objects Layer \label{concrete}}


The functions in this chapter are specific to certain Python object
types.  Passing them an object of the wrong type is not a good idea;
if you receive an object from a Python program and you are not sure
that it has the right type, you must perform a type check first;
for example, to check that an object is a dictionary, use
\cfunction{PyDict_Check()}.  The chapter is structured like the
``family tree'' of Python object types.

\warning{While the functions described in this chapter carefully check
the type of the objects which are passed in, many of them do not check
for \NULL{} being passed instead of a valid object.  Allowing \NULL{}
to be passed in can cause memory access violations and immediate
termination of the interpreter.}


\section{Fundamental Objects \label{fundamental}}

This section describes Python type objects and the singleton object
\code{None}.


\subsection{Type Objects \label{typeObjects}}

\obindex{type}
\begin{ctypedesc}{PyTypeObject}
  The C structure of the objects used to describe built-in types.
\end{ctypedesc}

\begin{cvardesc}{PyObject*}{PyType_Type}
  This is the type object for type objects; it is the same object as
  \code{type} and \code{types.TypeType} in the Python layer.
  \withsubitem{(in module types)}{\ttindex{TypeType}}
\end{cvardesc}

\begin{cfuncdesc}{int}{PyType_Check}{PyObject *o}
  Return true if the object \var{o} is a type object, including
  instances of types derived from the standard type object.  Return
  false in all other cases.
\end{cfuncdesc}

\begin{cfuncdesc}{int}{PyType_CheckExact}{PyObject *o}
  Return true if the object \var{o} is a type object, but not a
  subtype of the standard type object.  Return false in all other
  cases.
  \versionadded{2.2}
\end{cfuncdesc}

\begin{cfuncdesc}{int}{PyType_HasFeature}{PyObject *o, int feature}
  Return true if the type object \var{o} sets the feature
  \var{feature}.  Type features are denoted by single bit flags.
\end{cfuncdesc}

\begin{cfuncdesc}{int}{PyType_IS_GC}{PyObject *o}
  Return true if the type object includes support for the cycle
  detector; this tests the type flag \constant{Py_TPFLAGS_HAVE_GC}.
  \versionadded{2.0}
\end{cfuncdesc}

\begin{cfuncdesc}{int}{PyType_IsSubtype}{PyTypeObject *a, PyTypeObject *b}
  Return true if \var{a} is a subtype of \var{b}.
  \versionadded{2.2}
\end{cfuncdesc}

\begin{cfuncdesc}{PyObject*}{PyType_GenericAlloc}{PyTypeObject *type,
                                                  Py_ssize_t nitems}
  \versionadded{2.2}
\end{cfuncdesc}

\begin{cfuncdesc}{PyObject*}{PyType_GenericNew}{PyTypeObject *type,
                                            PyObject *args, PyObject *kwds}
  \versionadded{2.2}
\end{cfuncdesc}

\begin{cfuncdesc}{int}{PyType_Ready}{PyTypeObject *type}
  Finalize a type object.  This should be called on all type objects
  to finish their initialization.  This function is responsible for
  adding inherited slots from a type's base class.  Return \code{0}
  on success, or return \code{-1} and sets an exception on error.
  \versionadded{2.2}
\end{cfuncdesc}


\subsection{The None Object \label{noneObject}}

\obindex{None}
Note that the \ctype{PyTypeObject} for \code{None} is not directly
exposed in the Python/C API.  Since \code{None} is a singleton,
testing for object identity (using \samp{==} in C) is sufficient.
There is no \cfunction{PyNone_Check()} function for the same reason.

\begin{cvardesc}{PyObject*}{Py_None}
  The Python \code{None} object, denoting lack of value.  This object
  has no methods.  It needs to be treated just like any other object
  with respect to reference counts.
\end{cvardesc}

\begin{csimplemacrodesc}{Py_RETURN_NONE}
  Properly handle returning \cdata{Py_None} from within a C function.
  \versionadded{2.4}
\end{csimplemacrodesc}


\section{Numeric Objects \label{numericObjects}}

\obindex{numeric}


\subsection{Plain Integer Objects \label{intObjects}}

\obindex{integer}
\begin{ctypedesc}{PyIntObject}
  This subtype of \ctype{PyObject} represents a Python integer
  object.
\end{ctypedesc}

\begin{cvardesc}{PyTypeObject}{PyInt_Type}
  This instance of \ctype{PyTypeObject} represents the Python plain
  integer type.  This is the same object as \code{int} and
  \code{types.IntType}.
  \withsubitem{(in modules types)}{\ttindex{IntType}}
\end{cvardesc}

\begin{cfuncdesc}{int}{PyInt_Check}{PyObject *o}
  Return true if \var{o} is of type \cdata{PyInt_Type} or a subtype
  of \cdata{PyInt_Type}.
  \versionchanged[Allowed subtypes to be accepted]{2.2}
\end{cfuncdesc}

\begin{cfuncdesc}{int}{PyInt_CheckExact}{PyObject *o}
  Return true if \var{o} is of type \cdata{PyInt_Type}, but not a
  subtype of \cdata{PyInt_Type}.
  \versionadded{2.2}
\end{cfuncdesc}

\begin{cfuncdesc}{PyObject*}{PyInt_FromString}{char *str, char **pend,
                                               int base}
  Return a new \ctype{PyIntObject} or \ctype{PyLongObject} based on the
  string value in \var{str}, which is interpreted according to the radix in
  \var{base}.  If \var{pend} is non-\NULL{}, \code{*\var{pend}} will point to
  the first character in \var{str} which follows the representation of the
  number.  If \var{base} is \code{0}, the radix will be determined based on
  the leading characters of \var{str}: if \var{str} starts with \code{'0x'}
  or \code{'0X'}, radix 16 will be used; if \var{str} starts with
  \code{'0'}, radix 8 will be used; otherwise radix 10 will be used.  If
  \var{base} is not \code{0}, it must be between \code{2} and \code{36},
  inclusive.  Leading spaces are ignored.  If there are no digits,
  \exception{ValueError} will be raised.  If the string represents a number
  too large to be contained within the machine's \ctype{long int} type and
  overflow warnings are being suppressed, a \ctype{PyLongObject} will be
  returned.  If overflow warnings are not being suppressed, \NULL{} will be
  returned in this case.
\end{cfuncdesc}

\begin{cfuncdesc}{PyObject*}{PyInt_FromLong}{long ival}
  Create a new integer object with a value of \var{ival}.

  The current implementation keeps an array of integer objects for all
  integers between \code{-5} and \code{256}, when you create an int in
  that range you actually just get back a reference to the existing
  object. So it should be possible to change the value of \code{1}.  I
  suspect the behaviour of Python in this case is undefined. :-)
\end{cfuncdesc}

\begin{cfuncdesc}{PyObject*}{PyInt_FromSsize_t}{Py_ssize_t ival}
  Create a new integer object with a value of \var{ival}.
  If the value exceeds \code{LONG_MAX}, a long integer object is
  returned.

 \versionadded{2.5}
\end{cfuncdesc}

\begin{cfuncdesc}{long}{PyInt_AsLong}{PyObject *io}
  Will first attempt to cast the object to a \ctype{PyIntObject}, if
  it is not already one, and then return its value. If there is an
  error, \code{-1} is returned, and the caller should check
  \code{PyErr_Occurred()} to find out whether there was an error, or
  whether the value just happened to be -1.
\end{cfuncdesc}

\begin{cfuncdesc}{long}{PyInt_AS_LONG}{PyObject *io}
  Return the value of the object \var{io}.  No error checking is
  performed.
\end{cfuncdesc}

\begin{cfuncdesc}{unsigned long}{PyInt_AsUnsignedLongMask}{PyObject *io}
  Will first attempt to cast the object to a \ctype{PyIntObject} or
  \ctype{PyLongObject}, if it is not already one, and then return its
  value as unsigned long.  This function does not check for overflow.
  \versionadded{2.3}
\end{cfuncdesc}

\begin{cfuncdesc}{unsigned PY_LONG_LONG}{PyInt_AsUnsignedLongLongMask}{PyObject *io}
  Will first attempt to cast the object to a \ctype{PyIntObject} or
  \ctype{PyLongObject}, if it is not already one, and then return its
  value as unsigned long long, without checking for overflow.
  \versionadded{2.3}
\end{cfuncdesc}

\begin{cfuncdesc}{Py_ssize_t}{PyInt_AsSsize_t}{PyObject *io}
  Will first attempt to cast the object to a \ctype{PyIntObject} or
  \ctype{PyLongObject}, if it is not already one, and then return its
  value as \ctype{Py_ssize_t}.
  \versionadded{2.5}
\end{cfuncdesc}

\begin{cfuncdesc}{long}{PyInt_GetMax}{}
  Return the system's idea of the largest integer it can handle
  (\constant{LONG_MAX}\ttindex{LONG_MAX}, as defined in the system
  header files).
\end{cfuncdesc}

\subsection{Boolean Objects \label{boolObjects}}

Booleans in Python are implemented as a subclass of integers.  There
are only two booleans, \constant{Py_False} and \constant{Py_True}.  As
such, the normal creation and deletion functions don't apply to
booleans.  The following macros are available, however.

\begin{cfuncdesc}{int}{PyBool_Check}{PyObject *o}
  Return true if \var{o} is of type \cdata{PyBool_Type}.
  \versionadded{2.3}
\end{cfuncdesc}

\begin{cvardesc}{PyObject*}{Py_False}
  The Python \code{False} object.  This object has no methods.  It needs to
  be treated just like any other object with respect to reference counts.
\end{cvardesc}

\begin{cvardesc}{PyObject*}{Py_True}
  The Python \code{True} object.  This object has no methods.  It needs to
  be treated just like any other object with respect to reference counts.
\end{cvardesc}

\begin{csimplemacrodesc}{Py_RETURN_FALSE}
  Return \constant{Py_False} from a function, properly incrementing its
  reference count.
\versionadded{2.4}
\end{csimplemacrodesc}

\begin{csimplemacrodesc}{Py_RETURN_TRUE}
  Return \constant{Py_True} from a function, properly incrementing its
  reference count.
\versionadded{2.4}
\end{csimplemacrodesc}

\begin{cfuncdesc}{PyObject*}{PyBool_FromLong}{long v}
  Return a new reference to \constant{Py_True} or \constant{Py_False}
  depending on the truth value of \var{v}.
\versionadded{2.3}
\end{cfuncdesc}

\subsection{Long Integer Objects \label{longObjects}}

\obindex{long integer}
\begin{ctypedesc}{PyLongObject}
  This subtype of \ctype{PyObject} represents a Python long integer
  object.
\end{ctypedesc}

\begin{cvardesc}{PyTypeObject}{PyLong_Type}
  This instance of \ctype{PyTypeObject} represents the Python long
  integer type.  This is the same object as \code{long} and
  \code{types.LongType}.
  \withsubitem{(in modules types)}{\ttindex{LongType}}
\end{cvardesc}

\begin{cfuncdesc}{int}{PyLong_Check}{PyObject *p}
  Return true if its argument is a \ctype{PyLongObject} or a subtype
  of \ctype{PyLongObject}.
  \versionchanged[Allowed subtypes to be accepted]{2.2}
\end{cfuncdesc}

\begin{cfuncdesc}{int}{PyLong_CheckExact}{PyObject *p}
  Return true if its argument is a \ctype{PyLongObject}, but not a
  subtype of \ctype{PyLongObject}.
  \versionadded{2.2}
\end{cfuncdesc}

\begin{cfuncdesc}{PyObject*}{PyLong_FromLong}{long v}
  Return a new \ctype{PyLongObject} object from \var{v}, or \NULL{}
  on failure.
\end{cfuncdesc}

\begin{cfuncdesc}{PyObject*}{PyLong_FromUnsignedLong}{unsigned long v}
  Return a new \ctype{PyLongObject} object from a C \ctype{unsigned
  long}, or \NULL{} on failure.
\end{cfuncdesc}

\begin{cfuncdesc}{PyObject*}{PyLong_FromLongLong}{PY_LONG_LONG v}
  Return a new \ctype{PyLongObject} object from a C \ctype{long long},
  or \NULL{} on failure.
\end{cfuncdesc}

\begin{cfuncdesc}{PyObject*}{PyLong_FromUnsignedLongLong}{unsigned PY_LONG_LONG v}
  Return a new \ctype{PyLongObject} object from a C \ctype{unsigned
  long long}, or \NULL{} on failure.
\end{cfuncdesc}

\begin{cfuncdesc}{PyObject*}{PyLong_FromDouble}{double v}
  Return a new \ctype{PyLongObject} object from the integer part of
  \var{v}, or \NULL{} on failure.
\end{cfuncdesc}

\begin{cfuncdesc}{PyObject*}{PyLong_FromString}{char *str, char **pend,
                                                int base}
  Return a new \ctype{PyLongObject} based on the string value in
  \var{str}, which is interpreted according to the radix in
  \var{base}.  If \var{pend} is non-\NULL{}, \code{*\var{pend}} will
  point to the first character in \var{str} which follows the
  representation of the number.  If \var{base} is \code{0}, the radix
  will be determined based on the leading characters of \var{str}: if
  \var{str} starts with \code{'0x'} or \code{'0X'}, radix 16 will be
  used; if \var{str} starts with \code{'0'}, radix 8 will be used;
  otherwise radix 10 will be used.  If \var{base} is not \code{0}, it
  must be between \code{2} and \code{36}, inclusive.  Leading spaces
  are ignored.  If there are no digits, \exception{ValueError} will be
  raised.
\end{cfuncdesc}

\begin{cfuncdesc}{PyObject*}{PyLong_FromUnicode}{Py_UNICODE *u,
                                                 Py_ssize_t length, int base}
  Convert a sequence of Unicode digits to a Python long integer
  value.  The first parameter, \var{u}, points to the first character
  of the Unicode string, \var{length} gives the number of characters,
  and \var{base} is the radix for the conversion.  The radix must be
  in the range [2, 36]; if it is out of range, \exception{ValueError}
  will be raised.
  \versionadded{1.6}
\end{cfuncdesc}

\begin{cfuncdesc}{PyObject*}{PyLong_FromVoidPtr}{void *p}
  Create a Python integer or long integer from the pointer \var{p}.
  The pointer value can be retrieved from the resulting value using
  \cfunction{PyLong_AsVoidPtr()}.
  \versionadded{1.5.2}
  \versionchanged[If the integer is larger than LONG_MAX,
  a positive long integer is returned]{2.5}
 \end{cfuncdesc}

\begin{cfuncdesc}{long}{PyLong_AsLong}{PyObject *pylong}
  Return a C \ctype{long} representation of the contents of
  \var{pylong}.  If \var{pylong} is greater than
  \constant{LONG_MAX}\ttindex{LONG_MAX}, an \exception{OverflowError}
  is raised.
  \withsubitem{(built-in exception)}{\ttindex{OverflowError}}
\end{cfuncdesc}

\begin{cfuncdesc}{unsigned long}{PyLong_AsUnsignedLong}{PyObject *pylong}
  Return a C \ctype{unsigned long} representation of the contents of
  \var{pylong}.  If \var{pylong} is greater than
  \constant{ULONG_MAX}\ttindex{ULONG_MAX}, an
  \exception{OverflowError} is raised.
  \withsubitem{(built-in exception)}{\ttindex{OverflowError}}
\end{cfuncdesc}

\begin{cfuncdesc}{PY_LONG_LONG}{PyLong_AsLongLong}{PyObject *pylong}
  Return a C \ctype{long long} from a Python long integer.  If
  \var{pylong} cannot be represented as a \ctype{long long}, an
  \exception{OverflowError} will be raised.
  \versionadded{2.2}
\end{cfuncdesc}

\begin{cfuncdesc}{unsigned PY_LONG_LONG}{PyLong_AsUnsignedLongLong}{PyObject
                                                                 *pylong}
  Return a C \ctype{unsigned long long} from a Python long integer.
  If \var{pylong} cannot be represented as an \ctype{unsigned long
  long}, an \exception{OverflowError} will be raised if the value is
  positive, or a \exception{TypeError} will be raised if the value is
  negative.
  \versionadded{2.2}
\end{cfuncdesc}

\begin{cfuncdesc}{unsigned long}{PyLong_AsUnsignedLongMask}{PyObject *io}
  Return a C \ctype{unsigned long} from a Python long integer, without
  checking for overflow.
  \versionadded{2.3}
\end{cfuncdesc}

\begin{cfuncdesc}{unsigned PY_LONG_LONG}{PyLong_AsUnsignedLongLongMask}{PyObject *io}
  Return a C \ctype{unsigned long long} from a Python long integer, without
  checking for overflow.
  \versionadded{2.3}
\end{cfuncdesc}

\begin{cfuncdesc}{double}{PyLong_AsDouble}{PyObject *pylong}
  Return a C \ctype{double} representation of the contents of
  \var{pylong}.  If \var{pylong} cannot be approximately represented
  as a \ctype{double}, an \exception{OverflowError} exception is
  raised and \code{-1.0} will be returned.
\end{cfuncdesc}

\begin{cfuncdesc}{void*}{PyLong_AsVoidPtr}{PyObject *pylong}
  Convert a Python integer or long integer \var{pylong} to a C
  \ctype{void} pointer.  If \var{pylong} cannot be converted, an
  \exception{OverflowError} will be raised.  This is only assured to
  produce a usable \ctype{void} pointer for values created with
  \cfunction{PyLong_FromVoidPtr()}.
  \versionadded{1.5.2}
  \versionchanged[For values outside 0..LONG_MAX, both signed and
  unsigned integers are acccepted]{2.5}
\end{cfuncdesc}


\subsection{Floating Point Objects \label{floatObjects}}

\obindex{floating point}
\begin{ctypedesc}{PyFloatObject}
  This subtype of \ctype{PyObject} represents a Python floating point
  object.
\end{ctypedesc}

\begin{cvardesc}{PyTypeObject}{PyFloat_Type}
  This instance of \ctype{PyTypeObject} represents the Python floating
  point type.  This is the same object as \code{float} and
  \code{types.FloatType}.
  \withsubitem{(in modules types)}{\ttindex{FloatType}}
\end{cvardesc}

\begin{cfuncdesc}{int}{PyFloat_Check}{PyObject *p}
  Return true if its argument is a \ctype{PyFloatObject} or a subtype
  of \ctype{PyFloatObject}.
  \versionchanged[Allowed subtypes to be accepted]{2.2}
\end{cfuncdesc}

\begin{cfuncdesc}{int}{PyFloat_CheckExact}{PyObject *p}
  Return true if its argument is a \ctype{PyFloatObject}, but not a
  subtype of \ctype{PyFloatObject}.
  \versionadded{2.2}
\end{cfuncdesc}

\begin{cfuncdesc}{PyObject*}{PyFloat_FromString}{PyObject *str}
  Create a \ctype{PyFloatObject} object based on the string value in
  \var{str}, or \NULL{} on failure.
\end{cfuncdesc}

\begin{cfuncdesc}{PyObject*}{PyFloat_FromDouble}{double v}
  Create a \ctype{PyFloatObject} object from \var{v}, or \NULL{} on
  failure.
\end{cfuncdesc}

\begin{cfuncdesc}{double}{PyFloat_AsDouble}{PyObject *pyfloat}
  Return a C \ctype{double} representation of the contents of
  \var{pyfloat}.  If \var{pyfloat} is not a Python floating point
  object but has a \method{__float__} method, this method will first
  be called to convert \var{pyfloat} into a float.
\end{cfuncdesc}

\begin{cfuncdesc}{double}{PyFloat_AS_DOUBLE}{PyObject *pyfloat}
  Return a C \ctype{double} representation of the contents of
  \var{pyfloat}, but without error checking.
\end{cfuncdesc}


\subsection{Complex Number Objects \label{complexObjects}}

\obindex{complex number}
Python's complex number objects are implemented as two distinct types
when viewed from the C API:  one is the Python object exposed to
Python programs, and the other is a C structure which represents the
actual complex number value.  The API provides functions for working
with both.

\subsubsection{Complex Numbers as C Structures}

Note that the functions which accept these structures as parameters
and return them as results do so \emph{by value} rather than
dereferencing them through pointers.  This is consistent throughout
the API.

\begin{ctypedesc}{Py_complex}
  The C structure which corresponds to the value portion of a Python
  complex number object.  Most of the functions for dealing with
  complex number objects use structures of this type as input or
  output values, as appropriate.  It is defined as:

\begin{verbatim}
typedef struct {
   double real;
   double imag;
} Py_complex;
\end{verbatim}
\end{ctypedesc}

\begin{cfuncdesc}{Py_complex}{_Py_c_sum}{Py_complex left, Py_complex right}
  Return the sum of two complex numbers, using the C
  \ctype{Py_complex} representation.
\end{cfuncdesc}

\begin{cfuncdesc}{Py_complex}{_Py_c_diff}{Py_complex left, Py_complex right}
  Return the difference between two complex numbers, using the C
  \ctype{Py_complex} representation.
\end{cfuncdesc}

\begin{cfuncdesc}{Py_complex}{_Py_c_neg}{Py_complex complex}
  Return the negation of the complex number \var{complex}, using the C
  \ctype{Py_complex} representation.
\end{cfuncdesc}

\begin{cfuncdesc}{Py_complex}{_Py_c_prod}{Py_complex left, Py_complex right}
  Return the product of two complex numbers, using the C
  \ctype{Py_complex} representation.
\end{cfuncdesc}

\begin{cfuncdesc}{Py_complex}{_Py_c_quot}{Py_complex dividend,
                                          Py_complex divisor}
  Return the quotient of two complex numbers, using the C
  \ctype{Py_complex} representation.
\end{cfuncdesc}

\begin{cfuncdesc}{Py_complex}{_Py_c_pow}{Py_complex num, Py_complex exp}
  Return the exponentiation of \var{num} by \var{exp}, using the C
  \ctype{Py_complex} representation.
\end{cfuncdesc}


\subsubsection{Complex Numbers as Python Objects}

\begin{ctypedesc}{PyComplexObject}
  This subtype of \ctype{PyObject} represents a Python complex number
  object.
\end{ctypedesc}

\begin{cvardesc}{PyTypeObject}{PyComplex_Type}
  This instance of \ctype{PyTypeObject} represents the Python complex
  number type. It is the same object as \code{complex} and
  \code{types.ComplexType}.
\end{cvardesc}

\begin{cfuncdesc}{int}{PyComplex_Check}{PyObject *p}
  Return true if its argument is a \ctype{PyComplexObject} or a
  subtype of \ctype{PyComplexObject}.
  \versionchanged[Allowed subtypes to be accepted]{2.2}
\end{cfuncdesc}

\begin{cfuncdesc}{int}{PyComplex_CheckExact}{PyObject *p}
  Return true if its argument is a \ctype{PyComplexObject}, but not a
  subtype of \ctype{PyComplexObject}.
  \versionadded{2.2}
\end{cfuncdesc}

\begin{cfuncdesc}{PyObject*}{PyComplex_FromCComplex}{Py_complex v}
  Create a new Python complex number object from a C
  \ctype{Py_complex} value.
\end{cfuncdesc}

\begin{cfuncdesc}{PyObject*}{PyComplex_FromDoubles}{double real, double imag}
  Return a new \ctype{PyComplexObject} object from \var{real} and
  \var{imag}.
\end{cfuncdesc}

\begin{cfuncdesc}{double}{PyComplex_RealAsDouble}{PyObject *op}
  Return the real part of \var{op} as a C \ctype{double}.
\end{cfuncdesc}

\begin{cfuncdesc}{double}{PyComplex_ImagAsDouble}{PyObject *op}
  Return the imaginary part of \var{op} as a C \ctype{double}.
\end{cfuncdesc}

\begin{cfuncdesc}{Py_complex}{PyComplex_AsCComplex}{PyObject *op}
  Return the \ctype{Py_complex} value of the complex number \var{op}.
  \versionchanged[If \var{op} is not a Python complex number object
                  but has a \method{__complex__} method, this method
		  will first be called to convert \var{op} to a Python
		  complex number object]{2.6}
\end{cfuncdesc}



\section{Sequence Objects \label{sequenceObjects}}

\obindex{sequence}
Generic operations on sequence objects were discussed in the previous
chapter; this section deals with the specific kinds of sequence
objects that are intrinsic to the Python language.


\subsection{String Objects \label{stringObjects}}

These functions raise \exception{TypeError} when expecting a string
parameter and are called with a non-string parameter.

\obindex{string}
\begin{ctypedesc}{PyStringObject}
  This subtype of \ctype{PyObject} represents a Python string object.
\end{ctypedesc}

\begin{cvardesc}{PyTypeObject}{PyString_Type}
  This instance of \ctype{PyTypeObject} represents the Python string
  type; it is the same object as \code{str} and \code{types.StringType}
  in the Python layer.
  \withsubitem{(in module types)}{\ttindex{StringType}}.
\end{cvardesc}

\begin{cfuncdesc}{int}{PyString_Check}{PyObject *o}
  Return true if the object \var{o} is a string object or an instance
  of a subtype of the string type.
  \versionchanged[Allowed subtypes to be accepted]{2.2}
\end{cfuncdesc}

\begin{cfuncdesc}{int}{PyString_CheckExact}{PyObject *o}
  Return true if the object \var{o} is a string object, but not an
  instance of a subtype of the string type.
  \versionadded{2.2}
\end{cfuncdesc}

\begin{cfuncdesc}{PyObject*}{PyString_FromString}{const char *v}
  Return a new string object with a copy of the string \var{v} as value
  on success, and \NULL{} on failure.  The parameter \var{v} must not be
  \NULL{}; it will not be checked.
\end{cfuncdesc}

\begin{cfuncdesc}{PyObject*}{PyString_FromStringAndSize}{const char *v,
                                                         Py_ssize_t len}
  Return a new string object with a copy of the string \var{v} as value
  and length \var{len} on success, and \NULL{} on failure.  If \var{v} is
  \NULL{}, the contents of the string are uninitialized.
\end{cfuncdesc}

\begin{cfuncdesc}{PyObject*}{PyString_FromFormat}{const char *format, ...}
  Take a C \cfunction{printf()}-style \var{format} string and a
  variable number of arguments, calculate the size of the resulting
  Python string and return a string with the values formatted into
  it.  The variable arguments must be C types and must correspond
  exactly to the format characters in the \var{format} string.  The
  following format characters are allowed:

  % This should be exactly the same as the table in PyErr_Format.
  % One should just refer to the other.

  % The descriptions for %zd and %zu are wrong, but the truth is complicated
  % because not all compilers support the %z width modifier -- we fake it
  % when necessary via interpolating PY_FORMAT_SIZE_T.

  % %u, %lu, %zu should have "new in Python 2.5" blurbs.

  \begin{tableiii}{l|l|l}{member}{Format Characters}{Type}{Comment}
    \lineiii{\%\%}{\emph{n/a}}{The literal \% character.}
    \lineiii{\%c}{int}{A single character, represented as an C int.}
    \lineiii{\%d}{int}{Exactly equivalent to \code{printf("\%d")}.}
    \lineiii{\%u}{unsigned int}{Exactly equivalent to \code{printf("\%u")}.}
    \lineiii{\%ld}{long}{Exactly equivalent to \code{printf("\%ld")}.}
    \lineiii{\%lu}{unsigned long}{Exactly equivalent to \code{printf("\%lu")}.}
    \lineiii{\%zd}{Py_ssize_t}{Exactly equivalent to \code{printf("\%zd")}.}
    \lineiii{\%zu}{size_t}{Exactly equivalent to \code{printf("\%zu")}.}
    \lineiii{\%i}{int}{Exactly equivalent to \code{printf("\%i")}.}
    \lineiii{\%x}{int}{Exactly equivalent to \code{printf("\%x")}.}
    \lineiii{\%s}{char*}{A null-terminated C character array.}
    \lineiii{\%p}{void*}{The hex representation of a C pointer.
	Mostly equivalent to \code{printf("\%p")} except that it is
	guaranteed to start with the literal \code{0x} regardless of
	what the platform's \code{printf} yields.}
  \end{tableiii}

  An unrecognized format character causes all the rest of the format
  string to be copied as-is to the result string, and any extra
  arguments discarded.
\end{cfuncdesc}

\begin{cfuncdesc}{PyObject*}{PyString_FromFormatV}{const char *format,
                                                   va_list vargs}
  Identical to \function{PyString_FromFormat()} except that it takes
  exactly two arguments.
\end{cfuncdesc}

\begin{cfuncdesc}{Py_ssize_t}{PyString_Size}{PyObject *string}
  Return the length of the string in string object \var{string}.
\end{cfuncdesc}

\begin{cfuncdesc}{Py_ssize_t}{PyString_GET_SIZE}{PyObject *string}
  Macro form of \cfunction{PyString_Size()} but without error
  checking.
\end{cfuncdesc}

\begin{cfuncdesc}{char*}{PyString_AsString}{PyObject *string}
  Return a NUL-terminated representation of the contents of
  \var{string}.  The pointer refers to the internal buffer of
  \var{string}, not a copy.  The data must not be modified in any way,
  unless the string was just created using
  \code{PyString_FromStringAndSize(NULL, \var{size})}.
  It must not be deallocated.  If \var{string} is a Unicode object,
  this function computes the default encoding of \var{string} and
  operates on that.  If \var{string} is not a string object at all,
  \cfunction{PyString_AsString()} returns \NULL{} and raises
  \exception{TypeError}.
\end{cfuncdesc}

\begin{cfuncdesc}{char*}{PyString_AS_STRING}{PyObject *string}
  Macro form of \cfunction{PyString_AsString()} but without error
  checking.  Only string objects are supported; no Unicode objects
  should be passed.
\end{cfuncdesc}

\begin{cfuncdesc}{int}{PyString_AsStringAndSize}{PyObject *obj,
                                                 char **buffer,
                                                 Py_ssize_t *length}
  Return a NUL-terminated representation of the contents of the
  object \var{obj} through the output variables \var{buffer} and
  \var{length}.

  The function accepts both string and Unicode objects as input. For
  Unicode objects it returns the default encoded version of the
  object.  If \var{length} is \NULL{}, the resulting buffer may not
  contain NUL characters; if it does, the function returns \code{-1}
  and a \exception{TypeError} is raised.

  The buffer refers to an internal string buffer of \var{obj}, not a
  copy. The data must not be modified in any way, unless the string
  was just created using \code{PyString_FromStringAndSize(NULL,
  \var{size})}.  It must not be deallocated.  If \var{string} is a
  Unicode object, this function computes the default encoding of
  \var{string} and operates on that.  If \var{string} is not a string
  object at all, \cfunction{PyString_AsStringAndSize()} returns
  \code{-1} and raises \exception{TypeError}.
\end{cfuncdesc}

\begin{cfuncdesc}{void}{PyString_Concat}{PyObject **string,
                                         PyObject *newpart}
  Create a new string object in \var{*string} containing the contents
  of \var{newpart} appended to \var{string}; the caller will own the
  new reference.  The reference to the old value of \var{string} will
  be stolen.  If the new string cannot be created, the old reference
  to \var{string} will still be discarded and the value of
  \var{*string} will be set to \NULL{}; the appropriate exception will
  be set.
\end{cfuncdesc}

\begin{cfuncdesc}{void}{PyString_ConcatAndDel}{PyObject **string,
                                               PyObject *newpart}
  Create a new string object in \var{*string} containing the contents
  of \var{newpart} appended to \var{string}.  This version decrements
  the reference count of \var{newpart}.
\end{cfuncdesc}

\begin{cfuncdesc}{int}{_PyString_Resize}{PyObject **string, Py_ssize_t newsize}
  A way to resize a string object even though it is ``immutable''.
  Only use this to build up a brand new string object; don't use this
  if the string may already be known in other parts of the code.  It
  is an error to call this function if the refcount on the input string
  object is not one.
  Pass the address of an existing string object as an lvalue (it may
  be written into), and the new size desired.  On success, \var{*string}
  holds the resized string object and \code{0} is returned; the address in
  \var{*string} may differ from its input value.  If the
  reallocation fails, the original string object at \var{*string} is
  deallocated, \var{*string} is set to \NULL{}, a memory exception is set,
  and \code{-1} is returned.
\end{cfuncdesc}

\begin{cfuncdesc}{PyObject*}{PyString_Format}{PyObject *format,
                                              PyObject *args}
  Return a new string object from \var{format} and \var{args}.
  Analogous to \code{\var{format} \%\ \var{args}}.  The \var{args}
  argument must be a tuple.
\end{cfuncdesc}

\begin{cfuncdesc}{void}{PyString_InternInPlace}{PyObject **string}
  Intern the argument \var{*string} in place.  The argument must be
  the address of a pointer variable pointing to a Python string
  object.  If there is an existing interned string that is the same as
  \var{*string}, it sets \var{*string} to it (decrementing the
  reference count of the old string object and incrementing the
  reference count of the interned string object), otherwise it leaves
  \var{*string} alone and interns it (incrementing its reference
  count).  (Clarification: even though there is a lot of talk about
  reference counts, think of this function as reference-count-neutral;
  you own the object after the call if and only if you owned it before
  the call.)
\end{cfuncdesc}

\begin{cfuncdesc}{PyObject*}{PyString_InternFromString}{const char *v}
  A combination of \cfunction{PyString_FromString()} and
  \cfunction{PyString_InternInPlace()}, returning either a new string
  object that has been interned, or a new (``owned'') reference to an
  earlier interned string object with the same value.
\end{cfuncdesc}

\begin{cfuncdesc}{PyObject*}{PyString_Decode}{const char *s,
                                               Py_ssize_t size,
                                               const char *encoding,
                                               const char *errors}
  Create an object by decoding \var{size} bytes of the encoded
  buffer \var{s} using the codec registered for
  \var{encoding}.  \var{encoding} and \var{errors} have the same
  meaning as the parameters of the same name in the
  \function{unicode()} built-in function.  The codec to be used is
  looked up using the Python codec registry.  Return \NULL{} if
  an exception was raised by the codec.
\end{cfuncdesc}

\begin{cfuncdesc}{PyObject*}{PyString_AsDecodedObject}{PyObject *str,
                                               const char *encoding,
                                               const char *errors}
  Decode a string object by passing it to the codec registered for
  \var{encoding} and return the result as Python
  object. \var{encoding} and \var{errors} have the same meaning as the
  parameters of the same name in the string \method{encode()} method.
  The codec to be used is looked up using the Python codec registry.
  Return \NULL{} if an exception was raised by the codec.
\end{cfuncdesc}

\begin{cfuncdesc}{PyObject*}{PyString_Encode}{const char *s,
                                               Py_ssize_t size,
                                               const char *encoding,
                                               const char *errors}
  Encode the \ctype{char} buffer of the given size by passing it to
  the codec registered for \var{encoding} and return a Python object.
  \var{encoding} and \var{errors} have the same meaning as the
  parameters of the same name in the string \method{encode()} method.
  The codec to be used is looked up using the Python codec
  registry.  Return \NULL{} if an exception was raised by the
  codec.
\end{cfuncdesc}

\begin{cfuncdesc}{PyObject*}{PyString_AsEncodedObject}{PyObject *str,
                                               const char *encoding,
                                               const char *errors}
  Encode a string object using the codec registered for
  \var{encoding} and return the result as Python object.
  \var{encoding} and \var{errors} have the same meaning as the
  parameters of the same name in the string \method{encode()} method.
  The codec to be used is looked up using the Python codec registry.
  Return \NULL{} if an exception was raised by the codec.
\end{cfuncdesc}


\subsection{Unicode Objects \label{unicodeObjects}}
\sectionauthor{Marc-Andre Lemburg}{mal@lemburg.com}

%--- Unicode Type -------------------------------------------------------

These are the basic Unicode object types used for the Unicode
implementation in Python:

\begin{ctypedesc}{Py_UNICODE}
  This type represents the storage type which is used by Python
  internally as basis for holding Unicode ordinals.  Python's default
  builds use a 16-bit type for \ctype{Py_UNICODE} and store Unicode
  values internally as UCS2. It is also possible to build a UCS4
  version of Python (most recent Linux distributions come with UCS4
  builds of Python). These builds then use a 32-bit type for
  \ctype{Py_UNICODE} and store Unicode data internally as UCS4. On
  platforms where \ctype{wchar_t} is available and compatible with the
  chosen Python Unicode build variant, \ctype{Py_UNICODE} is a typedef
  alias for \ctype{wchar_t} to enhance native platform compatibility.
  On all other platforms, \ctype{Py_UNICODE} is a typedef alias for
  either \ctype{unsigned short} (UCS2) or \ctype{unsigned long}
  (UCS4).
\end{ctypedesc}

Note that UCS2 and UCS4 Python builds are not binary compatible.
Please keep this in mind when writing extensions or interfaces.

\begin{ctypedesc}{PyUnicodeObject}
  This subtype of \ctype{PyObject} represents a Python Unicode object.
\end{ctypedesc}

\begin{cvardesc}{PyTypeObject}{PyUnicode_Type}
  This instance of \ctype{PyTypeObject} represents the Python Unicode
  type.  It is exposed to Python code as \code{unicode} and
  \code{types.UnicodeType}.
\end{cvardesc}

The following APIs are really C macros and can be used to do fast
checks and to access internal read-only data of Unicode objects:

\begin{cfuncdesc}{int}{PyUnicode_Check}{PyObject *o}
  Return true if the object \var{o} is a Unicode object or an
  instance of a Unicode subtype.
  \versionchanged[Allowed subtypes to be accepted]{2.2}
\end{cfuncdesc}

\begin{cfuncdesc}{int}{PyUnicode_CheckExact}{PyObject *o}
  Return true if the object \var{o} is a Unicode object, but not an
  instance of a subtype.
  \versionadded{2.2}
\end{cfuncdesc}

\begin{cfuncdesc}{Py_ssize_t}{PyUnicode_GET_SIZE}{PyObject *o}
  Return the size of the object.  \var{o} has to be a
  \ctype{PyUnicodeObject} (not checked).
\end{cfuncdesc}

\begin{cfuncdesc}{Py_ssize_t}{PyUnicode_GET_DATA_SIZE}{PyObject *o}
  Return the size of the object's internal buffer in bytes.  \var{o}
  has to be a \ctype{PyUnicodeObject} (not checked).
\end{cfuncdesc}

\begin{cfuncdesc}{Py_UNICODE*}{PyUnicode_AS_UNICODE}{PyObject *o}
  Return a pointer to the internal \ctype{Py_UNICODE} buffer of the
  object.  \var{o} has to be a \ctype{PyUnicodeObject} (not checked).
\end{cfuncdesc}

\begin{cfuncdesc}{const char*}{PyUnicode_AS_DATA}{PyObject *o}
  Return a pointer to the internal buffer of the object.
  \var{o} has to be a \ctype{PyUnicodeObject} (not checked).
\end{cfuncdesc}

% --- Unicode character properties ---------------------------------------

Unicode provides many different character properties. The most often
needed ones are available through these macros which are mapped to C
functions depending on the Python configuration.

\begin{cfuncdesc}{int}{Py_UNICODE_ISSPACE}{Py_UNICODE ch}
  Return 1 or 0 depending on whether \var{ch} is a whitespace
  character.
\end{cfuncdesc}

\begin{cfuncdesc}{int}{Py_UNICODE_ISLOWER}{Py_UNICODE ch}
  Return 1 or 0 depending on whether \var{ch} is a lowercase character.
\end{cfuncdesc}

\begin{cfuncdesc}{int}{Py_UNICODE_ISUPPER}{Py_UNICODE ch}
  Return 1 or 0 depending on whether \var{ch} is an uppercase
  character.
\end{cfuncdesc}

\begin{cfuncdesc}{int}{Py_UNICODE_ISTITLE}{Py_UNICODE ch}
  Return 1 or 0 depending on whether \var{ch} is a titlecase character.
\end{cfuncdesc}

\begin{cfuncdesc}{int}{Py_UNICODE_ISLINEBREAK}{Py_UNICODE ch}
  Return 1 or 0 depending on whether \var{ch} is a linebreak character.
\end{cfuncdesc}

\begin{cfuncdesc}{int}{Py_UNICODE_ISDECIMAL}{Py_UNICODE ch}
  Return 1 or 0 depending on whether \var{ch} is a decimal character.
\end{cfuncdesc}

\begin{cfuncdesc}{int}{Py_UNICODE_ISDIGIT}{Py_UNICODE ch}
  Return 1 or 0 depending on whether \var{ch} is a digit character.
\end{cfuncdesc}

\begin{cfuncdesc}{int}{Py_UNICODE_ISNUMERIC}{Py_UNICODE ch}
  Return 1 or 0 depending on whether \var{ch} is a numeric character.
\end{cfuncdesc}

\begin{cfuncdesc}{int}{Py_UNICODE_ISALPHA}{Py_UNICODE ch}
  Return 1 or 0 depending on whether \var{ch} is an alphabetic
  character.
\end{cfuncdesc}

\begin{cfuncdesc}{int}{Py_UNICODE_ISALNUM}{Py_UNICODE ch}
  Return 1 or 0 depending on whether \var{ch} is an alphanumeric
  character.
\end{cfuncdesc}

These APIs can be used for fast direct character conversions:

\begin{cfuncdesc}{Py_UNICODE}{Py_UNICODE_TOLOWER}{Py_UNICODE ch}
  Return the character \var{ch} converted to lower case.
\end{cfuncdesc}

\begin{cfuncdesc}{Py_UNICODE}{Py_UNICODE_TOUPPER}{Py_UNICODE ch}
  Return the character \var{ch} converted to upper case.
\end{cfuncdesc}

\begin{cfuncdesc}{Py_UNICODE}{Py_UNICODE_TOTITLE}{Py_UNICODE ch}
  Return the character \var{ch} converted to title case.
\end{cfuncdesc}

\begin{cfuncdesc}{int}{Py_UNICODE_TODECIMAL}{Py_UNICODE ch}
  Return the character \var{ch} converted to a decimal positive
  integer.  Return \code{-1} if this is not possible.  This macro
  does not raise exceptions.
\end{cfuncdesc}

\begin{cfuncdesc}{int}{Py_UNICODE_TODIGIT}{Py_UNICODE ch}
  Return the character \var{ch} converted to a single digit integer.
  Return \code{-1} if this is not possible.  This macro does not raise
  exceptions.
\end{cfuncdesc}

\begin{cfuncdesc}{double}{Py_UNICODE_TONUMERIC}{Py_UNICODE ch}
  Return the character \var{ch} converted to a double.
  Return \code{-1.0} if this is not possible.  This macro does not raise
  exceptions.
\end{cfuncdesc}

% --- Plain Py_UNICODE ---------------------------------------------------

To create Unicode objects and access their basic sequence properties,
use these APIs:

\begin{cfuncdesc}{PyObject*}{PyUnicode_FromUnicode}{const Py_UNICODE *u,
                                                    Py_ssize_t size}
  Create a Unicode Object from the Py_UNICODE buffer \var{u} of the
  given size. \var{u} may be \NULL{} which causes the contents to be
  undefined. It is the user's responsibility to fill in the needed
  data.  The buffer is copied into the new object. If the buffer is
  not \NULL{}, the return value might be a shared object. Therefore,
  modification of the resulting Unicode object is only allowed when
  \var{u} is \NULL{}.
\end{cfuncdesc}

\begin{cfuncdesc}{PyObject*}{PyUnicode_FromStringAndSize}{const char *u,
                                                       Py_ssize_t size}
  Create a Unicode Object from the char buffer \var{u}.
  The bytes will be interpreted as being UTF-8 encoded. 
  \var{u} may also be \NULL{} which causes the
  contents to be undefined. It is the user's responsibility to fill
  in the needed data.  The buffer is copied into the new object.
  If the buffer is not \NULL{}, the return value might be a shared object.
  Therefore, modification of the resulting Unicode object is only allowed
  when \var{u} is \NULL{}.
  \versionadded{3.0}
\end{cfuncdesc}

\begin{cfuncdesc}{PyObject*}{PyUnicode_FromString}{const char*u}
   Create a Unicode object from an UTF-8 encoded null-terminated
   char buffer \var{u}.
   \versionadded{3.0}
\end{funcdesc}

\begin{cfuncdesc}{PyObject*}{PyUnicode_FromFormat}{const char *format, ...}
  Take a C \cfunction{printf()}-style \var{format} string and a
  variable number of arguments, calculate the size of the resulting
  Python unicode string and return a string with the values formatted into
  it.  The variable arguments must be C types and must correspond
  exactly to the format characters in the \var{format} string.  The
  following format characters are allowed:

  % The descriptions for %zd and %zu are wrong, but the truth is complicated
  % because not all compilers support the %z width modifier -- we fake it
  % when necessary via interpolating PY_FORMAT_SIZE_T.

  \begin{tableiii}{l|l|l}{member}{Format Characters}{Type}{Comment}
    \lineiii{\%\%}{\emph{n/a}}{The literal \% character.}
    \lineiii{\%c}{int}{A single character, represented as an C int.}
    \lineiii{\%d}{int}{Exactly equivalent to \code{printf("\%d")}.}
    \lineiii{\%u}{unsigned int}{Exactly equivalent to \code{printf("\%u")}.}
    \lineiii{\%ld}{long}{Exactly equivalent to \code{printf("\%ld")}.}
    \lineiii{\%lu}{unsigned long}{Exactly equivalent to \code{printf("\%lu")}.}
    \lineiii{\%zd}{Py_ssize_t}{Exactly equivalent to \code{printf("\%zd")}.}
    \lineiii{\%zu}{size_t}{Exactly equivalent to \code{printf("\%zu")}.}
    \lineiii{\%i}{int}{Exactly equivalent to \code{printf("\%i")}.}
    \lineiii{\%x}{int}{Exactly equivalent to \code{printf("\%x")}.}
    \lineiii{\%s}{char*}{A null-terminated C character array.}
    \lineiii{\%p}{void*}{The hex representation of a C pointer.
	Mostly equivalent to \code{printf("\%p")} except that it is
	guaranteed to start with the literal \code{0x} regardless of
	what the platform's \code{printf} yields.}
    \lineiii{\%U}{PyObject*}{A unicode object.}
    \lineiii{\%V}{PyObject*, char *}{A unicode object (which may be \NULL{})
	and a null-terminated C character array as a second parameter (which
	will be used, if the first parameter is \NULL{}).}
    \lineiii{\%S}{PyObject*}{The result of calling \function{PyObject_Unicode()}.}
    \lineiii{\%R}{PyObject*}{The result of calling \function{PyObject_Repr()}.}
  \end{tableiii}

  An unrecognized format character causes all the rest of the format
  string to be copied as-is to the result string, and any extra
  arguments discarded.
  \versionadded{3.0}
\end{cfuncdesc}

\begin{cfuncdesc}{PyObject*}{PyUnicode_FromFormatV}{const char *format,
                                                   va_list vargs}
  Identical to \function{PyUnicode_FromFormat()} except that it takes
  exactly two arguments.
  \versionadded{3.0}
\end{cfuncdesc}

\begin{cfuncdesc}{Py_UNICODE*}{PyUnicode_AsUnicode}{PyObject *unicode}
  Return a read-only pointer to the Unicode object's internal
  \ctype{Py_UNICODE} buffer, \NULL{} if \var{unicode} is not a Unicode
  object.
\end{cfuncdesc}

\begin{cfuncdesc}{Py_ssize_t}{PyUnicode_GetSize}{PyObject *unicode}
  Return the length of the Unicode object.
\end{cfuncdesc}

\begin{cfuncdesc}{PyObject*}{PyUnicode_FromEncodedObject}{PyObject *obj,
                                                      const char *encoding,
                                                      const char *errors}
  Coerce an encoded object \var{obj} to an Unicode object and return a
  reference with incremented refcount.
  
  String and other char buffer compatible objects are decoded
  according to the given encoding and using the error handling
  defined by errors.  Both can be \NULL{} to have the interface
  use the default values (see the next section for details).

  All other objects, including Unicode objects, cause a
  \exception{TypeError} to be set.

  The API returns \NULL{} if there was an error.  The caller is
  responsible for decref'ing the returned objects.
\end{cfuncdesc}

\begin{cfuncdesc}{PyObject*}{PyUnicode_FromObject}{PyObject *obj}
  Shortcut for \code{PyUnicode_FromEncodedObject(obj, NULL, "strict")}
  which is used throughout the interpreter whenever coercion to
  Unicode is needed.
\end{cfuncdesc}

% --- wchar_t support for platforms which support it ---------------------

If the platform supports \ctype{wchar_t} and provides a header file
wchar.h, Python can interface directly to this type using the
following functions. Support is optimized if Python's own
\ctype{Py_UNICODE} type is identical to the system's \ctype{wchar_t}.

\begin{cfuncdesc}{PyObject*}{PyUnicode_FromWideChar}{const wchar_t *w,
                                                     Py_ssize_t size}
  Create a Unicode object from the \ctype{wchar_t} buffer \var{w} of
  the given size.  Return \NULL{} on failure.
\end{cfuncdesc}

\begin{cfuncdesc}{Py_ssize_t}{PyUnicode_AsWideChar}{PyUnicodeObject *unicode,
                                             wchar_t *w,
                                             Py_ssize_t size}
  Copy the Unicode object contents into the \ctype{wchar_t} buffer
  \var{w}.  At most \var{size} \ctype{wchar_t} characters are copied
  (excluding a possibly trailing 0-termination character).  Return
  the number of \ctype{wchar_t} characters copied or -1 in case of an
  error.  Note that the resulting \ctype{wchar_t} string may or may
  not be 0-terminated.  It is the responsibility of the caller to make
  sure that the \ctype{wchar_t} string is 0-terminated in case this is
  required by the application.
\end{cfuncdesc}


\subsubsection{Built-in Codecs \label{builtinCodecs}}

Python provides a set of builtin codecs which are written in C
for speed. All of these codecs are directly usable via the
following functions.

Many of the following APIs take two arguments encoding and
errors. These parameters encoding and errors have the same semantics
as the ones of the builtin unicode() Unicode object constructor.

Setting encoding to \NULL{} causes the default encoding to be used
which is \ASCII.  The file system calls should use
\cdata{Py_FileSystemDefaultEncoding} as the encoding for file
names. This variable should be treated as read-only: On some systems,
it will be a pointer to a static string, on others, it will change at
run-time (such as when the application invokes setlocale).

Error handling is set by errors which may also be set to \NULL{}
meaning to use the default handling defined for the codec.  Default
error handling for all builtin codecs is ``strict''
(\exception{ValueError} is raised).

The codecs all use a similar interface.  Only deviation from the
following generic ones are documented for simplicity.

% --- Generic Codecs -----------------------------------------------------

These are the generic codec APIs:

\begin{cfuncdesc}{PyObject*}{PyUnicode_Decode}{const char *s,
                                               Py_ssize_t size,
                                               const char *encoding,
                                               const char *errors}
  Create a Unicode object by decoding \var{size} bytes of the encoded
  string \var{s}.  \var{encoding} and \var{errors} have the same
  meaning as the parameters of the same name in the
  \function{unicode()} builtin function.  The codec to be used is
  looked up using the Python codec registry.  Return \NULL{} if an
  exception was raised by the codec.
\end{cfuncdesc}

\begin{cfuncdesc}{PyObject*}{PyUnicode_Encode}{const Py_UNICODE *s,
                                               Py_ssize_t size,
                                               const char *encoding,
                                               const char *errors}
  Encode the \ctype{Py_UNICODE} buffer of the given size and return
  a Python string object.  \var{encoding} and \var{errors} have the
  same meaning as the parameters of the same name in the Unicode
  \method{encode()} method.  The codec to be used is looked up using
  the Python codec registry.  Return \NULL{} if an exception was
  raised by the codec.
\end{cfuncdesc}

\begin{cfuncdesc}{PyObject*}{PyUnicode_AsEncodedString}{PyObject *unicode,
                                               const char *encoding,
                                               const char *errors}
  Encode a Unicode object and return the result as Python string
  object. \var{encoding} and \var{errors} have the same meaning as the
  parameters of the same name in the Unicode \method{encode()} method.
  The codec to be used is looked up using the Python codec registry.
  Return \NULL{} if an exception was raised by the codec.
\end{cfuncdesc}

% --- UTF-8 Codecs -------------------------------------------------------

These are the UTF-8 codec APIs:

\begin{cfuncdesc}{PyObject*}{PyUnicode_DecodeUTF8}{const char *s,
                                               Py_ssize_t size,
                                               const char *errors}
  Create a Unicode object by decoding \var{size} bytes of the UTF-8
  encoded string \var{s}. Return \NULL{} if an exception was raised
  by the codec.
\end{cfuncdesc}

\begin{cfuncdesc}{PyObject*}{PyUnicode_DecodeUTF8Stateful}{const char *s,
                                               Py_ssize_t size,
                                               const char *errors,
                                               Py_ssize_t *consumed}
  If \var{consumed} is \NULL{}, behave like \cfunction{PyUnicode_DecodeUTF8()}.
  If \var{consumed} is not \NULL{}, trailing incomplete UTF-8 byte sequences
  will not be treated as an error. Those bytes will not be decoded and the
  number of bytes that have been decoded will be stored in \var{consumed}.
  \versionadded{2.4}
\end{cfuncdesc}

\begin{cfuncdesc}{PyObject*}{PyUnicode_EncodeUTF8}{const Py_UNICODE *s,
                                               Py_ssize_t size,
                                               const char *errors}
  Encode the \ctype{Py_UNICODE} buffer of the given size using UTF-8
  and return a Python string object.  Return \NULL{} if an exception
  was raised by the codec.
\end{cfuncdesc}

\begin{cfuncdesc}{PyObject*}{PyUnicode_AsUTF8String}{PyObject *unicode}
  Encode a Unicode objects using UTF-8 and return the result as
  Python string object.  Error handling is ``strict''.  Return
  \NULL{} if an exception was raised by the codec.
\end{cfuncdesc}

% --- UTF-16 Codecs ------------------------------------------------------ */

These are the UTF-16 codec APIs:

\begin{cfuncdesc}{PyObject*}{PyUnicode_DecodeUTF16}{const char *s,
                                               Py_ssize_t size,
                                               const char *errors,
                                               int *byteorder}
  Decode \var{length} bytes from a UTF-16 encoded buffer string and
  return the corresponding Unicode object.  \var{errors} (if
  non-\NULL{}) defines the error handling. It defaults to ``strict''.

  If \var{byteorder} is non-\NULL{}, the decoder starts decoding using
  the given byte order:

\begin{verbatim}
   *byteorder == -1: little endian
   *byteorder == 0:  native order
   *byteorder == 1:  big endian
\end{verbatim}

  and then switches if the first two bytes of the input data are a byte order
  mark (BOM) and the specified byte order is native order.  This BOM is not
  copied into the resulting Unicode string.  After completion, \var{*byteorder}
  is set to the current byte order at the.

  If \var{byteorder} is \NULL{}, the codec starts in native order mode.

  Return \NULL{} if an exception was raised by the codec.
\end{cfuncdesc}

\begin{cfuncdesc}{PyObject*}{PyUnicode_DecodeUTF16Stateful}{const char *s,
                                               Py_ssize_t size,
                                               const char *errors,
                                               int *byteorder,
                                               Py_ssize_t *consumed}
  If \var{consumed} is \NULL{}, behave like
  \cfunction{PyUnicode_DecodeUTF16()}. If \var{consumed} is not \NULL{},
  \cfunction{PyUnicode_DecodeUTF16Stateful()} will not treat trailing incomplete
  UTF-16 byte sequences (such as an odd number of bytes or a split surrogate pair)
  as an error. Those bytes will not be decoded and the number of bytes that
  have been decoded will be stored in \var{consumed}.
  \versionadded{2.4}
\end{cfuncdesc}

\begin{cfuncdesc}{PyObject*}{PyUnicode_EncodeUTF16}{const Py_UNICODE *s,
                                               Py_ssize_t size,
                                               const char *errors,
                                               int byteorder}
  Return a Python string object holding the UTF-16 encoded value of
  the Unicode data in \var{s}.  If \var{byteorder} is not \code{0},
  output is written according to the following byte order:

\begin{verbatim}
   byteorder == -1: little endian
   byteorder == 0:  native byte order (writes a BOM mark)
   byteorder == 1:  big endian
\end{verbatim}

  If byteorder is \code{0}, the output string will always start with
  the Unicode BOM mark (U+FEFF). In the other two modes, no BOM mark
  is prepended.

  If \var{Py_UNICODE_WIDE} is defined, a single \ctype{Py_UNICODE}
  value may get represented as a surrogate pair. If it is not
  defined, each \ctype{Py_UNICODE} values is interpreted as an
  UCS-2 character.

  Return \NULL{} if an exception was raised by the codec.
\end{cfuncdesc}

\begin{cfuncdesc}{PyObject*}{PyUnicode_AsUTF16String}{PyObject *unicode}
  Return a Python string using the UTF-16 encoding in native byte
  order. The string always starts with a BOM mark.  Error handling is
  ``strict''.  Return \NULL{} if an exception was raised by the
  codec.
\end{cfuncdesc}

% --- Unicode-Escape Codecs ----------------------------------------------

These are the ``Unicode Escape'' codec APIs:

\begin{cfuncdesc}{PyObject*}{PyUnicode_DecodeUnicodeEscape}{const char *s,
                                               Py_ssize_t size,
                                               const char *errors}
  Create a Unicode object by decoding \var{size} bytes of the
  Unicode-Escape encoded string \var{s}.  Return \NULL{} if an
  exception was raised by the codec.
\end{cfuncdesc}

\begin{cfuncdesc}{PyObject*}{PyUnicode_EncodeUnicodeEscape}{const Py_UNICODE *s,
                                               Py_ssize_t size}
  Encode the \ctype{Py_UNICODE} buffer of the given size using
  Unicode-Escape and return a Python string object.  Return \NULL{}
  if an exception was raised by the codec.
\end{cfuncdesc}

\begin{cfuncdesc}{PyObject*}{PyUnicode_AsUnicodeEscapeString}{PyObject *unicode}
  Encode a Unicode objects using Unicode-Escape and return the
  result as Python string object.  Error handling is ``strict''.
  Return \NULL{} if an exception was raised by the codec.
\end{cfuncdesc}

% --- Raw-Unicode-Escape Codecs ------------------------------------------

These are the ``Raw Unicode Escape'' codec APIs:

\begin{cfuncdesc}{PyObject*}{PyUnicode_DecodeRawUnicodeEscape}{const char *s,
                                               Py_ssize_t size,
                                               const char *errors}
  Create a Unicode object by decoding \var{size} bytes of the
  Raw-Unicode-Escape encoded string \var{s}.  Return \NULL{} if an
  exception was raised by the codec.
\end{cfuncdesc}

\begin{cfuncdesc}{PyObject*}{PyUnicode_EncodeRawUnicodeEscape}{const Py_UNICODE *s,
                                               Py_ssize_t size,
                                               const char *errors}
  Encode the \ctype{Py_UNICODE} buffer of the given size using
  Raw-Unicode-Escape and return a Python string object.  Return
  \NULL{} if an exception was raised by the codec.
\end{cfuncdesc}

\begin{cfuncdesc}{PyObject*}{PyUnicode_AsRawUnicodeEscapeString}{PyObject *unicode}
  Encode a Unicode objects using Raw-Unicode-Escape and return the
  result as Python string object. Error handling is ``strict''.
  Return \NULL{} if an exception was raised by the codec.
\end{cfuncdesc}

% --- Latin-1 Codecs -----------------------------------------------------

These are the Latin-1 codec APIs:
Latin-1 corresponds to the first 256 Unicode ordinals and only these
are accepted by the codecs during encoding.

\begin{cfuncdesc}{PyObject*}{PyUnicode_DecodeLatin1}{const char *s,
                                                     Py_ssize_t size,
                                                     const char *errors}
  Create a Unicode object by decoding \var{size} bytes of the Latin-1
  encoded string \var{s}.  Return \NULL{} if an exception was raised
  by the codec.
\end{cfuncdesc}

\begin{cfuncdesc}{PyObject*}{PyUnicode_EncodeLatin1}{const Py_UNICODE *s,
                                                     Py_ssize_t size,
                                                     const char *errors}
  Encode the \ctype{Py_UNICODE} buffer of the given size using
  Latin-1 and return a Python string object.  Return \NULL{} if an
  exception was raised by the codec.
\end{cfuncdesc}

\begin{cfuncdesc}{PyObject*}{PyUnicode_AsLatin1String}{PyObject *unicode}
  Encode a Unicode objects using Latin-1 and return the result as
  Python string object.  Error handling is ``strict''.  Return
  \NULL{} if an exception was raised by the codec.
\end{cfuncdesc}

% --- ASCII Codecs -------------------------------------------------------

These are the \ASCII{} codec APIs.  Only 7-bit \ASCII{} data is
accepted. All other codes generate errors.

\begin{cfuncdesc}{PyObject*}{PyUnicode_DecodeASCII}{const char *s,
                                                    Py_ssize_t size,
                                                    const char *errors}
  Create a Unicode object by decoding \var{size} bytes of the
  \ASCII{} encoded string \var{s}.  Return \NULL{} if an exception
  was raised by the codec.
\end{cfuncdesc}

\begin{cfuncdesc}{PyObject*}{PyUnicode_EncodeASCII}{const Py_UNICODE *s,
                                                    Py_ssize_t size,
                                                    const char *errors}
  Encode the \ctype{Py_UNICODE} buffer of the given size using
  \ASCII{} and return a Python string object.  Return \NULL{} if an
  exception was raised by the codec.
\end{cfuncdesc}

\begin{cfuncdesc}{PyObject*}{PyUnicode_AsASCIIString}{PyObject *unicode}
  Encode a Unicode objects using \ASCII{} and return the result as
  Python string object.  Error handling is ``strict''.  Return
  \NULL{} if an exception was raised by the codec.
\end{cfuncdesc}

% --- Character Map Codecs -----------------------------------------------

These are the mapping codec APIs:

This codec is special in that it can be used to implement many
different codecs (and this is in fact what was done to obtain most of
the standard codecs included in the \module{encodings} package). The
codec uses mapping to encode and decode characters.

Decoding mappings must map single string characters to single Unicode
characters, integers (which are then interpreted as Unicode ordinals)
or None (meaning "undefined mapping" and causing an error).

Encoding mappings must map single Unicode characters to single string
characters, integers (which are then interpreted as Latin-1 ordinals)
or None (meaning "undefined mapping" and causing an error).

The mapping objects provided must only support the __getitem__ mapping
interface.

If a character lookup fails with a LookupError, the character is
copied as-is meaning that its ordinal value will be interpreted as
Unicode or Latin-1 ordinal resp. Because of this, mappings only need
to contain those mappings which map characters to different code
points.

\begin{cfuncdesc}{PyObject*}{PyUnicode_DecodeCharmap}{const char *s,
                                               Py_ssize_t size,
                                               PyObject *mapping,
                                               const char *errors}
  Create a Unicode object by decoding \var{size} bytes of the encoded
  string \var{s} using the given \var{mapping} object.  Return
  \NULL{} if an exception was raised by the codec. If \var{mapping} is \NULL{}
  latin-1 decoding will be done. Else it can be a dictionary mapping byte or a
  unicode string, which is treated as a lookup table. Byte values greater
  that the length of the string and U+FFFE "characters" are treated as
  "undefined mapping".
  \versionchanged[Allowed unicode string as mapping argument]{2.4}
\end{cfuncdesc}

\begin{cfuncdesc}{PyObject*}{PyUnicode_EncodeCharmap}{const Py_UNICODE *s,
                                               Py_ssize_t size,
                                               PyObject *mapping,
                                               const char *errors}
  Encode the \ctype{Py_UNICODE} buffer of the given size using the
  given \var{mapping} object and return a Python string object.
  Return \NULL{} if an exception was raised by the codec.
\end{cfuncdesc}

\begin{cfuncdesc}{PyObject*}{PyUnicode_AsCharmapString}{PyObject *unicode,
                                                        PyObject *mapping}
  Encode a Unicode objects using the given \var{mapping} object and
  return the result as Python string object.  Error handling is
  ``strict''.  Return \NULL{} if an exception was raised by the
  codec.
\end{cfuncdesc}

The following codec API is special in that maps Unicode to Unicode.

\begin{cfuncdesc}{PyObject*}{PyUnicode_TranslateCharmap}{const Py_UNICODE *s,
                                               Py_ssize_t size,
                                               PyObject *table,
                                               const char *errors}
  Translate a \ctype{Py_UNICODE} buffer of the given length by
  applying a character mapping \var{table} to it and return the
  resulting Unicode object.  Return \NULL{} when an exception was
  raised by the codec.

  The \var{mapping} table must map Unicode ordinal integers to Unicode
  ordinal integers or None (causing deletion of the character).

  Mapping tables need only provide the \method{__getitem__()}
  interface; dictionaries and sequences work well.  Unmapped character
  ordinals (ones which cause a \exception{LookupError}) are left
  untouched and are copied as-is.
\end{cfuncdesc}

% --- MBCS codecs for Windows --------------------------------------------

These are the MBCS codec APIs. They are currently only available on
Windows and use the Win32 MBCS converters to implement the
conversions.  Note that MBCS (or DBCS) is a class of encodings, not
just one.  The target encoding is defined by the user settings on the
machine running the codec.

\begin{cfuncdesc}{PyObject*}{PyUnicode_DecodeMBCS}{const char *s,
                                               Py_ssize_t size,
                                               const char *errors}
  Create a Unicode object by decoding \var{size} bytes of the MBCS
  encoded string \var{s}.  Return \NULL{} if an exception was
  raised by the codec.
\end{cfuncdesc}

\begin{cfuncdesc}{PyObject*}{PyUnicode_DecodeMBCSStateful}{const char *s,
                                               int size,
                                               const char *errors,
                                               int *consumed}
  If \var{consumed} is \NULL{}, behave like
  \cfunction{PyUnicode_DecodeMBCS()}. If \var{consumed} is not \NULL{},
  \cfunction{PyUnicode_DecodeMBCSStateful()} will not decode trailing lead
  byte and the number of bytes that have been decoded will be stored in
  \var{consumed}.
  \versionadded{2.5}
\end{cfuncdesc}

\begin{cfuncdesc}{PyObject*}{PyUnicode_EncodeMBCS}{const Py_UNICODE *s,
                                               Py_ssize_t size,
                                               const char *errors}
  Encode the \ctype{Py_UNICODE} buffer of the given size using MBCS
  and return a Python string object.  Return \NULL{} if an exception
  was raised by the codec.
\end{cfuncdesc}

\begin{cfuncdesc}{PyObject*}{PyUnicode_AsMBCSString}{PyObject *unicode}
  Encode a Unicode objects using MBCS and return the result as
  Python string object.  Error handling is ``strict''.  Return
  \NULL{} if an exception was raised by the codec.
\end{cfuncdesc}

% --- Methods & Slots ----------------------------------------------------

\subsubsection{Methods and Slot Functions \label{unicodeMethodsAndSlots}}

The following APIs are capable of handling Unicode objects and strings
on input (we refer to them as strings in the descriptions) and return
Unicode objects or integers as appropriate.

They all return \NULL{} or \code{-1} if an exception occurs.

\begin{cfuncdesc}{PyObject*}{PyUnicode_Concat}{PyObject *left,
                                               PyObject *right}
  Concat two strings giving a new Unicode string.
\end{cfuncdesc}

\begin{cfuncdesc}{PyObject*}{PyUnicode_Split}{PyObject *s,
                                              PyObject *sep,
                                              Py_ssize_t maxsplit}
  Split a string giving a list of Unicode strings.  If sep is \NULL{},
  splitting will be done at all whitespace substrings.  Otherwise,
  splits occur at the given separator.  At most \var{maxsplit} splits
  will be done.  If negative, no limit is set.  Separators are not
  included in the resulting list.
\end{cfuncdesc}

\begin{cfuncdesc}{PyObject*}{PyUnicode_Splitlines}{PyObject *s,
                                                   int keepend}
  Split a Unicode string at line breaks, returning a list of Unicode
  strings.  CRLF is considered to be one line break.  If \var{keepend}
  is 0, the Line break characters are not included in the resulting
  strings.
\end{cfuncdesc}

\begin{cfuncdesc}{PyObject*}{PyUnicode_Translate}{PyObject *str,
                                                  PyObject *table,
                                                  const char *errors}
  Translate a string by applying a character mapping table to it and
  return the resulting Unicode object.

  The mapping table must map Unicode ordinal integers to Unicode
  ordinal integers or None (causing deletion of the character).

  Mapping tables need only provide the \method{__getitem__()}
  interface; dictionaries and sequences work well.  Unmapped character
  ordinals (ones which cause a \exception{LookupError}) are left
  untouched and are copied as-is.

  \var{errors} has the usual meaning for codecs. It may be \NULL{}
  which indicates to use the default error handling.
\end{cfuncdesc}

\begin{cfuncdesc}{PyObject*}{PyUnicode_Join}{PyObject *separator,
                                             PyObject *seq}
  Join a sequence of strings using the given separator and return the
  resulting Unicode string.
\end{cfuncdesc}

\begin{cfuncdesc}{int}{PyUnicode_Tailmatch}{PyObject *str,
                                                  PyObject *substr,
                                                  Py_ssize_t start,
                                                  Py_ssize_t end,
                                                  int direction}
  Return 1 if \var{substr} matches \var{str}[\var{start}:\var{end}] at
  the given tail end (\var{direction} == -1 means to do a prefix
  match, \var{direction} == 1 a suffix match), 0 otherwise.
  Return \code{-1} if an error occurred.
\end{cfuncdesc}

\begin{cfuncdesc}{Py_ssize_t}{PyUnicode_Find}{PyObject *str,
                                       PyObject *substr,
                                       Py_ssize_t start,
                                       Py_ssize_t end,
                                       int direction}
  Return the first position of \var{substr} in
  \var{str}[\var{start}:\var{end}] using the given \var{direction}
  (\var{direction} == 1 means to do a forward search,
  \var{direction} == -1 a backward search).  The return value is the
  index of the first match; a value of \code{-1} indicates that no
  match was found, and \code{-2} indicates that an error occurred and
  an exception has been set.
\end{cfuncdesc}

\begin{cfuncdesc}{Py_ssize_t}{PyUnicode_Count}{PyObject *str,
                                        PyObject *substr,
                                        Py_ssize_t start,
                                        Py_ssize_t end}
  Return the number of non-overlapping occurrences of \var{substr} in
  \code{\var{str}[\var{start}:\var{end}]}.  Return \code{-1} if an
  error occurred.
\end{cfuncdesc}

\begin{cfuncdesc}{PyObject*}{PyUnicode_Replace}{PyObject *str,
                                                PyObject *substr,
                                                PyObject *replstr,
                                                Py_ssize_t maxcount}
  Replace at most \var{maxcount} occurrences of \var{substr} in
  \var{str} with \var{replstr} and return the resulting Unicode object.
  \var{maxcount} == -1 means replace all occurrences.
\end{cfuncdesc}

\begin{cfuncdesc}{int}{PyUnicode_Compare}{PyObject *left, PyObject *right}
  Compare two strings and return -1, 0, 1 for less than, equal, and
  greater than, respectively.
\end{cfuncdesc}

\begin{cfuncdesc}{int}{PyUnicode_RichCompare}{PyObject *left, 
                                              PyObject *right, 
                                              int op}

  Rich compare two unicode strings and return one of the following:
  \begin{itemize}
    \item \code{NULL} in case an exception was raised
    \item \constant{Py_True} or \constant{Py_False} for successful comparisons
    \item \constant{Py_NotImplemented} in case the type combination is unknown
  \end{itemize}

   Note that \constant{Py_EQ} and \constant{Py_NE} comparisons can cause a
   \exception{UnicodeWarning} in case the conversion of the arguments to
   Unicode fails with a \exception{UnicodeDecodeError}.

   Possible values for \var{op} are
   \constant{Py_GT}, \constant{Py_GE}, \constant{Py_EQ},
   \constant{Py_NE}, \constant{Py_LT}, and \constant{Py_LE}.
\end{cfuncdesc}

\begin{cfuncdesc}{PyObject*}{PyUnicode_Format}{PyObject *format,
                                              PyObject *args}
  Return a new string object from \var{format} and \var{args}; this
  is analogous to \code{\var{format} \%\ \var{args}}.  The
  \var{args} argument must be a tuple.
\end{cfuncdesc}

\begin{cfuncdesc}{int}{PyUnicode_Contains}{PyObject *container,
                                           PyObject *element}
  Check whether \var{element} is contained in \var{container} and
  return true or false accordingly.

  \var{element} has to coerce to a one element Unicode
  string. \code{-1} is returned if there was an error.
\end{cfuncdesc}

\begin{cfuncdesc}{void}{PyUnicode_InternInPlace}{PyObject **string}
  Intern the argument \var{*string} in place.  The argument must be
  the address of a pointer variable pointing to a Python unicode string
  object.  If there is an existing interned string that is the same as
  \var{*string}, it sets \var{*string} to it (decrementing the
  reference count of the old string object and incrementing the
  reference count of the interned string object), otherwise it leaves
  \var{*string} alone and interns it (incrementing its reference
  count).  (Clarification: even though there is a lot of talk about
  reference counts, think of this function as reference-count-neutral;
  you own the object after the call if and only if you owned it before
  the call.)
\end{cfuncdesc}

\begin{cfuncdesc}{PyObject*}{PyUnicode_InternFromString}{const char *v}
  A combination of \cfunction{PyUnicode_FromString()} and
  \cfunction{PyUnicode_InternInPlace()}, returning either a new unicode
  string object that has been interned, or a new (``owned'') reference to
  an earlier interned string object with the same value.
\end{cfuncdesc}


\subsection{Buffer Objects \label{bufferObjects}}
\sectionauthor{Greg Stein}{gstein@lyra.org}

\obindex{buffer}
Python objects implemented in C can export a group of functions called
the ``buffer\index{buffer interface} interface.''  These functions can
be used by an object to expose its data in a raw, byte-oriented
format. Clients of the object can use the buffer interface to access
the object data directly, without needing to copy it first.

Two examples of objects that support
the buffer interface are strings and arrays. The string object exposes
the character contents in the buffer interface's byte-oriented
form. An array can also expose its contents, but it should be noted
that array elements may be multi-byte values.

An example user of the buffer interface is the file object's
\method{write()} method. Any object that can export a series of bytes
through the buffer interface can be written to a file. There are a
number of format codes to \cfunction{PyArg_ParseTuple()} that operate
against an object's buffer interface, returning data from the target
object.

More information on the buffer interface is provided in the section
``Buffer Object Structures'' (section~\ref{buffer-structs}), under
the description for \ctype{PyBufferProcs}\ttindex{PyBufferProcs}.

A ``buffer object'' is defined in the \file{bufferobject.h} header
(included by \file{Python.h}). These objects look very similar to
string objects at the Python programming level: they support slicing,
indexing, concatenation, and some other standard string
operations. However, their data can come from one of two sources: from
a block of memory, or from another object which exports the buffer
interface.

Buffer objects are useful as a way to expose the data from another
object's buffer interface to the Python programmer. They can also be
used as a zero-copy slicing mechanism. Using their ability to
reference a block of memory, it is possible to expose any data to the
Python programmer quite easily. The memory could be a large, constant
array in a C extension, it could be a raw block of memory for
manipulation before passing to an operating system library, or it
could be used to pass around structured data in its native, in-memory
format.

\begin{ctypedesc}{PyBufferObject}
  This subtype of \ctype{PyObject} represents a buffer object.
\end{ctypedesc}

\begin{cvardesc}{PyTypeObject}{PyBuffer_Type}
  The instance of \ctype{PyTypeObject} which represents the Python
  buffer type; it is the same object as \code{buffer} and 
  \code{types.BufferType} in the Python layer.
  \withsubitem{(in module types)}{\ttindex{BufferType}}.
\end{cvardesc}

\begin{cvardesc}{int}{Py_END_OF_BUFFER}
  This constant may be passed as the \var{size} parameter to
  \cfunction{PyBuffer_FromObject()} or
  \cfunction{PyBuffer_FromReadWriteObject()}.  It indicates that the
  new \ctype{PyBufferObject} should refer to \var{base} object from
  the specified \var{offset} to the end of its exported buffer.  Using
  this enables the caller to avoid querying the \var{base} object for
  its length.
\end{cvardesc}

\begin{cfuncdesc}{int}{PyBuffer_Check}{PyObject *p}
  Return true if the argument has type \cdata{PyBuffer_Type}.
\end{cfuncdesc}

\begin{cfuncdesc}{PyObject*}{PyBuffer_FromObject}{PyObject *base,
                                                  Py_ssize_t offset, Py_ssize_t size}
  Return a new read-only buffer object.  This raises
  \exception{TypeError} if \var{base} doesn't support the read-only
  buffer protocol or doesn't provide exactly one buffer segment, or it
  raises \exception{ValueError} if \var{offset} is less than zero. The
  buffer will hold a reference to the \var{base} object, and the
  buffer's contents will refer to the \var{base} object's buffer
  interface, starting as position \var{offset} and extending for
  \var{size} bytes. If \var{size} is \constant{Py_END_OF_BUFFER}, then
  the new buffer's contents extend to the length of the \var{base}
  object's exported buffer data.
\end{cfuncdesc}

\begin{cfuncdesc}{PyObject*}{PyBuffer_FromReadWriteObject}{PyObject *base,
                                                           Py_ssize_t offset,
                                                           Py_ssize_t size}
  Return a new writable buffer object.  Parameters and exceptions are
  similar to those for \cfunction{PyBuffer_FromObject()}.  If the
  \var{base} object does not export the writeable buffer protocol,
  then \exception{TypeError} is raised.
\end{cfuncdesc}

\begin{cfuncdesc}{PyObject*}{PyBuffer_FromMemory}{void *ptr, Py_ssize_t size}
  Return a new read-only buffer object that reads from a specified
  location in memory, with a specified size.  The caller is
  responsible for ensuring that the memory buffer, passed in as
  \var{ptr}, is not deallocated while the returned buffer object
  exists.  Raises \exception{ValueError} if \var{size} is less than
  zero.  Note that \constant{Py_END_OF_BUFFER} may \emph{not} be
  passed for the \var{size} parameter; \exception{ValueError} will be
  raised in that case.
\end{cfuncdesc}

\begin{cfuncdesc}{PyObject*}{PyBuffer_FromReadWriteMemory}{void *ptr, Py_ssize_t size}
  Similar to \cfunction{PyBuffer_FromMemory()}, but the returned
  buffer is writable.
\end{cfuncdesc}

\begin{cfuncdesc}{PyObject*}{PyBuffer_New}{Py_ssize_t size}
  Return a new writable buffer object that maintains its own memory
  buffer of \var{size} bytes.  \exception{ValueError} is returned if
  \var{size} is not zero or positive.  Note that the memory buffer (as
  returned by \cfunction{PyObject_AsWriteBuffer()}) is not specifically
  aligned.
\end{cfuncdesc}


\subsection{Tuple Objects \label{tupleObjects}}

\obindex{tuple}
\begin{ctypedesc}{PyTupleObject}
  This subtype of \ctype{PyObject} represents a Python tuple object.
\end{ctypedesc}

\begin{cvardesc}{PyTypeObject}{PyTuple_Type}
  This instance of \ctype{PyTypeObject} represents the Python tuple
  type; it is the same object as \code{tuple} and \code{types.TupleType}
  in the Python layer.\withsubitem{(in module types)}{\ttindex{TupleType}}.
\end{cvardesc}

\begin{cfuncdesc}{int}{PyTuple_Check}{PyObject *p}
  Return true if \var{p} is a tuple object or an instance of a subtype
  of the tuple type.
  \versionchanged[Allowed subtypes to be accepted]{2.2}
\end{cfuncdesc}

\begin{cfuncdesc}{int}{PyTuple_CheckExact}{PyObject *p}
  Return true if \var{p} is a tuple object, but not an instance of a
  subtype of the tuple type.
  \versionadded{2.2}
\end{cfuncdesc}

\begin{cfuncdesc}{PyObject*}{PyTuple_New}{Py_ssize_t len}
  Return a new tuple object of size \var{len}, or \NULL{} on failure.
\end{cfuncdesc}

\begin{cfuncdesc}{PyObject*}{PyTuple_Pack}{Py_ssize_t n, \moreargs}
  Return a new tuple object of size \var{n}, or \NULL{} on failure.
  The tuple values are initialized to the subsequent \var{n} C arguments
  pointing to Python objects.  \samp{PyTuple_Pack(2, \var{a}, \var{b})}
  is equivalent to \samp{Py_BuildValue("(OO)", \var{a}, \var{b})}.
  \versionadded{2.4}
\end{cfuncdesc}

\begin{cfuncdesc}{int}{PyTuple_Size}{PyObject *p}
  Take a pointer to a tuple object, and return the size of that
  tuple.
\end{cfuncdesc}

\begin{cfuncdesc}{int}{PyTuple_GET_SIZE}{PyObject *p}
  Return the size of the tuple \var{p}, which must be non-\NULL{} and
  point to a tuple; no error checking is performed.
\end{cfuncdesc}

\begin{cfuncdesc}{PyObject*}{PyTuple_GetItem}{PyObject *p, Py_ssize_t pos}
  Return the object at position \var{pos} in the tuple pointed to by
  \var{p}.  If \var{pos} is out of bounds, return \NULL{} and sets an
  \exception{IndexError} exception.
\end{cfuncdesc}

\begin{cfuncdesc}{PyObject*}{PyTuple_GET_ITEM}{PyObject *p, Py_ssize_t pos}
  Like \cfunction{PyTuple_GetItem()}, but does no checking of its
  arguments.
\end{cfuncdesc}

\begin{cfuncdesc}{PyObject*}{PyTuple_GetSlice}{PyObject *p,
                                               Py_ssize_t low, Py_ssize_t high}
  Take a slice of the tuple pointed to by \var{p} from \var{low} to
  \var{high} and return it as a new tuple.
\end{cfuncdesc}

\begin{cfuncdesc}{int}{PyTuple_SetItem}{PyObject *p,
                                        Py_ssize_t pos, PyObject *o}
  Insert a reference to object \var{o} at position \var{pos} of the
  tuple pointed to by \var{p}. Return \code{0} on success.
  \note{This function ``steals'' a reference to \var{o}.}
\end{cfuncdesc}

\begin{cfuncdesc}{void}{PyTuple_SET_ITEM}{PyObject *p,
                                          Py_ssize_t pos, PyObject *o}
  Like \cfunction{PyTuple_SetItem()}, but does no error checking, and
  should \emph{only} be used to fill in brand new tuples.  \note{This
  function ``steals'' a reference to \var{o}.}
\end{cfuncdesc}

\begin{cfuncdesc}{int}{_PyTuple_Resize}{PyObject **p, Py_ssize_t newsize}
  Can be used to resize a tuple.  \var{newsize} will be the new length
  of the tuple.  Because tuples are \emph{supposed} to be immutable,
  this should only be used if there is only one reference to the
  object.  Do \emph{not} use this if the tuple may already be known to
  some other part of the code.  The tuple will always grow or shrink
  at the end.  Think of this as destroying the old tuple and creating
  a new one, only more efficiently.  Returns \code{0} on success.
  Client code should never assume that the resulting value of
  \code{*\var{p}} will be the same as before calling this function.
  If the object referenced by \code{*\var{p}} is replaced, the
  original \code{*\var{p}} is destroyed.  On failure, returns
  \code{-1} and sets \code{*\var{p}} to \NULL{}, and raises
  \exception{MemoryError} or
  \exception{SystemError}.
  \versionchanged[Removed unused third parameter, \var{last_is_sticky}]{2.2}
\end{cfuncdesc}


\subsection{List Objects \label{listObjects}}

\obindex{list}
\begin{ctypedesc}{PyListObject}
  This subtype of \ctype{PyObject} represents a Python list object.
\end{ctypedesc}

\begin{cvardesc}{PyTypeObject}{PyList_Type}
  This instance of \ctype{PyTypeObject} represents the Python list
  type.  This is the same object as \code{list} and \code{types.ListType}
  in the Python layer.\withsubitem{(in module types)}{\ttindex{ListType}}
\end{cvardesc}

\begin{cfuncdesc}{int}{PyList_Check}{PyObject *p}
  Return true if \var{p} is a list object or an instance of a
  subtype of the list type.
  \versionchanged[Allowed subtypes to be accepted]{2.2}
\end{cfuncdesc}

\begin{cfuncdesc}{int}{PyList_CheckExact}{PyObject *p}
  Return true if \var{p} is a list object, but not an instance of a
  subtype of the list type.
  \versionadded{2.2}
\end{cfuncdesc}

\begin{cfuncdesc}{PyObject*}{PyList_New}{Py_ssize_t len}
  Return a new list of length \var{len} on success, or \NULL{} on
  failure.
  \note{If \var{length} is greater than zero, the returned list object's
        items are set to \code{NULL}.  Thus you cannot use abstract
        API functions such as \cfunction{PySequence_SetItem()} 
        or expose the object to Python code before setting all items to a
        real object with \cfunction{PyList_SetItem()}.}
\end{cfuncdesc}

\begin{cfuncdesc}{Py_ssize_t}{PyList_Size}{PyObject *list}
  Return the length of the list object in \var{list}; this is
  equivalent to \samp{len(\var{list})} on a list object.
  \bifuncindex{len}
\end{cfuncdesc}

\begin{cfuncdesc}{Py_ssize_t}{PyList_GET_SIZE}{PyObject *list}
  Macro form of \cfunction{PyList_Size()} without error checking.
\end{cfuncdesc}

\begin{cfuncdesc}{PyObject*}{PyList_GetItem}{PyObject *list, Py_ssize_t index}
  Return the object at position \var{pos} in the list pointed to by
  \var{p}.  The position must be positive, indexing from the end of the
  list is not supported.  If \var{pos} is out of bounds, return \NULL{}
  and set an \exception{IndexError} exception.
\end{cfuncdesc}

\begin{cfuncdesc}{PyObject*}{PyList_GET_ITEM}{PyObject *list, Py_ssize_t i}
  Macro form of \cfunction{PyList_GetItem()} without error checking.
\end{cfuncdesc}

\begin{cfuncdesc}{int}{PyList_SetItem}{PyObject *list, Py_ssize_t index,
                                       PyObject *item}
  Set the item at index \var{index} in list to \var{item}.  Return
  \code{0} on success or \code{-1} on failure.  \note{This function
  ``steals'' a reference to \var{item} and discards a reference to an
  item already in the list at the affected position.}
\end{cfuncdesc}

\begin{cfuncdesc}{void}{PyList_SET_ITEM}{PyObject *list, Py_ssize_t i,
                                              PyObject *o}
  Macro form of \cfunction{PyList_SetItem()} without error checking.
  This is normally only used to fill in new lists where there is no
  previous content.
  \note{This function ``steals'' a reference to \var{item}, and,
  unlike \cfunction{PyList_SetItem()}, does \emph{not} discard a
  reference to any item that it being replaced; any reference in
  \var{list} at position \var{i} will be leaked.}
\end{cfuncdesc}

\begin{cfuncdesc}{int}{PyList_Insert}{PyObject *list, Py_ssize_t index,
                                      PyObject *item}
  Insert the item \var{item} into list \var{list} in front of index
  \var{index}.  Return \code{0} if successful; return \code{-1} and
  set an exception if unsuccessful.  Analogous to
  \code{\var{list}.insert(\var{index}, \var{item})}.
\end{cfuncdesc}

\begin{cfuncdesc}{int}{PyList_Append}{PyObject *list, PyObject *item}
  Append the object \var{item} at the end of list \var{list}.
  Return \code{0} if successful; return \code{-1} and set an
  exception if unsuccessful.  Analogous to
  \code{\var{list}.append(\var{item})}.
\end{cfuncdesc}

\begin{cfuncdesc}{PyObject*}{PyList_GetSlice}{PyObject *list,
                                              Py_ssize_t low, Py_ssize_t high}
  Return a list of the objects in \var{list} containing the objects
  \emph{between} \var{low} and \var{high}.  Return \NULL{} and set
  an exception if unsuccessful.
  Analogous to \code{\var{list}[\var{low}:\var{high}]}.
\end{cfuncdesc}

\begin{cfuncdesc}{int}{PyList_SetSlice}{PyObject *list,
                                        Py_ssize_t low, Py_ssize_t high,
                                        PyObject *itemlist}
  Set the slice of \var{list} between \var{low} and \var{high} to the
  contents of \var{itemlist}.  Analogous to
  \code{\var{list}[\var{low}:\var{high}] = \var{itemlist}}.
  The \var{itemlist} may be \NULL{}, indicating the assignment
  of an empty list (slice deletion).
  Return \code{0} on success, \code{-1} on failure.
\end{cfuncdesc}

\begin{cfuncdesc}{int}{PyList_Sort}{PyObject *list}
  Sort the items of \var{list} in place.  Return \code{0} on
  success, \code{-1} on failure.  This is equivalent to
  \samp{\var{list}.sort()}.
\end{cfuncdesc}

\begin{cfuncdesc}{int}{PyList_Reverse}{PyObject *list}
  Reverse the items of \var{list} in place.  Return \code{0} on
  success, \code{-1} on failure.  This is the equivalent of
  \samp{\var{list}.reverse()}.
\end{cfuncdesc}

\begin{cfuncdesc}{PyObject*}{PyList_AsTuple}{PyObject *list}
  Return a new tuple object containing the contents of \var{list};
  equivalent to \samp{tuple(\var{list})}.\bifuncindex{tuple}
\end{cfuncdesc}


\section{Mapping Objects \label{mapObjects}}

\obindex{mapping}


\subsection{Dictionary Objects \label{dictObjects}}

\obindex{dictionary}
\begin{ctypedesc}{PyDictObject}
  This subtype of \ctype{PyObject} represents a Python dictionary
  object.
\end{ctypedesc}

\begin{cvardesc}{PyTypeObject}{PyDict_Type}
  This instance of \ctype{PyTypeObject} represents the Python
  dictionary type.  This is exposed to Python programs as
  \code{dict} and \code{types.DictType}.
  \withsubitem{(in module types)}{\ttindex{DictType}\ttindex{DictionaryType}}
\end{cvardesc}

\begin{cfuncdesc}{int}{PyDict_Check}{PyObject *p}
  Return true if \var{p} is a dict object or an instance of a
  subtype of the dict type.
  \versionchanged[Allowed subtypes to be accepted]{2.2}
\end{cfuncdesc}

\begin{cfuncdesc}{int}{PyDict_CheckExact}{PyObject *p}
  Return true if \var{p} is a dict object, but not an instance of a
  subtype of the dict type.
  \versionadded{2.4}
\end{cfuncdesc}

\begin{cfuncdesc}{PyObject*}{PyDict_New}{}
  Return a new empty dictionary, or \NULL{} on failure.
\end{cfuncdesc}

\begin{cfuncdesc}{PyObject*}{PyDictProxy_New}{PyObject *dict}
  Return a proxy object for a mapping which enforces read-only
  behavior.  This is normally used to create a proxy to prevent
  modification of the dictionary for non-dynamic class types.
  \versionadded{2.2}
\end{cfuncdesc}

\begin{cfuncdesc}{void}{PyDict_Clear}{PyObject *p}
  Empty an existing dictionary of all key-value pairs.
\end{cfuncdesc}

\begin{cfuncdesc}{int}{PyDict_Contains}{PyObject *p, PyObject *key}
  Determine if dictionary \var{p} contains \var{key}.  If an item
  in \var{p} is matches \var{key}, return \code{1}, otherwise return
  \code{0}.  On error, return \code{-1}.  This is equivalent to the
  Python expression \samp{\var{key} in \var{p}}.
  \versionadded{2.4}
\end{cfuncdesc}

\begin{cfuncdesc}{PyObject*}{PyDict_Copy}{PyObject *p}
  Return a new dictionary that contains the same key-value pairs as
  \var{p}.
  \versionadded{1.6}
\end{cfuncdesc}

\begin{cfuncdesc}{int}{PyDict_SetItem}{PyObject *p, PyObject *key,
                                       PyObject *val}
  Insert \var{value} into the dictionary \var{p} with a key of
  \var{key}.  \var{key} must be hashable; if it isn't,
  \exception{TypeError} will be raised.
  Return \code{0} on success or \code{-1} on failure.
\end{cfuncdesc}

\begin{cfuncdesc}{int}{PyDict_SetItemString}{PyObject *p,
            const char *key,
            PyObject *val}
  Insert \var{value} into the dictionary \var{p} using \var{key} as a
  key. \var{key} should be a \ctype{char*}.  The key object is created
  using \code{PyString_FromString(\var{key})}. Return \code{0} on
  success or \code{-1} on failure.
  \ttindex{PyString_FromString()}
\end{cfuncdesc}

\begin{cfuncdesc}{int}{PyDict_DelItem}{PyObject *p, PyObject *key}
  Remove the entry in dictionary \var{p} with key \var{key}.
  \var{key} must be hashable; if it isn't, \exception{TypeError} is
  raised.  Return \code{0} on success or \code{-1} on failure.
\end{cfuncdesc}

\begin{cfuncdesc}{int}{PyDict_DelItemString}{PyObject *p, char *key}
  Remove the entry in dictionary \var{p} which has a key specified by
  the string \var{key}.  Return \code{0} on success or \code{-1} on
  failure.
\end{cfuncdesc}

\begin{cfuncdesc}{PyObject*}{PyDict_GetItem}{PyObject *p, PyObject *key}
  Return the object from dictionary \var{p} which has a key
  \var{key}.  Return \NULL{} if the key \var{key} is not present, but
  \emph{without} setting an exception.
\end{cfuncdesc}

\begin{cfuncdesc}{PyObject*}{PyDict_GetItemString}{PyObject *p, const char *key}
  This is the same as \cfunction{PyDict_GetItem()}, but \var{key} is
  specified as a \ctype{char*}, rather than a \ctype{PyObject*}.
\end{cfuncdesc}

\begin{cfuncdesc}{PyObject*}{PyDict_Items}{PyObject *p}
  Return a \ctype{PyListObject} containing all the items from the
  dictionary, as in the dictionary method \method{items()} (see the
  \citetitle[../lib/lib.html]{Python Library Reference}).
\end{cfuncdesc}

\begin{cfuncdesc}{PyObject*}{PyDict_Keys}{PyObject *p}
  Return a \ctype{PyListObject} containing all the keys from the
  dictionary, as in the dictionary method \method{keys()} (see the
  \citetitle[../lib/lib.html]{Python Library Reference}).
\end{cfuncdesc}

\begin{cfuncdesc}{PyObject*}{PyDict_Values}{PyObject *p}
  Return a \ctype{PyListObject} containing all the values from the
  dictionary \var{p}, as in the dictionary method \method{values()}
  (see the \citetitle[../lib/lib.html]{Python Library Reference}).
\end{cfuncdesc}

\begin{cfuncdesc}{Py_ssize_t}{PyDict_Size}{PyObject *p}
  Return the number of items in the dictionary.  This is equivalent
  to \samp{len(\var{p})} on a dictionary.\bifuncindex{len}
\end{cfuncdesc}

\begin{cfuncdesc}{int}{PyDict_Next}{PyObject *p, Py_ssize_t *ppos,
                                    PyObject **pkey, PyObject **pvalue}
  Iterate over all key-value pairs in the dictionary \var{p}.  The
  \ctype{int} referred to by \var{ppos} must be initialized to
  \code{0} prior to the first call to this function to start the
  iteration; the function returns true for each pair in the
  dictionary, and false once all pairs have been reported.  The
  parameters \var{pkey} and \var{pvalue} should either point to
  \ctype{PyObject*} variables that will be filled in with each key and
  value, respectively, or may be \NULL{}.  Any references returned through
  them are borrowed.  \var{ppos} should not be altered during iteration.
  Its value represents offsets within the internal dictionary structure,
  and since the structure is sparse, the offsets are not consecutive.

  For example:

\begin{verbatim}
PyObject *key, *value;
Py_ssize_t pos = 0;

while (PyDict_Next(self->dict, &pos, &key, &value)) {
    /* do something interesting with the values... */
    ...
}
\end{verbatim}

  The dictionary \var{p} should not be mutated during iteration.  It
  is safe (since Python 2.1) to modify the values of the keys as you
  iterate over the dictionary, but only so long as the set of keys
  does not change.  For example:

\begin{verbatim}
PyObject *key, *value;
Py_ssize_t pos = 0;

while (PyDict_Next(self->dict, &pos, &key, &value)) {
    int i = PyInt_AS_LONG(value) + 1;
    PyObject *o = PyInt_FromLong(i);
    if (o == NULL)
        return -1;
    if (PyDict_SetItem(self->dict, key, o) < 0) {
        Py_DECREF(o);
        return -1;
    }
    Py_DECREF(o);
}
\end{verbatim}
\end{cfuncdesc}

\begin{cfuncdesc}{int}{PyDict_Merge}{PyObject *a, PyObject *b, int override}
  Iterate over mapping object \var{b} adding key-value pairs to dictionary
  \var{a}.
  \var{b} may be a dictionary, or any object supporting
  \function{PyMapping_Keys()} and \function{PyObject_GetItem()}.
  If \var{override} is true, existing pairs in \var{a} will
  be replaced if a matching key is found in \var{b}, otherwise pairs
  will only be added if there is not a matching key in \var{a}.
  Return \code{0} on success or \code{-1} if an exception was
  raised.
\versionadded{2.2}
\end{cfuncdesc}

\begin{cfuncdesc}{int}{PyDict_Update}{PyObject *a, PyObject *b}
  This is the same as \code{PyDict_Merge(\var{a}, \var{b}, 1)} in C,
  or \code{\var{a}.update(\var{b})} in Python.  Return \code{0} on
  success or \code{-1} if an exception was raised.
  \versionadded{2.2}
\end{cfuncdesc}

\begin{cfuncdesc}{int}{PyDict_MergeFromSeq2}{PyObject *a, PyObject *seq2,
                                             int override}
  Update or merge into dictionary \var{a}, from the key-value pairs in
  \var{seq2}.  \var{seq2} must be an iterable object producing
  iterable objects of length 2, viewed as key-value pairs.  In case of
  duplicate keys, the last wins if \var{override} is true, else the
  first wins.
  Return \code{0} on success or \code{-1} if an exception
  was raised.
  Equivalent Python (except for the return value):

\begin{verbatim}
def PyDict_MergeFromSeq2(a, seq2, override):
    for key, value in seq2:
        if override or key not in a:
            a[key] = value
\end{verbatim}

  \versionadded{2.2}
\end{cfuncdesc}


\section{Other Objects \label{otherObjects}}

\subsection{Class Objects \label{classObjects}}

\obindex{class}
Note that the class objects described here represent old-style classes,
which will go away in Python 3. When creating new types for extension
modules, you will want to work with type objects (section
\ref{typeObjects}).

\begin{ctypedesc}{PyClassObject}
  The C structure of the objects used to describe built-in classes.
\end{ctypedesc}

\begin{cvardesc}{PyObject*}{PyClass_Type}
  This is the type object for class objects; it is the same object as
  \code{types.ClassType} in the Python layer.
  \withsubitem{(in module types)}{\ttindex{ClassType}}
\end{cvardesc}

\begin{cfuncdesc}{int}{PyClass_Check}{PyObject *o}
  Return true if the object \var{o} is a class object, including
  instances of types derived from the standard class object.  Return
  false in all other cases.
\end{cfuncdesc}

\begin{cfuncdesc}{int}{PyClass_IsSubclass}{PyObject *klass, PyObject *base}
  Return true if \var{klass} is a subclass of \var{base}. Return false in
  all other cases.
\end{cfuncdesc}

\subsection{File Objects \label{fileObjects}}

\obindex{file}
Python's built-in file objects are implemented entirely on the
\ctype{FILE*} support from the C standard library.  This is an
implementation detail and may change in future releases of Python.

\begin{ctypedesc}{PyFileObject}
  This subtype of \ctype{PyObject} represents a Python file object.
\end{ctypedesc}

\begin{cvardesc}{PyTypeObject}{PyFile_Type}
  This instance of \ctype{PyTypeObject} represents the Python file
  type.  This is exposed to Python programs as \code{file} and
  \code{types.FileType}.
  \withsubitem{(in module types)}{\ttindex{FileType}}
\end{cvardesc}

\begin{cfuncdesc}{int}{PyFile_Check}{PyObject *p}
  Return true if its argument is a \ctype{PyFileObject} or a subtype
  of \ctype{PyFileObject}.
  \versionchanged[Allowed subtypes to be accepted]{2.2}
\end{cfuncdesc}

\begin{cfuncdesc}{int}{PyFile_CheckExact}{PyObject *p}
  Return true if its argument is a \ctype{PyFileObject}, but not a
  subtype of \ctype{PyFileObject}.
  \versionadded{2.2}
\end{cfuncdesc}

\begin{cfuncdesc}{PyObject*}{PyFile_FromString}{char *filename, char *mode}
  On success, return a new file object that is opened on the file
  given by \var{filename}, with a file mode given by \var{mode}, where
  \var{mode} has the same semantics as the standard C routine
  \cfunction{fopen()}\ttindex{fopen()}.  On failure, return \NULL{}.
\end{cfuncdesc}

\begin{cfuncdesc}{PyObject*}{PyFile_FromFile}{FILE *fp,
                                              char *name, char *mode,
                                              int (*close)(FILE*)}
  Create a new \ctype{PyFileObject} from the already-open standard C
  file pointer, \var{fp}.  The function \var{close} will be called
  when the file should be closed.  Return \NULL{} on failure.
\end{cfuncdesc}

\begin{cfuncdesc}{FILE*}{PyFile_AsFile}{PyObject *p}
  Return the file object associated with \var{p} as a \ctype{FILE*}.
\end{cfuncdesc}

\begin{cfuncdesc}{PyObject*}{PyFile_GetLine}{PyObject *p, int n}
  Equivalent to \code{\var{p}.readline(\optional{\var{n}})}, this
  function reads one line from the object \var{p}.  \var{p} may be a
  file object or any object with a \method{readline()} method.  If
  \var{n} is \code{0}, exactly one line is read, regardless of the
  length of the line.  If \var{n} is greater than \code{0}, no more
  than \var{n} bytes will be read from the file; a partial line can be
  returned.  In both cases, an empty string is returned if the end of
  the file is reached immediately.  If \var{n} is less than \code{0},
  however, one line is read regardless of length, but
  \exception{EOFError} is raised if the end of the file is reached
  immediately.
  \withsubitem{(built-in exception)}{\ttindex{EOFError}}
\end{cfuncdesc}

\begin{cfuncdesc}{PyObject*}{PyFile_Name}{PyObject *p}
  Return the name of the file specified by \var{p} as a string
  object.
\end{cfuncdesc}

\begin{cfuncdesc}{void}{PyFile_SetBufSize}{PyFileObject *p, int n}
  Available on systems with \cfunction{setvbuf()}\ttindex{setvbuf()}
  only.  This should only be called immediately after file object
  creation.
\end{cfuncdesc}

\begin{cfuncdesc}{int}{PyFile_Encoding}{PyFileObject *p, char *enc}
  Set the file's encoding for Unicode output to \var{enc}. Return
  1 on success and 0 on failure.
  \versionadded{2.3}
\end{cfuncdesc}

\begin{cfuncdesc}{int}{PyFile_SoftSpace}{PyObject *p, int newflag}
  This function exists for internal use by the interpreter.  Set the
  \member{softspace} attribute of \var{p} to \var{newflag} and
  \withsubitem{(file attribute)}{\ttindex{softspace}}return the
  previous value.  \var{p} does not have to be a file object for this
  function to work properly; any object is supported (thought its only
  interesting if the \member{softspace} attribute can be set).  This
  function clears any errors, and will return \code{0} as the previous
  value if the attribute either does not exist or if there were errors
  in retrieving it.  There is no way to detect errors from this
  function, but doing so should not be needed.
\end{cfuncdesc}

\begin{cfuncdesc}{int}{PyFile_WriteObject}{PyObject *obj, PyObject *p,
                                           int flags}
  Write object \var{obj} to file object \var{p}.  The only supported
  flag for \var{flags} is
  \constant{Py_PRINT_RAW}\ttindex{Py_PRINT_RAW}; if given, the
  \function{str()} of the object is written instead of the
  \function{repr()}.  Return \code{0} on success or \code{-1} on
  failure; the appropriate exception will be set.
\end{cfuncdesc}

\begin{cfuncdesc}{int}{PyFile_WriteString}{const char *s, PyObject *p}
  Write string \var{s} to file object \var{p}.  Return \code{0} on
  success or \code{-1} on failure; the appropriate exception will be
  set.
\end{cfuncdesc}


\subsection{Instance Objects \label{instanceObjects}}

\obindex{instance}
There are very few functions specific to instance objects.

\begin{cvardesc}{PyTypeObject}{PyInstance_Type}
  Type object for class instances.
\end{cvardesc}

\begin{cfuncdesc}{int}{PyInstance_Check}{PyObject *obj}
  Return true if \var{obj} is an instance.
\end{cfuncdesc}

\begin{cfuncdesc}{PyObject*}{PyInstance_New}{PyObject *class,
                                             PyObject *arg,
                                             PyObject *kw}
  Create a new instance of a specific class.  The parameters \var{arg}
  and \var{kw} are used as the positional and keyword parameters to
  the object's constructor.
\end{cfuncdesc}

\begin{cfuncdesc}{PyObject*}{PyInstance_NewRaw}{PyObject *class,
                                                PyObject *dict}
  Create a new instance of a specific class without calling its
  constructor.  \var{class} is the class of new object.  The
  \var{dict} parameter will be used as the object's \member{__dict__};
  if \NULL{}, a new dictionary will be created for the instance.
\end{cfuncdesc}


\subsection{Function Objects \label{function-objects}}

\obindex{function}
There are a few functions specific to Python functions.

\begin{ctypedesc}{PyFunctionObject}
  The C structure used for functions.
\end{ctypedesc}

\begin{cvardesc}{PyTypeObject}{PyFunction_Type}
  This is an instance of \ctype{PyTypeObject} and represents the
  Python function type.  It is exposed to Python programmers as
  \code{types.FunctionType}.
  \withsubitem{(in module types)}{\ttindex{MethodType}}
\end{cvardesc}

\begin{cfuncdesc}{int}{PyFunction_Check}{PyObject *o}
  Return true if \var{o} is a function object (has type
  \cdata{PyFunction_Type}).  The parameter must not be \NULL{}.
\end{cfuncdesc}

\begin{cfuncdesc}{PyObject*}{PyFunction_New}{PyObject *code,
                                             PyObject *globals}
  Return a new function object associated with the code object
  \var{code}. \var{globals} must be a dictionary with the global
  variables accessible to the function.

  The function's docstring, name and \var{__module__} are retrieved
  from the code object, the argument defaults and closure are set to
  \NULL{}.
\end{cfuncdesc}

\begin{cfuncdesc}{PyObject*}{PyFunction_GetCode}{PyObject *op}
  Return the code object associated with the function object \var{op}.
\end{cfuncdesc}

\begin{cfuncdesc}{PyObject*}{PyFunction_GetGlobals}{PyObject *op}
  Return the globals dictionary associated with the function object
  \var{op}.
\end{cfuncdesc}

\begin{cfuncdesc}{PyObject*}{PyFunction_GetModule}{PyObject *op}
  Return the \var{__module__} attribute of the function object \var{op}.
  This is normally a string containing the module name, but can be set
  to any other object by Python code.
\end{cfuncdesc}

\begin{cfuncdesc}{PyObject*}{PyFunction_GetDefaults}{PyObject *op}
  Return the argument default values of the function object \var{op}.
  This can be a tuple of arguments or \NULL{}.
\end{cfuncdesc}

\begin{cfuncdesc}{int}{PyFunction_SetDefaults}{PyObject *op,
                                               PyObject *defaults}
  Set the argument default values for the function object \var{op}.
  \var{defaults} must be \var{Py_None} or a tuple.

  Raises \exception{SystemError} and returns \code{-1} on failure.
\end{cfuncdesc}

\begin{cfuncdesc}{PyObject*}{PyFunction_GetClosure}{PyObject *op}
  Return the closure associated with the function object \var{op}.
  This can be \NULL{} or a tuple of cell objects.
\end{cfuncdesc}

\begin{cfuncdesc}{int}{PyFunction_SetClosure}{PyObject *op,
                                              PyObject *closure}
  Set the closure associated with the function object \var{op}.
  \var{closure} must be \var{Py_None} or a tuple of cell objects.

  Raises \exception{SystemError} and returns \code{-1} on failure.
\end{cfuncdesc}


\subsection{Method Objects \label{method-objects}}

\obindex{method}
There are some useful functions that are useful for working with
method objects.

\begin{cvardesc}{PyTypeObject}{PyMethod_Type}
  This instance of \ctype{PyTypeObject} represents the Python method
  type.  This is exposed to Python programs as \code{types.MethodType}.
  \withsubitem{(in module types)}{\ttindex{MethodType}}
\end{cvardesc}

\begin{cfuncdesc}{int}{PyMethod_Check}{PyObject *o}
  Return true if \var{o} is a method object (has type
  \cdata{PyMethod_Type}).  The parameter must not be \NULL{}.
\end{cfuncdesc}

\begin{cfuncdesc}{PyObject*}{PyMethod_New}{PyObject *func,
                                           PyObject *self, PyObject *class}
  Return a new method object, with \var{func} being any callable
  object; this is the function that will be called when the method is
  called.  If this method should be bound to an instance, \var{self}
  should be the instance and \var{class} should be the class of
  \var{self}, otherwise \var{self} should be \NULL{} and \var{class}
  should be the class which provides the unbound method..
\end{cfuncdesc}

\begin{cfuncdesc}{PyObject*}{PyMethod_Class}{PyObject *meth}
  Return the class object from which the method \var{meth} was
  created; if this was created from an instance, it will be the class
  of the instance.
\end{cfuncdesc}

\begin{cfuncdesc}{PyObject*}{PyMethod_GET_CLASS}{PyObject *meth}
  Macro version of \cfunction{PyMethod_Class()} which avoids error
  checking.
\end{cfuncdesc}

\begin{cfuncdesc}{PyObject*}{PyMethod_Function}{PyObject *meth}
  Return the function object associated with the method \var{meth}.
\end{cfuncdesc}

\begin{cfuncdesc}{PyObject*}{PyMethod_GET_FUNCTION}{PyObject *meth}
  Macro version of \cfunction{PyMethod_Function()} which avoids error
  checking.
\end{cfuncdesc}

\begin{cfuncdesc}{PyObject*}{PyMethod_Self}{PyObject *meth}
  Return the instance associated with the method \var{meth} if it is
  bound, otherwise return \NULL{}.
\end{cfuncdesc}

\begin{cfuncdesc}{PyObject*}{PyMethod_GET_SELF}{PyObject *meth}
  Macro version of \cfunction{PyMethod_Self()} which avoids error
  checking.
\end{cfuncdesc}


\subsection{Module Objects \label{moduleObjects}}

\obindex{module}
There are only a few functions special to module objects.

\begin{cvardesc}{PyTypeObject}{PyModule_Type}
  This instance of \ctype{PyTypeObject} represents the Python module
  type.  This is exposed to Python programs as
  \code{types.ModuleType}.
  \withsubitem{(in module types)}{\ttindex{ModuleType}}
\end{cvardesc}

\begin{cfuncdesc}{int}{PyModule_Check}{PyObject *p}
  Return true if \var{p} is a module object, or a subtype of a module
  object.
  \versionchanged[Allowed subtypes to be accepted]{2.2}
\end{cfuncdesc}

\begin{cfuncdesc}{int}{PyModule_CheckExact}{PyObject *p}
  Return true if \var{p} is a module object, but not a subtype of
  \cdata{PyModule_Type}.
  \versionadded{2.2}
\end{cfuncdesc}

\begin{cfuncdesc}{PyObject*}{PyModule_New}{const char *name}
  Return a new module object with the \member{__name__} attribute set
  to \var{name}.  Only the module's \member{__doc__} and
  \member{__name__} attributes are filled in; the caller is
  responsible for providing a \member{__file__} attribute.
  \withsubitem{(module attribute)}{
    \ttindex{__name__}\ttindex{__doc__}\ttindex{__file__}}
\end{cfuncdesc}

\begin{cfuncdesc}{PyObject*}{PyModule_GetDict}{PyObject *module}
  Return the dictionary object that implements \var{module}'s
  namespace; this object is the same as the \member{__dict__}
  attribute of the module object.  This function never fails.
  \withsubitem{(module attribute)}{\ttindex{__dict__}}
  It is recommended extensions use other \cfunction{PyModule_*()}
  and \cfunction{PyObject_*()} functions rather than directly
  manipulate a module's \member{__dict__}.
\end{cfuncdesc}

\begin{cfuncdesc}{char*}{PyModule_GetName}{PyObject *module}
  Return \var{module}'s \member{__name__} value.  If the module does
  not provide one, or if it is not a string, \exception{SystemError}
  is raised and \NULL{} is returned.
  \withsubitem{(module attribute)}{\ttindex{__name__}}
  \withsubitem{(built-in exception)}{\ttindex{SystemError}}
\end{cfuncdesc}

\begin{cfuncdesc}{char*}{PyModule_GetFilename}{PyObject *module}
  Return the name of the file from which \var{module} was loaded using
  \var{module}'s \member{__file__} attribute.  If this is not defined,
  or if it is not a string, raise \exception{SystemError} and return
  \NULL{}.
  \withsubitem{(module attribute)}{\ttindex{__file__}}
  \withsubitem{(built-in exception)}{\ttindex{SystemError}}
\end{cfuncdesc}

\begin{cfuncdesc}{int}{PyModule_AddObject}{PyObject *module,
                                           const char *name, PyObject *value}
  Add an object to \var{module} as \var{name}.  This is a convenience
  function which can be used from the module's initialization
  function.  This steals a reference to \var{value}.  Return
  \code{-1} on error, \code{0} on success.
  \versionadded{2.0}
\end{cfuncdesc}

\begin{cfuncdesc}{int}{PyModule_AddIntConstant}{PyObject *module,
                                                const char *name, long value}
  Add an integer constant to \var{module} as \var{name}.  This
  convenience function can be used from the module's initialization
  function. Return \code{-1} on error, \code{0} on success.
  \versionadded{2.0}
\end{cfuncdesc}

\begin{cfuncdesc}{int}{PyModule_AddStringConstant}{PyObject *module,
                                                   const char *name, const char *value}
  Add a string constant to \var{module} as \var{name}.  This
  convenience function can be used from the module's initialization
  function.  The string \var{value} must be null-terminated.  Return
  \code{-1} on error, \code{0} on success.
  \versionadded{2.0}
\end{cfuncdesc}


\subsection{Iterator Objects \label{iterator-objects}}

Python provides two general-purpose iterator objects.  The first, a
sequence iterator, works with an arbitrary sequence supporting the
\method{__getitem__()} method.  The second works with a callable
object and a sentinel value, calling the callable for each item in the
sequence, and ending the iteration when the sentinel value is
returned.

\begin{cvardesc}{PyTypeObject}{PySeqIter_Type}
  Type object for iterator objects returned by
  \cfunction{PySeqIter_New()} and the one-argument form of the
  \function{iter()} built-in function for built-in sequence types.
  \versionadded{2.2}
\end{cvardesc}

\begin{cfuncdesc}{int}{PySeqIter_Check}{op}
  Return true if the type of \var{op} is \cdata{PySeqIter_Type}.
  \versionadded{2.2}
\end{cfuncdesc}

\begin{cfuncdesc}{PyObject*}{PySeqIter_New}{PyObject *seq}
  Return an iterator that works with a general sequence object,
  \var{seq}.  The iteration ends when the sequence raises
  \exception{IndexError} for the subscripting operation.
  \versionadded{2.2}
\end{cfuncdesc}

\begin{cvardesc}{PyTypeObject}{PyCallIter_Type}
  Type object for iterator objects returned by
  \cfunction{PyCallIter_New()} and the two-argument form of the
  \function{iter()} built-in function.
  \versionadded{2.2}
\end{cvardesc}

\begin{cfuncdesc}{int}{PyCallIter_Check}{op}
  Return true if the type of \var{op} is \cdata{PyCallIter_Type}.
  \versionadded{2.2}
\end{cfuncdesc}

\begin{cfuncdesc}{PyObject*}{PyCallIter_New}{PyObject *callable,
                                             PyObject *sentinel}
  Return a new iterator.  The first parameter, \var{callable}, can be
  any Python callable object that can be called with no parameters;
  each call to it should return the next item in the iteration.  When
  \var{callable} returns a value equal to \var{sentinel}, the
  iteration will be terminated.
  \versionadded{2.2}
\end{cfuncdesc}


\subsection{Descriptor Objects \label{descriptor-objects}}

``Descriptors'' are objects that describe some attribute of an object.
They are found in the dictionary of type objects.

\begin{cvardesc}{PyTypeObject}{PyProperty_Type}
  The type object for the built-in descriptor types.
  \versionadded{2.2}
\end{cvardesc}

\begin{cfuncdesc}{PyObject*}{PyDescr_NewGetSet}{PyTypeObject *type,
					        struct PyGetSetDef *getset}
  \versionadded{2.2}
\end{cfuncdesc}

\begin{cfuncdesc}{PyObject*}{PyDescr_NewMember}{PyTypeObject *type,
					        struct PyMemberDef *meth}
  \versionadded{2.2}
\end{cfuncdesc}

\begin{cfuncdesc}{PyObject*}{PyDescr_NewMethod}{PyTypeObject *type,
                                                struct PyMethodDef *meth}
  \versionadded{2.2}
\end{cfuncdesc}

\begin{cfuncdesc}{PyObject*}{PyDescr_NewWrapper}{PyTypeObject *type,
						 struct wrapperbase *wrapper,
                                                 void *wrapped}
  \versionadded{2.2}
\end{cfuncdesc}

\begin{cfuncdesc}{PyObject*}{PyDescr_NewClassMethod}{PyTypeObject *type,
						     PyMethodDef *method}
  \versionadded{2.3}
\end{cfuncdesc}

\begin{cfuncdesc}{int}{PyDescr_IsData}{PyObject *descr}
  Return true if the descriptor objects \var{descr} describes a data
  attribute, or false if it describes a method.  \var{descr} must be a
  descriptor object; there is no error checking.
  \versionadded{2.2}
\end{cfuncdesc}

\begin{cfuncdesc}{PyObject*}{PyWrapper_New}{PyObject *, PyObject *}
  \versionadded{2.2}
\end{cfuncdesc}


\subsection{Slice Objects \label{slice-objects}}

\begin{cvardesc}{PyTypeObject}{PySlice_Type}
  The type object for slice objects.  This is the same as
  \code{slice} and \code{types.SliceType}.
  \withsubitem{(in module types)}{\ttindex{SliceType}}
\end{cvardesc}

\begin{cfuncdesc}{int}{PySlice_Check}{PyObject *ob}
  Return true if \var{ob} is a slice object; \var{ob} must not be
  \NULL{}.
\end{cfuncdesc}

\begin{cfuncdesc}{PyObject*}{PySlice_New}{PyObject *start, PyObject *stop,
                                          PyObject *step}
  Return a new slice object with the given values.  The \var{start},
  \var{stop}, and \var{step} parameters are used as the values of the
  slice object attributes of the same names.  Any of the values may be
  \NULL{}, in which case the \code{None} will be used for the
  corresponding attribute.  Return \NULL{} if the new object could
  not be allocated.
\end{cfuncdesc}

\begin{cfuncdesc}{int}{PySlice_GetIndices}{PySliceObject *slice, Py_ssize_t length,
                                           Py_ssize_t *start, Py_ssize_t *stop, Py_ssize_t *step}
Retrieve the start, stop and step indices from the slice object
\var{slice}, assuming a sequence of length \var{length}. Treats
indices greater than \var{length} as errors.

Returns 0 on success and -1 on error with no exception set (unless one
of the indices was not \constant{None} and failed to be converted to
an integer, in which case -1 is returned with an exception set).

You probably do not want to use this function.  If you want to use
slice objects in versions of Python prior to 2.3, you would probably
do well to incorporate the source of \cfunction{PySlice_GetIndicesEx},
suitably renamed, in the source of your extension.
\end{cfuncdesc}

\begin{cfuncdesc}{int}{PySlice_GetIndicesEx}{PySliceObject *slice, Py_ssize_t length,
                                             Py_ssize_t *start, Py_ssize_t *stop, Py_ssize_t *step,
                                             Py_ssize_t *slicelength}
Usable replacement for \cfunction{PySlice_GetIndices}.  Retrieve the
start, stop, and step indices from the slice object \var{slice}
assuming a sequence of length \var{length}, and store the length of
the slice in \var{slicelength}.  Out of bounds indices are clipped in
a manner consistent with the handling of normal slices.

Returns 0 on success and -1 on error with exception set.

\versionadded{2.3}
\end{cfuncdesc}


\subsection{Weak Reference Objects \label{weakref-objects}}

Python supports \emph{weak references} as first-class objects.  There
are two specific object types which directly implement weak
references.  The first is a simple reference object, and the second
acts as a proxy for the original object as much as it can.

\begin{cfuncdesc}{int}{PyWeakref_Check}{ob}
  Return true if \var{ob} is either a reference or proxy object.
  \versionadded{2.2}
\end{cfuncdesc}

\begin{cfuncdesc}{int}{PyWeakref_CheckRef}{ob}
  Return true if \var{ob} is a reference object.
  \versionadded{2.2}
\end{cfuncdesc}

\begin{cfuncdesc}{int}{PyWeakref_CheckProxy}{ob}
  Return true if \var{ob} is a proxy object.
  \versionadded{2.2}
\end{cfuncdesc}

\begin{cfuncdesc}{PyObject*}{PyWeakref_NewRef}{PyObject *ob,
                                               PyObject *callback}
  Return a weak reference object for the object \var{ob}.  This will
  always return a new reference, but is not guaranteed to create a new
  object; an existing reference object may be returned.  The second
  parameter, \var{callback}, can be a callable object that receives
  notification when \var{ob} is garbage collected; it should accept a
  single parameter, which will be the weak reference object itself.
  \var{callback} may also be \code{None} or \NULL{}.  If \var{ob}
  is not a weakly-referencable object, or if \var{callback} is not
  callable, \code{None}, or \NULL{}, this will return \NULL{} and
  raise \exception{TypeError}.
  \versionadded{2.2}
\end{cfuncdesc}

\begin{cfuncdesc}{PyObject*}{PyWeakref_NewProxy}{PyObject *ob,
                                                 PyObject *callback}
  Return a weak reference proxy object for the object \var{ob}.  This
  will always return a new reference, but is not guaranteed to create
  a new object; an existing proxy object may be returned.  The second
  parameter, \var{callback}, can be a callable object that receives
  notification when \var{ob} is garbage collected; it should accept a
  single parameter, which will be the weak reference object itself.
  \var{callback} may also be \code{None} or \NULL{}.  If \var{ob} is not
  a weakly-referencable object, or if \var{callback} is not callable,
  \code{None}, or \NULL{}, this will return \NULL{} and raise
  \exception{TypeError}.
  \versionadded{2.2}
\end{cfuncdesc}

\begin{cfuncdesc}{PyObject*}{PyWeakref_GetObject}{PyObject *ref}
  Return the referenced object from a weak reference, \var{ref}.  If
  the referent is no longer live, returns \code{None}.
  \versionadded{2.2}
\end{cfuncdesc}

\begin{cfuncdesc}{PyObject*}{PyWeakref_GET_OBJECT}{PyObject *ref}
  Similar to \cfunction{PyWeakref_GetObject()}, but implemented as a
  macro that does no error checking.
  \versionadded{2.2}
\end{cfuncdesc}


\subsection{CObjects \label{cObjects}}

\obindex{CObject}
Refer to \emph{Extending and Embedding the Python Interpreter},
section~1.12, ``Providing a C API for an Extension Module,'' for more
information on using these objects.


\begin{ctypedesc}{PyCObject}
  This subtype of \ctype{PyObject} represents an opaque value, useful
  for C extension modules who need to pass an opaque value (as a
  \ctype{void*} pointer) through Python code to other C code.  It is
  often used to make a C function pointer defined in one module
  available to other modules, so the regular import mechanism can be
  used to access C APIs defined in dynamically loaded modules.
\end{ctypedesc}

\begin{cfuncdesc}{int}{PyCObject_Check}{PyObject *p}
  Return true if its argument is a \ctype{PyCObject}.
\end{cfuncdesc}

\begin{cfuncdesc}{PyObject*}{PyCObject_FromVoidPtr}{void* cobj,
                                                    void (*destr)(void *)}
  Create a \ctype{PyCObject} from the \code{void *}\var{cobj}.  The
  \var{destr} function will be called when the object is reclaimed,
  unless it is \NULL{}.
\end{cfuncdesc}

\begin{cfuncdesc}{PyObject*}{PyCObject_FromVoidPtrAndDesc}{void* cobj,
	                          void* desc, void (*destr)(void *, void *)}
  Create a \ctype{PyCObject} from the \ctype{void *}\var{cobj}.  The
  \var{destr} function will be called when the object is reclaimed.
  The \var{desc} argument can be used to pass extra callback data for
  the destructor function.
\end{cfuncdesc}

\begin{cfuncdesc}{void*}{PyCObject_AsVoidPtr}{PyObject* self}
  Return the object \ctype{void *} that the \ctype{PyCObject}
  \var{self} was created with.
\end{cfuncdesc}

\begin{cfuncdesc}{void*}{PyCObject_GetDesc}{PyObject* self}
  Return the description \ctype{void *} that the \ctype{PyCObject}
  \var{self} was created with.
\end{cfuncdesc}

\begin{cfuncdesc}{int}{PyCObject_SetVoidPtr}{PyObject* self, void* cobj}
  Set the void pointer inside \var{self} to \var{cobj}.
  The \ctype{PyCObject} must not have an associated destructor.
  Return true on success, false on failure.
\end{cfuncdesc}


\subsection{Cell Objects \label{cell-objects}}

``Cell'' objects are used to implement variables referenced by
multiple scopes.  For each such variable, a cell object is created to
store the value; the local variables of each stack frame that
references the value contains a reference to the cells from outer
scopes which also use that variable.  When the value is accessed, the
value contained in the cell is used instead of the cell object
itself.  This de-referencing of the cell object requires support from
the generated byte-code; these are not automatically de-referenced
when accessed.  Cell objects are not likely to be useful elsewhere.

\begin{ctypedesc}{PyCellObject}
  The C structure used for cell objects.
\end{ctypedesc}

\begin{cvardesc}{PyTypeObject}{PyCell_Type}
  The type object corresponding to cell objects.
\end{cvardesc}

\begin{cfuncdesc}{int}{PyCell_Check}{ob}
  Return true if \var{ob} is a cell object; \var{ob} must not be
  \NULL{}.
\end{cfuncdesc}

\begin{cfuncdesc}{PyObject*}{PyCell_New}{PyObject *ob}
  Create and return a new cell object containing the value \var{ob}.
  The parameter may be \NULL{}.
\end{cfuncdesc}

\begin{cfuncdesc}{PyObject*}{PyCell_Get}{PyObject *cell}
  Return the contents of the cell \var{cell}.
\end{cfuncdesc}

\begin{cfuncdesc}{PyObject*}{PyCell_GET}{PyObject *cell}
  Return the contents of the cell \var{cell}, but without checking
  that \var{cell} is non-\NULL{} and a cell object.
\end{cfuncdesc}

\begin{cfuncdesc}{int}{PyCell_Set}{PyObject *cell, PyObject *value}
  Set the contents of the cell object \var{cell} to \var{value}.  This
  releases the reference to any current content of the cell.
  \var{value} may be \NULL{}.  \var{cell} must be non-\NULL{}; if it is
  not a cell object, \code{-1} will be returned.  On success, \code{0}
  will be returned.
\end{cfuncdesc}

\begin{cfuncdesc}{void}{PyCell_SET}{PyObject *cell, PyObject *value}
  Sets the value of the cell object \var{cell} to \var{value}.  No
  reference counts are adjusted, and no checks are made for safety;
  \var{cell} must be non-\NULL{} and must be a cell object.
\end{cfuncdesc}


\subsection{Generator Objects \label{gen-objects}}

Generator objects are what Python uses to implement generator iterators.
They are normally created by iterating over a function that yields values,
rather than explicitly calling \cfunction{PyGen_New}.

\begin{ctypedesc}{PyGenObject}
  The C structure used for generator objects.
\end{ctypedesc}

\begin{cvardesc}{PyTypeObject}{PyGen_Type}
  The type object corresponding to generator objects
\end{cvardesc}

\begin{cfuncdesc}{int}{PyGen_Check}{ob}
  Return true if \var{ob} is a generator object; \var{ob} must not be
  \NULL{}.
\end{cfuncdesc}

\begin{cfuncdesc}{int}{PyGen_CheckExact}{ob}
  Return true if \var{ob}'s type is \var{PyGen_Type}
  is a generator object; \var{ob} must not be
  \NULL{}.
\end{cfuncdesc}

\begin{cfuncdesc}{PyObject*}{PyGen_New}{PyFrameObject *frame}
  Create and return a new generator object based on the \var{frame} object.
  A reference to \var{frame} is stolen by this function.
  The parameter must not be \NULL{}.
\end{cfuncdesc}


\subsection{DateTime Objects \label{datetime-objects}}

Various date and time objects are supplied by the \module{datetime}
module.  Before using any of these functions, the header file
\file{datetime.h} must be included in your source (note that this is
not included by \file{Python.h}), and the macro
\cfunction{PyDateTime_IMPORT} must be invoked.  The macro puts a
pointer to a C structure into a static variable, 
\code{PyDateTimeAPI}, that is used by the following macros.

Type-check macros:

\begin{cfuncdesc}{int}{PyDate_Check}{PyObject *ob}
  Return true if \var{ob} is of type \cdata{PyDateTime_DateType} or
  a subtype of \cdata{PyDateTime_DateType}.  \var{ob} must not be
  \NULL{}.
  \versionadded{2.4}
\end{cfuncdesc}

\begin{cfuncdesc}{int}{PyDate_CheckExact}{PyObject *ob}
  Return true if \var{ob} is of type \cdata{PyDateTime_DateType}.
  \var{ob} must not be \NULL{}.
  \versionadded{2.4}
\end{cfuncdesc}

\begin{cfuncdesc}{int}{PyDateTime_Check}{PyObject *ob}
  Return true if \var{ob} is of type \cdata{PyDateTime_DateTimeType} or
  a subtype of \cdata{PyDateTime_DateTimeType}.  \var{ob} must not be
  \NULL{}.
  \versionadded{2.4}
\end{cfuncdesc}

\begin{cfuncdesc}{int}{PyDateTime_CheckExact}{PyObject *ob}
  Return true if \var{ob} is of type \cdata{PyDateTime_DateTimeType}.
  \var{ob} must not be \NULL{}.
  \versionadded{2.4}
\end{cfuncdesc}

\begin{cfuncdesc}{int}{PyTime_Check}{PyObject *ob}
  Return true if \var{ob} is of type \cdata{PyDateTime_TimeType} or
  a subtype of \cdata{PyDateTime_TimeType}.  \var{ob} must not be
  \NULL{}.
  \versionadded{2.4}
\end{cfuncdesc}

\begin{cfuncdesc}{int}{PyTime_CheckExact}{PyObject *ob}
  Return true if \var{ob} is of type \cdata{PyDateTime_TimeType}.
  \var{ob} must not be \NULL{}.
  \versionadded{2.4}
\end{cfuncdesc}

\begin{cfuncdesc}{int}{PyDelta_Check}{PyObject *ob}
  Return true if \var{ob} is of type \cdata{PyDateTime_DeltaType} or
  a subtype of \cdata{PyDateTime_DeltaType}.  \var{ob} must not be
  \NULL{}.
  \versionadded{2.4}
\end{cfuncdesc}

\begin{cfuncdesc}{int}{PyDelta_CheckExact}{PyObject *ob}
  Return true if \var{ob} is of type \cdata{PyDateTime_DeltaType}.
  \var{ob} must not be \NULL{}.
  \versionadded{2.4}
\end{cfuncdesc}

\begin{cfuncdesc}{int}{PyTZInfo_Check}{PyObject *ob}
  Return true if \var{ob} is of type \cdata{PyDateTime_TZInfoType} or
  a subtype of \cdata{PyDateTime_TZInfoType}.  \var{ob} must not be
  \NULL{}.
  \versionadded{2.4}
\end{cfuncdesc}

\begin{cfuncdesc}{int}{PyTZInfo_CheckExact}{PyObject *ob}
  Return true if \var{ob} is of type \cdata{PyDateTime_TZInfoType}.
  \var{ob} must not be \NULL{}.
  \versionadded{2.4}
\end{cfuncdesc}

Macros to create objects:

\begin{cfuncdesc}{PyObject*}{PyDate_FromDate}{int year, int month, int day}
  Return a \code{datetime.date} object with the specified year, month
  and day.
  \versionadded{2.4}
\end{cfuncdesc}

\begin{cfuncdesc}{PyObject*}{PyDateTime_FromDateAndTime}{int year, int month,
        int day, int hour, int minute, int second, int usecond}
  Return a \code{datetime.datetime} object with the specified year, month,
  day, hour, minute, second and microsecond.
  \versionadded{2.4}
\end{cfuncdesc}

\begin{cfuncdesc}{PyObject*}{PyTime_FromTime}{int hour, int minute,
        int second, int usecond}
  Return a \code{datetime.time} object with the specified hour, minute,
  second and microsecond.
  \versionadded{2.4}
\end{cfuncdesc}

\begin{cfuncdesc}{PyObject*}{PyDelta_FromDSU}{int days, int seconds,
        int useconds}
  Return a \code{datetime.timedelta} object representing the given number
  of days, seconds and microseconds.  Normalization is performed so that
  the resulting number of microseconds and seconds lie in the ranges
  documented for \code{datetime.timedelta} objects.
  \versionadded{2.4}
\end{cfuncdesc}

Macros to extract fields from date objects.  The argument must be an
instance of \cdata{PyDateTime_Date}, including subclasses (such as
\cdata{PyDateTime_DateTime}).  The argument must not be \NULL{}, and
the type is not checked:

\begin{cfuncdesc}{int}{PyDateTime_GET_YEAR}{PyDateTime_Date *o}
  Return the year, as a positive int.
  \versionadded{2.4}
\end{cfuncdesc}

\begin{cfuncdesc}{int}{PyDateTime_GET_MONTH}{PyDateTime_Date *o}
  Return the month, as an int from 1 through 12.
  \versionadded{2.4}
\end{cfuncdesc}

\begin{cfuncdesc}{int}{PyDateTime_GET_DAY}{PyDateTime_Date *o}
  Return the day, as an int from 1 through 31.
  \versionadded{2.4}
\end{cfuncdesc}

Macros to extract fields from datetime objects.  The argument must be an
instance of \cdata{PyDateTime_DateTime}, including subclasses.
The argument must not be \NULL{}, and the type is not checked:

\begin{cfuncdesc}{int}{PyDateTime_DATE_GET_HOUR}{PyDateTime_DateTime *o}
  Return the hour, as an int from 0 through 23.
  \versionadded{2.4}
\end{cfuncdesc}

\begin{cfuncdesc}{int}{PyDateTime_DATE_GET_MINUTE}{PyDateTime_DateTime *o}
  Return the minute, as an int from 0 through 59.
  \versionadded{2.4}
\end{cfuncdesc}

\begin{cfuncdesc}{int}{PyDateTime_DATE_GET_SECOND}{PyDateTime_DateTime *o}
  Return the second, as an int from 0 through 59.
  \versionadded{2.4}
\end{cfuncdesc}

\begin{cfuncdesc}{int}{PyDateTime_DATE_GET_MICROSECOND}{PyDateTime_DateTime *o}
  Return the microsecond, as an int from 0 through 999999.
  \versionadded{2.4}
\end{cfuncdesc}

Macros to extract fields from time objects.  The argument must be an
instance of \cdata{PyDateTime_Time}, including subclasses.
The argument must not be \NULL{}, and the type is not checked:

\begin{cfuncdesc}{int}{PyDateTime_TIME_GET_HOUR}{PyDateTime_Time *o}
  Return the hour, as an int from 0 through 23.
  \versionadded{2.4}
\end{cfuncdesc}

\begin{cfuncdesc}{int}{PyDateTime_TIME_GET_MINUTE}{PyDateTime_Time *o}
  Return the minute, as an int from 0 through 59.
  \versionadded{2.4}
\end{cfuncdesc}

\begin{cfuncdesc}{int}{PyDateTime_TIME_GET_SECOND}{PyDateTime_Time *o}
  Return the second, as an int from 0 through 59.
  \versionadded{2.4}
\end{cfuncdesc}

\begin{cfuncdesc}{int}{PyDateTime_TIME_GET_MICROSECOND}{PyDateTime_Time *o}
  Return the microsecond, as an int from 0 through 999999.
  \versionadded{2.4}
\end{cfuncdesc}

Macros for the convenience of modules implementing the DB API:

\begin{cfuncdesc}{PyObject*}{PyDateTime_FromTimestamp}{PyObject *args}
  Create and return a new \code{datetime.datetime} object given an argument
  tuple suitable for passing to \code{datetime.datetime.fromtimestamp()}.
  \versionadded{2.4}
\end{cfuncdesc}

\begin{cfuncdesc}{PyObject*}{PyDate_FromTimestamp}{PyObject *args}
  Create and return a new \code{datetime.date} object given an argument
  tuple suitable for passing to \code{datetime.date.fromtimestamp()}.
  \versionadded{2.4}
\end{cfuncdesc}


\subsection{Set Objects \label{setObjects}}
\sectionauthor{Raymond D. Hettinger}{python@rcn.com}

\obindex{set}
\obindex{frozenset}
\versionadded{2.5}

This section details the public API for \class{set} and \class{frozenset}
objects.  Any functionality not listed below is best accessed using the
either the abstract object protocol (including
\cfunction{PyObject_CallMethod()}, \cfunction{PyObject_RichCompareBool()},
\cfunction{PyObject_Hash()}, \cfunction{PyObject_Repr()},
\cfunction{PyObject_IsTrue()}, \cfunction{PyObject_Print()}, and
\cfunction{PyObject_GetIter()})
or the abstract number protocol (including
\cfunction{PyNumber_And()}, \cfunction{PyNumber_Subtract()},
\cfunction{PyNumber_Or()}, \cfunction{PyNumber_Xor()},
\cfunction{PyNumber_InPlaceAnd()}, \cfunction{PyNumber_InPlaceSubtract()},
\cfunction{PyNumber_InPlaceOr()}, and \cfunction{PyNumber_InPlaceXor()}).

\begin{ctypedesc}{PySetObject}
  This subtype of \ctype{PyObject} is used to hold the internal data for
  both \class{set} and \class{frozenset} objects.  It is like a
  \ctype{PyDictObject} in that it is a fixed size for small sets
  (much like tuple storage) and will point to a separate, variable sized
  block of memory for medium and large sized sets (much like list storage).
  None of the fields of this structure should be considered public and
  are subject to change.  All access should be done through the
  documented API rather than by manipulating the values in the structure.

\end{ctypedesc}

\begin{cvardesc}{PyTypeObject}{PySet_Type}
  This is an instance of \ctype{PyTypeObject} representing the Python
  \class{set} type.
\end{cvardesc}

\begin{cvardesc}{PyTypeObject}{PyFrozenSet_Type}
  This is an instance of \ctype{PyTypeObject} representing the Python
  \class{frozenset} type.
\end{cvardesc}


The following type check macros work on pointers to any Python object.
Likewise, the constructor functions work with any iterable Python object.

\begin{cfuncdesc}{int}{PyAnySet_Check}{PyObject *p}
  Return true if \var{p} is a \class{set} object, a \class{frozenset}
  object, or an instance of a subtype.
\end{cfuncdesc}

\begin{cfuncdesc}{int}{PyAnySet_CheckExact}{PyObject *p}
  Return true if \var{p} is a \class{set} object or a \class{frozenset}
  object but not an instance of a subtype.
\end{cfuncdesc}

\begin{cfuncdesc}{int}{PyFrozenSet_CheckExact}{PyObject *p}
  Return true if \var{p} is a \class{frozenset} object
  but not an instance of a subtype.
\end{cfuncdesc}

\begin{cfuncdesc}{PyObject*}{PySet_New}{PyObject *iterable}
  Return a new \class{set} containing objects returned by the
  \var{iterable}.  The \var{iterable} may be \NULL{} to create a
  new empty set.  Return the new set on success or \NULL{} on
  failure.  Raise \exception{TypeError} if \var{iterable} is
  not actually iterable.  The constructor is also useful for
  copying a set (\code{c=set(s)}).
\end{cfuncdesc}

\begin{cfuncdesc}{PyObject*}{PyFrozenSet_New}{PyObject *iterable}
  Return a new \class{frozenset} containing objects returned by the
  \var{iterable}.  The \var{iterable} may be \NULL{} to create a
  new empty frozenset.  Return the new set on success or \NULL{} on
  failure.  Raise \exception{TypeError} if \var{iterable} is
  not actually iterable.
\end{cfuncdesc}


The following functions and macros are available for instances of
\class{set} or \class{frozenset} or instances of their subtypes.

\begin{cfuncdesc}{int}{PySet_Size}{PyObject *anyset}
  Return the length of a \class{set} or \class{frozenset} object.
  Equivalent to \samp{len(\var{anyset})}.  Raises a
  \exception{PyExc_SystemError} if \var{anyset} is not a \class{set},
  \class{frozenset}, or an instance of a subtype.
  \bifuncindex{len}
\end{cfuncdesc}

\begin{cfuncdesc}{int}{PySet_GET_SIZE}{PyObject *anyset}
  Macro form of \cfunction{PySet_Size()} without error checking.
\end{cfuncdesc}

\begin{cfuncdesc}{int}{PySet_Contains}{PyObject *anyset, PyObject *key}
  Return 1 if found, 0 if not found, and -1 if an error is
  encountered.  Unlike the Python \method{__contains__()} method, this
  function does not automatically convert unhashable sets into temporary
  frozensets.  Raise a \exception{TypeError} if the \var{key} is unhashable.
  Raise \exception{PyExc_SystemError} if \var{anyset} is not a \class{set},
  \class{frozenset}, or an instance of a subtype.
\end{cfuncdesc}

The following functions are available for instances of \class{set} or
its subtypes but not for instances of \class{frozenset} or its subtypes.

\begin{cfuncdesc}{int}{PySet_Add}{PyObject *set, PyObject *key}
  Add \var{key} to a \class{set} instance.  Does not apply to
  \class{frozenset} instances.  Return 0 on success or -1 on failure.
  Raise a \exception{TypeError} if the \var{key} is unhashable.
  Raise a \exception{MemoryError} if there is no room to grow.
  Raise a \exception{SystemError} if \var{set} is an not an instance
  of \class{set} or its subtype.
\end{cfuncdesc}

\begin{cfuncdesc}{int}{PySet_Discard}{PyObject *set, PyObject *key}
  Return 1 if found and removed, 0 if not found (no action taken),
  and -1 if an error is encountered.  Does not raise \exception{KeyError}
  for missing keys.  Raise a \exception{TypeError} if the \var{key} is
  unhashable.  Unlike the Python \method{discard()} method, this function
  does not automatically convert unhashable sets into temporary frozensets.
  Raise \exception{PyExc_SystemError} if \var{set} is an not an instance
  of \class{set} or its subtype.
\end{cfuncdesc}

\begin{cfuncdesc}{PyObject*}{PySet_Pop}{PyObject *set}
  Return a new reference to an arbitrary object in the \var{set},
  and removes the object from the \var{set}.  Return \NULL{} on
  failure.  Raise \exception{KeyError} if the set is empty.
  Raise a \exception{SystemError} if \var{set} is an not an instance
  of \class{set} or its subtype.
\end{cfuncdesc}

\begin{cfuncdesc}{int}{PySet_Clear}{PyObject *set}
  Empty an existing set of all elements.
\end{cfuncdesc}

\chapter{Initialization, Finalization, and Threads
         \label{initialization}}

\begin{cfuncdesc}{void}{Py_Initialize}{}
  Initialize the Python interpreter.  In an application embedding 
  Python, this should be called before using any other Python/C API
  functions; with the exception of
  \cfunction{Py_SetProgramName()}\ttindex{Py_SetProgramName()},
  \cfunction{PyEval_InitThreads()}\ttindex{PyEval_InitThreads()},
  \cfunction{PyEval_ReleaseLock()}\ttindex{PyEval_ReleaseLock()},
  and \cfunction{PyEval_AcquireLock()}\ttindex{PyEval_AcquireLock()}.
  This initializes the table of loaded modules (\code{sys.modules}),
  and\withsubitem{(in module sys)}{\ttindex{modules}\ttindex{path}}
  creates the fundamental modules
  \module{__builtin__}\refbimodindex{__builtin__},
  \module{__main__}\refbimodindex{__main__} and
  \module{sys}\refbimodindex{sys}.  It also initializes the module
  search\indexiii{module}{search}{path} path (\code{sys.path}).
  It does not set \code{sys.argv}; use
  \cfunction{PySys_SetArgv()}\ttindex{PySys_SetArgv()} for that.  This
  is a no-op when called for a second time (without calling
  \cfunction{Py_Finalize()}\ttindex{Py_Finalize()} first).  There is
  no return value; it is a fatal error if the initialization fails.
\end{cfuncdesc}

\begin{cfuncdesc}{void}{Py_InitializeEx}{int initsigs}
  This function works like \cfunction{Py_Initialize()} if
  \var{initsigs} is 1. If \var{initsigs} is 0, it skips
  initialization registration of signal handlers, which
  might be useful when Python is embedded. \versionadded{2.4}
\end{cfuncdesc}

\begin{cfuncdesc}{int}{Py_IsInitialized}{}
  Return true (nonzero) when the Python interpreter has been
  initialized, false (zero) if not.  After \cfunction{Py_Finalize()}
  is called, this returns false until \cfunction{Py_Initialize()} is
  called again.
\end{cfuncdesc}

\begin{cfuncdesc}{void}{Py_Finalize}{}
  Undo all initializations made by \cfunction{Py_Initialize()} and
  subsequent use of Python/C API functions, and destroy all
  sub-interpreters (see \cfunction{Py_NewInterpreter()} below) that
  were created and not yet destroyed since the last call to
  \cfunction{Py_Initialize()}.  Ideally, this frees all memory
  allocated by the Python interpreter.  This is a no-op when called
  for a second time (without calling \cfunction{Py_Initialize()} again
  first).  There is no return value; errors during finalization are
  ignored.

  This function is provided for a number of reasons.  An embedding
  application might want to restart Python without having to restart
  the application itself.  An application that has loaded the Python
  interpreter from a dynamically loadable library (or DLL) might want
  to free all memory allocated by Python before unloading the
  DLL. During a hunt for memory leaks in an application a developer
  might want to free all memory allocated by Python before exiting
  from the application.

  \strong{Bugs and caveats:} The destruction of modules and objects in
  modules is done in random order; this may cause destructors
  (\method{__del__()} methods) to fail when they depend on other
  objects (even functions) or modules.  Dynamically loaded extension
  modules loaded by Python are not unloaded.  Small amounts of memory
  allocated by the Python interpreter may not be freed (if you find a
  leak, please report it).  Memory tied up in circular references
  between objects is not freed.  Some memory allocated by extension
  modules may not be freed.  Some extensions may not work properly if
  their initialization routine is called more than once; this can
  happen if an application calls \cfunction{Py_Initialize()} and
  \cfunction{Py_Finalize()} more than once.
\end{cfuncdesc}

\begin{cfuncdesc}{PyThreadState*}{Py_NewInterpreter}{}
  Create a new sub-interpreter.  This is an (almost) totally separate
  environment for the execution of Python code.  In particular, the
  new interpreter has separate, independent versions of all imported
  modules, including the fundamental modules
  \module{__builtin__}\refbimodindex{__builtin__},
  \module{__main__}\refbimodindex{__main__} and
  \module{sys}\refbimodindex{sys}.  The table of loaded modules
  (\code{sys.modules}) and the module search path (\code{sys.path})
  are also separate.  The new environment has no \code{sys.argv}
  variable.  It has new standard I/O stream file objects
  \code{sys.stdin}, \code{sys.stdout} and \code{sys.stderr} (however
  these refer to the same underlying \ctype{FILE} structures in the C
  library).
  \withsubitem{(in module sys)}{
    \ttindex{stdout}\ttindex{stderr}\ttindex{stdin}}

  The return value points to the first thread state created in the new
  sub-interpreter.  This thread state is made in the current thread
  state.  Note that no actual thread is created; see the discussion of
  thread states below.  If creation of the new interpreter is
  unsuccessful, \NULL{} is returned; no exception is set since the
  exception state is stored in the current thread state and there may
  not be a current thread state.  (Like all other Python/C API
  functions, the global interpreter lock must be held before calling
  this function and is still held when it returns; however, unlike
  most other Python/C API functions, there needn't be a current thread
  state on entry.)

  Extension modules are shared between (sub-)interpreters as follows:
  the first time a particular extension is imported, it is initialized
  normally, and a (shallow) copy of its module's dictionary is
  squirreled away.  When the same extension is imported by another
  (sub-)interpreter, a new module is initialized and filled with the
  contents of this copy; the extension's \code{init} function is not
  called.  Note that this is different from what happens when an
  extension is imported after the interpreter has been completely
  re-initialized by calling
  \cfunction{Py_Finalize()}\ttindex{Py_Finalize()} and
  \cfunction{Py_Initialize()}\ttindex{Py_Initialize()}; in that case,
  the extension's \code{init\var{module}} function \emph{is} called
  again.

  \strong{Bugs and caveats:} Because sub-interpreters (and the main
  interpreter) are part of the same process, the insulation between
  them isn't perfect --- for example, using low-level file operations
  like \withsubitem{(in module os)}{\ttindex{close()}}
  \function{os.close()} they can (accidentally or maliciously) affect
  each other's open files.  Because of the way extensions are shared
  between (sub-)interpreters, some extensions may not work properly;
  this is especially likely when the extension makes use of (static)
  global variables, or when the extension manipulates its module's
  dictionary after its initialization.  It is possible to insert
  objects created in one sub-interpreter into a namespace of another
  sub-interpreter; this should be done with great care to avoid
  sharing user-defined functions, methods, instances or classes
  between sub-interpreters, since import operations executed by such
  objects may affect the wrong (sub-)interpreter's dictionary of
  loaded modules.  (XXX This is a hard-to-fix bug that will be
  addressed in a future release.)
\end{cfuncdesc}

\begin{cfuncdesc}{void}{Py_EndInterpreter}{PyThreadState *tstate}
  Destroy the (sub-)interpreter represented by the given thread state.
  The given thread state must be the current thread state.  See the
  discussion of thread states below.  When the call returns, the
  current thread state is \NULL.  All thread states associated with
  this interpreter are destroyed.  (The global interpreter lock must
  be held before calling this function and is still held when it
  returns.)  \cfunction{Py_Finalize()}\ttindex{Py_Finalize()} will
  destroy all sub-interpreters that haven't been explicitly destroyed
  at that point.
\end{cfuncdesc}

\begin{cfuncdesc}{void}{Py_SetProgramName}{char *name}
  This function should be called before
  \cfunction{Py_Initialize()}\ttindex{Py_Initialize()} is called
  for the first time, if it is called at all.  It tells the
  interpreter the value of the \code{argv[0]} argument to the
  \cfunction{main()}\ttindex{main()} function of the program.  This is
  used by \cfunction{Py_GetPath()}\ttindex{Py_GetPath()} and some
  other functions below to find the Python run-time libraries relative
  to the interpreter executable.  The default value is
  \code{'python'}.  The argument should point to a zero-terminated
  character string in static storage whose contents will not change
  for the duration of the program's execution.  No code in the Python
  interpreter will change the contents of this storage.
\end{cfuncdesc}

\begin{cfuncdesc}{char*}{Py_GetProgramName}{}
  Return the program name set with
  \cfunction{Py_SetProgramName()}\ttindex{Py_SetProgramName()}, or the
  default.  The returned string points into static storage; the caller
  should not modify its value.
\end{cfuncdesc}

\begin{cfuncdesc}{char*}{Py_GetPrefix}{}
  Return the \emph{prefix} for installed platform-independent files.
  This is derived through a number of complicated rules from the
  program name set with \cfunction{Py_SetProgramName()} and some
  environment variables; for example, if the program name is
  \code{'/usr/local/bin/python'}, the prefix is \code{'/usr/local'}.
  The returned string points into static storage; the caller should
  not modify its value.  This corresponds to the \makevar{prefix}
  variable in the top-level \file{Makefile} and the
  \longprogramopt{prefix} argument to the \program{configure} script
  at build time.  The value is available to Python code as
  \code{sys.prefix}.  It is only useful on \UNIX.  See also the next
  function.
\end{cfuncdesc}

\begin{cfuncdesc}{char*}{Py_GetExecPrefix}{}
  Return the \emph{exec-prefix} for installed
  platform-\emph{de}pendent files.  This is derived through a number
  of complicated rules from the program name set with
  \cfunction{Py_SetProgramName()} and some environment variables; for
  example, if the program name is \code{'/usr/local/bin/python'}, the
  exec-prefix is \code{'/usr/local'}.  The returned string points into
  static storage; the caller should not modify its value.  This
  corresponds to the \makevar{exec_prefix} variable in the top-level
  \file{Makefile} and the \longprogramopt{exec-prefix} argument to the
  \program{configure} script at build  time.  The value is available
  to Python code as \code{sys.exec_prefix}.  It is only useful on
  \UNIX.

  Background: The exec-prefix differs from the prefix when platform
  dependent files (such as executables and shared libraries) are
  installed in a different directory tree.  In a typical installation,
  platform dependent files may be installed in the
  \file{/usr/local/plat} subtree while platform independent may be
  installed in \file{/usr/local}.

  Generally speaking, a platform is a combination of hardware and
  software families, e.g.  Sparc machines running the Solaris 2.x
  operating system are considered the same platform, but Intel
  machines running Solaris 2.x are another platform, and Intel
  machines running Linux are yet another platform.  Different major
  revisions of the same operating system generally also form different
  platforms.  Non-\UNIX{} operating systems are a different story; the
  installation strategies on those systems are so different that the
  prefix and exec-prefix are meaningless, and set to the empty string.
  Note that compiled Python bytecode files are platform independent
  (but not independent from the Python version by which they were
  compiled!).

  System administrators will know how to configure the \program{mount}
  or \program{automount} programs to share \file{/usr/local} between
  platforms while having \file{/usr/local/plat} be a different
  filesystem for each platform.
\end{cfuncdesc}

\begin{cfuncdesc}{char*}{Py_GetProgramFullPath}{}
  Return the full program name of the Python executable; this is 
  computed as a side-effect of deriving the default module search path 
  from the program name (set by
  \cfunction{Py_SetProgramName()}\ttindex{Py_SetProgramName()} above).
  The returned string points into static storage; the caller should
  not modify its value.  The value is available to Python code as
  \code{sys.executable}.
  \withsubitem{(in module sys)}{\ttindex{executable}}
\end{cfuncdesc}

\begin{cfuncdesc}{char*}{Py_GetPath}{}
  \indexiii{module}{search}{path}
  Return the default module search path; this is computed from the 
  program name (set by \cfunction{Py_SetProgramName()} above) and some
  environment variables.  The returned string consists of a series of
  directory names separated by a platform dependent delimiter
  character.  The delimiter character is \character{:} on \UNIX and Mac OS X,
  \character{;} on Windows.  The returned string points into
  static storage; the caller should not modify its value.  The value
  is available to Python code as the list
  \code{sys.path}\withsubitem{(in module sys)}{\ttindex{path}}, which
  may be modified to change the future search path for loaded
  modules.

  % XXX should give the exact rules
\end{cfuncdesc}

\begin{cfuncdesc}{const char*}{Py_GetVersion}{}
  Return the version of this Python interpreter.  This is a string
  that looks something like

\begin{verbatim}
"1.5 (#67, Dec 31 1997, 22:34:28) [GCC 2.7.2.2]"
\end{verbatim}

  The first word (up to the first space character) is the current
  Python version; the first three characters are the major and minor
  version separated by a period.  The returned string points into
  static storage; the caller should not modify its value.  The value
  is available to Python code as \code{sys.version}.
  \withsubitem{(in module sys)}{\ttindex{version}}
\end{cfuncdesc}

\begin{cfuncdesc}{const char*}{Py_GetPlatform}{}
  Return the platform identifier for the current platform.  On \UNIX,
  this is formed from the ``official'' name of the operating system,
  converted to lower case, followed by the major revision number;
  e.g., for Solaris 2.x, which is also known as SunOS 5.x, the value
  is \code{'sunos5'}.  On Mac OS X, it is \code{'darwin'}.  On Windows,
  it is \code{'win'}.  The returned string points into static storage;
  the caller should not modify its value.  The value is available to
  Python code as \code{sys.platform}.
  \withsubitem{(in module sys)}{\ttindex{platform}}
\end{cfuncdesc}

\begin{cfuncdesc}{const char*}{Py_GetCopyright}{}
  Return the official copyright string for the current Python version,
  for example

  \code{'Copyright 1991-1995 Stichting Mathematisch Centrum, Amsterdam'}

  The returned string points into static storage; the caller should
  not modify its value.  The value is available to Python code as
  \code{sys.copyright}.
  \withsubitem{(in module sys)}{\ttindex{copyright}}
\end{cfuncdesc}

\begin{cfuncdesc}{const char*}{Py_GetCompiler}{}
  Return an indication of the compiler used to build the current
  Python version, in square brackets, for example:

\begin{verbatim}
"[GCC 2.7.2.2]"
\end{verbatim}

  The returned string points into static storage; the caller should
  not modify its value.  The value is available to Python code as part
  of the variable \code{sys.version}.
  \withsubitem{(in module sys)}{\ttindex{version}}
\end{cfuncdesc}

\begin{cfuncdesc}{const char*}{Py_GetBuildInfo}{}
  Return information about the sequence number and build date and time 
  of the current Python interpreter instance, for example

\begin{verbatim}
"#67, Aug  1 1997, 22:34:28"
\end{verbatim}

  The returned string points into static storage; the caller should
  not modify its value.  The value is available to Python code as part
  of the variable \code{sys.version}.
  \withsubitem{(in module sys)}{\ttindex{version}}
\end{cfuncdesc}

\begin{cfuncdesc}{int}{PySys_SetArgv}{int argc, char **argv}
  Set \code{sys.argv} based on \var{argc} and \var{argv}.  These
  parameters are similar to those passed to the program's
  \cfunction{main()}\ttindex{main()} function with the difference that
  the first entry should refer to the script file to be executed
  rather than the executable hosting the Python interpreter.  If there
  isn't a script that will be run, the first entry in \var{argv} can
  be an empty string.  If this function fails to initialize
  \code{sys.argv}, a fatal condition is signalled using
  \cfunction{Py_FatalError()}\ttindex{Py_FatalError()}.
  \withsubitem{(in module sys)}{\ttindex{argv}}
  % XXX impl. doesn't seem consistent in allowing 0/NULL for the params; 
  % check w/ Guido.
\end{cfuncdesc}

% XXX Other PySys thingies (doesn't really belong in this chapter)

\section{Thread State and the Global Interpreter Lock
         \label{threads}}

\index{global interpreter lock}
\index{interpreter lock}
\index{lock, interpreter}

The Python interpreter is not fully thread safe.  In order to support
multi-threaded Python programs, there's a global lock that must be
held by the current thread before it can safely access Python objects.
Without the lock, even the simplest operations could cause problems in
a multi-threaded program: for example, when two threads simultaneously
increment the reference count of the same object, the reference count
could end up being incremented only once instead of twice.

Therefore, the rule exists that only the thread that has acquired the
global interpreter lock may operate on Python objects or call Python/C
API functions.  In order to support multi-threaded Python programs,
the interpreter regularly releases and reacquires the lock --- by
default, every 100 bytecode instructions (this can be changed with
\withsubitem{(in module sys)}{\ttindex{setcheckinterval()}}
\function{sys.setcheckinterval()}).  The lock is also released and
reacquired around potentially blocking I/O operations like reading or
writing a file, so that other threads can run while the thread that
requests the I/O is waiting for the I/O operation to complete.

The Python interpreter needs to keep some bookkeeping information
separate per thread --- for this it uses a data structure called
\ctype{PyThreadState}\ttindex{PyThreadState}.  There's one global
variable, however: the pointer to the current
\ctype{PyThreadState}\ttindex{PyThreadState} structure.  While most
thread packages have a way to store ``per-thread global data,''
Python's internal platform independent thread abstraction doesn't
support this yet.  Therefore, the current thread state must be
manipulated explicitly.

This is easy enough in most cases.  Most code manipulating the global
interpreter lock has the following simple structure:

\begin{verbatim}
Save the thread state in a local variable.
Release the interpreter lock.
...Do some blocking I/O operation...
Reacquire the interpreter lock.
Restore the thread state from the local variable.
\end{verbatim}

This is so common that a pair of macros exists to simplify it:

\begin{verbatim}
Py_BEGIN_ALLOW_THREADS
...Do some blocking I/O operation...
Py_END_ALLOW_THREADS
\end{verbatim}

The
\csimplemacro{Py_BEGIN_ALLOW_THREADS}\ttindex{Py_BEGIN_ALLOW_THREADS}
macro opens a new block and declares a hidden local variable; the
\csimplemacro{Py_END_ALLOW_THREADS}\ttindex{Py_END_ALLOW_THREADS}
macro closes the block.  Another advantage of using these two macros
is that when Python is compiled without thread support, they are
defined empty, thus saving the thread state and lock manipulations.

When thread support is enabled, the block above expands to the
following code:

\begin{verbatim}
    PyThreadState *_save;

    _save = PyEval_SaveThread();
    ...Do some blocking I/O operation...
    PyEval_RestoreThread(_save);
\end{verbatim}

Using even lower level primitives, we can get roughly the same effect
as follows:

\begin{verbatim}
    PyThreadState *_save;

    _save = PyThreadState_Swap(NULL);
    PyEval_ReleaseLock();
    ...Do some blocking I/O operation...
    PyEval_AcquireLock();
    PyThreadState_Swap(_save);
\end{verbatim}

There are some subtle differences; in particular,
\cfunction{PyEval_RestoreThread()}\ttindex{PyEval_RestoreThread()} saves
and restores the value of the  global variable
\cdata{errno}\ttindex{errno}, since the lock manipulation does not
guarantee that \cdata{errno} is left alone.  Also, when thread support
is disabled,
\cfunction{PyEval_SaveThread()}\ttindex{PyEval_SaveThread()} and
\cfunction{PyEval_RestoreThread()} don't manipulate the lock; in this
case, \cfunction{PyEval_ReleaseLock()}\ttindex{PyEval_ReleaseLock()} and
\cfunction{PyEval_AcquireLock()}\ttindex{PyEval_AcquireLock()} are not
available.  This is done so that dynamically loaded extensions
compiled with thread support enabled can be loaded by an interpreter
that was compiled with disabled thread support.

The global interpreter lock is used to protect the pointer to the
current thread state.  When releasing the lock and saving the thread
state, the current thread state pointer must be retrieved before the
lock is released (since another thread could immediately acquire the
lock and store its own thread state in the global variable).
Conversely, when acquiring the lock and restoring the thread state,
the lock must be acquired before storing the thread state pointer.

Why am I going on with so much detail about this?  Because when
threads are created from C, they don't have the global interpreter
lock, nor is there a thread state data structure for them.  Such
threads must bootstrap themselves into existence, by first creating a
thread state data structure, then acquiring the lock, and finally
storing their thread state pointer, before they can start using the
Python/C API.  When they are done, they should reset the thread state
pointer, release the lock, and finally free their thread state data
structure.

Beginning with version 2.3, threads can now take advantage of the 
\cfunction{PyGILState_*()} functions to do all of the above
automatically.  The typical idiom for calling into Python from a C
thread is now:

\begin{verbatim}
    PyGILState_STATE gstate;
    gstate = PyGILState_Ensure();

    /* Perform Python actions here.  */
    result = CallSomeFunction();
    /* evaluate result */

    /* Release the thread. No Python API allowed beyond this point. */
    PyGILState_Release(gstate);
\end{verbatim}

Note that the \cfunction{PyGILState_*()} functions assume there is only
one global  interpreter (created automatically by
\cfunction{Py_Initialize()}).  Python still supports the creation of
additional interpreters  (using \cfunction{Py_NewInterpreter()}), but
mixing multiple interpreters and the \cfunction{PyGILState_*()} API is
unsupported.

\begin{ctypedesc}{PyInterpreterState}
  This data structure represents the state shared by a number of
  cooperating threads.  Threads belonging to the same interpreter
  share their module administration and a few other internal items.
  There are no public members in this structure.

  Threads belonging to different interpreters initially share nothing,
  except process state like available memory, open file descriptors
  and such.  The global interpreter lock is also shared by all
  threads, regardless of to which interpreter they belong.
\end{ctypedesc}

\begin{ctypedesc}{PyThreadState}
  This data structure represents the state of a single thread.  The
  only public data member is \ctype{PyInterpreterState
  *}\member{interp}, which points to this thread's interpreter state.
\end{ctypedesc}

\begin{cfuncdesc}{void}{PyEval_InitThreads}{}
  Initialize and acquire the global interpreter lock.  It should be
  called in the main thread before creating a second thread or
  engaging in any other thread operations such as
  \cfunction{PyEval_ReleaseLock()}\ttindex{PyEval_ReleaseLock()} or
  \code{PyEval_ReleaseThread(\var{tstate})}\ttindex{PyEval_ReleaseThread()}.
  It is not needed before calling
  \cfunction{PyEval_SaveThread()}\ttindex{PyEval_SaveThread()} or
  \cfunction{PyEval_RestoreThread()}\ttindex{PyEval_RestoreThread()}.

  This is a no-op when called for a second time.  It is safe to call
  this function before calling
  \cfunction{Py_Initialize()}\ttindex{Py_Initialize()}.

  When only the main thread exists, no lock operations are needed.
  This is a common situation (most Python programs do not use
  threads), and the lock operations slow the interpreter down a bit.
  Therefore, the lock is not created initially.  This situation is
  equivalent to having acquired the lock:  when there is only a single
  thread, all object accesses are safe.  Therefore, when this function
  initializes the lock, it also acquires it.  Before the Python
  \module{thread}\refbimodindex{thread} module creates a new thread,
  knowing that either it has the lock or the lock hasn't been created
  yet, it calls \cfunction{PyEval_InitThreads()}.  When this call
  returns, it is guaranteed that the lock has been created and that the
  calling thread has acquired it.

  It is \strong{not} safe to call this function when it is unknown
  which thread (if any) currently has the global interpreter lock.

  This function is not available when thread support is disabled at
  compile time.
\end{cfuncdesc}

\begin{cfuncdesc}{int}{PyEval_ThreadsInitialized}{}
  Returns a non-zero value if \cfunction{PyEval_InitThreads()} has been
  called.  This function can be called without holding the lock, and
  therefore can be used to avoid calls to the locking API when running
  single-threaded.  This function is not available when thread support
  is disabled at compile time. \versionadded{2.4}
\end{cfuncdesc}

\begin{cfuncdesc}{void}{PyEval_AcquireLock}{}
  Acquire the global interpreter lock.  The lock must have been
  created earlier.  If this thread already has the lock, a deadlock
  ensues.  This function is not available when thread support is
  disabled at compile time.
\end{cfuncdesc}

\begin{cfuncdesc}{void}{PyEval_ReleaseLock}{}
  Release the global interpreter lock.  The lock must have been
  created earlier.  This function is not available when thread support
  is disabled at compile time.
\end{cfuncdesc}

\begin{cfuncdesc}{void}{PyEval_AcquireThread}{PyThreadState *tstate}
  Acquire the global interpreter lock and set the current thread
  state to \var{tstate}, which should not be \NULL.  The lock must
  have been created earlier.  If this thread already has the lock,
  deadlock ensues.  This function is not available when thread support
  is disabled at compile time.
\end{cfuncdesc}

\begin{cfuncdesc}{void}{PyEval_ReleaseThread}{PyThreadState *tstate}
  Reset the current thread state to \NULL{} and release the global
  interpreter lock.  The lock must have been created earlier and must
  be held by the current thread.  The \var{tstate} argument, which
  must not be \NULL, is only used to check that it represents the
  current thread state --- if it isn't, a fatal error is reported.
  This function is not available when thread support is disabled at
  compile time.
\end{cfuncdesc}

\begin{cfuncdesc}{PyThreadState*}{PyEval_SaveThread}{}
  Release the interpreter lock (if it has been created and thread
  support is enabled) and reset the thread state to \NULL, returning
  the previous thread state (which is not \NULL).  If the lock has
  been created, the current thread must have acquired it.  (This
  function is available even when thread support is disabled at
  compile time.)
\end{cfuncdesc}

\begin{cfuncdesc}{void}{PyEval_RestoreThread}{PyThreadState *tstate}
  Acquire the interpreter lock (if it has been created and thread
  support is enabled) and set the thread state to \var{tstate}, which
  must not be \NULL.  If the lock has been created, the current thread
  must not have acquired it, otherwise deadlock ensues.  (This
  function is available even when thread support is disabled at
  compile time.)
\end{cfuncdesc}

The following macros are normally used without a trailing semicolon;
look for example usage in the Python source distribution.

\begin{csimplemacrodesc}{Py_BEGIN_ALLOW_THREADS}
  This macro expands to
  \samp{\{ PyThreadState *_save; _save = PyEval_SaveThread();}.
  Note that it contains an opening brace; it must be matched with a
  following \csimplemacro{Py_END_ALLOW_THREADS} macro.  See above for
  further discussion of this macro.  It is a no-op when thread support
  is disabled at compile time.
\end{csimplemacrodesc}

\begin{csimplemacrodesc}{Py_END_ALLOW_THREADS}
  This macro expands to \samp{PyEval_RestoreThread(_save); \}}.
  Note that it contains a closing brace; it must be matched with an
  earlier \csimplemacro{Py_BEGIN_ALLOW_THREADS} macro.  See above for
  further discussion of this macro.  It is a no-op when thread support
  is disabled at compile time.
\end{csimplemacrodesc}

\begin{csimplemacrodesc}{Py_BLOCK_THREADS}
  This macro expands to \samp{PyEval_RestoreThread(_save);}: it is
  equivalent to \csimplemacro{Py_END_ALLOW_THREADS} without the
  closing brace.  It is a no-op when thread support is disabled at
  compile time.
\end{csimplemacrodesc}

\begin{csimplemacrodesc}{Py_UNBLOCK_THREADS}
  This macro expands to \samp{_save = PyEval_SaveThread();}: it is
  equivalent to \csimplemacro{Py_BEGIN_ALLOW_THREADS} without the
  opening brace and variable declaration.  It is a no-op when thread
  support is disabled at compile time.
\end{csimplemacrodesc}

All of the following functions are only available when thread support
is enabled at compile time, and must be called only when the
interpreter lock has been created.

\begin{cfuncdesc}{PyInterpreterState*}{PyInterpreterState_New}{}
  Create a new interpreter state object.  The interpreter lock need
  not be held, but may be held if it is necessary to serialize calls
  to this function.
\end{cfuncdesc}

\begin{cfuncdesc}{void}{PyInterpreterState_Clear}{PyInterpreterState *interp}
  Reset all information in an interpreter state object.  The
  interpreter lock must be held.
\end{cfuncdesc}

\begin{cfuncdesc}{void}{PyInterpreterState_Delete}{PyInterpreterState *interp}
  Destroy an interpreter state object.  The interpreter lock need not
  be held.  The interpreter state must have been reset with a previous
  call to \cfunction{PyInterpreterState_Clear()}.
\end{cfuncdesc}

\begin{cfuncdesc}{PyThreadState*}{PyThreadState_New}{PyInterpreterState *interp}
  Create a new thread state object belonging to the given interpreter
  object.  The interpreter lock need not be held, but may be held if
  it is necessary to serialize calls to this function.
\end{cfuncdesc}

\begin{cfuncdesc}{void}{PyThreadState_Clear}{PyThreadState *tstate}
  Reset all information in a thread state object.  The interpreter lock
  must be held.
\end{cfuncdesc}

\begin{cfuncdesc}{void}{PyThreadState_Delete}{PyThreadState *tstate}
  Destroy a thread state object.  The interpreter lock need not be
  held.  The thread state must have been reset with a previous call to
  \cfunction{PyThreadState_Clear()}.
\end{cfuncdesc}

\begin{cfuncdesc}{PyThreadState*}{PyThreadState_Get}{}
  Return the current thread state.  The interpreter lock must be
  held.  When the current thread state is \NULL, this issues a fatal
  error (so that the caller needn't check for \NULL).
\end{cfuncdesc}

\begin{cfuncdesc}{PyThreadState*}{PyThreadState_Swap}{PyThreadState *tstate}
  Swap the current thread state with the thread state given by the
  argument \var{tstate}, which may be \NULL.  The interpreter lock
  must be held.
\end{cfuncdesc}

\begin{cfuncdesc}{PyObject*}{PyThreadState_GetDict}{}
  Return a dictionary in which extensions can store thread-specific
  state information.  Each extension should use a unique key to use to
  store state in the dictionary.  It is okay to call this function
  when no current thread state is available.
  If this function returns \NULL, no exception has been raised and the
  caller should assume no current thread state is available.
  \versionchanged[Previously this could only be called when a current
  thread is active, and \NULL{} meant that an exception was raised]{2.3}
\end{cfuncdesc}

\begin{cfuncdesc}{int}{PyThreadState_SetAsyncExc}{long id, PyObject *exc}
  Asynchronously raise an exception in a thread. 
  The \var{id} argument is the thread id of the target thread;
  \var{exc} is the exception object to be raised.
  This function does not steal any references to \var{exc}.
  To prevent naive misuse, you must write your own C extension 
  to call this.  Must be called with the GIL held. 
  Returns the number of thread states modified; if it returns a number 
  greater than one, you're in trouble, and you should call it again 
  with \var{exc} set to \constant{NULL} to revert the effect.
  This raises no exceptions.
  \versionadded{2.3}
\end{cfuncdesc}

\begin{cfuncdesc}{PyGILState_STATE}{PyGILState_Ensure}{}
Ensure that the current thread is ready to call the Python
C API regardless of the current state of Python, or of its
thread lock.  This may be called as many times as desired
by a thread as long as each call is matched with a call to 
\cfunction{PyGILState_Release()}.  
In general, other thread-related APIs may 
be used between \cfunction{PyGILState_Ensure()} and \cfunction{PyGILState_Release()} calls as long as the 
thread state is restored to its previous state before the Release().
For example, normal usage of the \csimplemacro{Py_BEGIN_ALLOW_THREADS}
and \csimplemacro{Py_END_ALLOW_THREADS} macros is acceptable.
    
The return value is an opaque "handle" to the thread state when
\cfunction{PyGILState_Acquire()} was called, and must be passed to
\cfunction{PyGILState_Release()} to ensure Python is left in the same
state. Even though recursive calls are allowed, these handles
\emph{cannot} be shared - each unique call to
\cfunction{PyGILState_Ensure} must save the handle for its call to
\cfunction{PyGILState_Release}.
    
When the function returns, the current thread will hold the GIL.
Failure is a fatal error.
  \versionadded{2.3}
\end{cfuncdesc}

\begin{cfuncdesc}{void}{PyGILState_Release}{PyGILState_STATE}
Release any resources previously acquired.  After this call, Python's
state will be the same as it was prior to the corresponding
\cfunction{PyGILState_Ensure} call (but generally this state will be
unknown to the caller, hence the use of the GILState API.)
    
Every call to \cfunction{PyGILState_Ensure()} must be matched by a call to 
\cfunction{PyGILState_Release()} on the same thread.
  \versionadded{2.3}
\end{cfuncdesc}


\section{Profiling and Tracing \label{profiling}}

\sectionauthor{Fred L. Drake, Jr.}{fdrake@acm.org}

The Python interpreter provides some low-level support for attaching
profiling and execution tracing facilities.  These are used for
profiling, debugging, and coverage analysis tools.

Starting with Python 2.2, the implementation of this facility was
substantially revised, and an interface from C was added.  This C
interface allows the profiling or tracing code to avoid the overhead
of calling through Python-level callable objects, making a direct C
function call instead.  The essential attributes of the facility have
not changed; the interface allows trace functions to be installed
per-thread, and the basic events reported to the trace function are
the same as had been reported to the Python-level trace functions in
previous versions.

\begin{ctypedesc}[Py_tracefunc]{int (*Py_tracefunc)(PyObject *obj,
                                PyFrameObject *frame, int what,
                                PyObject *arg)}
  The type of the trace function registered using
  \cfunction{PyEval_SetProfile()} and \cfunction{PyEval_SetTrace()}.
  The first parameter is the object passed to the registration
  function as \var{obj}, \var{frame} is the frame object to which the
  event pertains, \var{what} is one of the constants
  \constant{PyTrace_CALL}, \constant{PyTrace_EXCEPTION},
  \constant{PyTrace_LINE}, \constant{PyTrace_RETURN},
  \constant{PyTrace_C_CALL}, \constant{PyTrace_C_EXCEPTION},
  or \constant{PyTrace_C_RETURN}, and \var{arg}
  depends on the value of \var{what}:

  \begin{tableii}{l|l}{constant}{Value of \var{what}}{Meaning of \var{arg}}
    \lineii{PyTrace_CALL}{Always \NULL.}
    \lineii{PyTrace_EXCEPTION}{Exception information as returned by
                            \function{sys.exc_info()}.}
    \lineii{PyTrace_LINE}{Always \NULL.}
    \lineii{PyTrace_RETURN}{Value being returned to the caller.}
    \lineii{PyTrace_C_CALL}{Name of function being called.}
    \lineii{PyTrace_C_EXCEPTION}{Always \NULL.}
    \lineii{PyTrace_C_RETURN}{Always \NULL.}
  \end{tableii}
\end{ctypedesc}

\begin{cvardesc}{int}{PyTrace_CALL}
  The value of the \var{what} parameter to a \ctype{Py_tracefunc}
  function when a new call to a function or method is being reported,
  or a new entry into a generator.  Note that the creation of the
  iterator for a generator function is not reported as there is no
  control transfer to the Python bytecode in the corresponding frame.
\end{cvardesc}

\begin{cvardesc}{int}{PyTrace_EXCEPTION}
  The value of the \var{what} parameter to a \ctype{Py_tracefunc}
  function when an exception has been raised.  The callback function
  is called with this value for \var{what} when after any bytecode is
  processed after which the exception becomes set within the frame
  being executed.  The effect of this is that as exception propagation
  causes the Python stack to unwind, the callback is called upon
  return to each frame as the exception propagates.  Only trace
  functions receives these events; they are not needed by the
  profiler.
\end{cvardesc}

\begin{cvardesc}{int}{PyTrace_LINE}
  The value passed as the \var{what} parameter to a trace function
  (but not a profiling function) when a line-number event is being
  reported.
\end{cvardesc}

\begin{cvardesc}{int}{PyTrace_RETURN}
  The value for the \var{what} parameter to \ctype{Py_tracefunc}
  functions when a call is returning without propagating an exception.
\end{cvardesc}

\begin{cvardesc}{int}{PyTrace_C_CALL}
  The value for the \var{what} parameter to \ctype{Py_tracefunc}
  functions when a C function is about to be called.
\end{cvardesc}

\begin{cvardesc}{int}{PyTrace_C_EXCEPTION}
  The value for the \var{what} parameter to \ctype{Py_tracefunc}
  functions when a C function has thrown an exception.
\end{cvardesc}

\begin{cvardesc}{int}{PyTrace_C_RETURN}
  The value for the \var{what} parameter to \ctype{Py_tracefunc}
  functions when a C function has returned.
\end{cvardesc}

\begin{cfuncdesc}{void}{PyEval_SetProfile}{Py_tracefunc func, PyObject *obj}
  Set the profiler function to \var{func}.  The \var{obj} parameter is
  passed to the function as its first parameter, and may be any Python
  object, or \NULL.  If the profile function needs to maintain state,
  using a different value for \var{obj} for each thread provides a
  convenient and thread-safe place to store it.  The profile function
  is called for all monitored events except the line-number events.
\end{cfuncdesc}

\begin{cfuncdesc}{void}{PyEval_SetTrace}{Py_tracefunc func, PyObject *obj}
  Set the tracing function to \var{func}.  This is similar to
  \cfunction{PyEval_SetProfile()}, except the tracing function does
  receive line-number events.
\end{cfuncdesc}


\section{Advanced Debugger Support \label{advanced-debugging}}
\sectionauthor{Fred L. Drake, Jr.}{fdrake@acm.org}

These functions are only intended to be used by advanced debugging
tools.

\begin{cfuncdesc}{PyInterpreterState*}{PyInterpreterState_Head}{}
  Return the interpreter state object at the head of the list of all
  such objects.
  \versionadded{2.2}
\end{cfuncdesc}

\begin{cfuncdesc}{PyInterpreterState*}{PyInterpreterState_Next}{PyInterpreterState *interp}
  Return the next interpreter state object after \var{interp} from the
  list of all such objects.
  \versionadded{2.2}
\end{cfuncdesc}

\begin{cfuncdesc}{PyThreadState *}{PyInterpreterState_ThreadHead}{PyInterpreterState *interp}
  Return the a pointer to the first \ctype{PyThreadState} object in
  the list of threads associated with the interpreter \var{interp}.
  \versionadded{2.2}
\end{cfuncdesc}

\begin{cfuncdesc}{PyThreadState*}{PyThreadState_Next}{PyThreadState *tstate}
  Return the next thread state object after \var{tstate} from the list
  of all such objects belonging to the same \ctype{PyInterpreterState}
  object.
  \versionadded{2.2}
\end{cfuncdesc}

\chapter{Memory Management \label{memory}}
\sectionauthor{Vladimir Marangozov}{Vladimir.Marangozov@inrialpes.fr}


\section{Overview \label{memoryOverview}}

Memory management in Python involves a private heap containing all
Python objects and data structures. The management of this private
heap is ensured internally by the \emph{Python memory manager}.  The
Python memory manager has different components which deal with various
dynamic storage management aspects, like sharing, segmentation,
preallocation or caching.

At the lowest level, a raw memory allocator ensures that there is
enough room in the private heap for storing all Python-related data
by interacting with the memory manager of the operating system. On top
of the raw memory allocator, several object-specific allocators
operate on the same heap and implement distinct memory management
policies adapted to the peculiarities of every object type. For
example, integer objects are managed differently within the heap than
strings, tuples or dictionaries because integers imply different
storage requirements and speed/space tradeoffs. The Python memory
manager thus delegates some of the work to the object-specific
allocators, but ensures that the latter operate within the bounds of
the private heap.

It is important to understand that the management of the Python heap
is performed by the interpreter itself and that the user has no
control over it, even if she regularly manipulates object pointers to
memory blocks inside that heap.  The allocation of heap space for
Python objects and other internal buffers is performed on demand by
the Python memory manager through the Python/C API functions listed in
this document.

To avoid memory corruption, extension writers should never try to
operate on Python objects with the functions exported by the C
library: \cfunction{malloc()}\ttindex{malloc()},
\cfunction{calloc()}\ttindex{calloc()},
\cfunction{realloc()}\ttindex{realloc()} and
\cfunction{free()}\ttindex{free()}.  This will result in 
mixed calls between the C allocator and the Python memory manager
with fatal consequences, because they implement different algorithms
and operate on different heaps.  However, one may safely allocate and
release memory blocks with the C library allocator for individual
purposes, as shown in the following example:

\begin{verbatim}
    PyObject *res;
    char *buf = (char *) malloc(BUFSIZ); /* for I/O */

    if (buf == NULL)
        return PyErr_NoMemory();
    ...Do some I/O operation involving buf...
    res = PyString_FromString(buf);
    free(buf); /* malloc'ed */
    return res;
\end{verbatim}

In this example, the memory request for the I/O buffer is handled by
the C library allocator. The Python memory manager is involved only
in the allocation of the string object returned as a result.

In most situations, however, it is recommended to allocate memory from
the Python heap specifically because the latter is under control of
the Python memory manager. For example, this is required when the
interpreter is extended with new object types written in C. Another
reason for using the Python heap is the desire to \emph{inform} the
Python memory manager about the memory needs of the extension module.
Even when the requested memory is used exclusively for internal,
highly-specific purposes, delegating all memory requests to the Python
memory manager causes the interpreter to have a more accurate image of
its memory footprint as a whole. Consequently, under certain
circumstances, the Python memory manager may or may not trigger
appropriate actions, like garbage collection, memory compaction or
other preventive procedures. Note that by using the C library
allocator as shown in the previous example, the allocated memory for
the I/O buffer escapes completely the Python memory manager.


\section{Memory Interface \label{memoryInterface}}

The following function sets, modeled after the ANSI C standard,
but specifying  behavior when requesting zero bytes,
are available for allocating and releasing memory from the Python heap:


\begin{cfuncdesc}{void*}{PyMem_Malloc}{size_t n}
  Allocates \var{n} bytes and returns a pointer of type \ctype{void*}
  to the allocated memory, or \NULL{} if the request fails.
  Requesting zero bytes returns a distinct non-\NULL{} pointer if
  possible, as if \cfunction{PyMem_Malloc(1)} had been called instead.
  The memory will not have been initialized in any way.
\end{cfuncdesc}

\begin{cfuncdesc}{void*}{PyMem_Realloc}{void *p, size_t n}
  Resizes the memory block pointed to by \var{p} to \var{n} bytes.
  The contents will be unchanged to the minimum of the old and the new
  sizes. If \var{p} is \NULL, the call is equivalent to
  \cfunction{PyMem_Malloc(\var{n})}; else if \var{n} is equal to zero, the
  memory block is resized but is not freed, and the returned pointer
  is non-\NULL.  Unless \var{p} is \NULL, it must have been
  returned by a previous call to \cfunction{PyMem_Malloc()} or
  \cfunction{PyMem_Realloc()}.
\end{cfuncdesc}

\begin{cfuncdesc}{void}{PyMem_Free}{void *p}
  Frees the memory block pointed to by \var{p}, which must have been
  returned by a previous call to \cfunction{PyMem_Malloc()} or
  \cfunction{PyMem_Realloc()}.  Otherwise, or if
  \cfunction{PyMem_Free(p)} has been called before, undefined
  behavior occurs. If \var{p} is \NULL, no operation is performed.
\end{cfuncdesc}

The following type-oriented macros are provided for convenience.  Note 
that \var{TYPE} refers to any C type.

\begin{cfuncdesc}{\var{TYPE}*}{PyMem_New}{TYPE, size_t n}
  Same as \cfunction{PyMem_Malloc()}, but allocates \code{(\var{n} *
  sizeof(\var{TYPE}))} bytes of memory.  Returns a pointer cast to
  \ctype{\var{TYPE}*}.  The memory will not have been initialized in
  any way.
\end{cfuncdesc}

\begin{cfuncdesc}{\var{TYPE}*}{PyMem_Resize}{void *p, TYPE, size_t n}
  Same as \cfunction{PyMem_Realloc()}, but the memory block is resized
  to \code{(\var{n} * sizeof(\var{TYPE}))} bytes.  Returns a pointer
  cast to \ctype{\var{TYPE}*}.
\end{cfuncdesc}

\begin{cfuncdesc}{void}{PyMem_Del}{void *p}
  Same as \cfunction{PyMem_Free()}.
\end{cfuncdesc}

In addition, the following macro sets are provided for calling the
Python memory allocator directly, without involving the C API functions
listed above. However, note that their use does not preserve binary
compatibility across Python versions and is therefore deprecated in
extension modules.

\cfunction{PyMem_MALLOC()}, \cfunction{PyMem_REALLOC()}, \cfunction{PyMem_FREE()}.

\cfunction{PyMem_NEW()}, \cfunction{PyMem_RESIZE()}, \cfunction{PyMem_DEL()}.


\section{Examples \label{memoryExamples}}

Here is the example from section \ref{memoryOverview}, rewritten so
that the I/O buffer is allocated from the Python heap by using the
first function set:

\begin{verbatim}
    PyObject *res;
    char *buf = (char *) PyMem_Malloc(BUFSIZ); /* for I/O */

    if (buf == NULL)
        return PyErr_NoMemory();
    /* ...Do some I/O operation involving buf... */
    res = PyString_FromString(buf);
    PyMem_Free(buf); /* allocated with PyMem_Malloc */
    return res;
\end{verbatim}

The same code using the type-oriented function set:

\begin{verbatim}
    PyObject *res;
    char *buf = PyMem_New(char, BUFSIZ); /* for I/O */

    if (buf == NULL)
        return PyErr_NoMemory();
    /* ...Do some I/O operation involving buf... */
    res = PyString_FromString(buf);
    PyMem_Del(buf); /* allocated with PyMem_New */
    return res;
\end{verbatim}

Note that in the two examples above, the buffer is always
manipulated via functions belonging to the same set. Indeed, it
is required to use the same memory API family for a given
memory block, so that the risk of mixing different allocators is
reduced to a minimum. The following code sequence contains two errors,
one of which is labeled as \emph{fatal} because it mixes two different
allocators operating on different heaps.

\begin{verbatim}
char *buf1 = PyMem_New(char, BUFSIZ);
char *buf2 = (char *) malloc(BUFSIZ);
char *buf3 = (char *) PyMem_Malloc(BUFSIZ);
...
PyMem_Del(buf3);  /* Wrong -- should be PyMem_Free() */
free(buf2);       /* Right -- allocated via malloc() */
free(buf1);       /* Fatal -- should be PyMem_Del()  */
\end{verbatim}

In addition to the functions aimed at handling raw memory blocks from
the Python heap, objects in Python are allocated and released with
\cfunction{PyObject_New()}, \cfunction{PyObject_NewVar()} and
\cfunction{PyObject_Del()}, or with their corresponding macros
\cfunction{PyObject_NEW()}, \cfunction{PyObject_NEW_VAR()} and
\cfunction{PyObject_DEL()}.

These will be explained in the next chapter on defining and
implementing new object types in C.

\chapter{Defining New Types
        \label{defining-new-types}}
\sectionauthor{Michael Hudson}{mwh@python.net}
\sectionauthor{Dave Kuhlman}{dkuhlman@rexx.com}
\sectionauthor{Jim Fulton}{jim@zope.com}

As mentioned in the last chapter, Python allows the writer of an
extension module to define new types that can be manipulated from
Python code, much like strings and lists in core Python.

This is not hard; the code for all extension types follows a pattern,
but there are some details that you need to understand before you can
get started.

\section{The Basics
    \label{dnt-basics}}

The Python runtime sees all Python objects as variables of type
\ctype{PyObject*}.  A \ctype{PyObject} is not a very magnificent
object - it just contains the refcount and a pointer to the object's
``type object''.  This is where the action is; the type object
determines which (C) functions get called when, for instance, an
attribute gets looked up on an object or it is multiplied by another
object.  These C functions are called ``type methods'' to distinguish
them from things like \code{[].append} (which we call ``object
methods'').

So, if you want to define a new object type, you need to create a new
type object.

This sort of thing can only be explained by example, so here's a
minimal, but complete, module that defines a new type:

\verbatiminput{noddy.c}

Now that's quite a bit to take in at once, but hopefully bits will
seem familiar from the last chapter.

The first bit that will be new is:

\begin{verbatim}
typedef struct {
    PyObject_HEAD
} noddy_NoddyObject;
\end{verbatim}

This is what a Noddy object will contain---in this case, nothing more
than every Python object contains, namely a refcount and a pointer to a type
object.  These are the fields the \code{PyObject_HEAD} macro brings
in.  The reason for the macro is to standardize the layout and to
enable special debugging fields in debug builds.  Note that there is
no semicolon after the \code{PyObject_HEAD} macro; one is included in
the macro definition.  Be wary of adding one by accident; it's easy to
do from habit, and your compiler might not complain, but someone
else's probably will!  (On Windows, MSVC is known to call this an
error and refuse to compile the code.)

For contrast, let's take a look at the corresponding definition for
standard Python integers:

\begin{verbatim}
typedef struct {
    PyObject_HEAD
    long ob_ival;
} PyIntObject;
\end{verbatim}

Moving on, we come to the crunch --- the type object.

\begin{verbatim}
static PyTypeObject noddy_NoddyType = {
    PyObject_HEAD_INIT(NULL)
    0,                         /*ob_size*/
    "noddy.Noddy",             /*tp_name*/
    sizeof(noddy_NoddyObject), /*tp_basicsize*/
    0,                         /*tp_itemsize*/
    0,                         /*tp_dealloc*/
    0,                         /*tp_print*/
    0,                         /*tp_getattr*/
    0,                         /*tp_setattr*/
    0,                         /*tp_compare*/
    0,                         /*tp_repr*/
    0,                         /*tp_as_number*/
    0,                         /*tp_as_sequence*/
    0,                         /*tp_as_mapping*/
    0,                         /*tp_hash */
    0,                         /*tp_call*/
    0,                         /*tp_str*/
    0,                         /*tp_getattro*/
    0,                         /*tp_setattro*/
    0,                         /*tp_as_buffer*/
    Py_TPFLAGS_DEFAULT,        /*tp_flags*/
    "Noddy objects",           /* tp_doc */
    0,		               /* tp_traverse */
    0,		               /* tp_clear */
    0,		               /* tp_richcompare */
    0,		               /* tp_weaklistoffset */
    0,		               /* tp_iter */
    0,		               /* tp_iternext */
    0,		               /* tp_methods */
    0,                         /* tp_members */
    0,                         /* tp_getset */
    0,                         /* tp_base */
    0,                         /* tp_dict */
    0,                         /* tp_descr_get */
    0,                         /* tp_descr_set */
    0,                         /* tp_dictoffset */
    0,                         /* tp_init */
    0,                         /* tp_alloc */
    PyType_GenericNew,         /* tp_new */
};
\end{verbatim}

Now if you go and look up the definition of \ctype{PyTypeObject} in
\file{object.h} you'll see that it has many more fields that the
definition above.  The remaining fields will be filled with zeros by
the C compiler, and it's common practice to not specify them
explicitly unless you need them.  

This is so important that we're going to pick the top of it apart still
further:

\begin{verbatim}
    PyObject_HEAD_INIT(NULL)
\end{verbatim}

This line is a bit of a wart; what we'd like to write is:

\begin{verbatim}
    PyObject_HEAD_INIT(&PyType_Type)
\end{verbatim}

as the type of a type object is ``type'', but this isn't strictly
conforming C and some compilers complain.  Fortunately, this member
will be filled in for us by \cfunction{PyType_Ready()}.

\begin{verbatim}
    0,                          /* ob_size */
\end{verbatim}

The \member{ob_size} field of the header is not used; its presence in
the type structure is a historical artifact that is maintained for
binary compatibility with extension modules compiled for older
versions of Python.  Always set this field to zero.

\begin{verbatim}
    "noddy.Noddy",              /* tp_name */
\end{verbatim}

The name of our type.  This will appear in the default textual
representation of our objects and in some error messages, for example:

\begin{verbatim}
>>> "" + noddy.new_noddy()
Traceback (most recent call last):
  File "<stdin>", line 1, in ?
TypeError: cannot add type "noddy.Noddy" to string
\end{verbatim}

Note that the name is a dotted name that includes both the module name
and the name of the type within the module. The module in this case is 
\module{noddy} and the type is \class{Noddy}, so we set the type name
to \class{noddy.Noddy}.

\begin{verbatim}
    sizeof(noddy_NoddyObject),  /* tp_basicsize */
\end{verbatim}

This is so that Python knows how much memory to allocate when you call
\cfunction{PyObject_New}.

\begin{verbatim}
    0,                          /* tp_itemsize */
\end{verbatim}

This has to do with variable length objects like lists and strings.
Ignore this for now.

Skipping a number of type methods that we don't provide, we set the
class flags to \constant{Py_TPFLAGS_DEFAULT}. 

\begin{verbatim}
    Py_TPFLAGS_DEFAULT,        /*tp_flags*/
\end{verbatim}

All types should include this constant in their flags.  It enables all
of the members defined by the current version of Python.

We provide a doc string for the type in \member{tp_doc}.

\begin{verbatim}
    "Noddy objects",           /* tp_doc */
\end{verbatim}

Now we get into the type methods, the things that make your objects
different from the others.  We aren't going to implement any of these
in this version of the module.  We'll expand this example later to 
have more interesting behavior.  

For now, all we want to be able to do is to create new \class{Noddy}
objects. To enable object creation, we have to provide a
\member{tp_new} implementation. In this case, we can just use the
default implementation provided by the API function
\cfunction{PyType_GenericNew}.

\begin{verbatim}
    PyType_GenericNew,         /* tp_new */
\end{verbatim}

All the other type methods are \NULL, so we'll go over them later
--- that's for a later section!

Everything else in the file should be familiar, except for some code
in \cfunction{initnoddy}:

\begin{verbatim}
    if (PyType_Ready(&noddy_NoddyType) < 0)
        return;
\end{verbatim}

This initializes the \class{Noddy} type, filing in a number of
members, including \member{ob_type} that we initially set to \NULL.

\begin{verbatim}
    PyModule_AddObject(m, "Noddy", (PyObject *)&noddy_NoddyType);
\end{verbatim}

This adds the type to the module dictionary.  This allows us to create
\class{Noddy} instances by calling the \class{Noddy} class:

\begin{verbatim}
import noddy
mynoddy = noddy.Noddy()
\end{verbatim}

That's it!  All that remains is to build it; put the above code in a
file called \file{noddy.c} and

\begin{verbatim}
from distutils.core import setup, Extension
setup(name="noddy", version="1.0",
      ext_modules=[Extension("noddy", ["noddy.c"])])
\end{verbatim}

in a file called \file{setup.py}; then typing

\begin{verbatim}
$ python setup.py build
\end{verbatim} %$ <-- bow to font-lock  ;-(

at a shell should produce a file \file{noddy.so} in a subdirectory;
move to that directory and fire up Python --- you should be able to
\code{import noddy} and play around with Noddy objects.

That wasn't so hard, was it?

Of course, the current Noddy type is pretty uninteresting. It has no
data and doesn't do anything. It can't even be subclasses.

\subsection{Adding data and methods to the Basic example}
    
Let's expend the basic example to add some data and methods.  Let's
also make the type usable as a base class. We'll create
a new module, \module{noddy2} that adds these capabilities:

\verbatiminput{noddy2.c}

This version of the module has a number of changes.

We've added an extra include:

\begin{verbatim}
#include "structmember.h"
\end{verbatim}

This include provides declarations that we use to handle attributes,
as described a bit later.

The name of the \class{Noddy} object structure has been shortened to
\class{Noddy}.  The type object name has been shortened to
\class{NoddyType}.

The  \class{Noddy} type now has three data attributes, \var{first},
\var{last}, and \var{number}.  The \var{first} and \var{last}
variables are Python strings containing first and last names. The
\var{number} attribute is an integer.

The object structure is updated accordingly:

\begin{verbatim}
typedef struct {
    PyObject_HEAD
    PyObject *first;
    PyObject *last;
    int number;
} Noddy;
\end{verbatim}

Because we now have data to manage, we have to be more careful about
object allocation and deallocation.  At a minimum, we need a
deallocation method:

\begin{verbatim}
static void
Noddy_dealloc(Noddy* self)
{
    Py_XDECREF(self->first);
    Py_XDECREF(self->last);
    self->ob_type->tp_free(self);
}
\end{verbatim}

which is assigned to the \member{tp_dealloc} member:

\begin{verbatim}
    (destructor)Noddy_dealloc, /*tp_dealloc*/
\end{verbatim}

This method decrements the reference counts of the two Python
attributes. We use \cfunction{Py_XDECREF} here because the
\member{first} and \member{last} members could be \NULL.  It then
calls the \member{tp_free} member of the object's type to free the
object's memory.  Note that the object's type might not be
\class{NoddyType}, because the object may be an instance of a
subclass.

We want to make sure that the first and last names are initialized to
empty strings, so we provide a new method:

\begin{verbatim}
static PyObject *
Noddy_new(PyTypeObject *type, PyObject *args, PyObject *kwds)
{
    Noddy *self;

    self = (Noddy *)type->tp_alloc(type, 0);
    if (self != NULL) {
        self->first = PyString_FromString("");
        if (self->first == NULL)
          {
            Py_DECREF(self);
            return NULL;
          }
        
        self->last = PyString_FromString("");
        if (self->last == NULL)
          {
            Py_DECREF(self);
            return NULL;
          }

        self->number = 0;
    }

    return (PyObject *)self;
}
\end{verbatim}

and install it in the \member{tp_new} member:

\begin{verbatim}
    Noddy_new,                 /* tp_new */
\end{verbatim}

The new member is responsible for creating (as opposed to
initializing) objects of the type.  It is exposed in Python as the
\method{__new__} method.  See the paper titled ``Unifying types and
classes in Python'' for a detailed discussion of the \method{__new__}
method.  One reason to implement a new method is to assure the initial
values of instance variables.  In this case, we use the new method to
make sure that the initial values of the members \member{first} and
\member{last} are not \NULL. If we didn't care whether the initial
values were \NULL, we could have used \cfunction{PyType_GenericNew} as
our new method, as we did before.  \cfunction{PyType_GenericNew}
initializes all of the instance variable members to NULLs.

The new method is a static method that is passed the type being
instantiated and any arguments passed when the type was called,
and that returns the new object created. New methods always accept
positional and keyword arguments, but they often ignore the arguments,
leaving the argument handling to initializer methods. Note that if the
type supports subclassing, the type passed may not be the type being
defined.  The new method calls the tp_alloc slot to allocate memory.
We don't fill the \member{tp_alloc} slot ourselves. Rather
\cfunction{PyType_Ready()} fills it for us by inheriting it from our
base class, which is \class{object} by default.  Most types use the
default allocation.

We provide an initialization function:

\begin{verbatim}
static PyObject *
Noddy_init(Noddy *self, PyObject *args, PyObject *kwds)
{
    PyObject *first=NULL, *last=NULL;

    static char *kwlist[] = {"first", "last", "number", NULL};

    if (! PyArg_ParseTupleAndKeywords(args, kwds, "|OOi", kwlist, 
                                      &first, &last, 
                                      &self->number))
        return NULL; 

    if (first) {
        Py_XDECREF(self->first);
        Py_INCREF(first);
        self->first = first;
    }

    if (last) {
        Py_XDECREF(self->last);
        Py_INCREF(last);
        self->last = last;
    }

    Py_INCREF(Py_None);
    return Py_None;
}
\end{verbatim}

by filling the \member{tp_init} slot.

\begin{verbatim}
    (initproc)Noddy_init,         /* tp_init */
\end{verbatim}

The \member{tp_init} slot is exposed in Python as the
\method{__init__} method. It is used to initialize an object after
it's created. Unlike the new method, we can't guarantee that the
initializer is called.  The initializer isn't called when unpickling
objects and it can be overridden.  Our initializer accepts arguments
to provide initial values for our instance. Initializers always accept
positional and keyword arguments.

We want to want to expose our instance variables as attributes. There
are a number of ways to do that. The simplest way is to define member
definitions:

\begin{verbatim}
static PyMemberDef Noddy_members[] = {
    {"first", T_OBJECT_EX, offsetof(Noddy, first), 0,
     "first name"},
    {"last", T_OBJECT_EX, offsetof(Noddy, last), 0,
     "last name"},
    {"number", T_INT, offsetof(Noddy, number), 0,
     "noddy number"},
    {NULL}  /* Sentinel */
};
\end{verbatim}

and put the definitions in the \member{tp_members} slot:

\begin{verbatim}
    Noddy_members,             /* tp_members */
\end{verbatim}

Each member definition has a member name, type, offset, access flags
and documentation string. See the ``Generic Attribute Management''
section below for details.

A disadvantage of this approach is that it doesn't provide a way to
restrict the types of objects that can be assigned to the Python
attributes.  We expect the first and last names to be strings, but any
Python objects can be assigned.  Further, the attributes can be
deleted, setting the C pointers to \NULL.  Even though we can make
sure the members are initialized to non-\NULL values, the members can
be set to \NULL if the attributes are deleted.

We define a single method, \method{name}, that outputs the objects
name as the concatenation of the first and last names.  

\begin{verbatim}
static PyObject *
Noddy_name(Noddy* self)
{
    static PyObject *format = NULL;
    PyObject *args, *result;

    if (format == NULL) {
        format = PyString_FromString("%s %s");
        if (format == NULL)
            return NULL;
    }

    if (self->first == NULL) {
        PyErr_SetString(PyExc_AttributeError, "first");
        return NULL;
    }

    if (self->last == NULL) {
        PyErr_SetString(PyExc_AttributeError, "last");
        return NULL;
    }

    args = Py_BuildValue("OO", self->first, self->last);
    if (args == NULL)
        return NULL;

    result = PyString_Format(format, args);
    Py_DECREF(args);
    
    return result;
}
\end{verbatim}

The method is implemented as a C function that takes a \class{Noddy} (or
\class{Noddy} subclass) instance as the first argument.  Methods
always take an instance as the first argument. Methods often take
positional and keyword arguments as well, but in this cased we don't
take any and don't need to accept a positional argument tuple or
keyword argument dictionary. This method is equivalent to the Python
method:

\begin{verbatim}
    def name(self):
       return "%s %s" % (self.first, self.last)
\end{verbatim}

Note that we have to check for the possibility that our \member{first}
and \member{last} members are \NULL.  This is because they can be
deleted, in which case they are set to \NULL.  It would be better to
prevent deletion of these attributes and to restrict the attribute
values to be strings.  We'll see how to do that in the next section.

Now that we've defined the method, we need to create an array of
method definitions:

\begin{verbatim}
static PyMethodDef Noddy_methods[] = {
    {"name", (PyCFunction)Noddy_name, METH_NOARGS,
     "Return the name, combining the first and last name"
    },
    {NULL}  /* Sentinel */
};
\end{verbatim}

and assign them to the \member{tp_methods} slot:

\begin{verbatim}
    Noddy_methods,             /* tp_methods */
\end{verbatim}

Note that used the \constant{METH_NOARGS} flag to indicate that the
method is passed no arguments.

Finally, we'll make our type usable as a base class.  We've written
our methods carefully so far so that they don't make any assumptions
about the type of the object being created or used, so all we need to
do is to add the \constant{Py_TPFLAGS_BASETYPE} to our class flag
definition:

\begin{verbatim}
    Py_TPFLAGS_DEFAULT | Py_TPFLAGS_BASETYPE, /*tp_flags*/
\end{verbatim}

We rename \cfunction{initnoddy} to \cfunction{initnoddy2}
and update the module name passed to \cfunction{Py_InitModule3}.

Finally, we update our \file{setup.py} file to build the new module:

\begin{verbatim}
from distutils.core import setup, Extension
setup(name="noddy", version="1.0",
      ext_modules=[
         Extension("noddy", ["noddy.c"]),
         Extension("noddy2", ["noddy2.c"]),
         ])
\end{verbatim}

\subsection{Providing finer control over data attributes}

In this section, we'll provide finer control over how the
\member{first} and \member{last} attributes are set in the
\class{Noddy} example. In the previous version of our module, the
instance variables \member{first} and \member{last} could be set to
non-string values or even deleted. We want to make sure that these
attributes always contain strings.

\verbatiminput{noddy3.c}

To provide greater control, over the \member{first} and \member{last}
attributes, we'll use custom getter and setter functions.  Here are
the functions for getting and setting the \member{first} attribute:

\begin{verbatim}
Noddy_getfirst(Noddy *self, void *closure)
{
    Py_INCREF(self->first);
    return self->first;
}

static int
Noddy_setfirst(Noddy *self, PyObject *value, void *closure)
{
  if (value == NULL) {
    PyErr_SetString(PyExc_TypeError, "Cannot delete the first attribute");
    return -1;
  }
  
  if (! PyString_Check(value)) {
    PyErr_SetString(PyExc_TypeError, 
                    "The first attribute value must be a string");
    return -1;
  }
      
  Py_DECREF(self->first);
  Py_INCREF(value);
  self->first = value;    

  return 0;
}
\end{verbatim}

The getter function is passed a \class{Noddy} object and a
``closure'', which is void pointer. In this case, the closure is
ignored. (The closure supports an advanced usage in which definition
data is passed to the getter and setter. This could, for example, be
used to allow a single set of getter and setter functions that decide
the attribute to get or set based on data in the closure.)

The setter function is passed the \class{Noddy} object, the new value,
and the closure. The new value may be \NULL, in which case the
attribute is being deleted.  In our setter, we raise an error if the
attribute is deleted or if the attribute value is not a string.

We create an array of \ctype{PyGetSetDef} structures:

\begin{verbatim}
static PyGetSetDef Noddy_getseters[] = {
    {"first", 
     (getter)Noddy_getfirst, (setter)Noddy_setfirst,
     "first name",
     NULL},
    {"last", 
     (getter)Noddy_getlast, (setter)Noddy_setlast,
     "last name",
     NULL},
    {NULL}  /* Sentinel */
};
\end{verbatim}

and register it in the \member{tp_getset} slot:

\begin{verbatim}
    Noddy_getseters,           /* tp_getset */
\end{verbatim}

to register out attribute getters and setters.  

The last item in a \ctype{PyGetSetDef} structure is the closure
mentioned above. In this case, we aren't using the closure, so we just
pass \NULL.

We also remove the member definitions for these attributes:

\begin{verbatim}
static PyMemberDef Noddy_members[] = {
    {"number", T_INT, offsetof(Noddy, number), 0,
     "noddy number"},
    {NULL}  /* Sentinel */
};
\end{verbatim}

With these changes, we can assure that the \member{first} and
\member{last} members are never NULL so we can remove checks for \NULL
values in almost all cases. This means that most of the
\cfunction{Py_XDECREF} calls can be converted to \cfunction{Py_DECREF}
calls. The only place we can't change these calls is in the
deallocator, where there is the possibility that the initialization of
these members failed in the constructor.

We also rename the module initialization function and module name in
the initialization function, as we did before, and we add an extra
definition to the \file{setup.py} file.

\section{Type Methods
         \label{dnt-type-methods}}

This section aims to give a quick fly-by on the various type methods
you can implement and what they do.

Here is the definition of \ctype{PyTypeObject}, with some fields only
used in debug builds omitted:

\verbatiminput{typestruct.h}

Now that's a \emph{lot} of methods.  Don't worry too much though - if
you have a type you want to define, the chances are very good that you
will only implement a handful of these.

As you probably expect by now, we're going to go over this and give
more information about the various handlers.  We won't go in the order
they are defined in the structure, because there is a lot of
historical baggage that impacts the ordering of the fields; be sure
your type initializaion keeps the fields in the right order!  It's
often easiest to find an example that includes all the fields you need
(even if they're initialized to \code{0}) and then change the values
to suit your new type.

\begin{verbatim}
    char *tp_name; /* For printing */
\end{verbatim}

The name of the type - as mentioned in the last section, this will
appear in various places, almost entirely for diagnostic purposes.
Try to choose something that will be helpful in such a situation!

\begin{verbatim}
    int tp_basicsize, tp_itemsize; /* For allocation */
\end{verbatim}

These fields tell the runtime how much memory to allocate when new
objects of this type are created.  Python has some builtin support
for variable length structures (think: strings, lists) which is where
the \member{tp_itemsize} field comes in.  This will be dealt with
later.

\begin{verbatim}
    char *tp_doc;
\end{verbatim}

Here you can put a string (or its address) that you want returned when
the Python script references \code{obj.__doc__} to retrieve the
docstring.
   
Now we come to the basic type methods---the ones most extension types
will implement.


\subsection{Finalization and De-allocation}

\index{object!deallocation}
\index{deallocation, object}
\index{object!finalization}
\index{finalization, of objects}

\begin{verbatim}
    destructor tp_dealloc;
\end{verbatim}

This function is called when the reference count of the instance of
your type is reduced to zero and the Python interpreter wants to
reclaim it.  If your type has memory to free or other clean-up to
perform, put it here.  The object itself needs to be freed here as
well.  Here is an example of this function:

\begin{verbatim}
static void
newdatatype_dealloc(newdatatypeobject * obj)
{
    free(obj->obj_UnderlyingDatatypePtr);
    obj->ob_type->tp_free(self);
}
\end{verbatim}

One important requirement of the deallocator function is that it
leaves any pending exceptions alone.  This is important since
deallocators are frequently called as the interpreter unwinds the
Python stack; when the stack is unwound due to an exception (rather
than normal returns), nothing is done to protect the deallocators from
seeing that an exception has already been set.  Any actions which a
deallocator performs which may cause additional Python code to be
executed may detect that an exception has been set.  This can lead to
misleading errors from the interpreter.  The proper way to protect
against this is to save a pending exception before performing the
unsafe action, and restoring it when done.  This can be done using the
\cfunction{PyErr_Fetch()}\ttindex{PyErr_Fetch()} and
\cfunction{PyErr_Restore()}\ttindex{PyErr_Restore()} functions:

\begin{verbatim}
static void
my_dealloc(PyObject *obj)
{
    MyObject *self = (MyObject *) obj;
    PyObject *cbresult;

    if (self->my_callback != NULL) {
        PyObject *err_type, *err_value, *err_traceback;
        int have_error = PyErr_Occurred() ? 1 : 0;

        if (have_error)
            PyErr_Fetch(&err_type, &err_value, &err_traceback);

        cbresult = PyObject_CallObject(self->my_callback, NULL);
        if (cbresult == NULL)
            PyErr_WriteUnraisable();
        else
            Py_DECREF(cbresult);

        if (have_error)
            PyErr_Restore(err_type, err_value, err_traceback);

        Py_DECREF(self->my_callback);
    }
    obj->ob_type->tp_free(self);
}
\end{verbatim}


\subsection{Object Presentation}

In Python, there are three ways to generate a textual representation
of an object: the \function{repr()}\bifuncindex{repr} function (or
equivalent backtick syntax), the \function{str()}\bifuncindex{str}
function, and the \keyword{print} statement.  For most objects, the
\keyword{print} statement is equivalent to the \function{str()}
function, but it is possible to special-case printing to a
\ctype{FILE*} if necessary; this should only be done if efficiency is
identified as a problem and profiling suggests that creating a
temporary string object to be written to a file is too expensive.

These handlers are all optional, and most types at most need to
implement the \member{tp_str} and \member{tp_repr} handlers.

\begin{verbatim}
    reprfunc tp_repr;
    reprfunc tp_str;
    printfunc tp_print;
\end{verbatim}

The \member{tp_repr} handler should return a string object containing
a representation of the instance for which it is called.  Here is a
simple example:

\begin{verbatim}
static PyObject *
newdatatype_repr(newdatatypeobject * obj)
{
    return PyString_FromFormat("Repr-ified_newdatatype{{size:\%d}}",
                               obj->obj_UnderlyingDatatypePtr->size);
}
\end{verbatim}

If no \member{tp_repr} handler is specified, the interpreter will
supply a representation that uses the type's \member{tp_name} and a
uniquely-identifying value for the object.

The \member{tp_str} handler is to \function{str()} what the
\member{tp_repr} handler described above is to \function{repr()}; that
is, it is called when Python code calls \function{str()} on an
instance of your object.  Its implementation is very similar to the
\member{tp_repr} function, but the resulting string is intended for
human consumption.  If \member{tp_str} is not specified, the
\member{tp_repr} handler is used instead.

Here is a simple example:

\begin{verbatim}
static PyObject *
newdatatype_str(newdatatypeobject * obj)
{
    return PyString_FromFormat("Stringified_newdatatype{{size:\%d}}",
                               obj->obj_UnderlyingDatatypePtr->size);
}
\end{verbatim}

The print function will be called whenever Python needs to "print" an
instance of the type.  For example, if 'node' is an instance of type
TreeNode, then the print function is called when Python code calls:

\begin{verbatim}
print node
\end{verbatim}

There is a flags argument and one flag, \constant{Py_PRINT_RAW}, and
it suggests that you print without string quotes and possibly without
interpreting escape sequences.

The print function receives a file object as an argument. You will
likely want to write to that file object.

Here is a sampe print function:

\begin{verbatim}
static int
newdatatype_print(newdatatypeobject *obj, FILE *fp, int flags)
{
    if (flags & Py_PRINT_RAW) {
        fprintf(fp, "<{newdatatype object--size: %d}>",
                obj->obj_UnderlyingDatatypePtr->size);
    }
    else {
        fprintf(fp, "\"<{newdatatype object--size: %d}>\"",
                obj->obj_UnderlyingDatatypePtr->size);
    }
    return 0;
}
\end{verbatim}


\subsection{Attribute Management}

For every object which can support attributes, the corresponding type
must provide the functions that control how the attributes are
resolved.  There needs to be a function which can retrieve attributes
(if any are defined), and another to set attributes (if setting
attributes is allowed).  Removing an attribute is a special case, for
which the new value passed to the handler is \NULL.

Python supports two pairs of attribute handlers; a type that supports
attributes only needs to implement the functions for one pair.  The
difference is that one pair takes the name of the attribute as a
\ctype{char*}, while the other accepts a \ctype{PyObject*}.  Each type
can use whichever pair makes more sense for the implementation's
convenience.

\begin{verbatim}
    getattrfunc  tp_getattr;        /* char * version */
    setattrfunc  tp_setattr;
    /* ... */
    getattrofunc tp_getattrofunc;   /* PyObject * version */
    setattrofunc tp_setattrofunc;
\end{verbatim}

If accessing attributes of an object is always a simple operation
(this will be explained shortly), there are generic implementations
which can be used to provide the \ctype{PyObject*} version of the
attribute management functions.  The actual need for type-specific
attribute handlers almost completely disappeared starting with Python
2.2, though there are many examples which have not been updated to use
some of the new generic mechanism that is available.


\subsubsection{Generic Attribute Management}

\versionadded{2.2}

Most extension types only use \emph{simple} attributes.  So, what
makes the attributes simple?  There are only a couple of conditions
that must be met:

\begin{enumerate}
  \item   The name of the attributes must be known when
          \cfunction{PyType_Ready()} is called.

  \item   No special processing is needed to record that an attribute
          was looked up or set, nor do actions need to be taken based
          on the value.
\end{enumerate}

Note that this list does not place any restrictions on the values of
the attributes, when the values are computed, or how relevant data is
stored.

When \cfunction{PyType_Ready()} is called, it uses three tables
referenced by the type object to create \emph{descriptors} which are
placed in the dictionary of the type object.  Each descriptor controls
access to one attribute of the instance object.  Each of the tables is
optional; if all three are \NULL, instances of the type will only have
attributes that are inherited from their base type, and should leave
the \member{tp_getattro} and \member{tp_setattro} fields \NULL{} as
well, allowing the base type to handle attributes.

The tables are declared as three fields of the type object:

\begin{verbatim}
    struct PyMethodDef *tp_methods;
    struct PyMemberDef *tp_members;
    struct PyGetSetDef *tp_getset;
\end{verbatim}

If \member{tp_methods} is not \NULL, it must refer to an array of
\ctype{PyMethodDef} structures.  Each entry in the table is an
instance of this structure:

\begin{verbatim}
typedef struct PyMethodDef {
    char        *ml_name;       /* method name */
    PyCFunction  ml_meth;       /* implementation function */
    int	         ml_flags;      /* flags */
    char        *ml_doc;        /* docstring */
} PyMethodDef;
\end{verbatim}

One entry should be defined for each method provided by the type; no
entries are needed for methods inherited from a base type.  One
additional entry is needed at the end; it is a sentinel that marks the
end of the array.  The \member{ml_name} field of the sentinel must be
\NULL.

XXX Need to refer to some unified discussion of the structure fields,
shared with the next section.

The second table is used to define attributes which map directly to
data stored in the instance.  A variety of primitive C types are
supported, and access may be read-only or read-write.  The structures
in the table are defined as:

\begin{verbatim}
typedef struct PyMemberDef {
    char *name;
    int   type;
    int   offset;
    int   flags;
    char *doc;
} PyMemberDef;
\end{verbatim}

For each entry in the table, a descriptor will be constructed and
added to the type which will be able to extract a value from the
instance structure.  The \member{type} field should contain one of the
type codes defined in the \file{structmember.h} header; the value will
be used to determine how to convert Python values to and from C
values.  The \member{flags} field is used to store flags which control
how the attribute can be accessed.

XXX Need to move some of this to a shared section!

The following flag constants are defined in \file{structmember.h};
they may be combined using bitwise-OR.

\begin{tableii}{l|l}{constant}{Constant}{Meaning}
  \lineii{READONLY \ttindex{READONLY}}
         {Never writable.}
  \lineii{RO \ttindex{RO}}
         {Shorthand for \constant{READONLY}.}
  \lineii{READ_RESTRICTED \ttindex{READ_RESTRICTED}}
         {Not readable in restricted mode.}
  \lineii{WRITE_RESTRICTED \ttindex{WRITE_RESTRICTED}}
         {Not writable in restricted mode.}
  \lineii{RESTRICTED \ttindex{RESTRICTED}}
         {Not readable or writable in restricted mode.}
\end{tableii}

An interesting advantage of using the \member{tp_members} table to
build descriptors that are used at runtime is that any attribute
defined this way can have an associated docstring simply by providing
the text in the table.  An application can use the introspection API
to retrieve the descriptor from the class object, and get the
docstring using its \member{__doc__} attribute.

As with the \member{tp_methods} table, a sentinel entry with a
\member{name} value of \NULL{} is required.  


% XXX Descriptors need to be explained in more detail somewhere, but
% not here.
%
% Descriptor objects have two handler functions which correspond to
% the \member{tp_getattro} and \member{tp_setattro} handlers.  The
% \method{__get__()} handler is a function which is passed the
% descriptor, instance, and type objects, and returns the value of the
% attribute, or it returns \NULL{} and sets an exception.  The
% \method{__set__()} handler is passed the descriptor, instance, type,
% and new value;


\subsubsection{Type-specific Attribute Management}

For simplicity, only the \ctype{char*} version will be demonstrated
here; the type of the name parameter is the only difference between
the \ctype{char*} and \ctype{PyObject*} flavors of the interface.
This example effectively does the same thing as the generic example
above, but does not use the generic support added in Python 2.2.  The
value in showing this is two-fold: it demonstrates how basic attribute
management can be done in a way that is portable to older versions of
Python, and explains how the handler functions are called, so that if
you do need to extend their functionality, you'll understand what
needs to be done.

The \member{tp_getattr} handler is called when the object requires an
attribute look-up.  It is called in the same situations where the
\method{__getattr__()} method of a class would be called.

A likely way to handle this is (1) to implement a set of functions
(such as \cfunction{newdatatype_getSize()} and
\cfunction{newdatatype_setSize()} in the example below), (2) provide a
method table listing these functions, and (3) provide a getattr
function that returns the result of a lookup in that table.  The
method table uses the same structure as the \member{tp_methods} field
of the type object.

Here is an example:

\begin{verbatim}
static PyMethodDef newdatatype_methods[] = {
    {"getSize", (PyCFunction)newdatatype_getSize, METH_VARARGS,
     "Return the current size."},
    {"setSize", (PyCFunction)newdatatype_setSize, METH_VARARGS,
     "Set the size."},
    {NULL, NULL, 0, NULL}           /* sentinel */
};

static PyObject *
newdatatype_getattr(newdatatypeobject *obj, char *name)
{
    return Py_FindMethod(newdatatype_methods, (PyObject *)obj, name);
}
\end{verbatim}

The \member{tp_setattr} handler is called when the
\method{__setattr__()} or \method{__delattr__()} method of a class
instance would be called.  When an attribute should be deleted, the
third parameter will be \NULL.  Here is an example that simply raises
an exception; if this were really all you wanted, the
\member{tp_setattr} handler should be set to \NULL.
   
\begin{verbatim}
static int
newdatatype_setattr(newdatatypeobject *obj, char *name, PyObject *v)
{
    (void)PyErr_Format(PyExc_RuntimeError, "Read-only attribute: \%s", name);
    return -1;
}
\end{verbatim}


\subsection{Object Comparison}

\begin{verbatim}
    cmpfunc tp_compare;
\end{verbatim}

The \member{tp_compare} handler is called when comparisons are needed
and the object does not implement the specific rich comparison method
which matches the requested comparison.  (It is always used if defined
and the \cfunction{PyObject_Compare()} or \cfunction{PyObject_Cmp()}
functions are used, or if \function{cmp()} is used from Python.)
It is analogous to the \method{__cmp__()} method.  This function
should return \code{-1} if \var{obj1} is less than
\var{obj2}, \code{0} if they are equal, and \code{1} if
\var{obj1} is greater than
\var{obj2}.
(It was previously allowed to return arbitrary negative or positive
integers for less than and greater than, respectively; as of Python
2.2, this is no longer allowed.  In the future, other return values
may be assigned a different meaning.)

A \member{tp_compare} handler may raise an exception.  In this case it
should return a negative value.  The caller has to test for the
exception using \cfunction{PyErr_Occurred()}.


Here is a sample implementation:

\begin{verbatim}
static int
newdatatype_compare(newdatatypeobject * obj1, newdatatypeobject * obj2)
{
    long result;
 
    if (obj1->obj_UnderlyingDatatypePtr->size <
        obj2->obj_UnderlyingDatatypePtr->size) {
        result = -1;
    }
    else if (obj1->obj_UnderlyingDatatypePtr->size >
             obj2->obj_UnderlyingDatatypePtr->size) {
        result = 1;
    }
    else {
        result = 0;
    }
    return result;
}
\end{verbatim}


\subsection{Abstract Protocol Support}

Python supports a variety of \emph{abstract} `protocols;' the specific
interfaces provided to use these interfaces are documented in the
\citetitle[../api/api.html]{Python/C API Reference Manual} in the
chapter ``\ulink{Abstract Objects Layer}{../api/abstract.html}.''

A number of these abstract interfaces were defined early in the
development of the Python implementation.  In particular, the number,
mapping, and sequence protocols have been part of Python since the
beginning.  Other protocols have been added over time.  For protocols
which depend on several handler routines from the type implementation,
the older protocols have been defined as optional blocks of handlers
referenced by the type object.  For newer protocols there are
additional slots in the main type object, with a flag bit being set to
indicate that the slots are present and should be checked by the
interpreter.  (The flag bit does not indicate that the slot values are
non-\NULL. The flag may be set to indicate the presense of a slot,
but a slot may still be unfilled.)

\begin{verbatim}
    PyNumberMethods   tp_as_number;
    PySequenceMethods tp_as_sequence;
    PyMappingMethods  tp_as_mapping;
\end{verbatim}

If you wish your object to be able to act like a number, a sequence,
or a mapping object, then you place the address of a structure that
implements the C type \ctype{PyNumberMethods},
\ctype{PySequenceMethods}, or \ctype{PyMappingMethods}, respectively.
It is up to you to fill in this structure with appropriate values. You
can find examples of the use of each of these in the \file{Objects}
directory of the Python source distribution.


\begin{verbatim}
    hashfunc tp_hash;
\end{verbatim}

This function, if you choose to provide it, should return a hash
number for an instance of your datatype. Here is a moderately
pointless example:

\begin{verbatim}
static long
newdatatype_hash(newdatatypeobject *obj)
{
    long result;
    result = obj->obj_UnderlyingDatatypePtr->size;
    result = result * 3;
    return result;
}
\end{verbatim}

\begin{verbatim}
    ternaryfunc tp_call;
\end{verbatim}

This function is called when an instance of your datatype is "called",
for example, if \code{obj1} is an instance of your datatype and the Python
script contains \code{obj1('hello')}, the \member{tp_call} handler is
invoked.

This function takes three arguments:

\begin{enumerate}
  \item
    \var{arg1} is the instance of the datatype which is the subject of
    the call. If the call is \code{obj1('hello')}, then \var{arg1} is
    \code{obj1}.

  \item
    \var{arg2} is a tuple containing the arguments to the call.  You
    can use \cfunction{PyArg_ParseTuple()} to extract the arguments.

  \item
    \var{arg3} is a dictionary of keyword arguments that were passed.
    If this is non-\NULL{} and you support keyword arguments, use
    \cfunction{PyArg_ParseTupleAndKeywords()} to extract the
    arguments.  If you do not want to support keyword arguments and
    this is non-\NULL, raise a \exception{TypeError} with a message
    saying that keyword arguments are not supported.
\end{enumerate}
       
Here is a desultory example of the implementation of the call function.

\begin{verbatim}
/* Implement the call function.
 *    obj1 is the instance receiving the call.
 *    obj2 is a tuple containing the arguments to the call, in this
 *         case 3 strings.
 */
static PyObject *
newdatatype_call(newdatatypeobject *obj, PyObject *args, PyObject *other)
{
    PyObject *result;
    char *arg1;
    char *arg2;
    char *arg3;

    if (!PyArg_ParseTuple(args, "sss:call", &arg1, &arg2, &arg3)) {
        return NULL;
    }
    result = PyString_FromFormat(
        "Returning -- value: [\%d] arg1: [\%s] arg2: [\%s] arg3: [\%s]\n",
        obj->obj_UnderlyingDatatypePtr->size,
        arg1, arg2, arg3);
    printf("\%s", PyString_AS_STRING(result));
    return result;
}
\end{verbatim}

XXX some fields need to be added here...


\begin{verbatim}
    /* Added in release 2.2 */
    /* Iterators */
    getiterfunc tp_iter;
    iternextfunc tp_iternext;
\end{verbatim}

These functions provide support for the iterator protocol.  Any object
which wishes to support iteration over its contents (which may be
generated during iteration) must implement the \code{tp_iter}
handler.  Objects which are returned by a \code{tp_iter} handler must
implement both the \code{tp_iter} and \code{tp_iternext} handlers.
Both handlers take exactly one parameter, the instance for which they
are being called, and return a new reference.  In the case of an
error, they should set an exception and return \NULL.

For an object which represents an iterable collection, the
\code{tp_iter} handler must return an iterator object.  The iterator
object is responsible for maintaining the state of the iteration.  For
collections which can support multiple iterators which do not
interfere with each other (as lists and tuples do), a new iterator
should be created and returned.  Objects which can only be iterated
over once (usually due to side effects of iteration) should implement
this handler by returning a new reference to themselves, and should
also implement the \code{tp_iternext} handler.  File objects are an
example of such an iterator.

Iterator objects should implement both handlers.  The \code{tp_iter}
handler should return a new reference to the iterator (this is the
same as the \code{tp_iter} handler for objects which can only be
iterated over destructively).  The \code{tp_iternext} handler should
return a new reference to the next object in the iteration if there is
one.  If the iteration has reached the end, it may return \NULL{}
without setting an exception or it may set \exception{StopIteration};
avoiding the exception can yield slightly better performance.  If an
actual error occurs, it should set an exception and return \NULL.


\subsection{Supporting the Cycle Collector
            \label{example-cycle-support}}

This example shows only enough of the implementation of an extension
type to show how the garbage collector support needs to be added.  It
shows the definition of the object structure, the
\member{tp_traverse}, \member{tp_clear} and \member{tp_dealloc}
implementations, the type structure, and a constructor --- the module
initialization needed to export the constructor to Python is not shown
as there are no special considerations there for the collector.  To
make this interesting, assume that the module exposes ways for the
\member{container} field of the object to be modified.  Note that
since no checks are made on the type of the object used to initialize
\member{container}, we have to assume that it may be a container.

\verbatiminput{cycle-gc.c}

Full details on the APIs related to the cycle detector are in
\ulink{Supporting Cyclic Garbarge
Collection}{../api/supporting-cycle-detection.html} in the
\citetitle[../api/api.html]{Python/C API Reference Manual}.


\subsection{More Suggestions}

Remember that you can omit most of these functions, in which case you
provide \code{0} as a value.

In the \file{Objects} directory of the Python source distribution,
there is a file \file{xxobject.c}, which is intended to be used as a
template for the implementation of new types.  One useful strategy
for implementing a new type is to copy and rename this file, then
read the instructions at the top of it.

There are type definitions for each of the functions you must
provide.  They are in \file{object.h} in the Python include
directory that comes with the source distribution of Python.

In order to learn how to implement any specific method for your new
datatype, do the following: Download and unpack the Python source
distribution.  Go the the \file{Objects} directory, then search the
C source files for \code{tp_} plus the function you want (for
example, \code{tp_print} or \code{tp_compare}).  You will find
examples of the function you want to implement.

When you need to verify that the type of an object is indeed the
object you are implementing and if you use xxobject.c as an starting
template for your implementation, then there is a macro defined for
this purpose. The macro definition will look something like this:

\begin{verbatim}
#define is_newdatatypeobject(v)  ((v)->ob_type == &Newdatatypetype)
\end{verbatim}

And, a sample of its use might be something like the following:

\begin{verbatim}
    if (!is_newdatatypeobject(objp1) {
        PyErr_SetString(PyExc_TypeError, "arg #1 not a newdatatype");
        return NULL;
    }
\end{verbatim}



% \chapter{Debugging \label{debugging}}
%
% XXX Explain Py_DEBUG, Py_TRACE_REFS, Py_REF_DEBUG.


\appendix
\chapter{Reporting Bugs}
\label{reporting-bugs}

Python is a mature programming language which has established a
reputation for stability.  In order to maintain this reputation, the
developers would like to know of any deficiencies you find in Python
or its documentation.

All bug reports should be submitted via the Python Bug Tracker on
SourceForge (\url{http://sourceforge.net/bugs/?group_id=5470}).  The
bug tracker offers a Web form which allows pertinent information to be
entered and submitted to the developers.

Before submitting a report, please log into SourceForge if you are a
member; this will make it possible for the developers to contact you
for additional information if needed.  If you are not a SourceForge
member but would not mind the developers contacting you, you may
include your email address in your bug description.  In this case,
please realize that the information is publically available and cannot
be protected.

The first step in filing a report is to determine whether the problem
has already been reported.  The advantage in doing so, aside from
saving the developers time, is that you learn what has been done to
fix it; it may be that the problem has already been fixed for the next
release, or additional information is needed (in which case you are
welcome to provide it if you can!).  To do this, search the bug
database using the search box near the bottom of the page.

If the problem you're reporting is not already in the bug tracker, go
back to the Python Bug Tracker
(\url{http://sourceforge.net/bugs/?group_id=5470}).  Select the
``Submit a Bug'' link at the top of the page to open the bug reporting
form.

The submission form has a number of fields.  The only fields that are
required are the ``Summary'' and ``Details'' fields.  For the summary,
enter a \emph{very} short description of the problem; less than ten
words is good.  In the Details field, describe the problem in detail,
including what you expected to happen and what did happen.  Be sure to
include the version of Python you used, whether any extension modules
were involved, and what hardware and software platform you were using
(including version information as appropriate).

The only other field that you may want to set is the ``Category''
field, which allows you to place the bug report into a broad category
(such as ``Documentation'' or ``Library'').

Each bug report will be assigned to a developer who will determine
what needs to be done to correct the problem.  If you have a
SourceForge account and logged in to report the problem, you will
receive an update each time action is taken on the bug.


\begin{seealso}
  \seetitle[http://www-mice.cs.ucl.ac.uk/multimedia/software/documentation/ReportingBugs.html]{How
        to Report Bugs Effectively}{Article which goes into some
        detail about how to create a useful bug report.  This
        describes what kind of information is useful and why it is
        useful.}

  \seetitle[http://www.mozilla.org/quality/bug-writing-guidelines.html]{Bug
        Writing Guidelines}{Information about writing a good bug
        report.  Some of this is specific to the Mozilla project, but
        describes general good practices.}
\end{seealso}


\chapter{History and License}
\input{license}

\documentclass{manual}

\title{Python/C API Reference Manual}

\input{boilerplate}

\makeindex			% tell \index to actually write the .idx file


\begin{document}

\maketitle

\ifhtml
\chapter*{Front Matter\label{front}}
\fi

\input{copyright}

\begin{abstract}

\noindent
This manual documents the API used by C and \Cpp{} programmers who
want to write extension modules or embed Python.  It is a companion to
\citetitle[../ext/ext.html]{Extending and Embedding the Python
Interpreter}, which describes the general principles of extension
writing but does not document the API functions in detail.

\warning{The current version of this document is incomplete.  I hope
that it is nevertheless useful.  I will continue to work on it, and
release new versions from time to time, independent from Python source
code releases.}

\end{abstract}

\tableofcontents


\input{intro}
\input{veryhigh}
\input{refcounting}
\input{exceptions}
\input{utilities}
\input{abstract}
\input{concrete}
\input{init}
\input{memory}
\input{newtypes}


% \chapter{Debugging \label{debugging}}
%
% XXX Explain Py_DEBUG, Py_TRACE_REFS, Py_REF_DEBUG.


\appendix
\chapter{Reporting Bugs}
\input{reportingbugs}

\chapter{History and License}
\input{license}

\input{api.ind}			% Index -- must be last

\end{document}
			% Index -- must be last

\end{document}
			% Index -- must be last

\end{document}
			% Index -- must be last

\end{document}
