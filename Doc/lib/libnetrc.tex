\section{\module{netrc} ---
         netrc file processing}

\declaremodule{standard}{netrc}
% Note the \protect needed for \file... ;-(
\modulesynopsis{Loading of \protect\file{.netrc} files.}
\moduleauthor{Eric S. Raymond}{esr@snark.thyrsus.com}
\sectionauthor{Eric S. Raymond}{esr@snark.thyrsus.com}


\versionadded{1.5.2}

The \class{netrc} class parses and encapsulates the netrc file format
used by the \UNIX{} \program{ftp} program and other FTP clients.

\begin{classdesc}{netrc}{\optional{file}}
A \class{netrc} instance or subclass instance encapsulates data from 
a netrc file.  The initialization argument, if present, specifies the
file to parse.  If no argument is given, the file \file{.netrc} in the
user's home directory will be read.  Parse errors will raise
\exception{NetrcParseError} with diagnostic information including the
file name, line number, and terminating token.
\end{classdesc}

\begin{excdesc}{NetrcParseError}
Exception raised by the \class{netrc} class when syntactical errors
are encountered in source text.  Instances of this exception provide
three interesting attributes:  \member{msg} is a textual explanation
of the error, \member{filename} is the name of the source file, and
\member{lineno} gives the line number on which the error was found.
\end{excdesc}


\subsection{netrc Objects \label{netrc-objects}}

A \class{netrc} instance has the following methods:

\begin{methoddesc}{authenticators}{host}
Return a 3-tuple \code{(\var{login}, \var{account}, \var{password})}
of authenticators for \var{host}.  If the netrc file did not
contain an entry for the given host, return the tuple associated with
the `default' entry.  If neither matching host nor default entry is
available, return \code{None}.
\end{methoddesc}

\begin{methoddesc}{__repr__}{}
Dump the class data as a string in the format of a netrc file.
(This discards comments and may reorder the entries.)
\end{methoddesc}

Instances of \class{netrc} have public instance variables:

\begin{memberdesc}{hosts}
Dictionary mapping host names to \code{(\var{login}, \var{account},
\var{password})} tuples.  The `default' entry, if any, is represented
as a pseudo-host by that name.
\end{memberdesc}

\begin{memberdesc}{macros}
Dictionary mapping macro names to string lists.
\end{memberdesc}

\note{Passwords are limited to a subset of the ASCII character set.
Versions of this module prior to 2.3 were extremely limited.  Starting with
2.3, all ASCII punctuation is allowed in passwords.  However, note that
whitespace and non-printable characters are not allowed in passwords.  This
is a limitation of the way the .netrc file is parsed and may be removed in
the future.}
