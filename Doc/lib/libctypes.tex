\newlength{\locallinewidth}
\setlength{\locallinewidth}{\linewidth}
\section{\module{ctypes} --- A foreign function library for Python.}
\declaremodule{standard}{ctypes}
\moduleauthor{Thomas Heller}{theller@python.net}
\modulesynopsis{A foreign function library for Python.}
\versionadded{2.5}

\code{ctypes} is a foreign function library for Python.


\subsection{ctypes tutorial\label{ctypes-ctypes-tutorial}}

This tutorial describes version 0.9.9 of \code{ctypes}.

Note: The code samples in this tutorial uses \code{doctest} to make sure
that they actually work.  Since some code samples behave differently
under Linux, Windows, or Mac OS X, they contain doctest directives in
comments.

Note: Quite some code samples references the ctypes \class{c{\_}int} type.
This type is an alias to the \class{c{\_}long} type on 32-bit systems.  So,
you should not be confused if \class{c{\_}long} is printed if you would
expect \class{c{\_}int} - they are actually the same type.


\subsubsection{Loading dynamic link libraries\label{ctypes-loading-dynamic-link-libraries}}

\code{ctypes} exports the \var{cdll}, and on Windows also \var{windll} and
\var{oledll} objects to load dynamic link libraries.

You load libraries by accessing them as attributes of these objects.
\var{cdll} loads libraries which export functions using the standard
\code{cdecl} calling convention, while \var{windll} libraries call
functions using the \code{stdcall} calling convention. \var{oledll} also
uses the \code{stdcall} calling convention, and assumes the functions
return a Windows \class{HRESULT} error code. The error code is used to
automatically raise \class{WindowsError} Python exceptions when the
function call fails.

Here are some examples for Windows, note that \code{msvcrt} is the MS
standard C library containing most standard C functions, and uses the
cdecl calling convention:
\begin{verbatim}
>>> from ctypes import *
>>> print windll.kernel32 # doctest: +WINDOWS
<WinDLL 'kernel32', handle ... at ...>
>>> print cdll.msvcrt # doctest: +WINDOWS
<CDLL 'msvcrt', handle ... at ...>
>>> libc = cdll.msvcrt # doctest: +WINDOWS
>>>
\end{verbatim}

Windows appends the usual '.dll' file suffix automatically.

On Linux, it is required to specify the filename \emph{including} the
extension to load a library, so attribute access does not work.
Either the \method{LoadLibrary} method of the dll loaders should be used,
or you should load the library by creating an instance of CDLL by
calling the constructor:
\begin{verbatim}
>>> cdll.LoadLibrary("libc.so.6") # doctest: +LINUX
<CDLL 'libc.so.6', handle ... at ...>
>>> libc = CDLL("libc.so.6")     # doctest: +LINUX
>>> libc                         # doctest: +LINUX
<CDLL 'libc.so.6', handle ... at ...>
>>>
\end{verbatim}

XXX Add section for Mac OS X.


\subsubsection{Accessing functions from loaded dlls\label{ctypes-accessing-functions-from-loaded-dlls}}

Functions are accessed as attributes of dll objects:
\begin{verbatim}
>>> from ctypes import *
>>> libc.printf
<_FuncPtr object at 0x...>
>>> print windll.kernel32.GetModuleHandleA # doctest: +WINDOWS
<_FuncPtr object at 0x...>
>>> print windll.kernel32.MyOwnFunction # doctest: +WINDOWS
Traceback (most recent call last):
  File "<stdin>", line 1, in ?
  File "ctypes.py", line 239, in __getattr__
    func = _StdcallFuncPtr(name, self)
AttributeError: function 'MyOwnFunction' not found
>>>
\end{verbatim}

Note that win32 system dlls like \code{kernel32} and \code{user32} often
export ANSI as well as UNICODE versions of a function. The UNICODE
version is exported with an \code{W} appended to the name, while the ANSI
version is exported with an \code{A} appended to the name. The win32
\code{GetModuleHandle} function, which returns a \emph{module handle} for a
given module name, has the following C prototype, and a macro is used
to expose one of them as \code{GetModuleHandle} depending on whether
UNICODE is defined or not:
\begin{verbatim}
/* ANSI version */
HMODULE GetModuleHandleA(LPCSTR lpModuleName);
/* UNICODE version */
HMODULE GetModuleHandleW(LPCWSTR lpModuleName);
\end{verbatim}

\var{windll} does not try to select one of them by magic, you must
access the version you need by specifying \code{GetModuleHandleA} or
\code{GetModuleHandleW} explicitely, and then call it with normal strings
or unicode strings respectively.

Sometimes, dlls export functions with names which aren't valid Python
identifiers, like \code{"??2@YAPAXI@Z"}. In this case you have to use
\code{getattr} to retrieve the function:
\begin{verbatim}
>>> getattr(cdll.msvcrt, "??2@YAPAXI@Z") # doctest: +WINDOWS
<_FuncPtr object at 0x...>
>>>
\end{verbatim}

On Windows, some dlls export functions not by name but by ordinal.
These functions can be accessed by indexing the dll object with the
odinal number:
\begin{verbatim}
>>> cdll.kernel32[1] # doctest: +WINDOWS
<_FuncPtr object at 0x...>
>>> cdll.kernel32[0] # doctest: +WINDOWS
Traceback (most recent call last):
  File "<stdin>", line 1, in ?
  File "ctypes.py", line 310, in __getitem__
    func = _StdcallFuncPtr(name, self)
AttributeError: function ordinal 0 not found
>>>
\end{verbatim}


\subsubsection{Calling functions\label{ctypes-calling-functions}}

You can call these functions like any other Python callable. This
example uses the \code{time()} function, which returns system time in
seconds since the \UNIX{} epoch, and the \code{GetModuleHandleA()} function,
which returns a win32 module handle.

This example calls both functions with a NULL pointer (\code{None} should
be used as the NULL pointer):
\begin{verbatim}
>>> print libc.time(None)
114...
>>> print hex(windll.kernel32.GetModuleHandleA(None)) # doctest: +WINDOWS
0x1d000000
>>>
\end{verbatim}

\code{ctypes} tries to protect you from calling functions with the wrong
number of arguments.  Unfortunately this only works on Windows.  It
does this by examining the stack after the function returns:
\begin{verbatim}
>>> windll.kernel32.GetModuleHandleA() # doctest: +WINDOWS
Traceback (most recent call last):
  File "<stdin>", line 1, in ?
ValueError: Procedure probably called with not enough arguments (4 bytes missing)
>>> windll.kernel32.GetModuleHandleA(0, 0) # doctest: +WINDOWS
Traceback (most recent call last):
  File "<stdin>", line 1, in ?
ValueError: Procedure probably called with too many arguments (4 bytes in excess)
>>>
\end{verbatim}

On Windows, \code{ctypes} uses win32 structured exception handling to
prevent crashes from general protection faults when functions are
called with invalid argument values:
\begin{verbatim}
>>> windll.kernel32.GetModuleHandleA(32) # doctest: +WINDOWS
Traceback (most recent call last):
  File "<stdin>", line 1, in ?
WindowsError: exception: access violation reading 0x00000020
>>>
\end{verbatim}

There are, however, enough ways to crash Python with \code{ctypes}, so
you should be careful anyway.

Python integers, strings and unicode strings are the only objects that
can directly be used as parameters in these function calls.

Before we move on calling functions with other parameter types, we
have to learn more about \code{ctypes} data types.


\subsubsection{Simple data types\label{ctypes-simple-data-types}}

\code{ctypes} defines a number of primitive C compatible data types :
\begin{quote}

\begin{longtable}[c]{|p{0.19\locallinewidth}|p{0.28\locallinewidth}|p{0.14\locallinewidth}|}
\hline
\textbf{
ctypes type
} & \textbf{
C type
} & \textbf{
Python type
} \\
\hline
\endhead

\class{c{\_}char}
 & 
\code{char}
 & 
character
 \\
\hline

\class{c{\_}byte}
 & 
\code{char}
 & 
integer
 \\
\hline

\class{c{\_}ubyte}
 & 
\code{unsigned char}
 & 
integer
 \\
\hline

\class{c{\_}short}
 & 
\code{short}
 & 
integer
 \\
\hline

\class{c{\_}ushort}
 & 
\code{unsigned short}
 & 
integer
 \\
\hline

\class{c{\_}int}
 & 
\code{int}
 & 
integer
 \\
\hline

\class{c{\_}uint}
 & 
\code{unsigned int}
 & 
integer
 \\
\hline

\class{c{\_}long}
 & 
\code{long}
 & 
integer
 \\
\hline

\class{c{\_}ulong}
 & 
\code{unsigned long}
 & 
long
 \\
\hline

\class{c{\_}longlong}
 & 
\code{{\_}{\_}int64} or
\code{long long}
 & 
long
 \\
\hline

\class{c{\_}ulonglong}
 & 
\code{unsigned {\_}{\_}int64} or
\code{unsigned long long}
 & 
long
 \\
\hline

\class{c{\_}float}
 & 
\code{float}
 & 
float
 \\
\hline

\class{c{\_}double}
 & 
\code{double}
 & 
float
 \\
\hline

\class{c{\_}char{\_}p}
 & 
\code{char *}
(NUL terminated)
 & 
string or
\code{None}
 \\
\hline

\class{c{\_}wchar{\_}p}
 & 
\code{wchar{\_}t *}
(NUL terminated)
 & 
unicode or
\code{None}
 \\
\hline

\class{c{\_}void{\_}p}
 & 
\code{void *}
 & 
integer or
\code{None}
 \\
\hline
\end{longtable}
\end{quote}

All these types can be created by calling them with an optional
initializer of the correct type and value:
\begin{verbatim}
>>> c_int()
c_long(0)
>>> c_char_p("Hello, World")
c_char_p('Hello, World')
>>> c_ushort(-3)
c_ushort(65533)
>>>
\end{verbatim}

Since these types are mutable, their value can also be changed
afterwards:
\begin{verbatim}
>>> i = c_int(42)
>>> print i
c_long(42)
>>> print i.value
42
>>> i.value = -99
>>> print i.value
-99
>>>
\end{verbatim}

Assigning a new value to instances of the pointer types \class{c{\_}char{\_}p},
\class{c{\_}wchar{\_}p}, and \class{c{\_}void{\_}p} changes the \emph{memory location} they
point to, \emph{not the contents} of the memory block (of course not,
because Python strings are immutable):
\begin{verbatim}
>>> s = "Hello, World"
>>> c_s = c_char_p(s)
>>> print c_s
c_char_p('Hello, World')
>>> c_s.value = "Hi, there"
>>> print c_s
c_char_p('Hi, there')
>>> print s                 # first string is unchanged
Hello, World
\end{verbatim}

You should be careful, however, not to pass them to functions
expecting pointers to mutable memory. If you need mutable memory
blocks, ctypes has a \code{create{\_}string{\_}buffer} function which creates
these in various ways.  The current memory block contents can be
accessed (or changed) with the \code{raw} property, if you want to access
it as NUL terminated string, use the \code{string} property:
\begin{verbatim}
>>> from ctypes import *
>>> p = create_string_buffer(3)      # create a 3 byte buffer, initialized to NUL bytes
>>> print sizeof(p), repr(p.raw)
3 '\x00\x00\x00'
>>> p = create_string_buffer("Hello")      # create a buffer containing a NUL terminated string
>>> print sizeof(p), repr(p.raw)
6 'Hello\x00'
>>> print repr(p.value)
'Hello'
>>> p = create_string_buffer("Hello", 10)  # create a 10 byte buffer
>>> print sizeof(p), repr(p.raw)
10 'Hello\x00\x00\x00\x00\x00'
>>> p.value = "Hi"      
>>> print sizeof(p), repr(p.raw)
10 'Hi\x00lo\x00\x00\x00\x00\x00'
>>>
\end{verbatim}

The \code{create{\_}string{\_}buffer} function replaces the \code{c{\_}buffer}
function (which is still available as an alias), as well as the
\code{c{\_}string} function from earlier ctypes releases.  To create a
mutable memory block containing unicode characters of the C type
\code{wchar{\_}t} use the \code{create{\_}unicode{\_}buffer} function.


\subsubsection{Calling functions, continued\label{ctypes-calling-functions-continued}}

Note that printf prints to the real standard output channel, \emph{not} to
\code{sys.stdout}, so these examples will only work at the console
prompt, not from within \emph{IDLE} or \emph{PythonWin}:
\begin{verbatim}
>>> printf = libc.printf
>>> printf("Hello, %s\n", "World!")
Hello, World!
14
>>> printf("Hello, %S", u"World!")
Hello, World!
13
>>> printf("%d bottles of beer\n", 42)
42 bottles of beer
19
>>> printf("%f bottles of beer\n", 42.5)
Traceback (most recent call last):
  File "<stdin>", line 1, in ?
ArgumentError: argument 2: exceptions.TypeError: Don't know how to convert parameter 2
>>>
\end{verbatim}

As has been mentioned before, all Python types except integers,
strings, and unicode strings have to be wrapped in their corresponding
\code{ctypes} type, so that they can be converted to the required C data
type:
\begin{verbatim}
>>> printf("An int %d, a double %f\n", 1234, c_double(3.14))
Integer 1234, double 3.1400001049
31
>>>
\end{verbatim}


\subsubsection{Calling functions with your own custom data types\label{ctypes-calling-functions-with-own-custom-data-types}}

You can also customize \code{ctypes} argument conversion to allow
instances of your own classes be used as function arguments.
\code{ctypes} looks for an \member{{\_}as{\_}parameter{\_}} attribute and uses this as
the function argument. Of course, it must be one of integer, string,
or unicode:
\begin{verbatim}
>>> class Bottles(object):
...     def __init__(self, number):
...         self._as_parameter_ = number
...
>>> bottles = Bottles(42)
>>> printf("%d bottles of beer\n", bottles)
42 bottles of beer
19
>>>
\end{verbatim}

If you don't want to store the instance's data in the
\member{{\_}as{\_}parameter{\_}} instance variable, you could define a \code{property}
which makes the data avaiblable.


\subsubsection{Specifying the required argument types (function prototypes)\label{ctypes-specifying-required-argument-types}}

It is possible to specify the required argument types of functions
exported from DLLs by setting the \member{argtypes} attribute.

\member{argtypes} must be a sequence of C data types (the \code{printf}
function is probably not a good example here, because it takes a
variable number and different types of parameters depending on the
format string, on the other hand this is quite handy to experiment
with this feature):
\begin{verbatim}
>>> printf.argtypes = [c_char_p, c_char_p, c_int, c_double]
>>> printf("String '%s', Int %d, Double %f\n", "Hi", 10, 2.2)
String 'Hi', Int 10, Double 2.200000
37
>>>
\end{verbatim}

Specifying a format protects against incompatible argument types (just
as a prototype for a C function), and tries to convert the arguments
to valid types:
\begin{verbatim}
>>> printf("%d %d %d", 1, 2, 3)
Traceback (most recent call last):
  File "<stdin>", line 1, in ?
ArgumentError: argument 2: exceptions.TypeError: wrong type
>>> printf("%s %d %f", "X", 2, 3)
X 2 3.00000012
12
>>>
\end{verbatim}

If you have defined your own classes which you pass to function calls,
you have to implement a \method{from{\_}param} class method for them to be
able to use them in the \member{argtypes} sequence. The \method{from{\_}param}
class method receives the Python object passed to the function call,
it should do a typecheck or whatever is needed to make sure this
object is acceptable, and then return the object itself, it's
\member{{\_}as{\_}parameter{\_}} attribute, or whatever you want to pass as the C
function argument in this case. Again, the result should be an
integer, string, unicode, a \code{ctypes} instance, or something having
the \member{{\_}as{\_}parameter{\_}} attribute.


\subsubsection{Return types\label{ctypes-return-types}}

By default functions are assumed to return integers.  Other return
types can be specified by setting the \member{restype} attribute of the
function object.

Here is a more advanced example, it uses the strchr function, which
expects a string pointer and a char, and returns a pointer to a
string:
\begin{verbatim}
>>> strchr = libc.strchr
>>> strchr("abcdef", ord("d")) # doctest: +SKIP
8059983
>>> strchr.restype = c_char_p # c_char_p is a pointer to a string
>>> strchr("abcdef", ord("d"))
'def'
>>> print strchr("abcdef", ord("x"))
None
>>>
\end{verbatim}

If you want to avoid the \code{ord("x")} calls above, you can set the
\member{argtypes} attribute, and the second argument will be converted from
a single character Python string into a C char:
\begin{verbatim}
>>> strchr.restype = c_char_p
>>> strchr.argtypes = [c_char_p, c_char]
>>> strchr("abcdef", "d")
'def'
>>> strchr("abcdef", "def")
Traceback (most recent call last):
  File "<stdin>", line 1, in ?
ArgumentError: argument 2: exceptions.TypeError: one character string expected
>>> print strchr("abcdef", "x")
None
>>> strchr("abcdef", "d")
'def'
>>>
\end{verbatim}

XXX Mention the \member{errcheck} protocol...

You can also use a callable Python object (a function or a class for
example) as the \member{restype} attribute.  It will be called with the
\code{integer} the C function returns, and the result of this call will
be used as the result of your function call. This is useful to check
for error return values and automatically raise an exception:
\begin{verbatim}
>>> GetModuleHandle = windll.kernel32.GetModuleHandleA # doctest: +WINDOWS
>>> def ValidHandle(value):
...     if value == 0:
...         raise WinError()
...     return value
...
>>>
>>> GetModuleHandle.restype = ValidHandle # doctest: +WINDOWS
>>> GetModuleHandle(None) # doctest: +WINDOWS
486539264
>>> GetModuleHandle("something silly") # doctest: +WINDOWS +IGNORE_EXCEPTION_DETAIL
Traceback (most recent call last):
  File "<stdin>", line 1, in ?
  File "<stdin>", line 3, in ValidHandle
WindowsError: [Errno 126] The specified module could not be found.
>>>
\end{verbatim}

\code{WinError} is a function which will call Windows \code{FormatMessage()}
api to get the string representation of an error code, and \emph{returns}
an exception.  \code{WinError} takes an optional error code parameter, if
no one is used, it calls \function{GetLastError()} to retrieve it.


\subsubsection{Passing pointers (or: passing parameters by reference)\label{ctypes-passing-pointers}}

Sometimes a C api function expects a \emph{pointer} to a data type as
parameter, probably to write into the corresponding location, or if
the data is too large to be passed by value. This is also known as
\emph{passing parameters by reference}.

\code{ctypes} exports the \function{byref} function which is used to pass
parameters by reference.  The same effect can be achieved with the
\code{pointer} function, although \code{pointer} does a lot more work since
it constructs a real pointer object, so it is faster to use \function{byref}
if you don't need the pointer object in Python itself:
\begin{verbatim}
>>> i = c_int()
>>> f = c_float()
>>> s = create_string_buffer('\000' * 32)
>>> print i.value, f.value, repr(s.value)
0 0.0 ''
>>> libc.sscanf("1 3.14 Hello", "%d %f %s",
...             byref(i), byref(f), s)
3
>>> print i.value, f.value, repr(s.value)
1 3.1400001049 'Hello'
>>>
\end{verbatim}


\subsubsection{Structures and unions\label{ctypes-structures-unions}}

Structures and unions must derive from the \class{Structure} and \class{Union}
base classes which are defined in the \code{ctypes} module. Each subclass
must define a \member{{\_}fields{\_}} attribute.  \member{{\_}fields{\_}} must be a list of
\emph{2-tuples}, containing a \emph{field name} and a \emph{field type}.

The field type must be a \code{ctypes} type like \class{c{\_}int}, or any other
derived \code{ctypes} type: structure, union, array, pointer.

Here is a simple example of a POINT structure, which contains two
integers named \code{x} and \code{y}, and also shows how to initialize a
structure in the constructor:
\begin{verbatim}
>>> from ctypes import *
>>> class POINT(Structure):
...     _fields_ = [("x", c_int),
...                 ("y", c_int)]
...
>>> point = POINT(10, 20)
>>> print point.x, point.y
10 20
>>> point = POINT(y=5)
>>> print point.x, point.y
0 5
>>> POINT(1, 2, 3)
Traceback (most recent call last):
  File "<stdin>", line 1, in ?
ValueError: too many initializers
>>>
\end{verbatim}

You can, however, build much more complicated structures. Structures
can itself contain other structures by using a structure as a field
type.

Here is a RECT structure which contains two POINTs named \code{upperleft}
and \code{lowerright}
\begin{verbatim}
>>> class RECT(Structure):
...     _fields_ = [("upperleft", POINT),
...                 ("lowerright", POINT)]
...
>>> rc = RECT(point)
>>> print rc.upperleft.x, rc.upperleft.y
0 5
>>> print rc.lowerright.x, rc.lowerright.y
0 0
>>>
\end{verbatim}

Nested structures can also be initialized in the constructor in
several ways:
\begin{verbatim}
>>> r = RECT(POINT(1, 2), POINT(3, 4))
>>> r = RECT((1, 2), (3, 4))
\end{verbatim}

Fields descriptors can be retrieved from the \emph{class}, they are useful
for debugging because they can provide useful information:
\begin{verbatim}
>>> print POINT.x
<Field type=c_long, ofs=0, size=4>
>>> print POINT.y
<Field type=c_long, ofs=4, size=4>
>>>
\end{verbatim}


\subsubsection{Structure/union alignment and byte order\label{ctypes-structureunion-alignment-byte-order}}

By default, Structure and Union fields are aligned in the same way the
C compiler does it. It is possible to override this behaviour be
specifying a \member{{\_}pack{\_}} class attribute in the subclass
definition. This must be set to a positive integer and specifies the
maximum alignment for the fields. This is what \code{{\#}pragma pack(n)}
also does in MSVC.

\code{ctypes} uses the native byte order for Structures and Unions.  To
build structures with non-native byte order, you can use one of the
BigEndianStructure, LittleEndianStructure, BigEndianUnion, and
LittleEndianUnion base classes.  These classes cannot contain pointer
fields.


\subsubsection{Bit fields in structures and unions\label{ctypes-bit-fields-in-structures-unions}}

It is possible to create structures and unions containing bit fields.
Bit fields are only possible for integer fields, the bit width is
specified as the third item in the \member{{\_}fields{\_}} tuples:
\begin{verbatim}
>>> class Int(Structure):
...     _fields_ = [("first_16", c_int, 16),
...                 ("second_16", c_int, 16)]
...
>>> print Int.first_16
<Field type=c_long, ofs=0:0, bits=16>
>>> print Int.second_16
<Field type=c_long, ofs=0:16, bits=16>
>>>
\end{verbatim}


\subsubsection{Arrays\label{ctypes-arrays}}

Arrays are sequences, containing a fixed number of instances of the
same type.

The recommended way to create array types is by multiplying a data
type with a positive integer:
\begin{verbatim}
TenPointsArrayType = POINT * 10
\end{verbatim}

Here is an example of an somewhat artifical data type, a structure
containing 4 POINTs among other stuff:
\begin{verbatim}
>>> from ctypes import *
>>> class POINT(Structure):
...    _fields_ = ("x", c_int), ("y", c_int)
...
>>> class MyStruct(Structure):
...    _fields_ = [("a", c_int),
...                ("b", c_float),
...                ("point_array", POINT * 4)]
>>>
>>> print len(MyStruct().point_array)
4
\end{verbatim}

Instances are created in the usual way, by calling the class:
\begin{verbatim}
arr = TenPointsArrayType()
for pt in arr:
    print pt.x, pt.y
\end{verbatim}

The above code print a series of \code{0 0} lines, because the array
contents is initialized to zeros.

Initializers of the correct type can also be specified:
\begin{verbatim}
>>> from ctypes import *
>>> TenIntegers = c_int * 10
>>> ii = TenIntegers(1, 2, 3, 4, 5, 6, 7, 8, 9, 10)
>>> print ii
<c_long_Array_10 object at 0x...>
>>> for i in ii: print i,
...
1 2 3 4 5 6 7 8 9 10
>>>
\end{verbatim}


\subsubsection{Pointers\label{ctypes-pointers}}

Pointer instances are created by calling the \code{pointer} function on a
\code{ctypes} type:
\begin{verbatim}
>>> from ctypes import *
>>> i = c_int(42)
>>> pi = pointer(i)
>>>
\end{verbatim}

XXX XXX Not correct: use indexing, not the contents atribute

Pointer instances have a \code{contents} attribute which returns the
ctypes' type pointed to, the \code{c{\_}int(42)} in the above case:
\begin{verbatim}
>>> pi.contents
c_long(42)
>>>
\end{verbatim}

Assigning another \class{c{\_}int} instance to the pointer's contents
attribute would cause the pointer to point to the memory location
where this is stored:
\begin{verbatim}
>>> pi.contents = c_int(99)
>>> pi.contents
c_long(99)
>>>
\end{verbatim}

Pointer instances can also be indexed with integers:
\begin{verbatim}
>>> pi[0]
99
>>>
\end{verbatim}

XXX What is this???
Assigning to an integer index changes the pointed to value:
\begin{verbatim}
>>> i2 = pi[0]
>>> i2
99
>>> pi[0] = 22
>>> i2
99
>>>
\end{verbatim}

It is also possible to use indexes different from 0, but you must know
what you're doing when you use this: You access or change arbitrary
memory locations when you do this. Generally you only use this feature
if you receive a pointer from a C function, and you \emph{know} that the
pointer actually points to an array instead of a single item.


\subsubsection{Pointer classes/types\label{ctypes-pointer-classestypes}}

Behind the scenes, the \code{pointer} function does more than simply
create pointer instances, it has to create pointer \emph{types} first.
This is done with the \code{POINTER} function, which accepts any
\code{ctypes} type, and returns a new type:
\begin{verbatim}
>>> PI = POINTER(c_int)
>>> PI
<class 'ctypes.LP_c_long'>
>>> PI(42) # doctest: +IGNORE_EXCEPTION_DETAIL
Traceback (most recent call last):
  File "<stdin>", line 1, in ?
TypeError: expected c_long instead of int
>>> PI(c_int(42))
<ctypes.LP_c_long object at 0x...>
>>>
\end{verbatim}


\subsubsection{Incomplete Types\label{ctypes-incomplete-types}}

\emph{Incomplete Types} are structures, unions or arrays whose members are
not yet specified. In C, they are specified by forward declarations, which
are defined later:
\begin{verbatim}
struct cell; /* forward declaration */

struct {
    char *name;
    struct cell *next;
} cell;
\end{verbatim}

The straightforward translation into ctypes code would be this, but it
does not work:
\begin{verbatim}
>>> class cell(Structure):
...     _fields_ = [("name", c_char_p),
...                 ("next", POINTER(cell))]
...
Traceback (most recent call last):
  File "<stdin>", line 1, in ?
  File "<stdin>", line 2, in cell
NameError: name 'cell' is not defined
>>>
\end{verbatim}

because the new \code{class cell} is not available in the class statement
itself.  In \code{ctypes}, we can define the \code{cell} class and set the
\member{{\_}fields{\_}} attribute later, after the class statement:
\begin{verbatim}
>>> from ctypes import *
>>> class cell(Structure):
...     pass
...
>>> cell._fields_ = [("name", c_char_p),
...                  ("next", POINTER(cell))]
>>>
\end{verbatim}

Lets try it. We create two instances of \code{cell}, and let them point
to each other, and finally follow the pointer chain a few times:
\begin{verbatim}
>>> c1 = cell()
>>> c1.name = "foo"
>>> c2 = cell()
>>> c2.name = "bar"
>>> c1.next = pointer(c2)
>>> c2.next = pointer(c1)
>>> p = c1
>>> for i in range(8):
...     print p.name,
...     p = p.next[0]
...
foo bar foo bar foo bar foo bar
>>>    
\end{verbatim}


\subsubsection{Callback functions\label{ctypes-callback-functions}}

\code{ctypes} allows to create C callable function pointers from Python
callables. These are sometimes called \emph{callback functions}.

First, you must create a class for the callback function, the class
knows the calling convention, the return type, and the number and
types of arguments this function will receive.

The CFUNCTYPE factory function creates types for callback functions
using the normal cdecl calling convention, and, on Windows, the
WINFUNCTYPE factory function creates types for callback functions
using the stdcall calling convention.

Both of these factory functions are called with the result type as
first argument, and the callback functions expected argument types as
the remaining arguments.

I will present an example here which uses the standard C library's
\function{qsort} function, this is used to sort items with the help of a
callback function. \function{qsort} will be used to sort an array of
integers:
\begin{verbatim}
>>> IntArray5 = c_int * 5
>>> ia = IntArray5(5, 1, 7, 33, 99)
>>> qsort = libc.qsort
>>> qsort.restype = None
>>>
\end{verbatim}

\function{qsort} must be called with a pointer to the data to sort, the
number of items in the data array, the size of one item, and a pointer
to the comparison function, the callback. The callback will then be
called with two pointers to items, and it must return a negative
integer if the first item is smaller than the second, a zero if they
are equal, and a positive integer else.

So our callback function receives pointers to integers, and must
return an integer. First we create the \code{type} for the callback
function:
\begin{verbatim}
>>> CMPFUNC = CFUNCTYPE(c_int, POINTER(c_int), POINTER(c_int))
>>>
\end{verbatim}

For the first implementation of the callback function, we simply print
the arguments we get, and return 0 (incremental development ;-):
\begin{verbatim}
>>> def py_cmp_func(a, b):
...     print "py_cmp_func", a, b
...     return 0
...
>>>
\end{verbatim}

Create the C callable callback:
\begin{verbatim}
>>> cmp_func = CMPFUNC(py_cmp_func)
>>>
\end{verbatim}

And we're ready to go:
\begin{verbatim}
>>> qsort(ia, len(ia), sizeof(c_int), cmp_func) # doctest: +WINDOWS
py_cmp_func <ctypes.LP_c_long object at 0x00...> <ctypes.LP_c_long object at 0x00...>
py_cmp_func <ctypes.LP_c_long object at 0x00...> <ctypes.LP_c_long object at 0x00...>
py_cmp_func <ctypes.LP_c_long object at 0x00...> <ctypes.LP_c_long object at 0x00...>
py_cmp_func <ctypes.LP_c_long object at 0x00...> <ctypes.LP_c_long object at 0x00...>
py_cmp_func <ctypes.LP_c_long object at 0x00...> <ctypes.LP_c_long object at 0x00...>
py_cmp_func <ctypes.LP_c_long object at 0x00...> <ctypes.LP_c_long object at 0x00...>
py_cmp_func <ctypes.LP_c_long object at 0x00...> <ctypes.LP_c_long object at 0x00...>
py_cmp_func <ctypes.LP_c_long object at 0x00...> <ctypes.LP_c_long object at 0x00...>
py_cmp_func <ctypes.LP_c_long object at 0x00...> <ctypes.LP_c_long object at 0x00...>
py_cmp_func <ctypes.LP_c_long object at 0x00...> <ctypes.LP_c_long object at 0x00...>
>>>
\end{verbatim}

We know how to access the contents of a pointer, so lets redefine our callback:
\begin{verbatim}
>>> def py_cmp_func(a, b):
...     print "py_cmp_func", a[0], b[0]
...     return 0
...
>>> cmp_func = CMPFUNC(py_cmp_func)
>>>
\end{verbatim}

Here is what we get on Windows:
\begin{verbatim}
>>> qsort(ia, len(ia), sizeof(c_int), cmp_func) # doctest: +WINDOWS
py_cmp_func 7 1
py_cmp_func 33 1
py_cmp_func 99 1
py_cmp_func 5 1
py_cmp_func 7 5
py_cmp_func 33 5
py_cmp_func 99 5
py_cmp_func 7 99
py_cmp_func 33 99
py_cmp_func 7 33
>>>
\end{verbatim}

It is funny to see that on linux the sort function seems to work much
more efficient, it is doing less comparisons:
\begin{verbatim}
>>> qsort(ia, len(ia), sizeof(c_int), cmp_func) # doctest: +LINUX
py_cmp_func 5 1
py_cmp_func 33 99
py_cmp_func 7 33
py_cmp_func 5 7
py_cmp_func 1 7
>>>
\end{verbatim}

Ah, we're nearly done! The last step is to actually compare the two
items and return a useful result:
\begin{verbatim}
>>> def py_cmp_func(a, b):
...     print "py_cmp_func", a[0], b[0]
...     return a[0] - b[0]
...
>>>
\end{verbatim}

Final run on Windows:
\begin{verbatim}
>>> qsort(ia, len(ia), sizeof(c_int), CMPFUNC(py_cmp_func)) # doctest: +WINDOWS
py_cmp_func 33 7
py_cmp_func 99 33
py_cmp_func 5 99
py_cmp_func 1 99
py_cmp_func 33 7
py_cmp_func 1 33
py_cmp_func 5 33
py_cmp_func 5 7
py_cmp_func 1 7
py_cmp_func 5 1
>>>
\end{verbatim}

and on Linux:
\begin{verbatim}
>>> qsort(ia, len(ia), sizeof(c_int), CMPFUNC(py_cmp_func)) # doctest: +LINUX
py_cmp_func 5 1
py_cmp_func 33 99
py_cmp_func 7 33
py_cmp_func 1 7
py_cmp_func 5 7
>>>
\end{verbatim}

So, our array sorted now:
\begin{verbatim}
>>> for i in ia: print i,
...
1 5 7 33 99
>>>
\end{verbatim}

\textbf{Important note for callback functions:}

Make sure you keep references to CFUNCTYPE objects as long as they are
used from C code. ctypes doesn't, and if you don't, they may be
garbage collected, crashing your program when a callback is made.


\subsubsection{Accessing values exported from dlls\label{ctypes-accessing-values-exported-from-dlls}}

Sometimes, a dll not only exports functions, it also exports
values. An example in the Python library itself is the
\code{Py{\_}OptimizeFlag}, an integer set to 0, 1, or 2, depending on the
\programopt{-O} or \programopt{-OO} flag given on startup.

\code{ctypes} can access values like this with the \method{in{\_}dll} class
methods of the type.  \var{pythonapi} �s a predefined symbol giving
access to the Python C api:
\begin{verbatim}
>>> opt_flag = c_int.in_dll(pythonapi, "Py_OptimizeFlag")
>>> print opt_flag
c_long(0)
>>>
\end{verbatim}

If the interpreter would have been started with \programopt{-O}, the sample
would have printed \code{c{\_}long(1)}, or \code{c{\_}long(2)} if \programopt{-OO} would have
been specified.

An extended example which also demonstrates the use of pointers
accesses the \code{PyImport{\_}FrozenModules} pointer exported by Python.

Quoting the Python docs: \emph{This pointer is initialized to point to an
array of ``struct {\_}frozen`` records, terminated by one whose members
are all NULL or zero. When a frozen module is imported, it is searched
in this table. Third-party code could play tricks with this to provide
a dynamically created collection of frozen modules.}

So manipulating this pointer could even prove useful. To restrict the
example size, we show only how this table can be read with
\code{ctypes}:
\begin{verbatim}
>>> from ctypes import *
>>>
>>> class struct_frozen(Structure):
...     _fields_ = [("name", c_char_p),
...                 ("code", POINTER(c_ubyte)),
...                 ("size", c_int)]
...
>>>
\end{verbatim}

We have defined the \code{struct {\_}frozen} data type, so we can get the
pointer to the table:
\begin{verbatim}
>>> FrozenTable = POINTER(struct_frozen)
>>> table = FrozenTable.in_dll(pythonapi, "PyImport_FrozenModules")
>>>
\end{verbatim}

Since \code{table} is a \code{pointer} to the array of \code{struct{\_}frozen}
records, we can iterate over it, but we just have to make sure that
our loop terminates, because pointers have no size. Sooner or later it
would probably crash with an access violation or whatever, so it's
better to break out of the loop when we hit the NULL entry:
\begin{verbatim}
>>> for item in table:
...    print item.name, item.size
...    if item.name is None:
...        break
...
__hello__ 104
__phello__ -104
__phello__.spam 104
None 0
>>>
\end{verbatim}

The fact that standard Python has a frozen module and a frozen package
(indicated by the negative size member) is not wellknown, it is only
used for testing. Try it out with \code{import {\_}{\_}hello{\_}{\_}} for example.

XXX Describe how to access the \var{code} member fields, which contain
the byte code for the modules.


\subsubsection{Surprises\label{ctypes-surprises}}

There are some edges in \code{ctypes} where you may be expect something
else than what actually happens.

Consider the following example:
\begin{verbatim}
>>> from ctypes import *
>>> class POINT(Structure):
...     _fields_ = ("x", c_int), ("y", c_int)
...
>>> class RECT(Structure):
...     _fields_ = ("a", POINT), ("b", POINT)
...
>>> p1 = POINT(1, 2)
>>> p2 = POINT(3, 4)
>>> rc = RECT(p1, p2)
>>> print rc.a.x, rc.a.y, rc.b.x, rc.b.y
1 2 3 4
>>> # now swap the two points
>>> rc.a, rc.b = rc.b, rc.a
>>> print rc.a.x, rc.a.y, rc.b.x, rc.b.y
3 4 3 4
\end{verbatim}

Hm. We certainly expected the last statement to print \code{3 4 1 2}.
What happended? Here are the steps of the \code{rc.a, rc.b = rc.b, rc.a}
line above:
\begin{verbatim}
>>> temp0, temp1 = rc.b, rc.a
>>> rc.a = temp0
>>> rc.b = temp1
\end{verbatim}

Note that \code{temp0} and \code{temp1} are objects still using the internal
buffer of the \code{rc} object above. So executing \code{rc.a = temp0}
copies the buffer contents of \code{temp0} into \code{rc} 's buffer.  This,
in turn, changes the contents of \code{temp1}. So, the last assignment
\code{rc.b = temp1}, doesn't have the expected effect.

Keep in mind that retrieving subobjects from Structure, Unions, and
Arrays doesn't \emph{copy} the subobject, instead it retrieves a wrapper
object accessing the root-object's underlying buffer.

Another example that may behave different from what one would expect is this:
\begin{verbatim}
>>> s = c_char_p()
>>> s.value = "abc def ghi"
>>> s.value
'abc def ghi'
>>> s.value is s.value
False
>>>
\end{verbatim}

Why is it printing \code{False}?  ctypes instances are objects containing
a memory block plus some descriptors accessing the contents of the
memory.  Storing a Python object in the memory block does not store
the object itself, instead the \code{contents} of the object is stored.
Accessing the contents again constructs a new Python each time!


\subsubsection{Bugs, ToDo and non-implemented things\label{ctypes-bugs-todo-non-implemented-things}}

Enumeration types are not implemented. You can do it easily yourself,
using \class{c{\_}int} as the base class.

\code{long double} is not implemented.
% Local Variables:
% compile-command: "make.bat"
% End: 

