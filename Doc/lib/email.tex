% Copyright (C) 2001,2002 Python Software Foundation
% Author: barry@zope.com (Barry Warsaw)

\section{\module{email} ---
	 An email and MIME handling package}

\declaremodule{standard}{email}
\modulesynopsis{Package supporting the parsing, manipulating, and
    generating email messages, including MIME documents.}
\moduleauthor{Barry A. Warsaw}{barry@zope.com}
\sectionauthor{Barry A. Warsaw}{barry@zope.com}

\versionadded{2.2}

The \module{email} package is a library for managing email messages,
including MIME and other \rfc{2822}-based message documents.  It
subsumes most of the functionality in several older standard modules
such as \refmodule{rfc822}, \refmodule{mimetools},
\refmodule{multifile}, and other non-standard packages such as
\module{mimecntl}.  It is specifically \emph{not} designed to do any
sending of email messages to SMTP (\rfc{2821}) servers; that is the
function of the \refmodule{smtplib} module.  The \module{email}
package attempts to be as RFC-compliant as possible, supporting in
addition to \rfc{2822}, such MIME-related RFCs as
\rfc{2045}-\rfc{2047}, and \rfc{2231}.

The primary distinguishing feature of the \module{email} package is
that it splits the parsing and generating of email messages from the
internal \emph{object model} representation of email.  Applications
using the \module{email} package deal primarily with objects; you can
add sub-objects to messages, remove sub-objects from messages,
completely re-arrange the contents, etc.  There is a separate parser
and a separate generator which handles the transformation from flat
text to the object model, and then back to flat text again.  There
are also handy subclasses for some common MIME object types, and a few
miscellaneous utilities that help with such common tasks as extracting
and parsing message field values, creating RFC-compliant dates, etc.

The following sections describe the functionality of the
\module{email} package.  The ordering follows a progression that
should be common in applications: an email message is read as flat
text from a file or other source, the text is parsed to produce an
object model representation of the email message, this model is
manipulated, and finally the model is rendered back into
flat text.

It is perfectly feasible to create the object model out of whole cloth
--- i.e. completely from scratch.  From there, a similar progression
can be taken as above.  

Also included are detailed specifications of all the classes and
modules that the \module{email} package provides, the exception
classes you might encounter while using the \module{email} package,
some auxiliary utilities, and a few examples.  For users of the older
\module{mimelib} package, or previous versions of the \module{email}
package, a section on differences and porting is provided.

\begin{seealso}
    \seemodule{smtplib}{SMTP protocol client}
\end{seealso}

\subsection{Representing an email message}
\declaremodule{standard}{email.Message}
\modulesynopsis{The base class representing email messages.}

The central class in the \module{email} package is the
\class{Message} class; it is the base class for the \module{email}
object model.  \class{Message} provides the core functionality for
setting and querying header fields, and for accessing message bodies.

Conceptually, a \class{Message} object consists of \emph{headers} and
\emph{payloads}.  Headers are \rfc{2822} style field names and
values where the field name and value are separated by a colon.  The
colon is not part of either the field name or the field value.

Headers are stored and returned in case-preserving form but are
matched case-insensitively.  There may also be a single
\emph{Unix-From} header, also known as the envelope header or the
\code{From_} header.  The payload is either a string in the case of
simple message objects, a list of \class{Message} objects for
multipart MIME documents, or a single \class{Message} instance for
\mimetype{message/rfc822} type objects.

\class{Message} objects provide a mapping style interface for
accessing the message headers, and an explicit interface for accessing
both the headers and the payload.  It provides convenience methods for
generating a flat text representation of the message object tree, for
accessing commonly used header parameters, and for recursively walking
over the object tree.

Here are the methods of the \class{Message} class:

\begin{classdesc}{Message}{}
The constructor takes no arguments.
\end{classdesc}

\begin{methoddesc}[Message]{as_string}{\optional{unixfrom}}
Return the entire formatted message as a string.  Optional
\var{unixfrom}, when true, specifies to include the \emph{Unix-From}
envelope header; it defaults to 0.
\end{methoddesc}

\begin{methoddesc}[Message]{__str__}{}
Equivalent to \method{aMessage.as_string(unixfrom=1)}.
\end{methoddesc}

\begin{methoddesc}[Message]{is_multipart}{}
Return 1 if the message's payload is a list of sub-\class{Message}
objects, otherwise return 0.  When \method{is_multipart()} returns 0,
the payload should either be a string object, or a single
\class{Message} instance.
\end{methoddesc}

\begin{methoddesc}[Message]{set_unixfrom}{unixfrom}
Set the \emph{Unix-From} (a.k.a envelope header or \code{From_}
header) to \var{unixfrom}, which should be a string.
\end{methoddesc}

\begin{methoddesc}[Message]{get_unixfrom}{}
Return the \emph{Unix-From} header.  Defaults to \code{None} if the
\emph{Unix-From} header was never set.
\end{methoddesc}

\begin{methoddesc}[Message]{add_payload}{payload}
Add \var{payload} to the message object's existing payload.  If, prior
to calling this method, the object's payload was \code{None}
(i.e. never before set), then after this method is called, the payload
will be the argument \var{payload}.

If the object's payload was already a list
(i.e. \method{is_multipart()} returns 1), then \var{payload} is
appended to the end of the existing payload list.

For any other type of existing payload, \method{add_payload()} will
transform the new payload into a list consisting of the old payload
and \var{payload}, but only if the document is already a MIME
multipart document.  This condition is satisfied if the message's
\mailheader{Content-Type} header's main type is either
\mimetype{multipart}, or there is no \mailheader{Content-Type}
header.  In any other situation,
\exception{MultipartConversionError} is raised.
\end{methoddesc}

\begin{methoddesc}[Message]{attach}{payload}
Synonymous with \method{add_payload()}.
\end{methoddesc}

\begin{methoddesc}[Message]{get_payload}{\optional{i\optional{, decode}}}
Return the current payload, which will be a list of \class{Message}
objects when \method{is_multipart()} returns 1, or a scalar (either a
string or a single \class{Message} instance) when
\method{is_multipart()} returns 0.

With optional \var{i}, \method{get_payload()} will return the
\var{i}-th element of the payload, counting from zero, if
\method{is_multipart()} returns 1.  An \exception{IndexError} will be raised
if \var{i} is less than 0 or greater than or equal to the number of
items in the payload.  If the payload is scalar
(i.e. \method{is_multipart()} returns 0) and \var{i} is given, a
\exception{TypeError} is raised.

Optional \var{decode} is a flag indicating whether the payload should be
decoded or not, according to the \mailheader{Content-Transfer-Encoding} header.
When true and the message is not a multipart, the payload will be
decoded if this header's value is \samp{quoted-printable} or
\samp{base64}.  If some other encoding is used, or
\mailheader{Content-Transfer-Encoding} header is
missing, the payload is returned as-is (undecoded).  If the message is
a multipart and the \var{decode} flag is true, then \code{None} is
returned.
\end{methoddesc}

\begin{methoddesc}[Message]{set_payload}{payload}
Set the entire message object's payload to \var{payload}.  It is the
client's responsibility to ensure the payload invariants.
\end{methoddesc}

The following methods implement a mapping-like interface for accessing
the message object's \rfc{2822} headers.  Note that there are some
semantic differences between these methods and a normal mapping
(i.e. dictionary) interface.  For example, in a dictionary there are
no duplicate keys, but here there may be duplicate message headers.  Also,
in dictionaries there is no guaranteed order to the keys returned by
\method{keys()}, but in a \class{Message} object, there is an explicit
order.  These semantic differences are intentional and are biased
toward maximal convenience.

Note that in all cases, any optional \emph{Unix-From} header the message
may have is not included in the mapping interface.

\begin{methoddesc}[Message]{__len__}{}
Return the total number of headers, including duplicates.
\end{methoddesc}

\begin{methoddesc}[Message]{__contains__}{name}
Return true if the message object has a field named \var{name}.
Matching is done case-insensitively and \var{name} should not include the
trailing colon.  Used for the \code{in} operator,
e.g.:

\begin{verbatim}
if 'message-id' in myMessage:
    print 'Message-ID:', myMessage['message-id']
\end{verbatim}
\end{methoddesc}

\begin{methoddesc}[Message]{__getitem__}{name}
Return the value of the named header field.  \var{name} should not
include the colon field separator.  If the header is missing,
\code{None} is returned; a \exception{KeyError} is never raised.

Note that if the named field appears more than once in the message's
headers, exactly which of those field values will be returned is
undefined.  Use the \method{get_all()} method to get the values of all
the extant named headers.
\end{methoddesc}

\begin{methoddesc}[Message]{__setitem__}{name, val}
Add a header to the message with field name \var{name} and value
\var{val}.  The field is appended to the end of the message's existing
fields.

Note that this does \emph{not} overwrite or delete any existing header
with the same name.  If you want to ensure that the new header is the
only one present in the message with field name
\var{name}, first use \method{__delitem__()} to delete all named
fields, e.g.:

\begin{verbatim}
del msg['subject']
msg['subject'] = 'Python roolz!'
\end{verbatim}
\end{methoddesc}

\begin{methoddesc}[Message]{__delitem__}{name}
Delete all occurrences of the field with name \var{name} from the
message's headers.  No exception is raised if the named field isn't
present in the headers.
\end{methoddesc}

\begin{methoddesc}[Message]{has_key}{name}
Return 1 if the message contains a header field named \var{name},
otherwise return 0.
\end{methoddesc}

\begin{methoddesc}[Message]{keys}{}
Return a list of all the message's header field names.  These keys
will be sorted in the order in which they were added to the message
via \method{__setitem__()}, and may contain duplicates.  Any fields
deleted and then subsequently re-added are always appended to the end
of the header list.
\end{methoddesc}

\begin{methoddesc}[Message]{values}{}
Return a list of all the message's field values.  These will be sorted
in the order in which they were added to the message via
\method{__setitem__()}, and may contain duplicates.  Any fields
deleted and then subsequently re-added are always appended to the end
of the header list.
\end{methoddesc}

\begin{methoddesc}[Message]{items}{}
Return a list of 2-tuples containing all the message's field headers and
values.  These will be sorted in the order in which they were added to
the message via \method{__setitem__()}, and may contain duplicates.
Any fields deleted and then subsequently re-added are always appended
to the end of the header list.
\end{methoddesc}

\begin{methoddesc}[Message]{get}{name\optional{, failobj}}
Return the value of the named header field.  This is identical to
\method{__getitem__()} except that optional \var{failobj} is returned
if the named header is missing (defaults to \code{None}).
\end{methoddesc}

Here are some additional useful methods:

\begin{methoddesc}[Message]{get_all}{name\optional{, failobj}}
Return a list of all the values for the field named \var{name}.  These
will be sorted in the order in which they were added to the message
via \method{__setitem__()}.  Any fields
deleted and then subsequently re-added are always appended to the end
of the list.

If there are no such named headers in the message, \var{failobj} is
returned (defaults to \code{None}).
\end{methoddesc}

\begin{methoddesc}[Message]{add_header}{_name, _value, **_params}
Extended header setting.  This method is similar to
\method{__setitem__()} except that additional header parameters can be
provided as keyword arguments.  \var{_name} is the header to set and
\var{_value} is the \emph{primary} value for the header.

For each item in the keyword argument dictionary \var{_params}, the
key is taken as the parameter name, with underscores converted to
dashes (since dashes are illegal in Python identifiers).  Normally,
the parameter will be added as \code{key="value"} unless the value is
\code{None}, in which case only the key will be added.

Here's an example:

\begin{verbatim}
msg.add_header('Content-Disposition', 'attachment', filename='bud.gif')
\end{verbatim}

This will add a header that looks like

\begin{verbatim}
Content-Disposition: attachment; filename="bud.gif"
\end{verbatim}
\end{methoddesc}

\begin{methoddesc}[Message]{get_type}{\optional{failobj}}
Return the message's content type, as a string of the form
\mimetype{maintype/subtype} as taken from the
\mailheader{Content-Type} header.
The returned string is coerced to lowercase.

If there is no \mailheader{Content-Type} header in the message,
\var{failobj} is returned (defaults to \code{None}).
\end{methoddesc}

\begin{methoddesc}[Message]{get_main_type}{\optional{failobj}}
Return the message's \emph{main} content type.  This essentially returns the
\var{maintype} part of the string returned by \method{get_type()}, with the
same semantics for \var{failobj}.
\end{methoddesc}

\begin{methoddesc}[Message]{get_subtype}{\optional{failobj}}
Return the message's sub-content type.  This essentially returns the
\var{subtype} part of the string returned by \method{get_type()}, with the
same semantics for \var{failobj}.
\end{methoddesc}

\begin{methoddesc}[Message]{get_params}{\optional{failobj\optional{, header}}}
Return the message's \mailheader{Content-Type} parameters, as a list.  The
elements of the returned list are 2-tuples of key/value pairs, as
split on the \character{=} sign.  The left hand side of the
\character{=} is the key, while the right hand side is the value.  If
there is no \character{=} sign in the parameter the value is the empty
string.  The value is always unquoted with \method{Utils.unquote()}.

Optional \var{failobj} is the object to return if there is no
\mailheader{Content-Type} header.  Optional \var{header} is the header to
search instead of \mailheader{Content-Type}.
\end{methoddesc}

\begin{methoddesc}[Message]{get_param}{param\optional{,
    failobj\optional{, header}}}
Return the value of the \mailheader{Content-Type} header's parameter
\var{param} as a string.  If the message has no \mailheader{Content-Type}
header or if there is no such parameter, then \var{failobj} is
returned (defaults to \code{None}).

Optional \var{header} if given, specifies the message header to use
instead of \mailheader{Content-Type}.
\end{methoddesc}

\begin{methoddesc}[Message]{get_charsets}{\optional{failobj}}
Return a list containing the character set names in the message.  If
the message is a \mimetype{multipart}, then the list will contain one
element for each subpart in the payload, otherwise, it will be a list
of length 1.

Each item in the list will be a string which is the value of the
\code{charset} parameter in the \mailheader{Content-Type} header for the
represented subpart.  However, if the subpart has no
\mailheader{Content-Type} header, no \code{charset} parameter, or is not of
the \mimetype{text} main MIME type, then that item in the returned list
will be \var{failobj}.
\end{methoddesc}

\begin{methoddesc}[Message]{get_filename}{\optional{failobj}}
Return the value of the \code{filename} parameter of the
\mailheader{Content-Disposition} header of the message, or \var{failobj} if
either the header is missing, or has no \code{filename} parameter.
The returned string will always be unquoted as per
\method{Utils.unquote()}.
\end{methoddesc}

\begin{methoddesc}[Message]{get_boundary}{\optional{failobj}}
Return the value of the \code{boundary} parameter of the
\mailheader{Content-Type} header of the message, or \var{failobj} if either
the header is missing, or has no \code{boundary} parameter.  The
returned string will always be unquoted as per
\method{Utils.unquote()}.
\end{methoddesc}

\begin{methoddesc}[Message]{set_boundary}{boundary}
Set the \code{boundary} parameter of the \mailheader{Content-Type} header
to \var{boundary}.  \method{set_boundary()} will always quote
\var{boundary} so you should not quote it yourself.  A
\exception{HeaderParseError} is raised if the message object has no
\mailheader{Content-Type} header.

Note that using this method is subtly different than deleting the old
\mailheader{Content-Type} header and adding a new one with the new boundary
via \method{add_header()}, because \method{set_boundary()} preserves the
order of the \mailheader{Content-Type} header in the list of headers.
However, it does \emph{not} preserve any continuation lines which may
have been present in the original \mailheader{Content-Type} header.
\end{methoddesc}

\begin{methoddesc}[Message]{walk}{}
The \method{walk()} method is an all-purpose generator which can be
used to iterate over all the parts and subparts of a message object
tree, in depth-first traversal order.  You will typically use
\method{walk()} as the iterator in a \code{for ... in} loop; each
iteration returns the next subpart.

Here's an example that prints the MIME type of every part of a message
object tree:

\begin{verbatim}
>>> for part in msg.walk():
>>>     print part.get_type('text/plain')
multipart/report
text/plain
message/delivery-status
text/plain
text/plain
message/rfc822
\end{verbatim}
\end{methoddesc}

\class{Message} objects can also optionally contain two instance
attributes, which can be used when generating the plain text of a MIME
message.

\begin{datadesc}{preamble}
The format of a MIME document allows for some text between the blank
line following the headers, and the first multipart boundary string.
Normally, this text is never visible in a MIME-aware mail reader
because it falls outside the standard MIME armor.  However, when
viewing the raw text of the message, or when viewing the message in a
non-MIME aware reader, this text can become visible.

The \var{preamble} attribute contains this leading extra-armor text
for MIME documents.  When the \class{Parser} discovers some text after
the headers but before the first boundary string, it assigns this text
to the message's \var{preamble} attribute.  When the \class{Generator}
is writing out the plain text representation of a MIME message, and it
finds the message has a \var{preamble} attribute, it will write this
text in the area between the headers and the first boundary.

Note that if the message object has no preamble, the
\var{preamble} attribute will be \code{None}.
\end{datadesc}

\begin{datadesc}{epilogue}
The \var{epilogue} attribute acts the same way as the \var{preamble}
attribute, except that it contains text that appears between the last
boundary and the end of the message.

One note: when generating the flat text for a \mimetype{multipart}
message that has no \var{epilogue} (using the standard
\class{Generator} class), no newline is added after the closing
boundary line.  If the message object has an \var{epilogue} and its
value does not start with a newline, a newline is printed after the
closing boundary.  This seems a little clumsy, but it makes the most
practical sense.  The upshot is that if you want to ensure that a
newline get printed after your closing \mimetype{multipart} boundary,
set the \var{epilogue} to the empty string.
\end{datadesc}


\subsection{Parsing email messages}
\section{\module{email.Parser} ---
         Parsing flat text email messages}

\declaremodule{standard}{email.Parser}
\modulesynopsis{Parse flat text email messages to produce a message
	        object tree.}
\sectionauthor{Barry A. Warsaw}{barry@zope.com}

\versionadded{2.2}

The \module{Parser} module provides a single class, the \class{Parser}
class, which is used to take a message in flat text form and create
the associated object model.  The resulting object tree can then be
manipulated using the \class{Message} class interface as described in
\refmodule{email.Message}, and turned over
to a generator (as described in \refmodule{emamil.Generator}) to
return the textual representation of the message.  It is intended that
the \class{Parser} to \class{Generator} path be idempotent if the
object model isn't modified in between.

\subsection{Parser class API}

\begin{classdesc}{Parser}{\optional{_class}}
The constructor for the \class{Parser} class takes a single optional
argument \var{_class}.  This must be callable factory (i.e. a function
or a class), and it is used whenever a sub-message object needs to be
created.  It defaults to \class{Message} (see
\refmodule{email.Message}).  \var{_class} will be called with zero
arguments.
\end{classdesc}

The other public \class{Parser} methods are:

\begin{methoddesc}[Parser]{parse}{fp}
Read all the data from the file-like object \var{fp}, parse the
resulting text, and return the root message object.  \var{fp} must
support both the \method{readline()} and the \method{read()} methods
on file-like objects.

The text contained in \var{fp} must be formatted as a block of \rfc{2822}
style headers and header continuation lines, optionally preceeded by a
\emph{Unix-From} header.  The header block is terminated either by the
end of the data or by a blank line.  Following the header block is the
body of the message (which may contain MIME-encoded subparts).
\end{methoddesc}

\begin{methoddesc}[Parser]{parsestr}{text}
Similar to the \method{parse()} method, except it takes a string
object instead of a file-like object.  Calling this method on a string
is exactly equivalent to wrapping \var{text} in a \class{StringIO}
instance first and calling \method{parse()}.
\end{methoddesc}

Since creating a message object tree from a string or a file object is
such a common task, two functions are provided as a convenience.  They
are available in the top-level \module{email} package namespace.

\begin{funcdesc}{message_from_string}{s\optional{, _class}}
Return a message object tree from a string.  This is exactly
equivalent to \code{Parser().parsestr(s)}.  Optional \var{_class} is
interpreted as with the \class{Parser} class constructor.
\end{funcdesc}

\begin{funcdesc}{message_from_file}{fp\optional{, _class}}
Return a message object tree from an open file object.  This is exactly
equivalent to \code{Parser().parse(fp)}.  Optional \var{_class} is
interpreted as with the \class{Parser} class constructor.
\end{funcdesc}

Here's an example of how you might use this at an interactive Python
prompt:

\begin{verbatim}
>>> import email
>>> msg = email.message_from_string(myString)
\end{verbatim}

\subsection{Additional notes}

Here are some notes on the parsing semantics:

\begin{itemize}
\item Most non-\code{multipart} type messages are parsed as a single
      message object with a string payload.  These objects will return
      0 for \method{is_multipart()}.
\item One exception is for \code{message/delivery-status} type
      messages.  Because such the body of such messages consist of
      blocks of headers, \class{Parser} will create a non-multipart
      object containing non-multipart subobjects for each header
      block.
\item Another exception is for \code{message/*} types (i.e. more
      general than \code{message/delivery-status}.  These are
      typically \code{message/rfc822} type messages, represented as a
      non-multipart object containing a singleton payload, another
      non-multipart \class{Message} instance.
\end{itemize}


\subsection{Generating MIME documents}
\section{\module{email.Generator} ---
         Generating flat text from an email message object tree}

\declaremodule{standard}{email.Generator}
\modulesynopsis{Generate flat text email messages to from a message
	        object tree.}
\sectionauthor{Barry A. Warsaw}{barry@zope.com}

\versionadded{2.2}

The \class{Generator} class is used to render a message object model
into its flat text representation, including MIME encoding any
sub-messages, generating the correct \rfc{2822} headers, etc.  Here
are the public methods of the \class{Generator} class.

\begin{classdesc}{Generator}{outfp\optional{, mangle_from_\optional{,
    maxheaderlen}}}
The constructor for the \class{Generator} class takes a file-like
object called \var{outfp} for an argument.  \var{outfp} must support
the \method{write()} method and be usable as the output file in a
Python 2.0 extended print statement.

Optional \var{mangle_from_} is a flag that, when true, puts a ``>''
character in front of any line in the body that starts exactly as
\samp{From } (i.e. \code{From} followed by a space at the front of the
line).  This is the only guaranteed portable way to avoid having such
lines be mistaken for \emph{Unix-From} headers (see
\url{http://home.netscape.com/eng/mozilla/2.0/relnotes/demo/content-length.html}
 for details).

Optional \var{maxheaderlen} specifies the longest length for a
non-continued header.  When a header line is longer than
\var{maxheaderlen} (in characters, with tabs expanded to 8 spaces),
the header will be broken on semicolons and continued as per
\rfc{2822}.  If no semicolon is found, then the header is left alone.
Set to zero to disable wrapping headers.  Default is 78, as
recommended (but not required) by \rfc{2822}.
\end{classdesc}

The other public \class{Generator} methods are:

\begin{methoddesc}[Generator]{__call__}{msg\optional{, unixfrom}}
Print the textual representation of the message object tree rooted at
\var{msg} to the output file specified when the \class{Generator}
instance was created.  Sub-objects are visited depth-first and the
resulting text will be properly MIME encoded.

Optional \var{unixfrom} is a flag that forces the printing of the
\emph{Unix-From} (a.k.a. envelope header or \code{From_} header)
delimiter before the first \rfc{2822} header of the root message
object.  If the root object has no \emph{Unix-From} header, a standard
one is crafted.  By default, this is set to 0 to inhibit the printing
of the \emph{Unix-From} delimiter.

Note that for sub-objects, no \emph{Unix-From} header is ever printed.
\end{methoddesc}

\begin{methoddesc}[Generator]{write}{s}
Write the string \var{s} to the underlying file object,
i.e. \var{outfp} passed to \class{Generator}'s constructor.  This
provides just enough file-like API for \class{Generator} instances to
be used in extended print statements.
\end{methoddesc}

As a convenience, see the methods \method{Message.as_string()} and
\code{str(aMessage)}, a.k.a. \method{Message.__str__()}, which
simplify the generation of a formatted string representation of a
message object.  For more detail, see \refmodule{email.Message}.


\subsection{Creating email and MIME objects from scratch}
Ordinarily, you get a message object structure by passing a file or
some text to a parser, which parses the text and returns the root
message object.  However you can also build a complete message
structure from scratch, or even individual \class{Message} objects by
hand.  In fact, you can also take an existing structure and add new
\class{Message} objects, move them around, etc.  This makes a very
convenient interface for slicing-and-dicing MIME messages.

You can create a new object structure by creating \class{Message}
instances, adding attachments and all the appropriate headers manually.
For MIME messages though, the \module{email} package provides some
convenient subclasses to make things easier.  Each of these classes
should be imported from a module with the same name as the class, from
within the \module{email} package.  E.g.:

\begin{verbatim}
import email.MIMEImage.MIMEImage
\end{verbatim}

or

\begin{verbatim}
from email.MIMEText import MIMEText
\end{verbatim}

Here are the classes:

\begin{classdesc}{MIMEBase}{_maintype, _subtype, **_params}
This is the base class for all the MIME-specific subclasses of
\class{Message}.  Ordinarily you won't create instances specifically
of \class{MIMEBase}, although you could.  \class{MIMEBase} is provided
primarily as a convenient base class for more specific MIME-aware
subclasses.

\var{_maintype} is the \mailheader{Content-Type} major type
(e.g. \mimetype{text} or \mimetype{image}), and \var{_subtype} is the
\mailheader{Content-Type} minor type 
(e.g. \mimetype{plain} or \mimetype{gif}).  \var{_params} is a parameter
key/value dictionary and is passed directly to
\method{Message.add_header()}.

The \class{MIMEBase} class always adds a \mailheader{Content-Type} header
(based on \var{_maintype}, \var{_subtype}, and \var{_params}), and a
\mailheader{MIME-Version} header (always set to \code{1.0}).
\end{classdesc}

\begin{classdesc}{MIMENonMultipart}{}
A subclass of \class{MIMEBase}, this is an intermediate base class for
MIME messages that are not \mimetype{multipart}.  The primary purpose
of this class is to prevent the use of the \method{attach()} method,
which only makes sense for \mimetype{multipart} messages.  If
\method{attach()} is called, a \exception{MultipartConversionError}
exception is raised.

\versionadded{2.2.2}
\end{classdesc}

\begin{classdesc}{MIMEMultipart}{\optional{subtype\optional{,
    boundary\optional{, _subparts\optional{, _params}}}}}

A subclass of \class{MIMEBase}, this is an intermediate base class for
MIME messages that are \mimetype{multipart}.  Optional \var{_subtype}
defaults to \mimetype{mixed}, but can be used to specify the subtype
of the message.  A \mailheader{Content-Type} header of
\mimetype{multipart/}\var{_subtype} will be added to the message
object.  A \mailheader{MIME-Version} header will also be added.

Optional \var{boundary} is the multipart boundary string.  When
\code{None} (the default), the boundary is calculated when needed.

\var{_subparts} is a sequence of initial subparts for the payload.  It
must be possible to convert this sequence to a list.  You can always
attach new subparts to the message by using the
\method{Message.attach()} method.

Additional parameters for the \mailheader{Content-Type} header are
taken from the keyword arguments, or passed into the \var{_params}
argument, which is a keyword dictionary.

\versionadded{2.2.2}
\end{classdesc}

\begin{classdesc}{MIMEAudio}{_audiodata\optional{, _subtype\optional{,
    _encoder\optional{, **_params}}}}

A subclass of \class{MIMENonMultipart}, the \class{MIMEAudio} class
is used to create MIME message objects of major type \mimetype{audio}.
\var{_audiodata} is a string containing the raw audio data.  If this
data can be decoded by the standard Python module \refmodule{sndhdr},
then the subtype will be automatically included in the
\mailheader{Content-Type} header.  Otherwise you can explicitly specify the
audio subtype via the \var{_subtype} parameter.  If the minor type could
not be guessed and \var{_subtype} was not given, then \exception{TypeError}
is raised.

Optional \var{_encoder} is a callable (i.e. function) which will
perform the actual encoding of the audio data for transport.  This
callable takes one argument, which is the \class{MIMEAudio} instance.
It should use \method{get_payload()} and \method{set_payload()} to
change the payload to encoded form.  It should also add any
\mailheader{Content-Transfer-Encoding} or other headers to the message
object as necessary.  The default encoding is base64.  See the
\refmodule{email.Encoders} module for a list of the built-in encoders.

\var{_params} are passed straight through to the base class constructor.
\end{classdesc}

\begin{classdesc}{MIMEImage}{_imagedata\optional{, _subtype\optional{,
    _encoder\optional{, **_params}}}}

A subclass of \class{MIMENonMultipart}, the \class{MIMEImage} class is
used to create MIME message objects of major type \mimetype{image}.
\var{_imagedata} is a string containing the raw image data.  If this
data can be decoded by the standard Python module \refmodule{imghdr},
then the subtype will be automatically included in the
\mailheader{Content-Type} header.  Otherwise you can explicitly specify the
image subtype via the \var{_subtype} parameter.  If the minor type could
not be guessed and \var{_subtype} was not given, then \exception{TypeError}
is raised.

Optional \var{_encoder} is a callable (i.e. function) which will
perform the actual encoding of the image data for transport.  This
callable takes one argument, which is the \class{MIMEImage} instance.
It should use \method{get_payload()} and \method{set_payload()} to
change the payload to encoded form.  It should also add any
\mailheader{Content-Transfer-Encoding} or other headers to the message
object as necessary.  The default encoding is base64.  See the
\refmodule{email.Encoders} module for a list of the built-in encoders.

\var{_params} are passed straight through to the \class{MIMEBase}
constructor.
\end{classdesc}

\begin{classdesc}{MIMEMessage}{_msg\optional{, _subtype}}
A subclass of \class{MIMENonMultipart}, the \class{MIMEMessage} class
is used to create MIME objects of main type \mimetype{message}.
\var{_msg} is used as the payload, and must be an instance of class
\class{Message} (or a subclass thereof), otherwise a
\exception{TypeError} is raised.

Optional \var{_subtype} sets the subtype of the message; it defaults
to \mimetype{rfc822}.
\end{classdesc}

\begin{classdesc}{MIMEText}{_text\optional{, _subtype\optional{, _charset}}}
A subclass of \class{MIMENonMultipart}, the \class{MIMEText} class is
used to create MIME objects of major type \mimetype{text}.
\var{_text} is the string for the payload.  \var{_subtype} is the
minor type and defaults to \mimetype{plain}.  \var{_charset} is the
character set of the text and is passed as a parameter to the
\class{MIMENonMultipart} constructor; it defaults to \code{us-ascii}.  No
guessing or encoding is performed on the text data.

\versionchanged[The previously deprecated \var{_encoding} argument has
been removed.  Encoding happens implicitly based on the \var{_charset}
argument]{2.4}
\end{classdesc}


\subsection{Headers, Character sets, and Internationalization}
\declaremodule{standard}{email.header}
\modulesynopsis{Representing non-ASCII headers}

\rfc{2822} is the base standard that describes the format of email
messages.  It derives from the older \rfc{822} standard which came
into widespread use at a time when most email was composed of \ASCII{}
characters only.  \rfc{2822} is a specification written assuming email
contains only 7-bit \ASCII{} characters.

Of course, as email has been deployed worldwide, it has become
internationalized, such that language specific character sets can now
be used in email messages.  The base standard still requires email
messages to be transferred using only 7-bit \ASCII{} characters, so a
slew of RFCs have been written describing how to encode email
containing non-\ASCII{} characters into \rfc{2822}-compliant format.
These RFCs include \rfc{2045}, \rfc{2046}, \rfc{2047}, and \rfc{2231}.
The \module{email} package supports these standards in its
\module{email.header} and \module{email.charset} modules.

If you want to include non-\ASCII{} characters in your email headers,
say in the \mailheader{Subject} or \mailheader{To} fields, you should
use the \class{Header} class and assign the field in the
\class{Message} object to an instance of \class{Header} instead of
using a string for the header value.  Import the \class{Header} class from the
\module{email.header} module.  For example:

\begin{verbatim}
>>> from email.message import Message
>>> from email.header import Header
>>> msg = Message()
>>> h = Header('p\xf6stal', 'iso-8859-1')
>>> msg['Subject'] = h
>>> print msg.as_string()
Subject: =?iso-8859-1?q?p=F6stal?=


\end{verbatim}

Notice here how we wanted the \mailheader{Subject} field to contain a
non-\ASCII{} character?  We did this by creating a \class{Header}
instance and passing in the character set that the byte string was
encoded in.  When the subsequent \class{Message} instance was
flattened, the \mailheader{Subject} field was properly \rfc{2047}
encoded.  MIME-aware mail readers would show this header using the
embedded ISO-8859-1 character.

\versionadded{2.2.2}

Here is the \class{Header} class description:

\begin{classdesc}{Header}{\optional{s\optional{, charset\optional{,
    maxlinelen\optional{, header_name\optional{, continuation_ws\optional{,
    errors}}}}}}}
Create a MIME-compliant header that can contain strings in different
character sets.

Optional \var{s} is the initial header value.  If \code{None} (the
default), the initial header value is not set.  You can later append
to the header with \method{append()} method calls.  \var{s} may be a
byte string or a Unicode string, but see the \method{append()}
documentation for semantics.

Optional \var{charset} serves two purposes: it has the same meaning as
the \var{charset} argument to the \method{append()} method.  It also
sets the default character set for all subsequent \method{append()}
calls that omit the \var{charset} argument.  If \var{charset} is not
provided in the constructor (the default), the \code{us-ascii}
character set is used both as \var{s}'s initial charset and as the
default for subsequent \method{append()} calls.

The maximum line length can be specified explicit via
\var{maxlinelen}.  For splitting the first line to a shorter value (to
account for the field header which isn't included in \var{s},
e.g. \mailheader{Subject}) pass in the name of the field in
\var{header_name}.  The default \var{maxlinelen} is 76, and the
default value for \var{header_name} is \code{None}, meaning it is not
taken into account for the first line of a long, split header.

Optional \var{continuation_ws} must be \rfc{2822}-compliant folding
whitespace, and is usually either a space or a hard tab character.
This character will be prepended to continuation lines.
\end{classdesc}

Optional \var{errors} is passed straight through to the
\method{append()} method.

\begin{methoddesc}[Header]{append}{s\optional{, charset\optional{, errors}}}
Append the string \var{s} to the MIME header.

Optional \var{charset}, if given, should be a \class{Charset} instance
(see \refmodule{email.charset}) or the name of a character set, which
will be converted to a \class{Charset} instance.  A value of
\code{None} (the default) means that the \var{charset} given in the
constructor is used.

\var{s} may be a byte string or a Unicode string.  If it is a byte
string (i.e. \code{isinstance(s, str)} is true), then
\var{charset} is the encoding of that byte string, and a
\exception{UnicodeError} will be raised if the string cannot be
decoded with that character set.

If \var{s} is a Unicode string, then \var{charset} is a hint
specifying the character set of the characters in the string.  In this
case, when producing an \rfc{2822}-compliant header using \rfc{2047}
rules, the Unicode string will be encoded using the following charsets
in order: \code{us-ascii}, the \var{charset} hint, \code{utf-8}.  The
first character set to not provoke a \exception{UnicodeError} is used.

Optional \var{errors} is passed through to any \function{unicode()} or
\function{ustr.encode()} call, and defaults to ``strict''.
\end{methoddesc}

\begin{methoddesc}[Header]{encode}{\optional{splitchars}}
Encode a message header into an RFC-compliant format, possibly
wrapping long lines and encapsulating non-\ASCII{} parts in base64 or
quoted-printable encodings.  Optional \var{splitchars} is a string
containing characters to split long ASCII lines on, in rough support
of \rfc{2822}'s \emph{highest level syntactic breaks}.  This doesn't
affect \rfc{2047} encoded lines.
\end{methoddesc}

The \class{Header} class also provides a number of methods to support
standard operators and built-in functions.

\begin{methoddesc}[Header]{__str__}{}
A synonym for \method{Header.encode()}.  Useful for
\code{str(aHeader)}.
\end{methoddesc}

\begin{methoddesc}[Header]{__unicode__}{}
A helper for the built-in \function{unicode()} function.  Returns the
header as a Unicode string.
\end{methoddesc}

\begin{methoddesc}[Header]{__eq__}{other}
This method allows you to compare two \class{Header} instances for equality.
\end{methoddesc}

\begin{methoddesc}[Header]{__ne__}{other}
This method allows you to compare two \class{Header} instances for inequality.
\end{methoddesc}

The \module{email.header} module also provides the following
convenient functions.

\begin{funcdesc}{decode_header}{header}
Decode a message header value without converting the character set.
The header value is in \var{header}.

This function returns a list of \code{(decoded_string, charset)} pairs
containing each of the decoded parts of the header.  \var{charset} is
\code{None} for non-encoded parts of the header, otherwise a lower
case string containing the name of the character set specified in the
encoded string.

Here's an example:

\begin{verbatim}
>>> from email.header import decode_header
>>> decode_header('=?iso-8859-1?q?p=F6stal?=')
[('p\xf6stal', 'iso-8859-1')]
\end{verbatim}
\end{funcdesc}

\begin{funcdesc}{make_header}{decoded_seq\optional{, maxlinelen\optional{,
    header_name\optional{, continuation_ws}}}}
Create a \class{Header} instance from a sequence of pairs as returned
by \function{decode_header()}.

\function{decode_header()} takes a header value string and returns a
sequence of pairs of the format \code{(decoded_string, charset)} where
\var{charset} is the name of the character set.

This function takes one of those sequence of pairs and returns a
\class{Header} instance.  Optional \var{maxlinelen},
\var{header_name}, and \var{continuation_ws} are as in the
\class{Header} constructor.
\end{funcdesc}


\subsection{Encoders}
\declaremodule{standard}{email.Encoders}
\modulesynopsis{Encoders for email message payloads.}

When creating \class{Message} objects from scratch, you often need to
encode the payloads for transport through compliant mail servers.
This is especially true for \mimetype{image/*} and \mimetype{text/*}
type messages containing binary data.

The \module{email} package provides some convenient encodings in its
\module{Encoders} module.  These encoders are actually used by the
\class{MIMEAudio} and \class{MIMEImage} class constructors to provide default
encodings.  All encoder functions take exactly one argument, the message
object to encode.  They usually extract the payload, encode it, and reset the
payload to this newly encoded value.  They should also set the
\mailheader{Content-Transfer-Encoding} header as appropriate.

Here are the encoding functions provided:

\begin{funcdesc}{encode_quopri}{msg}
Encodes the payload into quoted-printable form and sets the
\mailheader{Content-Transfer-Encoding} header to
\code{quoted-printable}\footnote{Note that encoding with
\method{encode_quopri()} also encodes all tabs and space characters in
the data.}.
This is a good encoding to use when most of your payload is normal
printable data, but contains a few unprintable characters.
\end{funcdesc}

\begin{funcdesc}{encode_base64}{msg}
Encodes the payload into base64 form and sets the
\mailheader{Content-Transfer-Encoding} header to
\code{base64}.  This is a good encoding to use when most of your payload
is unprintable data since it is a more compact form than
quoted-printable.  The drawback of base64 encoding is that it
renders the text non-human readable.
\end{funcdesc}

\begin{funcdesc}{encode_7or8bit}{msg}
This doesn't actually modify the message's payload, but it does set
the \mailheader{Content-Transfer-Encoding} header to either \code{7bit} or
\code{8bit} as appropriate, based on the payload data.
\end{funcdesc}

\begin{funcdesc}{encode_noop}{msg}
This does nothing; it doesn't even set the
\mailheader{Content-Transfer-Encoding} header.
\end{funcdesc}


\subsection{Exception classes}
\declaremodule{standard}{email.Errors}
\modulesynopsis{The exception classes used by the email package.}

The following exception classes are defined in the
\module{email.Errors} module:

\begin{excclassdesc}{MessageError}{}
This is the base class for all exceptions that the \module{email}
package can raise.  It is derived from the standard
\exception{Exception} class and defines no additional methods.
\end{excclassdesc}

\begin{excclassdesc}{MessageParseError}{}
This is the base class for exceptions thrown by the \class{Parser}
class.  It is derived from \exception{MessageError}.
\end{excclassdesc}

\begin{excclassdesc}{HeaderParseError}{}
Raised under some error conditions when parsing the \rfc{2822} headers of
a message, this class is derived from \exception{MessageParseError}.
It can be raised from the \method{Parser.parse()} or
\method{Parser.parsestr()} methods.

Situations where it can be raised include finding an envelope
header after the first \rfc{2822} header of the message, finding a
continuation line before the first \rfc{2822} header is found, or finding
a line in the headers which is neither a header or a continuation
line.
\end{excclassdesc}

\begin{excclassdesc}{BoundaryError}{}
Raised under some error conditions when parsing the \rfc{2822} headers of
a message, this class is derived from \exception{MessageParseError}.
It can be raised from the \method{Parser.parse()} or
\method{Parser.parsestr()} methods.

Situations where it can be raised include not being able to find the
starting or terminating boundary in a \mimetype{multipart/*} message
when strict parsing is used.
\end{excclassdesc}

\begin{excclassdesc}{MultipartConversionError}{}
Raised when a payload is added to a \class{Message} object using
\method{add_payload()}, but the payload is already a scalar and the
message's \mailheader{Content-Type} main type is not either
\mimetype{multipart} or missing.  \exception{MultipartConversionError}
multiply inherits from \exception{MessageError} and the built-in
\exception{TypeError}.

Since \method{Message.add_payload()} is deprecated, this exception is
rarely raised in practice.  However the exception may also be raised
if the \method{attach()} method is called on an instance of a class
derived from \class{MIMENonMultipart} (e.g. \class{MIMEImage}).
\end{excclassdesc}

Here's the list of the defects that the \class{FeedParser} can find while
parsing messages.  Note that the defects are added to the message where the
problem was found, so for example, if a message nested inside a
\mimetype{multipart/alternative} had a malformed header, that nested message
object would have a defect, but the containing messages would not.

All defect classes are subclassed from \class{email.Errors.MessageDefect}, but
this class is \emph{not} an exception!

\versionadded[All the defect classes were added]{2.4}

\begin{itemize}
\item \class{NoBoundaryInMultipartDefect} -- A message claimed to be a
      multipart, but had no \mimetype{boundary} parameter.

\item \class{StartBoundaryNotFoundDefect} -- The start boundary claimed in the
      \mailheader{Content-Type} header was never found.

\item \class{FirstHeaderLineIsContinuationDefect} -- The message had a
      continuation line as its first header line.

\item \class{MisplacedEnvelopeHeaderDefect} - A ``Unix From'' header was found
      in the middle of a header block.

\item \class{MalformedHeaderDefect} -- A header was found that was missing a
      colon, or was otherwise malformed.

\item \class{MultipartInvariantViolationDefect} -- A message claimed to be a
      \mimetype{multipart}, but no subparts were found.  Note that when a
      message has this defect, its \method{is_multipart()} method may return
      false even though its content type claims to be \mimetype{multipart}.
\end{itemize}


\subsection{Miscellaneous utilities}
\declaremodule{standard}{email.Utils}
\modulesynopsis{Miscellaneous email package utilities.}

There are several useful utilities provided with the \module{email}
package.

\begin{funcdesc}{quote}{str}
Return a new string with backslashes in \var{str} replaced by two
backslashes and double quotes replaced by backslash-double quote.
\end{funcdesc}

\begin{funcdesc}{unquote}{str}
Return a new string which is an \emph{unquoted} version of \var{str}.
If \var{str} ends and begins with double quotes, they are stripped
off.  Likewise if \var{str} ends and begins with angle brackets, they
are stripped off.
\end{funcdesc}

\begin{funcdesc}{parseaddr}{address}
Parse address -- which should be the value of some address-containing
field such as \mailheader{To} or \mailheader{Cc} -- into its constituent
\emph{realname} and \emph{email address} parts.  Returns a tuple of that
information, unless the parse fails, in which case a 2-tuple of
\code{(None, None)} is returned.
\end{funcdesc}

\begin{funcdesc}{dump_address_pair}{pair}
The inverse of \method{parseaddr()}, this takes a 2-tuple of the form
\code{(realname, email_address)} and returns the string value suitable
for a \mailheader{To} or \mailheader{Cc} header.  If the first element of
\var{pair} is false, then the second element is returned unmodified.
\end{funcdesc}

\begin{funcdesc}{getaddresses}{fieldvalues}
This method returns a list of 2-tuples of the form returned by
\code{parseaddr()}.  \var{fieldvalues} is a sequence of header field
values as might be returned by \method{Message.getall()}.  Here's a
simple example that gets all the recipients of a message:

\begin{verbatim}
from email.Utils import getaddresses

tos = msg.get_all('to')
ccs = msg.get_all('cc')
resent_tos = msg.get_all('resent-to')
resent_ccs = msg.get_all('resent-cc')
all_recipients = getaddresses(tos + ccs + resent_tos + resent_ccs)
\end{verbatim}
\end{funcdesc}

\begin{funcdesc}{decode}{s}
This method decodes a string according to the rules in \rfc{2047}.  It
returns the decoded string as a Python unicode string.
\end{funcdesc}

\begin{funcdesc}{encode}{s\optional{, charset\optional{, encoding}}}
This method encodes a string according to the rules in \rfc{2047}.  It
is not actually the inverse of \function{decode()} since it doesn't
handle multiple character sets or multiple string parts needing
encoding.  In fact, the input string \var{s} must already be encoded
in the \var{charset} character set (Python can't reliably guess what
character set a string might be encoded in).  The default
\var{charset} is \samp{iso-8859-1}.

\var{encoding} must be either the letter \character{q} for
Quoted-Printable or \character{b} for Base64 encoding.  If
neither, a \exception{ValueError} is raised.  Both the \var{charset} and
the \var{encoding} strings are case-insensitive, and coerced to lower
case in the returned string.
\end{funcdesc}

\begin{funcdesc}{parsedate}{date}
Attempts to parse a date according to the rules in \rfc{2822}.
however, some mailers don't follow that format as specified, so
\function{parsedate()} tries to guess correctly in such cases. 
\var{date} is a string containing an \rfc{2822} date, such as 
\code{"Mon, 20 Nov 1995 19:12:08 -0500"}.  If it succeeds in parsing
the date, \function{parsedate()} returns a 9-tuple that can be passed
directly to \function{time.mktime()}; otherwise \code{None} will be
returned.  Note that fields 6, 7, and 8 of the result tuple are not
usable.
\end{funcdesc}

\begin{funcdesc}{parsedate_tz}{date}
Performs the same function as \function{parsedate()}, but returns
either \code{None} or a 10-tuple; the first 9 elements make up a tuple
that can be passed directly to \function{time.mktime()}, and the tenth
is the offset of the date's timezone from UTC (which is the official
term for Greenwich Mean Time)\footnote{Note that the sign of the timezone
offset is the opposite of the sign of the \code{time.timezone}
variable for the same timezone; the latter variable follows the
\POSIX{} standard while this module follows \rfc{2822}.}.  If the input
string has no timezone, the last element of the tuple returned is
\code{None}.  Note that fields 6, 7, and 8 of the result tuple are not
usable.
\end{funcdesc}

\begin{funcdesc}{mktime_tz}{tuple}
Turn a 10-tuple as returned by \function{parsedate_tz()} into a UTC
timestamp.  It the timezone item in the tuple is \code{None}, assume
local time.  Minor deficiency: \function{mktime_tz()} interprets the
first 8 elements of \var{tuple} as a local time and then compensates
for the timezone difference.  This may yield a slight error around
changes in daylight savings time, though not worth worring about for
common use.
\end{funcdesc}

\begin{funcdesc}{formatdate}{\optional{timeval}}
Returns the time formatted as per Internet standards \rfc{2822}
and updated by \rfc{1123}.  If \var{timeval} is provided, then it
should be a floating point time value as expected by
\method{time.gmtime()}, otherwise the current time is used.
\end{funcdesc}


\subsection{Iterators}
\declaremodule{standard}{email.iterators}
\modulesynopsis{Iterate over a  message object tree.}

Iterating over a message object tree is fairly easy with the
\method{Message.walk()} method.  The \module{email.iterators} module
provides some useful higher level iterations over message object
trees.

\begin{funcdesc}{body_line_iterator}{msg\optional{, decode}}
This iterates over all the payloads in all the subparts of \var{msg},
returning the string payloads line-by-line.  It skips over all the
subpart headers, and it skips over any subpart with a payload that
isn't a Python string.  This is somewhat equivalent to reading the
flat text representation of the message from a file using
\method{readline()}, skipping over all the intervening headers.

Optional \var{decode} is passed through to \method{Message.get_payload()}.
\end{funcdesc}

\begin{funcdesc}{typed_subpart_iterator}{msg\optional{,
    maintype\optional{, subtype}}}
This iterates over all the subparts of \var{msg}, returning only those
subparts that match the MIME type specified by \var{maintype} and
\var{subtype}.

Note that \var{subtype} is optional; if omitted, then subpart MIME
type matching is done only with the main type.  \var{maintype} is
optional too; it defaults to \mimetype{text}.

Thus, by default \function{typed_subpart_iterator()} returns each
subpart that has a MIME type of \mimetype{text/*}.
\end{funcdesc}

The following function has been added as a useful debugging tool.  It
should \emph{not} be considered part of the supported public interface
for the package.

\begin{funcdesc}{_structure}{msg\optional{, fp\optional{, level}}}
Prints an indented representation of the content types of the
message object structure.  For example:

\begin{verbatim}
>>> msg = email.message_from_file(somefile)
>>> _structure(msg)
multipart/mixed
    text/plain
    text/plain
    multipart/digest
        message/rfc822
            text/plain
        message/rfc822
            text/plain
        message/rfc822
            text/plain
        message/rfc822
            text/plain
        message/rfc822
            text/plain
    text/plain
\end{verbatim}

Optional \var{fp} is a file-like object to print the output to.  It
must be suitable for Python's extended print statement.  \var{level}
is used internally.
\end{funcdesc}


\subsection{Differences from \module{email} v1 (up to Python 2.2.1)}

Version 1 of the \module{email} package was bundled with Python
releases up to Python 2.2.1.  Version 2 was developed for the Python
2.3 release, and backported to Python 2.2.2.  It was also available as
a separate distutils based package.  \module{email} version 2 is
almost entirely backwards compatible with version 1, with the
following differences:

\begin{itemize}
\item The \module{email.Header} and \module{email.Charset} modules
      have been added.
\item The pickle format for \class{Message} instances has changed.
      Since this was never (and still isn't) formally defined, this
      isn't considered a backwards incompatibility.  However if your
      application pickles and unpickles \class{Message} instances, be
      aware that in \module{email} version 2, \class{Message}
      instances now have private variables \var{_charset} and
      \var{_default_type}.
\item Several methods in the \class{Message} class have been
      deprecated, or their signatures changes.  Also, many new methods
      have been added.  See the documentation for the \class{Message}
      class for deatils.  The changes should be completely backwards
      compatible.
\item The object structure has changed in the face of
      \mimetype{message/rfc822} content types.  In \module{email}
      version 1, such a type would be represented by a scalar payload,
      i.e. the container message's \method{is_multipart()} returned
      false, \method{get_payload()} was not a list object, and was
      actually a \class{Message} instance.

      This structure was inconsistent with the rest of the package, so
      the object representation for \mimetype{message/rfc822} content
      types was changed.  In module{email} version 2, the container
      \emph{does} return \code{True} from \method{is_multipart()}, and
      \method{get_payload()} returns a list containing a single
      \class{Message} item.

      Note that this is one place that backwards compatibility could
      not be completely maintained.  However, if you're already
      testing the return type of \method{get_payload()}, you should be
      fine.  You just need to make sure your code doesn't do a
      \method{set_payload()} with a \class{Message} instance on a
      container with a content type of \mimetype{message/rfc822}.
\item The \class{Parser} constructor's \var{strict} argument was
      added, and its \method{parse()} and \method{parsestr()} methods
      grew a \var{headersonly} argument.  The \var{strict} flag was
      also added to functions \function{email.message_from_file()}
      and \function{email.message_from_string()}.
\item \method{Generator.__call__()} is deprecated; use
      \method{Generator.flatten()} instead.  The \class{Generator}
      class has also grown the \method{clone()} method.
\item The \class{DecodedGenerator} class in the
      \module{email.Generator} module was added.
\item The intermediate base classes \class{MIMENonMultipart} and
      \class{MIMEMultipart} have been added, and interposed in the
      class heirarchy for most of the other MIME-related derived
      classes.
\item The \var{_encoder} argument to the \class{MIMEText} constructor
      has been deprecated.  Encoding  now happens implicitly based
      on the \var{_charset} argument.
\item The following functions in the \module{email.Utils} module have
      been deprecated: \function{dump_address_pairs()},
      \function{decode()}, and \function{encode()}.  The following
      functions have been added to the module:
      \function{make_msgid()}, \function{decode_rfc2231()},
      \function{encode_rfc2231()}, and \function{decode_params()}.
\item The non-public function \function{email.Iterators._structure()}
      was added.
\end{itemize}

\subsection{Differences from \module{mimelib}}

The \module{email} package was originally prototyped as a separate
library called
\ulink{\module{mimelib}}{http://mimelib.sf.net/}.
Changes have been made so that
method names are more consistent, and some methods or modules have
either been added or removed.  The semantics of some of the methods
have also changed.  For the most part, any functionality available in
\module{mimelib} is still available in the \refmodule{email} package,
albeit often in a different way.

Here is a brief description of the differences between the
\module{mimelib} and the \refmodule{email} packages, along with hints on
how to port your applications.

Of course, the most visible difference between the two packages is
that the package name has been changed to \refmodule{email}.  In
addition, the top-level package has the following differences:

\begin{itemize}
\item \function{messageFromString()} has been renamed to
      \function{message_from_string()}.
\item \function{messageFromFile()} has been renamed to
      \function{message_from_file()}.
\end{itemize}

The \class{Message} class has the following differences:

\begin{itemize}
\item The method \method{asString()} was renamed to \method{as_string()}.
\item The method \method{ismultipart()} was renamed to
      \method{is_multipart()}.
\item The \method{get_payload()} method has grown a \var{decode}
      optional argument.
\item The method \method{getall()} was renamed to \method{get_all()}.
\item The method \method{addheader()} was renamed to \method{add_header()}.
\item The method \method{gettype()} was renamed to \method{get_type()}.
\item The method\method{getmaintype()} was renamed to
      \method{get_main_type()}.
\item The method \method{getsubtype()} was renamed to
      \method{get_subtype()}.
\item The method \method{getparams()} was renamed to
      \method{get_params()}.
      Also, whereas \method{getparams()} returned a list of strings,
      \method{get_params()} returns a list of 2-tuples, effectively
      the key/value pairs of the parameters, split on the \character{=}
      sign.
\item The method \method{getparam()} was renamed to \method{get_param()}.
\item The method \method{getcharsets()} was renamed to
      \method{get_charsets()}.
\item The method \method{getfilename()} was renamed to
      \method{get_filename()}.
\item The method \method{getboundary()} was renamed to
      \method{get_boundary()}.
\item The method \method{setboundary()} was renamed to
      \method{set_boundary()}.
\item The method \method{getdecodedpayload()} was removed.  To get
      similar functionality, pass the value 1 to the \var{decode} flag
      of the {get_payload()} method.
\item The method \method{getpayloadastext()} was removed.  Similar
      functionality
      is supported by the \class{DecodedGenerator} class in the
      \refmodule{email.Generator} module.
\item The method \method{getbodyastext()} was removed.  You can get
      similar functionality by creating an iterator with
      \function{typed_subpart_iterator()} in the
      \refmodule{email.Iterators} module.
\end{itemize}

The \class{Parser} class has no differences in its public interface.
It does have some additional smarts to recognize
\mimetype{message/delivery-status} type messages, which it represents as
a \class{Message} instance containing separate \class{Message}
subparts for each header block in the delivery status
notification\footnote{Delivery Status Notifications (DSN) are defined
in \rfc{1894}.}.

The \class{Generator} class has no differences in its public
interface.  There is a new class in the \refmodule{email.Generator}
module though, called \class{DecodedGenerator} which provides most of
the functionality previously available in the
\method{Message.getpayloadastext()} method.

The following modules and classes have been changed:

\begin{itemize}
\item The \class{MIMEBase} class constructor arguments \var{_major}
      and \var{_minor} have changed to \var{_maintype} and
      \var{_subtype} respectively.
\item The \code{Image} class/module has been renamed to
      \code{MIMEImage}.  The \var{_minor} argument has been renamed to
      \var{_subtype}.
\item The \code{Text} class/module has been renamed to
      \code{MIMEText}.  The \var{_minor} argument has been renamed to
      \var{_subtype}.
\item The \code{MessageRFC822} class/module has been renamed to
      \code{MIMEMessage}.  Note that an earlier version of
      \module{mimelib} called this class/module \code{RFC822}, but
      that clashed with the Python standard library module
      \refmodule{rfc822} on some case-insensitive file systems.

      Also, the \class{MIMEMessage} class now represents any kind of
      MIME message with main type \mimetype{message}.  It takes an
      optional argument \var{_subtype} which is used to set the MIME
      subtype.  \var{_subtype} defaults to \mimetype{rfc822}.
\end{itemize}

\module{mimelib} provided some utility functions in its
\module{address} and \module{date} modules.  All of these functions
have been moved to the \refmodule{email.Utils} module.

The \code{MsgReader} class/module has been removed.  Its functionality
is most closely supported in the \function{body_line_iterator()}
function in the \refmodule{email.Iterators} module.

\subsection{Examples}

Here are a few examples of how to use the \module{email} package to
read, write, and send simple email messages, as well as more complex
MIME messages.

First, let's see how to create and send a simple text message:

\begin{verbatim}
# Import smtplib for the actual sending function
import smtplib

# Here are the email pacakge modules we'll need
from email import Encoders
from email.MIMEText import MIMEText

# Open a plain text file for reading
fp = open(textfile)
# Create a text/plain message, using Quoted-Printable encoding for non-ASCII
# characters.
msg = MIMEText(fp.read(), _encoder=Encoders.encode_quopri)
fp.close()

# me == the sender's email address
# you == the recipient's email address
msg['Subject'] = 'The contents of %s' % textfile
msg['From'] = me
msg['To'] = you

# Send the message via our own SMTP server.  Use msg.as_string() with
# unixfrom=0 so as not to confuse SMTP.
s = smtplib.SMTP()
s.connect()
s.sendmail(me, [you], msg.as_string(0))
s.close()
\end{verbatim}

Here's an example of how to send a MIME message containing a bunch of
family pictures:

\begin{verbatim}
# Import smtplib for the actual sending function
import smtplib

# Here are the email pacakge modules we'll need
from email.MIMEImage import MIMEImage
from email.MIMEBase import MIMEBase

COMMASPACE = ', '

# Create the container (outer) email message.
# me == the sender's email address
# family = the list of all recipients' email addresses
msg = MIMEBase('multipart', 'mixed')
msg['Subject'] = 'Our family reunion'
msg['From'] = me
msg['To'] = COMMASPACE.join(family)
msg.preamble = 'Our family reunion'
# Guarantees the message ends in a newline
msg.epilogue = ''

# Assume we know that the image files are all in PNG format
for file in pngfiles:
    # Open the files in binary mode.  Let the MIMEIMage class automatically
    # guess the specific image type.
    fp = open(file, 'rb')
    img = MIMEImage(fp.read())
    fp.close()
    msg.attach(img)

# Send the email via our own SMTP server.
s = smtplib.SMTP()
s.connect()
s.sendmail(me, family, msg.as_string(unixfrom=0))
s.close()
\end{verbatim}

Here's an example\footnote{Thanks to Matthew Dixon Cowles for the
original inspiration and examples.} of how to send the entire contents
of a directory as an email message:

\begin{verbatim}
#!/usr/bin/env python

"""Send the contents of a directory as a MIME message.

Usage: dirmail [options] from to [to ...]*

Options:
    -h / --help
        Print this message and exit.

    -d directory
    --directory=directory
        Mail the contents of the specified directory, otherwise use the
        current directory.  Only the regular files in the directory are sent,
        and we don't recurse to subdirectories.

`from' is the email address of the sender of the message.

`to' is the email address of the recipient of the message, and multiple
recipients may be given.

The email is sent by forwarding to your local SMTP server, which then does the
normal delivery process.  Your local machine must be running an SMTP server.
"""

import sys
import os
import getopt
import smtplib
# For guessing MIME type based on file name extension
import mimetypes

from email import Encoders
from email.Message import Message
from email.MIMEAudio import MIMEAudio
from email.MIMEBase import MIMEBase
from email.MIMEImage import MIMEImage
from email.MIMEText import MIMEText

COMMASPACE = ', '


def usage(code, msg=''):
    print >> sys.stderr, __doc__
    if msg:
        print >> sys.stderr, msg
    sys.exit(code)


def main():
    try:
        opts, args = getopt.getopt(sys.argv[1:], 'hd:', ['help', 'directory='])
    except getopt.error, msg:
        usage(1, msg)

    dir = os.curdir
    for opt, arg in opts:
        if opt in ('-h', '--help'):
            usage(0)
        elif opt in ('-d', '--directory'):
            dir = arg

    if len(args) < 2:
        usage(1)

    sender = args[0]
    recips = args[1:]
    
    # Create the enclosing (outer) message
    outer = MIMEBase('multipart', 'mixed')
    outer['Subject'] = 'Contents of directory %s' % os.path.abspath(dir)
    outer['To'] = COMMASPACE.join(recips)
    outer['From'] = sender
    outer.preamble = 'You will not see this in a MIME-aware mail reader.\n'
    # To guarantee the message ends with a newline
    outer.epilogue = ''

    for filename in os.listdir(dir):
        path = os.path.join(dir, filename)
        if not os.path.isfile(path):
            continue
        # Guess the Content-Type: based on the file's extension.  Encoding
        # will be ignored, although we should check for simple things like
        # gzip'd or compressed files
        ctype, encoding = mimetypes.guess_type(path)
        if ctype is None or encoding is not None:
            # No guess could be made, or the file is encoded (compressed), so
            # use a generic bag-of-bits type.
            ctype = 'application/octet-stream'
        maintype, subtype = ctype.split('/', 1)
        if maintype == 'text':
            fp = open(path)
            # Note: we should handle calculating the charset
            msg = MIMEText(fp.read(), _subtype=subtype)
            fp.close()
        elif maintype == 'image':
            fp = open(path, 'rb')
            msg = MIMEImage(fp.read(), _subtype=subtype)
            fp.close()
        elif maintype == 'audio':
            fp = open(path, 'rb')
            msg = MIMEAudio(fp.read(), _subtype=subtype)
            fp.close()
        else:
            fp = open(path, 'rb')
            msg = MIMEBase(maintype, subtype)
            msg.add_payload(fp.read())
            fp.close()
            # Encode the payload using Base64
            Encoders.encode_base64(msg)
        # Set the filename parameter
        msg.add_header('Content-Disposition', 'attachment', filename=filename)
        outer.attach(msg)

    fp = open('/tmp/debug.pck', 'w')
    import cPickle
    cPickle.dump(outer, fp)
    fp.close()
    # Now send the message
    s = smtplib.SMTP()
    s.connect()
    s.sendmail(sender, recips, outer.as_string(0))
    s.close()


if __name__ == '__main__':
    main()
\end{verbatim}

And finally, here's an example of how to unpack a MIME message like
the one above, into a directory of files:

\begin{verbatim}
#!/usr/bin/env python

"""Unpack a MIME message into a directory of files.

Usage: unpackmail [options] msgfile

Options:
    -h / --help
        Print this message and exit.

    -d directory
    --directory=directory
        Unpack the MIME message into the named directory, which will be
        created if it doesn't already exist.

msgfile is the path to the file containing the MIME message.
"""

import sys
import os
import getopt
import errno
import mimetypes
import email


def usage(code, msg=''):
    print >> sys.stderr, __doc__
    if msg:
        print >> sys.stderr, msg
    sys.exit(code)


def main():
    try:
        opts, args = getopt.getopt(sys.argv[1:], 'hd:', ['help', 'directory='])
    except getopt.error, msg:
        usage(1, msg)

    dir = os.curdir
    for opt, arg in opts:
        if opt in ('-h', '--help'):
            usage(0)
        elif opt in ('-d', '--directory'):
            dir = arg

    try:
        msgfile = args[0]
    except IndexError:
        usage(1)

    try:
        os.mkdir(dir)
    except OSError, e:
        # Ignore directory exists error
        if e.errno <> errno.EEXIST: raise

    fp = open(msgfile)
    msg = email.message_from_file(fp)
    fp.close()

    counter = 1
    for part in msg.walk():
        # multipart/* are just containers
        if part.get_main_type() == 'multipart':
            continue
        # Applications should really sanitize the given filename so that an
        # email message can't be used to overwrite important files
        filename = part.get_filename()
        if not filename:
            ext = mimetypes.guess_extension(part.get_type())
            if not ext:
                # Use a generic bag-of-bits extension
                ext = '.bin'
            filename = 'part-%03d%s' % (counter, ext)
        counter += 1
        fp = open(os.path.join(dir, filename), 'wb')
        fp.write(part.get_payload(decode=1))
        fp.close()


if __name__ == '__main__':
    main()
\end{verbatim}
