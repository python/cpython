\section{\module{cgitb} ---
         Traceback manager for CGI scripts}

\declaremodule{standard}{cgitb}
\modulesynopsis{Configurable traceback handler for CGI scripts.}
\moduleauthor{Ka-Ping Yee}{ping@lfw.org}
\sectionauthor{Fred L. Drake, Jr.}{fdrake@acm.org}

\versionadded{2.2}
\index{CGI!exceptions}
\index{CGI!tracebacks}
\index{exception!in CGI scripts}
\index{traceback!in CGI scripts}

The \module{cgitb} module provides a special exception handler for CGI
scripts.  After this module is activated using the \function{enable()}
function, if an uncaught exception occurs, a detailed, formatted
report will be sent to the Web browser.  The report includes a
traceback showing excerpts of the source code for each level, as well
as the values of the arguments and local variables to currently
running functions, to help you debug the problem.  Optionally, you can
save this information to a file instead of sending it to the browser.

To enable this feature, simply add one line to the top of your CGI script:

\begin{verbatim}
import cgitb; cgitb.enable()
\end{verbatim}

The options to the \function{enable()} function control whether the
report is displayed in the browser and whether the report is logged
to a file for later analysis.


\begin{funcdesc}{enable}{\optional{display\optional{, logdir\optional{,
                         context}}}}
  This function causes the \module{cgitb} module to take over the
  interpreter's default handling for exceptions by setting the
  value of \code{\refmodule{sys}.excepthook}.
  \withsubitem{(in module sys)}{\ttindex{excepthook()}}

  The optional argument \var{display} defaults to \code{1} and can be set
  to \code{0} to suppress sending the traceback to the browser.
  If the argument \var{logdir} is present, the traceback reports are
  written to files.  The value of \var{logdir} should be a directory
  where these files will be placed.
  The optional argument \var{context} is the number of lines of
  context to display around the current line of source code in the
  traceback; this defaults to \code{5}.
\end{funcdesc}

\begin{funcdesc}{handler}{\optional{info}}
  This function handles an exception using the default settings
  (that is, show a report in the browser, but don't log to a file).
  This can be used when you've caught an exception and want to
  report it using \module{cgitb}.  The optional \var{info} argument
  should be a 3-tuple containing an exception type, exception
  value, and traceback object, exactly like the tuple returned by
  \code{\refmodule{sys}.exc_info()}.  If the \var{info} argument
  is not supplied, the current exception is obtained from
  \code{\refmodule{sys}.exc_info()}.
\end{funcdesc}
